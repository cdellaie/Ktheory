\begin{frame}{Motivations}

Soit $X$ un espace métrique dénombrable discret à géométrie bornée, i.e. $\sup_{x\in X} |B(x,R)|<\infty$ pour tout $R>0$. On note $H_X = l^2(X)\otimes H$ et 

\[C_R[X] = \{T\in \mathcal L(H_X) \text{ t.q. } T_{xy} \in \mathfrak K(H) \text{ et } prop(T) < R \}\]

\begin{definitionfr}[Algèbre de Roe]
\[C^*(X) = \overline{\cup_{R>0} C_R[X]} \subseteq \mathcal L(H_X).\]
\end{definitionfr}
\end{frame}

\begin{frame}
\begin{conj}[Baum-Connes coarse]
Soit $X$ un espace métrique dénombrable discret à géométrie bornée. Alors l'application d'assemblage
\[\mu_X : \varinjlim KK(C_0(P_d(X),\C) \rightarrow K(C^*(X))\]
est un isomorphisme.
\end{conj}

Soit $\Gamma$ un groupe finiment engendré par une partie symmétrique $S$. On note $|\Gamma|$ l'espace métrique obtenu en munissant $\Gamma$ de la métrique de la longueur des mots. \\

La conjecture de Baum-Connes coarse pour $|\Gamma |$ implique la conjecture de Novikov sur les hautes signatures associées à $\Gamma$. 
\end{frame}

\begin{frame}{Motivations}

\begin{thmfr}[Yu, 2010]
Soit $X$ de dimension asymptotique finie. Alors l'application d'assemblage
\[\mu_X : \varinjlim KK(C_0(P_d(X),\C) \rightarrow K(C^*(X))\]
est un isomorphisme.
\end{thmfr}

Dans la preuve : des groupes noté $QP_{\delta, r ,s, k}(X)$ et $QU_{\delta, r ,s, k}(X)$ qui approximent $K_0(C^*(X))$ et $K_1(C^*(X))$.

\end{frame}
