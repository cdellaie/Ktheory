%% Beamer Template of Sabine Hauert and R.~Eddie Wilson
%% Available on https://www.overleaf.com/latex/templates/eddies-minimum-beamer-style/qhcyjdqkbbgx#.V-vfg1SLS1I

%%%%%%%%%%%%%%%%%%%%%%%%%%%%%%%%%%%%%%%%%%%%%%
% Head matter - can we try to be consistent on
% included packages

\documentclass{beamer}
 \usepackage{sansmathaccent} 
\mode<presentation>
{\usetheme{default}
 \usecolortheme{default}
 \usefonttheme{default}
 \setbeamertemplate{navigation symbols}{}
 \setbeamertemplate{caption}[numbered]} 


% Block colors
\setbeamercolor{block body}{bg=blue!10,fg=black}
\setbeamercolor{block title}{bg=blue!30,fg=black}

% Pour que tikzcd fonctionne
%\[\begin{tikzcd}[ampersand replacement=\&, column sep=small]\end{tikzcd}\]


\usepackage[frenchb]{babel}
\usepackage{amsfonts}
\usepackage{amsmath}
\usepackage{amssymb}
%\usepackage[T1]{fontenc}
\usepackage[utf8]{inputenc}
\usepackage{amsthm}
\usepackage{graphicx}
\usepackage{tikz}

%%%%%%%%%%%%%%%%%%%%%%%%%%%%%%%%%%%%%%%%%%%%%%%%%%%%%%%
%% Setting for the nice arrows and nodes graphs  %%%%%%
%%%%%%%%%%%%%%%%%%%%%%%%%%%%%%%%%%%%%%%%%%%%%%%%%%%%%%%

\usetikzlibrary{arrows,positioning,decorations.markings} 
\tikzset{
    %Define standard arrow tip
    >=stealth',
    %Define style for boxes
    punkt/.style={
           rectangle,
           rounded corners,
           draw=black, very thick,
           text width=6.5em,
           minimum height=2em,
           text centered},
    % Define arrow style
    pil/.style={
           ->,
           thick,
           shorten <=2pt,
           shorten >=2pt,}
}
\tikzstyle{vecArrow} = [thick, decoration={markings,mark=at position
   1 with {\arrow[semithick]{open triangle 60}}},
   double distance=1.4pt, shorten >= 5.5pt,
   preaction = {decorate},
   postaction = {draw,line width=1.4pt, white,shorten >= 4.5pt}]
\tikzstyle{innerWhite} = [semithick, white,line width=1.4pt, shorten >= 4.5pt]

%%%%%%%%%%%%%%%%%%%%%%%%%%%%%%%%%%%%%%%%%%%%%%%%%%%%%%%%
%%%%%%%%%%%%%%%%%%%%%%%%%%%%%%%%%%%%%%%%%%%%%%%%%%%%%%%%
\usepackage{tikz-cd}
\usepackage{hyperref}
\usepackage{amssymb}
\usepackage{geometry}

\hypersetup{                    % parametrage des hyperliens
    colorlinks=true,                % colorise les liens
    breaklinks=true,                % permet les retours à la ligne pour les liens trop longs
    urlcolor= blue,                 % couleur des hyperliens
    linkcolor= blue,                % couleur des liens internes aux documents (index, figures, tableaux, equations,...)
    citecolor= cyan               % couleur des liens vers les references bibliographiques
    }

\theoremstyle{definition}
\newtheorem{definition}{Definition}
\newtheorem{thm}{Theorem}
\newtheorem{ex}{Exercice}
\newtheorem{lem}{Lemma}
\newtheorem*{dem}{Proof}
\newtheorem{prop}{Proposition}
\newtheorem{cor}{Corollary}
\newtheorem{conj}{Conjecture}
\newtheorem{Res}{Result}
\newtheorem{Expl}{Example}
\newtheorem{rk}{Remark}

\newcommand{\N}{\mathbb N}
\newcommand{\Z}{\mathbb Z}
\newcommand{\R}{\mathbb R}
\newcommand{\C}{\mathbb C}
\newcommand{\Hil}{\mathcal H}
\newcommand{\Mn}{\mathcal M _n (\mathbb C)}
\newcommand{\K}{\mathbb K}
\newcommand{\B}{\mathbb B}
\newcommand{\Cat}{\mathbb B / \mathbb K}
\newcommand{\G}{\mathcal G }

\setlength\parindent{0pt}


%%%%%%%%%%%%%%%%%%%%%%%%%%%%%%%%%%%%%%%%%%%%%%
% Formatting for title page
\title[First Steps with SCRATCH]{Principe de restriction pour les groupoïdes étales,\\ et une formule de Künneth pour leurs produits croisés}
\author{Clément Dell'Aiera}
\institute{IECL}
\date{15 Juin 2017}
%%%%%%%%%%%%%%%%%%%%%%%%%%%%%%%%%%%%%%%%%%%%%%

\begin{document}

%%% TITRE
\begin{frame}
  \titlepage
\begin{center}\includegraphics[width=5cm]{IECL.png}\end{center}
\end{frame}

\section{Motivations}
\begin{frame}{Motivations}

Soit $X$ un espace métrique dénombrable discret à géométrie bornée, i.e. $\sup_{x\in X} |B(x,R)|<\infty$ pour tout $R>0$. On note $H_X = l^2(X)\otimes H$ et 

\[C_R[X] = \{T\in \mathcal L(H_X) \text{ t.q. } T_{xy} \in \mathfrak K(H) \text{ et } prop(T) < R \}\]

\begin{definitionfr}[Algèbre de Roe]
\[C^*(X) = \overline{\cup_{R>0} C_R[X]} \subseteq \mathcal L(H_X).\]
\end{definitionfr}
\vspace{0.3 cm}
Si $X$ est une variété riemannienne non-compacte, $K(C^*(X))$ est le réceptacle pour les indices d'opérateurs différentiels elliptiques (approche développée par J. Roe).
\end{frame}

\begin{frame}{Motivations}
\begin{conj}[Baum-Connes coarse]
Soit $X$ un espace métrique dénombrable discret à géométrie bornée. Alors l'application d'assemblage
\[\mu_X : \varinjlim KK_*(C_0(P_d(X)),\C) \rightarrow K_*(C^*(X))\]
est un isomorphisme.
\end{conj}

Soit $\Gamma$ un groupe finiment engendré par une partie symmétrique $S$. \\
\vspace{0.3 cm}
On note $|\Gamma|$ l'espace métrique obtenu en munissant $\Gamma$ de la métrique de la longueur des mots. \\
\vspace{0.3 cm}
La conjecture de Baum-Connes coarse pour $|\Gamma |$ implique la conjecture de Novikov sur les hautes signatures associées à $\Gamma$. 
\end{frame}

\begin{frame}{Motivations}

\begin{thmfr}[Yu, 2010 \cite{Yu1}]
Soit $X$ de dimension asymptotique finie. Alors l'application d'assemblage
\[\mu_X : \varinjlim KK(C_0(P_d(X),\C) \rightarrow K(C^*(X))\]
est un isomorphisme.
\end{thmfr}
\vspace{0.3 cm}
Dans la preuve : des groupes d'obstruction notés $QP_{\delta, r ,s, k}(X)$ et $QU_{\delta, r ,s, k}(X)$ qui approximent la $K$-théorie. Ce sont les ancêtres de la $K$-théorie contrôlée.\\

\end{frame}

\begin{frame}{Motivations}

\begin{definitionfr}[Dimension asymptotique]
Un espace métrique est de dimension asymptotique $\leq d$ si, pour tout $R>0$, il existe un recouvrement $\mathcal U$ de $X$  tel que :
\vspace{0.3 cm}
\begin{itemize}
\item[$\bullet$] $\sup_{\mathcal U} diam(U)<\infty$ i.e. $\mathcal U$ est uniformément borné,
\vspace{0.3 cm}
\item[$\bullet$] $\mathcal U = \mathcal U_0 \coprod ... \coprod \mathcal U_d$ i.e. on peut colorier les ensembles par $d+1$ couleurs,
\vspace{0.3 cm}
\item[$\bullet$] $\inf_{U,U'\in\mathcal U_j} d(U,U')\geq R$, i.e. deux ensembles de la même couleur sont distants d'au moins $R$.
 \end{itemize}
\end{definitionfr}
\end{frame}

\begin{frame}{Motivations}
Par exemple, $asdim(\Z^2) \leq 2$.

\[\begin{tikzpicture}
\draw [very thin, gray] (0,0) grid[step=0.5] (6,5);
\end{tikzpicture}\]
%\[\begin{tikzpicture}\foreach \x in {0,...,9}\draw  ( \x , 0) node {$\bullet$};\end{tikzpicture}\]
\end{frame}


\begin{frame}{Motivations}
Par exemple, $asdim(\Z^2) \leq 2$.
\[\begin{tikzpicture}

\draw[fill, blue!60] (0,0) rectangle (2,1) ; 
\draw[fill, blue!60] (0,2) rectangle (2,3) ; 
\draw[fill, blue!60] (0,4) rectangle (2,5) ; 
\draw[fill, blue!60] (3,1) rectangle (5,2) ;
\draw[fill, blue!60] (3,3) rectangle (5,4) ; 

\draw[fill, blue!30] (4,0) rectangle (6,1) ; 
\draw[fill, blue!30] (4,2) rectangle (6,3) ; 
\draw[fill, blue!30] (4,4) rectangle (6,5) ; 
\draw[fill, blue!30] (1,1) rectangle (3,2) ; 
\draw[fill, blue!30] (1,3) rectangle (3,4) ; 

\draw [very thin, gray] (0,0) grid[step=0.5] (6,5);
\end{tikzpicture}\]
%\[\begin{tikzpicture}\foreach \x in {0,...,9}\draw  ( \x , 0) node {$\bullet$};\end{tikzpicture}\]
\end{frame}

\begin{frame}{Motivations}
Par exemple, $asdim(\Z^2) \leq 2$.
\[\begin{tikzpicture}

\draw[fill, blue!60] (0,0) rectangle (2,1) ; 
\draw[fill, blue!60] (0,2) rectangle (2,3) ; 
\draw[fill, blue!60] (0,4) rectangle (2,5) ; 
\draw[fill, blue!60] (3,1) rectangle (5,2) ;
\draw[fill, blue!60] (3,3) rectangle (5,4) ; 

\draw[fill, blue!30] (4,0) rectangle (6,1) ; 
\draw[fill, blue!30] (4,2) rectangle (6,3) ; 
\draw[fill, blue!30] (4,4) rectangle (6,5) ; 
\draw[fill, blue!30] (1,1) rectangle (3,2) ; 
\draw[fill, blue!30] (1,3) rectangle (3,4) ; 

\draw [very thin, gray] (0,0) grid[step=0.25] (6,5);
\end{tikzpicture}\]
%\[\begin{tikzpicture}\foreach \x in {0,...,9}\draw  ( \x , 0) node {$\bullet$};\end{tikzpicture}\]
\end{frame}

\begin{frame}{Motivations}
Par exemple, $asdim(\Z^2) \leq 2$.
\[\begin{tikzpicture}

\draw[fill, blue!60] (0,0) rectangle (2,1) ; 
\draw[fill, blue!60] (0,2) rectangle (2,3) ; 
\draw[fill, blue!60] (0,4) rectangle (2,5) ; 
\draw[fill, blue!60] (3,1) rectangle (5,2) ;
\draw[fill, blue!60] (3,3) rectangle (5,4) ; 

\draw[fill, blue!30] (4,0) rectangle (6,1) ; 
\draw[fill, blue!30] (4,2) rectangle (6,3) ; 
\draw[fill, blue!30] (4,4) rectangle (6,5) ; 
\draw[fill, blue!30] (1,1) rectangle (3,2) ; 
\draw[fill, blue!30] (1,3) rectangle (3,4) ; 

\draw [very thin, gray] (0,0) grid[step=0.1] (6,5);
\end{tikzpicture}\]
%\[\begin{tikzpicture}\foreach \x in {0,...,9}\draw  ( \x , 0) node {$\bullet$};\end{tikzpicture}\]
\pause
Les groupes $QP_{\delta, r ,s, k}(X)$ et $QU_{\delta, r ,s, k}(X)$ satisfont une suite de Mayer-Vietoris qui n'existe pas en $K$-théorie.
\end{frame}

\begin{frame}{Motivations}

Un cadre plus général : la $K$-théorie quantitative (Oyono-Oyono \& Yu, 2015 \cite{OY2}) pour les $C^*$-algèbres filtrées.\\
\vspace{0.3 cm}
Une preuve plus générale :

\begin{thmfr}[Oyono-Oyono Yu, 2017]
Soit $X$ de décomposition géométrique finie. Alors l'application d'assemblage
\[\mu_X : \varinjlim KK(C_0(P_d(X),\C) \rightarrow K(C^*(X))\]
est un isomorphisme.
\end{thmfr}

\end{frame}


\section{$K$-théorie contrôllée}
\begin{frame}{$K$-théorie contrôlée}
Soit $\mathcal E$ une structure coarse et $A$ une $C^*$-algèbre $\mathcal E$ filtrée. On définit les $\varepsilon$-$E$-unitaires
\[U^{\varepsilon, E}(A)= \{u\in A_E \text{ t.q. } ||u^*u-1||<\varepsilon\text{ et }||uu^*-1||<\varepsilon \}\]
et les $\varepsilon$-$E$-projections 
\[P^{\varepsilon, E}(A)= \{p\in A_E \text{ t.q. } p=p^*\text{ et }||p^2-p||<\varepsilon \}.\]
\end{frame}

\begin{frame}{$K$-théorie contrôlée}
Comme en $K$-théorie, 
\begin{itemize}
\item[$\bullet$] $P_\infty^{\varepsilon, E}(A)$ est la limite inductive algébrique des $P_n^{\varepsilon, E}(A)$ par rapports aux inclusions
\[\left\{\begin{array}{rcl}
	P^{\varepsilon,E}_n(A) 		& \rightarrow	& P^{\varepsilon,E}_{n+1}(A)\\ 
	p 		& \mapsto 	& \begin{pmatrix}p& 0 \\ 0&0 \end{pmatrix}
\end{array}\right.\]
\item[$\bullet$] $U_\infty^{\varepsilon, E}(A)$ est la limite inductive algébrique des $U_n^{\varepsilon, E}(A)$ par rapports à
\[\left\{\begin{array}{rcl}
	U^{\varepsilon,E}_n(A) 		& \rightarrow	& U^{\varepsilon,E}_{n+1}(A)\\ 
	u 		& \mapsto 	& \begin{pmatrix}u & 0 \\ 0& 1 \end{pmatrix}
\end{array}\right. .\]
\end{itemize}
\end{frame}

\begin{frame}{$K$-théorie contrôlée}
On munit $P_\infty^{\varepsilon, E}(A)\times \N$ et $U_\infty^{\varepsilon, E}(A)$ des relations d'équivalence suivantes:
\begin{itemize}
\item[$\bullet$] $(p,l) \sim (q,l')$ s'il existe une homotopie de quasi-projections $h\in P^{\varepsilon, E}_\infty(A[0,1])$ et un entier $k$ tel que 
\[h(0)=\begin{pmatrix} p & 0 \\ 0 & 1_{k+l'} \end{pmatrix} \text{ and }
h(1)=\begin{pmatrix} q & 0 \\ 0 & 1_{k+l} \end{pmatrix}\]
\item[$\bullet$] $u \sim v$ s'il existe une homotopie de quasi-unitaires $h\in U^{3\varepsilon, E\circ E}_\infty(A[0,1])$ tel que $h(0)= u \text{ and }h(1)=v$.\\
\end{itemize}
\end{frame}

\begin{frame}{$K$-théorie contrôlée}
Si $A$ est unitale,
\begin{itemize}
\item[$\bullet$] $K_0^{\varepsilon,E}(A) = P^{\varepsilon, E}_\infty(A)\times \N / \sim_{\varepsilon,E}$ 
\item[$\bullet$] $K_1^{\varepsilon,E}(A) = U^{\varepsilon, E}_\infty(A) / \sim_{\varepsilon,E}$.
\end{itemize}

Si $A$ n'est pas unitale,  
\[K_0^{\varepsilon,E}(A) = \{[p,l]_{\varepsilon,E} : p\in P^{\varepsilon,E}_\infty (\tilde A), l\in \N \text{ s.t. rank}(\kappa_0(\rho_A(p)))=l \}\]
et $K_1^{\varepsilon,E}(A)$ est défini par $U_\infty^{\varepsilon,E}(\tilde A)/ \sim_{\varepsilon,E}$.\\

\begin{definitionfr}
La $K$-théorie contrôlée d'une $C^*$-algèbre filtrée $(A,\mathcal E)$ est la famille de groupes abéliens 
\[\hat K_0(A) = (K_0^{\varepsilon,E}(A))_{\varepsilon\in (0,\frac{1}{4}),E\in\mathcal E} \text{ et } \hat K_1(A) = (K_1^{\varepsilon,E}(A))_{\varepsilon\in (0,\frac{1}{4}),E\in\mathcal E}.\]
\end{definitionfr}
\end{frame}

\begin{frame}{$K$-théorie contrôlée}
$\bullet$ Pour tout $\varepsilon <\varepsilon'$ et $E\subseteq E'$, on dispose de morphismes 
\[\iota_{\varepsilon,E}^{\varepsilon',E'} : K^{\varepsilon,E}(A)\rightarrow K^{\varepsilon',E'}(A) \]
tels que $\iota_{\varepsilon',E'}^{\varepsilon'',E''}\circ \iota_{\varepsilon,E}^{\varepsilon',E'} = \iota_{\varepsilon,E}^{\varepsilon'',E''}$ et $\iota_{\varepsilon,E}^{\varepsilon,E}= id_{K^{\varepsilon,E}(A)}$.\\
\vspace{0.3 cm}
$\bullet$ Pour tout $\varepsilon $ et $E\in\mathcal E$, on dispose de morphismes 
\[\iota_{\varepsilon,E} : K^{\varepsilon,E}(A)\rightarrow K(A) \]
tels que $\iota_{\varepsilon',E'}\circ \iota_{\varepsilon,E}^{\varepsilon',E'} = \iota_{\varepsilon,E}$.\\
\vspace{0.3 cm}
$\bullet$ Pour tout élément $x\in K(A)$ et tout $\varepsilon\in (0,\frac{1}{4})$, il existe $E\in\mathcal E$ et $y\in K^{\varepsilon,E}(A)$ tel que $x=\iota_{\varepsilon,E}(y)$.
\end{frame}

\begin{frame}{$K$-théorie contrôlée}
$\bullet$ Une paire de contrôle $(\alpha,\rho)$ est donnée par $\alpha\in (0,\frac{1}{4})$ et $k : (0,\frac{1}{4\alpha})\rightarrow \N $ croissante.
\begin{definitionfr}[Morphisme contrôlé]
Un morphisme $(\alpha,k)$-contrôlé $\hat F : \hat K(A) \rightarrow \hat K(B)$ est une famille de morphismes 
\[F^{\varepsilon,E}: K^{\varepsilon,E}(A) \rightarrow K^{\alpha\varepsilon,k_\varepsilon. E}(B)\] 
compatible avec les $\iota_{\varepsilon,E}^{\varepsilon',E'}$.
\end{definitionfr}
\vspace{0.3 cm}
On dit que $\hat F$ induit $F : K(A)\rightarrow K(B)$ en $K$-théorie si 
\[\iota_{\alpha \varepsilon,k_\varepsilon E}\circ F^{\varepsilon,E}=F.\]
\textbf{Remarque :} Notion d'isomorphisme contrôlé et de suite exacte contrôlée.
\end{frame}

\begin{frame}{$K$-théorie contrôlée}
$\bullet$ Tout $*$-homomorphisme filtré $\phi; A\rightarrow B$, i.e. $\phi(A_E)\subseteq B_E$, induit un morphisme contrôlé 
\[\phi_* : \hat K(A)\rightarrow \hat K(B).\]
$\bullet$ Morita équivalence : $A \rightarrow A\otimes\mathfrak K ; a\mapsto a\otimes e$ induit un isomorphisme
\[K^{\varepsilon,E}(A)\rightarrow K^{\varepsilon,E}(A\otimes\mathfrak K).\]
$\bullet$ Pour toute suite exacte complètement filtrée, il existe une suite exacte contrôlée à $6$ termes, ainsi qu'un bord contrôlé.\\
\vspace{0.3 cm}
$\bullet$ Périodicité de Bott contrôlée : $p\mapsto 1+(e^{2i\pi }-1)p$ induit un isomorphisme contrôlé
\[\beta_A :\hat K_0(A) \rightarrow  \hat K_1(SA).\]

%il existe une paire de contrôle $(\alpha_\beta,k_\beta)$ et un morphisme $(\alpha_\beta,k_\beta)$-contrôlé
%\[\beta_A : K^{\varepsilon, E}(SA)\rightarrow K^{\alpha_\beta\varepsilon, k_\beta(\varepsilon) . E}(A)\]
%qui admet un inverse contrôlé $D_A$.
\end{frame}

\section{Formule de Künneth}
\begin{frame}{Künneth}
Pour toutes $C^*$-algèbres $A$ et $B$, on définit
\[\alpha_{A,B} : K_*(A)\otimes K_*(B)\rightarrow K_*(A\otimes B) \quad ; \quad (x,y)\mapsto x\otimes   \tau_A(y).\]

\begin{definitionfr}
Une $C^*$-algèbre $A$ satisfait la formule de Künneth si, pour toute $C^*$-algèbre $B$ telle que $K_*(B)$ soit libre, $\alpha_{A,B}$ est un isomorphisme.
\end{definitionfr}

Si $A$ satisfait la formule de Künneth, alors, pour toute $C^*$-algèbre $B$, la suite suivante
\[0  \rightarrow K_*(A)\otimes K_*(B) \rightarrow K_*(A\otimes B)  \rightarrow Tor(K_*(A),K_*(B)) \rightarrow 0\]
est exacte.

\end{frame}

\begin{frame}{Künneth}
Si $A$ est $\mathcal E$-filtrée, $A\otimes B$ est $\mathcal E$-filtrée par $\{A_E\otimes B\}_E$. On définit le morphisme contrôllé  
\[\hat\alpha_{A,B} : \hat K_*(A)\otimes K_*(B)\rightarrow \hat K_*(A\otimes B) \quad ; \quad (x,y)\mapsto \hat\tau_A(y)(x),\]
qui induit $\alpha_{A,B}$ en $K$-théorie.\\
\end{frame}

\begin{frame}{Künneth}
Le morphisme contrôllé $\alpha_{A,B}$ est dit :
\begin{itemize}
 
\item[$\bullet$] quasi-injectif s'il existe $\lambda \geq 1$ tel que, pour tout $\varepsilon\in (0,\frac{1}{4\alpha_\tau \lambda})$ et $F\in\mathcal E$, il existe $F'\in\mathcal E$, et $F\subseteq F'$, tels que: 
\[\forall x\in K^{\varepsilon,F}(A)\otimes K(B)\text{ such that }\alpha_{A,B}^{\varepsilon,F}(x)=0 \text{ then }(\iota_{\varepsilon,F}^{\lambda\varepsilon,F'}\otimes id) (x) = 0.\] 

\item[$\bullet$] quasi-surjectif s'il existe $\lambda \geq 1$ tel que, pour tout $\varepsilon \in (0,\frac{1}{4\alpha_\tau})$ et $F\in\mathcal E$, il existe $F'\in\mathcal E$, $F\subseteq h_{\tau,\lambda\varepsilon}.F'$, tels que:
\[ \forall y\in K^{\varepsilon,F}(A\otimes B), \exists x\in K^{\lambda\varepsilon, F'}(A)\otimes K(B) \text{ such that }
\alpha^{\lambda\varepsilon,F'}_{A,B}(x)=\iota_{\varepsilon,F}^{\alpha\lambda\varepsilon,h_{\tau, \lambda\varepsilon}.F'}(y).\] 

\end{itemize}
\end{frame}

\begin{frame}{Künneth}
\begin{definitionfr}
Une $C^*$-algèbre $\mathcal E$-filtrée $A$ satisfait la formule de Künneth quantitative s'il existe $\lambda \geq 1$ tel que, pour toute $C^*$-algèbre $B$ telle que $K_*(B)$ soit libre, $\hat\alpha_{A,B}$ est quasi-injective et quasi-surjectif. 
\end{definitionfr}
\end{frame}

\begin{frame}{Künneth}
\begin{thmfr}
Soit $G$ un groupoïde étale de la classe $\mathcal C$, et soient $E\in\mathcal E$ un ensemble contrôlé de $G$ et $P_E(G)$ le complexe de Rips associé. Si, pour tout sous-groupoïde compact ouvert $H$ de $G$ et tout $H$-espace $V$ tel que l'application moment $p : V\rightarrow H^{(0)}$ soit localement injective, $\alpha_{Res_H^G(A),B}^{H,V}$ est un isomorphisme, alors $\alpha_{A,B}^{G,P_E(G)}$ est un isomorphisme pour toute $C^*$-algèbre $B$ telle que $K_*(B)$ est un groupe abélien libre.\\
\end{thmfr}
\end{frame}

\begin{frame}{Künneth}
\begin{thmfr}
Soit $G$ un groupoïde $\sigma$-compact étale et $A$ une $G$-algèbre. Si 
\begin{itemize}
\item[$\bullet$] $G$ vérifie la conjecture de Baum-Connes à coefficients,
\item[$\bullet$] $G$ est un groupoïde fortement propre,
\item[$\bullet$] pour tout sous-groupoïde compact $H$ de $G$ et tout $H$-espace $V$ telle que l'application moment $p : V\rightarrow H^{(0)}$ soit localement injective, $\alpha_{Res_H^G(A),B}^{G,V}$ est un isomorphisme.
\end{itemize} 
Alors $A\rtimes_r G$ vérifie la formule de Künneth contrôlée.
\end{thmfr}
\end{frame}

\begin{frame}{Künneth}
\end{frame}

\begin{frame}{Künneth}
\end{frame}


\section{Groupoïdes}
\begin{frame}{Motivations}

Soit $X$ un espace métrique dénombrable discret à géométrie bornée, i.e. $\sup_{x\in X} |B(x,R)|<\infty$ pour tout $R>0$. On note $H_X = l^2(X)\otimes H$ et 

\[C_R[X] = \{T\in \mathcal L(H_X) \text{ t.q. } T_{xy} \in \mathfrak K(H) \text{ et } prop(T) < R \}\]

\begin{definitionfr}[Algèbre de Roe]
\[C^*(X) = \overline{\cup_{R>0} C_R[X]} \subseteq \mathcal L(H_X).\]
\end{definitionfr}
\end{frame}

\begin{frame}
\begin{conj}[Baum-Connes coarse]
Soit $X$ un espace métrique dénombrable discret à géométrie bornée. Alors l'application d'assemblage
\[\mu_X : \varinjlim KK(C_0(P_d(X),\C) \rightarrow K(C^*(X))\]
est un isomorphisme.
\end{conj}

Soit $\Gamma$ un groupe finiment engendré par une partie symmétrique $S$. On note $|\Gamma|$ l'espace métrique obtenu en munissant $\Gamma$ de la métrique de la longueur des mots. \\

La conjecture de Baum-Connes coarse pour $|\Gamma |$ implique la conjecture de Novikov sur les hautes signatures associées à $\Gamma$. 
\end{frame}

\begin{frame}{Motivations}

\begin{thmfr}[Yu, 2010]
Soit $X$ de dimension asymptotique finie. Alors l'application d'assemblage
\[\mu_X : \varinjlim KK(C_0(P_d(X),\C) \rightarrow K(C^*(X))\]
est un isomorphisme.
\end{thmfr}

Dans la preuve : des groupes noté $QP_{\delta, r ,s, k}(X)$ et $QU_{\delta, r ,s, k}(X)$ qui approximent $K_0(C^*(X))$ et $K_1(C^*(X))$.

\end{frame}


\begin{frame}{Références}
\bibliographystyle{plain}
\bibliography{biblio2}
\end{frame} 

\begin{frame}{}
Merci de votre attention !
\end{frame}

\end{document}
