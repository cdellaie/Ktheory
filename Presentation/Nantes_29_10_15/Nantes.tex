\documentclass{beamer}
\usepackage[frenchb]{babel}
\usepackage{amsfonts}
\usepackage{amsmath}
\usepackage{amssymb}
%\usepackage[T1]{fontenc}
\usepackage[utf8]{inputenc}
\usepackage{amsthm}
\usepackage{graphicx}
\usepackage{tikz}
\usepackage{tikz-cd}
\usepackage{hyperref}
\usepackage{amssymb}
\usepackage{geometry}

\hypersetup{                    % parametrage des hyperliens
    colorlinks=true,                % colorise les liens
    breaklinks=true,                % permet les retours à la ligne pour les liens trop longs
    urlcolor= blue,                 % couleur des hyperliens
    linkcolor= blue,                % couleur des liens internes aux documents (index, figures, tableaux, equations,...)
    citecolor= cyan               % couleur des liens vers les references bibliographiques
    }

\theoremstyle{definition}
\newtheorem{definition}{Definition}
\newtheorem{thm}{Theorem}
\newtheorem{ex}{Exercice}
\newtheorem{lem}{Lemma}
\newtheorem*{dem}{Proof}
\newtheorem{prop}{Proposition}
\newtheorem{cor}{Corollary}
\newtheorem{conj}{Conjecture}
\newtheorem{Res}{Result}
\newtheorem{Expl}{Example}
\newtheorem{rk}{Remark}

\newcommand{\N}{\mathbb N}
\newcommand{\Z}{\mathbb Z}
\newcommand{\R}{\mathbb R}
\newcommand{\C}{\mathbb C}
\newcommand{\Hil}{\mathcal H}
\newcommand{\Mn}{\mathcal M _n (\mathbb C)}
\newcommand{\K}{\mathbb K}
\newcommand{\B}{\mathbb B}
\newcommand{\Cat}{\mathbb B / \mathbb K}
\newcommand{\G}{\mathcal G }

\setlength\parindent{0pt}

 %\usepackage[utf8]{inputenc}
\usetheme{CambridgeUS}

\title{Propagation en K-théorie}
\author{Clément Dell'Aiera}\institute{Université de Lorraine}

\begin{document}

\begin{frame}
\titlepage
\end{frame}

\section{Théorèmes de l'indice sur des variétés compactes}

\begin{frame}
On rappelle qu'un opérateur compact est un opérateur limite d'opérateurs de rang fini. \\
Un opérateur $T$ est dit de Fredholm s'il est inversible modulo les opérateurs compacts. Son noyau et son conoyau sont alors fini-dimentsionnels et on peut définir son indice :
\[Ind\ T = \dim Ker T-\dim Ker T^*\] 
Soit $M$ une variété différentielle, et $d$ la différentielle extérieure définie sur le complexe de De Rham des formes extérieures
\[\begin{tikzcd}[ampersand replacement=\&,column sep = small]
\Omega^0(M) \arrow{r}{d} \& \Omega^1(M) \arrow{r}{d} \& .. \arrow{r}{d} \& \Omega^n(M)  
\end{tikzcd}\]
\end{frame}

\begin{frame}
Une métrique riemannienne $g$ sur $M$ induit une mesure $\mu$ sur $M$ et un produit scalaire $(,)$ sur le cotangent $T^* M$ que l'on peut étendre aux formes :
\[\langle \alpha,\beta\rangle = \int_M (\alpha(x),\beta(x))\mu(dx)\]
On complète $\Omega^j_c(M)$par rapport à $\langle\rangle$ pour obtenir un complexe d'espaces de Hilbert $\Omega^*_{L^2}(M)$
\[\begin{tikzcd}[ampersand replacement=\&,column sep = small]
\Omega_{L^2}^0(M) \arrow{r}{d} \& \Omega_{L^2}^1(M) \arrow{r}{d} \& .. \arrow{r}{d} \& \Omega_{L^2}^n(M)  
\end{tikzcd}\]
C'est le complexe des formes de carré intégrable. L'opérateur $d$ est cette fois un \textbf{opérateur non borné}, soit $d^*$ son adjoint.
\end{frame}

\begin{frame}
$D= d+d^*$ est ce que l'on appelle un opérateur de Dirac généralisé. \\
C'est un opérateur non-borné auto-adjoint, on peut donc donner un sens à une expression $f(D)$ par calcul fonctionnel, pour toute $f$ borélienne bornée.\\
\begin{thm}[Régularité elliptique]
Si $\phi\in C_0(\R)$, alors $\phi(D)$ est un opérateur compact.
\end{thm}

Si $\chi : \R\rightarrow \R$ est continue, bornée et impaire telle que $\chi (t)\rightarrow_{+\infty} 1$, alors 
\begin{itemize}
\item[$\bullet$] $\chi(D)$ est un opérateur de Fredholm,
\item[$\bullet$] $\chi_1(D)-\chi_2(D)$ est compact.
\end{itemize}
On peut donc calculer l'indice de $\chi(D)$ !
\end{frame}

\begin{frame}
Il vaut $0$ ! \\
$\chi(D)$ est autoadjoint... Mais on peut séparer les formes de degré pair et impair pour obtenir une graduation
\[\Omega_{L^2}(M) = \Omega^{even}(M) \bigoplus \Omega^{odd}(M),\]
et $D$ est un opérateur impair par rapport à cette décomposition :
\[D= \begin{pmatrix}0 & D_-\\ D_+ & 0  \end{pmatrix}\]
$\chi(D_+)$ est de Fredholm, et on définit
\[Ind(D,\epsilon)=Ind\chi(D_+)\]
Cet indice est non nul en général, et ne dépend pas de $\chi$.\\
($\epsilon = \begin{pmatrix}1 & 0 \\ 0 & -1\end{pmatrix}$ est l'opérateur de graduation.)
\end{frame}

\begin{frame}
Les théorèmes de l'indice donnent une formule pour l'indice d'un opérateur de Dirac généralisé en fonction de données topologiques.\\
\textbf{Indice $=$ évaluation d'une classe caractéristique contre la classe fondamentale de la variété }
\[Ind \ (D,\epsilon) = \langle \mathcal I _D , [M]\rangle\]
\end{frame}

\subsection{Exemples}
\begin{frame}%[Gauss-Bonnet]
On suppose que la dimension $n$ de $M$ est paire, et comme auparavant
\[\Omega_{L^2}(M)=\Omega^{even}(M)\oplus \Omega^{odd}(M),\quad \epsilon = \begin{pmatrix}1 & 0\\ 0 & -1 \end{pmatrix}\]
 \[D=d+d^*\]
\begin{thm}[Gauss-Bonnet]
\[Ind \ (D ,\epsilon) = \left\{\begin{array}{l} \chi(M)\quad \text{caractéristique d'Euler}\\ \frac{1}{(2\pi)^{n/2}}\int_M Pf(R)\end{array}\right.\]
\end{thm}
\end{frame}

\begin{frame}
L'opérateur de Hodge $\star : \Omega^{k}_{L^2}(M)\rightarrow \Omega_{L^2}^{n-k}(M)$ implémente une autre graduation, et 
\[Ind \ (D,\star) =\left\{ \begin{array}{l}sg(M)  \quad \text{signature de la variété}\\ \langle \mathcal L_M, [M]\rangle \end{array}\right.\]
\end{frame}

\begin{frame}
Si $M$ admet une structure Spin,
\[\slashed D = \sum c(X_j) \nabla_{X_j}\]
alors 
\[Ind \ \slashed D= \langle \mathcal A,[M]\rangle\]
\end{frame}

\begin{frame}
\textbf{BUT :} Etendre ces techniques au cas non-compact.\\

\textbf{Problème :} $\phi(D)$ n'est plus un opérateur compact.\\

\textbf{Piste :} Utiliser la preuve des théorèmes de l'indices utilisant le noyau de la chaleur.\\
\end{frame}

\begin{frame}
\begin{prop}[Formule de Lin??]
\[\slashed D^2 =\nabla \nabla^* +\frac{1}{4} Sc \]
\end{prop}

\[Ind \ \slashed D = Tr(\epsilon e^{-t\slashed D^2})\]
Lorsque $t\rightarrow 0$, cette formule devient locale en $t$ et le terme dominant est l'inégrale sur $M$ d'une forme $\mathcal I(x)dx$ que l'on peut determiner explicitement, sa classe de cohomolgie donne $\mathcal I_D$.\\

\textbf{Idée :} Le semi-groupe de la chaleur construit une homotopie entre l'invariant local $\mathcal I_D$ et l'invariant global $Ind \ \slashed D$.  
\end{frame}

\section{Géométrie asymptotique}
\subsection{$C^*$-algèbres}

\begin{frame}
\begin{definition} Une $C^*$-algèbre est une sous-algèbre $A\subset \mathcal L(H)$
 fermée pour la norme d'opérateur et stable par adjonction.
\end{definition}
\textbf{Exemples :} 
\begin{itemize} 
\item[$\bullet$]$C_0(X)$ pour un espace localement compact $X$.
\item[$\bullet$] l'algèbre des opérateurs bornés $\mathcal L(H)$, des opérateurs compacts $\mathfrak K(H)$.
\end{itemize}
\end{frame}

\end{document}