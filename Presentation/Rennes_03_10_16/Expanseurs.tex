\begin{frame}
  \titlepage
\end{frame}

% PARTIE 1 
\section{Laplacien et marches aléatoires sur les graphes}

\begin{frame}{Laplacien}
Notations. Un graphe fini $G=(V,E)$.
 
\begin{itemize}
\item[$\bullet$] $V$ ensemble fini de $n$ sommets,
\item[$\bullet$] $E\subset V\times V$ l'ensemble des arêtes,
\item[$\bullet$] si $v\in V$, les voisins de $v$ : $N_v=\{w\in V / (v,w)\in E\}$,
\item[$\bullet$] le degré de $v$ : $deg(v)=|N_v|$.
\end{itemize}
Marche aléatoire uniforme sur un graphe :
\begin{enumerate}
\item $X_0=v_0$
\item $X_{n+1}|X_n\sim \mathcal U_{N_{X_n}}$. \\
\end{enumerate}
\end{frame}

\begin{frame}{Laplacien}
Le Laplacien sur un graphe $\Delta_G\in \mathcal L(l^2(X))$ est définit comme
\[(\Delta f) (x) = f(x) - \frac{1}{deg(x)}\sum_{y\in N_x} f(y) ,\]
si $f\in l^2(X)$.\\

Propriétés : $\Delta\geq 0$ et $\text{ker }\Delta\simeq \mathbb C $.\\

\[Sp(\Delta)= \lambda_0\leq \lambda_1\leq \lambda_2...\leq \lambda_n\]
et $\lambda_0=0$.
\end{frame}

%%%%%%%
%%%%%%%
%%%%%%%
%%%%%%%


%\section{Some \LaTeX{} Examples}
%\subsection{Tables and Figures}



% Commands to include a figure:
%\begin{figure}
%\includegraphics[width=\textwidth]{your-figure's-file-name}
%\caption{\label{fig:your-figure}Caption goes here.}
%\end{figure}

\begin{frame}{Eddie shows two columns technique}
\begin{columns}
\begin{column}{0.4\textwidth}
\begin{center}
\includegraphics[scale=0.75]{IECL.png}
\end{center}
\end{column}
\begin{column}{0.6\textwidth}
\begin{itemize}
\item This is cleanest way of putting text beside
pictures. 
\pause\medskip
\item Also illustrating {\tt pause} for reveals --- I 
use this only sparingly
\pause\medskip
\item We need a convention on scaling of included SCRATCH graphics. Here: source image obtained by screen grabbing from desktop SCRATCH 1.4. Hopefully this is consistent across OS etc.! 
\medskip
\item The scaling here is mega-generous - for consideration.
\end{itemize}
% NB \smallskip, \medskip, \bigskip much better for
% consistency than explicit spaces
\end{column}
\end{columns}
\end{frame}
