\begin{frame}
  \titlepage
\begin{center}\includegraphics[width=5cm]{IECL.png}\end{center}
\end{frame}

% PARTIE 1
\section{Un problème de théorie des réseaux}

\begin{frame}{Un problème de théorie des réseaux}
\begin{figure}
\href{run:InternetMap2012Full.avi}{\includegraphics[scale=0.35]{InternetMap2012.jpg}} 
\caption{Map}
\end{figure}
\end{frame}

\begin{frame}{Un problème de théorie des réseaux}
Peut-on construire un réseau aussi grand que l'on souhaite sans que le risque de casse augmente tout en maîtrisant les coûts ?\\
Modélisation :\\
\begin{itemize}
\item[$\bullet$] Taille = nombre de sommets,
\item[$\bullet$] Propriété à maximiser = connexité,
\item[$\bullet$] Coûts = nombre d'arêtes.
\end{itemize}
Comment mesurer la connexité ? $\Rightarrow$ Marche aléatoire
\end{frame}

% PARTIE 2 
\section{Laplacien et marches aléatoires sur les graphes}

\begin{frame}{Laplacien}
Notations. Un graphe fini $G=(V,E)$.
 
\begin{itemize}
\item[$\bullet$] $V$ ensemble fini de $n$ sommets,
\item[$\bullet$] $E\subset V\times V$ l'ensemble des arêtes,
\item[$\bullet$] si $v\in V$, les voisins de $v$ : $N_v=\{w\in V / (v,w)\in E\}$,
\item[$\bullet$] le degré de $v$ : $deg(v)=|N_v|$.
\end{itemize}
Marche aléatoire uniforme sur un graphe :
\begin{enumerate}
\item $X_0=v_0$
\item $X_{n+1}|X_n\sim \mathcal U_{N_{X_n}}$. \\
\end{enumerate}
\end{frame}

\begin{frame}{Laplacien}
Le Laplacien sur un graphe $\Delta_G\in \mathcal L(l^2(X))$ est définit comme
\[(\Delta f) (x) = f(x) - \frac{1}{deg(x)}\sum_{y\in N_x} f(y) ,\]
si $f\in l^2(X)$.\\

Propriétés : $\Delta\geq 0$ et $\text{ker }\Delta\simeq \mathbb C $.\\

\[Sp(\Delta)= \lambda_0\leq \lambda_1\leq \lambda_2...\leq \lambda_n\]
et $\lambda_0=0$.
\end{frame}

\begin{frame}{Laplacien}

Si $M\in\mathcal L(l^2(X))$ est l'opérateur de Markov associé à la marche aléatoire, 
\[(Mf)(v)=\frac{1}{deg(v)}\sum_{w \in N_v} f(w) = \mathbb E[f(X_1)|X_0=v],\]
alors : \[\Delta=Id-M\]
Donc $0<\lambda_1$ contrôle la vitesse de convergence vers la distribution d'équilibre.
\end{frame}

\begin{frame}{Laplacien}
Reformulation du problème :\\
Peut-on construire une suite de graphes finis $\{X_j\}$ :
\begin{itemize}
\item[$\bullet$] de plus en plus grands $|X_j|\rightarrow \infty$,
\item[$\bullet$] $k$-réguliers : $deg(X_j)\leq k$,
\item[$\bullet$] uniformément connexes : $\exists\varepsilon>0, \forall j, \lambda_1(X_j)>\varepsilon$ 
\end{itemize}
\end{frame}

% PARTIE 3

\section{Expanseurs}

\begin{frame}{Expanseurs}
\begin{defn}
Un expanseur est une suite de graphes finis $\{X_j\}$ :
\begin{itemize}
\item[$\bullet$] de plus en plus grands $|X_j|\rightarrow \infty$,
\item[$\bullet$] $k$-réguliers : $deg(X_j)\leq k$,
\item[$\bullet$] uniformément connexes : $\exists\varepsilon>0, \forall j, \lambda_1(X_j)>\varepsilon$ 
\end{itemize}
\end{defn}
\end{frame}

\begin{frame}{Expanseurs}
\begin{figure}\includegraphics[scale=0.25]{Graphs5.png}\caption{Une famille de graphes}\end{figure}
\end{frame}

\begin{frame}{Expanseurs}
\begin{figure}\includegraphics[scale=0.45]{Laplacian.png}\caption{Laplacien de la famille}\end{figure}
\end{frame}

\begin{frame}{Expanseurs}
Comment en construire ?
\end{frame}

\begin{frame}{Expanseurs} % Une motivation
Soit $\Gamma$ un groupe discret dénombrable finiement engendré, par une partie $S$ symmétrique telle que $e_\Gamma\not\in S $.\\
On construit un espace métrique $|\Gamma|$ grâce à la distance 
\[d_S(\gamma,\gamma')=l_S(\gamma^{-1}\gamma')\]
où $l_S$ est la longueurs des mots $l_S(g) = \min \{j \ |\ g=s_1 ...s_j \ , s_j\in S\}$.
Cette construction est canonique et ne dépend pas de $S$ !
\begin{thm}
Soient $S$ et $S'$ des parties finies génératrices. Alors les espaces métriques associés $(\Gamma,d)$ et $(\Gamma,d')$ sont quasi-isométriques.
\end{thm}
\end{frame}

\begin{frame}{Expanseurs}
\begin{figure}[h]\centering
\includegraphics[scale=0.35]{CayleyFree2.jpeg}
\caption{Boules dans $Cay(\mathbb F_2)$}
\label{fig:Cayley}
\end{figure}
\end{frame}

\begin{frame}{Expanseurs}
\begin{figure}[h]\centering
\includegraphics[scale=0.35]{CayleyFree3.jpeg}
\caption{Boules dans $Cay(\mathbb F_2)$}
\label{fig:Cayley2}
\end{figure}
\end{frame}
\begin{frame}{Expanseurs}
\begin{figure}[h]\centering
\includegraphics[scale=0.35]{CayleyFree4.jpeg}
\caption{Boules dans $Cay(\mathbb F_2)$}
\label{fig:Cayley3}
\end{figure}
\end{frame}
\begin{frame}{Expanseurs}
\begin{figure}[h]\centering
\includegraphics[scale=0.35]{CayleyFree5.jpeg}
\caption{Boules dans $Cay(\mathbb F_2)$}
\label{fig:Cayley4}
\end{figure}
\end{frame}

\begin{frame}{Expanseurs}
Recette pour fabriquer un expanseur.
\begin{itemize}
\item[$\bullet$] Choisir un groupe f.g. $\Gamma$, résiduellement fini par rapport à une famille de sous-groupes normaux $\Gamma_0 > \Gamma_1>...$ d'intersection triviale, et qui a la propriété T.
\item[$\bullet$] Former l'union dijointe (métrique...) des espaces $\Gamma/\Gamma_j$ :\[X_{\mathcal N}(\Gamma)=\sqcup_j \Gamma/\Gamma_j.\]
\item[$\bullet$] $X_{\mathcal N}(\Gamma)$ est un expanseur.
\end{itemize}

Exemple : $\Gamma = SL(3,\Z)$ et $ \Gamma_j = SL(3,p^j\Z)$.
\end{frame}

% PARTIE 4 
\section{Plongements uniformes}
\begin{frame}{Plongements uniformes}

\begin{defn}
Soit $X$ et $Y$ deux espaces métriques. Un plongement uniforme est une application $\phi : X\rightarrow Y$ telle qu'il existe deux applications croissantes $\rho_0,\rho_1 : \R_+^*\rightarrow \R_+^*$ vérifiant
\[\rho_0(d_X(x,y))\leq d_Y(\phi(x),\phi(y))\leq \rho_1(d_X(x,y))\] 
pour $x,y\in X$.
\end{defn}

\begin{itemize}
\item[$\bullet$] les isométries sont des plongements uniformes,
\item[$\bullet$] $\rho$ linéaires : les quasi-isométries,
\item[$\bullet$] $\Z\hookrightarrow \R$.
\end{itemize}
\end{frame}

\begin{frame}{Plongements uniformes} 
\begin{itemize}
\item[$\bullet$] Tout espace métrique se plonge isométriquement dans un esapce de Banach.
\item[$\bullet$] Qu'en est-il des espaces de Hibert.
\end{itemize}
On se fixe $H$ un espace de Hilbert séparable.
\end{frame}

\begin{frame}{Plongements uniformes} 
\begin{thm}
Si ${X_j}_j$ est un expanseur, alors $X$ ne peut pas être plongé uniformément dans $H$.
\end{thm}
\end{frame}

\begin{frame}{Conséquences} 

La conjecture de Baum-Connes coarse donne un moyen de calculer les indices d'opérateurs elliptiques abstraits dans un espace métrique.\\
Elle est démontrée pour les espaces métriques qui admettent un plongement uniforme dans $H$, et admet des contre-exemples.\\
Une version modifiée de cette conjecture a été démontrée pour une classe d'expanseurs. \\
\end{frame}
%%%%%%%
%%%%%%%
%%%%%%%
%%%%%%%
