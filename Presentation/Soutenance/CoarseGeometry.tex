\begin{frame}{Géométrie Coarse}

Soit $X$ un espace métrique discret à géométrie bornée. Le groupoïde coarse $G(X)$, introduit par G. Skandalis, J-L. Tu et G. Yu \cite{SkTuYu}, est un groupoïde étale associé à tout espace coarse $X$.\\
\vspace{0.3 cm}
\begin{definitionfr}
Il existe une seule structure de groupoïde étale sur 
\[G(X) = \cup_{E\in\mathcal E }\overline{E}\subseteq \beta (X\times X)\]
au dessus de $\beta X$ qui étende le groupoïde des paires $X\times X\rightrightarrows \beta X$.
\end{definitionfr} 

\end{frame}

\begin{frame}{Géométrie Coarse}
On pose $l_B^\infty = l_B^\infty(X,B\otimes \mathfrak K)$. C'est une $G(X)$-algèbre.\\
\vspace{0.3 cm}
\begin{propfr}[\cite{SkTuYu}]
Pour toute $C^*$-algèbre $B$, il existe un $*$-isomorphisme 
\[C^*(X,B) \cong l_B^\infty \rtimes_r G(X),\]
naturel en $B$.
\end{propfr} 
\vspace{0.3 cm}
Cette équivalence entre géométrie coarse et groupoïde s'étend au niveau de la $KK$-théorie.
\end{frame}


\begin{frame}{Géométrie Coarse}
On note $\iota : \{x\}\hookrightarrow G(X)$ l'inclusion d'un point $x\in X$.\\
\vspace{0.3 cm}
\begin{thmfr}
Soient $B$ une $C^*$-algèbre, $E\in\mathcal E_X$ un entourage et $l_B^\infty$ la $G$-algèbre $l^\infty(X,B\otimes \mathfrak K)$. Alors, pour tout $z\in RK^G(P_{\overline E}(G),l_B^\infty)$ et tout $\varepsilon\in(0,\frac{1}{4})$, l'égalité suivante est vérifiée :
\[(\Psi_B)_*\circ\mu^{\epsilon,\overline E}_{G,l_B^\infty} (z) = \mu_{X,B}^{\epsilon,E}(\iota^*(z)).\]
\end{thmfr}
\vspace{0.3 cm}
Ce théorème induit en $K$-théorie un résultat de G. Skandalis, J-L. Tu et G. Yu \cite{SkTuYu}, qui établit l'équivalence entre la conjecture de Baum-Connes coarse pour $X$ à coefficients dans $B$ et la conjecture de Baum-Connes pour $G(X)$ à coefficients dans $l^\infty(X,B\otimes \mathfrak K)$.
\end{frame}

\begin{frame}{Géométrie Coarse}
Le théorème précédent permet notamment de donner une version contrôlée d'un résultat de M. Finn-Sell \cite{FinnSellFibred}. \\
\vspace{0.3 cm}
\begin{corfr}
Soit $X$ un espace coarse qui admet un plongement fibré dans l'espace de Hilbert. Alors $\hat \mu_{X}^{max}$ est un isomorphisme contrôlé, i.e. $X$ vérifie la version maximale de la conjecture de Baum-Connes coarse contrôlée.\\
\end{corfr}
\end{frame}






