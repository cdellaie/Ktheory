\begin{frame}{Groupoïdes}
Soit $G \rightrightarrows G^{(0)} $ un groupoïde localement compact.\\
\vspace{0.3 cm}
On note $r,s:G \rightrightarrows G^{(0)}$ les applications source et but, $e : G^{(0)} \rightarrow G$ l'application unité, et 
\[G^{(2)} = \{(g,g') \in G\times G \text{ t.q. } s(g)=r(g')\}\]
les paires composables.\\
\vspace{0.3 cm}
On rappelle des constructions dues à P-Y. Le Gall, J. Renault, J-L. Tu.
\vspace{0.3 cm}
\begin{definitionfr}
Le groupoïde $G$ est dit étale si l'application but $r : G \rightarrow G^{(0)} $ est un homéomorphisme local.
\end{definitionfr}

\end{frame}

\begin{frame}{Groupoïdes}

Une $C(G^{(0)})$-algèbre $A$ est la donnée d'une $C^*$-algèbre $A$ muni d'un $*$-morphisme non-dégénéré $\theta : C(G^{(0)}) \rightarrow Z(\mathcal M(A))$.\\ 

\begin{definitionfr}
L'espace $C_c(G,A)$ des sections à support compact est défini comme
\[C_c(G,A) = \bigcup_U C_0(U)\otimes_s A\]
où $U$ parcourt les ouverts $U\subseteq G$ relativement compacts.
\end{definitionfr}

\end{frame}

\begin{frame}{Groupoïdes}

Une action de $G$ sur une $C(G^{(0)})$-algèbre $A$ est la donnée d'un isomorphisme de $C(G)$-algèbres 
\[\alpha : s^* A \rightarrow r^* A\]
tel que $\alpha_{e_x} = id$ et $\alpha_g \circ \alpha_{g'}  = \alpha_{gg'}$ pour tout $x\in G^{(0)}$ et toute paire composable $(g,g') \in G^{(2)}$.\\ 
\vspace{0.3 cm}
On munit $C_c(G,A)$ du produit de convolution 
\[(f_1\ast f_2)(g) = \sum_{h\in G^{r(g)}} f_1(h) \alpha_h(f_2(h^{-1}g)).\]
et de l'involution $\overline f(g)=\alpha_g(f(g^{-1})^*)$.\\

\end{frame}

\begin{frame}{Groupoïdes}
Le $A$-module hilbertien $L^2(G,A)$ est la complétion de $C_c(G,A)$ pour le produit scalaire 
\[\langle \xi ,\eta \rangle_x  = \sum_{g\in G^x} \xi(g^{-1})^* \eta(g^{-1}) \quad x\in G^{(0)} \]
sur lequel $C_c(G,A)$ est représenté par $\lambda(f) \xi = f\ast \xi$, pour tout $ f\in C_c(G,A)$ et $\xi\in L^2(G,A)$.\\
\end{frame}

\begin{frame}{Groupoïdes}
Notation : $EE' = \{gg' / (g,g')\in G^{(2)}\cap E\times E'\}$.
\vspace{0.3 cm}
\begin{definitionfr}[Produit croisé]
Le produit croisé $A\rtimes_r G$ est la $C^*$-algèbre obtenue en complétant $C_c(G,A)$ pour la norme $||f||_r=||\lambda(f)||$.
\end{definitionfr}
\vspace{0.3 cm}
On pose $\mathcal E$ l'ensemble des compacts symétriques non vides de $G$, muni de la loi de compostion $E\circ E' = EE' \cup E'E$. Alors $\mathcal E$ est une structure coarse et $A\rtimes_r G$ est $\mathcal E$-filtrée avec
\[A_E = \cup_{U\subseteq E	} C_0(U)\otimes_s A.\] 
\end{frame}

\begin{frame}{Groupoïdes}
Pour calculer la $K$-théorie de $A\rtimes_r G$, on dispose de l'application d'assemblage de Baum-Connes
\[\mu_{G,A} : K^{top}(G,A) \rightarrow K(A \rtimes_r G)\]
définie pour toute $G$-algèbre $A$ (J-L. Tu). Le membre de gauche est calculable par des méthodes de topologie algébrique classique.
\vspace{0.3 cm}
\begin{conj}[Baum-Connes]
Le groupoïde $G$ vérifie la conjecture de Baum-Connes à coefficients si $\mu_{G,A}$ est un ismomorphisme pour toute $G$-algèbre $A$.
\end{conj}
\vspace{0.3 cm}
\textbf{Rappel:} $K^{top}(G,A)=\varinjlim_{E\in \mathcal E} RK^G(P_E,A)$ où $P_E$ est le complexe de Rips associé à $G$ :
\[P_E = \{\eta\in Prob(G) \text{ t.q. } \exists x\in G^{(0)},g\in G^x / \ supp (\eta) \subseteq G^x \cap gE  \}\]
\end{frame}

