\begin{frame}{Introduction}
En physique, les modèles prévoient les valeurs de quantités mesurables, appelées \textbf{observables}. Ce sont des fonctions
\[f:\Omega \rightarrow \R\]
où $\Omega$ est l'espace de configuration du système. Ces observables commutent en physique classique.\\
\vspace{0.3 cm} 
Pour donner un modèle de l'atome d'hydrogène en accord avec l'expérience, Heisenberg propose en 1925 de remplacer les observables classiques par des "q-nombres", qui ne commutent pas.
\[pq -qp = i\hbar 1\]
\end{frame}

\begin{frame}{Introduction}
\begin{definitionfr}
Une $C^*$-algèbre est une $\C$-algèbre $A$ munie d'un antihomomorphisme involutif $*$ et d'une norme multiplicative $||.||$ telle que :
\begin{itemize}
\item[$\bullet$] $(A,||.||)$ est une algèbre de Banach,
\item[$\bullet$] $||a^*a||^2 = ||a||^2$ pour tout $a\in A$.
\end{itemize}
\end{definitionfr}

\begin{itemize}
\item[$\bullet$] l'algèbre des opérateurs bornés sur un espace de Hilbert $\mathcal L(H)$, 
\item[$\bullet$] l'algèbre des fonctions continues sur un espace localement compact $C_0(X)$.
\end{itemize}

\end{frame}

\begin{frame}{Introduction}
Il s'avère que toute $C^*$-algèbre commutative est équivalente à un espace localement compact.
\begin{thmfr}[Dualité de Gelfand]
Soit $A$ une $C^*$-algèbre commutative. Alors il existe un espace localement compact $X$ et un $*$-isomorphisme 
\[A\cong C_0(X).\]
\end{thmfr}

Alain Connes développe à partir des années 80 la géométrie non-commutative basée sur le principe suivant
\begin{block}{Géométrie Non-Commutative}
Une $C^*$-algèbre représente un "espace non-commutatif". 
\end{block}

\end{frame}

\begin{frame}{Introduction}
Dans le cas de la topologie classique, un des objets à déterminer est ce qu'on appelle l'homologie et la cohomologie d'un espace.\\
\vspace{0.3 cm}
Dans le cadre de la géométrie non-commutative, on voudra déterminer la $K$-théorie des certaines $C^*$-algèbres, qui est l'analogue des théories homologiques classiques. 
\vspace{0.3 cm}
\begin{block}{Objectif de la thèse}
Déterminer les groupes de $K$-théorie de certaines familles de $C^*$-algèbres.
\end{block}
\end{frame}

\begin{frame}{Introduction}
Pour cela, voici notre stratégie :\\
\vspace{0.3 cm}
\begin{itemize}
\item[$\bullet$] détecter la filtration naturelles des produits croisés de $C^*$-algèbres par des groupoîdes étales et des groupes quantiques discrets,
\vspace{0.3 cm}
\item[$\bullet$] approximer la $K$-théorie par la $K$-théorie contrôlée,
\vspace{0.3 cm}
\item[$\bullet$] dans le cas des $C^*$-algèbres de Roe et des produits croisés de groupoïdes étales, construire des applications d'assemblage,
\vspace{0.3 cm}
\item[$\bullet$] prouver une version contrôlée de la conjecture de Baum-Connes.
\end{itemize}
\end{frame}
