\usepackage[frenchb,british]{babel}
\usepackage{amsfonts}
\usepackage{amsmath}
\usepackage{amssymb}
%\usepackage[T1]{fontenc}
\usepackage[utf8]{inputenc}
\usepackage{amsthm}
\usepackage{graphicx}
\usepackage{tikz}
\usepackage{tikz-cd}
\usepackage{hyperref}
\usepackage{amssymb}
\hypersetup{                    % parametrage des hyperliens
    colorlinks=true,                % colorise les liens
    breaklinks=true,                % permet les retours à la ligne pour les liens trop longs
    urlcolor= blue,                 % couleur des hyperliens
    linkcolor= blue,                
    citecolor= blue               % couleur des liens vers les references bibliographiques
}
%Commandes
\theoremstyle{definition}
\newtheorem{conj}{Conjecture}

% French style
\newtheorem*{definitionfr}{Definition}
\newtheorem*{propfr}{Proposition}
\newtheorem*{thmfr}{Theorem}
\newtheorem*{corfr}{Corollaire}
\newtheorem*{lemfr}{Lemme}
\newtheorem{exple}{Exemple}

\newcommand{\N}{\mathbb N}
\newcommand{\Z}{\mathbb Z}
\newcommand{\R}{\mathbb R}
\newcommand{\C}{\mathbb C}
\newcommand{\Hil}{\mathcal H}
\newcommand{\Mn}{\mathcal M _n (\mathbb C)}
\newcommand{\K}{\mathbb K}
\newcommand{\B}{\mathbb B}\newcommand{\Cat}{\mathbb B / \mathbb K}
\newcommand{\G}{\mathcal G }
\setlength\parindent{0pt}