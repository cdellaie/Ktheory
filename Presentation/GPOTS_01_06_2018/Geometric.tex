\begin{frame}
Let $A$ be a $\mathcal E$-filtered $C^*$-algebra, $c\geq 1$ and $F\in \mathcal E$.
\begin{definition}
 A $F$-controlled Mayer-Vietoris pair with coercivity $c$ is a quadruple $(V_0, V_1, A^{(0)}, A^{(1)})$:
\begin{itemize}
\item[$\bullet$] the $V_i$'s are closed subspaces of $A_F$,
\item[$\bullet$] $A^{(i)}$ is a $C^*$-algebra containing \[ V_i + A_{F'} V_i + V_i A_{F'}  + A_{F'} V_i A_{F'}\]
with $F' = F^5$,
\item[$\bullet$] for every $E\leq F$, every $x\in M_n(A_E)$ can be written as a sum \[x=x_0+x_1\] where $x_i\in M_n(V_i \cap A_E)$ and $|| x_i|| \leq c||x||$,
\item[$\bullet$] for every $\varepsilon>0$, $E\leq F$ and every $\varepsilon$-close elements $x\in A_E^{(0)}$ and $y\in A_E^{(1)}$, i.e.
\[|| x-y || < \varepsilon,\]
there exists $z\in M_n( A_E^{(0)}\cap A_E^{(1)})$ such that \[ ||x-z|| < c\varepsilon \quad \text{and} \quad ||y-z|| < c\varepsilon .\]
\end{itemize}
If $\mathcal A$ and $\mathcal B$ are two families of $\mathcal E$-filtered $C^*$-algebras, we say that $\mathcal A$ $2$-decomposes over $\mathcal B$ if there exists a constant $c\geq 1$ such that, for every $A\in\mathcal A$, and every $E\in \mathcal E$, there exists a controlled Mayer-Vietoris pair $(V_0, V_1, A^{(0)}, A^{(1)})$ with coercivity $c$ with $A^{(0)}$, $A^{(1)}$ and $A^{(0)} \cap A^{(1)}$ belonging to $\mathcal B$.
\end{definition} 
\end{frame}

\begin{frame}
If in possession of a controlled Mayer-Vietoris pair $(V_0, V_1, A^{(0)}, A^{(1)})$ for a filtered $C^*$-algebra $A$, Theorem $3.10$ of \cite{OY4} allows to compute its controlled $K$-theory in terms of the controlled $K$-theory of the sub-$C^*$-algebras $A_i$. See \cite{OY4},\cite{DellAieraThesis} or \cite{dell2017controlled} for precise definitions about controlled morphisms and controlled exact sequences. 

\begin{thmfr}
For every $\mathcal E$-filtered $C^*$-algebra $A$, $E\in \mathcal E$ and every $E$-controlled Mayer-Vietoris pair $(V_0, V_1, A^{(0)}, A^{(1)})$, there exists a controlled sequence
%\[\begin{tikzcd}
% \hat K_*( A^{(0)}\cap A^{(1)} ) \arrow{r} & \hat K_*(A^{(0)}) \oplus \hat K_*(A^{(1)}) \arrow{r} & \hat K_*(A) \arrow{d} \\ 
 %\hat K_*(A) \arrow{u} & \hat K_*(A^{(0)}) \oplus \hat K_*(A^{(1)}) \arrow{l} & \hat K_*( A^{(0)}\cap A^{(1)} ) \arrow{l}
%\end{tikzcd}\]
which is controlled-exact up to order $E$.  
\end{thmfr}  
\end{frame}

\begin{frame}
This result allows H. Oyono-Oyono and G. Yu to prove a permanence result (\cite{OY4}, Theorem $4.12$).

\begin{thmfr}
Let $A$ be a $\mathcal E$-filtered $C^*$-algebra. If for every $E\in \mathcal E$ there exists a $E$-controlled Mayer-Vietoris pair $(V_0, V_1, A^{(0)}, A^{(1)})$ such that $A^{(0)}$, $A^{(1)}$ and $A^{(0)} \cap A^{(1)}$ satisfy the quantitative Künneth formula then $A$ satisfies the quantitative Künneth formula.   
\end{thmfr}
\end{frame}

\begin{frame}
Let $\mathcal E$ be a coarse structure. A $\mathcal E$-filtered $C^*$-algebra $A$ is said to be locally bootstrap if, for every $E\in \mathcal E$, there exists $F\in \mathcal E$ and a sub-$C^*$-algebra $A^{(F)}$ of $A$, which is in the bootstrap class $\mathcal B$ and satisfies
\[A_E \subseteq A^{(F)}\subseteq A_F. \]
Notice the following property: a locally bootstrap $C^*$-algebra is automatically bootstrap. It is indeed an inductive limit of bootstrap $C^*$-algebras. Denote by $C_{fand}^{(0)}$ the class of locally bootstrap $C^*$-algebras. Then, a $C^*$-algebra $A$ belongs to the class $C^{(n+1)}_{fand}$ if it is $2$-decomposabe over $C_{fand}^{(n)}$. \\
\end{frame}

\begin{frame}
The asymptotic nuclear dimension of $A$ is the smaller $n$ such that $A$ belongs to $C^{(n)}_{fand}$, and we denote by $C_{fand}$ the class of $C^*$-algebras with finite asymptotic nuclear dimension,
\[ C_{fand}  = \cup_{n\geq 0} C_{fand}^{(n)}.\]

The two previous result combines in the main result of \cite{OY4}.
\begin{thmfr}
Let $A$ be a filtered $C^*$-algebra with finite asymptotic nuclear dimension. Then $A$ satisfies the Künneth formula. \end{thmfr}

As an application, one gets that the uniform Roe algebra of a coarse space with finite asymptotic dimension satisfies the Künneth formula.\\
\end{frame}
