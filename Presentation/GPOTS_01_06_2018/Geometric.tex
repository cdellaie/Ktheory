\begin{frame}
Suppose now that a groupoid can be decomposed in such a way at every order into subgroupoids whose reduced $C^*$-algebra satisfy the Künneth formula. The previous permanence result shows that the reduced $C^*$-algebra still satisfies the Künneth formula. This lead us to introduce the following notion.

\begin{definition}
Let $\mathcal G$ and $\mathcal H$ be two families of \'etale groupoids. \\

We say that $\mathcal G$ is $d$-decomposable over $\mathcal F$ if, for every groupoid $G$ in $\mathcal G$, every symmetric compact subset $E\subseteq G$, there exist a covering of $E^{(0)} = s(E)=r(E)$ by $d+1$ open subsets 
\[E^{(0)} = U_0 \cup ... \cup U_d \] such that the groupoids generated by $G_{|U_i} \cap E$ all belongs to the class $\mathcal H$.\\

We say that $\mathcal G$ is finitely $d$-decomposable over $\mathcal F$ if there exist finitely many classes 
\[\mathcal G= \mathcal F_0 \ , \ \mathcal F_1 \ , \ ... \ , \ \mathcal F_k = \mathcal F \] 
such that $\mathcal F_j$ is $d$-decomposable over $\mathcal F_{j+1}$ for every $j$. The smallest integer, if it exists, $k$ realizing this condition is called the relative dimension of $\mathcal G$ w.r.t. $\mathcal F$ and is denoted by $dim_{d,\mathcal F}(\mathcal G)$.  
\end{definition}

In \cite{GWY}, E. Guentner, R.Willett and G. Yu introduced the notion of \textit{dynamical asymptotic dimension} for an \'etale groupoid. Unravelling the definition, one gets that the dynamical asymptotic dimension of $G$ is less than $d$ iff \[dim_{d,\text{Cpt}} (\{G\}) \leq 1,\]
where Cpt is the class of compact \'etale groupoids.\\

From the permanence result, one also have that, if $\mathcal F$ is any family of \'etale groupoids whose reduced $C^*$-algebras satisfy the Künneth formula, and $G$ is an \'etale groupoid such that \[dim_{1, \mathcal F} ( \{G\} ) < \infty, \]
then $C_r^*(G)$ satisfy the Künneth formula.
\end{frame}
%%%%%%%%%%
%%%%%%%%%%
\begin{frame}

\begin{definition}
If $\mathcal A$ and $\mathcal B$ are two families of $\mathcal E$-filtered $C^*$-algebras, we say that $\mathcal A$ $2$-decomposes over $\mathcal B$ if there exists a constant $c\geq 1$ such that, for every $A\in\mathcal A$, and every $E\in \mathcal E$, there exists a controlled Mayer-Vietoris pair $(V_0, V_1, A^{(0)}, A^{(1)})$ with coercivity $c$ with $A^{(0)}$, $A^{(1)}$ and $A^{(0)} \cap A^{(1)}$ belonging to $\mathcal B$.
\end{definition} 
\end{frame}


