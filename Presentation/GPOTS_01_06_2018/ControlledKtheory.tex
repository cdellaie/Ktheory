\begin{frame}{$K$-théorie contrôlée}
Soit $\mathcal E$ une structure coarse et $A$ une $C^*$-algèbre $\mathcal E$ filtrée. On définit les $\varepsilon$-$E$-unitaires
\[U^{\varepsilon, E}(A)= \{u\in A_E \text{ t.q. } ||u^*u-1||<\varepsilon\text{ et }||uu^*-1||<\varepsilon \}\]
et les $\varepsilon$-$E$-projections 
\[P^{\varepsilon, E}(A)= \{p\in A_E \text{ t.q. } p=p^*\text{ et }||p^2-p||<\varepsilon \}.\]
\end{frame}

\begin{frame}{$K$-théorie contrôlée}
Comme en $K$-théorie, 
\begin{itemize}
\item[$\bullet$] $P_\infty^{\varepsilon, E}(A)$ est la limite inductive algébrique des $P_n^{\varepsilon, E}(A)$ par rapports aux inclusions
\[\left\{\begin{array}{rcl}
	P^{\varepsilon,E}_n(A) 		& \rightarrow	& P^{\varepsilon,E}_{n+1}(A)\\ 
	p 		& \mapsto 	& \begin{pmatrix}p& 0 \\ 0&0 \end{pmatrix}
\end{array}\right.\]
\item[$\bullet$] $U_\infty^{\varepsilon, E}(A)$ est la limite inductive algébrique des $U_n^{\varepsilon, E}(A)$ par rapports à
\[\left\{\begin{array}{rcl}
	U^{\varepsilon,E}_n(A) 		& \rightarrow	& U^{\varepsilon,E}_{n+1}(A)\\ 
	u 		& \mapsto 	& \begin{pmatrix}u & 0 \\ 0& 1 \end{pmatrix}
\end{array}\right. .\]
\end{itemize}
\end{frame}

\begin{frame}{$K$-théorie contrôlée}
On munit $P_\infty^{\varepsilon, E}(A)\times \N$ et $U_\infty^{\varepsilon, E}(A)$ des relations d'équivalence suivantes:
\begin{itemize}
\item[$\bullet$] $(p,l) \sim (q,l')$ s'il existe une homotopie de quasi-projections $h\in P^{\varepsilon, E}_\infty(A[0,1])$ et un entier $k$ tel que 
\[h(0)=\begin{pmatrix} p & 0 \\ 0 & 1_{k+l'} \end{pmatrix} \text{ and }
h(1)=\begin{pmatrix} q & 0 \\ 0 & 1_{k+l} \end{pmatrix}\]
\item[$\bullet$] $u \sim v$ s'il existe une homotopie de quasi-unitaires $h\in U^{3\varepsilon, E\circ E}_\infty(A[0,1])$ tel que $h(0)= u \text{ and }h(1)=v$.\\
\end{itemize}
\end{frame}

\begin{frame}{$K$-théorie contrôlée}
Si $A$ est unitale,
\begin{itemize}
\item[$\bullet$] $K_0^{\varepsilon,E}(A) = P^{\varepsilon, E}_\infty(A)\times \N / \sim_{\varepsilon,E}$ 
\item[$\bullet$] $K_1^{\varepsilon,E}(A) = U^{\varepsilon, E}_\infty(A) / \sim_{\varepsilon,E}$.
\end{itemize}

Si $A$ n'est pas unitale,  
\[K_0^{\varepsilon,E}(A) = \{[p,l]_{\varepsilon,E} : p\in P^{\varepsilon,E}_\infty (\tilde A), l\in \N \text{ s.t. rank}(\kappa_0(\rho_A(p)))=l \}\]
et $K_1^{\varepsilon,E}(A)$ est défini par $U_\infty^{\varepsilon,E}(\tilde A)/ \sim_{\varepsilon,E}$.\\

\begin{definitionfr}
La $K$-théorie contrôlée d'une $C^*$-algèbre filtrée $(A,\mathcal E)$ est la famille de groupes abéliens 
\[\hat K_0(A) = (K_0^{\varepsilon,E}(A))_{\varepsilon\in (0,\frac{1}{4}),E\in\mathcal E} \text{ et } \hat K_1(A) = (K_1^{\varepsilon,E}(A))_{\varepsilon\in (0,\frac{1}{4}),E\in\mathcal E}.\]
\end{definitionfr}
\end{frame}

\begin{frame}{$K$-théorie contrôlée}
$\bullet$ Pour tout $\varepsilon <\varepsilon'$ et $E\subseteq E'$, on dispose de morphismes 
\[\iota_{\varepsilon,E}^{\varepsilon',E'} : K^{\varepsilon,E}(A)\rightarrow K^{\varepsilon',E'}(A) \]
tels que $\iota_{\varepsilon',E'}^{\varepsilon'',E''}\circ \iota_{\varepsilon,E}^{\varepsilon',E'} = \iota_{\varepsilon,E}^{\varepsilon'',E''}$ et $\iota_{\varepsilon,E}^{\varepsilon,E}= id_{K^{\varepsilon,E}(A)}$.\\
\vspace{0.3 cm}
$\bullet$ Pour tout $\varepsilon $ et $E\in\mathcal E$, on dispose de morphismes 
\[\iota_{\varepsilon,E} : K^{\varepsilon,E}(A)\rightarrow K(A) \]
tels que $\iota_{\varepsilon',E'}\circ \iota_{\varepsilon,E}^{\varepsilon',E'} = \iota_{\varepsilon,E}$.\\
\vspace{0.3 cm}
$\bullet$ Pour tout élément $x\in K(A)$ et tout $\varepsilon\in (0,\frac{1}{4})$, il existe $E\in\mathcal E$ et $y\in K^{\varepsilon,E}(A)$ tel que $x=\iota_{\varepsilon,E}(y)$.
\end{frame}

\begin{frame}{$K$-théorie contrôlée}
$\bullet$ Une paire de contrôle $(\alpha,\rho)$ est donnée par $\alpha\in (0,\frac{1}{4})$ et $k : (0,\frac{1}{4\alpha})\rightarrow \N $ croissante.
\begin{definitionfr}[Morphisme contrôlé]
Un morphisme $(\alpha,k)$-contrôlé $\hat F : \hat K(A) \rightarrow \hat K(B)$ est une famille de morphismes 
\[F^{\varepsilon,E}: K^{\varepsilon,E}(A) \rightarrow K^{\alpha\varepsilon,k_\varepsilon. E}(B)\] 
compatible avec les $\iota_{\varepsilon,E}^{\varepsilon',E'}$.
\end{definitionfr}
\vspace{0.3 cm}
On dit que $\hat F$ induit $F : K(A)\rightarrow K(B)$ en $K$-théorie si 
\[\iota_{\alpha \varepsilon,k_\varepsilon E}\circ F^{\varepsilon,E}=F.\]
\textbf{Remarque :} Notion d'isomorphisme contrôlé et de suite exacte contrôlée.
\end{frame}

\begin{frame}{$K$-théorie contrôlée}
$\bullet$ Tout $*$-homomorphisme filtré $\phi; A\rightarrow B$, i.e. $\phi(A_E)\subseteq B_E$, induit un morphisme contrôlé 
\[\phi_* : \hat K(A)\rightarrow \hat K(B).\]
$\bullet$ Morita équivalence : $A \rightarrow A\otimes\mathfrak K ; a\mapsto a\otimes e$ induit un isomorphisme
\[K^{\varepsilon,E}(A)\rightarrow K^{\varepsilon,E}(A\otimes\mathfrak K).\]
$\bullet$ Pour toute suite exacte complètement filtrée, il existe une suite exacte contrôlée à $6$ termes, ainsi qu'un bord contrôlé.\\
\vspace{0.3 cm}
$\bullet$ Périodicité de Bott contrôlée : $p\mapsto 1+(e^{2i\pi }-1)p$ induit un isomorphisme contrôlé
\[\beta_A :\hat K_0(A) \rightarrow  \hat K_1(SA).\]

%il existe une paire de contrôle $(\alpha_\beta,k_\beta)$ et un morphisme $(\alpha_\beta,k_\beta)$-contrôlé
%\[\beta_A : K^{\varepsilon, E}(SA)\rightarrow K^{\alpha_\beta\varepsilon, k_\beta(\varepsilon) . E}(A)\]
%qui admet un inverse contrôlé $D_A$.
\end{frame}