
%%%%%%%%%%%%%%%%%%%%%%%%%%%
\section{A list of books}
  
A list of books I like about general knowledge in science:

\begin{itemize}
\item[$\bullet$] L'aventure des nombres, Godefroy
\item[$\bullet$] L'autobigraphie de Paul Levy, Laurent Schwartz, et Yuri Manin.
\item[$\bullet$] Recoltes et semailles, Grothendieck.
\item[$\bullet$] Lee Smolin, The trouble with physics, the rise of String theory, the fall of a Science, and what comes next,
\item[$\bullet$] Julian Barbour, The End of Time, The next revolution in Physics,
\item[$\bullet$] Carlo Rovelli, Et si le temps n'existait pas, un peu de science subversive,
\item[$\bullet$] Mandlebrot, The (Mis)Behaviour of markets, Fractals and Chaos, the Mandelbrot set and beyond, The fractal geometry of nature.
\item[$\bullet$] Manjit Kumar:
\item[$\bullet$] Amir Alexander, Infinitesimal: How a Dangerous Mathematical Theory Shaped the Modern World
\item[$\bullet$] Ian Stewart, Does God play dice?
\item[$\bullet$] History of Statistics, Stielger
\item[$\bullet$] Logicomix\\
\end{itemize}

Overview and more specialized books:
\begin{itemize}
\item[$\bullet$] Moonshine beyond the Monster, Terry Gannon
\item[$\bullet$] Le theoreme d'uniformisation, Saint-Gervais
\item[$\bullet$] Invitation aux mathematiques de Fermat, Hellgouarch
\item[$\bullet$] Rached Mneime, tous ses livres!
\item[$\bullet$] Hubbard West pour les equa diff
\item[$\bullet$] Noether's theorem, Yvette K
\item[$\bullet$] Nother's wonderful theorem
\item[$\bullet$] The annus mirabellus of Einstein
\item[$\bullet$] The Road to Reality, Sir Roger Penrose \\
\end{itemize}

Books about Einstein: 
\begin{itemize}
\item[$\bullet$] Subtle is the Lord, Abraham Pais \cite{Pais1982}; biography of Einstein by someone who knew him;
\item[$\bullet$] Einstein's miraculous year: Five papers that changed the face of physics, Penrose \& Einstein \cite{Penrose2005Einstein}; English translations of the five papers Einstein published in 1905 while working at the patent office in Bern.  
\item[$\bullet$] Quantum: Einstein, Bohr, and the great debate about the nature of reality, Kumar \cite{Kumar}; history of quantum theory from Planck's blackbody radiation to the EPR paradox.
\end{itemize}
%%%%%%%%%%%%%%%%%%%%%%%%

%%%%%%%%%%%%%%%%%%%%%
\section{Groups}
%%%%%%%%%%%%%%%%%%%%%

\begin{itemize}
\item[$\bullet$] Amenable, a-T-menable, property T, with a diagram
\item[$\bullet$] Mapping class groups 
\item[$\bullet$] Profinite groups, locally profinite groups, $Aut(\overline{\mathbb Q} /\mathbb Q)$
\item[$\bullet$] Automorphism of a regular tree, the Grigorchuk group,
\item[$\bullet$] Lamplighter groups, usually 
\[\Z_2 \wr \Z = \oplus \Z_2 \rtimes \Z.\]
More generally, wreath products $ H\wr \Gamma = H^\Gamma\rtimes \Gamma$. 
\item[$\bullet$] $\begin{pmatrix}1 & 0 \\ 2 & 1\end{pmatrix}$ and $\begin{pmatrix}1 & 2 \\ 0 & 1\end{pmatrix}$ generate a free group of finite index in $SL(2,\Z)$. The corresponding semi-direct product $\Gamma = \Z^2\rtimes \mathbb F_2$ does not have Haagerup's property ( $(\Gamma, \Z^2)$ has relative property (T) ).\\

\item[$\bullet$] Cayley graphs: finite groups, symmetric groups, $\Z^2$, $\mathbb F_2$, $\Z$ with original generating sets. $B(1,2)$. Lamplighter groups: meta-abelian without finite presentation. \\

$SL(2,\Z)$ has presentation 
\[<x,y \ | \ x^4=1, x^2=y^3> \quad p,q \geq 1,\] 
and in this presentation, the quotient by $<x^2>$ is isomorphic to \[PSL(2,\Z)\cong \Z_2 \ast \Z_3.\] This gives a way to draw their Cayley graph easily.\\

\item[$\bullet$] Baumslag-Solitar monster
\[BS_{p,q} = <a,b \ | \ ab^p a^{-1} = b^q>\]
are nonhopfian when $p$ and $q$ are coprime and at least $2$. $BS_{p,q}$ is the Higman-Neumann-Neumann extension $HNN(\Z,p \Z, \theta)$ where $\theta(p)=q$: 
\[BS_{p,q} < Aut(T_{p+q}) \quad \text{where } T_{p+q}\text{ homogeneous tree of degree }p+q.\]
On the other hand, one has a non injective morphism $BS_{p,q}\rightarrow Aff(\R); a \mapsto \frac{qx}{p} ; b\mapsto x+1$. The diagonal morphism $BS_{p,pq} \rightarrow Aut(T_{p+q})\times Aff(\R)$ has discrete image and $BS_{p,q}$ has Haagerup's property (because both $Aut(T_{p+q})$ and $Aff(\R)$ have it.\\

\item[$\bullet$] Infinite torsion questions: subgroups of $GL(n, \Z[\frac{1}{p}])$.\\
\item[$\bullet$] Tarski monsters: $p$ a prime, then every $x$ generates a cyclic subgroup of order $p$, and the set of $x$ together with any element not contained in this cyclic subgroup subgroup generates $\Gamma$. 
\end{itemize}

Tessera and Arhantseva showed that there exists a group which is a split extension of two groups that are coarsely embeddable into Hilberrt space, and that does not admit such an embedding.\\

\begin{itemize}
\item[$\bullet$] \textbf{Amenability} Abelian, Compact, extension of such (Elementary amenable), Grigorchuk group: amenable but not elementary amenable (first example of finitely generated group with intermediate growth, i.e. faster than polynomial but subexponential). Every group with subexponential growth (equivalent to virtually nilpotent by Gromov's Polynomial growth theorem). When discrete, $\Gamma$ is amenable iff $C^*_r(\Gamma)$ is nuclear.\\

In terms of CP functions? $\Gamma$ is amenable iff there exists a net of commpactly supported continuous positive definite functions converging pointwise to $1$.\\

\item[$\bullet$] \textbf{Haagerup's property} Introduced by Haagerup on his work on the Free groups. Incidentatly, $\mathbb F_2$ is not amenable but has Haagerup's property. Stability by closed subgroups so $\mathbb F_ n$ and the free group with countably many generators. Equivalent to Gromov's a-T-menability and property FH in the locally compact case. Every such group satisfies the Baum-Connes conjecture with coefficients, and is $K$-amenable, i.e. 
\[\lambda \in KK_0(C_{max}(\Gamma), C^*_r(\Gamma))\] 
is invertible. Same for $SL(2,\Z)$. Amenable groups, Coxeter groups, Groups acting metrically properly on trees or spaces with walls. $SU(n,1)$ and $SO(n,1)$: $g\mapsto d(gx_0,x_0)$ is conditionally negative and definite, where $d$ is the hyperbolic distance and $x_0$ any point in real or complex projective space. Baumslag-Solitar 's groups $BS_{p,q}$.\\

In terms of CP functions? $G$ has Haagerup's property iff there exists a continuous proper conditionally negative definite function $G\rightarrow \R_+$, iff there exists a sequence of continuous normalized positive definite functions converging uniformly on compact subsets of $G$. \\

\item[$\bullet$] \textbf{Property T} Any compact group. $SL(n,\Z)$ for $n\geq 3$. Simple real Lie groups with real rank $\geq 2$ and their lattices: $SL(n,\R)$, $n\geq 3$; $SO(p,q)$, $p>q\geq 2$; $SO(p,p)$, $p\geq 3$. Simple algebraic groups of rank $\geq 2$ over a local field. $Sp(n,1)$, $n\geq 2$ which is a simple real Lie group of real rank $1$, and its lattices, which are dicrete countable hyperbolic groups. $Aut (\mathbb F_5)$. Mapping class groups are supposed to have property (T), but the proof is still not clear and contains gaps. \\ 
\[\text{Property (T) } + \text{ Haagerup } = \text{ Compact. } \]
In terms of CP functions? $\Gamma$ has (T) iff every sequence of continuous normalized positive definite functions that converges uniformly on compact subsets to $1$ converges uniformly to $1$. Or iff every continuous conditionally negative definite function on $\Gamma$ is bounded.\\

\item[$\bullet$] \textbf{Asymptotic dimension} $\text{asdim }|\Gamma| = \text{dim}_{nuc}( C_u^*(\Gamma)) $. $\Z^n$ of asymptotic dimension $n$. Asymptotic dimension of a tree is one: $asdim(\mathbb F_n) = 1$. Hyperbolic groups (trees from far away) are of finite asymptotic dimension (which can be arbitrarily large). Finitely generated solvable goups such that the abelian quotients are finitely generated have finite asymptotic dimension. Example of such: the group 
\[Sol = \Z^2\rtimes_A \Z \quad \text{where}\quad \begin{pmatrix} 2 & 1 \\ 1 & 1 \end{pmatrix},\]
\[ \{e\} < \Z < \Z^2 < Sol \quad \text{ with } Sol/ \Z^2 \cong \Z.\]
Every almost connected Lie group has finite asymptotic dimension, and any of their discrete subgroup. For instance $SL(n,\Z)$ for every $n$. Mapping class groups have finite asymptotic dimension. \\

All finite asymtotic dimension groups satisfy the Novikov conjecture.\\

The groups $\Z^{(\infty)}=\bigoplus_{j=0}^\infty \Z$ with $d(x,y)=\sum j |x_j -y_j|$ and $\Z \wr \Z$ have infinite asymptotic dimension.\\

\item[$\bullet$] \textbf{FDC} $\Z^{(\infty)}$ has FDC and infinite asymptotic dimension, but is not finitely generated. The following subgroup of $SL(2,\R)$ has FDC, infinite asymptotic dimension and is finitely generated:  
\[G= \left\{\begin{pmatrix} \pi^n  & P(\pi) \\ 0 & \pi^{-n}\end{pmatrix} | n\in \Z, P \text{ Laurent polynomial with integer coefficients}\right\},\]
with $\left\{\begin{pmatrix} 1  & P(\pi) \\ 0 & 1\end{pmatrix} \right\} \cong \Z \wr \Z $ as a subgroup (so infinite dimension). Any countable subgroup of $GL(n,R)$ for $R$ a commutative ring has FDC, countable subgroups of almost connected Lie groups, elementary amenable, finite asymptotic dimension and hyperbolic, all have FDC.
\\

\item[$\bullet$] \textbf{Property A} $|\Gamma|$ has property (A) iff $C_u^*(\Gamma)$ is nuclear (Ozawa, but Guentner-Kaminker...) iff $\beta \Gamma \rtimes \Gamma$ is amenable. Non-equivariant version of Haagerup's property. All FDC groups have (A).\\
\item[$\bullet$] \textbf{Coarsely embeddable into Hilbert space} $|\Gamma|$ coarsely embeds iff $\beta \Gamma \rtimes \Gamma$ is a-T-menable.\\
\end{itemize}

Other properties: hyperbolicity in Gromov's sense, $K$-amenability, polya-$\mathcal P$ (polyabelian = solvable?, polycyclic,..), virtually abelian or nilpotent, Rapid decay property,... Exactness: Gromov's monsters are the only groups known not to be exact. $C^*$-simplicity: nonabelian Free groups, \\

\newpage

%%%%%%%%%%%%%%%%%%%%%%%%%%%%%%%%%%%%%%%%%%%%%%%%%%%%%%%%%%%%%%%%%%%%%%%%%%%
\[\begin{tikzpicture}[node distance=1cm, auto,]
 %nodes
\node[punkt] (CEH) {Coarse embedding into Hilbert space};\\
\node[above=of CEH] (dummy) {};
\node[punkt, right=of dummy] (aTm) {Haagerup}
	edge[pil, bend left=45] node[auto] {} (CEH.east);
\node[punkt,left=of dummy] (A) {Property (A)}
	edge[pil, bend right=45] node[auto] {} (CEH.west); 
\node[punkt, above=of A] (FDC) {Finite decomposition complexity};
\node[punkt, above=of FDC] (asdim) {Finite asymptotic dimension};
\node[punkt,right=of asdim] (fin) {Finite};

\node[punkt, above=of aTm] (am) {Amenability};

\draw[vecArrow] (asdim) to (FDC);
\draw[vecArrow] (FDC) to (A);
\draw[vecArrow] (am) to (aTm);
\draw[vecArrow] (fin) to (asdim);
\draw[vecArrow] (fin) to (am);

\end{tikzpicture}\]
%%%%%%%%%%%%%%%%%%%%%%%%%%%%%%%%%%%%%%%%%%%%%%%%%%%%%%%%%%%%%%%%%%%%%%%%%%%
%\vspace{0.5in}\\
Stability:

\begin{table}[h]
\begin{tabular}{c|c|c|c|c|}
\cline{2-5}
                                                         & \textbf{Amenablitiy} & \textbf{Haagerup} & \textbf{(T)}            & \textbf{Baum-Connes}  \\ \hline
\multicolumn{1}{|l|}{\textbf{Product}}                   &            Yes          &                   &                      &                       \\ \hline
\multicolumn{1}{|l|}{\textbf{Subgroups}}                 &            Yes          &       No          & No, $\Z < SL(3,\Z)$   &                       \\ 
\multicolumn{1}{|l|}{}                                   &                         &   Closed yes      & Finite index yes   &                       \\ \hline
\multicolumn{1}{|l|}{\textbf{Quotients}}                 &            Yes          &                   &         Yes          &                       \\ \hline
\multicolumn{1}{|l|}{\textbf{Extensions}}                &            Yes          &                   &                      &                       \\ \hline
\multicolumn{1}{|l|}{\textbf{Direct limits}}             &            Yes          &                   &                      &                       \\ \hline
\multicolumn{1}{|l|}{\textbf{Free products}}             &No, $\mathbb F_2 = \Z \ast \Z$   &           &                      &                       \\ \hline
\multicolumn{1}{|l|}{\textbf{HNN extensions}}            &                      &                      &                      &                       \\ \hline
\multicolumn{1}{|c|}{\textbf{Free amalgamated }}         &                      &                      &                      &                       \\ 
\multicolumn{1}{|c|}{\textbf{ products}}                 &                      &                      &                      &                       \\ 
\hline
\end{tabular}
\end{table}

\begin{table}[h]
\begin{tabular}{c|c|c|c|}
\cline{2-4}
                                                         & \textbf{Finite Asymptotic }              & \textbf{FDC} & \textbf{(A)}    \\ 
							 & \textbf{dimension}                       &              &              \\ \hline
\multicolumn{1}{|c|}{\textbf{Product}}                   &              Yes     &            Yes    &                             \\ \hline
\multicolumn{1}{|c|}{\textbf{Subgroups}}                 &              Yes     &           Yes     &                             \\ \hline
\multicolumn{1}{|c|}{\textbf{Quotients}}                 &                      &                   &     By amenable subgroups   \\ \hline
\multicolumn{1}{|c|}{\textbf{Extensions}}                &            Yes       &            Yes    &     Yes                     \\ \hline
\multicolumn{1}{|c|}{\textbf{Direct limits}}             &                      &                   &     Yes                     \\ \hline
\multicolumn{1}{|c|}{\textbf{Direct unions}}             & No, $\Z ^{(\infty)}$ &            Yes    &                             \\ \hline
\multicolumn{1}{|c|}{\textbf{Free products}}             &                      &                   &     Yes                     \\ \hline
\multicolumn{1}{|c|}{\textbf{HNN extensions}}            &          Yes         &             Yes   &                             \\ \hline
\multicolumn{1}{|c|}{\textbf{Free amalgamated }}         &           Yes        &              Yes  &                             \\ 
\multicolumn{1}{|c|}{\textbf{ products}}                 &                      &                   &                             \\ \hline
\end{tabular}  
\end{table}

\newpage
Other group like objects, but with less properties.

\subsection{Groupoids}

\begin{itemize}
\item[$\bullet$] The \textit{coarse groupoid} $G(X)$: \'etale (even ample) with totally disconnected basis $\beta X$. Dynamical asymptotic dimension of asymptotic dimension of $X$. Amenable iff $X$ has property $A$. A-T-menable iff $X$ coarsely embeds into Hilbert space. 
\item[$\bullet$] HLS groupoid associated to a sequence of finite metric spaces $X_n$ equipped with maps $X_n \rightarrow \Gamma$ to a finitely generated group $\Gamma = \langle S \rangle$. 
\item[$\bullet$] Groupoids of germs of semigroup of partial homeomorphisms acting on a topological space
\item[$\bullet$] Full topological groups of an ample second-countable groupoid with compact base space
\item[$\bullet$] Tillings groupoids ($\Omega\rtimes G$, usually amenable)
\item[$\bullet$] Holonomy groupoids of a foliation
\item[$\bullet$] Action groupoids $X\rtimes \Gamma$, principal bundles groupoids $P\times_G P$, where $P \rightarrow X$ is a $G$-bundle
\item[$\bullet$] Equivalence relation groupoids
\end{itemize}

If $G$ is \'etale, $G$ is amenable iff $C^*_r(G)$ is nuclear. Amenability implies that the full $C^*$-algebra and the reduced coincides, but the converse is false by a result of Willett \cite{Willett2015non}.

\subsection{Quantum groups}

One of the most useful ideas used by quantum groups theorists is to try and adapt concepts from geometric group theory in their setting. We could think of it as a way to algebraize notions like amenability, a-T-menability, etc. In the case of a discrete group $\Gamma$, it is known \cite{GWY} that
\[asdim(\Gamma) = dim_{nuc} (l^\infty (\Gamma) \rtimes_r \Gamma).\] 
By analogy, define \[asdim(\hat G) = dim_{nuc} (l^\infty (\hat G) \rtimes_r \hat G)\]
for a discrete quantum group $(\hat G, \hat \Delta)$. Here
\[l^\infty(\hat G) = \prod_{x\in Irr(G)} B(H_x)\]
is naturally a $\hat G$-algebra. (Describe explicitely the action and give the example of $\hat{SU(2)}$.)\\

Remark post-discussion with Mehrdad: In general, $l^\infty (\hat G) \rtimes_r \hat G$ is not exact so its nuclear dimension is not finite. Such a definition is thus hopeless.\\

The natural filtration of any crossed-product of a $\hat G$-algebra by $\hat G$ is given by the \textit{coarse structure} $\mathcal E_G$ of the finite dimensional symmetric representations of the compact dual $G$. This suggests that the \textit{coarse geometry} of the discrete quantum group $\hat G$ is encoded in $\mathcal E_G$. Indeed, the first thing one can do is to define a notion of $S$-separation for $x,y\in Irr(G)$ and $S\subseteq Irr(G)$:
\[(x,y )\in \Delta_S  \text{ iff } \Delta(p_x) (p_y \otimes p_S)\neq 0.\] 

\begin{itemize}
\item[$\bullet$] is it true that $asdim(\hat G)=d$ iff for every $R\in \mathcal E_G$, there exists a partition
\[Irr(G) = U_0 \coprod U_1 \coprod \ ... \ \coprod U_d\]
such that each $U_i$ is a disjoint union $\coprod_j U_{ij}$ of uniformly bounded subsets $R$-separated:
\begin{enumerate}
\item there exists $S\in \mathcal E_G$ such that $(x,y)\in \Delta_S$ for every $x,y \in U_{ij}$,
\item if $x\in U_{ij}$ and $y\in U_{ik}$, $j\neq k$, then $(x,y)\neq \Delta_R$.\\
\end{enumerate}

\item[$\bullet$] in the presence of finite asymptotic dimension, do we get a controlled Mayer-Vietoris pair? 
\[ \prod_{x\in U_i} B(H_x) \rtimes_r \hat G\]

\item[$\bullet$] Define an assembly map for $l^\infty(\hat G)\rtimes_r \hat G$, and try to prove it is an isomorphism with \textit{controlled cutting and pasting} techniques.
\end{itemize}

\newpage

%%%%%%%%%%%%%%%%%%%%%%%%%%%
\section{$C^*$-algebras}
%%%%%%%%%%%%%%%%%%%%%%%%%%%

How to construct $C^*$-algebras?\\
\begin{itemize}
\item[$\bullet$] Finitely generated/presented $C^*$-algebras? You have to get \textit{bounded relations}. (See Loring's book, \textit{Lifiting solutions to perturbing problems in $C^*$-algebras} \cite{Loring}.) \\

\item[$\bullet$] Basic blocks: commutative $C_0(X)$, finite dimensional or matrix blocks $\oplus M_{d_k}(\C)$, $B(H)$ and its essential ideal $\mathfrak K(H)$, the Calkin quotient $Q(H)$,\\

\item[$\bullet$] Sum and tensor product\textbf{s}.\\ 
\item[$\bullet$] Convolution algebras.\\
\item[$\bullet$] Crossed product by an action by automorphisms by a group-like object: groups, groupoids, semi-groups, quantum groups. Stress that it is a kind of semi-direct product in the category of $C^*$-algebras: for instance, $A\rtimes_r \Gamma$ can be defined as a particular completion of the algebraic tensor product
\[A\otimes_{alg} C^*_r(\Gamma)\]
where the product is not the usual one, but twisted by the action of $\Gamma$ on $A$, or, it is the same, the coaction of $C^*_r(\Gamma)$ on $A$.\\
\end{itemize}

Example of $C^*$-algebras:\\

\begin{itemize}
\item[$\bullet$] CAR algebra $C^*\langle a_i , a_i a_j +a_j a_i = \delta_{ij} \rangle $ or $\bigotimes M_2(\C)$ or 
\[\varinjlim \ 
\left\{\begin{array}{rcl} 
M_{2^n}(\C) & \rightarrow & M_{2^{n+1}}(\C) \\ 
a           & \mapsto     & \begin{pmatrix} a & 0 \\ 0 & a\end{pmatrix}\end{array}\right.\]
\end{itemize}

Class of $C^*$-algebras.\\

\begin{itemize}

\item[$\bullet$] Nuclear $C^*$-algebras. Finite dimensional, commutative, AF-algebras are nuclear. $C_r^*(\Gamma)$ is nuclear iff $\Gamma$ is amenable (true for a discrete group, and more generally an \'etale groupoid). $C^*_u (X)$ is nuclear iff $X$ has Yu's property $A$. So $C_r^*(\mathbb F_2)$ is not nuclear but $l^\infty (\mathbb F_2) \rtimes_r \mathbb F_2$ is, giving an instance where nuclarity fails to pass to subalgebras.\\

\item[$\bullet$] The bootstrap class $\mathcal B$. If $\Gamma$ has Haagerup's property, then $C^*_r(\Gamma)$ is Bootstrap. It is a specialization of a famous result of Tu (\cite{TuThese}, lemma 10.6) that if $G$ is a-T-menable, there exists a $G$-proper $C^*$-algebra $A$, $KK^G$-equivalent to $\C$, such that $A\rtimes_r G$ is Bootstrap. \\

\item[$\bullet$] The class $\mathcal N$ of $C^*$-algebras $A$ such that the map 
\[\alpha_{A,B}: K_*(A)\otimes K_*(B) \rightarrow K_*(A \otimes B)\]
is an isomorphism for every $C^*$-algebra $B$ such that $K_*(B)$ is a free abelian group. In \cite{schochetRosenberg}, it is shown that $\mathcal N$ contains all of the bootstrap class.\\

\item[$\bullet$] Exact $C^*$-algebras. We say that $\Gamma$ is exact if $C_r^*(\Gamma)$ is exact. It is shown by Ozawa (completing work of Guentner and Kaminker \cite{GuentnerKaminkerExactness}) in \cite{OzawaExact} that $\Gamma$ is exact iff $\beta \Gamma \rtimes \Gamma$ is amenable iff $C^*_u(|\Gamma|)$ is nuclear.\\

\item[$\bullet$] Non exact $C^*$-algebras: \\

\begin{itemize}
\item[$\bullet$] For any integer sequence $k_n$ which tends to $\infty$ as $n$ goes to $\infty$, 
\[ \prod_{n\geq 0} M_{k_n} \]
is not exact. As a result, for any discrete quantum group $\hat G$ which is truely noncommutative, $l^\infty(\hat G)$ is not exact. So is any of its crossed-product, so that the naive definition of the uniform-Roe algebra 
\[l^\infty (\hat G) \rtimes_r \hat G\] 
is not exact, hence not nuclear.\\

\item[$\bullet$] The reduced $C^*$-algebra $C_r^*(\Gamma) $ of a finitely generated group whose Cayley graph contains expander. Using Ozawa's result \cite{OzawaExpander}, one can construct finite dimensional $C^*$-algebras $M_{X_n}$ such that 
\[ 0 \rightarrow C_r^*(\Gamma) \otimes \bigoplus M_{X_n} \rightarrow C_r^*(\Gamma) \otimes \prod M_{X_n} \rightarrow C_r^*(\Gamma) \otimes \left(\prod M_{X_n} /  \bigoplus M_{X_n} \right) \rightarrow 0\] is not exact in the middle.\\ The problem of the existence of such a group is an interesting question, which was stated by Gromov and proved rigorously by several people in the wake of this.\\	
\item[$\bullet$] If $\Gamma$ is a discrete group with Kirchberg's approximation property, then $\Gamma$ is amenable is equivalent to the maximal $C^*$-algebra $C^*(\Gamma)$ is exact (\cite{BrownOzawa}, prop 3.7.11). \\

Any residually finite group satisfies Kirchberg's approximation property, hence \textit{any nonamenable residually finite group has a non-exact maximal $C^*$-algebra.} For instance, $C^*(\mathbb F_n)$ and $C^*(SL(n,\Z))$, $n\geq 2$, are not-exact, for the reason that
\[\begin{tikzcd}0 \arrow{r} & J \otimes_{min} C^*(\Gamma) \arrow{r} & C^*(\Gamma) \otimes_{min} C^*(\Gamma)  \arrow{r} & C^*_r(\Gamma)\otimes_{min} C^*(\Gamma)  \arrow{r} & 0 \end{tikzcd}\]
is not exact, where $J = \text{ ker } C^*(\Gamma) \rightarrow  C_r^*(\Gamma)  $.\\

 Recall that $\Gamma$ has Kirchberg's approximation property if 
\[\lambda \times \rho : C^*(\Gamma) \otimes_{alg} C^*(\Gamma) \rightarrow B(l^2 \Gamma) \]
is min-continuous, i.e. extends to $C^*(\Gamma) \otimes_{min} C^*(\Gamma)$.\\
\end{itemize}

\end{itemize}

One can define analog of approximation properties in the setting of $K$-theory. \\

\begin{itemize}
\item[$\bullet$] $A$ is $K$-nuclear if the class of the natural map
\[ p_{A,B} : A\otimes_{max} B \rightarrow A\otimes_{min} B \]
is invertible as an element of $KK(A\otimes_{max} B,A\otimes_{min} B)$.\\
\item[$\bullet$] $G$ is $K$-amenable if the class of the regular representation
\[ \lambda_{G} : C_{max}^*(G) \rightarrow C_{r}^*(G) \]
is invertible as an element of $KK(C_{max}^*(G),C_{r}^*(G))$.\\
\end{itemize}

For instance, Skandalis proves in \cite{SkandalisNotion} that, if $\Lambda$ is an infinite hyperbolic property T group, then $C_r^*(\Lambda)$ is not $K$-nuclear. In particular, it is not $KK$-equivalent to a nuclear $C^*$-algebra, and cannot be Bootstrap. This completely renders proving the Baum-Connes conjecture by mean of Dirac-Dual-Dirac method hopeless. An example of such a group is given by any lattice in $Sp(n,1)$ for instance. (higher rank algebraic semisimple groups?) \\

After developing a restriction principle for groupoids, a natural question was to find a $C^*$-algebra coming from a groupoid crossed-product that we were able to prove that it satisfied the Künneth formula, while still not being a consequence of previous results. One could have started with the so called HLS groupoid $G_{\mathcal N}(\Gamma)$ associated to a residually finite finitely generated group $\Gamma$ and a nested sequence of decreasing  normal sugroups of finite index $\mathcal N$.\\

One always has the following exact sequence of $*$-algebras
\[ 0 \rightarrow \oplus \C[\Gamma_n] \rightarrow C_c(G) \rightarrow \C [\Gamma]  \rightarrow 0\]      
which induces the following exact sequence of $C^*$-algebras
\[ 0 \rightarrow \oplus \C[\Gamma_n] \rightarrow C_r^*(G) \rightarrow C^*_{\mathcal N} (\Gamma)  \rightarrow 0\]   
where $C^*_{\mathcal N}(\Gamma)$ is the completion of $\C[\Gamma]$ w.r.t. to the norm
\[||x||_{\mathcal N} = \sup_{N\in \mathcal N} ||\lambda_{N} (x)||\quad x\in \C[\Gamma] \]
induced by the quasi-regular representations $\lambda_{N} : C_{max}^*(\Gamma) \rightarrow \mathcal L(l^2(\Gamma/ N))$.  \\ 

Now this exact sequence intertwines the Baum-Connes assembly maps, and the Baum-Connes conjecture for $G_{\mathcal N}(\Gamma)$ is equivalent to $\mu_{\Gamma,\mathcal N}$ being an isomorphism. \\

\begin{itemize}
\item[$\bullet$] If $\Gamma= \mathbb F_2$ and 
\[N_n = \cap ker \phi \]
for $\phi$ running accross all group homomorphisms from $\Gamma$ to a finite group of cardinality less than $n$, then $C_{\mathcal N}^*(\Gamma) \cong C_{max}^*(\Gamma)$ and $G$ satifies the Baum-Connes conjecture, is ample and satisfies the restriction condition. So we get that $C_r^*(G)$ satisfies the Künneth formula. It is still a result that one can get using the fact that $\Gamma$ being a-T-menable, it is $K$-amenable. Hence $C^*_{max}(\Gamma)$ and $C_r^*(\Gamma)$ are $KK$-equivalent and bootstrap, so that $C_r^*(G)$ also is by extension stability of bootstrapness. A remark of R. Willett is worth mentioning: $\mathbb F_2$ being the fundamental group of the wedge of two circles, it is $KK$-equivalent to $C(\mathbb S^1 \wedge \mathbb S^1)$.\\
\item[$\bullet$] One can artificially try to get rid of bootstrapiness by spatially tensoring this exact sequence by $C_r^*(\Lambda)$ for a infinite hyperbolic property T group. One then get the extension
\[ 0 \rightarrow \oplus \C[\Gamma_n] \otimes_{min} C_r^*(\Lambda) \rightarrow C_r^*(G\times \Lambda) \rightarrow C^*_{\mathcal N} (\Gamma)\otimes_{min} C_r^*(\Lambda)   \rightarrow 0.\]
The restriction principle applies for the groupoid $G_{\mathcal N}(\Gamma)\times\Lambda$, and induces that its reduced $C^*$-algebra satisfies the Künneth formula. But then again, one can deduce this from a previous result, namely the restriction principle for groups. Indeed, apply it to $\Lambda$ with coefficient on the trivial bootstrap $\Lambda$-algebra $C_r^*(G)$.	\\ 
\item[$\bullet$] Bekka shows that ??
\end{itemize}

%%%%%%%%%%%%%%%%%%%%%%%%%%%%%%%%%%%%%%%%%%%%%
\section{Useful constructions in $KK$-theory}
%%%%%%%%%%%%%%%%%%%%%%%%%%%%%%%%%%%%%%%%%%%%%

This section tries to compile interesting constructions in bivariant $KK$-theory that can be applied for the study of apporximation properties of $C^*$-algebras.

\subsection*{Extensions, boundaries and $KK_1$}

Kasparov-Stinespring and $E^{(\pi,T)}$.\\
Toeplitz and suspension..

\subsection*{Mapping cone and double cone constructions}

The mapping cone of a $*$-homomorphism $\phi: A \rightarrow B$ is the $C^*$-algebra 
\[C_\phi =\{ (a,f) \in A\oplus B[0,1] \ | \ f(0)= 0 \text{ and } f(1)=\phi(a) \}.\]
The mapping cone naturally fits in the short exact sequence 
\[s: \begin{tikzcd} 0 \arrow{r} & SB \arrow{r} & C_\phi \arrow{r} & A \arrow{r} & 0\end{tikzcd}\] 
and the boundary map of this sequence coincides with $\phi_*$ modulo suspension, i.e.
\[ \partial_{SB,CB} \otimes \partial_s = \phi_*.\]
(This remains true for any homology or cohomology theory.)\\

Given a sequence \[s: \begin{tikzcd} 0 \arrow{r} & A \arrow{r}{\alpha} & B \arrow{r}{\beta} & C \arrow{r} & 0\end{tikzcd}\] 
with zero composition, consider the natural inclusion $\gamma : A \rightarrow C_\phi$. We call $C_\gamma$ the \textit{double cone }of $s$ and denote it by $C(s)$. Notice that $C(A\otimes C(s)) = A \otimes C(s)$.

\begin{prop}[see \cite{ChabertEO} rk 4.3 and \cite{HLS} section 1]
The sequence $K(A)\rightarrow K(B) \rightarrow K(C)$ is exact iff $K(C(s))=0$.
\end{prop}  

For the proof, use that $K(C(s))=0$ iff $\gamma_*$ is an isomorphism. This property is useful to show failure of $K$-exactness. 

\begin{cor}
If $A$ is in $\mathcal N$ (satisfies the K\"{u}nneth formula and separable...), then $A$ is $K$-exact. 
\end{cor}

\begin{proof}
If $A$ is not $K$-exact, there exists a short exact sequence $s$ such that $A\otimes s$ is not exact in $K$-theory. Then $K(A\otimes C(s))\neq 0$ while $K(C(s))=0$, which prevents $A$ from satisfying the K\"{u}nneth formula.
\end{proof}

\subsection*{Geometric injective and projective resolutions}

%%%%%%%%%%%%%
\section{Baum-Connes}
%%%%%%%%%%%%%%

\begin{itemize}
\item[$\bullet$] Compact groups, or better: proper groupoids: Green-Julg. \\
\item[$\bullet$] Proof for the integer group $\Z$: \\

A model for $\underline E \Gamma$ is the space of finitely supported probabilty measures. For $\Z$, the barycenter map 
\[\left\{ \begin{array}{rcl}
\underline E \Z & \rightarrow & \R \\ 
\sum_{n} p_n \delta_n & \mapsto & \sum_n p_n n \end{array}\right.\]
is an equivariant continuous map homotopic to the identity.\\

Prove the isomorphism
\[ RK_*^\Z(\R, B) \cong RK_*(\mathbb{S}^1,B),\]
under which the assembly map sends the Toeplitz extension, which is a generator of the right side, to a generator of the $K$-theory group of $C^*(\Z)$.\\

\item[$\bullet$] For the free group on two elements $\mathbb F_2$, take $\mathbb F_2$'s Cayley graph $T$ as a model for $\underline E \mathbb F_2 =E\mathbb F_2$, and $B\mathbb F_2$ is the wedge of two circles,
\[ RK_*^{\mathbb F_2}(T, B) \cong RK_*(\mathbb{S}^1 \wedge \mathbb{S}^1 ,B),\]
and then?\\
	
\item[$\bullet$] Connes-Kasparov: proof by representation theory (Wasserman, etc)\\
\item[$\bullet$] Kasparov's Conspectus: towards Higson-Kasparov paper and the proof for Haagerup (J-L. Tu's general version in $KK$-theory, plus the beautiful result that aTmenability implies bootstrap)\\

\item[$\bullet$] Ideas from Coarse geometry, and Yu and Roe's work, SkandalisTuYu etc.\\

\item[$\bullet$] A $\gamma$-element is a class $\gamma \in KK^\Gamma(\C,\C)$ such that there exists a $\underline E \Gamma \rtimes \Gamma$-algebra $A$ and elements 
\[ \eta\in KK^\Gamma(\C,A) \quad \text{and}\quad D\in KK^\Gamma(A,\C)\]
such that $\gamma = \eta \otimes_A D$ and $p^*(\gamma)=1$ in $KK^\Gamma(C_0(\underline E\Gamma),C_0(\underline E\Gamma))$. See \cite{TuNovikov}.\\

Then $\gamma$ and $D\otimes \eta$ are projections, and $\gamma$ is unique. \\

\begin{thm}
If $\Gamma$ has a $\gamma$-element, then $K^{top}(\underline E\Gamma,B)$ identifies with 
\[K((A\otimes B)\rtimes_r \Gamma) p\]
where $p= j_\Gamma(\Sigma_{\underline E \Gamma, B}(\gamma))$ and the assembly maps $\mu_{r,\Gamma}$ and $\mu_{max,\Gamma}$ are injective. Moreover if
\[j_\Gamma(\gamma)_* : K(B\rtimes\Gamma) \rightarrow K(B\rtimes \Gamma) \]
is the identity, $\mu_{max,\Gamma}$ is an isomorphism. If $\gamma=1$, then $\lambda_*\in KK(C_{max}(\Gamma), C^*_r(\Gamma))$ is invertible and $\mu_{r,\Gamma}$ and $\mu_{max,\Gamma}$ are isomorphisms.\\
\end{thm}

\item[$\bullet$] Rub\'en asked: \\

Do you know a group satisfying Baum-Connes but which doesn't have a $\gamma$-element equal to $1$?   Do you know a group which is not $K$-amenable? \\

\textbf{Answer:} Any non compact group having property T cannot have $\gamma=1$, because the class of the Kazhdan projection is not zero in $K_0(C_{max}^*(\Gamma))$ but is in $K_0(C_{r}^*(\Gamma))$. For any infinite hyperbolic group $\Gamma$ having T, its reduced $C^*$-algebra $C^*_r(\Gamma)$ is not K-nuclear (\cite{skandalis1988notion}), so any lattice in $Sp(n,1)$ works out. For instance: $Sp_{n,1}(\Z)$.\\

Do you know when $\Gamma$ is amenable is equivalent to its maximal $C^*$-algebra being exact? When does $C_r^*(\Gamma)$ is exact implies $C^*(Gamma)$ is exact.\\

\textbf{Answer:} When $\Gamma$ has Kirchberg's approximation property, i.e. $\lambda \otimes \rho$ extends to $C^*(\Gamma)\otimes_{min} C^*(\Gamma)$, then $\Gamma$ is amenable iff $C^*(\Gamma)$ is exact.\\
 
\item[$\bullet$] Direct splitting method (Nishikawa 2018 \cite{nishikawa2018direct}):\\

\begin{definition} A Kasparov cycle $(H,T)\in E^\Gamma(\C,\C)$ has property $(\gamma)$ if there exists a non-degenerate representation of the $\Gamma$-algebra $(C_0(\underline E \Gamma),\alpha)$,
\[\pi : C_0(\underline E \Gamma) \rightarrow \mathcal L(H),\]
such that
\[\gamma \mapsto [\alpha_\gamma(\phi ), T] \in C_0(\Gamma, \mathfrak K(H)) \quad\forall \phi \in C_0(\underline E\Gamma)\]
and 
\[\int_\Gamma \alpha_\gamma(c^\frac{1}{2}) T \alpha_\gamma(c^\frac{1}{2})d\mu_\Gamma - T\in \mathfrak K(H),\]
for some cutoff function $c$ on $\Gamma$ and Haar measure $\mu_\Gamma$. (integral in the strong topology)\\
\end{definition}

If such a pair $(H,T)$ and $\pi$ is given, define:\\

\begin{itemize}
\item the $\Gamma$-equivariant Hilbert $A$-module $\tilde H  = H\otimes l^2(\Gamma)\otimes A$,
\item the Fredholm operator $(\tilde T)_{\gamma\gamma} = \gamma T \gamma^*$,
\item the representation $\tilde \pi = \pi \otimes \rho_{\Gamma,A}$, where $\rho_{\Gamma,A}$ is the right regular representation on $l^2(\Gamma)\otimes A$.\\
\end{itemize}

Then $(\tilde H, \tilde \pi, \tilde T)$ defines a class in 
\[\gamma\in KK_0(C_0(\underline E\Gamma) \otimes (A\rtimes_r \Gamma), A)\] 
and the splitting map is defined as
\[\nu_{\Gamma,A} : \left\{ \begin{array}{rcl}
K_*(A\rtimes_r \Gamma) & \rightarrow & KK_*^\Gamma(C_0(\underline E \Gamma ), A) \\
z & \mapsto & \tau_{C_0(\underline E\Gamma)}(z)\otimes \gamma 
\end{array}\right.\]
It is functorial in $A$ w.r.t. $\Gamma$-equivariant $*$-homomorphisms. The main result is the following:\\

\begin{thm}
The composition $\mu_{\Gamma,A}\circ \nu_{\Gamma,A}$ coincides the endomorphism of $K_*(A\rtimes_r \Gamma)$ induced by $(H,T)$.
\end{thm}

\end{itemize}

%%%%%%%%%%%%%%%%%%%%%%%%%%%%%%%%%%%
\section{GPOTS \& NCGOA 2018}
%%%%%%%%%%%%%%%%%%%%%%%%%%%%%%%%%%%

\subsection{Arnaud Brothier: some representations of the Thompson group}

\subsection{Piotr Nowak: Property T for $Out(\mathbb F_n)$}

\subsection{Wilhem Winter: Relative nuclear dimension}

\subsection{Rufus Willett: Exactness and exotic crossed-product}

%%%%%%%%%%%%%%%%%%%%%%%%%%%%%%%%%%%
\section{Coarse geometry \& dynamics}
%%%%%%%%%%%%%%%%%%%%%%%%%%%%%%%%%%%

%%%%%%%%%%%%%%%%%%%%%%%
%%%%%%%%%%%%%%%%%%%%%%%
\section{Langlands}   %
%%%%%%%%%%%%%%%%%%%%%%%
%%%%%%%%%%%%%%%%%%%%%%%

A modular form of weight $k$ is a section of 
\[\Lambda^{k+2}T^*M.\]
The projective space of the $\N$-graded algebra \[A=\bigoplus \Lambda^{k+2}T^*M\] is the compactification of the modular curve
\[\mathbb P(A) \cong \tilde{\mathcal C}.\]
If $F=\mathbb Q$ and $G= GL_2$, the finite part of the adele
\[\mathbb A_f = \prod_{\text{finite places}} F_\nu = \prod_{p \in \mathcal P} \mathbb Q_p\]
is ?? and $G(\hat{\Z})$ is the maximal compact of $G(\mathbb A_f)$ with $G(\mathbb A_f ) / G(\hat \Z)$ being two copies of the upper half plane $\mathbb H$, and $G(\mathbb A_\infty) \backslash G(\mathbb A) / G(K)$ is the modular curve.\\

Is the right part $G(\mathbb A_f)/ G(K)$ is isomorphic to the inductive limit $G(\Z / p^k \Z)$ ?\\

Yes if $G= GL_1$: \[ \varinjlim \Z / p^k \Z = \mathbb Q_p / \Z_p.\]

\section{Haagerup property, cocycles and the mapping class group}

If $\Sigma$ is a closed oriented connected surface (with marked points), we denote by $Mod(\Sigma)$ its so-called mapping class group.\\

In \cite{CostantinoMartelli} are used bounded representations of the mapping class group parametrized by a complex number $z\in \mathbb D$:
\[\pi_z : \Gamma \rightarrow \mathcal L(H).\]
Here, $H$ is the Hilbert space obtained as the free Hilbert space generated by multicurves having a finite number of intersections with a fixed triangulation $\tau$ of $\Sigma$. 

%%%%%%%%%%%%%%%%%%%%%
\section{Hawaii}
%%%%%%%%%%%%%%%%%%%%

\subsection{HLS groupoids}

Let $(\Gamma,\mathcal N)$ be an \textit{approximated group} and $G_{\mathcal N}$ its associated HLS groupoid. Then:

\begin{itemize}
\item[$\bullet$] $G_{\mathcal N}$ is amenable iff $\Gamma$ is amenable,\\

\item[$\bullet$] if $G_{\mathcal N}$ is a-T-menable, then $\Gamma$ is a-T-menable. The converse doesn't hold: in \cite{HLS}, the authors construct an approximated pair $(\mathbb F_2, \mathbb N)$ such that the assembly map $\mu_{G_{\mathcal N},r }$ is not surjective, even if $\mathbb F_2$ is a-t-menable. \\

\item[$\bullet$] $G_{\mathcal N}$ has T iff $\Gamma$ has T,\\

\item[$\bullet$] the algebraic exact sequence
\[\begin{tikzcd}0 \arrow{r} & \bigoplus_n \C[\Gamma_n] \arrow{r} & C_c(G_{\mathcal N})\arrow{r} &  \C [\Gamma]\arrow{r} & 0 \end{tikzcd}\]
extends to 
\[\begin{tikzcd}0 \arrow{r} & \bigoplus_n C^*_{r}(\Gamma_n) \arrow{r} & C^*_r(G_{\mathcal N})\arrow{r} &  C^*_{r,\infty}(\Gamma)\arrow{r} & 0 \end{tikzcd},\]
where the right side algebra is the completion of $\C[\Gamma]$ w.r.t. the norm 
\[ ||x||_{r,\infty} = \sup \{ ||y ||_r : q(y)=x\} \quad \forall x \in \C [\Gamma].\]
This is not an exotic crossed product functor, but one can still define an assembly map $\mu_{\Gamma, r, \infty}$ as the composition of $\mu_{\Gamma,max}$ with the induced at the level of $K$-theory of the quotient map $C_{max}^*(\Gamma) \rightarrow C_{r,\infty}^*(\Gamma) $. This exact sequence and the one induced by the decomposition of $G^0 = \overline \N$ is $\N$ and $\infty$ intertwines the assembly maps so that the next point follows:\\

\item[$\bullet$] $G_{\mathcal N}$ satisfies BC iff $\Gamma$ satifies BC for $\mu_{\Gamma,r,\infty}$.\\

\item[$\bullet$] If $\Gamma$ has T, then if $\mu_\Gamma$ is injective (which is the case for all closed subgroups of connected Lie groups), then $\mu_{\mathcal G_{\mathcal N}}$ fails to be surjective.\\

\item[$\bullet$] \textbf{Congruence subgroup property.} If $\Gamma$ has c.s.p. , then the assembly map fails to be surjective for any HLS groupoid $G_{\mathcal N}(\Gamma)$. If one can find such a groupoid which is a-T-menable for $SO(n,1)$, then this would imply Serre's c.s.p. conjecture: Any lattice in $SO(n,1)$ does not have c.s.p.\\
\end{itemize}

A useful fact from \cite{HLS}: 
\[\begin{tikzcd}0 \arrow{r} & J \arrow{r}{\alpha} & A \arrow{r}{\beta} &  B \arrow{r} & 0 \end{tikzcd}\]
is exact implies that the cone $C_\gamma$ of the natural inclusion $\gamma : J \rightarrow C_\beta$ has vanishin K-groups: \[K_*(C_\gamma) = 0.\]

\subsection{Visit to PennState, September 18th to 21st 2018}

Michael Francis:

\begin{itemize}
\item[$\bullet$] Dixmier-Malliavin theorem: for every Lie group $G$,
\[ C_c^\infty(G) \ast C_c^\infty(G) = C_c^\infty(G). \]
The idea is to decompose, in the real case, the Dirac mass at $0$ as a derivative of $\delta_0 =g^{(n)}$ for some $g\in C^{n-2}(\R)$, but this doesn't quite do the job, so that they show that there exists $g\in C_c^\infty(G)$ and $a_n$ going to $0$ as fast as needed so that 
\[\delta_ 0 = \sum a_k g^{(k)}.\]
The result follows from $f=f\ast \delta = (\sum (-1)^k a_k f^{(k)}) \ast g$. Michael extended this result to Lie groupoids.\\
\item[$\bullet$] a remak of John Roe in his lectures on Coarse Geometry, that there exists a Svarc-Milnor theorem for foliations.\\
\end{itemize}

Sarah Browne:

\begin{itemize}
\item[$\bullet$] Advice to read a book Nate Brown gave her, \textit{Lifting solutions to perturbing problems in $C^*$-algebras}by Terry A. Loring (Fields Institute Monographs). In here canbefind te definition o semi projective $C^*$-algebras.
\end{itemize}

\newpage
%%%%%%%%%%%%%%%%%%%%%%%%%%%%%%%%%%%%%%%%%%%%%%%%%%%%%%%%%%%%%%%%%%%%%%%%%%%%%%%%%%%%%%%%%%%%%%%%%%%%%%
\subsection{Visit to Texas A\&M, February 12th to 14th 2019, and University of Houston February 15th}
%%%%%%%%%%%%%%%%%%%%%%%%%%%%%%%%%%%%%%%%%%%%%%%%%%%%%%%%%%%%%%%%%%%%%%%%%%%%%%%%%%%%%%%%%%%%%%%%%%%%%%%

\subsubsection*{K\"{u}nneth formula, useless stability and extensions} If for every $r$, $X$ is $2$-decomposable w.r.t. a family of uniformly embeddable spaces (into Hilbert space), then $X$ is itself CEH. This contradicts the (false) example I gave in the preprint with Christian of a group whose uniform Roe algebra satisfies the K\"{u}nneth formula. For this I used a split extension of two CEH groups, which is not itself CEH built by Arzhantseva and Tessera (\cite{arzhantseva2016admitting}). A misuse of the fibering theorem made me believe that this group was $2$-decomposable w.r.t. a uniformly CEH family. \\

The main ingredients for this result are the following, and are all contained in a paper of Dadarlat and Guentner (see \cite{dadarlat2007uniform}). Let us fix some notation: $X$ is a discrete metric space, and $\mathcal U$ is a cover, by which we mean a collection of subset of $X$ whose union form $X$. We say that $\mathcal U$
\begin{itemize}
\item[$\bullet$] has Lebesgue number at least $L$ ($Leb(\mathcal U) \geq L$) if any ball of radius $L$ is completely contained in some $U\in \mathcal U$, 
\item[$\bullet$] has $R$-multiplicity less than $k$ ($R-mult(\mathcal U) \geq R$) if any ball of radius $R$ interesect at most $k$ elements of $\mathcal U$. If ignored, $R$ is zero. 
\item[$\bullet$] is $R$-separated if any two elements of $\mathcal U$ a at least $R$-apart,
\item[$\bullet$] is $(k,R)$-separated if $\mathcal U$ admits a partition into $k+1$ families which are $R$-separated.
\end{itemize}

By a partition of unity $\phi$ subordinated to $\mathcal U$, we mean a collection of function $\{\phi_U\}_{U\in \mathcal U}$, each $\phi_U : X \rightarrow [0,1]$ being zero outside of $U$, and such that $\sum_U \phi_U(x) = 1$ for every $x\in X$. We say $Lip(\phi_U)\leq C$ is there exist $\delta >0$ such that if $d(x,y)<\delta$, $|f(x)-f(y)|< C$. And $Lip_{l^1(\mathcal U)}(\phi)< C$ if there exists $\delta$ such that if $d(x,y)< \delta$, $\| \phi(x)- \phi(y) \|_{l^1(\mathcal U)} =\sum_U |\phi_U(x)-\phi_U(y)| < C$.
\begin{enumerate}
\item if $\mathcal U$ is a cover of $X$ with $Leb(\mathcal U) \geq L$ and $mult(\mathcal U)\leq k$ then
\[\phi_U (x) = \frac{d(x,X -U)}{\sum_{V\in \mathcal U} d(x,X -V)}\]
defines a PDU such that 
\[Lip(\phi_U)\leq \frac{2k+3}{L} \text{ and }Lip_{l^1(\mathcal U)} \leq \frac{(2k+2)(2k+3)}{L}.\]
\item if $\mathcal U$ is $(k,L)$-separated, then $mult(\mathcal U) \leq k+1$; 
\item if $\mathcal U$ is $(k,2R)$-separated, then $R-mult(\mathcal U) \leq k+1$;
\item if $L-mult(\mathcal U)\leq k+1$, $Leb(\mathcal U_L)\geq L$;
\item \textit{Summary:} If $\mathcal U$ is $(k,2L)$-separated, then $mult(\mathcal U_L)\leq k+1$ and $Leb(\mathcal U_L)\geq L $. Also $L-mult(\mathcal U )\leq k+1$. 
\end{enumerate}

Using \cite{dadarlat2007uniform}, thm 3.2, the result follows.\\

An interesting question was the following. The rule that associate to a coarse space $X$ its coarse groupoid is a functor, where the domain category is the one of coarse spaces and coarse maps, and the target category is the one of groupoid and generalized morphisms.\\

Does any generalized morphism between coarse groupoids arises from a coarse map?\\

The answer should be no. Thake an extension of groups
\[1 \rightarrow H \rightarrow G \rightarrow C \rightarrow 1,\]
maybe split. Then the quotient arrow should give a generalized morphism between the coarse groupoids, but it is not a coarse map as soon as $H$ is unbounded.\\

Can we show that under suitable conditions, if $C^*_u|H|$ and $C^*_u|C|$ satisfy the K\"{u}nneth formula, then so does $C^*_u|G|$?  

%%%%%%%%%%%%
\subsubsection*{Wreath product and counterexample} In \cite{WillettYuGromov}, thm 8.3, Willett and Yu constructed a counterexample for the Baum-Connes conjecture \textit{with coefficients}. Let $G$ be a group whose Cayley graph contains an expander $X$. $G$ does not act on $X$, but we can enlarge $X$ and set 
\[N_\infty (X) = \cup_{R>0} N_R(X),\]
so that $G$ acts on $N_\infty(X)$ and on 
\[A= l^\infty(N_\infty (X)) = \overline{\cup_{R>0}l^\infty(N_R(X))}.\]
Then the Baum-Connes assembly map for $G$ with coefficients in $A$ fails to be surjective.\\

Now can we force this $C^*$-algebraic coefficients into a group so as to build a counterexample for Baum-Connes without coefficients? For instance, take the unrestricted wreath product 
\[\Gamma=\Z_2 \wr_X G = \prod_{X} \Z_2 \rtimes G.\]
One sees that the reduced $C^*$-algebra $C_r^*(\Gamma)$ should really look like $A\rtimes_r G$. Indeed, $N =\prod_X \Z_2$ is normal abelian in $\Gamma$, so that $C_r^*(\Gamma) \cong C^*_r(N)\rtimes_r G \cong C(\hat N)\rtimes_r G$.  

%%%%%%%%%%%%%
\subsubsection*{Matui's conjecture}

Matui's conjecture states that 
\[K_i (C_r^*(G)) \cong \bigoplus_k H^{2k+i}(G)\]
for every second countable \'etale essentially principal minimal groupoid with base space homeomorphic to a Cantor space. See \cite{MatuiSurvey} for a survey, by Matui himself.\\

Basically, there are two directions that one can take: proving it, for instance for groupoids with FAD, of even for groupoids satisfying the Baum-Connes conjecture? on the other direction: try to find a counterexample.\\

A counterexample was given by Scarparo in \cite{scarparo2018homology}. The groupoid is obtained as $G=\Omega \rtimes \Gamma$, where $\Gamma$ is the infinite dihedral group $\Z\rtimes \Z_2$ and the action is given by shift on $\varinjlim \Gamma/ \Gamma_i$, $\Gamma_j = n_j \Z \rtimes \Z$. Is it FAD? A-T-menable?\\

Pick a strictly increasing sequence of integers $(n_i)$ such that $n_i | n_{i+1}$ and look at the following box-space
\[X = \coprod_{i\geq 1} | \Z_{n_i}|,\]
which has property A.\\

Does Matui's conjecture hold for the coarse groupoid $G(X)$?\\ 

Recall that $C^*_r(G) \cong C_u^*(X)$ and let us compute its $K$-theory. \\

Here are some ideas on how to do it. $K_0$ is big. Show that 
\[K_1(C^*_u(X)) \cong \{ (k_j)_{j\in \Z} \ |\ k \text{ is bounded } \} /  \{ k=(k_j)_{j\in \Z} \ |\ \lim_j k =0 \}.\]
\begin{itemize}
\item[$\bullet$] Let $u_j\in M_{n_j}$ be the shift on $\Z_{n_i}$. Then any unitary in $U_N(C^*_u(X))$ is stably homotopic to a product $(u_j^{k_j})_j$.
\item[$\bullet$] Such a product is not trivial in $K$-theory. To show that a unitary is not trivial in $K_1$, one can for instance compute the trace of 
\[D(u)u^{-1}\]
where $D$ is a derivation. For instance $D(u)= [N,u]$ where $N$ is the propagation operator (multiplication by the length).\\
Remark: if $D$ is the derivation on the circle, then $\int D(u)u^{-1}$ is the winding number.  
\item[$\bullet$] Take the exact sequence associated to the decompostion $\beta X = X \cup \partial \beta X$, which because $X$ has property A (ghost operators are compact) is
\[0\rightarrow \mathfrak K \rightarrow C_u^*(X) \rightarrow l^{\infty}(X)/c_{0}(X) \rtimes_r \Z \rightarrow 0\]
then Pimsner-Voiculescu should conclude. 
\item[$\bullet$] Mayer-Vietoris argument: $X$ splits into two copies of $U$, coarse unions of lines of length $\frac{n_j}{2}$, with intersection $W$ the coarse union of two-points spaces that are further and further apart, so that the $K$-theory is 
\[K(C^*_u(\coprod_{n\geq N} W_n)) \cong K(l^\infty \N).\]
Compute $K(C_u^*U)$.
\end{itemize} 

To compute the homology, first read \cite{MatuiSurvey}.

\subsubsection*{Failure of $K$-exactness}

Show that if $X$ is an expander, then $C^*_u(X)$ is not $K$-exact.

\subsubsection*{Automata groups and exotic construction}

Can one use automata groups to build an example of a discrete group which has A but is not FDC?\\

Andrzej Zuk build a example of an amenable automata group which is not elementary amenable. The idea of FDC is that it should be the "elementary property A" groups.

%%%%%%%%%%%%%%
%%%%%%%%%%%%%%
\newpage
\section{Mayer-Vietoris}

\section{Quantum groups}

\section{Property T}

\section{Number theory}

\section{Fock spaces, CuntzKrieger algebras, and second quantization}












