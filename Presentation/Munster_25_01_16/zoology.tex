

\section{A list of books}
  
A list of books I like about general knowledge in science:

\begin{itemize}
\item[$\bullet$] L'aventure des nombres, Godefroy
\item[$\bullet$] L'autobigraphie de Paul Levy, Laurent Schwartz, et Yuri Manin.
\item[$\bullet$] Recoltes et semailles, Grothendieck.
\item[$\bullet$] Carlo Rovelli, Lee Smolin and Julian Barbour, The trouble with Physics or the End of Time?
\item[$\bullet$] Mandlebrot, several books on fractals.
\item[$\bullet$] Manjit Kumar:
\end{itemize}

\section{Seminar}

\subsection{Cartan subalgebras}

\subsection{Classification and the UCT}

\section{Groups}

\begin{itemize}
\item[$\bullet$] Amenable, a-T-menable, property T, with a diagram
\item[$\bullet$] Mapping class groups 
\item[$\bullet$] Profinite groups, locally profinite groups, $Aut(\overline{\mathbb Q} /\mathbb Q)$
\item[$\bullet$] Automorphism of a regular tree, the Grigorchuk group,
\item[$\bullet$] Lamplighter groups $ H^\Gamma\rtimes \Gamma$, usually 
\[\oplus \Z_2 \rtimes \Z.\]
\end{itemize}

\section{$C^*$-algebras}

\begin{itemize}
\item[$\bullet$] CAR algebra $C^*\langle a_i , a_i a_j +a_j a_i = \delta_{ij} \rangle $ or $\bigotimes M_2(\C)$ or 
\[\varinjlim \ 
\left\{\begin{array}{rcl} 
M_{2^n}(\C) & \rightarrow & M_{2^{n+1}}(\C) \\ 
a           & \mapsto     & \begin{pmatrix} a & 0 \\ 0 & a\end{pmatrix}\end{array}\right.\]
\end{itemize}


\section{Noncommutative geometry}

\subsection{Why $SU_q(2)$?}

Apparently, some people are interested in deformation of classical Lie groups such as $SU_q(2)$, which is the Hopf algebra generated by $3$ generators $E,F,K$ satisfying the relations 
\[R.\]

I wanted to understand where these relations are coming from, which led me to interesting ideas developed by several people, including Yuri Manin. The idea is to define $SU_q(2)$ as a special group like object of the automorphism group of some noncommutative space, the quantum plane.\\

Let $k$ be a field. The free (noncommutative) $k$-algebra on $n$ generators is denoted by $k\langle x_1,... ,x_n\rangle$.

\begin{definition}
A quadratic algebra 
\[A= \oplus_{i\geq 0} A_i\] 
is a $\N$-graded finitely generated algebra such that:
\begin{itemize}
\item[$\bullet$] $A_0 = k$, and $A_1$ generates $A$,
\item[$\bullet$] the relations on generators are in $A_1 \otimes A_1$. 
\end{itemize} 
The quadratic algebra $A$ is said to be a Frobenius algebra of dimension $d$ if moreover 
\begin{itemize}
\item[$\bullet$] $A_d= k$ and $A_i =0$ for all $i>d$,
\item[$\bullet$] the multiplication map
\[m : A_i \otimes A_{d-i} \rightarrow A_d\]
is a perfect duality.
\end{itemize}
\end{definition}

The main example is the quantum plane
\[\mathbb A_q^{2} = k\langle x,y \rangle / (xy -qyx)\]
where $q\in k^\times$. More generally, the quantum space of dimension $n|m$ is
\[ \mathbb A_q^{n|m} = k\langle x_1,.. ,x_n , \eta_1,...,\eta_m \rangle / (x_i x_j - q x_j x_i , q \eta_i \eta_j +  \eta_j \eta_i).\]
This example is suppose to come from physics. In quantum field theories, physicists deal with two kind of particles, bosons and fermions, and use commuting variables for one type, and anticommuting for the other. One object they appeal to are called supermanifolds, which are manifolds enriched with anticommuting variables. Formally, it means they look at rigged spaces $(X,\mathcal O)$ locally isomorphic to $(\R^n, C^\infty [\eta_1,...,\eta_m])$, where $C^\infty [\eta_1,...,\eta_m]$ is the free sheaf of rings generated by anticommuting variables $\eta_i$ over the smooth complex valued functions $C^\infty(\R^n)$.\\

Remark that a quadratic algebra $A$ is a quotient of $k\langle x_1,... ,x_n\rangle$ by elements $r_\alpha \in A_1 \otimes A_1$, which we will denote as 
\[A= k\langle x_1,... ,x_n\rangle / (r_\alpha)\]
or \[A= \langle A_1, R_A\rangle\]
with $R_A \subseteq A_1 \otimes A_1$.\\

Manin defines the quantum dual of a quadratic algebra as
\[A^{!} = k\langle x^i\rangle / (r^\beta)\]
where $r^\beta_{ij}r^{ij}_\alpha = 0$, i.e. $R_{A^!}=R_{A}^\perp$. Then, the quantum endormorphisms between two quadratic algebra is
\[Hom(A,B) =k\langle z^j_i\rangle / (r_\alpha^\beta)\]
where $r_\alpha^\beta = r_\alpha^{ij}r^\beta_{kl} z_i^k z_j^l$. If $End(A)= Hom(A,A)$, then $End(A)$ satisfies the universal property to be intial in the category of $k$-algebras $(B,\beta)$ endowed with an algebra homomorphism $\beta: A \rightarrow A\otimes B$.\\ 

If one does that to the quantum plane $\mathbb A_q^2$, one stil doesn't find quite $M_q(2)$: half of the relations are missing. Also 
\[(\mathbb A_q^{2|0})^! = \mathbb A_q^{0|2}?\] Exercise.

\subsection{TQFT}

the category of bordisms $Bord^d$, and \[Z : Bord^d \rightarrow \mathcal C .\]
In dimension $1$: vector spaces, in dimension $2$: comodules over Hopf algebras. 


