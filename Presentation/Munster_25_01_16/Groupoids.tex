\section{Simple examples for Baum-Connes for groupoids}

This is a question asked by Sayan Chakraborty : find a simple example of the Baum-Connes conjecture for groupoids. \\

We found that one should be able to do actual computations in $K$-theory, like determining generators of $K$-group of some known $C^*$-algebras, and to prove Baum-Connes by hand in some simple examples. The only one we managed to actually do by hand was Baum-Connes for $\R^n$. (Do it !) \\

The simplest example would be to take the groupoid associated to an action of a group on a topological space $\mathcal G = X\rtimes G$. The first thing we want to do is to describe the classifying space for proper actions.\\

Suppose the groupoid étale equipped with a proper length. A simple model, from J-L. Tu \cite{TuBC2}, is given by the inductive limite of the spaces
\[Z_d=\left\{\nu \in \mathcal M(\mathcal G), s.t. \exists x,\text{if } g\in \text{supp }\nu \text{ then } l(g)\leq d , g\in \mathcal G^x\right\}.\]

Indeed, suppose $Y$ is a $\mathcal G$-proper cocompact space, then $Y\rtimes \mathcal G$ is a proper groupoid, so there exists a cutt-off function $c : Y\rightarrow [0,1]$ such that : 
\[\sum_{g\in \mathcal G^{p(y)}} c(yg) = 1,\forall y \in Y.\]
Now define \[y \mapsto \sum_{g\in \mathcal G^{p(y)}} c(yg)\delta_g\]
which is a $\mathcal G$-equivariant continuous map. Moreover $Z_d$ is proper and cocompact, and there exists a $d$ s.t. the map takes its values in it.\\

Now if $\mathcal G = X\rtimes G$, $Z_d \simeq X\times Z'_d$ where $Z_d= \left\{ \nu \in \mathcal M(G), s.t. \text{if } g\in \text{supp }\nu \text{ then } l(g)\leq d\right\}$, so that
$KK^{\mathcal G}(\Delta,A)\simeq KK^G(\Delta', A)$, where $\Delta$ and $\Delta'$ are respectively the $0$-dimensional part of the equivariant complexes $Z_d$ and $Z'_d$. This is true because the action of $G$ on $Z'_d$ is proper and cocompact, see lemma $3.6$ of \cite{TuBC2}. Now a standard Mayer-Vietoris argument (theorem $3.8$ \cite{TuBC2}) concludes to show that 
$K^{top}(\mathcal G,A)\simeq K^{top}(G,A)$.\\

As $C^*_r \mathcal G = C_0(X)\rtimes_r G$, we see that the Baum-Connes assembly map for $\mathcal G$ with coefficients in $A$ is equivalent to 
\[K_*^{top}(G,A)\rightarrow K_*((A\otimes C_0(X))\rtimes G).\]

Now we can look for concrete examples.
\subsection{Non commutative tori}

Question : Compute the generators of non-commutative tori. (Sayan did it)
\subsection{Principal bundle over $U(2)$}
This is an example from Olivier Gabriel's talk in Montpellier. \\

Take the principal bundle $U(2)\rightarrow U(2)/\mathbb T^2\simeq \mathbb S^2$. You can foliate the fibers by an irrational rotation $\theta$, so that you have an action of $\R$ on $C(U(2))$. Reducing to a complete transversal (take $SU(2)$ ), the algebra $C(U(2))\rtimes \R$ turns out to be Morita equivalent to $\underline A= C(SU(2))\rtimes \Z$ (a general result of foliation groupoids I think ). $\underline A$ can be reduced to $C(\overline D)\otimes A_\theta$ and to $Ind_{\mathbb T^2}^{U(2)}\ A_\theta$.\\

Question : Compute the generators of the $K$-theory of $\underline A$.

\subsection{Foliations}
\subsection{An example from physics}
In Alain Connes' book, we can read the following example.\\

Take the $2$-torus $M=\mathbb T^2$. Its fundamental group $\Gamma=\Z^2$ acts on its universal cover $\tilde M=\R^2$ by isometries, and the electromagnetic field $A$ gives a two-form $w$ (its curvature) on $\tilde M$, so a $2$-cocycle on the fundamental groupoid of $\tilde M$ :
\[w(\tilde x,\tilde y,\tilde z)=e^{2i\pi \int_{\Delta}\tilde w}\]
where $\Delta$ a geodesic triangle between the $3$ points. It turns out that $H^2(\Z^2,\mathbb T^2)=\mathbb S^1$, so that $\tilde w$ determines a number $\theta\in [0,1)$, and the twisted reduced algebra of the fundamental groupoid w.r.t. $\tilde w$ is equal to $A_\theta=C(\mathbb T^2 )\rtimes_{r,\theta} \Z^2$. This situation generalizes to general manifold whose fundamental cover are equiped with a line bundle and a conection. We can then associate a $2$-cocycle on the fundamental groupoid of $\tilde M$ to the curvature of the line bundle. \\

A question : Does the twisted crossed-product has applications to Yang-Mills theories ?
