\section{Funky questions, ideas of talks}

\subsection{Expanders}

Here are some interesting questions I had after a talk on expanders. 

\subsubsection{Plan of the talk}

I first gave a motivation for considering expanders. Namely, we are interested in the following network theory problem : can we construct a network as big as we want, such that the cost is controlled and which is not subject to easy failure ?\\

Building a network as big as we want means we want to consider a family of graphs $X_j = (V_j,E_j)$ such that $|V_j|\rightarrow +\infty$, and controlling the cost means that $deg(X_j) < k$ for all $j$. But what does "not easily subject to failure" means ? For this, I want to explain why we should ask our family to stay well connected and why the second value of the discrete Laplacian is a good way to measure that.\\

The idea is to relate the Laplacian to the uniform random walk on the graph, and to show that $\lambda_1(X)$ controlls the speed of convergence of the uniform random walk to the stationary measure which is the uniform probability on the graph, given by $\nu(x)= C. deg(x)$.\\

A family of graphs satifying the previous conditions and such that $\lambda_1(X_j) >c >0$ is called an expander. If time allows, one can then elaborate on metric properties of this type of graphs. The impossibility to embed them coarsely into any separable Hilbert space, and the relations to the Baum-Connes conjecture are close to my work.

\subsubsection{Questions}

\begin{itemize}
\item[$\bullet$] Paolo Pigato : What is the dynamic at the limit ?
\item[$\bullet$] Anne Briquet : Is $\lambda_1(X)$ such a good way to measure the connectedness of a graph, if you consider the phenomenon of cuttoff for finite Markov Chains.
\end{itemize} 

\subsection{Ideas of funky talks}

\begin{itemize}
\item[$\bullet$] What is the relation between the Fourier transform and quantum groups ?
\item[$\bullet$] What is the relation between the Runge Kutta methode and renormalization in QFT ? 
\item[$\bullet$] What is the relation between Brownian motion and second quantization ?
\end{itemize} 
