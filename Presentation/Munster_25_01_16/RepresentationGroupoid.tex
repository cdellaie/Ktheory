\section{Representations of groupoids}

\textbf{Mettre les references et des rappels sur les champs continus et mesurables d'espaces de Hilbert.}\\

This section is a reminder on the different notions of representations for groupoids that exists. Let us first begin by a reminder on continuous fields of $C^*$-algebras and Hilbert spaces.\\

A continuous field of $C^*$-algebras over a locally compact space $X$ is a pair 
\[ \left( \{ A_x \}_{x\in X} , \Gamma_A \right) \]
where:\\
\begin{itemize}
\item[$\bullet$] $A_x$ is a $C^*$-algebra,\\
\item[$\bullet$] $\Gamma_A \subset \prod_{x\in X} A_x$ is a $*$-algebra such that $x\mapsto || \gamma (x) ||$ is continuous for every $\gamma \in \Gamma_A$, and if $\sigma \in \prod_{x\in X} A_x$ is locally uniformly approximable by sections, then $\sigma\in\Gamma_A$,\\
\item[$\bullet$] $\{ \gamma(x)\}_{x\in X}$ is dense in $A_x$.\\
\end{itemize}

On the other hand a $C_0(X)$-algebra is a $C^*$-algebra $A$ endowed with a nondegenerate $*$-homomorphism 
\[\phi : C_0(X) \rightarrow M(Z(A)). \]
Any field $\left( \{ A_x \}_{x\in X} , \Gamma_A \right) $ over $X$ defines a $C_0(X)$-algebra 
\[C^*(\Gamma_A) := \{\gamma\in \Gamma_A \text{ s.t. } x\mapsto ||\gamma(x) || \in C_0(X)\}.\]
A $C_0(X)$-algebra $A$ is continuous if $x\mapsto || a_x||$ is continuous for each $a\in A$. Here, $a_x$ denotes the image of $a$ under the map $A \rightarrow A / \phi(I_x)A$,with $I_x$ the ideal of functions vanishing at $x$.\\

There is a correspondence between these two notions.\\

Let $G$ be a locally compact groupoid. Renault defines a representation of $G$ as the following data:\\

\begin{itemize}
\item[$\bullet$] a measure $\mu$ on $G^0$,\\
\item[$\bullet$] a measurable field of Hilbert space $(\mathcal H , \mu)$ over $G^0$,\\
\item[$\bullet$] a family of bounded operators $L_g : \mathcal H_{s(g)} \rightarrow \mathcal H_{r(g)}$ for each $g\in G$ satisfying $L_{g_1} L_{g_2} = L_{g_1 g_2}$ for every $(g_1,g_2)\in G^2$, $L_{e_x}= id_{\mathcal H_x}$, and 
\[g\mapsto \langle L_g(\xi_{s(g)}) , \eta_{r(g)} \rangle \] 
is measurable for every pair of measurable sections. \\
\end{itemize}

Recall the following: given a $C_0(G^0)$-Hilbert module $E$, a unitary representation $G$ is a unitary 
\[V \in\mathcal L_{s^* C_0(G^0)}(s^* E, r^* E)\]
which satisfies $V_1 V_2 = \Delta V$, where:\\

\begin{itemize}
\item[$\bullet$] $V_i $ is the $C_0(G^{2})$-operator induced from $V$ by the projection $p_i : G^2 \rightarrow G$,\\
\item[$\bullet$] $\Delta $ is the (comultiplication) map $C_0(G)\rightarrow M(C_0(G^{2}))$-operator induced by the multiplication $\Delta : G^2 \rightarrow G$.\\
\end{itemize}

Fiberwise this gives you a more restrictive class than the representations in the sense of Renault. Indeed, in the case of a trivial groupoid over a locally compact space $X$, the spectral theorem ensures that any $*$-representation on a Hilbert space $H$
\[\pi : C_0(X) \rightarrow \mathcal L(H)\]
disintegrates into a representation in the sense of Renault on a field of Hilbert space. However, our $\{V_g\}$ gives a continuous field of representation over a continuous field of Hilbert space, which is a priori stronger.\\ 

As in the case of groups, one can try to define a integrated representation, by 
\[ (V(f)\xi )_x = \int_{g\in G^x} f(g) V_g (\xi_{s(g)}) d\lambda^x (g) \quad f \in C_c(G) , \xi \in E.\]
This defines a map $C_c(G)\rightarrow \mathcal B(E)$, where $\mathcal B (E)$ denotes the bounded operator of $E$ seen as a Banach space. But this map is not even multiplicative!\\

Instead, consider $G$ to be étale, and 
\[V(f)_{xy} = \sum_{g\in G_y^x} f(g) V_g ,\]
so that 
\[(V(f)\xi)_{x}= \sum_{y\in X} V(f)_{xy} \xi_y = \sum_{g\in G^x} f(g)V_g(\xi_{s(g)}),\]
but this time
\[V(f\ast g)_{xz} = \sum_{y\in X} V(f)_{xy} V(g)_{yz}.\]
If $G$ is not étale, suppose there is a measure $\mu$ on $G^0$.