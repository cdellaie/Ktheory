\section{Le principe de Maxwell}

Dans son livre, Structure des systèmes dynamiques, J-M. Souriau donne un cadre symplectique à la mécanique classique, qui s'étend sans difficulté à celui de la relativité restreinte. Le voici. On se donne une variété $V$ feuilletée par une $2$-forme $\sigma$, telle que l'espace des feuilles $U=V/\text{ker }\mathcal F$ soit une variété sur laquelle $\sigma$ induit une structure symplectique. La première doit être pensée comme l'espace d'évolution du système physique que l'on étudie, et la seconde comme l'espace des mouvements du système.\\

Arrivé là, un expérimentateur doit pouvoir communiquer ses mesures à un second expérimentateur, et l'on doit donc se donner une règle qui permet de passer d'un référentiel à l'autre. Cela est fait grâce à l'action d'un groupe. Soit donc $G$ un groupe de Lie, qui agit sur $V$ de façon à préserver les feuilles, et qui induit donc une action par symplectomorphismes sur $U$.\\

Sous certaines conditions (si $\mathfrak g=[\mathfrak g,\mathfrak g]$ ), on peut associer une application moment $\mu : U\rightarrow \mathfrak g^*$, où $\mathfrak g^*$ est le dual de l'algèbre de Lie de $G$, qui vérifie
\[\sigma(Z_U(x))=-d\langle \mu(x),Z\rangle\quad, \forall Z\in \mathfrak g.\]

En mécanique classique, $V = \R\times \R^6$, et l'on passe d'un référentiel galiléen à un autre par transformations du groupe de Galilée, i.e. $G=\text{Gal}(3)$, remplacé par le groupe de Poincaré en relativité restreinte. Le défaut d'équivariance de l'application moment, i.e. 
\[\theta_x(a) = \underline{a}_{\mathfrak g^*} \mu(x)-\mu(\underline{a}_Ux)\quad ,\forall a\in G,x\in U \]
définit un $\mathfrak g^*$-cocyle de $G$, dont on peut montrer que la classe de cohomologie ne dépend ni de $x$ (à vérifier) ni de $\mu$. Cette classe, notée $m\in H^*(G)$, exprime la masse du système dans le cadre galilén. Une résultat remarquable est que le groupe de Poincaré a une cohomologie triviale, ce qui donne une preuve mathématique simple qu'un système n'a pas de masse intrinsèque en relativité restreinte.\\

\begin{itemize}
\item[$\bullet$] Souriau mentionne dans son livre que ce formalisme peut s'étendre en relativité générale, au prix d'abandonner les groupes de Lie pour les pseudo-groupes. A l'heure actuelle, je n'ai pas trouvé encore où ni comment le faire. Ce formalisme mettant en jeu des actions de groupes (et possiblement de pseudo-groupes, donc de groupoïdes ), je me demande si une réécriture en terme de groupoïdes ne permettrait pas d'éclarcir la méthode, ou encore d'apporter des réponses à des situations plus complexes en donnant de bons invariants. Par exemple, on peut penser à l'action de $\R$ sur le tore $\mathbb T$ par le flot donné par la forme $dy-\theta dx$, avec $\theta$ un irrationel.

\item[$\bullet$] Au départ, mon intérêt pour ce formalisme me vient de la question suivante : quel est le lien entre la $K$-théorie et la "masse" d'un système physique ? Ce lien est mentionné dans un des deux articles (plus précis !) d'E. Witten sur la $K$-théorie, où il explique que la $K$-théorie d'une certaine algèbre labelle naturellement les valeurs possibles de la charge conservée d'un champ de Ramond-Ramond. Ensuite, il est connu que si l'on observe un électron dans un cristal apériodique, son énergie est donnée par un élément de la $K$-théorie d'une certaine algèbre (la $C^*$-algèbre du groupoïde de l'action du groupe d'isométrie des pavages sur l'orbite d'icelui). Bien sûr, cette question est entièrement basée sur des analogies, dont la plus simple est que la charge électrique joue pour le champ életro-magnétique le rôle de la masse pour le champ gravitationnel. 
\end{itemize}

\subsection{Un exemple}

En mécanique classique, $V=\{y=(r,v,t)\in \R^N\times \R^N\times \R\}$ et \[\sigma(dy)(d'y)=\sum_j \langle m_jdv_j - F_jdt , d'r_j-v_j d't\rangle-\sum_j \langle m_j d'v_j -F_j d't, dr_j-v_j dt \rangle.\]
L'idée est de remplacer le système $m\frac{\partial v}{\partial t} = F,\frac{\partial x}{\partial t} =v$ par $\sigma$, et de passer de l'espace de phase $R^N\times R^N$ par $V$. Ce faisant, le temps devient une variable comme les autres, ce qui permet de traiter le cadre de la relativité restreinte,où un changement de repère mélange coordonnées de temps et d'espace, avec le même formalisme que celui de Galilée.\\

Voici d'ailleurs
\begin{itemize}
\item[$\bullet$] le groupe de Galilée 
\[Gal(3)=\left\{\begin{pmatrix} A & v & x \\ 0 & 1 & t \\ 0 & 0 & 1 \end{pmatrix}, A\in SO(3), v,x\in \R^3,t\in \R\right\}\]
Si l'on passe $R$ à $R'$ par une telle transformation, l'origine de $R'$ est translatée de $x$ à vitesse $v$ par rapport à celle de $R$, et il faut appliquer $A$ au repère de $R$ pour obtenir celui de $R'$.
\item[$\bullet$] Le groupe de Lorentz $O(3,1)$ est le groupe de $E_{3,1}$ qui préserve la forme de Minkowski $dt^2-dx^2$. Le groupe de Poincaré est donné par
\[P(3,1)=O(3,1)\rtimes E_{3,1} =\left\{\begin{pmatrix} A & v \\ 0 & 1 \end{pmatrix}, A\in O(3,1),v\in E_{3,1}\right\}.\]
On voit que le groupe de Galilée en est un sous-groupe, et que $P(3,1)$ permet lui de mélanger coordonnées spatiales et temporelles. 
\end{itemize}


































