One of the most celebrated results in representation theory is Pontryagin duality. Recall that, to any locally compact abelian group $G$, one can associate its dual, i.e. the set of homomorphisms from $G$ to the circle group, endowed with the topology of pointwise convergence. Using commutativity of $G$, the bidual is itself a locally compact abelian group. Moreover, taking the bidual of a group, one recovers the original group. This result gives a conceptual understanding of the Fourier transform, and one would want to drop the commutativity assumption. 
%One way to achieve this led to the notion of locally compact quantum groups. 
Adding some hypothesis to Gelfand duality, it is known that compact Hausdorff groups are equivalent to unital commutative $C^*$-algebras endowed with a coproduct that satisfies an additional density condition. Dropping commutativity, one obtains the definition of compact quantum groups. 

Discrete quantum groups can also be naturally defined in a way that puts them in duality with compact quantum groups; strikingly, most of the results on representations of discrete or compact groups, such as Peter-Weyl theory, extends to this setting. This process of abstraction is one of the motivations for studying quantum groups; however, they also appear naturally in a wide range of areas such as solvable models in statistical mechanics, knot theory, deformations of Lie algebras, and others. They also provide very interesting $C^*$-algebras, often appearing as quantum symmetries of some noncommutative objects. These reasons fostered the study of quantum groups and their $K$-theory. We shall now describe how the Buam-Connes conjecture can be extended to the quantum group setting.      

There is a version of the Baum-Connes conjecture \emph{with coefficients}, claiming that a certain assembly map 
\begin{equation}\label{bcc}
\mu_{\Gamma,A}: RK^\Gamma_*(\underline{E}\Gamma,A) \rightarrow K_*(A\rtimes_r \Gamma)
\end{equation}
generalizing that in line \eqref{bc ass} above is an isomorphism.  Here $A$ is a $C^*$-algebra on which the group $\Gamma$ acts by automorphisms and $A\rtimes_r \Gamma$ is the associated reduced crossed product, while $RK^\Gamma_*(\underline{E}\Gamma,A)$ is the topological $K$-theory of $G$ with coefficients in $A$.

In order to extend the conjecture to the realm of quantum groups (in the sense of Woronowicz \cite{Wo}), Ralf Meyer and Ryszard Nest reformulated the classical Baum-Connes conjecture with coefficients in \cite{MeyerNest} using the machinery of triangulated categories. Let $\Gamma$ be a group. One can consider the category $\mathfrak{KK}^\Gamma$, whose objects are separable $C^*$-algebras endowed with an action of $\Gamma$ by automorphisms, and morphisms from $A$ to $B$ are elements of $KK_0^\Gamma(A,B)$, an abelian group constructed by Gennadi Kasparov in \cite{Kasparov}. The composition of morphisms is given by an associative bilinear pairing (see \cite{Kasparov}) called the \emph{Kasparov product}.

Meyer and Nest showed in \cite{MeyerNest} that $\mathfrak{KK}^\Gamma$ is actually a triangulated category; roughly triangulated categories provide a very general setting in which one can do (some) homological algebra.  They then show that if $F$ is the functor on $\mathfrak{KK}^\Gamma$ defined by $F(A)= K_*(A\rtimes_r \Gamma)$, then there is a \emph{localization}\footnote{Precisely, at the subcategory of compactly induced objects.} $\mathbb{L}F$ of $F$ such that the corresponding natural transformation $\mathbb L F(A) \rightarrow F(A)$ is naturally isomorphic to the Baum-Connes assembly map with coefficients in $A$. 

Now, an advantage of this abstract picture of the Baum-Connes conjecture is that it allows one to avoid the use of the so called \emph{classifying space for proper actions}  $\underline E \Gamma$ as was done in \cite{BaumConnesHigson}; this is of particular interest in the setting of quantum groups, where the correct analogue of $\underline E \Gamma$ is not at all clear.  Indeed, given a (locally compact) quantum group $\mathbb G$, the category $\mathfrak{KK}^{\mathbb G}$ is again triangulated and the functor $F(A)= K_*(A\rtimes_r \mathbb G)$ still makes good sense. The assembly map for quantum groups is defined to be the natural transformation $\mathbb L F\rightarrow F$ for an appropriate localization $\mathbb{L}F$ that naturally analogizes that used in the classical setting. The Baum-Connes conjecture for $\mathbb{G}$ (with coefficients) is the claim that the natural transformation $\mathbb L F\rightarrow F$ is an isomorphism. It has been proved in some cases, for example by Voigt in \cite{Voigt1} and \cite{Voigt2}. 

The project that we want now to present proposes a notion of proper action for a quantum group (which we assume discrete for simplicity), and  a more direct approach to the quantum group Baum-Connes conjecture.  The extension of the notion of properness to the quantum case remains unsuccessful, despite a great deal of effort on the topic.

%The first step is to reformulate in an algebraic fashion the notion of classifying space. If $\Gamma$ is a discrete group and $F\subseteq \Gamma$ a finite subset, recall that 
%\[P_F(\Gamma)= \{ \eta \in Prob(\Gamma) \text{ s.t. supp}(\eta ) \subseteq \gamma F \text{ for some } \gamma \in\Gamma \}\] 
%is called the Rips complex of $\Gamma$. Endowed with the weak-$*$ topology, it is a $G-CW$-complex, and $\cup_F P_F(\Gamma)$ is a model for the classifying space $\underline E \Gamma$ that appears in the left hand side of the Baum-Connes conjecture. 
 
%Let $\mathcal F$ be the collection of finite subsets $F\subseteq \Gamma$, seen as a category (it is a poset). If $C$ is any abelian category, denote by $\hat{ \mathcal C}$ the category of functors $\mathcal F \rightarrow \mathcal C$, with morphisms $\hat{\mathcal C}(X,Y)$ given by natural transformations $X \rightarrow Y\circ \rho$, where $\rho :  \mathcal F \rightarrow \mathcal F$ is a non-decreasing map. 

In \cite{OY2}, Herv\'{e} Oyono-Oyono and Guoliang Yu introduced an approximation of $K$-theory called quantitative $K$-theory. In PI Dell'Aiera's thesis \cite{DellAieraThesis}, this approximation was extended to general \emph{filtered} $C^*$-algebras: here a filtered $C^*$-algebra $A$ has a collection of linear self-adjoint subspaces $(A_S)_{S\in \mathcal{F}}$ indexed by a partially ordered commutative monoid $\mathcal{F}$, and satisfying certain natural compatibility conditions.  To any filtered $C^*$-algebra $A$, one can associate its quantitative $K$-theory groups $\hat K_*(A)$; roughly these remember which subspace $A_S$ a particular class comes from.  For any $\Gamma-C^*$-algebra $A$, the reduced crossed product $A\rtimes_r \Gamma$ is naturally filtered by the collection $\mathcal F$ of finite subsets of $\Gamma$ ($A_S$ is just the elements of $A\rtimes_r\Gamma$ supported on $S$) and $\hat K_* (A\rtimes_r \Gamma)$ is well defined. It follows from PI Dell'Aiera's PhD thesis that $A\mapsto \hat K_*(A\rtimes_r \Gamma)$ is then a functor from $\mathfrak{KK}^\Gamma$ to an abelian category $\hat C$, a slight modification of the category $C$ of ($\Z_2$-graded) abelian groups.

 %I think you talk about these earlier so I won't precise what it is 

The first project here is to compute the controlled analogue of Meyer-Nest's localization functor $\mathbb{L}F$ in the controlled setting.

\begin{project} Compute the localisation of $\hat F( A) = \hat K_*(A\rtimes_r \Gamma) $. Describe the relation between $\mathbb L \hat F(A)\rightarrow \hat F(A)$ and the controlled assembly map
$\hat \mu_{\Gamma,A}$.	
\end{project}

We are fairly sure at this point that the answer is the filtered functor $\mathbb L \hat F(A) = (RK_*^\Gamma(P_S(\Gamma),A))_{S\in \mathcal{F}}$, and that $\mathbb L \hat F (A) \rightarrow \hat F(A)$ is naturally isomorphic to the controlled assembly map $\hat \mu_{\Gamma,A}$.  Here $S$ is a finite subset of $\Gamma$, and $P_S(\Gamma)$ is the associated \emph{Rips complex}, an approximation to the classifying space $\underline{E}\Gamma$ appearing on the left hand side of the Baum-Connes conjecture as in line \eqref{bcc}.  On the other hand, the controlled assembly maps $\hat\mu_{\Gamma,A}$ were defined in \cite{OY2} and \cite{DellAieraThesis} for any discrete group $\Gamma$ and any $\Gamma-C^*$-algebra $A$. These assembly maps approximate the classical ones in a precise way, and satisfy stronger stability properties. These stability properties allow one to give proofs of the Baum-Connes conjecture in some cases without using the Dirac-dual-Dirac argument, as discussed in the introduction.  Incidentally, one of the main motivations for the work of Meyer-Nest \cite{MeyerNest} in the classical case is to give a clean approach to various stability properties of the Baum-Connes conjecture; we suspect their proofs will go over to the controlled setting.

Now, let $\mathbb G$ a compact quantum group and $\hat{\mathbb G}$ its dual, which is a discrete quantum group. It was shown in \cite{DellAieraThesis} that, for every $\hat{\mathbb G}$-algebra $A$, $A\rtimes_r \hat{\mathbb G}$ is filtered by the set $\mathcal E$ of equivalence classes of finite dimensional representations of $\mathbb G$

\begin{project} Define a 
$ \hat{\mathbb G} $-algebra $\Delta_\pi$ for every class $\pi\in \mathcal E$, such that 
\[\mathbb L \hat F(A) \cong (\pi\mapsto KK^{\hat{\mathbb G}}(\Delta_\pi,A)).\]   
\end{project}

If this can be done (as we expect), then $\Delta_\pi$ would be the natural equivalent of the Rips complex for discrete quantum groups. Its existence would shed light on the notion of proper action in the quantum case. In the case of a discrete quantum group $\hat{\mathbb G}$ and a class $\pi\in \mathcal E$, we can construct a $C^*$-algebra $\Delta_\pi$ which reduces to the continuous functions on the Rips complex $C_0(P_d(\Gamma))$ when $\hat{\mathbb G}= \Gamma$ (in which case $\mathcal E\cong \mathcal F$), so this $\Delta_\pi$ is indeed a natural candidate for the right $C^*$-algebra in general. This construction would also allow the definition of a controlled assembly map for discrete quantum groups. This last point could lead to new techniques to prove instances of the Baum-Connes conjecture for quantum groups (the current proofs use the Dirac-dual-Dirac method), currently a very underdeveloped areas despite the interesting progress of Voigt and others mentioned above. Such a definition of the assembly map would give back its interpretation as an generalized index map, while preserving the equivalence with the current abstract definition; we hope this would lead to new intuition. % This is new, you can erase if you think it is not necessary 
Finally, we just mention that the above should generalize, and we hope to also carry this out.

\begin{project} Extend these projects to the setting of general locally compact groups, and quantum groups.
\end{project}
