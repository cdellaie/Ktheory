\documentclass[11pt]{article}

\bibliographystyle{abbrv}

\usepackage{amsfonts}

\usepackage{amsthm}

\usepackage{amssymb}

\usepackage{amsmath}

\usepackage[all]{xy}

\newcommand{\N}{\mathbb{N}}

\newcommand{\Z}{\mathbb{Z}}

\newcommand{\R}{\mathbb{R}}

\newcommand{\Q}{\mathbb{Q}}

\newcommand{\C}{\mathbb{C}}

\newcommand{\T}{\mathbb{T}}

\newcommand{\Manoa}{M\=anoa}

\newcommand{\Hawaii}{Hawai\kern.05em`\kern.05em\relax i}

\linespread{1.05}


\textwidth=5.75truein
\textheight=8.5truein

%
% dimensions to play with
%
%\textheight=8.0truein   %for mac
%\textwidth=6.0truein

\oddsidemargin=0.3truein
\evensidemargin=0.3truein

%\topmargin=-0.35truein %original
\topmargin=-0.25truein
% \headheight=0.16truein
\headheight=13pt
\headsep=0.3truein

\footskip=0.5truein

\parindent=18pt
\tolerance=6000
\parskip=0pt

\begin{document}



\section*{Project Summary}


\textbf{Overview}\\
The PI asks for funds to support his own research in fundamental mathematics; for supporting graduate students; and for travel to disseminate results and continue collaborations.\\

\noindent\textbf{Intellectual Merit.}\\
Groups are fundamental objects in mathematics, as they give a way of formalizing the symmetries of various geometric spaces (and other more general mathematical objects).   Associated to a group are various so-called K-theoretic invariants, which measure in some sense how `twisted' the group is.  This corresponds to the `twistedness' of the various geometric spaces that the group can be manifested as symmetries of.  There are important conjectural guesses at the structure of these K-theoretic invariants due to Baum-Connes and Farrell-Jones; if correct, these conjectures would have wide-ranging implications throughout analysis, algebra, and geometry.

This proposal consists of four broad projects.  Two of these push the ideas above to more general ways of encoding symmetries: groupoids and quantum groups respectively.  Both groupoids and groups have their origins modeling symmetries of certain systems from quantum physics, but now manifest all over mathematics.  The methods we propose should both lead to new results, and to a much more concrete understanding of existing theory.  Another project aims to better understand counterexamples to the Baum-Connes conjecture via an extension of the definition of property (T) -- a property for groups that makes them very rigid, in some sense -- to topological groupoids.  The fourth project aims to transfer ideas from the analytic realm of the Baum-Connes conjecture to the purely algebraic realm of the Farrell-Jones conjecture to overcome fundamental barriers in our understanding of the latter.

PI Willett has been working on related issues for around ten years with a variety of collaborators; much of the above builds on his earlier work.  PI Dell'Aiera completed his PhD in summer 2017 and has expertise in groupoids, quantum groups, and computational techniques in $K$-theory; much of the above will use ideas he developed in his thesis and that are new to PI Willett.\\

\noindent\textbf{Broader Impacts.}\\
The PIs will continue their work training graduate students, both through directly mentoring PhD and MA students, and through organizing and speaking in seminars at their home institution.  PI Willett will continue his work broadening access to the PhD program at his home institution, and broadening the education it provides, as well as his work improving mathematics general education.  PI Dell'Aiera will educate mathematics students on the connections of the field to physics.  Both PIs will disseminate their results through conference talks, and will give lecture series with the express aim of educating junior researchers.

\end{document}
