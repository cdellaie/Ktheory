\documentclass[11pt]{article}

\bibliographystyle{abbrv}

\usepackage{amsfonts}

\usepackage{amsthm}

\usepackage{amssymb}

\usepackage{amsmath}

\usepackage[all]{xy}

\newcommand{\N}{\mathbb{N}}

\newcommand{\Z}{\mathbb{Z}}

\newcommand{\R}{\mathbb{R}}

\newcommand{\Q}{\mathbb{Q}}

\newcommand{\C}{\mathbb{C}}

\newcommand{\T}{\mathbb{T}}

\newcommand{\Manoa}{M\=anoa}

\newcommand{\Hawaii}{Hawai\kern.05em`\kern.05em\relax i}

\linespread{1.05}


\textwidth=5.75truein
\textheight=8.5truein

%
% dimensions to play with
%
%\textheight=8.0truein   %for mac
%\textwidth=6.0truein

\oddsidemargin=0.3truein
\evensidemargin=0.3truein

%\topmargin=-0.35truein %original
\topmargin=-0.25truein
% \headheight=0.16truein
\headheight=13pt
\headsep=0.3truein

\footskip=0.5truein

\parindent=18pt
\tolerance=6000
\parskip=0pt

\begin{document}



\section*{Project Summary}


\textbf{Overview}\\
We propose a research project in Operator Algebras and Noncommutative Geometry. We are asking for funds to support research, travel to spread results and continue collaborations and share the outcome of the research.\\

\noindent\textbf{Intellectual Merit.}\\
Attached to groups, spaces and their actions are C*-algebras (a special class of Banach algebras) and their K-theory groups. A nice generalization that encompass groups, dynamics and metric spaces are topological groupoids, from which C*-algebras can be built (and their K-theory groups). The project is divided into two broad parts:

(1) Try to understand how a notion of finite decomposition for family of groupoids can be used to give new results about the Kunneth formula and the Baum-Connes conjecture for groupoids. This could potentially give new examples of groupoids that satisfy the conjecture. 

(2) Attempting to import coarse geometric ideas into the realm of C*-algebraic quantum groups. This can be done in several steps. First, study proper actions for duals of compact quantum groups and to try to define a classifying space for proper actions. Then define an assembly map for these quantum groups and study the relation with the one previously defined by Meyer and Nest. 

Similar questions have been treated during PI's thesis (in the setting of groupoids) and with several collaborators that are specialists in these areas (Christian Bonicke for groupoids, Ruben Martos for quantum groups).
\\

\noindent\textbf{Broader Impacts.}\\
The PI will continue his work training graduate students through organizing and speaking in seminars at his home institution. 

PI Dell?Aiera will educate mathematics students on the connections of the field to physics and will disseminate his results
through conference talks, and will give lecture series with the express aim of educating junior researchers.

PI Dell'Aiera intends to set up a blog to communicate about noncommutative geometry, and mathematics in general, to a broad auidence. Involving the graduate students and the participants of the seminar could also participate in their training.


\end{document}
