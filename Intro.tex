\section{Introduction}
\subsection{Notations}

Pour une $C^*$-algèbre $A$ non nécessairement unitale, on note $A^+$ la $C^*$-algèbre unitale qui la contient en tant qu'idéal bilatère, définie par :

\[\begin{array}{c}A^+=\{(a,\lambda)\in A\times \C \} \\ (a,\lambda)(b,\mu)=(ab+\lambda b +\mu a,\lambda\mu)\end{array}\]
 On a alors une suite exacte :
\[\begin{tikzcd}[column sep = small] 0 \arrow{r} &  A \arrow{r} & A^+ \arrow{r}{\pi_\C} & \C \arrow{r} & 0 \end{tikzcd}.\]

On rappelle que pour tout semi-groupe abélien $S$, il existe un groupe $G_S$, appelé groupe de Grothendieck de $S$, et un morphisme de semi-groupe $\mu : S\rightarrow G_S$ tels que, pour tout groupe $G$ et tout morphisme de semi-groupe $\alpha : S\rightarrow G$, il existe un unique morphisme de groupe $\tilde \alpha : G_S \rightarrow G$ vérifiant $\alpha = \mu \circ \tilde\alpha$. 

\[\begin{tikzcd} S \arrow{r}{\alpha} \arrow[dotted]{d}{\exists !\tilde\alpha}& G \\
	G_S \arrow{ur}{\mu} & \end{tikzcd}\]
 
Enfin, une remarque sur les limites inductives de $C^*$-algèbres. Par système inductif, on entend une famille de morphismes $\{\phi_{ij}:A_j\rightarrow A_i\}_{i>j}$, où $i$ et $j$ sont éléments d'un ensemble partiellement ordonné, vérifiant la condition de cohérence :
\[\phi_{ij}\circ\phi_{jk}=\phi_{ik}\quad \text{si }\ i>j>k.\]
Il existe alors un objet universel $A_\infty$, appelé la limite inductive algébrique du système $\{A_i; \phi_{ij}\}$, et des morphismes canoniques $\phi_i : A_i \rightarrow A_\infty$ qui rendent le diagramme suivant commutatif :
\[\begin{tikzcd} A_i \arrow{r}{\phi_i} & A_\infty \\
			A_j \arrow{ur}{\phi_j}\arrow{u}{\phi_{ij}} & 
\end{tikzcd}\]
et tel que $A_\infty = \cup \phi_j(A_j)$. Cet objet est universel au sens où, pour tout autre $A'_\infty$ et morphismes $\phi_i' : A_i\rightarrow A'_\infty$ qui font commuter le précédent diagramme, alors il existe un unique morphisme $A_\infty \rightarrow A'_\infty$ tel que le diagramme :
\[\begin{tikzcd}A_i \arrow{r}{\phi_i}\arrow{dr}{\phi'_i}& A_\infty\arrow[dotted]{d} \\
		    A_j \arrow{u}{\phi_{ij}}\arrow{ur}{\phi_j}\arrow{r}{\phi'_j}& A'_\infty
\end{tikzcd}\] 
commute. De plus, si chaque $\phi'_j$ est injectif, la flèche en pointillés l'est aussi. Et elle est surjective si $N= \cup \phi'_j A_j$. \\

Si $x=\phi_j(a_j)$,
\[\alpha(x):=\sup_i\{||\phi_{ij}(a_j)||\}\]
définit une semi-$C*$-norme sur $A_\infty$. On peut alors quotienter par l'idéal des éléments qui annulent $\alpha$, puis compléter par rapport à la norme obtenue sur le quotient. On étend ainsi la limite inductive à la catégorie des $C^*$-algèbre. Dans la suite de ce rapport, lorsque l'on parlera de limite inductive, cela désignera par défaut cette construction.
 
\subsection{Rappels sur les produits tensoriels de $C^*$-algèbres}

Cette section présente les résultats qui seront utilisés sur les produits tensoriels de $C^*$-algèbres. Toutes les preuves des affirmations non justifiées peuvent être trouvées dans le livre de Murphy ~\cite{Murphy} par exemple.\\

On rappelle que le produit tensoriel de $2$ espaces vectoriels $E$ et $F$ est défini comme l'unique (à isomorphisme près) espace vectoriel $E\otimes F$ muni d'une application bilinéaire $\pi : E\times F \rightarrow E\otimes F$ tel que, pour tout espace vectoriel $W$ et toute application bilinéaire $\phi : E \times F \rightarrow W $, il existe une unique application linéaire $\varphi :E\otimes F \rightarrow W$ telle que $\phi = \varphi\circ \pi$. On note $\pi(x,y)=x\otimes y$. Ce sont ces éléments, appelés tenseurs élémentaires, qui engendrent $E\otimes F$ comme espace vectoriel.\\ 

Le lemme suivant donne l'unicité à isomorphisme près :

\begin{lem}
Soient $E$ et $F$ deux espaces vectoriels sur un corps $K$. S'il existe deux $K$-espaces vectoriels $V_1$ et $V_2$ munis d'applications bilinéaires $\pi_j : E\times F \rightarrow V_j$ telles que, pour tout espace vectoriel $W$, toute application bilinéaire $E\times F \rightarrow W$,  se factorise uniquement via $\pi_1$ et $\pi_2$, alors $V_1$ et $V_2$ sont isomorphes en tant que $K$-espaces vectoriels.
\end{lem}


\[\begin{tikzcd}
V_1 \arrow{dr}& \arrow{l}{\pi_1}	E\times F \arrow{d}\arrow{r}{\pi_2}	& \arrow{dl}V_2 \\
			 & 		W		&	\\
\end{tikzcd}\]

\begin{dem}
En factorisant $\pi_j$ via $\pi_j$, il existe deux uniques applications linéaires $\phi_1 : V_2\rightarrow V_1$ et $\phi_2 : V_1\rightarrow V_2$ telles que :
\[\begin{array}{c}\pi_1=\phi_1\circ \pi_2 \\ \pi_2=\phi_2\circ \pi_1.\end{array}\]

Montrons que ces deux applications sont inverses. Comme :
\[\phi_1\circ \phi_2 \circ \pi_1 = \phi_1\circ \pi_2 =\pi_1 ,\]
$\phi_1\circ \phi_2 = id_{V_1}$ par unicité de la factorisation de $\pi_1$ via $\pi_1$.
Symétriquement, on démontre que : $\phi_2\circ \phi_1 = id_{V_2}$, et le résultat est démontré.\\
\qed
\end{dem}

Le problème qui va se poser, si l'on veut définir des produits tensoriels d'espaces vectoriels topologiques par exemple, est celui de la topologie que l'on veut définir sur celui-ci. Alexandre Grothendieck a étudié ces constructions dans sa thèse, voir le séminiaire Bourbaki ~\cite{GrothendieckNuc} pour une présentation. C'est au cours de sa thèse que A. Grothendieck a d'ailleurs introduit la nucléarité, notion clé pour les $C^*$-algèbres. On verra en effet que les $C^*$-algèbres nucléaires sont exactes : le foncteur obtenu en tensorisant par elle-même préserve l'exactitude des complexes de $C^*$-algèbres.\\

Sur les espaces de Hilbert, notre travail est simplifié : il existe un unique produit scalaire sur le produit tensoriel algébrique vérifiant : $\langle x\otimes x', y\otimes y'\rangle=\langle x, y\rangle \langle x',y' \rangle$. La complétion du produit tensoriel algébrique de $H$ et $K$ par rapport à ce produit scalaire est noté $H\hat \otimes K$. Ce résultat peut alors être transféré aux $C^*$-algèbres, grâce à leurs représentations sur des espaces de Hilbert. 

\begin{prop}
Soit $A$ et $B$ deux $C^*$-algèbres et $(H,\varphi)$ et $(K,\psi)$ deux représentations associées. Alors, il existe un unique $*$-homomorphisme $\pi : A\otimes B \rightarrow B(H\hat\otimes K)$ tel que $\pi(a\otimes b)= \varphi(a)\otimes \psi(b)$. De plus, $\pi$ est injectif si $\varphi$ et $\psi$ le sont.
\end{prop} 

On appelle représentation universelle d'une $C^*$-algèbre $A$ la somme directe de toutes les représentations $(H_\tau,\phi_\tau)$, $\tau$ parcourant l'espace des états de $A$, la représentation associée dérivant de la construction GNS. On peut alors définir deux normes sur le produit tensoriel algébrique de $2$ $C^*$-algèbres $A$ et $B$.

\begin{definition}
Soit $A$ et $B$ deux $C^*$-algèbres de représentations universelles $(H_A, \phi_A)$ et $(H_B, \phi_B)$. Soit $\pi$ l'unique $*$-homomorphisme donné par la proposition précédente : $\pi(a\otimes b ) = \phi_A(a)\otimes \phi_B(b)$.
\begin{itemize}
\item Le produit tensoriel spatial $A\otimes_{min} B$ est défini comme la complétion du produit tensoriel algébrique $A\otimes B$ par rapport à la norme 
\[||.||_{min}\left\{\begin{array}{rcl} A\otimes B & \rightarrow & \R_+ \\ c & \mapsto & ||\pi(c)||\end{array}\right.\]
\item Le produit tensoriel maximal $A\otimes_{max} B$ est défini comme la complétion du produit tensoriel algébrique $A\otimes B$ par rapport à la norme 
\[||.||_{max}\left\{\begin{array}{rcl} A\otimes B & \rightarrow & \R_+ \\ c & \mapsto & \max_{p} p(c)\end{array}\right.\]
où $p$ parcourt l'ensemble des semi-$C^*$-normes sur $A\otimes B$.
\end{itemize}
\end{definition}

On se servira de la propriété suivante : pour tout $*$-homomorphismes de $C^*$-algèbres $\varphi : A\rightarrow B$ et $\psi :  A'\rightarrow B'$, il existe un unique $*$-homomorphisme $\pi : A\otimes_{min} A' \rightarrow B\otimes_{min} B'$ tel que 
\[\pi(a\otimes a')=\varphi(a)\otimes \psi (a')\quad , \forall a\in A,a'\in A'.\]
De plus, $\pi $ est injective si $\varphi$ et $\psi$ le sont. On notera $\varphi\otimes_{min}\psi$ à la place de $\pi$ .\\

\begin{definition}
Une $C^*$-algèbre est dite nucléaire s'il n'existe qu'une seule $C^*$-norme sur le produit tensoriel algébrique $A\otimes B$, pour toute $C^*$-algèbre $B$.
\end{definition}

Dans les parties suivantes de ce rapport, lorsque l'on tensorisera par des $C^*$-algèbres nucléaires, on omettra le symbole $min$, et le produit tensoriel topologique sera noté comme  le produit tensoriel algébrique.

\begin{thm}
Soit $B$ une $C^*$-algèbre et
\begin{tikzcd}[column sep = small, row sep = tiny]
0 \arrow{r} & A' \arrow{r}{\varphi} & A \arrow{r}{\psi} & A'' \arrow{r} & 0 \\
\end{tikzcd}
une suite exacte de $C^*$-algèbres. Si le produit tensoriel algébrique $A''\otimes B$ n'admet qu'une seule $C^*$-norme, ce qui arrive lorsque $A''$ ou $B$ est nucléaire, alors la suite 
\[\begin{tikzcd}[column sep = small, row sep = tiny]
0 \arrow{r} & A' \otimes_{min} B\arrow{r}{\tilde\varphi} & A\otimes_{min} B \arrow{r}{\tilde\psi} & A'' \otimes_{min} B\arrow{r} & 0 \\
\end{tikzcd}\]
reste exacte.\\
On a noté $\tilde\varphi=\varphi\otimes_{min} id_B$ et $\tilde\psi = \psi\otimes_{min} id_B$.
\label{Nuclear}
\end{thm}

\begin{dem}
Soit $a''\otimes b \in A''\otimes B$. Par surjectivité, il existe $a\in A$ tel que $\psi(a)=a''$, et donc $\tilde\psi(a\otimes b ) = a''\otimes b$. Les éléments $a''\otimes b$ générant $A''\otimes B$, $\tilde \psi$ est surjective.\\

L'identité de $B$ et $\varphi$ étant des $*$-homomorphismes injectifs, $\tilde\varphi = \varphi \otimes_{min}id_B$ est injectif.\\

Observons que $\text{Im} \ \tilde\varphi =\text{Im}\ \varphi \otimes_{min} B\subset A\otimes_{min} B$ est un idéal.  On vérifie facilement que $\text{Im} \ \tilde\varphi \subset \text{ker}\ \tilde \psi $. Soit donc $R=\text{Im} \ \tilde\varphi$ et $f$ l'application canonique $(A\otimes_{min} B)/ R \rightarrow A''\otimes_{min} B$ obtenue en factorisant $\tilde \psi$. Nous allons construire une application $g : A''\otimes_{min} B \rightarrow (A\otimes_{min} B)/ R $ qui vérifie $g\circ f = id $. Cela montrera que $f$ est injective et donc que $\text{Im} \ \tilde\varphi = \text{ker}\ \tilde \psi $.\\

Soit $a''\in A''$. On choisit $a\in A$ tels que $\psi(a)=a''$ et on définit : 
\[\begin{array}{rcl}
A''\times B & \rightarrow & (A\otimes B) /R \\
a'', b          & \mapsto     & a\otimes b \ (\text{mod} \ R)\\
\end{array}\] 

Cette application ne dépend pas de la préimage de $a''$ choisie : si $\psi(a_1)=\psi(a_2)=a''$, alors $\psi(a_1-a_2)=0$ donc $a_1-a_2=\varphi(a')$ pour un certain $a'\in A'$, d'où $a_1\otimes b -a_2\otimes b = \varphi(a')\otimes b$ et donc $a_1\otimes b = a_2\otimes b \quad \text{mod}\ R$.\\
De plus, la fonction $A''\otimes B\rightarrow \R_+ :  x\mapsto \max (||g(x)||, ||x||_{min})$ est une $C^*$-norme. Par hypothèse, elle est donc égale à $||.||_{min}$, ce qui montre que $g$ est continue.
Etant bilinéaire et continue, cette application se factorise en l'application $g : A''\otimes_{min} B \rightarrow (A\otimes_{min} B) /R$ recherchée. En effet :
\[gf(a\otimes b + R )= g\tilde\psi(a\otimes b)= g(\psi(a)\otimes b)=a\otimes b +R\]
donc $g\circ f = id_{(A\otimes_{min} B)/R}$.

\qed
\end{dem}

Ce théorème sera utile pour la construction de l'extension de Toeplitz. Il sert aussi à la preuve du 

\begin{thm}
Soit
\begin{tikzcd}[column sep = small, row sep = tiny]
0 \arrow{r} & A' \arrow{r}{\varphi} & A \arrow{r}{\psi} & A'' \arrow{r} & 0 \\
\end{tikzcd}
une suite exacte de $C^*$-algèbres. Si $A'$ et $A''$ sont nucléaires, alors $A$ l'est aussi.
\label{NucExtension}
\end{thm}

Les $C^*$-algèbres finies dimensionelles ainsi que les $C^*$-algèbres commutatives sont nucléaires.

\subsection{Applications complètement positives et suites exactes}

Une application bornée $\sigma : A \rightarrow B$ entre deux $C^*$-algèbres unitales est dite complètement positive si $\sigma (1)=1$ et :
\[\sum_{i,j} b_i\sigma(a_i a_j^*) b_j^* \geq 0\]
pour tout entier $n$, tout $a_1$, ...,$a_n$ dans $A$ et $b_1$, ..., $b_n$ dans $B$. \\
Le résultat suivant caractérise les applications complètement positives.

\begin{thm}[Stinespring]
Soit $A$ une $C^*$-algèbre unitale. Une application unitale $\sigma : A \rightarrow B(H)$ est complètement positive ssi il existe :
\begin{itemize}
\item une isométrie $V : H \rightarrow H_1$  
\item une représentation non-dégénérée $\rho : A \rightarrow B(H_1)$
\end{itemize} 
telles que $\sigma(a)=V^*\rho(a)V$ pour tout $a\in A$.
\end{thm}

Ce résultat permet de montrer que si $\sigma : A \rightarrow A'$ est une application complètement positive, alors l'application $\sigma \otimes 1$, définie sur le produit tensoriel algébrique de manière évidente, s'étend en une application complètement positive $\sigma \otimes 1 : A\otimes B\rightarrow A'\otimes B$.\\
En effet, $\sigma$ étant complètement positive, elle est, en gardant les même notation, de la forme $\sigma(a)=V^*\rho(a)V$. Mais alors : 
\[(\sigma\otimes 1)(a\otimes b)=(V^*\rho(a)V) \otimes b =(V\otimes 1)^* (a \otimes b) (V\otimes 1).\]
 Les éléments $a \otimes b$ générant le produit tensoriel $A\otimes B$, l'extension par continuité de $\sigma \otimes 1$ est bien complètement positive. \\
Cette remarque nous permettra plus loin de construire l'extension de Toeplitz sans passer par la nucléarité. Il suffira alors de remarquer que lorsque l'on a une suite exacte courte scindée de $C^*$-algèbres, si la section est complètement positive, alors elle s'étend en une section complètement positive de la suite obtenue en tensorisant par une $C^*$-algèbre quelconque.\\

\begin{prop} Soit 
\begin{tikzcd}[column sep = small , row sep = tiny]
0 \arrow{r} & A' \arrow{r}{\varphi} & A \arrow{r}{\psi} & A'' \arrow{r} & 0 \\
\end{tikzcd}
une suite exacte de $C^*$-algèbres. Si la surjection $\psi$ a une section complètement positive, alors, pour toute $C^*$-algèbre $B$, la suite 
\[\begin{tikzcd}[column sep = small, , row sep = tiny]
0 \arrow{r} & A' \otimes_{min} B\arrow{r}{\tilde\varphi} & A\otimes_{min} B \arrow{r}{\tilde\psi} & A'' \otimes_{min} B\arrow{r} & 0 \\
\end{tikzcd}\]
reste exacte.
\label{CPexactness}
\end{prop}

\begin{dem}
L'injectivité de $\tilde\varphi$ et la surjectivité de $\tilde\psi$ fonctionne comme pour la preuve du théorème \ref{NucExtension}. On réitère la même technique pour montrer que la suite tensorisée est exacte au milieu. Toutefois, on ne peut plus utiliser l'unicité d'une $C^*$-norme sur $A''\otimes_{min} B$, et on a aucune assurance que l'application 
\[g : A''\times B \rightarrow (A\otimes_{min} B) / R\]
se factorise depuis $A"\otimes B$. \\

On peut par contre étendre la section complètement positive $\sigma : A"\rightarrow A$ en une section complètement positive $\tilde\sigma : A"\otimes_{min} B\rightarrow A\otimes_{min} B$. Alors $\tilde\psi = f\circ\pi$ et $\tilde\psi \circ \tilde\sigma = id_{A"\otimes B}$, donc $f\circ \pi\circ\tilde\sigma= id$, où 
\[\pi : A\otimes_{min} B \rightarrow (A\otimes_{min} B )/ R\]
est la projection naturelle sur $R=\text{Im}\ \tilde\varphi$.\\

 Pour montrer que $f$ est injective, il suffit donc de remarquer que $\pi\circ \tilde\sigma$ est un inverse à droite de $f$.

\qed
\end{dem}