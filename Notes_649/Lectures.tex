\section{Lecture 1}

Metric spaces
Cayley graphs

\section{Lecture 2}

What kind of morphisms should we use between metric spaces? We would like to find maps that capture the large scale geometric information between spaces.\\

We already have seen that isometries, i.e. maps $f : X\rightarrow Y$ such that $d(f(x),f(y)) = d(x,y)$ are too rigid. For instance, we would like to say that two Cayley graphs of the same finitely generated group look the same from far away. 

\begin{definition}
A quasi-isometry between metric spaces is a map $f : X\rightarrow Y$ such that there exist $L,C>0$ satisfying
\[\frac{1}{L} d(x,y) - C \leq d(f(x),f(y)) \leq L d(x,y) + C \quad \forall x,y \in X,\]
and there exists $r>0$ such that 
\[Y = \cup_{x\in X} B(f(x), r).\]
\end{definition}

Example: $(G,d_S)$ and $(G,d_T)$ are quasi-isometric. (Proof in class)\\

An action of a discrete group $G$ on $X$ is properly discontinuous if for every compact subset $K\subset X$
\[ | \{ g\in G  \ : \ gK \cap K \neq \emptyset \} | <\infty.\]

\begin{thm}
Let $X$ be a proper geodesic space, and $G$ a discrete group acting on $X$ by isometries. If the action is properly discontinuous and cocompact then $G$ is finitely generated and is quasi-isometric to $X$.
\end{thm} 

$S = \{ g \ : \ B(x_0, r)\cap B(gx_0,r)\neq \emptyset \}$.

Applications: 
\begin{itemize}
\item[$\bullet$] if $M$ is a compact Riemannian manifold and $G=\pi_1(M)$, then $G$ and the universal cover of $M$ are quasi-isometric (with the lifted $G$-invariant metric)
\item[$\bullet$] $PSL(2, \Z)$ is quasi-isometric to the upper-half plane with the hyperbolic metric $\frac{dx^2+dy^2}{y}$, hence to $\mathbb F_2$. The first point gives the generators.\\
\end{itemize}

Coarse maps and coarse embedding: definitions.\\

\begin{prop}
Any metric space admits an isometric embedding into a Banach space.
\end{prop}

$\phi(x) : X\rightarrow l^\infty (X) ;  s \mapsto d(s,x) - d(s,x_0)$.\\

We will be interested in the following question: when does a space a coarse embedding into a real Hilbert space?\\

Some examples of Hilbert spaces: $l^2(X)$, $l^2(X,H)$, $L^2(\Omega, \mu)$, random variable with second moments with $E(XY)$.\\

Definition of Hilbert spaces\\

A remark: the inner product can have a non-trivial null-space, Cauchy Schwartz still holds. In particular, the null-space will be a subspace and inner product is definite on the quotient. The \textit{separation-completion} of a real vector space $V$ will be 
\[H = \overline{V /N }^{\| .\|}.\]
It is a real Hilbert space.\\
 
Example: we have seen that $\Z^n$ is quasi-isometric to $\R^n$ with the Manhatthan metric, itself quasi-isometric to the Euclidean metric so $Z^n$ is CEH. Indeed
\[\sum_{i=1}^n |x_i | = \langle x,\epsilon\rangle \leq \| x\|_2 \|\epsilon \|_2 = \sqrt{n} \| x\|_2 .\]
Also: trees are CEH, $l^1(\mathbb N)$ is CEH. (Proofs)\\

Definition: positive definite kernels.\\

Example: Let $\phi: X\rightarrow H$ be any map. Then $k(x,y) = \langle \phi(x) , \phi(y)\rangle $ is a pdk.

\begin{thm}
Let $k: X\times X \rightarrow \R$ be a positive definite kernel. Then there exists a real Hilbert space and a map $\phi: X \rightarrow H$ such that 
\[k(x,y) = \langle \phi(x) , \phi(y)\rangle  \quad \forall x,y \in X.\]
\end{thm}

\begin{proof}
Let $V$ be the linear space of finitely supported real functions on $X$. Define
\[ \langle f , g \rangle = \sum_{x,y} f(x)f(y) k(x,y) \quad \forall x,y\in X.\]
It might happen than this inner-product has a nullspace. Let $H$ be the separation-completion of $V$.
Define \[\phi : X\rightarrow H ; x \mapsto e_x\]
where $e_x(x') = 1_{x=x'}$. Then
\[ \langle \phi(x) , \phi(y)\rangle = \sum_{x,y} e_x(x')e_y(y') k(x',y') = k(x,y). \] 
\end{proof}

\section{Lecture 3}

\subsection{Kernels}

A symmetric kernel on $X$ is a function $k: X\times X \rightarrow \R $ such that $k(x,y) = k(y,x)$. It is
\begin{itemize}
\item[$\bullet$] of positive type if for every finite subsets $F \subset X$
\[\langle v , K_F v \rangle \geq 0 \quad \forall v \in l^2(F).\] 
\item[$\bullet$] of conditionally negative type if for every finite subsets $F \subset X$
\[\langle v , K_F v \rangle \leq 0 \quad \forall v \in l^2(F)\text{ s.t. }\sum_{x\in F} v_x = 0.\] 
\end{itemize}

Examples: constants are (PT) and (CNT), for any map $f : X\rightarrow \R$, $k(x,y) =f(x)f(y)$ is (PT). If $k$ is (PT), $-k$ is (CNT). It is the conditionally that is here important, otherwise, it would be the opposite of being positive.\\

Operations on kernels: (PT) and (CNT) both form a cone (stability by addition and multiplication by a positive scalar), they are closed for the topology of simple convergence and by Schur multiplication. \\

Pb: show that Schur multiplication preserves (PT).\\

\begin{prop}
Let $k: X\times X\rightarrow \R $ be a positive type function on $X$ and $f(x)=\sum_n a_n z^n$ an entire function with positive coefficients. Then
\[\sum_n a_n k(x,y)^n\]
defines a positive type function. 
\end{prop}

\begin{thm} Schoenberg's lemma. Let $k: X\times X\rightarrow \R $ be a symmetric kernel on $X$. Then $k$ is (CNT) iff 
\[\forall t> 0 , \exp (-tk(x,y))\]
is of positive type.
\end{thm}

\begin{proof}
If $e^{-tk(x,y)}$ is (PT) for every $t>0$, then 
\[k(x,y) =\lim_{t\rightarrow 0} \frac{1-e^{-tk(x,y)}}{t}\]
is (CNT).\\

If $k(x,y)$ is (CNT), then there exists a real Hilbert space $H$ and a map $\phi: X \rightarrow H$ such that 
\[k(x,y) = \| \phi(x) - \phi(y) \|^2 \quad \forall x,y \in X.\]
Then
\[e^{tk(x,y)} = e^{-t \| \phi(x) \|^2 } e^{-t \| \phi(y) \|^2 } e^{2t \langle \phi(x), \phi(y) \rangle } \]
is a Schur product of a kernel of the type $f(x)f(y)$ with the exponential of a positive type, so it is positive. 
\end{proof}

Remark on the GNS construction.

When $k(x,y)$ is (CNT) then the separation completion of 
\[V = \{a : X\rightarrow \R \text{ finitely supported s.t. }\ : \ \sum_x a_x = 0\}\]
with respect to the positive bilinear form 
\[\langle a, b \rangle = -\frac{1}{2}\sum_{x,y} a_x b_x k(x,y) \]
gives a real Hilbert space $H$ and map $\phi: X \rightarrow H$ such that 
\[k(x,y) = \| \phi(x)-\phi(y) \|^2 \]
and $span\{\phi(x) - \phi(x_0)\}_{x\in X} $ is dense in $H$, where $x_0$ is any fixed point.\\ 

The couple $(H, \phi)$ is unique up to affine isomorphism: if $(K, \Psi)$ is another such couple, there exists a unique affine isomorphism $A: H \rightarrow K$ such that $\Psi =A\circ \phi $. 
 
\subsection{Kernels and approximation properties}

A function $\eta: X\times X \rightarrow \R$ is
\begin{itemize}
\item[$\bullet$] proper if $\eta^{-1}([-r,r])$ is an entourage for every $r>0$,
\item[$\bullet$] $\sup_{(x,y)\in \Delta_R} | \eta(x,y) | < \infty$.
\end{itemize}

\begin{thm}
Let $X$ be a countable discrete space with bounded geometry. Then the following are equivalent:
\begin{itemize}
\item[$\bullet$] X admits a coarse embedding into Hilbert space
\item[$\bullet$] there exists a proper \text{conditionally negative} type function that is bounded on entourages.
\end{itemize}
\end{thm}

First implication: $\eta(x,y) = \| \phi(x) - \phi(y)\|^2$.\\

Definitions: $G$ residually finite w.r.t. $\mathcal N = \{N_n\}$, $q_n: G \rightarrow G_n = G/N_n$, box space
\[X_{\mathcal N} = \coprod Cay(G_n, q_n(S)).\]
Haagerup's property or a-T-menability.

\begin{prop}
If $X_{\mathcal N}$ is (CEH), then $G$ is a-T-menable.
\end{prop}

\begin{proof}
By definition, we get a sequence of negative type functions 
\[\phi_n : G_n \rightarrow H\]
with 
\[\rho_-(l_n(x)) \leq \phi_n(x) \leq \rho_+(l_n(g)).\]
The composition $\phi_n\circ q_n$
\end{proof}

Remark: the converse does not hold: the free group on $n$ generators is a-T-menable, and any finitely generated group is a quotient of a $\mathbb F_n$, so we get a coarse embedding
\[X(G) \rightarrow X(\mathbb F_n)\]
so that the converse to the last proposition would imply that any box space is (CEH). But any infinite residually finite property (T) group contradicts this, $SL(3,\Z)$ for instance.

\section{Lecture 4}

Positive type and conditionally negative type functions on groups: the $G$-equivariance gives a unitary (or orthogonal representation), i.e.
\[\phi: G\rightarrow H \iff (H,\pi,v) \text{ GNS triple}.\] 
GNS construction $\phi(g) = \langle v,\pi(g)v\rangle$, i.e. $\phi(s^{-1}t) = \langle \pi(s)v,\pi(t)v\rangle$ wich is positive. $\eta(g)= \|\pi(g)v -v\|$. \\

$G$ is a-T-menable (metrically proper action by affine isometries on a real Hilbert space) iff there exists a $G$-equivariant proper negative type function on $G$.\\

Consequence: there is a homotopy between the trivial representation and the regular representation.
