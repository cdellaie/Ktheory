\section{Lecture 1}

Metric spaces\\
Cayley graphs\\

Ref: chapter 1 of Nowak, Yu, Large Scale geometry.
\section{Lecture 2}

What kind of morphisms should we use between metric spaces? We would like to find maps that capture the large scale geometric information between spaces.\\

We already have seen that isometries, i.e. maps $f : X\rightarrow Y$ such that $d(f(x),f(y)) = d(x,y)$ are too rigid. For instance, we would like to say that two Cayley graphs of the same finitely generated group look the same from far away. 

\begin{definition}
A quasi-isometry between metric spaces is a map $f : X\rightarrow Y$ such that there exist $L,C>0$ satisfying
\[\frac{1}{L} d(x,y) - C \leq d(f(x),f(y)) \leq L d(x,y) + C \quad \forall x,y \in X,\]
and there exists $r>0$ such that 
\[Y = \cup_{x\in X} B(f(x), r).\]
\end{definition}

Example: $(G,d_S)$ and $(G,d_T)$ are quasi-isometric. (Proof in class)\\

An action of a discrete group $G$ on $X$ is properly discontinuous if for every compact subset $K\subset X$
\[ | \{ g\in G  \ : \ gK \cap K \neq \emptyset \} | <\infty.\]

\begin{thm}
Let $X$ be a proper geodesic space, and $G$ a discrete group acting on $X$ by isometries. If the action is properly discontinuous and cocompact then $G$ is finitely generated and is quasi-isometric to $X$.
\end{thm} 

$S = \{ g \ : \ B(x_0, r)\cap B(gx_0,r)\neq \emptyset \}$.

Applications: 
\begin{itemize}
\item[$\bullet$] if $M$ is a compact Riemannian manifold and $G=\pi_1(M)$, then $G$ and the universal cover of $M$ are quasi-isometric (with the lifted $G$-invariant metric)
\item[$\bullet$] $PSL(2, \Z)$ is quasi-isometric to the upper-half plane with the hyperbolic metric $\frac{dx^2+dy^2}{y}$, hence to $\mathbb F_2$. The first point gives the generators.\\
\end{itemize}

Coarse maps and coarse embedding: definitions.\\

\begin{prop}
Any metric space admits an isometric embedding into a Banach space.
\end{prop}

$\phi(x) : X\rightarrow l^\infty (X) ;  s \mapsto d(s,x) - d(s,x_0)$.\\

We will be interested in the following question: when does a space a coarse embedding into a real Hilbert space?\\

Some examples of Hilbert spaces: $l^2(X)$, $l^2(X,H)$, $L^2(\Omega, \mu)$, random variable with second moments with $E(XY)$.\\

Definition of Hilbert spaces\\

A remark: the inner product can have a non-trivial null-space, Cauchy Schwartz still holds. In particular, the null-space will be a subspace and inner product is definite on the quotient. The \textit{separation-completion} of a real vector space $V$ will be 
\[H = \overline{V /N }^{\| .\|}.\]
It is a real Hilbert space.\\
 
Example: we have seen that $\Z^n$ is quasi-isometric to $\R^n$ with the Manhatthan metric, itself quasi-isometric to the Euclidean metric so $Z^n$ is CEH. Indeed
\[\sum_{i=1}^n |x_i | = \langle x,\epsilon\rangle \leq \| x\|_2 \|\epsilon \|_2 = \sqrt{n} \| x\|_2 .\]
Also: trees are CEH, $l^1(\mathbb N)$ is CEH. (Proofs)\\

Definition: positive definite kernels.\\

Example: Let $\phi: X\rightarrow H$ be any map. Then $k(x,y) = \langle \phi(x) , \phi(y)\rangle $ is a pdk.

\begin{thm}
Let $k: X\times X \rightarrow \R$ be a positive definite kernel. Then there exists a real Hilbert space and a map $\phi: X \rightarrow H$ such that 
\[k(x,y) = \langle \phi(x) , \phi(y)\rangle  \quad \forall x,y \in X.\]
\end{thm}

\begin{proof}
Let $V$ be the linear space of finitely supported real functions on $X$. Define
\[ \langle f , g \rangle = \sum_{x,y} f(x)f(y) k(x,y) \quad \forall x,y\in X.\]
It might happen than this inner-product has a nullspace. Let $H$ be the separation-completion of $V$.
Define \[\phi : X\rightarrow H ; x \mapsto e_x\]
where $e_x(x') = 1_{x=x'}$. Then
\[ \langle \phi(x) , \phi(y)\rangle = \sum_{x,y} e_x(x')e_y(y') k(x',y') = k(x,y). \] 
\end{proof}

Ref: Chapter 1 and 5 of Nowak, Yu, Large scale geometry.

\section{Lecture 3}

Last time we defined coarse embeddings and quasi-isometries. An interesting example of a coarse embedding that may not be a quasi isometriy is the inclusion of a subgroup $H$ into a group $G$.

To see that, consider the Baumslag Solitar group 
\[G =BS(1,2) = \langle a, b \  : b^{-1}ab = a^2 \rangle ,\]
and let $H = \langle a \rangle \cong \Z$. One sees that, whereas $l_H(a^n) = n$, the relation $a^{2^n} = b^{-n}a b^n$ ensures that $l_G(a^{2^n}) \leq 2n+ 1$, so that $l(a^n)$ is of the same order as $log_2(n)$. 

\subsection{Kernels}

A symmetric kernel on $X$ is a function $k: X\times X \rightarrow \R $ such that $k(x,y) = k(y,x)$. It is
\begin{itemize}
\item[$\bullet$] of positive type if for every finite subsets $F \subset X$
\[\langle v , K_F v \rangle \geq 0 \quad \forall v \in l^2(F).\] 
\item[$\bullet$] of conditionally negative type if for every finite subsets $F \subset X$
\[\langle v , K_F v \rangle \leq 0 \quad \forall v \in l^2(F)\text{ s.t. }\sum_{x\in F} v_x = 0.\] 
\end{itemize}

Examples: constants are (PT) and (CNT), for any map $f : X\rightarrow \R$, $k(x,y) =f(x)f(y)$ is (PT). If $k$ is (PT), $-k$ is (CNT). It is the conditionally that is here important, otherwise, it would be the opposite of being positive.\\

Operations on kernels: (PT) and (CNT) both form a cone (stability by addition and multiplication by a positive scalar), they are closed for the topology of simple convergence and by Schur multiplication. \\

Pb: show that Schur multiplication preserves (PT).\\

\begin{prop}
Let $k: X\times X\rightarrow \R $ be a positive type function on $X$ and $f(x)=\sum_n a_n z^n$ an entire function with positive coefficients. Then
\[\sum_n a_n k(x,y)^n\]
defines a positive type function. 
\end{prop}

\begin{thm} Schoenberg's lemma. Let $k: X\times X\rightarrow \R $ be a symmetric kernel on $X$. Then $k$ is (CNT) iff 
\[\forall t> 0 , \exp (-tk(x,y))\]
is of positive type.
\end{thm}

\begin{proof}
If $e^{-tk(x,y)}$ is (PT) for every $t>0$, then 
\[k(x,y) =\lim_{t\rightarrow 0} \frac{1-e^{-tk(x,y)}}{t}\]
is (CNT).\\

If $k(x,y)$ is (CNT), then there exists a real Hilbert space $H$ and a map $\phi: X \rightarrow H$ such that 
\[k(x,y) = \| \phi(x) - \phi(y) \|^2 \quad \forall x,y \in X.\]
Then
\[e^{tk(x,y)} = e^{-t \| \phi(x) \|^2 } e^{-t \| \phi(y) \|^2 } e^{2t \langle \phi(x), \phi(y) \rangle } \]
is a Schur product of a kernel of the type $f(x)f(y)$ with the exponential of a positive type, so it is positive. 
\end{proof}

Remark on the GNS construction.

When $k(x,y)$ is (CNT) then the separation completion of 
\[V = \{a : X\rightarrow \R \text{ finitely supported s.t. }\ : \ \sum_x a_x = 0\}\]
with respect to the positive bilinear form 
\[\langle a, b \rangle = -\frac{1}{2}\sum_{x,y} a_x b_x k(x,y) \]
gives a real Hilbert space $H$ and map $\phi: X \rightarrow H$ such that 
\[k(x,y) = \| \phi(x)-\phi(y) \|^2 \]
and $span\{\phi(x) - \phi(x_0)\}_{x\in X} $ is dense in $H$, where $x_0$ is any fixed point.\\ 

The couple $(H, \phi)$ is unique up to affine isomorphism: if $(K, \Psi)$ is another such couple, there exists a unique affine isomorphism $A: H \rightarrow K$ such that $\Psi =A\circ \phi $. 
 
\subsection{Kernels and approximation properties}

A function $\eta: X\times X \rightarrow \R$ is
\begin{itemize}
\item[$\bullet$] proper if $\eta^{-1}([-r,r])$ is an entourage for every $r>0$,
\item[$\bullet$] $\sup_{(x,y)\in \Delta_R} | \eta(x,y) | < \infty$.
\end{itemize}

\begin{thm}
Let $X$ be a countable discrete space with bounded geometry. Then the following are equivalent:
\begin{itemize}
\item[$\bullet$] X admits a coarse embedding into Hilbert space
\item[$\bullet$] there exists a proper \text{conditionally negative} type function that is bounded on entourages.
\end{itemize}
\end{thm}

First implication: $\eta(x,y) = \| \phi(x) - \phi(y)\|^2$.\\

Definitions: $G$ residually finite w.r.t. $\mathcal N = \{N_n\}$, $q_n: G \rightarrow G_n = G/N_n$, box space
\[X_{\mathcal N} = \coprod Cay(G_n, q_n(S)).\]
Haagerup's property or a-T-menability.

\begin{prop}
If $X_{\mathcal N}$ is (CEH), then $G$ is a-T-menable.
\end{prop}

\begin{proof}
By definition, we get a sequence of negative type functions 
\[\phi_n : G_n \rightarrow H\]
with 
\[\rho_-(l_n(x)) \leq \phi_n(x) \leq \rho_+(l_n(g)).\]
The composition $\phi_n\circ q_n$
\end{proof}

Remark: the converse does not hold: the free group on $n$ generators is a-T-menable, and any finitely generated group is a quotient of a $\mathbb F_n$, so we get a coarse embedding
\[X(G) \rightarrow X(\mathbb F_n)\]
so that the converse to the last proposition would imply that any box space is (CEH). But any infinite residually finite property (T) group contradicts this, $SL(3,\Z)$ for instance.\\

Ref: Chapter 5 and 6 of Nowak, Yu, Large scale geometry, and Appendix C of Bekka, Valette, Kazdhan Property (T)

\section{Lecture 4}

Definition of the Laplacian on a graph: for any Hilbert space $H$,
\[(\Delta f)(x) = f(x) - \frac{1}{N_x}\sum_{y : (x,y)\in E} f(y)\quad \forall f\in l^2(X,H).\]
On a finite graph, the Laplacian is a symmetric (or selfadjoint in the complex case) positive matrix with kernel the constant functions, and its orthogonal is 
\[ker(\Delta)^\perp = \{ f\in l^2(X) \ : \ \sum_{x\in X} f(x) = 0\}.\]

\subsection{Expanders}

\begin{definition}
An expander sequence is a sequence of finite graphs $X_n= (V_n, E_n)$ such that 
\begin{itemize}
\item[$\bullet$] $\lim_n |V_n| = +\infty $,
\item[$\bullet$] $d=\sup_n deg(X_n)< \infty$,
\item[$\bullet$] there exists a constant $c>0$ such that, for every $n$, $sp(\Delta_{X_n}) \subset \{0\} \cup (c,\infty)$.
\end{itemize}
\end{definition} 

Out of an expander sequence, we build its \textit{coarse disjoint union}
\[X = \coprod X_n\]
which is the disjoint union as a set, endowed with a metric $d$ which restricts to the graph metric on any of the graphs, and such that $\lim_{i,j\rightarrow \infty} d(X_i,X_j)=+\infty$. It is $X$ that we call an expander.\\

The last condition means that for every $f\in l^2(X,H)$ such that $\sum_{x\in X_n} f(x)=0$, the inequality
\[\sum_{(x,y)\in E_n} \| f(x)-f(y) \|^2 \geq c \sum_{x\in X_n} \| f(x)\|^2    \]
holds.

\begin{thm}
Expanders cannot admit a coarse embedding into Hilbert space.
\end{thm}

\begin{proof} 
I think this proof is due to Higson. Let $f: X \rightarrow H$ be a coarse embedding. One gets vectors $f_n\in l^2(X_n,H)$ by restricting $f$ to $X_n$. We can translate each $f_n$ without changing the fact that it is a coarse embedding, so we can suppose $\sum_{x\in X_n} f_n(x)=0$, hence
\[\sum_{(x,y)\in E_n} \| f_n(x)-f_n(y) \|^2 \geq c \sum_{x\in X_n} \| f(x_n)\|^2.\]
The term on the left is less than $d |V_n| \rho_+^2(1)$ so that we get a inequality of the type
\[\sum_{x\in X_n} \| f(x_n)\|^2 \leq A |V_n|\]
with $A$ a constant independent of $n$. That is possible only if at least half of the terms are less than $2A$. But that is a contradiction with the properness of $f$, since it means that $f_n$ maps more than $\frac{|X_n|}{2}$ terms into $B(0,2A)$.\\	 
\end{proof}

We will understand better the following diagram
\[\begin{tikzcd}
\text{amenablity} \arrow{r}\arrow{d} & \text{a-T-menability} \arrow{d} \\
\text{ property (A)} \arrow{r} & \text{ (CEH) }  
\end{tikzcd}\]
where the top row is a $G$-equivariant version of the bottom one, whereas the left side corresponds to finitely supported kernels, relaxed into a $c_0$ condition on the right. There is a parallel diagram with property (T), and we will see what should be a non-equivariant version of property (T). There is a way to restore equivariance by considering the \textit{coarse groupoid}.

\subsection{GNS construction}

We have seen that for any symmetric kernel $k: X\times X \rightarrow \R$ of positive type, there exists a pair $GNS(k)=(H,\phi)$ satisfying
\begin{itemize}
\item[$\bullet$] $H$ is a real Hilbert space,
\item[$\bullet$] $\phi : X\rightarrow H$ is a map such that $\langle \phi(x),\phi(y) \rangle = k(x,y)$ and $span \{ \phi(x)\ : \ x\in X \} $ is dense in $H$. 
\end{itemize}

Moreover, the pair $(H,\phi)$ satisfies the following universal property: let $(K,\psi)$ be any other pair such that $\langle \psi(x),\psi(y) \rangle = k(x,y)$. Then there exists a unique linear isometry $T: H\rightarrow K$ such that $\psi = T \circ \phi$.\\

Indeed, the identity 
\[\langle \sum_i a_i \delta_{x_i}, \sum_j b_j \delta_{y_j} \rangle  = \sum_{i,j} a_i b_j k(x_i,_j) =\langle \sum_i a_i \psi(x_i), \sum_j b_j \psi(y_j) \rangle\]
proves that the map $\sum_i a_i \delta_{x_i}\mapsto  \sum_i a_i \psi(x_i)$ extends to an isometry $T:H \rightarrow K $ satisfying $\psi = T \circ \phi$. Uniqueness is obvious.\\

Now, say that $X=G$ is a group, and that the kernel is $G$-equivariant, i.e.
\[k(gx,gy) = k(x,y)\quad \forall g,x,y\in G.\]
Such a function is entirely determined by 
\[\tilde k (g) = k(e,g),\]
which is then what is called a positive type function on the group $G$.\\

In this case, the GNS construction and its universal property allows to build a unitary representation. Indeed 
\[\langle \phi(gx),\phi(gy) \rangle = k(gx,gy) = k(x,y)\] 
ensures the existence of a unique linear isometry $\pi_g: H \rightarrow H$ such that $\phi(gx) = \pi_g \phi(x)$, $\forall g,x \in G$. The uniqueness also ensures that $\pi_s \pi_t = \pi_{st} $ and $\pi_e = id_H$. Together with symmetry ($\tilde k(g^{-1})=\tilde k(g)$), it implies that $\pi_g^*=\pi_{g^{-1}}$. So that, if $v= \phi(e)\in H$, we have that the GNS pair $(H,\phi)$ is actually equivalent to a triple $(H,\pi, v)$ where 
\begin{itemize}
\item[$\bullet$] $H$ is a real Hilbert space,
\item[$\bullet$] $\pi : G\rightarrow U(H)$ is a unitary representation, 
\item[$\bullet$] $v\in H$ is a vector,
\end{itemize}
such that $span \{ \pi_g (v)\}_{g\in G}$ is dense in $H$ (we say $v$ is \textit{cyclic} for $\pi$), and $\tilde k(g)=\langle v , \pi_g (v)\rangle$, for all $g\in G$.\\

The triple $(H,\pi,v)$ is called the GNS triple associated to the positive type function $\tilde k$. It satisfies the following universal property: if $(K,\sigma , w)$ is any other triple such that $\tilde k(g)=\langle w , \sigma_g (w)\rangle$, then there exists a unique linear isometry $T: H\rightarrow K $ such that $Tv= w$ and $T\pi_g = \sigma_g T$.\\

As a problem, show that the GNS triple of a cyclic representation is unitarily equivalent to the original representation, i.e. 
\[GNS(\pi_{v,v}) \cong (H,\pi,v).\]  

As a result, we get that the family of equivalence classes of cyclic representation is a set, isomorphic to the set of positive type functions on $G$.
\[\{ \pi: G\rightarrow U(H) \text{ cyclic } \} / \sim \cong \{\phi : G \rightarrow \R \text{ positive type}\}.\] 

We will see next time how we can use a natural topology on the space of bounded functions on $G$ to topologize the space of cyclic representations of $G$. (This will be equivalent to the Fell topology)

\section{Lecture 5}
Recall that cyclic unitary representations corresponds to positive type functions on the group via the GNS construction. Cauchy-Schwartz ensures that
\[|\phi(g) | = |\langle v_\phi , \pi_\phi(g)v_\phi \rangle | \leq \|v_\phi \|^2,\]
hence $\phi\in l^\infty(G)$, and $\|\phi\|_\infty = |\phi(e)| =\|v_\phi \|^2$. But $(l^\infty(G), \|\cdot\|_\infty )$ is a Banach space. 
\[\tilde G\cong \mathcal C :=\{\phi: G\rightarrow \R \text{ PT s.t. }\phi(e) = 1\}\] 
Positive type functions form a cone closed w.r.t the topology of pointwise convergence, so the weak-$*$ topology. $\mathcal C$ is a closed convex subspace, so that Krein-Milman ensures that the set of extremal points is weak-$*$ dense in $\mathcal C$.

\begin{prop}
$Ext(\mathcal C)$ corresponds to the irreducible unitary representations via the GNS-construction.
\end{prop}  
 
This is an easy consequence of how the operations on positive type functions tranlate at the level of representations.
\[GNS(t\phi)\cong (H_\phi, \pi_\phi, \sqrt{t}\cdot v_\phi)\]
\[GNS(\phi_1+\phi_2)\cong (H_{\phi_1} \oplus H_{\phi_2}, \pi_{\phi_1}\oplus \pi_{\phi_2}, v_{\phi_1}\oplus v_{\phi_2})\]
\[GNS(t\phi)\cong (H_{\phi_1} \otimes H_{\phi_2}, \pi_{\phi_1}\otimes \pi_{\phi_2}, v_{\phi_1}\otimes v_{\phi_2})\]

The set of equivalence classes of unitary representations is endowed with the Fell topology: it is the topology of uniform convergence of coefficients on compact subsets. It corresponds exactly to the weak-$*$ toplogy on the space of normalized positive type functions.

\begin{thm} [Raikov]
The weak-$*$ topology and the topology of uniform convergence on compact subsets, i.e. for every net $\phi_i$ of positive type normalized functions 
\[ \forall f \in l^1(G), \quad \lim_i\langle \phi_i , f\rangle = \langle \phi , f\rangle \iff \forall F\subset G\text{ finite}, \quad \lim_i\|\phi_i -\phi\|_{\infty , F} = 0 \] 
\end{thm} 
\begin{proof}
Let $\phi_i$ converge uniformly on every finite subsets to $\phi$. Let $f\in l^1(G)$ and $\varepsilon>0$. There exists a finite subset $F\subset G$ such that 
\[\sum_{g\in G-F} |f(g)| \leq \frac{\varepsilon}{4}\]
and a rank $n$ such that 
\[\forall i\geq n, \|\phi_i -\phi\|_{\infty , F} \leq \frac{\varepsilon }{2\sum_{F} |f(g)| }\]
and 
\[\begin{split}
|\langle \phi - \phi_i , f \rangle | & \leq \|\phi_i -\phi\|_{\infty , F} \sum_{F} |f(g)|+2\sum_{G-F}|f(g)| \\
				& \leq \varepsilon \\ 
\end{split}\]
So that $\phi_i$ converges to $\phi$ in the weak-$*$ topology.\\
 
Let $\phi_i$ converge to $\phi$ in the weak-$*$ topology. Let $F\subset G$ be finite, and $\varepsilon>0$. Define $f=\chi_F\in l^1(G)$ and, up to changing the value of $f$ to $-1$, we get that, ultimately,  
\[ |\phi(g)-\phi_i(g)|\leq \sum_{g\in F} |\phi(g)-\phi_i(g)| = \langle \phi - \phi_i , f \rangle   \leq \varepsilon \quad \forall g\in F\]
which ensures that $\|\phi_i-\phi\|_{\infty, F}\leq \varepsilon$.
\end{proof}

Going back to a-T-menability.

\begin{definition} We say that $G$ is a-T-menable, or has Haagerup's property, if there exists a (metrically) proper action of $G$ on a real Hilbert space by affine isometries, i.e there exists a group morphism 
\[\alpha:\left\{\begin{array}{rcl} G & \rightarrow & Aff(H) \\ v & \mapsto & \alpha_g(v) = \pi_g(v)+b_g\end{array}\right.\] 
with 
\begin{itemize}
\item[$\bullet$] $\pi: G \rightarrow U(H)$ is a unitary representation for $G$,
\item[$\bullet$] $b: G\rightarrow H$ is a cocycle for $\pi$, meaning $b_{st} = \pi_s b_t+ b_s$, for every $s,t \in G$,
\item[$\bullet$] $\alpha$ metrically proper, i.e. $\lim_{l(g)\rightarrow \infty} \|b_g\|^2 =\infty$.
\end{itemize}
\end{definition}
Remarks:
\begin{itemize}
\item[$\bullet$] the cocycle relation ensures that $\alpha $ is a group morphism. In particular, it also yields 
\[ \|b_s- b_t\|^2 = \| b_{s^{-1}t} \|^2 ,\]
ensuring that $\phi(g)=\|b_g- b_e\|^2$ is a conditionally negative type function on $G$.
\item[$\bullet$] 
\end{itemize}

\begin{thm} If $G$ is a-T-menable, then $G$ admits a coarse embedding into Hilbert space.
\end{thm}
\begin{proof}
It suffices to show that $\phi(g)=\|b_g- b_e\|^2$ is proper, which is guaranted by the fact that $\lim_{l(g)\rightarrow \infty} \|b_g\|^2 =\infty$.\\
\end{proof}

An interesting consequence is the following. 
\begin{prop}
If $G$ is a-T-menable, there exists a continuous path of representations between the trivial representation and the quasi-regular representation $\lambda_{G/H}$ for some subgroup $H<G$.
\end{prop}

\begin{proof}
\[\phi_t(g) = \exp(-t\eta(g))\]
\end{proof} 
\section{Lecture 6}

Positive type and conditionally negative type functions on groups: the $G$-equivariance gives a unitary (or orthogonal representation), i.e.
\[\phi: G\rightarrow H \iff (H,\pi,v) \text{ GNS triple}.\] 
GNS construction $\phi(g) = \langle v,\pi(g)v\rangle$, i.e. $\phi(s^{-1}t) = \langle \pi(s)v,\pi(t)v\rangle$ which is positive. $\eta(g)= \|\pi(g)v -v\|$. \\

$G$ is a-T-menable (metrically proper action by affine isometries on a real Hilbert space) iff there exists a $G$-equivariant proper negative type function on $G$.\\

Consequence: there is a homotopy between the trivial representation and the regular representation.\\

Gelfand Raikov theorem
