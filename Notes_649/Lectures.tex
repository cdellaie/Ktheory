\section{Lecture 1}

Metric spaces\\
Cayley graphs\\

Ref: chapter 1 of Nowak, Yu, Large Scale geometry.
\section{Lecture 2}

What kind of morphisms should we use between metric spaces? We would like to find maps that capture the large scale geometric information between spaces.\\

We already have seen that isometries, i.e. maps $f : X\rightarrow Y$ such that $d(f(x),f(y)) = d(x,y)$ are too rigid. For instance, we would like to say that two Cayley graphs of the same finitely generated group look the same from far away. 

\begin{definition}
A quasi-isometry between metric spaces is a map $f : X\rightarrow Y$ such that there exist $L,C>0$ satisfying
\[\frac{1}{L} d(x,y) - C \leq d(f(x),f(y)) \leq L d(x,y) + C \quad \forall x,y \in X,\]
and there exists $r>0$ such that 
\[Y = \cup_{x\in X} B(f(x), r).\]
\end{definition}

Example: $(G,d_S)$ and $(G,d_T)$ are quasi-isometric. (Proof in class)\\

An action of a discrete group $G$ on $X$ is properly discontinuous if for every compact subset $K\subset X$
\[ | \{ g\in G  \ : \ gK \cap K \neq \emptyset \} | <\infty.\]

\begin{thm}
Let $X$ be a proper geodesic space, and $G$ a discrete group acting on $X$ by isometries. If the action is properly discontinuous and cocompact then $G$ is finitely generated and is quasi-isometric to $X$.
\end{thm} 

$S = \{ g \ : \ B(x_0, r)\cap B(gx_0,r)\neq \emptyset \}$.

Applications: 
\begin{itemize}
\item[$\bullet$] if $M$ is a compact Riemannian manifold and $G=\pi_1(M)$, then $G$ and the universal cover of $M$ are quasi-isometric (with the lifted $G$-invariant metric)
\item[$\bullet$] $PSL(2, \Z)$ is quasi-isometric to the upper-half plane with the hyperbolic metric $\frac{dx^2+dy^2}{y}$, hence to $\mathbb F_2$. The first point gives the generators.\\
\end{itemize}

Coarse maps and coarse embedding: definitions.\\

\begin{prop}
Any metric space admits an isometric embedding into a Banach space.
\end{prop}

$\phi(x) : X\rightarrow l^\infty (X) ;  s \mapsto d(s,x) - d(s,x_0)$.\\

We will be interested in the following question: when does a space a coarse embedding into a real Hilbert space?\\

Some examples of Hilbert spaces: $l^2(X)$, $l^2(X,H)$, $L^2(\Omega, \mu)$, random variable with second moments with $E(XY)$.\\

Definition of Hilbert spaces\\

A remark: the inner product can have a non-trivial null-space, Cauchy Schwartz still holds. In particular, the null-space will be a subspace and inner product is definite on the quotient. The \textit{separation-completion} of a real vector space $V$ will be 
\[H = \overline{V /N }^{\| .\|}.\]
It is a real Hilbert space.\\
 
Example: we have seen that $\Z^n$ is quasi-isometric to $\R^n$ with the Manhatthan metric, itself quasi-isometric to the Euclidean metric so $Z^n$ is CEH. Indeed
\[\sum_{i=1}^n |x_i | = \langle x,\epsilon\rangle \leq \| x\|_2 \|\epsilon \|_2 = \sqrt{n} \| x\|_2 .\]
Also: trees are CEH, $l^1(\mathbb N)$ is CEH. (Proofs)\\

Definition: positive definite kernels.\\

Example: Let $\phi: X\rightarrow H$ be any map. Then $k(x,y) = \langle \phi(x) , \phi(y)\rangle $ is a pdk.

\begin{thm}
Let $k: X\times X \rightarrow \R$ be a positive definite kernel. Then there exists a real Hilbert space and a map $\phi: X \rightarrow H$ such that 
\[k(x,y) = \langle \phi(x) , \phi(y)\rangle  \quad \forall x,y \in X.\]
\end{thm}

\begin{proof}
Let $V$ be the linear space of finitely supported real functions on $X$. Define
\[ \langle f , g \rangle = \sum_{x,y} f(x)f(y) k(x,y) \quad \forall x,y\in X.\]
It might happen than this inner-product has a nullspace. Let $H$ be the separation-completion of $V$.
Define \[\phi : X\rightarrow H ; x \mapsto e_x\]
where $e_x(x') = 1_{x=x'}$. Then
\[ \langle \phi(x) , \phi(y)\rangle = \sum_{x,y} e_x(x')e_y(y') k(x',y') = k(x,y). \] 
\end{proof}

Ref: Chapter 1 and 5 of Nowak, Yu, Large scale geometry.

\section{Lecture 3}

Last time we defined coarse embeddings and quasi-isometries. An interesting example of a coarse embedding that may not be a quasi isometriy is the inclusion of a subgroup $H$ into a group $G$.

To see that, consider the Baumslag Solitar group 
\[G =BS(1,2) = \langle a, b \  : b^{-1}ab = a^2 \rangle ,\]
and let $H = \langle a \rangle \cong \Z$. One sees that, whereas $l_H(a^n) = n$, the relation $a^{2^n} = b^{-n}a b^n$ ensures that $l_G(a^{2^n}) \leq 2n+ 1$, so that $l(a^n)$ is of the same order as $log_2(n)$. 

\subsection{Kernels}

A symmetric kernel on $X$ is a function $k: X\times X \rightarrow \R $ such that $k(x,y) = k(y,x)$. It is
\begin{itemize}
\item[$\bullet$] of positive type if for every finite subsets $F \subset X$
\[\langle v , K_F v \rangle \geq 0 \quad \forall v \in l^2(F).\] 
\item[$\bullet$] of conditionally negative type if for every finite subsets $F \subset X$
\[\langle v , K_F v \rangle \leq 0 \quad \forall v \in l^2(F)\text{ s.t. }\sum_{x\in F} v_x = 0.\] 
\end{itemize}

Examples: constants are (PT) and (CNT), for any map $f : X\rightarrow \R$, $k(x,y) =f(x)f(y)$ is (PT). If $k$ is (PT), $-k$ is (CNT). It is the conditionally that is here important, otherwise, it would be the opposite of being positive.\\

Operations on kernels: (PT) and (CNT) both form a cone (stability by addition and multiplication by a positive scalar), they are closed for the topology of simple convergence and by Schur multiplication. \\

Pb: show that Schur multiplication preserves (PT).\\

\begin{prop}
Let $k: X\times X\rightarrow \R $ be a positive type function on $X$ and $f(x)=\sum_n a_n z^n$ an entire function with positive coefficients. Then
\[\sum_n a_n k(x,y)^n\]
defines a positive type function. 
\end{prop}

\begin{thm} Schoenberg's lemma. Let $k: X\times X\rightarrow \R $ be a symmetric kernel on $X$. Then $k$ is (CNT) iff 
\[\forall t> 0 , \exp (-tk(x,y))\]
is of positive type.
\end{thm}

\begin{proof}
If $e^{-tk(x,y)}$ is (PT) for every $t>0$, then 
\[k(x,y) =\lim_{t\rightarrow 0} \frac{1-e^{-tk(x,y)}}{t}\]
is (CNT).\\

If $k(x,y)$ is (CNT), then there exists a real Hilbert space $H$ and a map $\phi: X \rightarrow H$ such that 
\[k(x,y) = \| \phi(x) - \phi(y) \|^2 \quad \forall x,y \in X.\]
Then
\[e^{-tk(x,y)} = e^{-t \| \phi(x) \|^2 } e^{-t \| \phi(y) \|^2 } e^{2t \langle \phi(x), \phi(y) \rangle } \]
is a Schur product of a kernel of the type $f(x)f(y)$ with the exponential of a positive type, so it is positive. 
\end{proof}

Remark on the GNS construction.

When $k(x,y)$ is (CNT) then the separation completion of 
\[V = \{a : X\rightarrow \R \text{ finitely supported s.t. }\ : \ \sum_x a_x = 0\}\]
with respect to the positive bilinear form 
\[\langle a, b \rangle = -\frac{1}{2}\sum_{x,y} a_x b_x k(x,y) \]
gives a real Hilbert space $H$ and map $\phi: X \rightarrow H$ such that 
\[k(x,y) = \| \phi(x)-\phi(y) \|^2 \]
and $span\{\phi(x) - \phi(x_0)\}_{x\in X} $ is dense in $H$, where $x_0$ is any fixed point.\\ 

The couple $(H, \phi)$ is unique up to affine isomorphism: if $(K, \Psi)$ is another such couple, there exists a unique affine isomorphism $A: H \rightarrow K$ such that $\Psi =A\circ \phi $. 
 
\subsection{Kernels and approximation properties}

A function $\eta: X\times X \rightarrow \R$ is
\begin{itemize}
\item[$\bullet$] proper if $\eta^{-1}([-r,r])$ is an entourage for every $r>0$,
\item[$\bullet$] $\sup_{(x,y)\in \Delta_R} | \eta(x,y) | < \infty$.
\end{itemize}

\begin{thm}
Let $X$ be a countable discrete space with bounded geometry. Then the following are equivalent:
\begin{itemize}
\item[$\bullet$] X admits a coarse embedding into Hilbert space
\item[$\bullet$] there exists a proper \text{conditionally negative} type function that is bounded on entourages.
\end{itemize}
\end{thm}

First implication: $\eta(x,y) = \| \phi(x) - \phi(y)\|^2$.\\

Definitions: $G$ residually finite w.r.t. $\mathcal N = \{N_n\}$, $q_n: G \rightarrow G_n = G/N_n$, box space
\[X_{\mathcal N} = \coprod Cay(G_n, q_n(S)).\]
Haagerup's property or a-T-menability.

\begin{prop}
If $X_{\mathcal N}$ is (CEH), then $G$ is a-T-menable.
\end{prop}

\begin{proof}
By definition, we get a sequence of negative type functions 
\[\phi_n : G_n \rightarrow H\]
with 
\[\rho_-(l_n(x)) \leq \phi_n(x) \leq \rho_+(l_n(g)).\]
The composition $\phi_n\circ q_n$
\end{proof}

Remark: the converse does not hold: the free group on $n$ generators is a-T-menable, and any finitely generated group is a quotient of a $\mathbb F_n$, so we get a coarse embedding
\[X(G) \rightarrow X(\mathbb F_n)\]
so that the converse to the last proposition would imply that any box space is (CEH). But any infinite residually finite property (T) group contradicts this, $SL(3,\Z)$ for instance.\\

Ref: Chapter 5 and 6 of Nowak, Yu, Large scale geometry, and Appendix C of Bekka, Valette, Kazdhan Property (T)

\section{Lecture 4}

Definition of the Laplacian on a graph: for any Hilbert space $H$,
\[(\Delta f)(x) = f(x) - \frac{1}{N_x}\sum_{y : (x,y)\in E} f(y)\quad \forall f\in l^2(X,H).\]
On a finite graph, the Laplacian is a symmetric (or selfadjoint in the complex case) positive matrix with kernel the constant functions, and its orthogonal is 
\[ker(\Delta)^\perp = \{ f\in l^2(X) \ : \ \sum_{x\in X} f(x) = 0\}.\]

\subsection{Expanders}

\begin{definition}
An expander sequence is a sequence of finite graphs $X_n= (V_n, E_n)$ such that 
\begin{itemize}
\item[$\bullet$] $\lim_n |V_n| = +\infty $,
\item[$\bullet$] $d=\sup_n deg(X_n)< \infty$,
\item[$\bullet$] there exists a constant $c>0$ such that, for every $n$, $sp(\Delta_{X_n}) \subset \{0\} \cup (c,\infty)$.
\end{itemize}
\end{definition} 

Out of an expander sequence, we build its \textit{coarse disjoint union}
\[X = \coprod X_n\]
which is the disjoint union as a set, endowed with a metric $d$ which restricts to the graph metric on any of the graphs, and such that $\lim_{i,j\rightarrow \infty} d(X_i,X_j)=+\infty$. It is $X$ that we call an expander.\\

The last condition means that for every $f\in l^2(X,H)$ such that $\sum_{x\in X_n} f(x)=0$, the inequality
\[\sum_{(x,y)\in E_n} \| f(x)-f(y) \|^2 \geq c \sum_{x\in X_n} \| f(x)\|^2    \]
holds.

\begin{thm}
Expanders cannot admit a coarse embedding into Hilbert space.
\end{thm}

\begin{proof} 
I think this proof is due to Higson. Let $f: X \rightarrow H$ be a coarse embedding. One gets vectors $f_n\in l^2(X_n,H)$ by restricting $f$ to $X_n$. We can translate each $f_n$ without changing the fact that it is a coarse embedding, so we can suppose $\sum_{x\in X_n} f_n(x)=0$, hence
\[\sum_{(x,y)\in E_n} \| f_n(x)-f_n(y) \|^2 \geq c \sum_{x\in X_n} \| f(x_n)\|^2.\]
The term on the left is less than $d |V_n| \rho_+^2(1)$ so that we get a inequality of the type
\[\sum_{x\in X_n} \| f(x_n)\|^2 \leq A |V_n|\]
with $A$ a constant independent of $n$. That is possible only if at least half of the terms are less than $2A$. But that is a contradiction with the properness of $f$, since it means that $f_n$ maps more than $\frac{|X_n|}{2}$ terms into $B(0,2A)$.\\	 
\end{proof}

We will understand better the following diagram
\[\begin{tikzcd}
\text{amenablity} \arrow{r}\arrow{d} & \text{a-T-menability} \arrow{d} \\
\text{ property (A)} \arrow{r} & \text{ (CEH) }  
\end{tikzcd}\]
where the top row is a $G$-equivariant version of the bottom one, whereas the left side corresponds to finitely supported kernels, relaxed into a $c_0$ condition on the right. There is a parallel diagram with property (T), and we will see what should be a non-equivariant version of property (T). There is a way to restore equivariance by considering the \textit{coarse groupoid}.

\subsection{GNS construction}

We have seen that for any symmetric kernel $k: X\times X \rightarrow \R$ of positive type, there exists a pair $GNS(k)=(H,\phi)$ satisfying
\begin{itemize}
\item[$\bullet$] $H$ is a real Hilbert space,
\item[$\bullet$] $\phi : X\rightarrow H$ is a map such that $\langle \phi(x),\phi(y) \rangle = k(x,y)$ and $span \{ \phi(x)\ : \ x\in X \} $ is dense in $H$. 
\end{itemize}

Moreover, the pair $(H,\phi)$ satisfies the following universal property: let $(K,\psi)$ be any other pair such that $\langle \psi(x),\psi(y) \rangle = k(x,y)$. Then there exists a unique linear isometry $T: H\rightarrow K$ such that $\psi = T \circ \phi$.\\

Indeed, the identity 
\[\langle \sum_i a_i \delta_{x_i}, \sum_j b_j \delta_{y_j} \rangle  = \sum_{i,j} a_i b_j k(x_i,_j) =\langle \sum_i a_i \psi(x_i), \sum_j b_j \psi(y_j) \rangle\]
proves that the map $\sum_i a_i \delta_{x_i}\mapsto  \sum_i a_i \psi(x_i)$ extends to an isometry $T:H \rightarrow K $ satisfying $\psi = T \circ \phi$. Uniqueness is obvious.\\

Now, say that $X=G$ is a group, and that the kernel is $G$-equivariant, i.e.
\[k(gx,gy) = k(x,y)\quad \forall g,x,y\in G.\]
Such a function is entirely determined by 
\[\tilde k (g) = k(e,g),\]
which is then what is called a positive type function on the group $G$.\\

In this case, the GNS construction and its universal property allows to build a unitary representation. Indeed 
\[\langle \phi(gx),\phi(gy) \rangle = k(gx,gy) = k(x,y)\] 
ensures the existence of a unique linear isometry $\pi_g: H \rightarrow H$ such that $\phi(gx) = \pi_g \phi(x)$, $\forall g,x \in G$. The uniqueness also ensures that $\pi_s \pi_t = \pi_{st} $ and $\pi_e = id_H$. Together with symmetry ($\tilde k(g^{-1})=\tilde k(g)$), it implies that $\pi_g^*=\pi_{g^{-1}}$. So that, if $v= \phi(e)\in H$, we have that the GNS pair $(H,\phi)$ is actually equivalent to a triple $(H,\pi, v)$ where 
\begin{itemize}
\item[$\bullet$] $H$ is a real Hilbert space,
\item[$\bullet$] $\pi : G\rightarrow U(H)$ is a unitary representation, 
\item[$\bullet$] $v\in H$ is a vector,
\end{itemize}
such that $span \{ \pi_g (v)\}_{g\in G}$ is dense in $H$ (we say $v$ is \textit{cyclic} for $\pi$), and $\tilde k(g)=\langle v , \pi_g (v)\rangle$, for all $g\in G$.\\

The triple $(H,\pi,v)$ is called the GNS triple associated to the positive type function $\tilde k$. It satisfies the following universal property: if $(K,\sigma , w)$ is any other triple such that $\tilde k(g)=\langle w , \sigma_g (w)\rangle$, then there exists a unique linear isometry $T: H\rightarrow K $ such that $Tv= w$ and $T\pi_g = \sigma_g T$.\\

As a problem, show that the GNS triple of a cyclic representation is unitarily equivalent to the original representation, i.e. 
\[GNS(\pi_{v,v}) \cong (H,\pi,v).\]  

As a result, we get that the family of equivalence classes of cyclic representation is a set, isomorphic to the set of positive type functions on $G$.
\[\{ \pi: G\rightarrow U(H) \text{ cyclic } \} / \sim \cong \{\phi : G \rightarrow \R \text{ positive type}\}.\] 

We will see next time how we can use a natural topology on the space of bounded functions on $G$ to topologize the space of cyclic representations of $G$. (This will be equivalent to the Fell topology)

\section{Lecture 5}
Recall that cyclic unitary representations corresponds to positive type functions on the group via the GNS construction. Cauchy-Schwartz ensures that
\[|\phi(g) | = |\langle v_\phi , \pi_\phi(g)v_\phi \rangle | \leq \|v_\phi \|^2,\]
hence $\phi\in l^\infty(G)$, and $\|\phi\|_\infty = |\phi(e)| =\|v_\phi \|^2$. But $(l^\infty(G), \|\cdot\|_\infty )$ is a Banach space. 
\[\tilde G\cong \mathcal C :=\{\phi: G\rightarrow \R \text{ PT s.t. }\phi(e) = 1\}\] 
Positive type functions form a cone closed w.r.t the topology of pointwise convergence, so the weak-$*$ topology. $\mathcal C$ is a closed convex subspace, so that Krein-Milman ensures that the set of extremal points is weak-$*$ dense in $\mathcal C$.

\begin{prop}
$Ext(\mathcal C)$ corresponds to the irreducible unitary representations via the GNS-construction.
\end{prop}  
 
This is an easy consequence of how the operations on positive type functions tranlate at the level of representations.
\[GNS(t\phi)\cong (H_\phi, \pi_\phi, \sqrt{t}\cdot v_\phi)\]
\[GNS(\phi_1+\phi_2)\cong (H_{\phi_1} \oplus H_{\phi_2}, \pi_{\phi_1}\oplus \pi_{\phi_2}, v_{\phi_1}\oplus v_{\phi_2})\]
\[GNS(t\phi)\cong (H_{\phi_1} \otimes H_{\phi_2}, \pi_{\phi_1}\otimes \pi_{\phi_2}, v_{\phi_1}\otimes v_{\phi_2})\]

The set of equivalence classes of unitary representations is endowed with the Fell topology: it is the topology of uniform convergence of coefficients on compact subsets. It corresponds exactly to the weak-$*$ toplogy on the space of normalized positive type functions.

\begin{thm} [Raikov]
The weak-$*$ topology and the topology of uniform convergence on compact subsets, i.e. for every net $\phi_i$ of positive type normalized functions 
\[ \forall f \in l^1(G), \quad \lim_i\langle \phi_i , f\rangle = \langle \phi , f\rangle \iff \forall F\subset G\text{ finite}, \quad \lim_i\|\phi_i -\phi\|_{\infty , F} = 0 \] 
\end{thm} 
\begin{proof}
Let $\phi_i$ converge uniformly on every finite subsets to $\phi$. Let $f\in l^1(G)$ and $\varepsilon>0$. There exists a finite subset $F\subset G$ such that 
\[\sum_{g\in G-F} |f(g)| \leq \frac{\varepsilon}{4}\]
and a rank $n$ such that 
\[\forall i\geq n, \|\phi_i -\phi\|_{\infty , F} \leq \frac{\varepsilon }{2\sum_{F} |f(g)| }\]
and 
\[\begin{split}
|\langle \phi - \phi_i , f \rangle | & \leq \|\phi_i -\phi\|_{\infty , F} \sum_{F} |f(g)|+2\sum_{G-F}|f(g)| \\
				& \leq \varepsilon \\ 
\end{split}\]
So that $\phi_i$ converges to $\phi$ in the weak-$*$ topology.\\
 
Let $\phi_i$ converge to $\phi$ in the weak-$*$ topology. Let $F\subset G$ be finite, and $\varepsilon>0$. Define $f=\chi_F\in l^1(G)$ and, up to changing the value of $f$ to $-1$, we get that, ultimately,  
\[ |\phi(g)-\phi_i(g)|\leq \sum_{g\in F} |\phi(g)-\phi_i(g)| = \langle \phi - \phi_i , f \rangle   \leq \varepsilon \quad \forall g\in F\]
which ensures that $\|\phi_i-\phi\|_{\infty, F}\leq \varepsilon$.
\end{proof}

Problem: Show that unireps separate points, i.e. for any $s\neq t$ in $G$, there exists a unirep $\pi$ such that $\pi(s)\neq \pi(t)$. \\

Going back to a-T-menability.

\begin{definition} We say that $G$ is a-T-menable, or has Haagerup's property, if there exists a (metrically) proper action of $G$ on a real Hilbert space by affine isometries, i.e there exists a group morphism 
\[\alpha:\left\{\begin{array}{rcl} G & \rightarrow & Aff(H) \\ v & \mapsto & \alpha_g(v) = \pi_g(v)+b_g\end{array}\right.\] 
with 
\begin{itemize}
\item[$\bullet$] $\pi: G \rightarrow U(H)$ is a unitary representation for $G$,
\item[$\bullet$] $b: G\rightarrow H$ is a cocycle for $\pi$, meaning $b_{st} = \pi_s b_t+ b_s$, for every $s,t \in G$,
\item[$\bullet$] $\alpha$ metrically proper, i.e. $\lim_{l(g)\rightarrow \infty} \|b_g\|^2 =\infty$.
\end{itemize}
\end{definition}
Remarks:
\begin{itemize}
\item[$\bullet$] the cocycle relation ensures that $\alpha $ is a group morphism. In particular, it also yields 
\[ \|b_s- b_t\|^2 = \| b_{s^{-1}t} \|^2 ,\]
ensuring that $\phi(g)=\|b_g- b_e\|^2$ is a conditionally negative type function on $G$.
\item[$\bullet$] 
\end{itemize}

\begin{thm} If $G$ is a-T-menable, then $G$ admits a coarse embedding into Hilbert space.
\end{thm}
\begin{proof}
It suffices to show that $\phi(g)=\|b_g- b_e\|^2$ is proper, which is guaranted by the fact that $\lim_{l(g)\rightarrow \infty} \|b_g\|^2 =\infty$.\\
\end{proof}

An interesting consequence is the following. 
\begin{prop}
If $G$ is a-T-menable, there exists a continuous path of representations between the trivial representation and the quasi-regular representation $\lambda_{G/H}$ for some subgroup $H<G$.
\end{prop}

\begin{proof} By Schoenberg's lemma, for every $t>0$, 
\[\phi_t(g) = \exp(-t\eta(g))\]
is of positive type. Let $\Psi$ be the characteristic function of $\{ g\ : \ \eta(g) = 0\}$. Let s show that $\phi_t$ converges uniformly on finite subsets to $\Psi$ as $t$ diverges to infinity. Let $F\subset G$ be finite, then its obvious isn't it?\\

Claim: $H= \{ g\ : \ \eta(g) = 0\}$ is a subgroup of $G$. Indeed $\eta(e)=0$ and 
\[\eta(s^{-1}t) \leq (\eta(s)+\eta(t))^2. \]

\[\|a\|^2_\Psi = \sum_{x\in G/H} (\sum_{s\in xH} a_s)^2\]
so that $N = \{a \ : \ \forall x\in G, \sum_{s\in xH} a_s = 0\}$ and
\[V/N \cong span \{ \chi_{xH}\}_{x\in G} \]
hence $H_{\Psi}\cong l^2(G/H)$ and $\pi_\Psi(g)\chi_{xH}= \chi_{gxH}$, $v_\Psi = \chi_{H}$, so that $GNS(\Psi)$ is isomorphic to the quasi-regular representation on $G/H$.\\	

This gives $GNS(\phi_t)$ converges in the Fell's topology to $\lambda_{G/H}$.
\end{proof} 

References: Bekka, Valette, Property T (Annex C)\\
Nowak Yu, Chapter on isometric actins on Banach spaces\\

\section{Lecture 6}

A last remark on affine actions of groups by isometries on Hilbert spaces. In the same way that a positive type kernel is equivalent to a Hilbert space $H$ and a map $\phi : X \rightarrow H$, there is an algebraic relation for cocyles: a cocycle for an isometric representation $G\rightarrow Aff(H)$ is equivalent to a negative type function $G\rightarrow \R$. Indeed, given such a cocycle $b: G \rightarrow H$,
\[\phi(s^{-1}t) = \|b_s-b_t\|^2=\| b_{s^{-1}t}\|^2\]
defines a negative type function, and the GNS representation associated to a CNT on $G$ will give a cocyle. Moreover, if the group is of bounded geometry, there is automatically a nondecreasing function $\rho_+$ such that $\|b_s-b_t\|\leq \rho_+(d(s,t))$, since 
\[\phi(g) \leq \sup_{x\in B(e,r)} \phi(x) \quad \forall g : |g|\leq r, \]
because the balls are finite. The only missing condition for $b$ to be a coarse embedding is the bound on the left hand side, which is equivalent to being proper. Notice that the opposite of a proper cocycle is any so called coboundary 
\[b_g = v-\pi_g(v)\]
which is bounded. In that case, $\alpha_g(v)=v$ is a constant representation.\\

Then: discussion on the origin of amenability, i.e. the Banach Tarski paradox. Vitali's theorem: there exists measure define on the whole of $P(\R)$ such that $\mu([0,1]) = 1$ and which is $\sigma$-additive and invariant by translation, whence the need to introduce $\sigma$-algebras. In the other way, we can define finitely additive measures, which we can show to be equivalent to means.\\

Let $X$ be a metric space, possibly endowed with an isometric action of $G$.

\begin{definition}
A mean on $X$ is a normalized ($m(1_X)=1$) positive ($m(f)\geq 0$ when $f\geq 0$) linear functional $m: l^\infty(X)\rightarrow \C$. We say that $m$ is $G$-invariant if 
\[m(\alpha_g(f))=m(f) \quad \forall f\in l^\infty (X), g\in G.\] 
\end{definition}

A mean is automatically continuous of norm $1$, and is thus an element of the dual $l^\infty(X)^*$. The space of means is a convex that is compact for the weak-$*$ topology.

\begin{definition}
A group $G$ is amenable if any of these equivalent conditions are satisfied:
\begin{itemize}
\item[$\bullet$] there exists a $G$-invariant mean on $l^\infty(G)$,
\item[$\bullet$] there exists a finitely additive measure on $G$, invariant by $G$-translation, 
\item[$\bullet$] for all $\varepsilon>0, r>0$, there exists a finite subset $F \subset G$ such that 
\[ \forall |g|\leq r, \frac{|F\Delta gF|}{|F|}< \varepsilon\]
\item[$\bullet$] for all $\varepsilon>0, r>0$, there exists a finite subset $F \subset G$ such that
\[\frac{|\partial_r F|}{|F|}<\varepsilon\]
\end{itemize}
\end{definition}

The definition via means is powerful to show that the following are amenable.
\begin{itemize}
\item[$\bullet$] finite groups
\[m(f)=\frac{1}{|G|}\sum_{g\in G} f(g),\]
\item[$\bullet$] if $G$ is amenable, so is $G/N$ where $N$ is a normal subgroup. Define a map $l^{\infty}(G/N)\rightarrow l^\infty(G)$ by identifying functions on $G/N$ to functions on $G$ that are constant on cosets,
\item[$\bullet$] abelian groups. Dualize the action of $G$ on $l^\infty(G)$ to the convex compact $K$ of means on $G$: we get affine maps $\tilde \alpha_g : K\rightarrow K$ for each $K$. By Kakutani's theor	em, they each have at least a fixed point. Since the affine maps commutes, the fixed points are stabilized so that for every finite subset $F\subset G$,
\[\cap_F Fix(\tilde \alpha_g)\]
is nonempty. Baire's theorem ensures that $\cap_G Fix(\tilde \alpha_g)$ is non-empty, so there exists a $G$-invariant mean.
\end{itemize} 

The implication from definition by Folner sets to the one by mean also uses a compactness argument: for any Folner set, define 
\[m_F(f)= \frac{1}{|F|}\sum_{g\in F}f(g).\]
Then $\{m_F\}$ is a bounded net in $K$ so that there exists an accumulation point which is easily shown to be an invariant mean.\\

For extensions, amenability satifises the ``2 out of 3'' property. This implies, knowing that abelian groups are amenable, that solvable groups are amenable.\\

The free group is not amenable. If a group is not amenable, then there is a constant $C>0$ such that 
\[\frac{|\partial_r F|}{|F|}\geq C\]
for every finite subset $F\subset G$. In particular
\[|B(e,n)| = \prod_{k=1,n} (1 + \frac{|S(e,k)|}{|B(e,k)|}) \geq (1+C)^n.\]
 In particular, subexponential growth implies amenability. So $\Z^n$ is amenable (not using that it is abelian).
 
Next time:
Hulancki-Reiter conidition for amenability, Property A and Higson-Roe condition (analog of Reiter), and kernel characterization 
\begin{thm}
X has property (A) iff for every $r>0, \varepsilon>0$, there exist $s>0$ and a positive type kernel $\phi: X\times X\rightarrow \R$ such that 
\begin{itemize}
\item[$\bullet$] $supp \ \phi \subset \Delta_s$,
\item[$\bullet$] if $d(x,y)\leq r$, $|1-\phi(x,y)|<\varepsilon$.
\end{itemize}
\end{thm}
Version for amenability: equivariance.\\

Then: formulation in terms of Schur multipliers.

References: Choimet, Queff\'elec, Analyse math\'ematiques, les grands th\'eor\`emes du XXe si\`ecle.\\
Nowak Yu, Chapter on amenability\\

\section{Lecture 7}

Let us prove that amenability implies a-T-menability. For this we will use yet another characterization of amenablity, called the Hulanicki-Reiter condition.

\begin{thm}
$G$ is amenable iff for all $r>0,\varepsilon >0$, there exists a finitely supported function $f\in l^1(G)_{+,1}$ such that 
\[ \| f -s\cdot f\|_1 \leq \varepsilon \quad \forall |s|\leq r. \]
\end{thm}
\begin{proof}
If $G$ is amenable, let $F$ be a $(\varepsilon, r)$-Folner set. The function $f=\frac{1}{|F|}\chi_F$ satisfies the condition\\

Let $f\in l^1(G)_{+,1}$ be as above. Up to a minor approximation, we can suppose $f$ is a step function with rational values: $f(x)\in \{0, \frac{1}{N}, ... , \frac{N-1}{N}, 1\}$. Define
\[F = \{ (g,i) \ : \ i\leq f(g)\} \subset G\times \mathbb N,\]
which is a grid under the graph of $f$, so that, as $f$ is a step function, 
\[1 = \| f \|_1 = \frac{|F|}{N}\]
and if $|s|\leq r$,
\[\frac{|F\Delta sF|}{N} = \frac{|F\Delta sF|}{|F|}= \|f- s\cdot f\|_1 \leq \varepsilon\]
so that $F$ is a $(\varepsilon, r)$-Folner set.
\end{proof}

Remark: we could have taken $f$ in $l^2(G)_{+,1}$. (Problem)

\begin{cor}
If $G$ is amenable, $G$ is a-T-menable.
\end{cor}

\begin{proof}
For each $n$, choose $f_n\in l^2(G)_{+,1}$ a $(2^{-n},n)$-Hulanicki-Reiter function:
\[\|f_n-s\cdot f_n\|_2 \leq \frac{1}{2^n} \quad \forall |s|\leq n.\]
Define $\pi_g = \oplus \lambda_g$ the $l^2$-sum of the left regular representation on $\bigoplus_n l^2(G)$, and 
\[b_g = \oplus_n (f_n - \lambda_g(f_n)).\]
Then 
\begin{itemize}
\item[$\bullet$] $b_g$ is well defined in $l^2$,
\[ \|b_g\|^2 = \sum_n \|f_n - g\cdot f_n\|^2 \leq \sum_{n= 1}^{|g|}\|f_n - g\cdot f_n\|^2 + \sum_{n\geq |g|+1} \frac{1}{4^n}. \]
\item[$\bullet$] $b_g$ is a cocyle for $\pi_g$ (direct computation),
\item[$\bullet$] $b_g$ is proper. Indeed for any $n$, there is a $N_n$ such that $\|f_n -s\cdot f_n\|^2 = 2$ if $|s|\geq N_n$. We can suppose $N_n$ to be increasing, and 
\[\|b_g\|^2 \geq 2|\{n \ : \ N_n \leq |g|\}| \rightarrow +\infty.\]
\end{itemize}
\end{proof}

In his 2000 paper in Inventiones Mathematicae, Guoliang Yu introduced the follwing property.

\begin{definition}
$X$ has property (A) iff for every positive numbers $\varepsilon, r >0$, there exist $s>0$ and a family of finite subsets $\{A_x\}_{x\in X}$ such that 
\[\frac{|A_x \Delta A_y|}{|A_x\cap A_y|} \leq \varepsilon \quad \forall d(x,y)\leq r\]
and $A_x\subset B(x,s)\times \mathbb N$.
\end{definition}

Show that property (A) is a coarse invariant.

Examples:
\begin{itemize}
\item[$\bullet$] finite groups,
\item[$\bullet$] amenable groups,
\item[$\bullet$] trees.
\end{itemize}

In particular, $\mathbb F_n$ is a tree, so it has property (A), while it is not amenable.

We have a analog of the Hulanicki-Reiter condition, called the Higson-Roe condition.

\begin{thm} 
$X$ has property (A) iff for every $r,\varepsilon>0$, there exist $s>0$ and $f:X\rightarrow l^1(X)_{+,1}$ such that 
\[\|f(x)-f(y)\|_1 \leq \varepsilon \quad \forall d(x,y)\leq r\]
and $\text{supp }f\subset B(x,s)$.
\end{thm}

\begin{proof}
If $\{A_x\}$ is a family of subset as in the definition of property (A), then, if $A_x(y) = A_x \cap \{y\}\times \mathbb N$, then 
\[f(x) = \frac{1}{|A_x|}\sum_y |A_x(y)|\delta_y \in l^1(X)_{+,1}\]
satisfies the conditions.\\

If $f: X\rightarrow l^1(X)_{+,1}$ as above is given, then, as for amenability, suppose that $f$ is a step function and define 
\[A_x = \{(x,i)\ : \ i\leq f(x)\}\subset supp(f)\times \mathbb N \subset B(x,s)\times \mathbb N.\]
Then $1 = \|f(x)\|_1= \frac{|A_x|}{N}$, and 
\[\frac{|A_x\Delta A_y|}{|A_x\cap A_y|}\]  

\[ \|f(x)-f(y)\|= \frac{1}{N}\sum ||A_x(z)|-|A_y(z)||  \]  
\end{proof}

References: Nowak-Yu, Chapter on property A

\section{Lecture 8}

We finished the proof of the Higson Roe condition. We use the Higson Roe condition to show the following theorems.

\begin{thm}
If $G$ is an amenable group, it has property (A).
\end{thm}

\begin{proof}
Let $\varepsilon >0, r>0$, and pick a Hulanicki-Reiter function $f\in l^2(G)_{+,1}$, finitely supported and $(\varepsilon, r)$-equivariant. Define $\tilde f (g) = \lambda_{g}^*(f)$. This defines a $(\varepsilon, r)$-Higson-Roe function for $\tilde f : G \rightarrow l^2(G)_{+,1}$.
\end{proof}

\begin{thm}
If $X$ has property (A), it is coarsely embeddable into a Hilbert space.
\end{thm}

\begin{proof}
Let $f_n : X \rightarrow l^2(X)_{+,1}$ be $(\varepsilon_n,r_n)$-Higson-Roe functions for $X$, with $\varepsilon_n = 2^{-n}$ and $r_n = n$. There exists $s_n>0$ such that $supp f_n(x)\subset B(x,s_n)$. We can suppose that $(s_n)_n$ increases and diverges to $+\infty$. Notice that
\[\begin{split}
\sum_n \|f_n(x) - f_n(y)\|^2 & \leq \sum_{n=0}^{d(x,y)} \|f_n(x) - f_n(y)\|^2 +\sum_{n\geq d(x,y)+1} \frac{1}{4^n} \\
				& \leq \sum_{n=0}^{d(x,y)} \|f_n(x) - f_n(y)\|^2 + 2.\\
\end{split}\]
Fix $x_0\in X$. The inequality above says that  
\[F(x) = \bigoplus_n (f_n(x) - f_n(x_0))\]
is well defined as an element of $l^2(X \times \mathbb N)$. It also ensures that
\[ \|F(x)-F(y)\|^2 \leq 2 d(x,y) +2.\]
Moreover, if $d(x,y)> 2s_n$, $\|f_n(x) - f_n(y)\|^2 = 2$, so that 
\[\|F(x)-F(y)\|^2 \geq | \{ n \ : \ 2s_n \leq  d(x,y)\} = \rho_-(d(x,y)),\]
with $\lim \rho_- =\infty$, so that $F$ is a coarse embedding. 
\end{proof}

\subsection{Kernel characterization of property (A)}

Definition of $\C_s[X]$ and $C^*_u(X)$, reminder on how to take the square root of a positive operator in a $C^*$-algebra, via polynomila approximation.

\begin{thm}[Tu]
$X$ has property (A) iff for every $\varepsilon> 0, r>0$, there exists a symmetric normalized positive type kernel $\phi : X\times X \rightarrow \R$ with finite propagation such that $\|1-\phi\|_{\infty, \Delta_r}<\varepsilon$. 
\end{thm}

\begin{proof}
If $X$ has property (A), pick a Higson-Roe function $f: X\rightarrow l^2_{+,1}(X)$ and define $\phi(x,y)=\langle f(y), f(x)\rangle$. It satisfies all the coniditions of the theorem.\\

Let $\varepsilon, r>0$ and $\phi$ be a kernel as above. Let $s= prop(\phi)$. Then
\[(T\xi)(x)= \sum_{y\in X} \phi(x,y)\xi(y) \quad \forall \xi \in l^2(X)\]
defines a bounded operator. Indeed, by the Cauchy-Schwartz inequality
\[\begin{split}
\langle \eta, T\xi \rangle & \leq \sum_{x,y} |\phi(x,y)| |\xi_y| |\eta_x| \\
				& \leq \left( \sum_{x,y}|\phi(x,y)||\xi_y|^2\right)^{\frac{1}{2}}\cdot \left( \sum_{x,y} |\phi(x,y)| |\eta_x|^2 \right)^{\frac{1}{2}}\\
				& \leq N(S) \|\xi \|\|\eta\|
\end{split}\] 
hence $\|T\|\leq N(S)$. Then let $P$ be a polynomial of degre $d$ such that $\|P(T)-T^2\|\leq \varepsilon$, and define $f: X\rightarrow l^2(X)$ by  
\[f(x) = S\delta_x. \]
This is a Higson Roe function for $X$.
\end{proof}

Discussion on examples: $\mathbb F_n$ and $SL(2,\Z)$ are a-T-menable, but not amenable. Property (T) together with a-T-menability implies finite. Hyperbolic groups have finite asymptotic dimension, hence property (A). So any infinite hyperbolic property (T) group, such as $Sp_{n,1}(\Z)$, has property (A) but is not a-T-menable.  

\section{Lecture 9}

We first give a characterization of amenability in terms of positive type functions. 

\begin{prop}
$G$ is amenable iff for every $r> 0, \varepsilon>0$, there exists a finitely supported normalized positive type function $\phi : G \rightarrow \mathbb R$ such that $|1-\phi (g)|< \varepsilon , \forall |g|\leq r$.
\end{prop}

\begin{proof}
If $G$ is amenable, let $f\in l^2(G)_{+,1}$ be a $(\varepsilon,r)$-Reiter's function, and define $\phi(g) = \langle f , g\cdot f\rangle$. Then $\phi$ is a positive type function that satisfies all the properties above.\\

Any finitely supported normalized positive type function induces a finite propagation kernel, and thus a Schur multiplier 
\[S_\phi : C^*_u(X) \rightarrow C^*_u(X).\]
Pick a net of positive type functions as above for $\varepsilon = 2^{-n}$ and $r=n$. Then the restriction of $S_{\phi_n}$ to $l^\infty(G)$ gives a sequence from which we can extract a convergent subsequence to a mean. (details) 
\end{proof}

\begin{thm}
Let $G$ be a finitely generated residually finite group. Then $G$ is amenable iff any box space $X_{\mathcal N}(G)$ has property (A).
\end{thm}

\begin{proof}
A normalized finite propagation kernel on a coarse disjoint union $\coprod X_n$ has to ultimately stabilize the $X_n$, hence we get a sequence of positive type kernels
\[k_n : X_n \times X_n \rightarrow \R \quad  \forall n \geq N.\]
Define 
\[\phi_n (g) =\frac{1}{|G_n|}\sum_{t\in G_n} k_n(t,tq_n(g)) \quad \forall g \in G.\]
Then $\phi_n: G\rightarrow \R$ is a normalized positive type function on $G$. Take a converging subsequence. The key property to show that the limit is finitely supported and is close to $1$ is that there exists a rank such that the canonical projection $q_n : G\rightarrow G_n$ induces an isometry between $B_G(e,s)$ and $B_{G_n}(\hat e,s)$.  
\end{proof}

A property called \textit{fibred coarse embedding into Hilbert space} was defined by Chen, Wang and Yu in 2012 in their investigation of the maximal coarse Baum-Connes conjecture.It turns out that it is exactly the notion that was missing to characterize a-T-menability of the group w.r.t. the box space.

\begin{thm}[Chen, Wang and Wang]
Let $G$ be a finitely generated residually finite group. Then $G$ is a-T-menable iff any box space $X_{\mathcal N}(G)$ admits a fibred coarse embedding into Hilbert space.
\end{thm}  

Then: definition of almost invariant vector and property (T). Examples: finite groups have property (T), and amenable groups which have (T) are finite. (Reiter's conidtion shows that the left regular representation has almost invariant vectors).

\begin{prop}
Let $G$ be a discrete group with Haagerup's property. If $G$ has property (T). then $G$ is finite.
\end{prop}

\section{Lecture 10}
Positive type and conditionally negative type functions on groups: the $G$-equivariance gives a unitary (or orthogonal representation), i.e.
\[\phi: G\rightarrow H \iff (H,\pi,v) \text{ GNS triple}.\] 
GNS construction $\phi(g) = \langle v,\pi(g)v\rangle$, i.e. $\phi(s^{-1}t) = \langle \pi(s)v,\pi(t)v\rangle$ which is positive. $\eta(g)= \|\pi(g)v -v\|$. \\

$G$ is a-T-menable (metrically proper action by affine isometries on a real Hilbert space) iff there exists a $G$-equivariant proper negative type function on $G$.\\

Consequence: there is a homotopy between the trivial representation and the regular representation.\\

Gelfand Raikov theorem\\

Do a table with the different definitions and their formulation: historical, kernels and positive functions, Schur multipliers, geometrical, representation theoretical.

\begin{table}[h]
\begin{tabular}{c|c|c|c|c|}
\cline{2-5}
                                                         & \textbf{Amenability} & \textbf{Haagerup} & \textbf{(A)}            & \textbf{(CEH)}  \\ \hline
\multicolumn{1}{|l|}{\textbf{Geometrical}}                   &  $\forall \varepsilon, r>0, \exists F\subset G \text{ finite s.t. 	}\frac{|F\Delta s F|}{|F|}\leq \varepsilon$  &                   & $\forall \varepsilon, r>0, \exists \{A_x\}_{x\in X}, A_x\subset X\times \mathbb N \text{ finite s.t. 	}\frac{|A_x\Delta A_y|}{|A_x\cap A_y|}\leq \varepsilon \forall (x,y)\in \Delta_r$ and                 &                       \\ \hline
\multicolumn{1}{|l|}{\textbf{Kernels}}                   &                     &                   &                      &                       \\ \hline
\multicolumn{1}{|l|}{\textbf{Reiter's type}}                   &                     &                   &                      &                       \\ \hline
\hline
\end{tabular}
\end{table} 

\begin{table}[h]
\begin{tabular}{c|c|c|}
%\cline{2-5}
                                                         & \textbf{(A)}            & \textbf{(CEH)}  \\ \hline
\multicolumn{1}{|l|}{\textbf{Geometrical}}               & $\forall \varepsilon, r>0, \exists \{A_x\}_{x\in X}, A_x\subset X\times \mathbb N \text{ finite s.t. }\frac{|A_x\Delta A_y|}{|A_x\cap A_y|}\leq \varepsilon \forall (x,y)\in \Delta_r$ and &                       \\ \hline
\multicolumn{1}{|l|}{\textbf{Kernels}}                   &                      &                       \\ \hline
\multicolumn{1}{|l|}{\textbf{Reiter's type}}             &                      &                       \\ \hline
\hline
\end{tabular}
\end{table} 
