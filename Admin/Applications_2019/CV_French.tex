\documentclass[a4paper,11pt]{article} 
\usepackage[frenchb]{babel}
%\usepackage[T1]{fontenc} 
\usepackage[utf8]{inputenc}      
\usepackage{url}                


\pagestyle{empty}             
\usepackage{vmargin}           
\setmarginsrb{3cm}{3cm}{3cm}{3cm}{0cm}{0cm}{0cm}{0cm}

% Marge gauche, haute, droite, basse; espace entre la marge et le texte à
% gauche, en  haut, à droite, en bas

% Pour laisser de l'espace entre les lignes du tableau
\newcommand\espace{\vrule height 20pt width 0pt}

% Pour mes grands titres
\newcommand{\titre}[1]{%
	\begin{center}
	\bigskip
	\rule{\textwidth}{1pt}
	\par\vspace{0.1cm}
        \textbf{\large #1}
	\par\rule{\textwidth}{1pt}
	\end{center}
	\bigskip
	}

\begin{document}

\begin{flushleft}
Clément Dell'Aiera \\
Department of Mathematics, University of Hawaii\\
2565 McCarthy Mall, Keller 401A \\
Honolulu HI 96\ 822 \\

\medskip
%Tél.: 06 74 62 99 52

E-mail: dellaiera.clement@gmail.com


\end{flushleft}
\begin{flushleft}
Nationalit\'e : Fran\c{c}aise \\
Date de naissance : 22/03/1990 in Metz (Moselle).
\end{flushleft}

\vspace{1.5cm}
\begin{center}
\par\huge{\textbf{Curriculum Vit\ae} }
\end{center}

%%%%%%%%%%%%%%%%%%
\titre{\'Education}
%#############%%%%

\begin{tabular}{cp{0.8\textwidth}}

\textbf{August 2017--pr\'esent} &  \textbf{Assistant Professor} at U.H. Manoa (non tenure track) \\
						& Dept. of Mathematics\\
						& Rank I3-M09. \\ 
\espace
\textbf{2014-2017} &  \textbf{Doctorat} sous la supervision de Hervé  Oyono-Oyono et Andrzej Zuk\\
						& Titre \textit{``Controlled K-theory for groupoids and applications"}. \\
\espace
\textbf{2010--2014} &  \textbf{ENS Cachan} (Antenne de Bretagne) \\
				    & 	\'Ecole Normale Supérieure, D\'epartement de Math\'ematiques \\
                              & \textbf{ENSAE Paristech}\\
				   & Paris Graduate School of Economics, Statistics and Finance\\
                                   & \textbf{Master Recherche en Math\'ematiques Fondamentales}\\  & University Paris~VII-Diderot. \\
                                   & \textbf{Agrégation de Mathématiques} \\
				& 2013, rang: $41^e$. \\
\espace

\espace
\textbf{2007--2010} &\textbf{Classes préparatoires MP$^*$ } \\
					& Nancy, Lycée Henri Poincaré\\

\espace
\textbf{2007} & \textbf{Baccalauréat} (série S, option math\'ematiques) 
 \\

\end{tabular}

\newpage
%%%%%%%%%%%%%%%%%%%%%%%%%%
\titre{Research}
%#########################

\textbf{Publication list:} 
\begin{enumerate}
\item \textit{Topological property T for groupoids}, with Rufus Willett. Preprint on arxiv, november 2018. Submitted for publication.
\item \textit{Going-Down functors and the Künneth-formula for crossed products by ample groupoids}, with Christian Bönicke. Transactions of the American Mathematical Society (2019).
\item \textit{A K\"{u}nneth formula for \'etale groupoids}, preprint available on personal webpage.
\item \textit{Controlled $K$-theory for groupoids and applications to Coarse Geometry}, Journal of Functional Analysis 275.7 (2018): 1756-1807.
\end{enumerate}
\espace


\textbf{Research talks :}\\

\begin{itemize}
\item[\textbf{Year 2019}]
\item[$\bullet$] September 4, \textit{Exhaustive representations of Roe algebras}, Noncommutative Geometry Seminar, University of Houston.
\item[$\bullet$] September 2, \textit{Exhaustive representations of Roe algebras}, Noncommutative Geometry Seminar, Texas A\&M. 
\item[$\bullet$] July 21-August 10: visit to the Shanghai Center for Mathematical Sciences with Guoliang Yu, talk July 31 at Fudan University: \textit{Groupoids and approximation properties of $C^*$-algebras}. 
\item[$\bullet$] May 29, ``Restriction principle and the Künneth formula'', GPOTS 2019, Texas A\&M.
\item[$\bullet$] May 2, ``Restriction principle and the Künneth formula'', Noncommutative Geometry Festival, ``NCG and Representation theory'', Washington University, Saint-Louis.
\item[$\bullet$] February 13, Noncommutative Geometry Seminar, Texas A\&M.
\end{itemize}
%\espace

\begin{itemize}
\item[\textbf{Year 2018}]
\item[$\bullet$] November 7, ``Dynamical Property T'', Noncommutative Geometry Seminar, Texas A\&M.
\item[$\bullet$] November 5, ``Dynamical Property T'', Noncommutative Geometry Seminar, University of Houston.
\item[$\bullet$] September 20, ``Dynamical Property T'', Noncommutative geometry Seminar, PennState University.
\item[$\bullet$] June 2018, ``C*-alg\`ebres g\'eom\'etriques et applications en g\'eom\'etrie coarse", S\'eminaire d'AO, Paris Diderot.
\item[$\bullet$] May 2018, ``Geometric C*-algebras: applications to the K\"unneth formula", GPOTS 2018, Miami University.
\item[$\bullet$] February 2018, ``Geometric C*-algebras and Coarse structures", Workshop on computability of K-theory for C*-algebras, Texas A\&M University.
\end{itemize}

\begin{itemize}
\item[\textbf{Year 2017}]
\item[$\bullet$] June, ``Principe de restriction pour les groupoïdes étales. Application à une formule de Künneth pour leurs produits croisés.", Séminaire d'Algèbres d'Opérateurs, Paris-Diderot (Paris 7).
\end{itemize}
\newpage
\begin{itemize}
\item[\textbf{Year 2016}]
\item[$\bullet$] December, ``Controlled K-theory for groupoids. Applications to Coarse Geometry", Kleines Seminar, Münster.
\item[$\bullet$] December, ``K-théorie quantitative et applications", Arbre de Noël du GDR Géométrie Non-commutative, Albi.
\item[$\bullet$] May, ``Asymptotic dimension for étale groupoids", Noncommutative Geometry Seminar, IECL, Metz-Nancy
\end{itemize}

\begin{itemize}
\item[\textbf{Year 2015}]
\item[$\bullet$] 2 D\'ecembre 2015, ``Controlled $K$-theory for groupoids", Arbre de Noël du GDR Géométrie Non-Commutative, Montpellier
\end{itemize}
\espace

\textbf{Talks directed to non-specialists :}\\

\begin{itemize}
\item[$\bullet$] February 2018, ``From a notion of dynamical dimension to cutting and pasting algebras", Analysis Seminar, University of Hawai'i.
%\item[$\bullet$] December 2017, "Expanders", Young Researchers' Day, IECL, Metz- Nancy
\item[$\bullet$] October 2016, ``Expanders", Landau Seminar, IRMAR, Rennes %: From Network reliability to K-theory", Landau Seminar, IRMAR, Rennes
\item[$\bullet$] February 2016, ``Applications of groupoids to Physics", Young Researchers Seminar, IECL, Metz-Nancy
\item[$\bullet$] October 2015, ``The Novikov conjecture for groups with finite asymptotic dimension", Henri Lebesgue Workshop, Nantes
\item[$\bullet$] October 2015, ``Propagation in $K$-theory", Young Researchers Seminar, IECL, Metz-Nancy
\item[$\bullet$] January 2015, ``Introduction to groupoids $C^*$-algebras", Young Researchers Seminar, IECL, Metz-Nancy
\end{itemize}
\espace

\textbf{Other activities:} 
\begin{itemize}
\item[$\bullet$] Fall 2019: co-organizer of the conference \textit{$K$-theory and $C^*$-algebras}, University of Hawai'i at M\={a}noa, December 2-6 2019. 
\item[$\bullet$] 2017-present : I currently co-organize, with Erik Guentner and Rufus Willett, the Noncommutative Geometry Seminar of the department of Mathematics of the University of Hawai'i at M\={a}noa. The list of talks is available on my personal webpage.
\item[$\bullet$] 2018-present: participant to the Geometric Group Theory seminar, University of Hawai'i at M\={a}noa, organized by Asaf Hadari and Andrew Sales.
\item[$\bullet$] 2016-2017 : Co-organizer with Matthieu Brachet of the Young Researchers Seminar, IECL, Metz-Nancy and Représentant du personnel au conseil du Laboratoire de l'IECL, Collège C.
\end{itemize}

\newpage

%%%%%%%%%%%%%%%%%%%%%%%%%%%%
\titre{Teaching Experience}
%%%%%%%%%%%%%%%%%%%%%%%%%%%%

\begin{itemize}

\item[$\bullet$] \textbf{ Year 2020 :} University of Hawaii at Manoa.\\
				Spring, Topics class for graduate students.\\
				Math 244: Calculus IV\\
\item[$\bullet$] \textbf{ Year 2019 :} University of Hawaii at Manoa.\\
					Fall, Math 244: Calculus IV.\\
					Spring, Math 307: Linear Algebra and Differential Equations.\\
					\textit{Duties : Lecturing, writing up Worksheets, Exams, and Finals. Grading.}\\

\item[$\bullet$] \textbf{ Year 2018 :} University of Hawaii at Manoa.\\
					Fall, Math 307: Linear Algebra and Differential Equations.\\
					Spring, Math 307: Linear Algebra and Differential Equations.\\
					\textit{Duties : Lecturing, writing up Worksheets, Exams, and Finals. Grading.}\\

\item[$\bullet$] \textbf{ Year 2017 :} University of Hawaii at Manoa.\\
					Fall, Math 203: Calculus for Business and the Social Sciences.\\
					\textit{Duties : Lecturing, writing up Worksheets, Exams, and Finals. Grading.}\\

\item[$\bullet$] \textbf{ Year 2016-2017 :} University of Lorraine.\\
					Teaching Assistant, Master 2. Preparatory Class for the Agrégation.\\
					\textit{Duties : Preparatory Class in Analysis, Computer sessions for ``Statistics and Probability".}\\
					Teaching Assistant, Master 1.\\
					\textit{Duties : Exercices sessions for ``Stochastic Processes and Probability", Computer sessions for ``Statistics and Time Series".}\\
					Teaching Assistant, L2 et L3 Mathematics. \\
					\textit{Duties : Computer sessions for the class ``Symbolic Computation".}\\   
\item[$\bullet$] \textbf{ Year 2015 :} University of Lorraine.\\
					Teaching Assistant, Master 1.\\
					\textit{Duties : Exercices sessions for ``Stochastic Processes and Probability", Computer sessions for ``Statistics and Time Series".}\\
					Teaching Assistant, L2 et L3 Mathematics. \\
					\textit{Duties : Exercices sessions in Analysis and in Algebra. Computer sessions for the class ``Symbolic Computation". Tutor for an undergraduate student working under the supervision of Pr. J-L. Tu.}\\   
					
\item[$\bullet$] \textbf{ Year 2014 :}  University of Lorraine.\\
					Teaching Assistant, Master 1. \\
					\textit{Duties : Exercices sessions for ``Stochastic Processes and Probability", Computer sessions for ``Statistics and Time Series".}\\
					Teaching Assistant, L2 et L3 Mathematics. \\
					\textit{Duties : Exercices sessions in Analysis and in Algebra. Introduction to LateX.}\\
\item[$\bullet$] \textbf{ Year 2013 :} Lycée Sainte Marie de Neuilly, Teaching Assistant in Mathematics. \\
					\textit{Duties : Oral examinations of the ``Classes Préparatoires" BL} \\
\item[$\bullet$] \textbf{ Year 2011 :} University Paris~II Assas.\\
					Teaching Assistant in Data Analysis, Licence 3 Eco-gestion.\\
					\textit{Duties : Exercices Sessions.}\\
					Lycée Sainte Marie de Neuilly, Teaching Assistant in Mathematics. \\
					\textit{Duties : Oral examinations of the ``Classes Préparatoires" BL} 
\end{itemize}

%%%%%%%%%%%%%%%%%%%%%%%%%%
\titre{Work Experience}
%###################%%%%%%

\begin{itemize}
\medskip
\item[$\bullet$] \textbf{Internships}: \\

During my education at ENSAE Paristech : \\

\begin{tabular}{cp{0.8\textwidth}}
\textbf{2014} & Master's thesis for the ENSAE under the supervision of Pr. Jérémy Jakubowicz, (Information geometry and deep learning), on applying deep learning and information geometry methods to image processing by neural networks.\\
\textbf{2012}&  $10$-weeks research internship under the supervision of Pr. Cristina Butucea on the theme of Statistics applied to Quantum Optics at University of Marne-la-Vallée (LAMA).		\\
\textbf{2011} & $2$-months mission in Bolivia with the humanitarian association Mission Potosi.\\
\end{tabular}
\\

\textbf{Master's thesis} under the supervision of Pr. Hervé Oyono-Oyono, Six term exact sequences in $K$-theory for crossed products of $C^*$-algebras by $Z$.

\medskip

\item[$\bullet$] \textbf{Professional Experience}\\ 2010--2012 : Worked for SANEF (Motorway tollbooth) in Saint-Avold.\\

\medskip
\end{itemize}

\newpage

%%%%%%%%%%%%%%%%%%%%%%%%%%%%%%%%%%%%%%
\titre{Languages and computer skills}
%#########################%%%%%%%%%%%%

\begin{itemize} 
\medskip
\item[$\bullet$] \textbf{English} Fluent. TOEIC 915/990.
\medskip
\item[$\bullet$] \textbf{Spanish} Oral.
\medskip
\item[$\bullet$] \textbf{Softwares} Pack Office, LateX, Python, Scilab, R, Objective Caml, C/C++, html.
\end{itemize}

\end{document}
