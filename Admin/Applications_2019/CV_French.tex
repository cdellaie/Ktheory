\documentclass[a4paper,11pt]{article} 
\usepackage[frenchb]{babel}
%\usepackage[T1]{fontenc} 
\usepackage[utf8]{inputenc}      
\usepackage{url}                


\pagestyle{empty}             
\usepackage{vmargin}           
\setmarginsrb{3cm}{3cm}{3cm}{3cm}{0cm}{0cm}{0cm}{0cm}

% Marge gauche, haute, droite, basse; espace entre la marge et le texte à
% gauche, en  haut, à droite, en bas

% Pour laisser de l'espace entre les lignes du tableau
\newcommand\espace{\vrule height 20pt width 0pt}

% Pour mes grands titres
\newcommand{\titre}[1]{%
	\begin{center}
	\bigskip
	\rule{\textwidth}{1pt}
	\par\vspace{0.1cm}
        \textbf{\large #1}
	\par\rule{\textwidth}{1pt}
	\end{center}
	\bigskip
	}

\begin{document}

\begin{flushleft}
Clément Dell'Aiera \\
Department of Mathematics, University of Hawaii\\
2565 McCarthy Mall, Keller 401A \\
Honolulu HI 96\ 822 \\

\medskip
%Tél.: 06 74 62 99 52

E-mail: dellaiera.clement@gmail.com


\end{flushleft}
\begin{flushleft}
Nationalit\'e : Fran\c{c}aise \\
Date de naissance : 22/03/1990 \`a Metz (Moselle).
\end{flushleft}

\vspace{1.5cm}
\begin{center}
\par\huge{\textbf{Curriculum Vit\ae} }
\end{center}

%%%%%%%%%%%%%%%%%%
\titre{\'Education}
%#############%%%%

\begin{tabular}{cp{0.8\textwidth}}

\textbf{Ao\^{u}t 2017--pr\'esent} &  \textbf{Temporary Assistant Professor} \\	
					& University of Hawaii at Manoa \\
						& Department of Mathematics\\
						%& Rank I3-M09. \\ 
\espace
\textbf{2014-2017} &  \textbf{Doctorat} \\
			& Universit\'e de Lorraine \\
			& Superviseurs: Hervé Oyono-Oyono et Andrzej Zuk.\\ 
						& Titre: \textit{``Controlled K-theory for groupoids and applications"}. \\
\espace
\textbf{2010--2014} &  \textbf{ENS Cachan} (Antenne de Bretagne) \\
				    & 	\'Ecole Normale Supérieure, D\'epartement de Math\'ematiques \\
                              & \textbf{ENSAE Paristech}\\
				   & Paris Graduate School of Economics, Statistics and Finance\\
                                   & \textbf{Master Recherche en Math\'ematiques Fondamentales}\\  & Universit\'e Paris~VII-Diderot. \\
                                   & \textbf{Agrégation de Mathématiques} \\
				& 2013, rang: $41^e$. \\
\espace

\espace
\textbf{2007--2010} &\textbf{Classes préparatoires MP$^*$ } \\
					& Nancy, Lycée Henri Poincaré\\

\espace
\textbf{2007} & \textbf{Baccalauréat} (série S, option math\'ematiques) 
 \\

\end{tabular}

\newpage
%%%%%%%%%%%%%%%%%%%%%%%%%%
\titre{Recherche}
%#########################

\textbf{Publications:} 
\begin{enumerate}
\item \textit{Topological property T for groupoids}, avec Rufus Willett. Preprint sur arxiv, novembre 2018. Soumis pour publication.
\item \textit{Going-Down functors and the Künneth-formula for crossed products by ample groupoids}, avec Christian Bönicke. Transactions of the American Mathematical Society (2019).
\item \textit{A K\"{u}nneth formula for \'etale groupoids}, preprint sur ma page personnelle.
\item \textit{Controlled $K$-theory for groupoids and applications to Coarse Geometry}, Journal of Functional Analysis 275.7 (2018): 1756-1807.
\end{enumerate}
\espace


\textbf{Expos\'es de mes travaux de recherche:}\\

\begin{itemize}
\item[\textbf{2019}]
\item[$\bullet$] 4 Septembre, \textit{Exhaustive representations of Roe algebras}, Noncommutative Geometry Seminar, University of Houston.
\item[$\bullet$] 2 Septembre, \textit{Exhaustive representations of Roe algebras}, Noncommutative Geometry Seminar, Texas A\&M. 
\item[$\bullet$] 21 Juillet-10 Ao\^{u}t: visiteur au Shanghai Center for Mathematical Sciences avec Guoliang Yu, expos\'e le 30 juillet \`a Fudan University: \textit{Groupoids and approximation properties of $C^*$-algebras}. 
\item[$\bullet$] 29 Mai, ``Restriction principle and the Künneth formula'', GPOTS 2019, Texas A\&M.
\item[$\bullet$] 2 Mai, ``Restriction principle and the Künneth formula'', Noncommutative Geometry Festival, ``NCG and Representation theory'', Washington University, Saint-Louis.
\item[$\bullet$] 13 F\'evrier, Noncommutative Geometry Seminar, Texas A\&M.
\end{itemize}
%\espace

\begin{itemize}
\item[\textbf{2018}]
\item[$\bullet$] 7 Novembre, ``Dynamical Property T'', Noncommutative Geometry Seminar, Texas A\&M.
\item[$\bullet$] 5 Novembre, ``Dynamical Property T'', Noncommutative Geometry Seminar, University of Houston.
\item[$\bullet$] 20 Septembre, ``Dynamical Property T'', Noncommutative geometry Seminar, PennState University.
\item[$\bullet$] Juin, ``C*-alg\`ebres g\'eom\'etriques et applications en g\'eom\'etrie coarse", Séminaire d'Algèbres d'Opérateurs, Paris Diderot.
\item[$\bullet$] Mai, ``Geometric C*-algebras: applications to the K\"unneth formula", GPOTS 2018, Miami University.
\item[$\bullet$] F\'evrier, ``Geometric C*-algebras and Coarse structures", Workshop on computability of K-theory for C*-algebras, Texas A\&M University.
\end{itemize}

\begin{itemize}
\item[\textbf{2017}]
\item[$\bullet$] Juin, ``Principe de restriction pour les groupoïdes étales. Application à une formule de Künneth pour leurs produits croisés.", Séminaire d'Algèbres d'Opérateurs, Paris-Diderot (Paris 7).
\end{itemize}
\newpage
\begin{itemize}
\item[\textbf{2016}]
\item[$\bullet$] D\'ecembre, ``Controlled K-theory for groupoids. Applications to Coarse Geometry", Kleines Seminar, Münster.
\item[$\bullet$] D\'ecembre, ``K-théorie quantitative et applications", Arbre de Noël du GDR Géométrie Non-commutative, Albi.
\item[$\bullet$] Mai, ``Asymptotic dimension for étale groupoids", Noncommutative Geometry Seminar, IECL, Metz-Nancy
\end{itemize}

\begin{itemize}
\item[\textbf{2015}]
\item[$\bullet$] 2 D\'ecembre 2015, ``Controlled $K$-theory for groupoids", Arbre de Noël du GDR Géométrie Non-Commutative, Montpellier
\end{itemize}
\espace

\textbf{Expos\'es \`a destination de non-sp\'ecialistes:}\\

\begin{itemize}
\item[$\bullet$] F\'evrier 2018, ``From a notion of dynamical dimension to cutting and pasting algebras", Analysis Seminar, University of Hawai'i.
%\item[$\bullet$] December 2017, "Expanders", Young Researchers' Day, IECL, Metz- Nancy
\item[$\bullet$] Octobre 2016, ``Expanders", Landau Seminar, IRMAR, Rennes %: From Network reliability to K-theory", Landau Seminar, IRMAR, Rennes
\item[$\bullet$] F\'evrier 2016, ``Applications of groupoids to Physics", Young Researchers Seminar, IECL, Metz-Nancy
\item[$\bullet$] Octobre 2015, ``The Novikov conjecture for groups with finite asymptotic dimension", Henri Lebesgue Workshop, Nantes
\item[$\bullet$] Octobre 2015, ``Propagation in $K$-theory", Young Researchers Seminar, IECL, Metz-Nancy
\item[$\bullet$] Janvier 2015, ``Introduction to groupoids $C^*$-algebras", Young Researchers Seminar, IECL, Metz-Nancy
\end{itemize}
\espace

\textbf{Autres activit\'es:} 
\begin{itemize}
\item[$\bullet$] Fall 2019: co-organisation de la conf\'erence \textit{$K$-theory and $C^*$-algebras}, \`a University of Hawai'i at M\={a}noa, 2-6 D\'ecembre 2019. 
\item[$\bullet$] 2017-pr\'esent : co-organisation, avec Erik Guentner et Rufus Willett, du Noncommutative Geometry Seminar du d\'epartement de Math\'ematiques de University of Hawai'i at M\={a}noa. La liste des expos\'es est disponible sur ma page personnelle.
\item[$\bullet$] 2018-pr\'esent: participant au s\'eminaire Geometric Group Theory, University of Hawai'i at M\={a}noa, organis\'e par Asaf Hadari et Andrew Sales.
\item[$\bullet$] 2016-2017 : Co-organisation avec Matthieu Brachet du s\'eminaire Jeunes Chercheurs, IECL, Metz-Nancy et Représentant du personnel au conseil du Laboratoire de l'IECL, Collège C.
\end{itemize}

\newpage

%%%%%%%%%%%%%%%%%%%%%%%%%%%%
\titre{Enseignements}
%%%%%%%%%%%%%%%%%%%%%%%%%%%%

\begin{itemize}

\item[$\bullet$] \textbf{2020:} University of Hawaii at Manoa.\\
				Spring, Topics class for graduate students: \textit{Large Scale Geometry and Operator Algebras}\\
				Spring, Math 244: Calculus IV\\
\item[$\bullet$] \textbf{2019:} University of Hawaii at Manoa.\\
					Fall, Math 244: Calculus IV.\\
					Spring, Math 307: Linear Algebra and Differential Equations.\\
					\textit{Duties : Lecturing, writing up Worksheets, Exams, and Finals. Grading.}\\

\item[$\bullet$] \textbf{2018:} University of Hawaii at Manoa.\\
					Fall, Math 307: Linear Algebra and Differential Equations.\\
					Spring, Math 307: Linear Algebra and Differential Equations.\\
					\textit{Duties : Lecturing, writing up Worksheets, Exams, and Finals. Grading.}\\

\item[$\bullet$] \textbf{2017:} University of Hawaii at Manoa.\\
					Fall, Math 203: Calculus for Business and the Social Sciences.\\
					\textit{Duties : Lecturing, writing up Worksheets, Exams, and Finals. Grading.}\\

\item[$\bullet$] \textbf{2016-2017:} Universit\'e of Lorraine.\\
					M2, le\c{c}ons d'Analyse pour l'Agrégation, TP de Mod\'elisation Statistiques et Probabilit\'es.\\
					M1,	charg\'e de TD pour le cours Processus Statistiques et Probabilit\'e, TP pour Statistiques et S\'eries Temporelles. \\
					L2 et L3: TD et kh\^{o}lles. \\  
\item[$\bullet$] \textbf{2015:} Universit\'e of Lorraine.\\
					M1, Charg\'e de TD pour Processus Stochastiques et Probabilit\'e, TP pour Statistiques et S\'eries Temporelles.\\
					L2 et L3: Charg\'e de TD d'alg\`ebre et d'analyse. TP pour Calcul Formel.\\ 
					Aide \`a l'encadrement d'un \'etudiant de L3 pour un stage sous la supervision J-L. Tu.\\
					
\item[$\bullet$] \textbf{2014:}  Universit\'e of Lorraine.\\
					M1: Charg\'e de TD pour Processus Stochastiques et Probabilit\'e, TP pour Statistiques et S\'eries Temporelles.\\ 
					L2 et L3: Charg\'e de TD d'alg\`ebre et d'analyse. Introduction \`a LateX.\\

\item[$\bullet$] \textbf{2013:} Lycée Sainte Marie de Neuilly\\ 
					K\^{o}lles de math\'ematiques en classes préparatoires BL. \\
\item[$\bullet$] \textbf{2011:} Universit\'e Paris~II Assas.\\
					Charg\'e de TD en Analyse de Donn\'ees, Licence 3 \'Eco-gestion.\\
					Lycée Sainte Marie de Neuilly \\
					K\^{o}lles de math\'ematiques en classes préparatoires BL. \\
\end{itemize}

%%%%%%%%%%%%%%%%%%%%%%%%%%
\titre{Stages et m\'emoires}
%###################%%%%%%

\begin{itemize}
\medskip
\item[$\bullet$] Pendant ma scolarit\'e \`a l'ENSAE Paristech : \\

\begin{tabular}{cp{0.8\textwidth}}
\textbf{2014} & M\'emoire de fin d'\'etude pour l'ENSAE sous la supervision de Jérémy Jakubowicz, (Information geometry and deep learning). Applications du deep learning et de la g\'eom\'etrie de l'information \`a la reconnaissance de caract\`eres manuscrits par des r\'eseaux de neurones.\\
\textbf{2012}&  Stage de recherche de $10$ semaines sous la supervision de Cristina Butucea sur le th\`eme \textit{Statistiques appliqu\'es \`a l'optique quantique} \`a l'Universit\'e de Marne-la-Vallée (LAMA).		\\
\textbf{2011} & Stage humanitaire de $2$ mois en Bolivie avec l'association Mission Potosi.\\
\end{tabular}
\\
\espace
\item[$\bullet$] Master de math\'ematiques fondamentales: \\

\begin{tabular}{cp{0.8\textwidth}}
\textbf{2014} & M\'emoire de fin d'\'etude sous la supervision de Hervé Oyono-Oyono, Suite exacte \`a six termes en $K$-th\'eorie pour les produits crois\'es de $C^*$-alg\`ebres par $Z$.\\
\end{tabular}
\\
\medskip
\end{itemize}

%%%%%%%%%%%%%%%%%%%%%%%%%%%%%%%%%%%%%%
\titre{Langues et informatique}
%#########################%%%%%%%%%%%%

\begin{itemize} 
\medskip
\item[$\bullet$] \textbf{English} Courant. TOEIC 915/990.
\medskip
\item[$\bullet$] \textbf{Spanish} Oral.
\medskip
\item[$\bullet$] \textbf{Softwares} Pack Office, LateX, Python, Scilab, R, Objective Caml, C/C++, html.
\end{itemize}

\end{document}
