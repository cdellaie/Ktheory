\section{Groupoids and approximation properties}

The following projects revolve around the idea of building $C^*$-algebras with exotic properties using topological groupoids and $*$-representations of their convolution algebra.

\subsection*{Property T}

\textit{with Rufus Willett, University of Hawai'i at Manoa}\\

In \cite{DellWillett}, the author and Willett introduced a topological version of Property (T) for topological groupoids. This definition generalizes that of groups, but also recover \textit{Geometric Property (T)}, defined by Willett and Yu in \cite{WillettYu} for metric spaces with bounded geometry. It also is related to property $( \tau )$, a representation theoretic property for residually finite groups, giving a criterion to build expanders, and to \textit{measurable Property T} defined by Anantharaman-Delaroche in\cite{anantharamanT}. These connections provide evidence that our definition at least make sense, but its real interest lies in the following results.
\begin{itemize}
\item[$\bullet$] In the \'etale case, property (T) can be characterized as a spectral gap condition for operators analogous to the Laplacian. 
\item[$\bullet$] Property (T), together with some technical assumptions, lead to \textit{failure of inner-exactness}. We show that this result holds for the coarse groupoid associated to an expander, and to the HLS groupoid associated to a residually finite group with property $(\tau )$. 
\end{itemize}
In ongoing work, we try to apply topological property (T) to failure of K-exactness. For instance, it is shown by Spakula in \cite{SpakulaBoxSpace} that residually finite groups satisfying a strengthening of property $(\tau)$ have their uniform Roe algebra, which is a groupoid reduced $C^*$-algebra in disguise, are not $K$-exact. This suggest that a stronger form of topological property (T) would lead to failure of $K$-exactness for the reduced $C^*$-algebra of our groupoid. We can aleady show this for uniform Roe algebras of some expanders. 

\subsection*{Exhaustivity of representations}

\textit{with Yu Qiao}\\

Motivated by questions coming from spectral theory of N-body Hamiltonians, and characterization of Fredholm integral operators and Toeplitz operators, Nistor and Prudhon introduced in \cite{NistorPrudhon} the concept of \textit{exhaustive families of representations}. A family of $*$-representations of a $C^*$-algebra $A$ is \textit{exhaustive} if every irreducible $*$-representation is weakly contained in one of the representation of the family.

In particular they prove that, for a groupoid that satisfies the Effros-Hahn conjecture and have amenable isotropy groups, the family of regular representations is exhaustive for the maximal $C^*$-algebra of the groupoid. This result has a famous result of Exel \cite{Exel} for corollary.

We are interested to applications of exhaustivity to \textit{Roe algebras} and \textit{HLS groupoids}. 
 
\subsection*{Matui's conjecture}


\section{Hilbert-Hadamard spaces and the Novikov conjecture}

\section{Controlled $K$-theory}

\subsection*{Baum-Connes for groupoids with finite complexity}

\textit{with Rufus Willett, University of Hawai'i at Manoa}\\

\subsection*{Getting rid of the $\varepsilon$ and $R$'s}