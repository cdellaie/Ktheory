\subsection{Groupoids and approximation properties}

The following projects revolve around the idea of building $C^*$-algebras with exotic properties using topological groupoids and $*$-representations of their convolution algebra.

\subsubsection*{Property T}

\textit{with Rufus Willett, University of Hawai'i at Manoa}\\

In \cite{DellWillett}, Rufus Willett and the author introduced a topological version of Property (T) for topological groupoids. This definition generalizes that of groups, but also recover \textit{Geometric Property (T)}, defined by Willett and Yu in \cite{WillettYu} for metric spaces with bounded geometry. It also is related to property $( \tau )$, a representation theoretic property for residually finite groups, giving a criterion to build expanders, and to \textit{measurable Property T} defined by Anantharaman-Delaroche in \cite{anantharamanT}. These connections provide evidence that our definition at least make sense, but its real interest lies in the following results.\\
\begin{itemize}
\item[$\bullet$] In the \'etale case, property (T) can be characterized as a spectral gap condition for operators analogous to the Laplacian on a compactly generated \'etale groupoid. This allows to characterize property (T) as a spectral gap condition and the existence of a Kazdhan projection. \\
\item[$\bullet$] Property (T), together with some technical assumptions, lead to \textit{failure of inner-exactness}. We show that this result holds for the coarse groupoid associated to an expander, and to the HLS groupoid associated to a residually finite group with property $(\tau )$. \\
\end{itemize}
In ongoing work, we try to apply topological property (T) to failure of K-exactness. For instance, it is shown by Spakula in \cite{SpakulaNonexact} that residually finite groups satisfying a strengthening of property $(\tau)$ have their uniform Roe algebra, which is a groupoid reduced $C^*$-algebra in disguise, are not $K$-exact. This suggest that a stronger form of topological property (T) would lead to failure of $K$-exactness for the reduced $C^*$-algebra of our groupoid. %We can aleady show this for uniform Roe algebras of some expanders. 

\begin{project}
Finding a condition for compactly generated \'etale groupoids which, together with property (T), ensures that its reduced $C^*$-algebra is not $K$-exact. 
\end{project}

\subsubsection*{Exhaustivity of representations}

\textit{with Yu Qiao}\\

Motivated by questions coming from spectral theory of N-body Hamiltonians, and characterization of Fredholm integral operators and Toeplitz operators, Nistor and Prudhon introduced in \cite{NistorPrudhon} the concept of \textit{exhaustive families of representations}. A family of $*$-representations of a $C^*$-algebra $A$ is \textit{exhaustive} if every irreducible $*$-representation is weakly contained in one of the representation of the family.\\

In particular they prove that, for a groupoid that satisfies the Effros-Hahn conjecture and have amenable isotropy groups, the family of regular representations is exhaustive for the maximal $C^*$-algebra of the groupoid. This result has for consequence a famous result of Exel \cite{exel2014invertibility} about invertibility of elements of amenable groupoids $C^*$-algebras.\\

We are interested to applications of exhaustivity to \textit{Roe algebras} and \textit{HLS groupoids}. Indeed, Skandalis, Tu and Yu defined in \cite{SkTuYu} a so-called \textit{coarse groupoid} $G(X)$ associated to a coarse metric space $X$, whose reduced $C^*$-algebra coincides with the uniform Roe algebra of $X$. This groupoid has features very similar to \textit{Fredholm groupoids}, intensively used in index theory: the base space $G^0 = \beta X$ has a dense open subset $U=X$ such that $G_{|U}\cong U\times U$. We denote by $F$ the boundary $\beta X - X$.

\begin{project}
Find conditions ensuring that the family of regular representations restricted to the boundary $\{\lambda_x\}_{x\in F}$ is exhaustive and/or giving an isomorphism $C_r^*G_F \cong C^*_r G / C^*_r G_U$.
\end{project}
 
For instance, we already know that this fails for coarse groupoids of some expanders, HLS groupoid of non amenable groups. 
\subsubsection*{Matui's conjecture}


\section{Hilbert-Hadamard spaces and the Novikov conjecture}

\section{Controlled $K$-theory}

Controlled, or quantitative, $K$-theory is a refinement of operator $K$-theory that has been developed following the celebrated proof of the Novikov conjecture for groups with asymptotic dimension by Guoliang Yu \cite{Yu1}. In this proof appear some groups build from almost-projections and almost-unitaries, with finite propagation, of the Roe algebra $C*(X)$ of a coarse space $X$. These groups are in essence $K$-theory groups with the added possibility of controlling the propagation of the classes, propagation coming from the distance function. This allows to apply \textit{controlled cutting and pasting} techniques to compute the $K$-theory groups of $C^*(X)$, i.e. Mayer-Vietoris like exact sequences, but involving \textit{almost ideals} instead of genuine ideals. In the case of finite asymptotic dimension, these ideals reduce to finite dimensional $C^*$-algebras, i.e. matrix algebras.\\

In \cite{OY1}, Oyono-Oyono and Yu applied this idea to the computation of the $K$-theory of the Roe algebras of some expanders. They formalized quantitative $K$-theory for $C^*$-algebras similar to reduced $C^*$-algebras of group $C^*_r(G)$ and Roe algebras $C^*(X)$ in \cite{OY2}, and in \cite{dell2018controlled}, the author generalized this approach to the setting of groupoid $C^*$-algebras and quantum groups, and general coarse spaces. These techniques were also applied to computations in $K$-theory:\\
\begin{itemize} 
\item[$\bullet$] Guentner, Willett and Yu proved in \cite{GWY2} the Baum-Connes conjecture for group actions which have \textit{finite dynamical complexity}, a transcription in the setting of topological group actions of \textit{finite decomposition complexity}, a weakening of \textit{finite asymptotic dimension}.\\   
\item[$\bullet$] Oyono-Oyono and Yu proved the the K\"unneth formula \cite{oyono2019quantitative} and the coarse Baum-Connes conjecture \cite{} for spaces with \textit{finite decomposition complexity}.\\
\end{itemize}

Let us point out that hese results do not give new examples where the Baum-Connes conjecture, coarse or not, holds. All these coarse spaces are coarsely embeddable into Hilbert spaces and all these groups or groups actions are a-T-menable, which falls under the monumental results of \cite{HigsonKasparov}, \cite{TuThese} and \cite{Yu2}. Their interest lie in that these last papers are notoriously hard, and the controlled philosophy gives new proofs which follows the classical cutting and pasting technique.\\  

\subsubsection*{Baum-Connes for groupoids with finite complexity}
  
\textit{with Rufus Willett, University of Hawai'i at Manoa}\\

The notion of \textit{finite dynamical complexity} naturally generalizes to \'etale groupoids, and these should be natural candidates for which the Baum-Connes conjecture can be proven using controlled $K$-theory. In a recent preprint \cite{willett2019decompositions}, Rufus Willett defined \textit{local decomposabity} for $C^*$-algebras, inspired by the above decomposition properties. He is then able to derive a Mayer-Vietoris like sequence involving a $C^*$-algebra which decomposes in that sense over two of its subalgebras. If they are ideals, one recover the classical Mayer-Vietoris exact sequence. This is used to establish a stability result for the K\"unneth formula and a stability result for the vanishing of $K$-theory groups.\\

\begin{project}
Prove the Baum-Connes conjecture for groupoids with finite dynamical complexity by local decomposability techniques.  
\end{project} 

\subsubsection*{Getting rid of the $\varepsilon$ and $R$'s}

The use of controlled $K$-theory is quite cumbersome. One has to keep track of $\varepsilon $ and $R$'s while doing already complicated $K$-theoretic computations. The following project is in the line of applying the controlled philosophy to $K$-theoretic computations, while trying to avoid keeping track of the propagation to every step. The idea is to define a global object in which appoximate relations become exact. \\

Let $A$ be a unital filtered $C^*$-algebra: $A$ is the closure of the union of a family  $\{A_r\}_{r>0}$ of closed self-adjoint subspaces such that $A_r A_s \subset A_{r+s}$. Define $\hat {l^\infty} (A)$ to be the closure in $\prod_{\N} A \otimes \mathfrak K$ of the union of 
\[\mathcal A_r = \prod_{i\in \N} A_r \otimes \mathfrak K \]
when $r>0$. Let $w\in \beta \N$ be a principal ultrafilter and define $\hat A_w$ to be the quotient of $\hat {l^\infty} (A)$ by the ideal 
\[ I_w = \{ (a_i) | \lim_w \|a_i \| = 0\}.\]
We call \textit{approximate ideal} of $A$ an ideal of $\hat A_w$. It can be shown that finite asymtotic dimension and finite dynamical complexity give rise to approximate ideals in $\hat{C^*(X)}_w$ and $\hat{C_r^*(G)}_w$ respectively. Moreover, in these cases, we have a genuine Mayer-Vietoris decomposition which, when evaluated at a finite place, gives back the quantitatve Mayer-Vietoris exact sequence. We thus can translate approximate ideals and quantitative relations into ideals and exact relations in the global algebra $\hat {l^\infty} (A)$.  

\begin{project}
Investigate stability properties of $K$-theoretic computations and the K\"unneth formula using $\hat {l^\infty} (A)$.
\end{project}  



























