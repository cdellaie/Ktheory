\documentclass[a4paper]{article}

\usepackage[frenchb]{babel}
\usepackage{amsfonts}
\usepackage{amsmath}
\usepackage{amssymb}
%\usepackage[T1]{fontenc}
\usepackage[utf8]{inputenc}
\usepackage{amsthm}
\usepackage{graphicx}
\usepackage{tikz}
\usepackage{tikz-cd}
\usepackage{hyperref}
\usepackage{amssymb}
\usepackage{geometry}

\hypersetup{                    % parametrage des hyperliens
    colorlinks=true,                % colorise les liens
    breaklinks=true,                % permet les retours à la ligne pour les liens trop longs
    urlcolor= blue,                 % couleur des hyperliens
    linkcolor= blue,                % couleur des liens internes aux documents (index, figures, tableaux, equations,...)
    citecolor= cyan               % couleur des liens vers les references bibliographiques
    }

\theoremstyle{definition}
\newtheorem{definition}{Definition}
\newtheorem{thm}{Theorem}
\newtheorem{ex}{Exercice}
\newtheorem{lem}{Lemma}
\newtheorem*{dem}{Proof}
\newtheorem{prop}{Proposition}
\newtheorem{cor}{Corollary}
\newtheorem{conj}{Conjecture}
\newtheorem{Res}{Result}
\newtheorem{Expl}{Example}
\newtheorem{rk}{Remark}

\newcommand{\N}{\mathbb N}
\newcommand{\Z}{\mathbb Z}
\newcommand{\R}{\mathbb R}
\newcommand{\C}{\mathbb C}
\newcommand{\Hil}{\mathcal H}
\newcommand{\Mn}{\mathcal M _n (\mathbb C)}
\newcommand{\K}{\mathbb K}
\newcommand{\B}{\mathbb B}
\newcommand{\Cat}{\mathbb B / \mathbb K}
\newcommand{\G}{\mathcal G }

\setlength\parindent{0pt}

\geometry{hmargin=2.5cm,vmargin=1.5cm}

\title{Teaching statement}
% and PhD description}

\date{}
\author{ Clément Dell'Aiera}


%\usepackage{fullpage}

\begin{document}

\maketitle

This document gives my teaching statement, followed by a record of my teaching experience. 
 
\section{Statement}

My primary goal as a teacher is to make students able to learn by themselves. I believe it builds up their mathematical’s confidence so that they are more willing to learn mathematics from other sources. I will now give details  on the method I use to try and do that.\\

\textbf{Experience.} I have been teaching since 2011, as a student, a PhD candidate and then as a postodoctoral fellow, both in France and in the US. Having the chance of being a pure mathematician that went through an Engineering School, I was assigned classes both in Pure and Applied Mathematics. This complementarty is really something I enjoy. Teaching Statistics or Computer sessions made me more aware of how to use computer, especially programming, in order to help students understand and manipulate objects they often find abstract. I know often show my students a piece of code and graphs in my lectures, even in Linear Algebra of Number Theory classes. On the other hand, I always underline the importance of understanding the abstract objects used in models for instance. As an example, one can think of all the hypothesis a statistical model has to satisfy in order to make predictions valid.\\

Another experience that changed my teaching practice is moving from France to the US. Teaching styles are really different. In my birth country, focus is put on content. The class has to be a challenge even for the best students. This made me aware of being structured and wanting to fit a lot of material into my lectures. After one year and a half of teaching in Hawaii though, I have learn to emphasize simple examples, to focus on not more than two new ideas during a lecture. As a result, I believe I am more careful about what the all students take out of the class and more sensitive to their needs.\\    

I organize my class in the following way: if there is one, follow the reference book, while always explaining things in several distinct ways so that the students see the actual lecture as a complement. I always write some notes in a personal file that is accessible online on my website, and assign homework at each class. Having regular homework allows to know how the class is doing, and also to locate which students need help, and which can do extra work if they are interested. My personal website is always updated with notes, homework and a schedule, which offers a summary of each class. My lectures always start by handing out the graded homework and by a quick reminder of last class. While teaching, I am trying to ask questions to the class, and really work toward having an active audience. This can be fostered by always being positive, even facing a ``wrong" answer. The end of the class is devoted as much as possible to 10 minutes of problems solving by the students, as a way to use what they just heard about.\\
   
%I should point out that during all these experiences, I always produced the written exercises myself, and collaborated for the final exam.  One of the main experience I have to share with students is my Applied Mathematics background : before turning to Fundamental Mathematics and studying at the Ecole Normale Supérieure (Cachan, Antenne de Bretagne), I studied in an engineering school, the ENSAE Paristech, specialized in Statistics, Finance and Economics. I was trained in Data Science, Statistics, etc. I always like to find concrete applications of the mathematics I am teaching. For example, one can even found that expanders, a notion from Coarse Geometry, has applications in network building !\\

\textbf{Curiosity.} The more the students are interested, the more they are willing to put some efforts into the class. I found that tickling their curiosity often gives great results. For example, I was very amazed by one of my undergraduate teacher who, during a class on linear algebra, explained to us how self-adjoint matrices were supposed to represent Hamiltonians in Quantum Mechanics, and how their spectra represented quantities that one can actually measures. This remark got me interested in Quantum Mechanics, and later, in Operator Algebra! The point is to show the students that they are not learning some very specialized mathematics, but are learning basics of a much greater structure. Often, I try to give examples that draw bridges between the class and very different topics. For example, I taught a class in Symbolic Calculus, and took advantage of a computer session to talk to my students about cryptography, and how arithmetic algorithms are used when paying by credit card, or communicating sensible information. Calculus classes are also a great place to talk about all models used in Physics, Biology, etc, that use differential equations. In Statistics, I tried to keep up with modern accessible things, and to teach about neural networks as well as linear models. Another example of interaction between different fields are expanders, a notion from Coarse Geometry, which has applications in electrical networks building !\\

\textbf{Learning by doing.} Practicing by yourself is crucial in order to learn the content of a class. I always propose homework, and I encourage students to give their answers back, so that I can give them feedback. I also reserve some time for communication during the class, so that the student can ask questions. The last practice I want to foster is to get the students to read. It can be survey or research articles, it can be mathematical books, but I try as much as possible to convince them to go to the library and expend their knowledge by reading, and not just searching the Internet (which they are usually already good at). \\

\textbf{Questions.} I am convinced that students learn a lot better if they are practicing when the class is over. To that end, I always encourage them to send any question by email if they didn’t have the time to ask during the class. I try as much as possible to listen to them and to produce a environment in which they feel safe to ask anything they did not understand. Usually, I always ask if anyone wants me to give more details on the subject of the class, and I focus on being receptive. By questioning them, they usually feel free to speak when they see they are not the only ones who missed something.\\

\textbf{Mentoring Students.} I was asked by Jean-Louis Tu to help an undergraduate student who was doing a research internship under his supervision last year at Elie Cartan’s Institute in Metz. His goal was to try and read an article on the Bost-Connes system. This system is a C*-dynamical system whose partition function is the famous Zêta function. I found it very fulfilling to mentor this student. His questions forced me to be as clear as possible, and we managed to study the basics on C*-algebras, and to look at the Pontryagin duality from the C*-algebraic point of view. \\

In Hawaii, the graduate students and I interact on a weekly basis, mainly during the Noncommutative Geometry Seminar. With Rufus Willett (my mentor) and erik Guentner, we organize the talks. Usually we present some paper in several lecture style talks, and occasionally ask a graduate student to give a talk if we think the results are not too technical.\\

\section{Teaching record}

\begin{itemize}
\item[$\bullet$] \textbf{ Spring 2019 :} University of Hawaii at Manoa.\\
					Math 649D, Topics class for graduate students: \textit{Large Scale Geometry and Operator Algebras}\\
					Math 244: Calculus IV\\
					\textit{Duties : Lecturing, writing up Worksheets, Exams, and Finals. Grading.}\\
\item[$\bullet$] \textbf{ Fall 2019 :} University of Hawaii at Manoa.\\
					Math 244: Calculus IV.\\
					\textit{Duties : Lecturing, writing up Worksheets, Exams, and Finals. Grading.}\\

\item[$\bullet$] \textbf{ Spring 2019 :} University of Hawaii at Manoa.\\
					Math 307: Linear Algebra and Differential Equations.\\
					\textit{Duties : Lecturing, writing up Worksheets, Exams, and Finals. Grading.}\\

\item[$\bullet$] \textbf{ Fall 2018 :} University of Hawaii at Manoa.\\
					Math 307: Linear Algebra and Differential Equations.\\
					\textit{Duties : Lecturing, writing up Worksheets, Exams, and Finals. Grading.}\\

\item[$\bullet$] \textbf{ Spring 2018 :} University of Hawaii at Manoa.\\
					 Math 307: Linear Algebra and Differential Equations.\\
					\textit{Duties : Lecturing, writing up Worksheets, Exams, and Finals. Grading.}\\

\item[$\bullet$] \textbf{ Fall 2017 :} University of Hawaii at Manoa.\\
				 	Math 203: Calculus for Business and the Social Sciences.\\
					\textit{Duties : Lecturing, writing up Worksheets, Exams, and Finals. Grading.}\\

\item[$\bullet$] \textbf{ Spring 2017 :} University of Lorraine.\\
					Teaching Assistant, Master 2. \\
					\textit{Duties : Computer sessions for ``Statistics and Probability".}\\
					Preparatory Class for the Agrégation de Mathématiques.\\
					\textit{Duties : Preparatory Class in Analysis (Le\c{c}ons d'Analyse).}\\
	
\item[$\bullet$] \textbf{ Fall 2016 :} University of Lorraine.\\
					Teaching Assistant, Master 1.\\
					\textit{Duties : Exercices sessions (travaux dirigés) for ``Stochastic Processes and Probability", Computer sessions (travaux dirigés) for ``Statistics and Time Series".}\\

\item[$\bullet$] \textbf{ Spring 2016 :} University of Lorraine.\\	
					Teaching Assistant, L2 et L3 Mathematics. \\
					\textit{Duties : Computer sessions (travaux dirigés) for the class ``Symbolic Computation" (Calcul Formel), Teaching classes in algebra (undergraduate group theory).}\\  
 
\item[$\bullet$] \textbf{ Fall 2015 :} University of Lorraine.\\
					Teaching Assistant, Master 1.\\
					\textit{Duties : Exercices sessions (travaux dirigés) for ``Stochastic Processes and Probability", Computer sessions for "Statistics and Time Series".}\\

\item[$\bullet$] \textbf{ Spring 2015 :} University of Lorraine.\\	
					Teaching Assistant, L2 et L3 Mathematics. \\
					\textit{Duties : Exercices sessions in Analysis and in Algebra. Computer sessions (travaux dirigés) for the class ``Symbolic Computation". Tutor for an undergraduate student working under the supervision of Pr. J-L. Tu. on $C^*$-algebras and number theory.}\\   
					
\item[$\bullet$] \textbf{ Fall 2014 :}  University of Lorraine.\\
					Teaching Assistant, Master 1. \\
					\textit{Duties : Exercices sessions for ``Stochastic Processes and Probability", Computer sessions for "Statistics and Time Series".}\\
					Teaching Assistant, L2 et L3 Mathematics. \\
					\textit{Duties : Exercices sessions in Analysis and in Algebra. Introduction to LateX.}\\
\item[$\bullet$] \textbf{ Year 2013 :} Lycée Sainte Marie de Neuilly, Teaching Assistant in Mathematics. \\
					\textit{Duties : Oral examinations of the ``Classes Préparatoires" BL} \\
\item[$\bullet$] \textbf{ Year 2011 :} University Paris~II Assas.\\
					Teaching Assistant in Data Analysis, Licence 3 Eco-gestion.\\
					\textit{Duties : Exercices Sessions.}\\
					Lycée Sainte Marie de Neuilly, Teaching Assistant in Mathematics. \\
					\textit{Duties : Oral examinations of the ``Classes Préparatoires" BL} 
\end{itemize}

\bibliographystyle{plain}
\bibliography{biblio2} 
%\nocite{*}

\end{document}
