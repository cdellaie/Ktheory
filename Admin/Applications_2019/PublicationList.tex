\documentclass[a4paper]{article}

\usepackage[frenchb]{babel}
\usepackage{amsfonts}
\usepackage{amsmath}
\usepackage{amssymb}
%\usepackage[T1]{fontenc}
\usepackage[utf8]{inputenc}
\usepackage{amsthm}
\usepackage{graphicx}
\usepackage{tikz}

%%%%%%%%%%%%%%%%%%%%%%%%%%%%%%%%%%%%%%%%%%%%%%%%%%%%%%%
%% Setting for the nice arrows and nodes graphs  %%%%%%
%%%%%%%%%%%%%%%%%%%%%%%%%%%%%%%%%%%%%%%%%%%%%%%%%%%%%%%

\usetikzlibrary{arrows,positioning,decorations.markings} 
\tikzset{
    %Define standard arrow tip
    >=stealth',
    %Define style for boxes
    punkt/.style={
           rectangle,
           rounded corners,
           draw=black, very thick,
           text width=6.5em,
           minimum height=2em,
           text centered},
    % Define arrow style
    pil/.style={
           ->,
           thick,
           shorten <=2pt,
           shorten >=2pt,}
}
\tikzstyle{vecArrow} = [thick, decoration={markings,mark=at position
   1 with {\arrow[semithick]{open triangle 60}}},
   double distance=1.4pt, shorten >= 5.5pt,
   preaction = {decorate},
   postaction = {draw,line width=1.4pt, white,shorten >= 4.5pt}]
\tikzstyle{innerWhite} = [semithick, white,line width=1.4pt, shorten >= 4.5pt]

%%%%%%%%%%%%%%%%%%%%%%%%%%%%%%%%%%%%%%%%%%%%%%%%%%%%%%%%
%%%%%%%%%%%%%%%%%%%%%%%%%%%%%%%%%%%%%%%%%%%%%%%%%%%%%%%%
\usepackage{tikz-cd}
\usepackage{hyperref}
\usepackage{amssymb}
\usepackage{geometry}

\hypersetup{                    % parametrage des hyperliens
    colorlinks=true,                % colorise les liens
    breaklinks=true,                % permet les retours à la ligne pour les liens trop longs
    urlcolor= blue,                 % couleur des hyperliens
    linkcolor= blue,                % couleur des liens internes aux documents (index, figures, tableaux, equations,...)
    citecolor= cyan               % couleur des liens vers les references bibliographiques
    }

\theoremstyle{definition}
\newtheorem{definition}{Definition}
\newtheorem{thm}{Theorem}
\newtheorem{ex}{Exercice}
\newtheorem{lem}{Lemma}
\newtheorem*{dem}{Proof}
\newtheorem{prop}{Proposition}
\newtheorem{cor}{Corollary}
\newtheorem{conj}{Conjecture}
\newtheorem{Res}{Result}
\newtheorem{Expl}{Example}
\newtheorem{rk}{Remark}

\newcommand{\N}{\mathbb N}
\newcommand{\Z}{\mathbb Z}
\newcommand{\R}{\mathbb R}
\newcommand{\C}{\mathbb C}
\newcommand{\Hil}{\mathcal H}
\newcommand{\Mn}{\mathcal M _n (\mathbb C)}
\newcommand{\K}{\mathbb K}
\newcommand{\B}{\mathbb B}
\newcommand{\Cat}{\mathbb B / \mathbb K}
\newcommand{\G}{\mathcal G }

\setlength\parindent{0pt}

\geometry{hmargin=2.5cm,vmargin=1.5cm}

\title{Publication List}
% and PhD description}

\date{}
\author{ Clément Dell'Aiera}


%\usepackage{fullpage}

\begin{document}

\maketitle

\begin{enumerate}
\item \textit{Topological property T for groupoids}, with Rufus Willett. Preprint on the \href{https://arxiv.org/abs/1811.07085}{arxiv}, november 2018. Submitted for publication.\\

Property (T) is a representation theoretic notion for topological groups, introduced by Kazdhan \cite{kazhdan1967connection} in 1967 in order to show that some lattices in some Lie groups are finitely generated. In a rather unexpected manner, such a particular goal gave one of the most fruitful notions, with applications ranging from pure to applied mathematics. Property (T) has been used intensively in representation theory and operator algebras, one of the most spectacular applications being the first non probabilistic proof of the existence of expander graphs by Margulis. To us, the main fact of interest is the equivalence between property (T) and the existence of a  very exotic projection in the maximal $C^*$-algebra of the group, called the \textit{Kazdhan projection}. This projection makes various short sequences of $C^*$-algebras not exact, leading to a failure, in some cases, of the Baum-Connes conjecture with coefficients. Trying to adapt these construction to the metric setting, Willett and Yu define in \cite{WillettYu} the \textit{Geometric Property (T)} for metric spaces with bounded geometry and carefully study its link with the Coarse Baum-Connes conjecture and the approximation properties of Roe algebras.\\

To a metric space with bounded geometry $X$ one can associate an ample groupoid $G(X)$ such that  $C^*_u(X)$ and $C_r^*(G(X))$ are naturally isomorphic with intertwined assembly maps. This suggest that one can define a version of property (T) for groupoids which generlizes both usual property (T) for groups and its geometric version. This is what we did, defining the concept of \textit{topological property (T)} for groupoids. After proving that topological property (T) indeed generalizes former similar notions, we study its relation to a-T-menability and also apply it to different situations. We prove for instance that property $(\tau)$ (a notion related to expander graphs) for residually finite groups is equivalent to a certain groupoid having topological property (T). In the last part, we give a condition ensuring that a property (T) groupoid is not inner-exact in $K$-theory. This is a strong property, giving obstructions to the Baum-Connes conjecture, studied at the end of the paper.

\item \textit{Going-Down functors and the Künneth-formula for crossed products by ample groupoids}, with Christian Bönicke. Preprint on the \href{https://arxiv.org/abs/1810.04415}{arxiv}, october 2018. Accepted in Transactions of the American Society.\\

\item \textit{A K\"{u}nneth formula for \'etale groupoids}, preprint available on \href{http://math.hawaii.edu/~dellaiera/Research.html}{personal webpage}.\\

\item \textit{Controlled $K$-theory for groupoids and applications to Coarse Geometry}, Journal of Functional Analysis, Volume 275, Issue 7, October 2018, Pages 1756-1807. Preprint on the \href{https://arxiv.org/abs/1710.06099}{arxiv}. 
\end{enumerate}

\newpage
\bibliographystyle{plain}
\bibliography{biblio2} 
%\nocite{*}

\end{document}
