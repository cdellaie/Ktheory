\documentclass[a4paper]{article}

\usepackage[frenchb]{babel}
\usepackage{amsfonts}
\usepackage{amsmath}
\usepackage{amssymb}
%\usepackage[T1]{fontenc}
\usepackage[utf8]{inputenc}
\usepackage{amsthm}
\usepackage{graphicx}
\usepackage{tikz}
\usepackage{tikz-cd}
\usepackage{hyperref}
\usepackage{amssymb}
\usepackage{geometry}

\hypersetup{                    % parametrage des hyperliens
    colorlinks=true,                % colorise les liens
    breaklinks=true,                % permet les retours à la ligne pour les liens trop longs
    urlcolor= blue,                 % couleur des hyperliens
    linkcolor= blue,                % couleur des liens internes aux documents (index, figures, tableaux, equations,...)
    citecolor= cyan               % couleur des liens vers les references bibliographiques
    }

\theoremstyle{definition}
\newtheorem{definition}{Definition}
\newtheorem{thm}{Theorem}
\newtheorem{ex}{Exercice}
\newtheorem{lem}{Lemma}
\newtheorem*{dem}{Proof}
\newtheorem{prop}{Proposition}
\newtheorem{cor}{Corollary}
\newtheorem{conj}{Conjecture}
\newtheorem{Res}{Result}
\newtheorem{Expl}{Example}
\newtheorem{rk}{Remark}

\newcommand{\N}{\mathbb N}
\newcommand{\Z}{\mathbb Z}
\newcommand{\R}{\mathbb R}
\newcommand{\C}{\mathbb C}
\newcommand{\Hil}{\mathcal H}
\newcommand{\Mn}{\mathcal M _n (\mathbb C)}
\newcommand{\K}{\mathbb K}
\newcommand{\B}{\mathbb B}
\newcommand{\Cat}{\mathbb B / \mathbb K}
\newcommand{\G}{\mathcal G }

\setlength\parindent{0pt}

\geometry{hmargin=2.5cm,vmargin=1.5cm}

\title{Publication List}
% and PhD description}

\date{}
\author{ Clément Dell'Aiera}


%\usepackage{fullpage}

\begin{document}

\maketitle

\section*{Publications}
\begin{enumerate}
\item \textit{Going-Down functors and the Künneth-formula for crossed products by ample groupoids}, with Christian Bönicke. Transactions of the American Society (2019).\\

In commutative algebraic topology, the K\"unneth formula allows to compute the cohomology of a product $H^*(X\times Y)$ using $H^*(X)$ and $H^*(Y)$. In the noncommutative setting, the K\"unneth formula deals with the computation of the operator $K$-theory groups $K_*(A \otimes B)$ of $C^*$-algebras knowing $K_*(A)$ and $K_*(B)$ ($\otimes$ being the spatial tensor product). Precisely, we say that $A$ satisfies the K\"unneth formula if a canonical map $\alpha_{A,B}: K(A)\otimes K(B) \rightarrow K(A\otimes B)$ is an isomorphism for every $C^*$-algebra $B$ with free abelian $K$-theory groups. It is known to hold for all \textit{bootstrap} $C^*$-algebras \cite{rosenberg1987kunneth}, and also for group $C^*$-algebras when the group is amenable (even a-T-menable) \cite{BaumConnesHigson}\cite{TuThese}. \\

In this paper, we prove the K\"unneth formula for $C^*$-algebras associated to ample groupoids satisfying the Baum-Connes conjecture such that all compact open subgroupoids satisfy the K\"unneth formula. This is a generalization of the restriction principle for groups developed by Chabert, Echterhoff et Oyono-Oyono \cite{ChabertEOY}. The interest of such work is the generous amount of $C^*$-algebras admitting a groupoid model, that is, which can be realized by (ample) groupoids $C^*$-algebras. The last part of the paper focuses on applications in Coarse Geometry. The K\"unneth formula is shown to hold for uniform Roe algebras of bounded geometry metric spaces which admit an coarse embedding into Hilbert space, and for the maximal Roe algebra of spaces admitting a fibred coarse embedding. Finally, we give an example of a space which admits no coarse embedding but whose uniform Roe algebras still satisfies the K\"unneth formula. This last example does not seem to be provable using previous methods, and use the full power of the restriction principle for groupoids.\\

\item \textit{Controlled $K$-theory for groupoids and applications to Coarse Geometry}, Journal of Functional Analysis, Volume 275, Issue 7, October 2018, Pages 1756-1807. \\

This paper summarizes my PhD thesis, where controlled $K$-theory is developed to the setting of \'etale groupoids, coarse spaces and quantum groups. This generalizes Herv\'e Oyono-Oyono et Guoliang Yu's quantitative $K$-theory for coarse spaces (see \cite{OY1}\cite{OY2}\cite{OY3}\cite{oyono2019quantitative}). \\

After definition and careful study of controlled $K$-theory, we show some of its main properties (approximating usual $K$-theory for instance, having a lot of exact sequences), we build controlled assembly maps, with value into those controlled $K$-groups, which factorizes the usual assembly maps. We state a controlled Baum-Connes conjecture for groupoids and coarse spaces, and relate it to the usual Baum-Connes conjectures. The end of the paper is dedicated to applications in Coarse Geometry.                                                                                                                                           
\end{enumerate}

\section*{Preprints}

\begin{enumerate}
\item \textit{Topological property T for groupoids}, with Rufus Willett. Preprint on the \href{https://arxiv.org/abs/1811.07085}{arxiv}, november 2018. Submitted for publication.\\

Property (T) is a representation theoretic notion for topological groups, introduced by Kazdhan \cite{kazhdan} in 1967 in order to show that some lattices in some Lie groups are finitely generated. In a rather unexpected manner, such a particular goal gave one of the most fruitful notions, with applications ranging from pure to applied mathematics. Property (T) has been used intensively in representation theory and operator algebras, one of the most spectacular applications being the first non probabilistic proof of the existence of expander graphs by Margulis. To us, the main fact of interest is the equivalence between property (T) and the existence of a  very exotic projection in the maximal $C^*$-algebra of the group, called the \textit{Kazdhan projection}. This projection makes various short sequences of $C^*$-algebras not exact, leading to a failure, in some cases, of the Baum-Connes conjecture with coefficients. Trying to adapt these construction to the metric setting, Willett and Yu define in \cite{WillettYu} the \textit{Geometric Property (T)} for metric spaces with bounded geometry and carefully study its link with the Coarse Baum-Connes conjecture and the approximation properties of Roe algebras.\\

To a metric space with bounded geometry $X$ one can associate an ample groupoid $G(X)$ such that  $C^*_u(X)$ and $C_r^*(G(X))$ are naturally isomorphic with intertwined assembly maps. This suggest that one can define a version of property (T) for groupoids which generlizes both usual property (T) for groups and its geometric version. This is what we did, defining the concept of \textit{topological property (T)} for groupoids. After proving that topological property (T) indeed generalizes former similar notions, we study its relation to a-T-menability and also apply it to different situations. We prove for instance that property $(\tau)$ (a notion related to expander graphs) for residually finite groups is equivalent to a certain groupoid having topological property (T). In the last part, we give a condition ensuring that a property (T) groupoid is not inner-exact in $K$-theory. This is a strong property, giving obstructions to the Baum-Connes conjecture, studied at the end of the paper.\\

\item \textit{A K\"{u}nneth formula for \'etale groupoids}, preprint available on \href{http://math.hawaii.edu/~dellaiera/Research.html}{personal webpage}.\\

In this paper, we prove the K\"unneth formula for $C^*$-algebras associated to \'etale groupoids satisfying the Baum-Connes conjecture such that all compact open subgroupoids satisfy the K\"unneth formula. This is a generalization of the restriction principle for groups developed by Chabert, Echterhoff et Oyono-Oyono \cite{ChabertEOY} and is another version of the work performed with Christian Bönicke. The constructions are nevertheless different. The induction functor is defined in an algebraic way, avoiding the continuous fields language, and we allow our groupoids to be \'etale instead of ample, with a conidition on proper actions (they should be locally induced by compact open subgroupoids, which can be thought of having enough compact open groupoid. The ample case of course fully fulfills this criterion). We still think that property should be studied in its own right.

%This paper is a preliminary version of the previous article. Discussing with Christian B\"onicke, we realized our respective work was overlapping and decided to turn it into a collaboration. 
\end{enumerate}

\bibliographystyle{plain}
\bibliography{biblio} 
%\nocite{*}

\end{document}
