
\documentclass[a4paper,11pt]{article} 
\usepackage[frenchb]{babel}
%\usepackage[T1]{fontenc} 
\usepackage[utf8]{inputenc}      
\usepackage{url}             
\usepackage{amsfonts}
\usepackage{amsmath}
\usepackage{amssymb}
\usepackage{amsthm}
\usepackage{hyperref}

%\theoremstyle{definition}
\newtheorem{theorem}{Th\'eor\`eme}
\newtheorem{thm}{Th\'eor\`eme}   
\setlength\parindent{0pt}

\pagestyle{empty}             
\usepackage{vmargin}           
\setmarginsrb{3cm}{3cm}{3cm}{3cm}{0cm}{0cm}{0cm}{0cm}

% Marge gauche, haute, droite, basse; espace entre la marge et le texte à
% gauche, en  haut, à droite, en bas

% Pour laisser de l'espace entre les lignes du tableau
\newcommand\espace{\vrule height 20pt width 0pt}

% Pour mes grands titres
\newcommand{\titre}[1]{%
	\begin{center}
	\bigskip
	\rule{\textwidth}{1pt}
	\par\vspace{0.1cm}
        \textbf{\large #1}
	\par\rule{\textwidth}{1pt}
	\end{center}
	\bigskip
	}

\begin{document}

\begin{flushleft}
Clément Dell'Aiera \\
Department of Mathematics, University of Hawaii\\
2565 McCarthy Mall, Keller 401A \\
Honolulu HI 96822 \\

\medskip
%Tél.: 06 74 62 99 52

E-mail: dellaiera.clement@gmail.com


\end{flushleft}
\begin{flushleft}
Nationalit\'e : Fran\c{c}aise \\
Date de naissance : 22/03/1990 \`{a} Metz (Moselle).
\end{flushleft}

\vspace{1.5cm}
\begin{center}
\par\huge{\textbf{Curriculum Vit\ae} }
\end{center}

%%%%%%%%%%%%%%%%%%
\titre{\'{E}ducation et emploi}
%#############%%%%

\begin{tabular}{cp{0.8\textwidth}}

\textbf{Ao\^{u}t 2017--pr\'{e}sent} &  \textbf{Assistant Professor} at U.H. Manoa (non-tenure track)  \\
						& Department of Mathematics\\
						&  Rank I3-M09 \\
						%& \textbf{Enseignement} \\
						%& Spring 2019: Math 307, Linear Algebra \& Differential Equations\\  
						%& Fall 2017: Math 203, Spring \& Fall 2018: Math 307\\
						\espace
\textbf{2014-2017} &  \textbf{Doctorat en Math\'ematiques} sous la supervision de \\
						& Hervé Oyono-Oyono et la co-direction d'Andrzej Zuk \\	
						& \textit{``Controlled K-theory for groupoids and applications"} \\
\espace
\textbf{2010--2014} &  \textbf{ENS Cachan} (Antenne de Bretagne) \\
				    & 	\'Ecole Normale Supérieure, D\'epartement de Math\'ematiques \\
                              & \textbf{ENSAE Paristech}\\
				&	Paris Graduate School of Economics, Statistics and Finance\\
                                   & \textbf{Master en Math\'ematiques Fondamentales}\\  & Universit\'e Paris~VII-Diderot \\
                                   & \textbf{Agrégation externe de Mathématiques} (2013) \\ 
							& Rang $41^e$ \\
\espace

\espace
\textbf{2007--2010} &\textbf{Classes préparatoires MP$^*$ } \\
					& Nancy, Lycée Henri Poincaré\\

\espace
\textbf{2007} & \textbf{Baccalauréat} (série S, sp\'ecialit\'e math\'ematiques) \\		& Mention Tr\`es bien \\
 \\

\end{tabular}
%%%%%%%%%%%%%%%%%%%%%%

\newpage
%%%%%%%%%%%%%%%%%%%%%%%%%%%%%%%%%%%%%%
\titre{Langues et informatique}
%#########################%%%%%%%%%%%%

\begin{tabular}{cp{0.8\textwidth}}

\textbf{Langues} &  \textbf{English} \\
					& Usage courant. TOEIC 915/990. Expatriation de 2 ann\'ees aux USA.\\
						&  \textbf{Spanish} \\
						& Oral \\
						\espace
\textbf{Informatique} & \textbf{Pack Office} \\	
						& Utilisation quotidienne \\	
						&  \textbf{LateX} \\
						& R\'edaction d'articles scientifiques et de notes de cours \\	
						&  \textbf{Python, Scilab, R, C/C++}\\
						& Utilisation r\'eguli\`ere pour enseignement et recherche \\			
						& \textbf{Objective Caml} \\	
						& Apprentissage en Classe Pr\'eparatoire \\			
						& \textbf{CSS, html} \\
						& R\'edaction de pages web \\			
						& \textbf{Github} \\
						& Utilisation quotidienne\\			
\end{tabular}

\titre{Exp\'erience}
%###################
Mes recherches se situent actuellement dans le domaine de la \textit{g\'eom\'etrie non-commutative} initi\'e par Alain Connes au d\'ebut des ann\'ees 90. Il s'agit, pour simplifier, d'utiliser des \textit{alg\`ebres d'op\'erateurs} pour r\'esoudre des probl\`emes de topologie alg\'ebrique. Les alg\`ebres d'op\'erateurs sont les structures math\'ematiques utilis\'ees en m\'ecanique quantique, et le programme d'Alain Connes s'inspire et s'applique en effet \`a la physique.\\

Mon activit\'e pr\'esente se partage entre la recherche, des enseignements, et des participations \`a certaines activit\'es administratives de l'University of Hawai'i at Manoa. \\

Avant cela, j'ai, en parall\`ele de mes \'etudes de math\'ematiques fondamentales, \'et\'e \'el\`eve \`a l'ENSAE Paristech. J'y ai suivi la sp\'ecialisation \textit{Data Science}.\\

\begin{tabular}{cp{0.8\textwidth}}
\textbf{2014} & M\'emoire de fin d'\'etude sous la supervision de Jérémy Jakubowicz, (Information geometry and deep learning): impl\'ementation sous Python de techniques d'apprentissage par r\'eseaux de neurones, deep learning et g\'eom\'etrie de l'information appliqu\'ees \`a la reconnaissance de caract\`eres manuscrits.\\
\espace
\textbf{2012}&  Stage de recherche sous la supervision de Cristina Butucea sur les Statistiques appliqu\'ees \`a l'optique quantique \`a l'Universit\'e de Marne-la-Vallée (LAMA). Plus pr\'ecis\'ement, \'etude de la tomographie quantique comme support pour la transmission de \textit{qubits} en \textit{informatique quantique}.\\
\end{tabular}
\\

% La scolarit\'e \`a l'ENSAE est ponctu\'ee de projet de groupes
%\bibliographystyle{plain}
%\bibliography{biblio} 
\end{document}






















