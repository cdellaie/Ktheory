\section{Research Project}

This section gives details on my future work. These are ideas that I would like to pursue, but I am off course very interested in other projects for which I would be qualified.\\

The first project is the continuation of my PhD thesis : to study controlled Mayer-Vietoris decompositions associated to étale groupoids. The idea is to consider a decomposition of the base space into open subsets, which are not supposed to be invariant. This leads to a Mayer-Vietoris exact sequence in $K$-homology for the groupoid, but also to a controlled Mayer-Vietoris exact sequence in controlled $K$-theory. The first goal is to show that the controlled assembly maps respects in a controlled way these two Mayer-Vietoris exact sequences. This leads to stability results of the controlled Baum-Connes Conjecture, which could be used to show the Baum-Connes conjecture for some kind of groupoids. The highlight of this method is that a proof actually gives you an algorithm to compute the $K$-theory of the crossed-product $C^*$-algebra of the groupoid, which is far from true in the usual cases, e.g. for a-T-menable groupoids. %For an example, take the topological groupoid associated to a manifold with corners, which is amenable, hence satisfies the Baum-Connes conjecture with coefficients, but for which only some examples of actual computations are known. 
This could be applied to a new proof of the Baum-Connes conjecture for étale groupoids with finite dynamic asymptotic dimension as defined by E. Guentner, R. Willett, and G. Yu \cite{GWY}.\\

The next ideas are part of a second project and the author really thinks that they are worth to be developed.\\

First, one could develop a theory of decomposition complexity for étale groupoids, and, if properly defined, the technique using Mayer-Vietoris decomposition should really work for this notion. More precisely, one should be able to prove that groupoids with finite decomposition complexity satifiy the Baum-Connes conjecture. The work would include showing explicit examples of groupoids which are not of finite dynamic asymptotic dimension, but are of finite decomposition complexity. The point of using both this kind of decomposition and controlled $K$-theory is that it is more algebraic in nature and could be adapted to other assembly conjectures, such as the Farell-Jones conjecture or the algebraic Novikov conjecture. \cite{RTY} \\

On another level, one could develop the range of applications of controlled $K$-theory. Indeed, the formalism is easily generalized to more abstract situations, see my PhD dissertation for more details. This could in principle be used to give a filtration on other $C^*$-algebras coming from general topological groupoids, quantum groups or reduced $C^*$-algebras twisted by finite dimensional representations, which are Banach algebras. Another very promising application of controlled $K$-theory is classification of $C^*$-algebras. \\

Controlled $K$-theory for filtered Banach algebras would also hopefully lead to interesting ideas. One can off course always ask why the $K$-theory of the reduced $C^*$-algebra is chosen to be the range of the assembly map. Some situations require to consider other Banach algebras, possibly lacking the $C^*$-property. In this area, $L^p$-version of the Baum-Connes conjecture are heavily studied, and a controlled $K$-theory could be of strong interest.\\

This last idea would also be the starting point of a more ambitious project : developing property T for topological groupoids. I am very interested in a paper of C. Dru{\c{t}}u and P. Nowak \cite{DrutuNowak} in which they relate generalized property T for groups to spectral gap of families of isometric representations. This leads to idempotents in completions of convolution algebras. The latter are only $C^*$-algebras if the representations of the family are all unitary. The generalization of these idea to groupoids could be used to define a property T for Coarse Spaces $X$ using the Coarse Groupoid $G(X)$. Then one could study the link between this property T for $G(X)$ and the geometric property T of R. Willett and G. Yu \cite{WY}. The long term goal of this line of thought is to develop a so called Oka principle for Coarse Spaces.\\

Please do not hesitate to contact me if further information is needed.\\

Sincerely,\\
Clément Dell’Aiera

