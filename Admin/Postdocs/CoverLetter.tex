November 15, 2016

Dear Committee Members,\\

I wish to apply for a Visiting Faculty position in the Department of Mathematics at Texas A$\&$M University. Currently, I am a graduate student at the University of Lorraine at Metz (France) in the Elie Cartan’s Institute, and am finishing a PhD thesis under the supervision of Pr. Hervé Oyono-Oyono, and codirection of Pr. Andrzej Zuk.\\

My primary research goals are directed towards Noncommutative geometry, K-theory of C*-algebras and Index Theory. As a PhD student, I followed the route of Pr. Hervé Oyono-Oyono and Pr. Guoliang Yu by using a new version of K-theory which they developed during the past years in order to study propagation effects in index theorems.\\

My work consisted first in defining a modified version of this controlled K-theory in order to extend it for topological groupoids. I was then able to construct a controlled version of the Baum-Connes assembly map for an étale groupoid that factorizes the usual one for groupoids. This allows to establish strongest results in Coarse Geometry , e.g. one can show that the controlled Coarse Baum-Connes conjecture holds for any metric space which admits a fibered coarse embedding into a Hilbert space. H. Oyono-Oyono and G. Yu obtained a similar statement using controlled K-theory for spaces which admits a Coarse embedding into a Hilbert space, which is a stronger condition to fulfill. \\

Other applications are very natural to prove once one has the power of this formalism, including a controlled Künneth formula for the K-theory of the reduced C*-algebra of ample groupoids. The highlights of the controlled versions of classical properties such as satisfying the Baum-Connes conjecture or the Künneth formula are that these are more stable, which is the second part of focus of my PhD thesis. The idea behind this line of thought is to use a weak type of decomposition of groupoids which could be seen as a groupoid version of decomposition complexity, and is related to the notion of dynamic asymptotic dimension, developed recently by Pr. Rufus Willett, Pr. Erik Guentner and Pr. Yu. This part of my work is not completely finished yet, but I am working very hard to finish it this year.\\

My work is not published yet, and, if desired, I would be happy to provide a preprint of the results I described above. My future research plans are aimed at developing ideas coming from Coarse Geometry to topological groupoids, in order to find new applications, not only in Coarse Geometry but in other groupoid related areas, such as foliations or group actions. I would also like to explore controlled K-theory for filtered Banach algebras, in order to tackle non-C*-completions of Roe algebras or of groupoid convolution algebras. This would be a start towards an Oka principle applied to Coarse Spaces. \\

Beyond my research, I had a wide range of teaching experiences. I have been a teaching assistant for undergraduate students almost without interruption for four years in very different areas : from data analysis and linear models to algebra, arithmetic and analysis. I also taught practical computing sessions on machine learning. As I graduated from an engineering school before turning to fundamental mathematics, I can teach applied as well as fundamental mathematics, and I am very sensitive about applications of science. It could be a real asset for your department, and I will work hard to be the best teacher I can. \\

Finally, I would like to mention that I was organizer of the PhD and Postdocs Students seminar of the department of mathematics of the University of Lorraine. In my opinion, social life and interactions with other research fields are very crucial aspects of research. I invited young researchers from various fields, from Statistics to Quantum groups, and from other laboratories. If I was given the opportunity, this is an objective I would pursue in my future career. \\

Enclosed are my curriculum vitae, publication record, teaching and research statement. Please do not hesitate to contact me if further information is needed. \\

Sincerely,\\

Clément Dell’Aiera\\
3 rue du Pont Saint-Marcel\\
57000 METZ (FRANCE)\\
