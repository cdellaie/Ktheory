
\documentclass[a4paper,11pt]{article} 
\usepackage[frenchb]{babel}
%\usepackage[T1]{fontenc} 
\usepackage[utf8]{inputenc}      
\usepackage{url}                


\pagestyle{empty}             
\usepackage{vmargin}           
\setmarginsrb{3cm}{3cm}{3cm}{3cm}{0cm}{0cm}{0cm}{0cm}

% Marge gauche, haute, droite, basse; espace entre la marge et le texte à
% gauche, en  haut, à droite, en bas

% Pour laisser de l'espace entre les lignes du tableau
\newcommand\espace{\vrule height 20pt width 0pt}

% Pour mes grands titres
\newcommand{\titre}[1]{%
	\begin{center}
	\bigskip
	\rule{\textwidth}{1pt}
	\par\vspace{0.1cm}
        \textbf{\large #1}
	\par\rule{\textwidth}{1pt}
	\end{center}
	\bigskip
	}

\begin{document}

\begin{flushleft}
Clément Dell'Aiera \\
Department of Mathematics, University of Hawaii\\
2565 McCarthy Mall, Keller 401A \\
Honolulu HI 96\ 822 \\

\medskip
%Tél.: 06 74 62 99 52

E-mail: dellaiera.clement@gmail.com


\end{flushleft}
\begin{flushleft}
Nationality : French \\
Date of Birth : 03/22/1990 in Metz (Moselle).
\end{flushleft}

\vspace{1.5cm}
\begin{center}
\par\huge{\textbf{Curriculum Vit\ae} }
\end{center}

%%%%%%%%%%%%%%%%%%
\titre{Education}
%#############%%%%

\begin{tabular}{cp{0.8\textwidth}}

\textbf{August 2017--present} &  \textbf{Assistant Professor} at U.H. Manoa  \\
						& Dept. of Mathematics\\
						& Rank I3-M09. \\
						& \textbf{Teaching duties} \\
						& Fall 2017: Math 203, Spring \& Fall 2018: Math 307\\ 
\espace
\textbf{2014-2017} &  \textbf{PhD student} under the supervision of Pr. Hervé  \\
						& Oyono-Oyono on the theme ``Controlled K-theory \\
						& for groupoids and applications". \\
\espace
\textbf{2010--2014} &  \textbf{ENS Cachan} (Antenne de Bretagne) \\
				    & 	Ecole Normale Supérieure, Department of Mathematics \\
                              & \textbf{ENSAE Paristech}\\
				&	Paris Graduate School of Economics, Statistics and Finance\\
                                   & \textbf{Master in Pure Mathematics}\\  & University Paris~VII-Diderot. \\
                                   & \textbf{Agrégation de Mathématiques} (2013) : rank $41^e$. \\
				& National competitive examination to be high school teacher\\
\espace

\espace
\textbf{2007--2010} &\textbf{Classes préparatoires MP$^*$ } \\
					& Nancy, Lycée Henri Poincaré\\
					& Get prepared for the highly competitive entrance \\
				& examinations to the top French Engineering Schools.\\

\espace
\textbf{2007} & \textbf{Baccalauréat} (série S, option mathematics) 
 \\

\end{tabular}

\newpage
%%%%%%%%%%%%%%%%%%%%%%%%%%
\titre{Research}
%#########################

\textbf{Publication list:} 
\begin{enumerate}
\item \textit{Controlled $K$-theory for groupoids and applications to Coarse Geometry}, Journal of Functional Analysis. 
\item \textit{A K\"{u}nneth formula for \'etale groupoids}, preprint available.
\end{enumerate}
\espace

\textbf{Research talks :}\\

\begin{itemize}
\item[$\bullet$] June 2018, ``C*-alg\`ebres g\'eom\'etriques et applications en g\'eom\'etrie coarse", S\'eminaire d'AO, Paris Diderot.
\item[$\bullet$] May 2018, ``Geometric C*-algebras: applications to the K\"unneth formula", GPOTS 2018, Miami University.
\item[$\bullet$] February 2018, ``Geometric C*-algebras and Coarse structures", Workshop on computability of K-theory for C*-algebras, Texas A\&M University.
\item[$\bullet$] June 2017, ``Principe de restriction pour les groupoïdes étales. Application à une formule de Künneth pour leurs produits croisés.", Séminaire d'Algèbres d'Opérateurs, Paris-Diderot (Paris 7).
\item[$\bullet$] December 2016, ``Controlled K-theory for groupoids. Applications to Coarse Geometry", Kleines Seminar, Münster.
\item[$\bullet$] December 2016, ``K-théorie quantitative et applications", Arbre de Noël du GDR Géométrie Non-commutative, Albi.
\item[$\bullet$] May 2016, ``Asymptotic dimension for étale groupoids", Noncommutative Geometry Seminar, IECL, Metz-Nancy
\item[$\bullet$] December 2015, ``Controlled $K$-theory for groupoids", Arbre de Noël du GDR Géométrie Non-Commutative, Montpellier
\end{itemize}
\espace

\textbf{Talks directed to non-specialists :}\\

\begin{itemize}
\item[$\bullet$] February 2018, ``From a notion of dynamical dimension to cutting and pasting algebras", Analysis Seminar, University of Hawaii.
%\item[$\bullet$] December 2017, "Expanders", Young Researchers' Day, IECL, Metz- Nancy
\item[$\bullet$] October 2016, ``Expanders", Landau Seminar, IRMAR, Rennes %: From Network reliability to K-theory", Landau Seminar, IRMAR, Rennes
\item[$\bullet$] February 2016, ``Applications of groupoids to Physics", Young Researchers Seminar, IECL, Metz-Nancy
\item[$\bullet$] October 2015, ``The Novikov conjecture for groups with finite asymptotic dimension", Henri Lebesgue Workshop, Nantes
\item[$\bullet$] October 2015, ``Propagation in $K$-theory", Young Researchers Seminar, IECL, Metz-Nancy
\item[$\bullet$] January 2015, ``Introduction to groupoids $C^*$-algebras", Young Researchers Seminar, IECL, Metz-Nancy
\end{itemize}
\espace

\textbf{Other activities:} 
\begin{itemize}
\item[$\bullet$] 2017-2018 : I currently co-organize, with Erik Guentner and Rufus Willett, the Noncommutative Geometry Seminar of the department of Mathematics of the University of Hawaii at Manoa. The list of talks is available on my personal webpage.
\item[$\bullet$] 2016-2017 : Co-organizer with Matthieu Brachet of the Young Researchers Seminar, IECL, Metz-Nancy and Représentant du personnel au conseil du Laboratoire de l'IECL, Collège C.
\end{itemize}

\newpage
%%%%%%%%%%%%%%%%%%%%%%%%%%%%
\titre{Teaching Experience}
%%%%%%%%%%%%%%%%%%%%%%%%%%%%

\begin{itemize}
\item[$\bullet$] \textbf{ Year 2017 :} University of Hawaii at Manoa.\\
					Fall, Math 203: Calculus for Business and the Social Sciences.\\
					Spring, Math 307: Linear Algebra and Differential Equations.\\
					\textit{Duties : Lecturing, writing up Worksheets, Exams, and Finals. Grading.}\\

\item[$\bullet$] \textbf{ Year 2016 :} University of Lorraine.\\
					Teaching Assistant, Master 2. Preparatory Class for the Agrégation.\\
					\textit{Duties : Preparatory Class in Analysis, Computer sessions for "Statistics and Probability".}\\
					Teaching Assistant, Master 1.\\
					\textit{Duties : Exercices sessions for "Stochastic Processes and Probability", Computer sessions for "Statistics and Time Series".}\\
					Teaching Assistant, L2 et L3 Mathematics. \\
					\textit{Duties : Computer sessions for the class "Symbolic Computation".}\\   
\item[$\bullet$] \textbf{ Year 2015 :} University of Lorraine.\\
					Teaching Assistant, Master 1.\\
					\textit{Duties : Exercices sessions for "Stochastic Processes and Probability", Computer sessions for "Statistics and Time Series".}\\
					Teaching Assistant, L2 et L3 Mathematics. \\
					\textit{Duties : Exercices sessions in Analysis and in Algebra. Computer sessions for the class "Symbolic Computation". Tutor for an undergraduate student working under the supervision of Pr. J-L. Tu.}\\   
					
\item[$\bullet$] \textbf{ Year 2014 :}  University of Lorraine.\\
					Teaching Assistant, Master 1. \\
					\textit{Duties : Exercices sessions for "Stochastic Processes and Probability", Computer sessions for "Statistics and Time Series".}\\
					Teaching Assistant, L2 et L3 Mathematics. \\
					\textit{Duties : Exercices sessions in Analysis and in Algebra. Introduction to LateX.}\\
\item[$\bullet$] \textbf{ Year 2013 :} Lycée Sainte Marie de Neuilly, Teaching Assistant in Mathematics. \\
					\textit{Duties : Oral examinations of the ``Classes Préparatoires" BL} \\
\item[$\bullet$] \textbf{ Year 2011 :} University Paris~II Assas.\\
					Teaching Assistant in Data Analysis, Licence 3 Eco-gestion.\\
					\textit{Duties : Exercices Sessions.}\\
					Lycée Sainte Marie de Neuilly, Teaching Assistant in Mathematics. \\
					\textit{Duties : Oral examinations of the ``Classes Préparatoires" BL} 
\end{itemize}

\newpage
%%%%%%%%%%%%%%%%%%%%%%%%%%
\titre{Work Experience}
%###################%%%%%%

\begin{itemize}
\medskip
\item[$\bullet$] \textbf{Internships}: \\

During my education at ENSAE Paristech : \\

\begin{tabular}{cp{0.8\textwidth}}
\textbf{2014} & Master's thesis for the ENSAE under the supervision of Pr. Jérémy Jakubowicz, (Information geometry and deep learning), on applying deep learning and information geometry methods to image processing by neural networks.\\
\textbf{2012}&  $10$-weeks research internship under the supervision of Pr. Cristina Butucea on the theme of Statistics applied to Quantum Optics at University of Marne-la-Vallée (LAMA).		\\
\textbf{2011} & $2$-months mission in Bolivia with the humanitarian association Mission Potosi.\\
\end{tabular}
\\

\textbf{Master's thesis} under the supervision of Pr. Hervé Oyono-Oyono, Six term exact sequences in $K$-theory for crossed products of $C^*$-algebras by $Z$.

\medskip

\item[$\bullet$] \textbf{Professional Experience}\\ 2010--2012 : Worked for SANEF (Motorway tollbooth) in Saint-Avold.\\

\medskip
\end{itemize}

%%%%%%%%%%%%%%%%%%%%%%%%%%%%%%%%%%%%%%
\titre{Languages and computer skills}
%#########################%%%%%%%%%%%%

\begin{itemize} 
\medskip
\item[$\bullet$] \textbf{English} Fluent. TOEIC 915/990.
\medskip
\item[$\bullet$] \textbf{Spanish} Oral.
\medskip
\item[$\bullet$] \textbf{Softwares} Pack Office, LateX, Python, Scilab, R, Objective Caml, C/C++, html.
\end{itemize}

\end{document}
