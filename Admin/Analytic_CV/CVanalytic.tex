
\documentclass[a4paper,11pt]{article} 
\usepackage[frenchb]{babel}
%\usepackage[T1]{fontenc} 
\usepackage[utf8]{inputenc}      
\usepackage{url}             
\usepackage{amsfonts}
\usepackage{amsmath}
\usepackage{amssymb}
\usepackage{amsthm}
\usepackage{hyperref}

%\theoremstyle{definition}
\newtheorem{theorem}{Th\'eor\`eme}
\newtheorem{thm}{Th\'eor\`eme}   

\usepackage[frenchb]{babel}
\usepackage{amsfonts}
\usepackage{amsmath}
\usepackage{amssymb}
%\usepackage[T1]{fontenc}
\usepackage[utf8]{inputenc}
\usepackage{amsthm}
\usepackage{graphicx}
\usepackage{tikz}

%%%%%%%%%%%%%%%%%%%%%%%%%%%%%%%%%%%%%%%%%%%%%%%%%%%%%%%
%% Setting for the nice arrows and nodes graphs  %%%%%%
%%%%%%%%%%%%%%%%%%%%%%%%%%%%%%%%%%%%%%%%%%%%%%%%%%%%%%%

\usetikzlibrary{arrows,positioning,decorations.markings} 
\tikzset{
    %Define standard arrow tip
    >=stealth',
    %Define style for boxes
    punkt/.style={
           rectangle,
           rounded corners,
           draw=black, very thick,
           text width=6.5em,
           minimum height=2em,
           text centered},
    % Define arrow style
    pil/.style={
           ->,
           thick,
           shorten <=2pt,
           shorten >=2pt,}
}
\tikzstyle{vecArrow} = [thick, decoration={markings,mark=at position
   1 with {\arrow[semithick]{open triangle 60}}},
   double distance=1.4pt, shorten >= 5.5pt,
   preaction = {decorate},
   postaction = {draw,line width=1.4pt, white,shorten >= 4.5pt}]
\tikzstyle{innerWhite} = [semithick, white,line width=1.4pt, shorten >= 4.5pt]

%%%%%%%%%%%%%%%%%%%%%%%%%%%%%%%%%%%%%%%%%%%%%%%%%%%%%%%%
%%%%%%%%%%%%%%%%%%%%%%%%%%%%%%%%%%%%%%%%%%%%%%%%%%%%%%%%
\usepackage{tikz-cd}
\usepackage{hyperref}
\usepackage{amssymb}
\usepackage{geometry}

\hypersetup{                    % parametrage des hyperliens
    colorlinks=true,                % colorise les liens
    breaklinks=true,                % permet les retours à la ligne pour les liens trop longs
    urlcolor= blue,                 % couleur des hyperliens
    linkcolor= blue,                % couleur des liens internes aux documents (index, figures, tableaux, equations,...)
    citecolor= cyan               % couleur des liens vers les references bibliographiques
    }

\theoremstyle{definition}
\newtheorem{definition}{Definition}
\newtheorem{thm}{Theorem}
\newtheorem{ex}{Exercice}
\newtheorem{lem}{Lemma}
\newtheorem*{dem}{Proof}
\newtheorem{prop}{Proposition}
\newtheorem{cor}{Corollary}
\newtheorem{conj}{Conjecture}
\newtheorem{Res}{Result}
\newtheorem{Expl}{Example}
\newtheorem{rk}{Remark}

\newcommand{\N}{\mathbb N}
\newcommand{\Z}{\mathbb Z}
\newcommand{\R}{\mathbb R}
\newcommand{\C}{\mathbb C}
\newcommand{\Hil}{\mathcal H}
\newcommand{\Mn}{\mathcal M _n (\mathbb C)}
\newcommand{\K}{\mathbb K}
\newcommand{\B}{\mathbb B}
\newcommand{\Cat}{\mathbb B / \mathbb K}
\newcommand{\G}{\mathcal G }

\setlength\parindent{0pt}


\setlength\parindent{0pt}

\pagestyle{empty}             
\usepackage{vmargin}           
\setmarginsrb{3cm}{3cm}{3cm}{3cm}{0cm}{0cm}{0cm}{0cm}

% Marge gauche, haute, droite, basse; espace entre la marge et le texte à
% gauche, en  haut, à droite, en bas

% Pour laisser de l'espace entre les lignes du tableau
\newcommand\espace{\vrule height 20pt width 0pt}

% Pour mes grands titres
\newcommand{\titre}[1]{%
	\begin{center}
	\bigskip
	\rule{\textwidth}{1pt}
	\par\vspace{0.1cm}
        \textbf{\large #1}
	\par\rule{\textwidth}{1pt}
	\end{center}
	\bigskip
	}

\begin{document}

\begin{flushleft}
Clément Dell'Aiera \\
Department of Mathematics, ENS Lyon\\
46 allée d’Italie\\
69342 Lyon Cedex 07\\
FRANCE\\

\medskip
E-mail: clement.dellaiera@ens-lyon.fr

\end{flushleft}
\begin{flushleft}
Nationalit\'e : Fran\c{c}aise \\
Date de naissance : 22/03/1990 \`{a} Metz (Moselle).
\end{flushleft}

\vspace{1.5cm}
\begin{center}
\par\huge{\textbf{Curriculum Vit\ae} }
\end{center}

%%%%%%%%%%%%%%%%%%
\titre{\'{E}ducation et emploi}
%#############%%%%

\begin{tabular}{cp{0.8\textwidth}}

\textbf{Septembre 2020--présent} &  \textbf{Agr\'eg\'e pr\'eparateur} à l'ENS Lyon \\
						& Dept. of Mathematics, UMPA\\

\espace

\textbf{2017--2020} &  \textbf{Temporary Assistant Professor} at U.H. Manoa  \\
						& Dept. of Mathematics\\
						%& Rank I3-M09. \\
\espace
\textbf{2014-2017} &  \textbf{Doctorat en Math\'ematiques} sous la supervision de \\
						& Hervé Oyono-Oyono et la co-direction d'Andrzej Zuk \\	
						& \textit{``Controlled K-theory for groupoids and applications"} \\
\espace
\textbf{2010--2014} &  \textbf{ENS Cachan} (Antenne de Bretagne) \\
				    & 	\'Ecole Normale Supérieure, D\'epartement de Math\'ematiques \\
                              & \textbf{ENSAE Paristech}\\
				&	Paris Graduate School of Economics, Statistics and Finance\\
                                   & \textbf{Master en Math\'ematiques Fondamentales}\\  & Universit\'e Paris~VII-Diderot \\
                                   & \textbf{Agrégation externe de Mathématiques} (2013) \\ 
							& Rang $41^e$ \\
\espace

\espace
\textbf{2007--2010} &\textbf{Classes préparatoires MP$^*$ } \\
					& Nancy, Lycée Henri Poincaré\\

\espace
\textbf{2007} & \textbf{Baccalauréat} (série S, sp\'ecialit\'e math\'ematiques) 
 \\

\end{tabular}

\newpage
%%%%%%%%%%%%%%%%%%%%%%%%%%
\titre{Activit\'es de recherche}
%#########################

\textbf{Liste de publications:} 
\begin{enumerate}
\item \textit{Topological property T for groupoids}, avec Rufus Willett. Preprint sur arxiv, novembre 2018. Soumis pour publication.
\item \textit{Going-Down functors and the Künneth-formula for crossed products by ample groupoids}, avec Christian Bönicke. Transactions of the American Society, 2019.
\item \textit{A K\"{u}nneth formula for \'etale groupoids}, preprint disponible sur ma webpage personnelle.
\item \textit{Controlled $K$-theory for groupoids and applications to Coarse Geometry}, Journal of Functional Analysis,Volume 275, Issue 7, October 2018, Pages 1756-1807. 
\end{enumerate}

\espace

\textbf{S\'eminaires:}

\begin{itemize}
\item[$\bullet$] 2020-2021: co-organisateur, avec Ruben Martos, Frank Taipe et Makoto Yamashita, du Quantum Group Seminar.\\
Participant aux groupes de travail et séminaires suivants: 
\begin{enumerate}
\item Actions! organisé par Damien Gaboriau, 
\item Lifting properties for $C^*$-algebras, organisé par Mikael de la Salle, 
\item Théorie des nombres, organisé par Frédérique Déglise. \\
\end{enumerate}

\item[$\bullet$] 2017-2020: co-organisateur, avec Erik Guentner et Rufus Willett, du Noncommutative Geometry Seminar of the department of Mathematics of the University of Hawaii at Manoa. Notes et programme sont disponibles sur ma page.\\

\item[$\bullet$] Spring 2019: participant au s\'eminaire de \textit{Geometric Group Theory}, organis\'e par Asaf Hadari et Andrew Sale, sur le th\`eme \textit{Right-Angled Artin Groups}.  \\
\item[$\bullet$] Fall 2018 \& Spring 2019: co-organisateur, avec Piper Harron, Sarah Post et Andrew Sale, du \textit{Colloquium de math\'ematiques}, un rendez-vous bi-mensuel ou mensuel \`a destination de tout le d\'epartement. \\%Le but est de pr\'evoir des expos\'es (suivis de rafra\^{i}chissements) accessibles aux math\'ematiciens de tout bords, professeurs, postdoctorants ainsi que th\'esards.\\
\item[$\bullet$] Spring 2018: participant au s\'eminaire \textit{Topological Quantum Field Theory}, organis\'e par Sarah Post. Notes de mes expos\'es disponibles sur ma page personnelle. \\
\item[$\bullet$] 2016-2017 : co-organisateur avec Matthieu Brachet, du \textit{S\'eminaire doctorants} en math\'ematiques de l'IECL, Metz-Nancy.
\end{itemize}

\espace

\textbf{Responsabilit\'es administratives:} \\
\begin{itemize}
%\item[$\bullet$] ... \\
\item[$\bullet$] 2016-2017 : Représentant du personnel au conseil du Laboratoire de l'IECL, Collège C.\\
\end{itemize}

\newpage
\titre{Expos\'es des travaux de recherche (orateur)}

%%%%%%%%%%%%%%%%%%%%%%%%%%%%%%
\begin{tabular}{cp{0.8\textwidth}}


\textbf{2021}   & \textbf{From Fredholm operators to topological invariants} \\
                & 27/01, Young Mathematicians Seminar, Aarhus University\\ %https://math.au.dk/en/currently/activities/event/item/6212/
                & \textbf{La compactification de Borel-Serre} \\
                & 11/01, Séminaire Borel, ENS Lyon \\
\\
\textbf{2020}
                & \textbf{Property (T), projective representations and LP, after Ioana-Spaas-Wiersma}\\
                & 18/12, Séminaire Lifting properties for $C^*$-algebras, ENS Lyon \\
                & \textbf{Propriété T dynamique et géométrique} \\ % https://indico.math.cnrs.fr/event/6064/    
                & 21/09, Séminaire de Géométrie, Groupes et Dynamique, ENS Lyon \\
                & \textbf{Actions métriques de groupes et leurs algèbres d'opérateurs}\\  %https://indico.math.cnrs.fr/event/6038/
                & 05/09, Théminaire, ENS Lyon \\
\\

\textbf{2019} 	& \textbf{The restriction principle and the Künneth formula}\\ % https://www.math.tamu.edu/~kerr/gpots2019/program.html
				& 29/05, GPOTS, Texas A\&M University \\   
				& \textbf{The restriction principle and the Künneth formula}\\	
				& 02/05, NCG Festival, Washington University, Saint-Louis \\ %https://sites.google.com/site/ncgwustl/program \\
				& \textbf{Decomposition complexity, a dynamical approach}\\
				& 15/02, Analysis Seminar, University of Houston. \\ %https://www.math.uh.edu/analysis/
				& \textbf{Decomposition complexity, a dynamical approach}\\
				& 13/02, Noncommutative Geometry Seminar, Texas A\&M.  \\ % http://www.math.tamu.edu/seminars/noncomgeom/
				\espace
\textbf{2018} & \textbf{Property T for topological groupoids}\\
				& 07/11, Noncommutative Geometry Seminar, Texas A\&M. \\ % http://www.math.tamu.edu/seminars/noncomgeom/
				& \textbf{Dynamical Property T}\\
				& 05/11, Noncommutative Geometry Seminar, University of Houston. \\ %https://www.math.uh.edu/analysis/Fall18.php
				& \textbf{Dynamical Property T}\\
				& 20/09, Noncommutative geometry Seminar, PennState University.\\
				& \textbf{C*-alg\`ebres g\'eom\'etriques et applications en g\'eom\'etrie coarse}\\
				& Juin, S\'eminaire d'AO, Paris Diderot.\\
				& \textbf{Geometric C*-algebras: applications to the K\"unneth formula}\\
				& Mai, GPOTS 2018, Miami University.\\
				& \textbf{Geometric C*-algebras and Coarse structures}\\
				& F\'evrier, Workshop on computability of K-theory for C*-algebras, Texas A\&M University.\\
\end{tabular}  

\begin{tabular}{cp{0.8\textwidth}}
				%\espace	
\textbf{2017} & \textbf{Principe de restriction pour les groupoïdes étales. Application à une formule de Künneth pour leurs produits croisés}\\
				& Juin, Séminaire d'Algèbres d'Opérateurs, Paris-Diderot (Paris 7).\\
				\espace	
\textbf{2016} & \textbf{Controlled K-theory for groupoids \& applications to Coarse Geometry}\\
				& D\'ecembre, Kleines Seminar, Münster.\\
				& \textbf{K-théorie quantitative et applications}\\
				& D\'ecembre, Arbre de Noël du GDR Géométrie Non-commutative, Albi.\\
				& \textbf{Asymptotic dimension for étale groupoids} \\
				& Mai, Noncommutative Geometry Seminar, IECL, Metz-Nancy.\\
				\espace	
\textbf{2015} & \textbf{Controlled $K$-theory for groupoids}\\
				& D\'ecembre, Arbre de Noël du GDR Géométrie Non-Commutative, Montpellier.\\
\end{tabular}

\vfill
\textit{Note: Ne sont pas comptabilis\'es ici les expos\'es donn\'es aux diff\'erents s\'eminaires locaux auxquels je participe. Leur liste est disponible sur la page du s\'eminaire \href{https://clementstuff.wordpress.com/noncommutative-geometry/}{ici}.}
%%%%%%%%%%%%%%%%%%%%%%%%%%%%%%
\newpage
%%%%%%%%%%%%%%%%%%%%%
%%%%%%%%%%%%%%%%%%%%%%%%%%%%
\titre{Enseignements}
%%%%%%%%%%%%%%%%%%%%%%%%%%%%
\begin{tabular}{cp{0.8\textwidth}}

\textbf{ 2019 } 	& \textbf{University of Hawaii at Manoa}\\
					& Spring, Math 307: Linear Algebra and Differential Equations.\\
					& \textit{Cours magistraux, r\'edactions et corrections des devoirs et examens}\\
\espace
\textbf{ 2018 } 	& \textbf{University of Hawaii at Manoa}\\
					& Fall, Math 307: Linear Algebra and Differential Equations.\\
					& Spring, Math 307: Linear Algebra and Differential Equations.\\
					& \textit{Cours magistraux, r\'edactions et corrections des devoirs et examens}\\
\espace
\textbf{ 2017 } 	& \textbf{University of Hawaii at Manoa}\\
					& Fall, Math 203: Calculus for Business and the Social Sciences.\\
					& \textit{Cours magistraux, r\'edactions et corrections des devoirs et examens}\\
\espace
\textbf{ 2016 } 	& \textbf{Universit\'e de Lorraine}\\
					&	Master 2, le\c{c}ons d'Analyse pour l'Agrégation, TP de Mod\'elisation Statistiques et Probabilit\'es.\\
					&	Master 1,	charg\'e de TD pour le cours Processus Statistiques et Probabilit\'e, TP pour Statistiques et S\'eries Temporelles. \\
					&	L2 et L3: TD et kh\^{o}lles. \\
\espace
\textbf{ 2015 } 	& \textbf{Universit\'e de Lorraine}\\
			 		& Master 1, Charg\'e de TD pour Processus Stochastiques et Probabilit\'e, TP pour Statistiques et S\'eries Temporelles.\\
					& L2 et L3: Charg\'e de TD d'alg\`ebre et d'analyse. TP pour Calcul Formel.\\ 
					& Aide \`a l'encadrement d'un \'etudiant de L3 pour un stage sous la supervision J-L. Tu.\\   
\espace		
\textbf{ 2014 } 	& \textbf{Universit\'e de Lorraine}\\
     					& Master 1: Charg\'e de TD pour Processus Stochastiques et Probabilit\'e, TP pour Statistiques et S\'eries Temporelles.\\ 
					& L2 et L3: Charg\'e de TD d'alg\`ebre et d'analyse. Introduction \`a LateX.\\
%\end{tabular}
%\begin{tabular}{cp{0.8\textwidth}}
\espace
 \textbf{ 2013 }   & \textbf{Lycée Sainte Marie de Neuilly} \\
					& K\^{o}lles de mathematics en classes préparatoires BL.\\
\espace
\textbf{ 2011 } 	& \textbf{University Paris~II Assas}\\
					& Charg\'e de TD en Analyse de Donn\'ees, Licence 3 \'Eco-gestion.\\
					& \textbf{Lycée Sainte Marie de Neuilly} \\
					& K\^{o}lles de math\'ematiques en classes préparatoires BL. \\
\end{tabular}

\newpage
%%%%%%%%%%%%%%%%%%%%%%%%%%
\titre{Autre}
%###################%%%%%%
\textbf{Expos\'es \`a destination de non-sp\'ecialistes:}\\

\begin{itemize}
\item[$\bullet$] F\'evrier 2018, ``From a notion of dynamical dimension to cutting and pasting algebras", Analysis Seminar, University of Hawaii\\
\item[$\bullet$] D\'ecembre 2017, ``Expanseurs", No\"el des Doctorants, IECL, Metz- Nancy\\
\item[$\bullet$] Octobre 2016, ``Expanseurs", S\'eminaire Landau, IRMAR, Rennes \\ %: From Network reliability to K-theory", Landau Seminar, IRMAR, Rennes
\item[$\bullet$] F\'evrier 2016, ``Applications des groupo\"{i}des en Physique", S\'eminaire Doctorants, IECL, Metz-Nancy\\
\item[$\bullet$] Octobre 2015, ``The Novikov conjecture for groups with finite asymptotic dimension", Henri Lebesgue Workshop, Nantes\\
\item[$\bullet$] Octobre 2015, ``Propagation in $K$-theory", S\'eminaire Doctorants, IECL, Metz-Nancy\\
\item[$\bullet$] Janvier 2015, ``Introduction aux groupo\"{i}des et \`a leurs $C^*$-alg\`ebres", S\'eminaire Doctorants, IECL, Metz-Nancy\\
\end{itemize}

%\begin{itemize}
\espace
\textbf{M\'emoire de Master} sous la supervision de Hervé Oyono-Oyono, \textit{Six term exact sequences in $K$-theory for crossed products of $C^*$-algebras by Z}.\\

\vfill
\textbf{Math\'ematiques appliqu\'ees} Lors de ma scolarit\'e \`a l'ENSAE Paristech:\\
\medskip
%\item[$\bullet$] \textbf{Internships}: \\
%During my education at ENSAE Paristech : \\
\espace
\begin{tabular}{cp{0.8\textwidth}}
\textbf{2014} & M\'emoire de fin d'\'etude sous la supervision of Jérémy Jakubowicz, \textit{Information geometry and deep learning}. Application de m\'ethode de deep learning et de g\'eom\'etrie de l'information \` la reconnaissance de caract\`res manuscrits. Impl\'ementation de r\'eseaux de neurones utilisant notamment une descente de gradient riemannienne.\\
\espace
\textbf{2012}&  Stage de $10$ semaines sous la supervision de Cristina Butucea sur le th\`eme \textit{Statistiques appliqu\'ees \`a l'optique quantique} \`a l'Universit\'e de Marne-la-Vallée (LAMA).		\\
\end{tabular}
\\

%\medskip
%\end{itemize}

%%%%%%%%%%%%%%%%%%%%%%%%%%%%%%%%%%%%%%
\titre{Langues et informatique}
%#########################%%%%%%%%%%%%

\begin{itemize} 
\medskip
\item[$\bullet$] \textbf{English} Fluent. TOEIC 915/990.
\medskip
\item[$\bullet$] \textbf{Spanish} Oral.
\medskip
\item[$\bullet$] \textbf{Softwares} Pack Office, LateX, Python, Scilab, R, Objective Caml, C/C++, html.
\end{itemize}

\newpage

%%%%%%%%%%%%%%%%%%%%%%%%%%%%%%%%%%%%%%%%%%%%%%%%%%%%%%%%%%%%%%%%%%%%%%%%%
\titre{Description d\'etaill\'ee des travaux et projets de recherches}
%%%%%%%%%%%%%%%%%%%%%%%%%%%%%%%%%%%%%%%%%%%%%%%%%%%%%%%%%%%%%%%%%%%%%%%%%

%Cette partie vise \`a d\'etailler les travaux d\'ej\`a accomplis ainsi que futurs. 
Le texte se divise en trois parties. La premi\`ere pr\'esente le domaine d'int\'er\^et du candidat, d'une mani\`ere g\'en\'erale, afin d'introduire les objets et les questions sur lesquelles nous travaillons. La seconde partie d\'etaille les travaux publi\'es ainsi que les sujets d\'ej\`a termin\'es. La derni\`ere partie se consacre \`a des questions ou des projets qui se veulent des ouvertures vers des travaux futurs. La r\'edaction, qui d\'ebute sans assumer quelconque connaissance en g\'eom\'etrie non-commutative et alg\`ebres d'op\'erateurs, exige de fa\c{c}on croissante une connaissance de plus en plus pouss\'ee de ces domaines. Ainsi, %la lectrice 
les lecteurs curieux de nos activit\'es peuvent se limiter \`a la premi\`ere section, l\`a o\`u les sp\'ecialistes peuvent sans probl\`eme directement commencer par la partie suivante.

\subsection*{Alg\`ebres d'op\'erateurs et g\'eom\'etrie non-commutative}

Pour d\'efinir notre champ d'\'etude, nous pr\'esenterons rapidement les objets \'etudi\'es et les questions pos\'ees \`a leur sujet.\\

Les alg\`ebres d'op\'erateurs sont n\'ees avec les travaux de Murray et Von Neumann qui, dans 4 articles fondateurs \cite{murray1936rings}, d\'efinissent les alg\`ebres de Von Neuman (alors appel\'ees \textit{rings of operators}). Les motivations proviennent ici de la physique: la m\'ecanique quantique date d'il y a moins de 30 ans et consiste encore \`a l'\'epoque d'une liste de r\`egles de calcul, que les physiciens s'efforcent de synth\'etiser en une th\'eorie coh\'erente. Rappelons qu'une alg\`ebre de Von Neumann est une sous-alg\`ebre auto-adjointe des op\'erateurs born\'es sur un espace de  Hilbert, \'egale \`a son bicommutant. La signification physique est la suivante: une mesure effectu\'ee sur un syst\`eme quantique est repr\'esent\'ee par un op\'erateur auto-adjoint $a$ sur un espace de Hilbert $H$. Si $\xi \in H$ est un vecteur unitaire, la mesure de l'observable $a$ dans l'\'etat $\xi$ est donn\'ee par la variable al\'eatoire associ\'ee \`a la forme lin\'eaire $\mu(f) = \langle \xi , f(a) \xi \rangle$ o\`u $f$ est une fonction continue sur le spectre de $a$.  \\

Dans les ann\'ees 40, les alg\`ebres d'op\'erateurs se voient appliqu\'ees en analyse harmonique apr\`es les travaux de Gelfand et Naimark. C'est \`a ce moment que naissent les $C^*$-alg\`ebres. La condition du bicommutant est remplac\'ee par ``\^etre ferm\'ee pour la topologie induite par la norme d'op\'erateurs''. La situation apr\`es la seconde guerre voient les applications fleurir: classification de facteurs de type III avec Alain Connes \cite{connes1976classification}, polyn\^ome de Jones \cite{jones1990hecke}, qui offre le premier invariant topologique pour les noeuds capable de distinguer un noeud de son symm\'etrique-miroir, th\'eorie quantique des champs alg\'ebrique \cite{haag1964algebraic}, application en th\'eorie de l'indice et \`a la topologie des feuilletage (Alain Connes \cite{connesfoliations}), formulation g\'eometrico-alg\'ebrique d'une version du mod\`ele standard, utilisant notamment les \textit{triplets spectraux} d'Alain Connes \cite{chamseddine1996universal}, applications d'assemblages de Baum-Connes \cite{BaumConnesHigson} et leur succ\`es retentissant en topologie alg\'ebrique avec une solution tr\`es g\'en\'erale \`a la conjecture de Novikov (voir \cite{HigsonKasparov}), etc. \\

Le type de questions qui nous int\'eressent se situent \`a la fronti\`ere entre la g\'eom\'etrie m\'etrique, la topologie alg\'ebrique et les propri\'et\'es d'approximation. Ces derni\`eres datent plus ou moins des travaux de th\`ese de Grothendieck, qui a d\'ebut\'e en analyse fonctionnelle. Alors venu faire sa th\`ese \`a Nancy avec Dieudonn\'e et Schwartz, Grothendieck se voit propos\'e de cr\'eer une notion de produit tensoriel pour les espaces vectoriels topologiques. Ce probl\`eme lui vient de Schwartz, qui aimerait simplifier sa preuve du th\'eor\`eme des noyaux, qui dit (en simplifiant) que toute op\'erateur lin\'eaire born\'e admet ``une matrice", c'est-\`a-dire un noyau, si l'on admet les distributions comme noyaux possibles. En dimension finie, ce r\'esultat est rapidement prouv\'e: il s'agit de montrer que l'espace vectoriel des op\'erateurs lin\'eaires de $V$ dans $W$ est isomorphe \`a $V^* \otimes W$. La suite est bien connue: Grothendieck d\'efinit non pas un, mais toute une famille de produit tensoriels, qui admettent un \'el\'ement maximal et un \'el\'ement minimal. Les espaces pour lesquels ces deux notions co\"incident, et n'ont donc qu'un seul produit tensoriel, sont appel\'es \textit{nucl\'eaires}, en r\'ef\'erence au th\'eor\`eme.\\

Permettons nous un bon dans le temps, et passons aux $C^*$-alg\`ebres. Elles aussi poss\`edent des produits tensoriels, avec un produit tensoriel minimal (dit \textit{spatial}) et un \textit{maximal}. Les $C^*$-alg\`ebres nucl\'eaires sont celles n'en poss\'edant qu'un seul. Une question naturelle est alors de savoir si toute les $C^*$-alg\`ebres sont nucl\'eaires ou non. C'est \`a ce moment que l'on appelle la th\'eorie des groupes au secours. Lors de ses travaux sur les d\'ecompositions paradoxales, Von Neumann a d\'efini la notion de groupe moyennable. Un groupe discret $\Gamma$ est \textit{moyennable} s'il existe une \textit{moyenne}, i.e. une forme lin\'eaire born\'ee $m : l^\infty (\Gamma) \rightarrow \mathbb C$ telle que $m(1_{l^\infty(\Gamma)}) = 1$ et qui soit $\Gamma$-invariante ($\Gamma$ agit naturellement par translation sur $l^\infty (\Gamma)$). Les groupes moyennables comprennent les groupes ab\'eliens, finis, r\'esolubles, le groupe de Grigorchuk, du \textit{lamplighter}, etc.\\

Maintenant, \`a un groupe discret $\Gamma$ est associ\'ee sa $C^*$-alg\'ebre r\'eduite $C^*_r(\Gamma)$. Il se trouve que le th\'eor\`eme suivant clos la question de la nucl\'earit\'e de telles $C^*$-alg\`ebres. 

\begin{theorem}En gardant les m\^emes notations, $\Gamma$ est moyennable ssi $C^*_r(\Gamma)$ est nucl\'eaire.
\end{theorem}

Il suffit donc de prendre n'importe quel groupe non-moyennable, par exemple le groupe libre \`a 2 g\'en\'erateurs $\mathbb F_2$ ou encore $SL(2,\mathbb Z)$, pour produire une $C^*$-alg\`ebre non nucl\'eaires.\\    

Il existe pl\'ethore de propri\'etes d'approximations, et qui sont souvent reli\'ees \`a des propri\'et\'es d'objets provenant d'autres domaines, \`a qui l'on associe une $C^*$-alg\`ebre. Un bon exemple est la $C^*$-alg\`ebre de Roe uniforme $C^*_u(X)$ associ\'ee \`a un espace m\'etrique $X$. La nucl\'earit\'e d'icelle est \'equivalente \`a l'espace $X$ satisfaisant la propri\'et\'e A de Yu \cite{nowak2008property}. Ces $C^*$-alg\`ebres sont importantes en topologie alg\'ebrique. Leurs groupes de $K$-th\'eorie sont le r\'eceptacle des indices pseudo-differentiels abstraits (i.e. des classes de $K$-homologie). Ici, l'on utilise la $K$-th\'eorie op\'eratorielle, qui est une g\'en\'eralisation de la $K$-th\'eorie topologique d'Atiyah aux alg\`ebres de Banach. Elle offre un analogue \`a la cohomologie dans le cas commutatif (et c'est bien un th\'eorie cohomologique au sens de Steenrod). Elle est \`a ne pas confondre avec la $K$-th\'eorie alg\'ebrique, m\^eme si pour une large classe de $C^*$-alg\`ebres, elles co\"incident. L'exemple le plus frappant est peut-\^etre la cas de la conjecture de Novikov. La preuve la plus g\'en\'erale qui existe \`a ce jour utilise la conjecture de Baum-Connes, qui donne un algorithme pour calculer les groupes de $K$-th\'eorie de $C^*_r(\Gamma)$ et $C_u^*(X)$: si un groupe est a-T-moyennable, ou s'il admet un plongement uniforme dans un espace de Hilbert s\'eparable, alors il satisfait la conjecture de Novikov. En particulier, cela est vrai de tous les groupes moyennables, ou encore hyperboliques au sens de Gromov.\\

Dans mes travaux, j'utilise ces techniques pour \'etudier principalement des espaces m\'etriques, des actions de groupes et leurs $C^*$-alg\`ebres. La notion unificatrice derri\`ere ces concepts est celle de \textit{groupo\"ide topologique}, qui est une g\'en\'eralisation \`a la fois de celle de groupe et d'espace. La partie semblable \`a un groupe est not\'ee $G$, tandis que cette derni\`ere est ``au-dessus" d'un espace topologique $G^0$, appel\'e \textit{base}. Elle est assez souple pour pouvoir aussi \^etre associ\'ee \`a des actions topologiques de groupe (la donn\'ee d'un morphisme $G\rightarrow Homeo(X)$) ou des espaces m\'etriques. Les groupo\"ides $G$ qui nous int\'eressent poss\'edent naturellement des $C^*$-alg\`ebres $C_r^*(G)$ d\'efinis comme dans le cas des groupes comme l'adh\'erence, pour la norme d'op\'erateur dans $B(L^2(G))$, des op\'erateurs de convolution par $C_c(G)$.

%\newpage
%%%%%%%%%%%%%%%%%%%%%%%%%%%%
\subsection*{Publications}
%%%%%%%%%%%%%%%%%%%%%%%%%%%%

Les trois principales questions concernant la $K$-th\'eorie des $C^*$-alg\'ebres auxquelles je m'int\'eresse sont la conjecture de Baum-Connes, la formule de K\"unneth et la formule de coefficients universels. Nous pr\'esentons maintenant les travaux publi\'es. \`A noter: en cas de convocation, le candidat pr\'esentera au choix soit ses travaux sur la propri\'et\'e T, soit ceux sur la formule de K\"unneth.

\subsection{Dynamic asymptotic dimension and Matui's HK conjecture}

En préparation, disponible sur demande. Avec Christian B\"onicke, Jamie Gabe et Rufus Willett.\\

Dans \cite{Guentner:2014aa}, Guentner, Willett et Yu ont introduit une notion de \textit{dimension asymptotique dynamique} qui généralise la dimension asymptotique de Gromov. Elle s'applique aux actions de groupes discrets par homéomorphismes sur des espaces localement compacts, et plus généralement aux groupoïdes localement compact et étales. 

Cet article explore les implications de la dimension asymptotique dynamique sur l'homologie des groupoïdes et la $K$-théorie des $C^*$-algèbres de groupoïdes. 

La théorie de l'homologie pour les groupoïdes définie par Crainic et Moerdijk \cite{Crainic2000} a générée un intérêt croissant depuis les travaux de Matui \cite{Matui2012}. Notre résultat principal est que tout groupoïde $\sigma$-compact, principal et ample $G$ de dimension asymptotique dynamique $dad(G)$ finie est de dimension homologique finie.
\begin{thmx}
$$ d=dad(G)<\infty \implies \forall n>d, H_n(G)=0$$
et $H_d(G)$ est un groupe libre.
\end{thmx}

La preuve de ce résultat repose sur une description des groupes d'homologie en termes d'espaces simpliciaux munis d'une action de $G$. Cela permet de décrire ces groupes au moyen du foncteur Tor classique, situation naturelle analogue au cas des groupes. 

\begin{thmx}
Pour tout groupoïde ample $G$ tel que $G^0$ soit $\sigma$-compact, il existe un isomorphisme canonique 
$$ H_*(G)\cong \text{Tor}_*^{\Z[G]}(\Z[G^0],\Z[G^0]).$$
\end{thmx}

Le premier résultat a des conséquences importantes pour la conjecture suivante, formulée par Matui dans \cite{Matui2016}.

\begin{conjecture*}[HK]
	Soit $G$ un groupoïde minimal essentiellement principal et ample. Alors
	$$K_i(C_r^*(G))\cong \bigoplus_{n\geq 0}H_{2n+i}(G),\quad i=0,1$$
\end{conjecture*}

Bien que l'on doive un contre-exemple à Scarparo \cite{Scarparo2020} dans le cas où les groupes d'isotropie contiennent de la torsion, la conjecture a été confirmée dans le cas principal sans même la restriction de minimalité \cite{Farsi2019,Matui2016,Ortega2020,Yi20}. Grâce à une suite spectrale récemment contruite par Proietti et Yamashita \cite{Proietti2020}, nous confirmons la conjecture HK pour une large classe de groupoïdes de petite dimension.

\begin{corx}
	Soit $G$ un groupoïde principal, $\sigma$-compact, ample, de dimension asymptotique dynamique au plus $2$. Alors $G$ satisfait la conjecture HK, i.e.
	$$K_0(C_r^*(G))\cong H_0(G)\oplus H_2(G),\quad K_1(C_r^*(G))\cong H_1(G).$$
\end{corx}

Dans la cas de la dimension $1$, ce résultat permet de classifier complètement les $C^*$-algèbres réduites de groupoïdes qui le satisfont. En effet, pour un groupoïde ample et principal, la dimension nucléaire de la $C^*$-algèbre réduite est bornée par la dimension asymptotique dynamique (\cite{Guentner:2014aa}), $C^*_r(G)$ est donc classifiable par l'invariant d'Elliot.   

\begin{corx}
	Soit $G$ un groupoïde principal, $\sigma$-compact, ample, de dimension asymptotique dynamique au plus $2$. Alors 
	$$\mathrm{Ell}(C_r^*(G))=(H_0(G),H_0(G)^+,[1_{G^{(0)}}],H_1(G),M(G),p).$$
\end{corx}

Comme application, nous étudions des exemples en géométrie \textit{coarse}. Grâce au groupoïde coarse de \cite{SkTuYu}, nous obtenons un résultat qui n'apparaît pas dans la littérature (sauf pour le degré $0$, traité dans \cite{Ara2020}).

\begin{thmx}\label{Thm:D}
	Soit $X$ un espace métrique à géométrie bornée et $G(X)$ son groupoïde coarse. Il exist un isomorphisme canonique	
    $$H_*(G(X))\cong H_*^{\mathrm{uf}}(X)$$
	où le terme de droite est l'homologie uniformément finie de $X$, suivant Block and Weinberger \cite{Block1992}.
\end{thmx}

Comme la dimension asymptotique de $X$ au sens de Gromov coïncide avec $dad(G(X))$, une combinaison des résultats précédents donnent les corrolaires suivants, purement géométriques. 

\begin{corx}
	Soit $X$ un espace métrique à géométrie bornée. 
	\begin{enumerate}
		\item Si $\mathrm{asdim}(X)\leq 2$, alors
		$$K_0(C_u^*(X))\cong H_0^{\mathrm{uf}}(X)\oplus H_2^{\mathrm{uf}}(X)\quad K_1(C_u^*(X))\cong H_1^{\mathrm{uf}}(X).$$
		\item Si $\mathrm{asdim}(X)\leq 3$ et $X$ est non-moyennable, alors
		$$K_0(C_u^*(X))\cong H_2^{\mathrm{uf}}(X)\quad K_1(C_u^*(X))\cong H_1^{\mathrm{uf}}(X)\oplus H_3^{\mathrm{uf}}(X).$$
	\end{enumerate}
\end{corx}


\subsubsection*{Topological property T for groupoids} 
Avec Rufus Willett. Soumis pour publication.\\

La propri\'et\'e T est une condition sur les repr\'esentations unitaires d'un groupe. Elle a \'et\'e d\'efinie par Kazdhan \cite{kazhdan1967connection} en 1967 afin de prouver que certains r\'eseaux de certains groupes de Lie sont finiement engendr\'es. De fa\c{c}on inattendue, elle se r\'ev\'ela \^etre tr\`es fructueuse en th\'eorie des repr\'esentations et en alg\`ebres d'op\'erateurs. Citons par exemple la premi\`ere preuve non probabiliste par Margulis de l'existence de graphes expanseurs. L'aspect qui nous int\'eresse ici est que, si $\Gamma$ a la propri\'et\'e T, il existe un projecteur $p$ dans $C^*_r(\Gamma)$ (la projection de Kazdhan) dont les propri\'et\'es sont tr\'es exotiques. Il est la raison pour laquelle la propri\'et\'e T est rest\'ee pendant longtemps une obstruction \`a la conjecture de Baum-Connes (jusqu'aux travaux de Vincent Lafforgue \cite{lafforgue2002k} qui prouve la dite conjecture pour certains groupes ayant T). Partant de ce constat, Willett et Yu d\'efinissent dans \cite{WillettYu} un \textit{propri\'et\'e T g\'eom\'etrique} pour les espaces m\'etriques afin d'\'etudier ses liens avec la conjecture de Baum-Connes et les propri\'et\'es d'approximation.\\

Le point de d\'epart de ce travail est le constat qu'\`a un espace m\'etrique $X$ peut \^etre associ\'e un groupo\"ide topologique $G(X)$, \'etale \`a base totalement discontinue, telle que les $C^*_u(X)$ et $C_r^*(G(X))$ soit naturellement isomorphes, et qui entrelacent les applications d'assemblages associ\'ees. Nous avons alors d\'efini une \textit{propri\'et\'e T topologique} pour les groupo\"ides topologiques. L'on montre bien entendu que cela g\'en\'eralise le cas des groupes et des espaces m\'etriques. L'article s'attaque ensuite \`a appliquer cette propri\'et\'e T dynamique \`a d'autres situations: il est par exemple prouv\'e qu'un groupe finiement engendr\'e r\'esiduellement fini a la propri\'et\'e $\tau$ (une condition pour fabriquer des graphes expanseurs) ssi un certain groupo\"ide a la propri\'et\'e T. La derni\`ere partie donne une condition sur un groupo\"ide ayant la propri\'et\'e T qui assure que sa $C^*$-alg\`ebre est non-exacte en $K$-th\'eorie. Cela est une propri\'et\'e tr\`es forte, qui a des applications pour la conjecture de Baum-Connes, \'etudi\'ees \`a la fin du papier. 

\subsubsection*{Going-Down functors and the Künneth-formula for crossed products by ample groupoids}

Transactions of the American Mathematical Society 372, no. 11 (2019): 8159-8194, avec Christian Bönicke.\\

En topologie alg\'ebrique commutative, la formule de K\"unneth donne une mani\`ere de calculer la cohomologie d'un produit $H^*(X\times Y)$ en utilisant $H^*(X)$ et $H^*(Y)$. En passant au non-commutatif, la formule de K\"unneth \'etudie le calcul de $K_*(A \otimes B)$ en fonction de $K_*(A)$ et $K_*(B)$ ($\otimes$ est ici le produit tensoriel spatial). Plus pr\'ecisement, on dit que $A$ satisfait la formule de K\"unneth si une certaine application $\alpha_{A,B}: K(A)\otimes K(B) \rightarrow K(A\otimes B)$ est un isomorphisme pour toute $C^*$-alg\`ebre $B$ dont les groupes de $K$-th\'eorie sont libres ab\'eliens. Cette formule est v\'erifi\'ee pour une classe de $C^*$-alg\`ebres appel\'ees \textit{bootstrap} \cite{rosenberg1987kunneth}, et aussi pour les $C^*$-alg\'ebres de groupes moyennables (et m\^eme a-T-moyennables) \cite{BaumConnesHigson}\cite{TuThese}. \\

Dans cet article, nous montrons la formule de K\"unneth pour des $C^*$-alg\`ebre de groupo\"ides amples qui satisfont la conjecture de Baum-Connes et dont tous les sous-groupo\"ides compact ouverts v\'erifient la formule de K\"unneth. Ce travail est une g\'en\'eralisation du principe de restriction d\'evelopp\'e par Chabert, Echterhoff et Oyono-Oyono \cite{ChabertEOY}. La premi\`ere motivation derri\`ere ce travail est que de nombreuses $C^*$-alg\`ebres peuvent se r\'ealiser comme des $C^*$-alg\`ebres de groupo\"ides amples qui satisfont cette condition. Nous donnons pour finir des applications en g\'eom\'etrie m\'etrique. La formule de K\"unneth est d\'emontr\'ee pour les alg\`ebres de Roe uniformes d'espaces se plongeant uniform\'ement dans un Hilbert, pour l'alg\`ebre de Roe maximale d'espaces admettant une plongement uniforme fibr\'e dans un Hilbert et pour l'alg\`ebre de Roe d'un espace n'admettant pas de tel plongement. (L'int\'er\^et du dernier exemple est de ne pas \^etre atteignable par les m\'ethodes d\'ej\`a existantes dans la litt\'erature.)  

\subsubsection*{A K\"{u}nneth formula for \'etale groupoids} Preprint disponible sur ma webpage personnelle.\\

Cet article est une version pr\'eliminaire de l'article ci-dessus. En discutant avec Christian B\"onicke, nous avons r\'ealis\'e que certains de nos r\'esultats se recoupaient et avons d\'ecid\'e de collaborer. L'article est cependant diff\'erent. Le foncteur d'induction est par exemple d\'efinit d'une fa\c{c}on plus alg\'ebrique, et l'on introduit une propri\'et\'e pour les groupo\"ides \'etales (d'\^etre localement compactement induit) qui m\'eriterait d'\^etre \'etudi\'ee.

\subsubsection*{Controlled $K$-theory for groupoids and applications to Coarse Geometry} 

Journal of Functional Analysis,Volume 275, Issue 7, October 2018, Pages 1756-1807.\\	 

Cet article constitue mon travail de th\`ese o\`u est construite une version de la $K$-th\'eorie op\'eratorielle qui prend en compte les effets de propagation dans les groupo\"ides, les espaces m\'etriques et les groupes quantiques. Ceci g\'en\'eralise les travaux effectu\'es par Herv\'e Oyono-Oyono et Guoliang Yu sur les espaces m\'etriques (voir \cite{OY1}\cite{OY2}\cite{OY3}\cite{oyono2019quantitative}). \\

Une fois la $K$-th\'eorie contr\^ol\'ee d\'efinie et ses propri\'et\'es d\'emontr\'ees (elle approxime la $K$-th\'eorie usuelle, comporte beaucoup de suites exactes, etc), nous construisons des applications d'assemblage \`a valeurs dans la $K$-th\'eorie contr\^ol\'ee, et \'enon\c{c}ons une conjecture de Baum-Connes contr\^ol\'ee. Les diff\'erentes versions de cette conjecture (pour les groupes, les groupo\"ides, les espaces m\'etriques, version contr\^ol\'ee ou non) sont compar\'ees. La fin de l'article est consacr\'ee \`a des applications en g\'eom\'etrie m\'etrique.  

\subsubsection*{S\'eminaires et vulgarisation}

Le s\'eminaire de g\'eom\'etrie non-commutative est anim\'e par Erik Guentner, Rufus Willett et moi-m\^eme. Nous d\'ecidons en g\'en\'eral d'un ou plusieurs th\`emes au d\'ebut de semestre et donnons ensuite des expos\'es toutes les semaines \`a un groupe de 5 \`a 10 \'etudiants en th\`ese, ainsi qu'\`a quelques professeurs. Les th\`emes trait\'es pour l'instant sont les suivants:
\begin{itemize}
\item[$\bullet$] les $C^*$-alg\`ebres et les sous-alg\`ebres de Cartan;
\item[$\bullet$] la propri\'et\'e T;
\item[$\bullet$] la moyennabilit\'e des groupes;
\item[$\bullet$] $C^*$-simplicit\'e des groupes discrets;
\item[$\bullet$] quasi-localit\'e.\\
\end{itemize}

Je me consacre aussi \`a d'autres s\'eminaires. J'ai notamment donn\'e des expos\'es aux s\'eminaires suivants.\\

\begin{itemize}
\item[$\bullet$] \textit{Topological Quantum Field Theories}: le s\'eminaire avant au d\'epart pour but d'\'etudier les repr\'esentations des alg\`ebres de Hopf ainsi que leur apparition dans la construction de TQFT en dimenion 2. Les TQFT sont des formulations alg\'ebriques de certaines int\'egrales (de Feynman) utilis\'ees par les physiciens en th\'eorie quantique des champs. Bien que math\'ematiquement mal d\'efinies dans certains cas, les calculs donnent des r\'esultats qui concordent avec l'exp\'erience \`a un degr\'es de pr\'ecision tr\`es fin. Certains math\'ematiciens ont propos\'e d'interpr\'eter la valeur de certaines de ces int\'egrales comme la valeur d'un foncteur $Z : Bord_n \rightarrow C$ entre $n$-cat\'egories.  J'ai pr\'esent\'e la classification des TQFT en dimension 2 et 3, une introduction \`a l'hyoth\`ese du cobordisme et \`a sa r\'esolution par Lurie \cite{Lurie}, ainsi que la reformulation dans ce contexte du polyn\^ome de Jones obtenue par Witten dans \cite{WittenJones}.\\

\item[$\bullet$] \textit{Right-Angled Artin Groups}: ce s\'eminaire se veut une introduction \`a la th\'eorie g\'eom\'etrique des groupes, en prenant comme exemple les groupes d'Artin \`a angle droit (RAAGs) et les mapping class groups. J'y ai donn\'e un expos\'e sur le th\'eor\`eme affirmant que tout sous-groupe d'un RAAG qui est 2-g\'en\'er\'e est soit ab\'elien soit libre (le s\'eminaire vient de d\'ebuter...).\\

\item[$\bullet$] \textit{S\'eminaire d'Analyse}. J'y ai donn\'e un expos\'e introduisant \`a la g\'eom\'etrie non-commutative.
\end{itemize}

%%%%%%%%%%%%%%%%%%%%%%%%%%%%%%%%%%%
\subsection*{Projets de recherche}
%%%%%%%%%%%%%%%%%%%%%%%%%%%%%%%%%%%

Voici quelques th\`emes de recherche que je souhaiterais \'etudier dans les ann\'ees qui suivent. Je suis bien s\^ur ouvert \`a d'autres sujets si de futurs coll\`egues ou collaborateurs ont des suggestions.\\

\begin{itemize} 
\item[$\bullet$] Inspir\'es par la notion de \textit{dimension asymptotique} (Gromov, \cite{gromov1993asymptotic}) en th\'eorie g\'eom\'etrique des groupes, Guentner,  Willett et Yu d\'efinissent \cite{Guentner:2014aa} une dimension asymptotique dynamique, utile pour les actions topologiques de groupes et les espaces m\'etriques par exemple. Cette notion a d\'ej\`a \'et\'e appliqu\'ee en th\'eorie des $C^*$-alg\`ebres \cite{deeley2018nuclear} et en topologie des vari\'et\'es \cite{bartels2019farrell}. La d\'efinition est donn\'ee de mani\`ere g\'en\'erale pour tout groupo\"ide \'etale, et peut \^etre relax\'ee en \textit{complexit\'e finie}: en simplifiant, les groupo\"ides dont l'adh\'erence est compacte sont ``de complexit\'e nulle"; et un groupo\"ide est dit de complexit\'e finie si il peut \^etre localement d\'ecoup\'e en deux parties, et ce un nombre fini de fois, qui sont elles m\^emes de complexit\'e nulle. Dans le cas des actions de groupes (associ\'ees \`a un certain type de groupo\"ides), la conjecture de Baum-Connes a \'et\'e d\'emontr\'ee par Guentner, Willet et Yu \cite{GWY2}.\\

\textbf{Projet:} D\'emontrer la conjecture de Baum-Connes pour les groupo\"ides \'etales de \textit{dimension asymptotique finie} (plus g\'en\'eralement de \textit{complexit\'e finie}).\\

Ce projet a plusieurs motivations. La premi\`ere est de donner une preuve plus \'el\'ementaire des travaux de Tu \cite{TuThese} sur les groupo\"ides moyennables, notamment pour le th\'eor\`eme des coefficients universels (UCT) pour les $C^*$-alg\`ebres de groupo\"ides simples, nucl\'eaires, s\'eparables et unitales. Ce r\'esultat a r\'ecement pris de l'importance dans le programme de classification des $C^*$-alg\'ebres d'Elliot; pourtant, la preuve est tr\`es dense et est comprise par peu de personnes. Ce projet devrait aussi aboutir \`a une meilleure compr\'ehension de la $K$-th\'eorie des $C^*$-alg\'ebres. Ultimement, il pourrait aboutir \`a de nouveaux r\'esultats sur la conjecture de Baum-Connes, en d\'efinissant une classe de groupo\"ides de mani\`ere inductive comme pour la complexit\'e finie, la complexit\'e nulle \'etant d\'efinie cette fois non pas comme ``d'adh\'erence compacte" mais par une classe de groupo\"ides satisfaisant la conjecture de Baum-Connes (moyennables par exemple, ou plus g\'en\'eralement a-T-moyennables).\\

La strat\'egie est ici claire pour la preuve: la classe de complexit\'e nulle satisfaisant l'\'enonc\'e du th\'eor\`eme, il suffit de d\'emontrer que la conjecture est stable pour des d\'ecompositions de groupo\"ides. Essentiellement, cela a d\'ej\`a \'et\'e fait par l'auteur et B\"onicke pour la formule de K\"unneth dans \cite{BonickeDellAiera}. Appliquer la m\^eme id\'ee \`a Baum-Connes est naturel, m\^eme si des difficult\'es techniques devront \^etre surmont\'ees.\\

\item[$\bullet$] Les \textit{conjectures de Matui} concernent la comparaison entre l'homologie et la $K$-th\'eorie de certains groupo\"ides topologiques \cite{matui2016topological}. Plus pr\'ecis\'ement, la conjecture HK pr\'evoit que si $G$ est un groupo\"ide \'etale essentiellement principal, minimal, dont l'espace des unit\'es est un Cantor, alors
\[\bigoplus_{k\geq 0} H_{2k+i}(G) \cong K_i(C^*_r(G))\quad \forall i\in \{0,1\}.\]
Dans le cas d'actions de groupe topologiques, ces hypoth\`eses correspondent \`a des actions par hom\'eomorphismes de groupes discrets sur des espaces de Cantor. Un r\'ecent contre-exemple \cite{Scarparo2020} fait appara\^itre l'importance de l'isotropie comme obstruction. \\

\textbf{Projet:} \'Etudier la conjecture dans le cas des groupo\"ides associ\'es \`a des espaces m\'etriques. \\

Avec B\"onicke, nous avons d\'ej\`a effectu\'e des calculs sur des cas particuliers (des \textit{box-spaces} associ\'es \`a des groupes moyennables r\'esiduellement finis, comme $\mathbb Z$, ou le groupe de Grigorchuk). La motivation derri\`ere ce projet serait de donner des exemples ou contre-exemples \`a la conjecture. Dans le cas o\`u elle est valide, cela donnerait des exemples de calculs explicites de groupes d'\textit{homologie coarse} \cite{NowakYu}, une th\'eorie homologique pour les espaces m\'etriques notoirement difficile \`a calculer.\\

%Je souhaiterais \'etudier ces conjectures dans le cas des groupo\"ides coarses et ceux associ\'es \`a des actions. \\ 

\item[$\bullet$] Le travail effectu\'e dans \cite{DellWillett} par l'auteur et Willett se situe dans un courant qui vise \`a \'eclaircir les ph\'enom\`enes qui emp\^echent la r\'ealisation de la conjecture de Baum-Connes \textit{coarse} (le cas m\'etrique donc) ou plus g\'en\'eralement pour les groupo\"ides (voir \cite{HigsonLaffSk} et \cite{WillettYu} par exemple). Pr\'ecisons, pour le lecteur pas tr\`es convaincu par l'int\'er\^et des groupo\"ides, qu'ils englobent le cas de la conjecture \`a coefficients, qui, elle, est connue pour avoir des contre-exemples. Grossi\`erement, la conjecture ne peut pas se r\'ealiser en pr\'esence de certaines variantes de la propri\'et\'e T pour les groupo\"ides. \\ 

Dans ce cas, nous pouvons montrer l'existence d'un projecteur de Kazdhan, aux propri\'et\'e si exotiques que, moyennant des renforcements \`a la propr\'et\'e T, ne peut pas se trouver dans l'image de l'application d'assemblage de Baum-Connes.\\

\textbf{Projet:} D\'ecrire avec plus de pr\'ecision les liens entre la propri\'et\'e T dynamique et l'obstruction \`a la conjecture de Baum-Connes.\\

Il semble probable de pouvoir prouver que la propri\'et\'e T dynamique constitue, moyennant des conditions suppl\'ementaires, une obstruction \`a l'exactitude en $K$-th\'eorie de certaines $C^*$-alg\`ebres, comme dans les exemples de Skandalis \cite{skandalis1988notion}. Des calculs d\'ej\`a effectu\'es sur des classes d'exemples confirment cette affirmation.\\

\item[$\bullet$] L'application de la $K$-th\'eorie contr\^ol\'ee aux groupes quantiques \'etait assez impr\'evue au moment de la r\'edaction de ma th\`ese. Elle fait ressortir la filtration naturelle d'un groupe quantique compact donn\'ee par ses repr\'esentations de dimension finie comme une donn\'ee g\'eom\'etrique. De l\`a, une question naturelle se pose: peut-on adapter des techniques de th\'eorie g\'eom\'etrique des groupes aux groupes quantiques? Par exemple, la d\'efinition de \textit{dimension asymptotique} pour un groupe quantique ou encore celle de son alg\`ebre de Roe sont des probl\`emes int\'eressants et non-triviaux. Pour illustrer cela, dans le cas d'un groupe discret $G$, il est bien connu que son alg\`ebre de Roe uniforme $C^*_u(G)$ est isomorphe \`a $l^\infty(G)\rtimes_r G$, et qu'elle est nucl\'eaire si $G$ est de dimension asymptotique finie. La m\^eme d\'efinition est bien trop na\"ive pour un groupe quantique discret: $l^\infty(G)$ est bien d\'efinie et est munie d'une action de $G$, permettant de former le produit crois\'e. Toutefois, dans les cas int\'eressants (r\'eellement non-commutatifs), $l^\infty(G)$ n'est m\^eme pas une $C^*$-alg\`ebre nucl\'eaire.  \\

\textbf{Projet:} D\'evelopper des techniques s'inspirant de th\'eorie g\'eom\'etrique des groupes pour les groupes quantiques.\\

D'autres questions peuvent \^etre abord\'ees pour les groupes quantiques, par exemple construire des ``bons'' espaces ou $C^*$-alg\`ebres sur lesquels ils agissent. Je pense notamment \`a un classifiant des actions propres, qui permettrait une d\'efinition constructive de l'application d'assemblage de Baum-Connes dans ce cadre (actuellement, elle est d\'efinie comme localisation d'un certain foncteur de cat\'egories triangul\'ees \cite{MeyerNest}). On peut aussi penser \`a la fronti\`ere de Furstenberg qui permet d'\'etudier la $C^*$-simplicit\'e de la $C^*$-alg\`ebre de groupes discrets \cite{kalantar2014boundaries}. Ces derni\`eres questions semblent toutefois tr\`es difficiles.\\ 

%Je souhaiterais pousser l'\'etude de ces th\`emes plus loin. \\

%\item[$\bullet$] De mani\`ere autonome, je me forme en th\'eorie g\'eom\'etrique des groupes et en physique math\'ematique.  

\end{itemize}
\newpage
\bibliographystyle{plain}
\bibliography{biblio} 
\end{document}






















