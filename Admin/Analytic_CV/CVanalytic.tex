
\documentclass[a4paper,11pt]{article} 
\usepackage[frenchb]{babel}
%\usepackage[T1]{fontenc} 
\usepackage[utf8]{inputenc}      
\usepackage{url}                
\setlength\parindent{0pt}

\pagestyle{empty}             
\usepackage{vmargin}           
\setmarginsrb{3cm}{3cm}{3cm}{3cm}{0cm}{0cm}{0cm}{0cm}

% Marge gauche, haute, droite, basse; espace entre la marge et le texte à
% gauche, en  haut, à droite, en bas

% Pour laisser de l'espace entre les lignes du tableau
\newcommand\espace{\vrule height 20pt width 0pt}

% Pour mes grands titres
\newcommand{\titre}[1]{%
	\begin{center}
	\bigskip
	\rule{\textwidth}{1pt}
	\par\vspace{0.1cm}
        \textbf{\large #1}
	\par\rule{\textwidth}{1pt}
	\end{center}
	\bigskip
	}

\begin{document}

\begin{flushleft}
Clément Dell'Aiera \\
Department of Mathematics, University of Hawaii\\
2565 McCarthy Mall, Keller 401A \\
Honolulu HI 96\ 822 \\

\medskip
%Tél.: 06 74 62 99 52

E-mail: dellaiera.clement@gmail.com


\end{flushleft}
\begin{flushleft}
Nationalit\'e : Fran\c{c}aise \\
Date de naissance : 22/03/1990 \`{a} Metz (Moselle).
\end{flushleft}

\vspace{1.5cm}
\begin{center}
\par\huge{\textbf{Curriculum Vit\ae} }
\end{center}

%%%%%%%%%%%%%%%%%%
\titre{\'{E}ducation et emploi}
%#############%%%%

\begin{tabular}{cp{0.8\textwidth}}

\textbf{Ao\^{u}t 2017--pr\'{e}sent} &  \textbf{Assistant Professor} at U.H. Manoa (non-tenure track)  \\
						& Department of Mathematics\\
						&  Rank I3-M09 \\
						%& \textbf{Enseignement} \\
						%& Spring 2019: Math 307, Linear Algebra \& Differential Equations\\  
						%& Fall 2017: Math 203, Spring \& Fall 2018: Math 307\\
						\espace
\textbf{2014-2017} &  \textbf{PhD student} sous la supervision de Hervé Oyono-Oyono et la  \\
						& co-direction d'Andrzej Zuk, titre: \\	
						& \textit{``Controlled K-theory for groupoids and applications"} \\
\espace
\textbf{2010--2014} &  \textbf{ENS Cachan} (Antenne de Bretagne) \\
				    & 	Ecole Normale Supérieure, D\'epartement de Math\'ematiques \\
                              & \textbf{ENSAE Paristech}\\
				&	Paris Graduate School of Economics, Statistics and Finance\\
                                   & \textbf{Master en Math\'ematiques Fondamentales}\\  & Universit\'e Paris~VII-Diderot \\
                                   & \textbf{Agrégation externe de Mathématiques} (2013) \\ 
							& Rang $41^e$ \\
\espace

\espace
\textbf{2007--2010} &\textbf{Classes préparatoires MP$^*$ } \\
					& Nancy, Lycée Henri Poincaré\\

\espace
\textbf{2007} & \textbf{Baccalauréat} (série S, sp\'ecialit\'e math\'ematiques) 
 \\

\end{tabular}

\newpage
%%%%%%%%%%%%%%%%%%%%%%%%%%
\titre{Activit\'es de recherche}
%#########################

\textbf{Liste de publications:} 
\begin{enumerate}
\item \textit{Topological property T for groupoids}, avec Rufus Willett. Preprint sur arxiv, novembre 2018. Soumis pour publication.
\item \textit{Going-Down functors and the Künneth-formula for crossed products by ample groupoids}, avec Christian Bönicke. Preprint sur arxiv, octobre 2018. Soumis pour publication.
\item \textit{A K\"{u}nneth formula for \'etale groupoids}, preprint disponible sur ma webpage personnelle.
\item \textit{Controlled $K$-theory for groupoids and applications to Coarse Geometry}, Journal of Functional Analysis,Volume 275, Issue 7, October 2018, Pages 1756-1807. 
\end{enumerate}

\espace

\textbf{S\'eminaires:} \\

\begin{itemize}
\item[$\bullet$] 2017-pr\'esent : co-organisateur, avec Erik Guentner et Rufus Willett, du Noncommutative Geometry Seminar of the department of Mathematics of the University of Hawaii at Manoa. Je maintiens une page personelle o\`u sont disponibles des notes que je r\'edige en TeX ainsi que le planning.\\

\item[$\bullet$] Spring 2019: participant au s\'eminaire de \textit{Geometric Group Theory}, organis\'e par Andrew Sale, sur le th\`eme \textit{Right-Angled Artin Groups}.  \\
\item[$\bullet$] Fall 2018 \& Spring 2019: co-organisateur, avec Piper Harron, Sarah Post et Andrew Sale, du \textit{Colloquium de math\'ematiques}, un rendez-vous bi-mensuel ou mensuel \`a destination de tout le d\'epartement. Le but est de pr\'evoir des expos\'es (suivis de rafra\^{i}chissements) accessibles aux math\'ematiciens de tout bords, professeurs, postdoctorants ainsi que th\'esards.\\
\item[$\bullet$] Spring 2018: participant au s\'eminaire \textit{Topological Quantum Field Theory}, organis\'e par Sarah Post. J'ai redig\'e des notes des expos\'es que j'y ai donn\'e, disponibles sur ma page personnelle. \\
\item[$\bullet$] 2016-2017 : co-organisateur avec Matthieu Brachet, du \textit{S\'eminaire doctorants} en math\'ematiques de l'IECL, Metz-Nancy.
\end{itemize}

\espace

\textbf{Responsabilit\'es administratives:} \\
\begin{itemize}
\item[$\bullet$] ... \\
\item[$\bullet$] 2016-2017 : Représentant du personnel au conseil du Laboratoire de l'IECL, Collège C.\\
\end{itemize}

\newpage
\titre{Research talks }

%%%%%%%%%%%%%%%%%%%%%%%%%%%%%%
\begin{tabular}{cp{0.8\textwidth}}

\textbf{2019} & \textbf{Noncommutative Geometry Festival 2019}\\	
				& Pr\'evu 29/04-03/05, Washington University, Saint-Louis \\ %https://sites.google.com/site/ncgwustl/program \\
				& \textbf{Decomposition complexity, a dynamical approach}\\
				& 15/02, Analysis Seminar, University of Houston. \\ %https://www.math.uh.edu/analysis/
				& \textbf{Decomposition complexity, a dynamical approach}\\
				& 13/02, Noncommutative Geometry Seminar, Texas A\&M.  \\ % http://www.math.tamu.edu/seminars/noncomgeom/
				\espace
\textbf{2018} & \textbf{Property T for topological groupoids}\\
				& 07/11, Noncommutative Geometry Seminar, Texas A\&M. \\ % http://www.math.tamu.edu/seminars/noncomgeom/
				& \textbf{Dynamical Property T}\\
				& 05/11, Noncommutative Geometry Seminar, University of Houston. \\ %https://www.math.uh.edu/analysis/Fall18.php
				& \textbf{Dynamical Property T}\\
				& 20/09, Noncommutative geometry Seminar, PennState University.\\
				& \textbf{C*-alg\`ebres g\'eom\'etriques et applications en g\'eom\'etrie coarse}\\
				& Juin, S\'eminaire d'AO, Paris Diderot.\\
				& \textbf{Geometric C*-algebras: applications to the K\"unneth formula}\\
				& Mai, GPOTS 2018, Miami University.\\
				& \textbf{Geometric C*-algebras and Coarse structures}\\
				& F\'evrier, Workshop on computability of K-theory for C*-algebras, Texas A\&M University.\\
				\espace	
\textbf{2017} & \textbf{Principe de restriction pour les groupoïdes étales. Application à une formule de Künneth pour leurs produits croisés}\\
				& Juin, Séminaire d'Algèbres d'Opérateurs, Paris-Diderot (Paris 7).\\
				\espace	
\textbf{2016} & \textbf{Controlled K-theory for groupoids \& applications to Coarse Geometry}\\
				& D\'ecembre, Kleines Seminar, Münster.\\
				& \textbf{K-théorie quantitative et applications}\\
				& D\'ecembre, Arbre de Noël du GDR Géométrie Non-commutative, Albi.\\
				& \textbf{Asymptotic dimension for étale groupoids} \\
				& Mai, Noncommutative Geometry Seminar, IECL, Metz-Nancy.\\
				\espace	
\textbf{2015} & \textbf{Controlled $K$-theory for groupoids}\\
				& D\'ecembre, Arbre de Noël du GDR Géométrie Non-Commutative, Montpellier.\\
\end{tabular}

\vfill
\textit{Note: Ne sont pas comptabilis\'es ici les expos\'es donn\'es aux diff\'erents s\'eminaires locaux auxquels je participe. Leur liste est disponible sur la page du s\'eminaire ici.}
%%%%%%%%%%%%%%%%%%%%%%%%%%%%%%
\newpage
%%%%%%%%%%%%%%%%%%%%%
%%%%%%%%%%%%%%%%%%%%%%%%%%%%
\titre{Enseignements}
%%%%%%%%%%%%%%%%%%%%%%%%%%%%
\begin{tabular}{cp{0.8\textwidth}}

\textbf{ 2019 } 	& \textbf{University of Hawaii at Manoa}\\
					& Spring, Math 307: Linear Algebra and Differential Equations.\\
					& \textit{Cours magistraux, r\'edactions et corrections des devoirs et examens}\\
\espace
\textbf{ 2018 } 	& \textbf{University of Hawaii at Manoa}\\
					& Fall, Math 307: Linear Algebra and Differential Equations.\\
					& Spring, Math 307: Linear Algebra and Differential Equations.\\
					& \textit{Cours magistraux, r\'edactions et corrections des devoirs et examens}\\
\espace
\textbf{ 2017 } 	& \textbf{University of Hawaii at Manoa}\\
					& Fall, Math 203: Calculus for Business and the Social Sciences.\\
					& \textit{Cours magistraux, r\'edactions et corrections des devoirs et examens}\\
\espace
\textbf{ 2016 } 	& \textbf{Universit\'e de Lorraine}\\
					&	Master 2, le\c{c}ons d'Analyse pour l'Agrégation, TP de Mod\'elisation Statistiques et Probabilit\'es.\\
					&	Master 1,	charg\'e de TD pour le cours Processus Statistiques et Probabilit\'e, TP pour Statistiques et S\'eries Temporelles. \\
					&	L2 et L3: TD et kh\^{o}lles. \\
\espace
\textbf{ 2015 } 	& \textbf{Universit\'e de Lorraine}\\
			 		& Master 1, Charg\'e de TD pour Processus Stochastiques et Probabilit\'e, TP pour Statistiques et S\'eries Temporelles.\\
					& L2 et L3: Charg\'e de TD d'alg\`ebre et d'analyse. TP pour Calcul Formel.\\ 
					& Aide \`a l'encadrement d'un \'etudiant de L3 pour un stage sous la supervision J-L. Tu.\\   
\espace		
\textbf{ 2014 } 	& \textbf{Universit\'e de Lorraine}\\
     					& Master 1: Charg\'e de TD pour Processus Stochastiques et Probabilit\'e, TP pour Statistiques et S\'eries Temporelles.\\ 
					& L2 et L3: Charg\'e de TD d'alg\`ebre et d'analyse. Introduction \`a LateX.\\
%\end{tabular}
%\begin{tabular}{cp{0.8\textwidth}}
\espace
 \textbf{ 2013 }   & \textbf{Lycée Sainte Marie de Neuilly} \\
					& K\^{o}lles de mathematics en classes préparatoires BL.\\
\espace
\textbf{ 2011 } 	& \textbf{University Paris~II Assas}\\
					& Charg\'e de TD en Analyse de Donn\'ees, Licence 3 Eco-gestion.\\
					& \textbf{Lycée Sainte Marie de Neuilly} \\
					& K\^{o}lles de math\'ematiques en classes préparatoires BL. \\
\end{tabular}

\newpage
%%%%%%%%%%%%%%%%%%%%%%%%%%
\titre{Autre}
%###################%%%%%%
\textbf{Expos\'es \`a destination de non-sp\'ecialistes:}\\

\begin{itemize}
\item[$\bullet$] F\'evrier 2018, ``From a notion of dynamical dimension to cutting and pasting algebras", Analysis Seminar, University of Hawaii\\
\item[$\bullet$] D\'ecembre 2017, ``Expanseurs", Noel des Doctorants, IECL, Metz- Nancy\\
\item[$\bullet$] Octobre 2016, ``Expanseurs", S\'eminaire Landau, IRMAR, Rennes \\ %: From Network reliability to K-theory", Landau Seminar, IRMAR, Rennes
\item[$\bullet$] F\'evrier 2016, ``Applications des groupo\"{i}des en Physique", S\'eminaire Doctorants, IECL, Metz-Nancy\\
\item[$\bullet$] Octobre 2015, ``The Novikov conjecture for groups with finite asymptotic dimension", Henri Lebesgue Workshop, Nantes\\
\item[$\bullet$] Octobre 2015, ``Propagation in $K$-theory", S\'eminaire Doctorants, IECL, Metz-Nancy\\
\item[$\bullet$] Janvier 2015, ``Introduction aux groupo\"{i}des et \`a leurs $C^*$-alg\`ebres", S\'eminaire Doctorants, IECL, Metz-Nancy\\
\end{itemize}

%\begin{itemize}
\espace
\textbf{M\'emoire de Master} sous la supervision de Hervé Oyono-Oyono, \textit{Six term exact sequences in $K$-theory for crossed products of $C^*$-algebras by Z}.\\

\vfill
\textbf{Math\'ematiques appliqu\'ees} Lors de ma scolarit\'e \`a l'ENSAE Paristech:\\
\medskip
%\item[$\bullet$] \textbf{Internships}: \\
%During my education at ENSAE Paristech : \\
\espace
\begin{tabular}{cp{0.8\textwidth}}
\textbf{2014} & M\'emoire de fin d'\'etude sous la supervision of Jérémy Jakubowicz, \textit{Information geometry and deep learning}. Application de m\'ethode de deep learning et de g\'eom\'etrie de l'information \` la reconnaissance de caract\`res manuscrits. Impl\`mentation de r\'eseaux de neurones utilisant notament une descente de gradient riemannienne.\\
\espace
\textbf{2012}&  Stage de $10$ semaines sous la supervision de Cristina Butucea sur le th\`me \textit{Statistiques appliqu\'ees \`a l'optique quantique} \`a l'University de Marne-la-Vallée (LAMA).		\\
\end{tabular}
\\

%\medskip
%\end{itemize}

%%%%%%%%%%%%%%%%%%%%%%%%%%%%%%%%%%%%%%
\titre{Languages and computer skills}
%#########################%%%%%%%%%%%%

\begin{itemize} 
\medskip
\item[$\bullet$] \textbf{English} Fluent. TOEIC 915/990.
\medskip
\item[$\bullet$] \textbf{Spanish} Oral.
\medskip
\item[$\bullet$] \textbf{Softwares} Pack Office, LateX, Python, Scilab, R, Objective Caml, C/C++, html.
\end{itemize}

\newpage

\titre{Description d\'etaill\'ee des travaux et projets de recherches}

Cette partie vise \`a d\'etailler les travaux d\'ej\`a accomplis ainsi que futurs. Le texte se divise en trois parties. La premi\`ere pr\'esente le domaine d'inter\^et du candidat, d'une mani\`ere g\'en\'erale, afin d'introduire les objets et les questions sur lesquelles nous travaillons. La seconde partie d\'etaille les travaux publi\'es ainsi que les sujets d\'ej\`a accomplis. La derni\`ere partie se consacre \`a des questions ou des projets qui se veulent des ouvertures vers des travaux futurs. La r\'edaction, qui d\'ebute sans assumer quelconque connaissance en g\'eometrie non-commutative et alg\`ebres d'op\'erateurs, exige de fa\c{c}on croissante une connaissance de plus en plus pouss\'ee de ces domaines. Ainsi, la lectrice curieuse de nos activit\'es peut se limiter \`a la premi\`ere section, l\`a o\`u la sp\'ecialiste peut sans probl\`eme directement commencer par la partie suivante.\\

\end{document}
