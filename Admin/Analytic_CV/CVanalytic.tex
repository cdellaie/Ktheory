
\documentclass[a4paper,11pt]{article} 
\usepackage[frenchb]{babel}
%\usepackage[T1]{fontenc} 
\usepackage[utf8]{inputenc}      
\usepackage{url}                


\pagestyle{empty}             
\usepackage{vmargin}           
\setmarginsrb{3cm}{3cm}{3cm}{3cm}{0cm}{0cm}{0cm}{0cm}

% Marge gauche, haute, droite, basse; espace entre la marge et le texte à
% gauche, en  haut, à droite, en bas

% Pour laisser de l'espace entre les lignes du tableau
\newcommand\espace{\vrule height 20pt width 0pt}

% Pour mes grands titres
\newcommand{\titre}[1]{%
	\begin{center}
	\bigskip
	\rule{\textwidth}{1pt}
	\par\vspace{0.1cm}
        \textbf{\large #1}
	\par\rule{\textwidth}{1pt}
	\end{center}
	\bigskip
	}

\begin{document}

\begin{flushleft}
Clément Dell'Aiera \\
Department of Mathematics, University of Hawaii\\
2565 McCarthy Mall, Keller 401A \\
Honolulu HI 96\ 822 \\

\medskip
%Tél.: 06 74 62 99 52

E-mail: dellaiera.clement@gmail.com


\end{flushleft}
\begin{flushleft}
Nationalit\'e : Fran\c{c}aise \\
Date de naissance : 22/03/1990 \`{a} Metz (Moselle).
\end{flushleft}

\vspace{1.5cm}
\begin{center}
\par\huge{\textbf{Curriculum Vit\ae} }
\end{center}

%%%%%%%%%%%%%%%%%%
\titre{\'{E}ducation et emploi}
%#############%%%%

\begin{tabular}{cp{0.8\textwidth}}

\textbf{Ao\^{u}t 2017--pr\'{e}sent} &  \textbf{Assistant Professor} at U.H. Manoa (non-tenure track)  \\
						& Department of Mathematics\\
						& Rank I3-M09. \\
						& \textbf{Enseignement} \\
						& Spring 2019: Math 307, Linear Algebra \& Differential Equations\\  
						& Fall 2017: Math 203, Spring \& Fall 2018: Math 307\\
						\espace
\textbf{2014-2017} &  \textbf{PhD student} sous la supervision de Hervé Oyono-Oyono et la  \\
						& codirection de Andrzej Zuk, titre: \\	
						& \textit{``Controlled K-theory for groupoids and applications"} \\
\espace
\textbf{2010--2014} &  \textbf{ENS Cachan} (Antenne de Bretagne) \\
				    & 	Ecole Normale Supérieure, D\'epartement de Math\'ematiques \\
                              & \textbf{ENSAE Paristech}\\
				&	Paris Graduate School of Economics, Statistics and Finance\\
                                   & \textbf{Master en Math\'ematiques Fondamentales}\\  & University Paris~VII-Diderot. \\
                                   & \textbf{Agrégation externe de Mathématiques} (2013) \\ 
							& rang $41^e$. \\
\espace

\espace
\textbf{2007--2010} &\textbf{Classes préparatoires MP$^*$ } \\
					& Nancy, Lycée Henri Poincaré\\

\espace
\textbf{2007} & \textbf{Baccalauréat} (série S, sp\'ecialit\'e math\'ematiques) 
 \\

\end{tabular}

\newpage
%%%%%%%%%%%%%%%%%%%%%%%%%%
\titre{Activit\'es de recherche}
%#########################

\textbf{Liste de publications:} 
\begin{enumerate}
\item \textit{Topological property T for groupoids}, avec Rufus Willett. Preprint sur arxiv, novembre 2018. Soumis pour publication.
\item \textit{Going-Down functors and the Künneth-formula for crossed products by ample groupoids}, avec Christian Bönicke. Preprint sur arxiv, octobre 2018. Soumis pour publication.
\item \textit{A K\"{u}nneth formula for \'etale groupoids}, preprint disponible sur ma webpage personnelle.
\item \textit{Controlled $K$-theory for groupoids and applications to Coarse Geometry}, Journal of Functional Analysis,Volume 275, Issue 7, October 2018, Pages 1756-1807. 
\end{enumerate}

\espace

\textbf{S\'eminaires:} \\

\begin{itemize}
\item[$\bullet$] 2017-pr\'esent : co-organisateur, avec Erik Guentner et Rufus Willett, le Noncommutative Geometry Seminar of the department of Mathematics of the University of Hawaii at Manoa. Je maintiens une page personelle o\`u sont disponibles des notes que je r\'edige en tex ainsi que le planning.\\
\item[$\bullet$] Spring 2019: participant au s\'eminaire de Geometric Group Theory, organis\'e par Andrew Sale, sur le th\`eme \textit{Right Angled Artin Groups}  \\
\item[$\bullet$] Fall 2018 \& Spring 2019: co-organisateur, avec Piper Harron, Sarah Post et Andrew Sale, du colloquiim de math\'ematiques, un rendez-vous bi-mensuel ou mensuel \`a destination de tout le d\'epartement. Le but est de pr\'evoir des expos\'es accessibles aux mathematiciens de tout bords, professeurs, postdoctorants ainsi que th\'esards.\\
\item[$\bullet$] 2016-2017 : co-organisateur avec Matthieu Brachet, du \textit{S\'eminaire doctorants} en math\'ematiques de l'IECL, Metz-Nancy.
\end{itemize}

\espace

\textbf{Responsabilit\'es administratives:} \\
\begin{itemize}
\item[$\bullet$] ... \\
\item[$\bullet$] 2016-2017 : Représentant du personnel au conseil du Laboratoire de l'IECL, Collège C.\\
\end{itemize}

\newpage
\titre{Research talks }

%%%%%%%%%%%%%%%%%%%%%%%%%%%%%%
\begin{tabular}{cp{0.8\textwidth}}

\textbf{2019} & \textbf{Noncommutative Geometry Festival 2019}\\	
				& Pr\'evu 29/04-03/05, Washington University, Saint-Louis \\ %https://sites.google.com/site/ncgwustl/program \\
				& \textbf{Decomposition complexity, a dynamical approach}\\
				& 15/02, Analysis Seminar, University of Houston. \\ %https://www.math.uh.edu/analysis/
				& \textbf{Decomposition complexity, a dynamical approach}\\
				& 13/02, Noncommutative Geometry Seminar, Texas A\&M.  \\ % http://www.math.tamu.edu/seminars/noncomgeom/
				\espace
\textbf{2018} & \textbf{Property T for topological groupoids}\\
				& 07/11, Noncommutative Geometry Seminar, Texas A\&M. \\ % http://www.math.tamu.edu/seminars/noncomgeom/
				& \textbf{Dynamical Property T}\\
				& 05/11, Noncommutative Geometry Seminar, University of Houston. \\ %https://www.math.uh.edu/analysis/Fall18.php
				& \textbf{Dynamical Property T}\\
				& 20/09, Noncommutative geometry Seminar, PennState University.\\
				& \textbf{C*-alg\`ebres g\'eom\'etriques et applications en g\'eom\'etrie coarse}\\
				& Juin, S\'eminaire d'AO, Paris Diderot.\\
				& \textbf{Geometric C*-algebras: applications to the K\"unneth formula}\\
				& Mai, GPOTS 2018, Miami University.\\
				& \textbf{Geometric C*-algebras and Coarse structures}\\
				& F\'evrier, Workshop on computability of K-theory for C*-algebras, Texas A\&M University.\\
				\espace	
\textbf{2017} & \textbf{Principe de restriction pour les groupoïdes étales. Application à une formule de Künneth pour leurs produits croisés}\\
				& Juin, Séminaire d'Algèbres d'Opérateurs, Paris-Diderot (Paris 7).\\
				\espace	
\textbf{2016} & \textbf{Controlled K-theory for groupoids \& applications to Coarse Geometry}\\
				& D\'ecembre, Kleines Seminar, Münster.\\
				& \textbf{K-théorie quantitative et applications}\\
				& D\'ecembre, Arbre de Noël du GDR Géométrie Non-commutative, Albi.\\
				& \textbf{Asymptotic dimension for étale groupoids} \\
				& Mai, Noncommutative Geometry Seminar, IECL, Metz-Nancy.\\
				\espace	
\textbf{2015} & \textbf{Controlled $K$-theory for groupoids}\\
				& D\'ecembre, Arbre de Noël du GDR Géométrie Non-Commutative, Montpellier.\\
\end{tabular}

\vfill
\textit{Note: Ne sont pas comptabilis\'es ici les expos\'es donn\'es aux diff\'erents s\'eminaires locaux auxquels je participe. Leur liste est disponible sur la page du s\'eminaire ici.}
%%%%%%%%%%%%%%%%%%%%%%%%%%%%%%
\newpage
%%%%%%%%%%%%%%%%%%%%%

\textbf{Talks directed to non-specialists :}\\

\begin{itemize}
\item[$\bullet$] February 2018, ``From a notion of dynamical dimension to cutting and pasting algebras", Analysis Seminar, University of Hawaii.
%\item[$\bullet$] December 2017, "Expanders", Young Researchers' Day, IECL, Metz- Nancy
\item[$\bullet$] October 2016, ``Expanders", Landau Seminar, IRMAR, Rennes %: From Network reliability to K-theory", Landau Seminar, IRMAR, Rennes
\item[$\bullet$] February 2016, ``Applications of groupoids to Physics", Young Researchers Seminar, IECL, Metz-Nancy
\item[$\bullet$] October 2015, ``The Novikov conjecture for groups with finite asymptotic dimension", Henri Lebesgue Workshop, Nantes
\item[$\bullet$] October 2015, ``Propagation in $K$-theory", Young Researchers Seminar, IECL, Metz-Nancy
\item[$\bullet$] January 2015, ``Introduction to groupoids $C^*$-algebras", Young Researchers Seminar, IECL, Metz-Nancy
\end{itemize}
\espace

%%%%%%%%%%%%%%%%%%%%%%%%%%%%
\titre{Teaching Experience}
%%%%%%%%%%%%%%%%%%%%%%%%%%%%

\begin{itemize}

\item[$\bullet$] \textbf{ Year 2019 :} University of Hawaii at Manoa.\\
					Spring, Math 307: Linear Algebra and Differential Equations.\\
					\textit{Duties : Lecturing, writing up Worksheets, Exams, and Finals. Grading.}\\

\item[$\bullet$] \textbf{ Year 2018 :} University of Hawaii at Manoa.\\
					Fall, Math 307: Linear Algebra and Differential Equations.\\
					Spring, Math 307: Linear Algebra and Differential Equations.\\
					\textit{Duties : Lecturing, writing up Worksheets, Exams, and Finals. Grading.}\\

\item[$\bullet$] \textbf{ Year 2017 :} University of Hawaii at Manoa.\\
					Fall, Math 203: Calculus for Business and the Social Sciences.\\
					\textit{Duties : Lecturing, writing up Worksheets, Exams, and Finals. Grading.}\\

\item[$\bullet$] \textbf{ Year 2016-2017 :} University of Lorraine.\\
					Teaching Assistant, Master 2. Preparatory Class for the Agrégation.\\
					\textit{Duties : Preparatory Class in Analysis, Computer sessions for "Statistics and Probability".}\\
					Teaching Assistant, Master 1.\\
					\textit{Duties : Exercices sessions for "Stochastic Processes and Probability", Computer sessions for "Statistics and Time Series".}\\
					Teaching Assistant, L2 et L3 Mathematics. \\
					\textit{Duties : Computer sessions for the class "Symbolic Computation".}\\   
\item[$\bullet$] \textbf{ Year 2015 :} University of Lorraine.\\
					Teaching Assistant, Master 1.\\
					\textit{Duties : Exercices sessions for "Stochastic Processes and Probability", Computer sessions for "Statistics and Time Series".}\\
					Teaching Assistant, L2 et L3 Mathematics. \\
					\textit{Duties : Exercices sessions in Analysis and in Algebra. Computer sessions for the class "Symbolic Computation". Tutor for an undergraduate student working under the supervision of Pr. J-L. Tu.}\\   
					
\item[$\bullet$] \textbf{ Year 2014 :}  University of Lorraine.\\
					Teaching Assistant, Master 1. \\
					\textit{Duties : Exercices sessions for "Stochastic Processes and Probability", Computer sessions for "Statistics and Time Series".}\\
					Teaching Assistant, L2 et L3 Mathematics. \\
					\textit{Duties : Exercices sessions in Analysis and in Algebra. Introduction to LateX.}\\
\item[$\bullet$] \textbf{ Year 2013 :} Lycée Sainte Marie de Neuilly, Teaching Assistant in Mathematics. \\
					\textit{Duties : Oral examinations of the ``Classes Préparatoires" BL} \\
\item[$\bullet$] \textbf{ Year 2011 :} University Paris~II Assas.\\
					Teaching Assistant in Data Analysis, Licence 3 Eco-gestion.\\
					\textit{Duties : Exercices Sessions.}\\
					Lycée Sainte Marie de Neuilly, Teaching Assistant in Mathematics. \\
					\textit{Duties : Oral examinations of the ``Classes Préparatoires" BL} 
\end{itemize}

%%%%%%%%%%%%%%%%%%%%%%%%%%
\titre{Work Experience}
%###################%%%%%%

\begin{itemize}
\medskip
\item[$\bullet$] \textbf{Internships}: \\

During my education at ENSAE Paristech : \\

\begin{tabular}{cp{0.8\textwidth}}
\textbf{2014} & Master's thesis for the ENSAE under the supervision of Pr. Jérémy Jakubowicz, (Information geometry and deep learning), on applying deep learning and information geometry methods to image processing by neural networks.\\
\textbf{2012}&  $10$-weeks research internship under the supervision of Pr. Cristina Butucea on the theme of Statistics applied to Quantum Optics at University of Marne-la-Vallée (LAMA).		\\
\textbf{2011} & $2$-months mission in Bolivia with the humanitarian association Mission Potosi.\\
\end{tabular}
\\

\textbf{Master's thesis} under the supervision of Pr. Hervé Oyono-Oyono, Six term exact sequences in $K$-theory for crossed products of $C^*$-algebras by $Z$.

\medskip

\item[$\bullet$] \textbf{Professional Experience}\\ 2010--2012 : Worked for SANEF (Motorway tollbooth) in Saint-Avold.\\

\medskip
\end{itemize}

%%%%%%%%%%%%%%%%%%%%%%%%%%%%%%%%%%%%%%
\titre{Languages and computer skills}
%#########################%%%%%%%%%%%%

\begin{itemize} 
\medskip
\item[$\bullet$] \textbf{English} Fluent. TOEIC 915/990.
\medskip
\item[$\bullet$] \textbf{Spanish} Oral.
\medskip
\item[$\bullet$] \textbf{Softwares} Pack Office, LateX, Python, Scilab, R, Objective Caml, C/C++, html.
\end{itemize}

\end{document}
