%\usepackage[frenchb,british]{babel}
\usepackage{verbatim}
\usepackage{amsfonts}
\usepackage{amsthm}
\usepackage{amsmath}
\usepackage{amssymb}
\usepackage{mathabx}
%\usepackage[T1]{fontenc}
\usepackage[utf8]{inputenc}
\usepackage{enumerate}
\usepackage[all]{xy}
\usepackage{graphicx}
\usepackage{tikz}
\usepackage{tikz-cd}
\usepackage{hyperref}

\hypersetup{                    % parametrage des hyperliens
    colorlinks=true,                % colorise les liens
    breaklinks=true,                % permet les retours ?� la ligne pour les liens trop longs
    urlcolor= blue,                 % couleur des hyperliens
   linkcolor= blue,                % couleur des liens internes aux documents (index, figures, tableaux, equations,...)
    citecolor= blue               % couleur des liens vers les references bibliographiques
    }

%Commandes

%\theoremstyle{definition}

\theoremstyle{plain}
%\newtheorem{theorem}{Theorem}[section]
\newtheorem{lemma}[theorem]{Lemma}
\newtheorem{corollary}[theorem]{Corollary}
\newtheorem{proposition}[theorem]{Proposition}
\newtheorem{conjecture}[theorem]{Conjecture}
\newtheorem{definition-theorem}[theorem]{Definition / Theorem}

% without a number

\newtheorem*{conjecture*}{Conjecture}
\newtheorem*{theorem*}{Theorem}
\newtheorem*{corollary*}{Corollary}

\theoremstyle{definition}
\newtheorem{definition}[theorem]{Definition}
\newtheorem{example}[theorem]{Example}
\newtheorem{notation}[theorem]{Notation}
\newtheorem{convention}[theorem]{Convention}

\newcommand{\N}{\mathbb N}
\newcommand{\Z}{\mathbb Z}
\newcommand{\Q}{\mathbb{Q}}
\newcommand{\R}{\mathbb R}
\newcommand{\C}{\mathbb C}
\newcommand{\Hil}{\mathcal H}
\newcommand{\Mn}{\mathcal M _n (\mathbb C)}
%\newcommand{\K}{\mathbb K}
\newcommand{\B}{\mathbb B}
\newcommand{\Cat}{\mathbb B / \mathbb K}
\newcommand{\G}{\mathcal G }

\newcommand{\T}{\mathbb{T}}
\newcommand{\h}{\mathcal{H}}
\newcommand{\U}{\mathcal{U}}
\newcommand{\V}{\mathcal{V}}
\newcommand{\Nc}{\mathcal{N}}
\newcommand{\M}{\mathcal{M}}

\newcommand{\Manoa}{M\=anoa}
\newcommand{\Hawaii}{Hawai\kern.05em`\kern.05em\relax i}


%\newtheorem{thm}[theorem]{Theorem}
  \newtheorem{kor}[theorem]{Corollary}
  \newtheorem{thmx}{Theorem}
   \renewcommand{\thethmx}{\Alph{thmx}}
  \newtheorem{corx}[thmx]{Corollary}
  \newtheorem{prop}[theorem]{Proposition}
  \newtheorem{quest}[theorem]{Question}
 % \newtheorem{conjecture}[theorem]{Vermutung}
  \theoremstyle{definition}
 \newtheorem{defi}[theorem]{Definition}
  \newtheorem{rem}[theorem]{Remark}
  \newenvironment{beweis}%
    {\begin{proof}[Beweis]}
    {\end{proof}}
  \newtheorem{ex}[theorem]{Example}
    \newtheorem{exs}[theorem]{Examples}



\newcommand{\norm}[1]{\lVert#1\rVert}   %Norm{} befehl
\newcommand{\abs}[1]{\lvert#1\rvert}
\newcommand{\KK}{\mathrm{KK}}
\newcommand{\RKK}{\mathcal{R}\mathrm{KK}}
\newcommand{\Tor}{\mathrm{Tor}}
\newcommand{\dad}{\mathrm{dad}}
\newcommand{\asdim}{\mathrm{asdim}}
\newcommand{\K}{\mathrm K}
\newcommand{\EE}{\mathbb E}
\newcommand{\RR}{\mathbb R}
\newcommand{\QQ}{\mathbb Q}
\newcommand{\CC}{\mathbb C}
\newcommand{\NN}{\mathbb N}
\newcommand{\ZZ}{\mathbb Z}
\newcommand{\FF}{\mathbb F}
\newcommand{\TT}{\mathbb T}
\newcommand{\HH}{\mathbb H}
\newcommand{\lk}{\langle}
\newcommand{\rk}{\rangle}
\newcommand{\id}{\text{id}}
\newcommand{\eps}{\varepsilon}
\newcommand{\str}{{s\!\!\!}\times_{\!r}}

\newcommand{\Gmod}{G\textrm{-}\mathbf{mod}}

\setcounter{MaxMatrixCols}{19}
\usepackage{tikz}
\usetikzlibrary{matrix,arrows}
\usepackage{hyperref}





















