\documentclass[a4paper]{article}

\usepackage[frenchb]{babel}
\usepackage{amsfonts}
\usepackage{amsmath}
\usepackage{amssymb}
%\usepackage[T1]{fontenc}
\usepackage[utf8]{inputenc}
\usepackage{amsthm}
\usepackage{graphicx}
\usepackage{tikz}
\usepackage{tikz-cd}
\usepackage{hyperref}
\usepackage{amssymb}
\usepackage{geometry}

\hypersetup{                    % parametrage des hyperliens
    colorlinks=true,                % colorise les liens
    breaklinks=true,                % permet les retours à la ligne pour les liens trop longs
    urlcolor= blue,                 % couleur des hyperliens
    linkcolor= blue,                % couleur des liens internes aux documents (index, figures, tableaux, equations,...)
    citecolor= cyan               % couleur des liens vers les references bibliographiques
    }

\theoremstyle{definition}
\newtheorem{definition}{Definition}
\newtheorem{thm}{Theorem}
\newtheorem{ex}{Exercice}
\newtheorem{lem}{Lemma}
\newtheorem*{dem}{Proof}
\newtheorem{prop}{Proposition}
\newtheorem{cor}{Corollary}
\newtheorem{conj}{Conjecture}
\newtheorem{Res}{Result}
\newtheorem{Expl}{Example}
\newtheorem{rk}{Remark}

\newcommand{\N}{\mathbb N}
\newcommand{\Z}{\mathbb Z}
\newcommand{\R}{\mathbb R}
\newcommand{\C}{\mathbb C}
\newcommand{\Hil}{\mathcal H}
\newcommand{\Mn}{\mathcal M _n (\mathbb C)}
\newcommand{\K}{\mathbb K}
\newcommand{\B}{\mathbb B}
\newcommand{\Cat}{\mathbb B / \mathbb K}
\newcommand{\G}{\mathcal G }

\setlength\parindent{0pt}

\geometry{hmargin=2.5cm,vmargin=1.5cm}

\title{Cover Letter}
% and PhD description}

\date{}
\author{ Clément Dell'Aiera}


%\usepackage{fullpage}

\begin{document}

\maketitle
%\newpage
%\tableofcontents
January 28, 2019

Dear Committee Members,\\

I wish to apply for a Instructorship in Mathematics position in the Department of Mathematics of the \'Ecole Polytechnique de Lausanne. Currently, I am an Assistant Professor (non-tenure track) at the University of Hawai'i at Manoa (USA) under the mentorship of Rufus Willett. He recruted me after I comlpleted a PhD thesis under the supervision of Pr. Hervé Oyono-Oyono, and codirection of Pr. Andrzej Zuk.\\

My primary research would best be described as the intersection of functional analysis, dynamical system and metric geometry (or \textit{Coarse Geometry}). In a broad sense, I would say that most of my work consists in building interesting Banach algebra out of data that comes from dynamical systems or metric spaces. The rest of my time is devoted to use these algebras to tackle problems in algebraic topology (mainly the Novikov conjecture or the Baun-Connes conjecture). What is understood under the vague term \textit{interesting} is exotic, with unusual properties, such as \textit{non nuclear, non exact} $C^*$-algebras (a particular kind of Banach algebras). And a precise way to link these algebras to algebraic topology is to study their $K$-theory (a cohomology theory). For instance, out of a nice proper metric space, say a Riemannian manifold, one can build a $C^*$-algebra called the \textit{Roe algebra}. It turns out that its $K$-theory is the natural receptacle for index of elliptic operator on the manifold, a fact of primary importance in the setting of index theorems.\\    

Here is a summary of my previous work. \\

My PhD thesis developed an extension of quantitative $K$-theory to the setting of \textit{topological groupoids}. These are a versatile tools that can be used to encode dynamical systems, metric spaces, locally compact groups, locally compact spaces, etc. The reason one should try to encode one's object of interest into a groupoid is that they are the perfect place to perform convolution, hence to build operator algebras. Quantitative $K$-theory is a refined version of $K$-theory, more computable than its parent, the usual $K$-theory. My main contribution was to introduce general filtrations on $C^*$-algebras with respect to \textit{coarse structures} (in a sense I defined). Coarse structures already existed in the metric setting. This generalization, while intented for topological groupoids, nevertheless allowed to define coarse structure on algebraic objects such as discrete quantum groups. This was followed by a definition of the controlled $K$-theory groups of a filtered $C^*$-algebra. The second part of the work was a careful study showing that controlled $K$-theory actually was a generalization of quantitative $K$-theory. The final part was the definition of so called \textit{assembly maps} with values in the controlled $K$-theory which factorize the usual Baum-Connes assembly map for groupoids. This allows to establish strongest results in Coarse Geometry , e.g. one can show that the controlled Coarse Baum-Connes conjecture holds for any metric space which admits a fibered coarse embedding into a Hilbert space. H. Oyono-Oyono and G. Yu obtained a similar statement using controlled K-theory for spaces which admits a Coarse embedding into a Hilbert space, which is a stronger condition to fulfill.\\ 

Applying these techniques further, I was able, in a collaboration with Christian B\"onicke, to prove , under suitable conditions, that some $C^*$-algebras associated to groupoids (some of their crossed products) satisfy the K\"unneth formula, an important formula in operator $K$-theory. One of the only known result in this domain is that a-T-menable groupoids (e.g. groupoids corresponding to amenable actions) have their $C^*$-algebras which satisfies the K\"unnet formula. This work contains two ingredients, an extension of the Going-Down principle to \'etale groupoids, and application of controlled $K$-theory. The last step was inspired by a coarse geometric notion, \textit{finite decomposition complexity}, which, once translated in the setting of groupoids, gave a stability result for the K\"unneth formula. This gives an interesting example of a reduced $C^*$-algebra of a groupoid which satisfies the K\"unneth formula and is not a-T-menable.\\

The most recent work was the definition and study of property T for toplogical groupoids. This work is a collaboration with Rufus Willett. \\

%%%%%%
My primary research goals are directed towards Noncommutative geometry, K-theory of C*-algebras and Index Theory. As a PhD student, I followed the route of Pr. Hervé Oyono-Oyono and Pr. Guoliang Yu by using a new version of K-theory which they developed during the last decade in order to study propagation effects in index theorems.\\

My work consisted first in defining a modified version of this controlled K-theory in order to extend it for topological groupoids. I was then able to construct a controlled version of the Baum-Connes assembly map for an étale groupoid that factorizes the usual one for groupoids. This allows to establish strongest results in Coarse Geometry , e.g. one can show that the controlled Coarse Baum-Connes conjecture holds for any metric space which admits a fibered coarse embedding into a Hilbert space. H. Oyono-Oyono and G. Yu obtained a similar statement using controlled K-theory for spaces which admits a Coarse embedding into a Hilbert space, which is a stronger condition to fulfill. I was also able to extend, in a collaboration with Christian B\"onicke, \\

Other applications are very natural to prove once one has the power of this formalism, including a controlled Künneth formula for the K-theory of the reduced C*-algebra of ample groupoids. The highlights of the controlled versions of classical properties such as satisfying the Baum-Connes conjecture or the Künneth formula are that these are more stable, which is the second part of focus of my PhD thesis. The idea behind this line of thought is to use a weak type of decomposition of groupoids which could be seen as a groupoid version of decomposition complexity, and is related to the notion of dynamic asymptotic dimension, developed recently by Pr. Rufus Willett, Pr. Erik Guentner and Pr. Yu. This part of my work is not completely finished yet, but I am working very hard to finish it this year.\\

All of my work has been submitted for publication. My future research plans are aimed at developing ideas coming from Coarse Geometry to topological groupoids, in order to find new applications, not only in Coarse Geometry but in other groupoid related areas, such as foliations or group actions. I would also like to explore controlled K-theory for filtered Banach algebras, in order to tackle non-C*-completions of Roe algebras or of groupoid convolution algebras. This would be a start towards an Oka principle applied to Coarse Spaces. \\

Beyond my research, I had a wide range of teaching experiences. I have been a teaching assistant for undergraduate students almost without interruption for four years in very different areas : from data analysis and linear models to algebra, arithmetic and analysis. I also taught practical computing sessions on machine learning. As I graduated from an engineering school before turning to fundamental mathematics, I can teach applied as well as fundamental mathematics, and I am very sensitive about applications of science. It could be a real asset for your department, and I will work hard to be the best teacher I can. \\

Finally, I would like to mention that I was organizer of the PhD and Postdocs Students seminar of the department of mathematics of the University of Lorraine. In my opinion, social life and interactions with other research fields are very crucial aspects of research. I invited young researchers from various fields, from Statistics to Quantum groups, and from other laboratories. If I was given the opportunity, this is an objective I would pursue in my future career. \\

Enclosed are my curriculum vitae, publication record, teaching and research statement. Please do not hesitate to contact me if further information is needed. \\

Sincerely,\\

Clément Dell’Aiera\\
3 rue du Pont Saint-Marcel\\
57000 METZ (FRANCE)\\

\newpage
\bibliographystyle{plain}
\bibliography{biblio2} 
%\nocite{*}

\end{document}
