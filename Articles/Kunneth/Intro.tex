\section{Introduction}

This article presents a proof of a Künneth formula for crossed products of $C^*$-algebras by action of \'etale groupoids. Recall that a $C^*$-algebra $A$ satisfies the Künneth formula if, for every $C^*$-algebra $B$, the following sequence 
\[\begin{tikzcd}[column sep = small] 
0 \arrow{r} & K_*(A)\otimes K_*(B)\arrow{r} & K_*(A\otimes B) \arrow{r} & Tor(K_*(A),K_*(B))\arrow{r} & 0 
\end{tikzcd}\]
is exact. \\

We are interested in $C^*$-algebras coming from a crossed product of a $C^*$-algebra by an action of an \'etale groupoid. Recall the following definition.

\begin{definition}
An étale groupoid is given by two topological spaces, the space of arrows $G$ and the space of units $G^{(0)}$ endowed with:
\begin{itemize}
\item[$\bullet$] continuous maps $s,r : G \rightrightarrows G^{(0)}$ which are local homeomorphisms,
\item[$\bullet$] a topological embedding $e: G^{(0)}\rightarrow G$ called the unit map, and a continuous involution $inv : G\rightarrow G; g\mapsto g^{-1}$ called the inverse map,
\item[$\bullet$] a multiplication map $G\times_{s,r}G\rightarrow G; (g,g')\mapsto gg'$ such that $(gg')g'' = g(g'g'')$, $gg^{-1}= e_{r(g)}$, $g^{-1}g= e_{s(g)}$
\end{itemize}
\end{definition}

$C^*$-algebras can be fibred over a base space $X$, which gives the notion of a $C(X)$-algebra. A $C(X)$-algebra $A$ has fibers $A_x$ for every $x\in X$. A groupoid can act on a $C(G^{(0)})$-algebra by an automorphism $\alpha : s^* A \rightarrow r^* A $ of $C(G)$-algebras. Here $s^* A$ denotes the pull back of along $s$, which is a $C(G)$-algebra with fibers $(s^* A)_g \cong A_{s(g)}$. Then one can construct the $*$-algebra of sections with compact support $C_c(G,A)$, which can be seen as compactly supported sections $f : G \rightarrow A$ such that $f(g)\in A_{r(g)}$. Then, the crossed product $A\rtimes_r G$ is defined as the completion of $C_c(G,A)$ with respect to the norm $||f||_r =\sup \{f \ast \xi : \xi \in C_c(G_x,A)||\xi||_2\leq 1,x\in G^{(0)}\}$, where $||.||_2$ is the norm on the Hilbert $A_x$-module $l^2(G_x,A)$. When $A= \C$ with trivial action, we get the reduced $C^*$-algebra $C^*_r(G)$.\\
 
Many $C^*$-algebras can be described using the crossed product construction. For instance:
\begin{itemize}
\item[$\bullet$] the irrational rotation algebra, or so called noncommutative torus, can be seen as 
\[C^*_r(\mathbb S^1 \rtimes \Z)\cong C_0(\mathbb S^1)\rtimes_r \Z ,\]
\item[$\bullet$] the Toeplitz algebra, Cuntz algebras, amd AF-algebras can be seen as reduced $C^*$-algebras of well chosen groupoids, see \cite{RenaultDynamical}.
\item[$\bullet$] In \cite{SkTuYu}, G. Skandalis, J-L. Tu and G. Yu introduced the coarse groupoid $G(X)$ associated to any discrete metric space with bounded geometry (more generally any coarse space with bounded geometry) $X$. The Roe algebra $C^*(X)$ and the uniform Roe algebra $C_u^*(X)$ are respectively isomorphic to $l^\infty (X,\mathfrak K)\rtimes_r G(X)$ and $C_r^*(G(X))$. 
\end{itemize}

To prove the main theorem, we will use a technique developed by J. Chabert, S. Echterhoff and H. Oyono-Oyono in \cite{ChabertEOY} called the Going Down Principle. It allows one to reduced a statement about the $K$-theory of crossed product to the analog for compact subgroups. We will adapt these techniques to the setting of groupoids. For this, we first need to define induction to $G$ of $H$-equivariant Hilbert modules with respect to a closed open subgroupoid $H$. Then, we define a class $\mathcal C$ of groupoids, whose proper actions are locally induced from compact open subgroupoids. These groupoids behaves well in topological $K$-theory with respect to restriction to compact open subgroupoids, see lemma \ref{Restriction}.
