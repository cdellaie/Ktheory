\documentclass{article}

\bibliographystyle{abbrv}

%\usepackage{showkeys}

\usepackage{verbatim}

\usepackage{amsfonts}

\usepackage{amsthm}

\usepackage{amssymb}

\usepackage{amsmath}

\usepackage{mathabx}

\usepackage{enumerate}

\usepackage[all]{xy}

\usepackage{graphicx}

\usepackage{hyperref}

\newcommand{\N}{\mathbb{N}}

\newcommand{\Z}{\mathbb{Z}}

\newcommand{\R}{\mathbb{R}}

\newcommand{\Q}{\mathbb{Q}}

\newcommand{\C}{\mathbb{C}}

\newcommand{\T}{\mathbb{T}}

\newcommand{\h}{\mathcal{H}}

\newcommand{\Manoa}{M\=anoa}

\newcommand{\Hawaii}{Hawai\kern.05em`\kern.05em\relax i}


\theoremstyle{plain}
\newtheorem{theorem}{Theorem}[section]
\newtheorem{lemma}[theorem]{Lemma}
\newtheorem{corollary}[theorem]{Corollary}
\newtheorem{proposition}[theorem]{Proposition}
\newtheorem{conjecture}[theorem]{Conjecture}
\newtheorem{definition-theorem}[theorem]{Definition / Theorem}

% without a number

\newtheorem*{conjecture*}{Conjecture}
\newtheorem*{theorem*}{Theorem}

\theoremstyle{definition}
\newtheorem{definition}[theorem]{Definition}
\newtheorem{example}[theorem]{Example}
\newtheorem{notation}[theorem]{Notation}
\newtheorem{convention}[theorem]{Convention}


\theoremstyle{remark}
\newtheorem{remark}[theorem]{Remark}
\newtheorem{remarks}[theorem]{Remarks}

% without a number

\newtheorem*{example*}{Example}  
\newtheorem*{remark*}{Remark}


\begin{document}


\section{Definitions}

For the sake of maintaining consistency, we record the following conventions for a locally compact groupoid $G$.  They match those of Renault \cite[Chapter 1]{Renault:1980fk}, Anantharaman-Delaroche and Renault \cite[Introduction]{Anantharaman-Delaroche:2000mw} and Brown and Ozawa \cite[Section 5.6]{Brown:2008qy} (the last only treats the \'{e}tale case) up to minor notational issues.

\begin{itemize}
\item The unit space is denoted $G^{(0)}$.  General elements of $G$ are generally denoted $g,h$ and elements of $G^{(0)}$ are denoted $x,y$.  
\item The source and range maps $r:G\to G^{(0)}$ and $s:G\to G^{(0)}$ are continuous and open\footnote{Sometimes openness is not included in the definition; however, it is a necessary condition if $G$ is to admit a Haar measure \cite[Proposition I.2.4]{Renault:1980fk}, so we include it explicitly.}.
\item An ordered pair of elements $(g,h)\in G\times G$ is composable if $s(g)=r(h)$; in this case their composition is denoted $gh$ and satisfies $s(gh)=s(h)$ and $r(gh)=r(g)$ (thus our convention is set up so that composition of groupoid elements mimics composition of functions; some authors use the opposite convention).
\item If $A$ is a subset of $G^{(0)}$, then $G^A:=r^{-1}(A)$ and $G_A:=s^{-1}(A)$.  For $A,B\subseteq X$, $G_A^B:=G^B\cap G_A$.  For $x\in G^{(0)}$, $G^x$ is shorthand for $G^{\{x\}}$ and similarly for $G_x$.
\item  \cite[pages 14-15]{Anantharaman-Delaroche:2000mw}, \cite[Definition 2.2]{Renault:1980fk} $G$ is equipped with a fixed \emph{(left) Haar-system}: a collection of measures $\{\mu^x~|~x\in G^{(0)}\}$ on $G$ such that the support of each $\mu^x$ is exactly $G^x$ (concretely, this means that for any open subset $U$ of $G$, $\mu^x(U)>0$ if and only if $U\cap G^x$ is non-empty); for each $f\in C_c(G)$, the function
$$
G^{(0)}\to \C,~~~x\mapsto \int_G f(h)d\mu^x(h)
$$
is continuous; for any $g\in G$ and $f\in C_c(G)$, 
$$
\int_G f(gh)d\mu^{s(g)}(h)=\int_G f(h)d\mu^{r(g)}(h).
$$
\item $G$ is \'{e}tale if $r$ is a local homeomorphism.  This forces each $G^x$ to be discrete, and Haar measure on an \'{e}tale groupoid will always be taken to be counting measure on each $G^x$.
\item The vector space $C_c(G)$ is given the structure of a $*$-algebra via the convolution product
$$
(f_1*f_2)(g):=\int_G f_1(gh)f_2(h^{-1})d\mu^{s(g)}(h)
$$
and involution
$$
f^*(g):=\overline{f(g^{-1})}
$$
(\cite[Proposition II.1.1]{Renault:1980fk} shows that these operations make $C_c(G)$ a topological $*$-algebra).
\item  \cite[page 22]{Renault:1980fk} Let $I:G\to G$ be the homeomorphism sending elements to their inverses, and for each $x\in G^{(0)}$, define $\mu_x:=I_*\mu^x$.
\item \cite[page 141]{Anantharaman-Delaroche:2000mw}, \cite[page 50]{Renault:1980fk} Given $f\in C_c(G)$, there are range and source \emph{$I$-norms} defined by
$$
\|f\|_{I,r}:=\sup_{x\in G^{(0)}}\int_G|f(g)|d\mu^x(g)
$$
and 
$$
\|f\|_{I,s}:=\sup_{x\in G^{(0)}}\int_G|f(g)|d\mu_x(g)=\sup_{x\in G^{(0)}}\int_G|f(g^{-1})|d\mu^x(g).
$$
The \emph{$I$-norm} of $f$ is 
$$
\|f\|_I:=\max\{\|f\|_{I,r},~\|f\|_{I,s}\}.
$$
\end{itemize}

\begin{definition}\label{rep}
A \emph{representation} of $C_c(G)$ is a non-degenerate $*$-homomorphism
$$
\pi:C_c(G)\to \mathcal{B}(\h)
$$
from $C_c(G)$ to the $C^*$-algebra of bounded operators on a Hilbert space $\h$, which is moreover bounded for the $I$-norm:
$$
\|\pi(f)\|_{\mathcal{B}(\h)}\leq \|f\|_I
$$
for all $f\in C_c(G)$.
\end{definition}

We will often leave the $*$-homomorphism $\pi$ implicit in our notation, and just say things like `$\h$ is a representation of $C_c(G)$'.

\begin{comment}
\begin{remark}\label{group}
If $G$ is a locally compact \emph{unimodular} group (equipped with a fixed Haar measure), then $C_c(G)$ as above is the same as the usual group convolution algebra, and boundedness for the $I$-norm is the same as boundedness for the usual $L^1$ norm.  If not, one easily checks that the vector space isomorphism defined by
$$
\phi:C_c(G)\to C_c(G),~~~\phi(f)(g)=(f(g)\Delta(g))^{-1/2}
$$
intertwines the $*$-algebra structure above and the more usual $*$-algebra structure for a group that has the same convolution as above, but involution defined by 
$$
f^*(g):=\overline{f(g^{-1})}\Delta(g^{-1}).
$$
Using this, one sees using standard arguments that $I$-bounded non-degenerate $*$-representations of $C_c(G)$ (with our $*$-algebra structure) are in one-to-one correspondence with unitary representations of $G$.
\end{remark}
\end{comment}

\begin{remark}\label{brem}
Let $\pi:C_c(G)\to \mathcal{B}(\h)$ be a nondegenerate $*$-homomorphism.  If $G$ is \'{e}tale, it is not difficult to see that $\pi$ is automatically bounded for the $I$-norm, and thus a representation.  The discussion in \cite[Appendix A]{Sims:2012fk} shows that if $G$ is second countable and $\pi$ is assumed continuous for the standard inductive limit topology on $C_c(G)$, then $\pi$ is also automatically continuous for the $I$-norm, and thus a representation.

We do not know if this is true in general (it seems unlikely, even if $G$ is a group, without assuming at least some form of continuity).  
\end{remark}

Let $G$ be a locally compact groupoid  equipped with a Haar system.  

\begin{definition}\label{constant}
Define a linear map
$$
\Phi:C_c(G)\to C_c(G^{(0)})
$$
by 
$$
\Phi(f)(x)=\int_{G} f(g) d\mu^x(g).
$$

Let $\h$ be a representation of $C_c(G)$.  An element $\xi$ of $\h$ is said to be \emph{constant} or \emph{invariant} if
$$
f\xi=\Phi(f)\xi
$$
for all $f\in C_c(G)$.  Write $\h_c$ for the closed subspace\footnote{It is not in general a subrepresentation.} of constant elements.
\end{definition}  

\begin{definition}\label{kaz}
An open subset $K$ of $G$ is a \emph{Kazhdan set} if there exists $c>0$ such that for any $*$-representation $\h$ of $C_c(G)$ and $\xi\in \h_c^\perp$ there exists $f\in C_c(G)$ supported in $K$ with $\|f\|_I\leq 1$ and such that 
$$
\|f\xi-\Phi(f)\xi\|\geq c\|\xi\|.
$$
The groupoid $G$ has \emph{property (T)} if it admits a Kazhdan set with compact closure.
\end{definition}

The following lemma shows that property (T) as above does not have much content in the absence of an invariant probability measure on $C_0(G^{(0)})$.

\begin{lemma}\label{inv meas}
There is a representation $\h$ of $G$ with $C_c(G)$ with non-trivial $\h_c$ non-trivial if and only if there is an invariant probability measure on $G^{(0)}$.  
\end{lemma}

\begin{proof}
Here, maybe.
\end{proof}






\section{Examples}

In this section, we look at some examples.

\subsection*{Groups}

The basic example is that of groups.  Here we show that our version of property (T) reduces to the usual one.  The following definition is \cite[Definition 1.1.3]{Bekka:2000kx}.

\begin{definition}\label{gpt}
Let $G$ be a locally compact group.  Let 
$$
u:G\to \mathcal{U}(\h)
$$
be a strongly continuous unitary representation of $G$.  Say that a vector $\xi\in \h$ is \emph{invariant} or \emph{constant} if $u_g\xi=\xi$ for all $g\in G$.

A subset $K$ of $G$ is a \emph{Kazhdan set} if there exists $c>0$ such that for any strongly continuous unitary representation 
$$
u:G\to \mathcal{U}(\h)
$$
of $G$ with no non-zero invariant vectors and all $\xi\in \h$ there exists $g\in K$ such that 
$$
\|u_g\xi-\xi\|\geq c\|\xi\|.
$$
The group $G$ has property (T) if it admits a compact Kazhdan set.
\end{definition}

We now have two definitions of `Kazhdan set' for groups: Definition \ref{gpt} and the specialisation of Definition \ref{kaz}!  Temporarily, if $G$ is a locally compact group let us say a \emph{group Kazhdan set} a Kazhdan set in the sense of Definition \ref{gpt} and a \emph{groupoid Kazhdan set} a Kazhdan set in the sense of Definition \ref{kaz}, and similarly for the notions of invariant vector.  However, the notions are closely related.  

For simplicity, we restrict to the case of unimodular groups (I did not yet check the details in the non-unimodular case).  First, we have a lemma relating the two notions of constant vector.

\begin{lemma}\label{cons lem}
Let $G$ be a unimodular locally compact group.  If $\pi:C_c(G)\to \mathcal{B}$ is a representation of $C_c(G)$ in the sense of Definition \ref{rep} then there is a unique unitary representation $u:G\to \mathcal{U}(\h)$ such that
$$
\pi(f)=\int_G f(g)u_g d\mu(g)\quad  \text{for all }f\in C_c(G).
$$
Conversely, if $u:G\to \mathcal{U}(\h)$ is a strongly continuous unitary representation of $G$, then the formula in the displayed line defines a representation of $C_c(G)$ in the sense of Definition \ref{rep}.  

Moreover, if $\pi$ and $u$ are related as above, then the constant vectors defined with respect to $\pi$ are the same as those defined with respect to $u$.
\end{lemma}

\begin{proof}
Note first that as $G$ is unimodular, the $I$-norm on $C_c(G)$ is just the $L^1$-norm, and thus representations of $C_c(G)$ in the sense of Definition \ref{rep} extend uniquely to nondegenerate $*$-representations of $L^1(G)$.  The claimed correspondence between $\pi$s and $u$s now follows from \cite[Section 6.2]{Echterhoff:2009qd}, for example.

To compare constant vectors, note first that $\Phi:C_c(G)\to C_c(G^{(0)})=\C$ as in Definition \ref{constant} is just integration with respect to the Haar measure in this case.  Say $\xi$ is constant for $u$.  Then for any $f\in C_c(G)$,
$$
\pi(f)\xi=\int_G f(g)  u_g\xi d\mu(g)=\int_G f(g)\xi d\mu(g)=\Big(\int_G f(g)d\mu(g)\Big)\xi=\Phi(f)\xi,
$$
so $\xi$ is constant for $\pi$.  Conversely, say $\xi$ is constant for $\pi$.   Let $(f_i)_{i\in I}$ be a \emph{Dirac net} in the sense of \cite[page 28]{Echterhoff:2009qd}, so in particular, $\int f_i=1$ for all $i$.  Then \cite[Lemma 6.2.2]{Echterhoff:2009qd} implies that for any $g\in G$, 
$$
\lim_{i\in I}u_gu(f_i) \xi=u_g\xi.
$$
On the other hand, $u_gu(f_i)=u(h_i)$, where $h_i$ is the left shift of $f_i$ by $g$ (and in particular still has integral one) and therefore 
$$
u_g\xi=\lim_{i\in I}u(h_i)\xi=\lim_{i\in I} \Phi(h_i)\xi=\lim_{i\in I} \Big(\int h_i\Big)\xi=\xi.
$$
Hence $\xi$ is invariant for $u$.
\end{proof}

\begin{proposition}\label{2kaz}
Let $G$ be a unimodular locally compact group.  
\begin{enumerate}
\item If $K$ is a groupoid Kazhdan set, then it is also a group Kazhdan set.
\item If $K$ is a group Kazhdan set then there is a neighbourhood $U$ of the identity in $G$ such that if $V\subseteq U$ is open and contains the identity, then $V\cdot K$ is a groupoid Kazhdan set. 
\end{enumerate}
\end{proposition}

\begin{proof}
Assume first that $K$ is a groupoid Kazhdan set with associated constant $c>0$, and let $u:G\to \mathcal{U}(\h)$ be a unitary representation of $G$ with no non-zero invariant vectors.  Using Lemma \ref{cons lem}, we may integrate $u$ to a representation $\pi$ which also has no non-zero invariant vectors.  Hence with notation as in Definition \ref{rep}, we have that $\h=\h_c^\perp$.  Let $\xi$ be an element of $\h$.  As $K$ is a groupoid Kazhdan set with respect to $c>0$, there exists $f\in C_c(G)$ with $\|f\|_I\leq 1$ and support in $K$ such that
\begin{equation}\label{geninq}
\|f\xi-\Phi(f)\xi\|\geq c\|\xi\|.
\end{equation}
Writing out the left-hand-side more fully gives
\begin{align*}
\Big\|\int_G f(g)u_g\xi d\mu(g)-\Big(\int_G f(g)d\mu(g)\Big)\xi\Big\| & \leq \int_G |f(g)|\|u_g\xi-\xi\|d\mu(g) \\ & \leq \|f\|_I\sup_{g\in K}\|u_g\xi-\xi\| \\ & \leq \sup_{g\in K}\|u_g\xi-\xi\|.
\end{align*}
Combining this with line \eqref{geninq} gives
$$
\sup_{g\in K}\|u_g\xi-\xi\|\geq c\|\xi\|,
$$
and so there exists $g\in K$ with 
$$
\|u_g\xi-\xi\|\geq \frac{c}{2}\|\xi\|.
$$
Hence $K$ is also a group Kazhdan set.\\

Conversely, say $K$ is a group Kazhdan set with associated constant $c>0$.  Let $\pi:C_c(G)\to \mathcal{B}(\h)$ be a representation of $C_c(G)$, and as in Lemma \ref{cons lem}, let $u:G\to \mathcal{U}(\h)$ be the associated representation of $G$, with the same constant vectors.  Let $\xi$ be a vector in $\h_c^\perp$, and note that as $u$ is a unitary representation of a group, it restricts to a representation on $\h_c^\perp$.  Hence there exists $g\in K$ with $\|u_g\xi-\xi\|\geq c\|\xi\|$.  Let $(f_U)$ be a Dirac net in $C_c(G)$ as on \cite[page 22]{Echterhoff:2009qd}, indexed by the collection of all open subsets of $G$ containing the identity, ordered by reverse inclusion.  Then 
$$
\lim_U \pi(f_U)u_g\xi=u_g\xi,
$$
and therefore there is open $U\owns e$ such that for all open $V\owns e$ contained in $U$, we have that 
$$
\|\pi(f_V)u_g\xi-\xi\|\geq \frac{c}{2}\|\xi\|.
$$ 
Let $h_V$ be the right shift of $f_V$ by $g$, and note that $\int h_V=1$ whence $\Phi(h_V)\xi=\xi$ for all $V$.  Hence the previous displayed line gives
$$
\|\pi(h_V)\xi-\xi\|\geq \frac{c}{2}\|\xi\|.
$$
As $\|h_V\|_{L^1(G)}=\|h_V\|_I=1$ for all $V$, and as $h_V$ is supported in $V\cdot K$, this completes the proof.
\end{proof}

\begin{corollary}\label{gpgpd}
Let $G$ be a locally compact group.  Then it has property (T) in the sense of Definition \ref{kaz} if and only if it has it in the sense of Definition \ref{gpt}. 
\end{corollary}

\begin{proof}
Say $G$ has groupoid property (T) with $K$ an open Kazhdan set with compact closure.  Then Proposition \ref{2kaz} $K$ is a group Kazhdan set.  As a superset of a group Kazhdan set is a group Kazhdan set, $\overline{K}$ is thus a compact group Kazhdan set, so $G$ has group property (T).  

Conversely, say $G$ has group property (T), and let $K$ be a compact groupoid Kazhdan set.  Note that $K\cdot U$ has compact closure whenever $U\subseteq G$ is an open set with compact closure.  Proposition \ref{2kaz} implies that $K\cdot V$ is a groupoid Kazhdan set for some open $V\owns e$ with $V$ having compact closure (use that $G$ is locally compact), and so $G\cdot V$ is a groupoid Kazhdan set with compact closure.
\end{proof}






\subsection*{Coarse spaces}

\begin{proposition}\label{coarse t}
Let $X$ be a bounded geometry metric space.  Then the associated coarse groupoid has property (T) if and only if $X$ has geometric property (T) in the sense of \cite[Definition 3.3]{Willett:2013cr}.
\end{proposition}

\begin{proof}
Here.
\end{proof}




\subsection*{Transformation groups}

\begin{proposition}\label{trans t}
Let $G$ be a unimodular locally compact group acting on a locally compact space $X$, and fixing a probability measure.  Then the transformation groupoid $X\rtimes G$ has property (T) if and only if $G$ does.  Probably the general statement says something like: property (T) if and only if the family of representations of $G$ arising from restriciting representations of $G$ in the sense of Renault does not weakly contain the trivial representation without actually containing it.
\end{proposition} 

\begin{proof}
Here.
\end{proof}







%\section{Kazhdan projections}

%Let $G$ be a locally compact groupoid, which we assume throughout has property (T).

%\begin{proposition}\label{cons}
%Let $K$ be a compact generating set for $G$, and let $f:G\to \R_+$ be a compactly supported continuous function, strictly positive on $K$.  Define
%$$
%\Delta_f:=f^*f-\Phi(f^*f).  ???
%$$  
%Then there exists $c>0$ such that in any representation $\h$ of $C_c(G)$, the spectrum of $\Delta_f$ as an operator on $\h$ is contained in a set of the form 
%$$
%\{0\}\cup [c,\infty)
%$$
%Moreover, if $0$ appears as a spectral value of $\Delta_f$ in some representation $\h$, then it is an eigenvalue, and the corresponding eigenspace consists precisely of the constant vectors $\h_c$.
%\end{proposition}



\section{The rest}

The following topics might be good to include.  Some might be quite technically demanding, and I'm not completely sure what's possible.

\begin{itemize}
\item Property (T) gives rise to Kazhdan projections in the presence of an invariant probability measure.
\item Property (T) is probably a Morita invariant.  
\item In the presence of an invariant probability measure on $G^{(0)}$, property (T) is probably incompatible with the Haagerup property.
\item Relationship with `Anantharaman-Delaroche property (T)'.  I do not know what the answer is here, but it might be interesting to try to work this out.
\item Relationship with `Cohomological property (T)'.  This was introduced by Mimura, Ozawa, Sako, and Suzuki in the context of coarse spaces \cite[Section 6]{Mimura:2014uq}.  It should generalize (I think...) to other groupoids.  It should imply our property (T), but not the other way around.  
\end{itemize}











\bibliography{Generalbib}


\end{document}