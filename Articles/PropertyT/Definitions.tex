\section{Definitions}

Let $G$ be a locally compact groupoid endowed with a Haar system. 

\begin{definition}
Define the following map
\[ E : \left\{ \begin{array}{rcl} 
C_c(G) & \rightarrow & C_c(G^{(0)}) \\
f & \mapsto & x\mapsto \int_{G^{x}} f(g) d\lambda^x(g)
\end{array}\right. .\]
\end{definition}

If $(\pi,H)$ is a representation of $C_c(G)$, define $H^\pi$ to be the subspace of the vectors $\xi \in H$ such that 
\[ f\xi = E(f)\xi \quad \forall f \in C_c(G).\]
These are called the invariant vectors. The subspace $H^\pi$ is naturally complemented, and we denote by $H_\pi$ its complementary so that $H = H^\pi \oplus H_\pi$.\\


Let $\mathcal F$ be a family of representation of $C_c(G)$.

\begin{definition}
Let $K$ be an open subset of $G$ and $c > 0$. The couple $(K,c)$ is called a Kazdhan pair for $\mathcal F$ if, in every representation $\pi $ of $\mathcal F$ and every $\xi \in H_\pi$, there exists $f\in C_c(G)$ with $||f||_I\leq 1$ supported in $K$ such that
\[ ||f\xi - E(f)\xi|| \geq c || \xi || .\]
The groupoid $G$ is said to have property $(T)_{\mathcal F}$ if there exists an open subset $K\subseteq G$ with compact closure and a positive number $c$ such that $(K,c)$ is a Kazdhan pair.  
\end{definition}
