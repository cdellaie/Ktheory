\section{Applications to Coarse Geometry}

We present in this section a result on the equivalence between the controlled assembly map for a discrete metric space with bounded geometry $X$ with coefficients in a $C^*$-algebra $B$ and the controlled assembly map for the coarse groupoid $G(X)$ with coefficients in the $G(X)$-algebra $l^\infty(X,B\otimes \mathfrak K)$. This result is applied to show that any such space that admits a fibred coarse embedding into Hilbert space satisfies the maximal controlled Baum-Connes conjecture.

\subsection{Equivalence between the controlled coarse assembly map for $X$ and the controlled assembly map for $G$ with coefficients in $l^\infty(X,\mathfrak K)$}

In this section, we prove how the result of G. Skandalis, J.-L. Tu and G. Yu \cite{SkTuYu} extends to the setting of controlled $K$-theory. \\

Recall from theorem \ref{IsomCoarseGroupoid} that, for every $C^*$-algebra $B$, there exists a natural isomorphism of $C^*$-algebras 
\[\Psi_B : l^\infty(X,B\otimes\mathfrak K)\rtimes_r G(X)\rightarrow C^*(X,B).\]
Moreover, it is filtered in the strong sense : for all $R>0$, $\Psi_B(C_{\overline\Delta_R}(G,B))= C_R[X,B]$, where $R=\sup_E d$.\\

The following theorem is proved in \cite{SkTuYu}. It states the equivalence between the coarse Baum-Connes conjecture with coefficients in $B$ and the Baum-Connes conjecture for $G(X)$ with coefficients in $l^\infty(X,B\otimes\mathfrak K)$.  

\begin{thm}[\cite{SkTuYu}]
Let $X$ be a discrete metric space with bounded geometry. Let $\Psi_B$ be the isomorphism of theorem \ref{IsomCoarseGroupoid}, $x\in X$ and $\iota :\{x\}\rightarrow G(X)$ be the natural inclusion of groupoids. Denote by $G=G(X)$ the coarse groupoid of $X$ and by $\tilde B$ the $G$-algebra $l^\infty (X,B\otimes\mathfrak K)$. Then, for every controlled subset $E\subseteq X\times X$, the following diagram is commutative with vertical arrows being isomorphisms :
\[\begin{tikzcd}
RK_*^G(P_{\overline{E}}(G),\tilde B) \arrow{r}{\mu_{G,\tilde B}^{\overline E}}\arrow{d}{\iota^*}& K_*(\tilde B\rtimes_r G)\arrow{d}{(\Psi_B)_*}\\
RK_*(P_E(X),B) \arrow{r}{\mu_{X,B}^E}& K_*(C^*(X,B))
\end{tikzcd},\]
where $\iota^*$ is the natural transformation induced by $\iota$ and $d= \sup_E d$.
\end{thm}

We shall prove a controlled analogue of this result which induces it in $K$-theory. We need the following lemmas. %s.

%\begin{lem}
%Let $G$ be an étale groupoid,$x\in G$, $Z$ a proper $G$-space and $B$ a $C^*$-algebra. Denote by $\tilde A$ the $G$-algebra $C_0(Z)$ and $\iota : \{x\} \rightarrow G$ the natural inclusion of groupoids. Then :
%\[\iota^* : RK^G(Z,l^\infty_B)\rightarrow KK(A_x,B)\]
%is an isomorphism of $\Z_2$-graded abelian groups. 
%\end{lem}

%\begin{dem}
%We define an inverse for $\iota^*$ : for $z=[H_{B_x},\pi,T]\in KK(A_x,B_x)$, define $\eta(z)= [H_B,\tilde\pi,\tilde T]$ where 
%\[(\tilde\pi) = \pi \otimes id\]
%\qed
%\end{dem}
\begin{lem}[\cite{SkTuYu}]\label{iota}
Let $x\in X$ and $\iota : \{x\}\hookrightarrow G$ the natural inclusion of groupoids. Then 
\[\iota^* : KK^G(C_0(P_{\overline E}(G),\tilde B) \rightarrow KK(C_0(P_{E}(X),B) \]
is an isomorphism of $\Z_2$-graded abelian groups.
\end{lem}

The reader can find a proof in \cite{SkTuYu} (Lemma $4.7$). We recall the explicit construction of the inverse 
\[j:KK(C_0(P_{E}(X),B) \rightarrow KK^G(C_0(P_{\overline E}(G),\tilde B)\] 
of $\iota^*$. Let $(H_B,\pi,T)\in\mathbb E (C_0(P_{E}(X),B)$ be a standard $K$-cycle. Let $\tilde B= l^\infty(X,B\otimes\mathfrak K)$ seen as Hilbert module over itself, $(\tilde \pi (a)\xi)(x) = \pi(a(x))\xi(x)$ and $(\tilde T\xi ) (x) = T\xi(x)$, for every $x\in X$ and $\xi\in E$. Then $j([H_B,\pi,T])=[\tilde B,\tilde \pi, \tilde T]$.

\begin{lem} Let $E\subseteq X\times X$ be controlled subset and $B$ be $C^*$-algebra. Denote $C_0(P_E(X))$ by $A$, $C_0(P_{\overline E}(G))$ by $\tilde A$ and $\tilde B = l^\infty(X,B\otimes \mathfrak K)$. Then, for every $z\in KK^G(\tilde A,\tilde B)$, the following equality of controlled morphisms holds :
\[\hat\sigma_X(\iota^*(z))\circ (\Psi_A)_* = (\Psi_B)_*\circ \hat J_G(z).\]  
\end{lem}

%\begin{dem}
%Let $z\in KK_1^G(\tilde A,\tilde B)$ be represented by the $K$-cycle $[H_{\tilde B},\pi,T]$ and let $P=\frac{1+T}{2}$. We can suppose that $T$ is $G$-equivariant. %We can suppose that $T$ is $G$-equivariant, i.e. $T = T'\otimes id_{L^2(G)}$ for some $T'\in \mathcal L_B(H\otimes B)$. 
%Recall that $E^{(\pi,T)} = \{(x,P_G\pi_G(x)P_G+y : x\in \tilde A\rtimes_r G,y\in (\tilde B\rtimes_r G)\otimes\mathfrak K\}$.\\

%Let us show how to extend $\Psi$ to $E^{(\pi,T)}$. For any $C^*$-algebra $B$, $\tilde B$ is naturally a $C^*$-subalgebra of both $\tilde B\rtimes_r G$ and $C^*(X,B)$, and the two inclusions commute modulo $\Psi_B$. We have a diagram :
%\[\begin{tikzcd} 
%  \  & B \arrow[bend left]{rdd}{\iota_3^B}& \\
%  \ &\tilde B \arrow{u}{ev_x}\arrow[hookrightarrow]{ld}{\iota_1^B}\arrow[hookrightarrow]{rd}{\iota_2^B} &  \\ 
%\tilde B\rtimes_rG \arrow{rr}{\Psi_B} &  &  C^*(X,B) 
%\end{tikzcd}\] 
%where the lower triangle is commutative. The map $(x,y)\mapsto (\Psi_A(x), (\Psi_B)_*(y))$ induces a morphism 
%\[\Psi_E : E^{(\psi,T)}  \rightarrow  E^{(\psi_x,T_x)} \] 
%which sends $(x,P_G \psi_G(x)P_G + y)$ to $(\Psi_A(x), (P_x)_X(\psi_x)_X(\Psi_A(x))(P_x)_X+(\Psi_B)_*(y))$. Indeed, as $T_G = (\iota_1)_*(T)$, we have $(\Psi_B)_*(T_G)=(\iota_2)_*(T)=(T_x)_X$. Also, the relations $(\iota_1^A)_*\circ\pi = \pi_G\circ \iota_1^A$ and $(\iota_2^A)_*\circ\pi_x = (\pi_x)_X\circ \iota_2^A$ are easy to derive, which lead to $(\Psi_B)_*\circ \pi_G \circ \iota_1^A= (\iota_2^B)_*\circ \pi_x = (\pi_x)_X\circ \Psi_A\circ \iota_1^A$. By extending $G$-equivariantly to $\tilde A  \rtimes_r G$, we have $(\Psi_B)_*(\pi_G(a))=(\pi_x)_X(\Psi_A(a))$.\\

%This map makes the following diagram commute
%\[\begin{tikzcd}[column sep = small]
%0\arrow{r} & K_{\tilde B\rtimes G}\arrow{r}\arrow{d}{(\Psi_B)_*} & E^{(\psi,T)} \arrow{r}\arrow{d}{\Psi_E}& \tilde A\rtimes_r G\arrow{r}\arrow{d}{\Psi_A} & 0\\
%0\arrow{r} & K_{C^*(X,B)}\arrow{r} & E^{(\psi_x,T_x)} \arrow{r}& C^*(X,A)\arrow{r} & 0
%\end{tikzcd}.\]
%Now remark \ref{rk3.8} gives $((\Psi_B)_*)_*\circ D_{\tilde A\rtimes_rG}^{K_{\tilde B\rtimes_G}} = D_{C^*(X,A)}^{K_{C^*(X,B)}}\circ (\Psi_A)_*$, and if we compose by the Morita equivalence, we get 
%\[\hat\sigma_X(\iota^*(z)) \circ (\Psi_A)_* = (\Psi_B)_*\circ \hat J_G(z).\]\qed\end{dem}

%%% NEW PROOF
%\begin{dem}
%Let $z\in KK_1^G(\tilde A,\tilde B)$. Let $\iota^*(z)$ be represented by the $K$-cycle $[H_{B},\pi,T]\in\mathbb E(A,B)$, and let $P=\frac{1+T}{2}$. Denote by $(\tilde B,\tilde \pi ,\tilde T)\in \mathbb E^G(\tilde A,\tilde B)$ the representative of $j(\iota^*(z))=z$ constructed as in lemma \ref{iota}. Recall that $E^{(\pi,T)} = \{(x,P\pi(x)P+y : x\in A,y\in B\otimes\mathfrak K\}$, and $E^{(\pi,T)}_X=C^*(X,E^{(\pi,T)})$. By \ref{Kasparov1}, $J_G(z)$ is given by the controlled boundary of $E_G^{(\tilde \pi,\tilde T)}$. We add in this proof a $G$ in subscript of $E^{(\tilde \pi,\tilde T)}$ for convenience of notation.\\

%We shall define a $*$-homomorphism from $E^{(\pi,T)}_X$ to $E_G^{(\tilde \pi,\tilde T)}$ that intertwines the two extensions. Let $\Phi_B : %C^*(X,B)\rightarrow \tilde B\rtimes_r G$ be the inverse of $\Psi_B$, for every $C^*$-algebra $B$. Define 
%\[\Phi_E : \left\{
%\begin{array}{rcl}
%E^{(\pi,T)}_X   & \rightarrow   & E_G^{(\tilde \pi,\tilde T)} \\
%(a,y)		& \mapsto	& (\Phi_A(a),\mathcal M_{\tilde B\rtimes_r G}\circ\Phi_B(y))\\
%\end{array}\right.\]

%which sends $(x,P_X \pi_X(x)P_X + y)$ to $(\Phi_A(x), (\tilde P)_G(\tilde \pi)_G(\Phi_A(x))(\tilde P)_G+(\mathcal M_{\tilde B\rtimes_r G}\circ\Phi_B)_*(y))$. This map makes the following diagram commutes
%\[\begin{tikzcd}[column sep = small]
%0\arrow{r} & C^*(X,B) \arrow{r}\arrow{d}{\mathcal M_{\tilde B\rtimes_r G}\circ\Phi_B} & E^{(\pi,T)}_X  \arrow{r}\arrow{d}{\Phi_E} & 
%	C^*(X,A)\arrow{r}\arrow{d}{\Phi_A} & 0 \\
%0\arrow{r} & K_{\tilde B\rtimes G}\arrow{r} & E_G^{(\tilde \pi,\tilde T)} \arrow{r}& \tilde A\rtimes_r G\arrow{r} & 0
%\end{tikzcd}.\]
%By remark \ref{rk3.8}, we get 
%\[D_{K_{\tilde B\rtimes G}, E_G^{(\tilde \pi,\tilde T)}} \circ (\Phi_A)_* =\mathcal M_{\tilde B\rtimes_r G} \circ (\phi_B)_* \circ D_{ C^*(X,B), E^{(\pi,T)}_X},\]
%hence, composing with Morita equivalence, and $(\Psi_A)_*$ and $(\Psi_B)_*$,
%\[ (\Psi_B)_*\circ \hat J_G(z) = \hat\sigma_X(\iota^*(z)) \circ (\Psi_A)_*.\]
%\qed
%\end{dem}
%%%

%%% NEW NEW PROOF
\begin{dem}
Let $z\in KK_1^G(\tilde A,\tilde B)$. Let $\iota^*(z)$ be represented by the $K$-cycle $[H_{B},\pi,T]\in\mathbb E(A,B)$, and let $P=\frac{1+T}{2}$. Denote by $(\tilde B,\tilde \pi ,\tilde T)\in \mathbb E^G(\tilde A,\tilde B)$ the representative of $j(\iota^*(z))=z$ constructed as in lemma \ref{iota} and $\tilde P=\frac{1+\tilde T}{2}$. Recall that 
\[E^{(\pi,T)} = \{(x,P\pi(x)P+y : x\in A,y\in B\otimes\mathfrak K\},\] 
and $E^{(\pi,T)}_X=C^*(X,E^{(\pi,T)})$. \\

First, notice that $z$ is the boundary element in $KK^G(\tilde A,\tilde B)$ of the following extension 
\[0 \rightarrow \tilde B \rightarrow E'\rightarrow \tilde A \rightarrow 0\]
where $E'$ is the $G$-algebra $\{ (a,\tilde P \tilde \pi(a) \tilde P+y) : a\in \tilde A, y \in \tilde B  \}\subseteq \tilde A\oplus \mathcal M(\tilde B) $, and the $*$-homomorphisms are the obvious ones. Set 
\[E'_G=\{ (a,\tilde P_G \tilde \pi_G(a) \tilde P_G+y) : a\in \tilde A\rtimes_r G, y \in \tilde B\rtimes_r G    )\}.\] 
We take the previous extension under the reduced crossed product to get the following extension
\[0 \rightarrow \tilde B\rtimes_r G \rightarrow E_G'\rightarrow \tilde A\rtimes_r G \rightarrow 0.\]
By \ref{Kasparov1}, $J_G(z)$ is given by the controlled boundary of $E'_G$. \\

We shall define a $*$-homomorphism from $E'\rtimes G$ to $E^{(\pi,T)}_X$ that intertwines the two extensions. Extend the $*$-isomorphism $\Psi_B : \tilde B \rtimes_r G \rightarrow C^*(X,B)$ to $\tilde \Psi_B : \mathcal M(\tilde B \rtimes_r G) \rightarrow \mathcal M(C^*(X,B))$. Set $\Psi_{E'} (a,y) = (\Psi_A(a),\tilde \Psi_B(y)) $ for every $(a,y)\in E'_G$. This map makes the following diagram commutes
\[
\begin{tikzcd}[column sep = small]
0\arrow{r} & \tilde B\rtimes_r G \arrow{r} \arrow{d}{\Psi_B} & E'_G \arrow{r}\arrow{d}{\Psi_{E'}} &
			 \tilde A\rtimes_r G\arrow{r}\arrow{d}{\Psi_A} & 0 \\
0\arrow{r} & C^*(X,B) \arrow{r} & E^{(\pi,T)}_X  \arrow{r} & C^*(X,A)\arrow{r} & 0 
\end{tikzcd}.
\]
By remark \ref{rk3.8}, we get 
\[  (\Psi_B)_* \circ D_{\tilde B\rtimes_r G, E'_G} = D_{ C^*(X,B), E^{(\pi,T)}_X} \circ (\Psi_A)_*,\]
hence,
\[ (\Psi_B)_*\circ \hat J_G(z) = \hat\sigma_X(\iota^*(z)) \circ (\Psi_A)_*.\]
\qed
\end{dem}

%%% END NEWPROOF

\begin{thm}\label{BCCeq}
Let $B$ be a $C^*$-algebra, $E\in \mathcal E_X$ an entourage and $\overline E \in \mathcal E_G$ the corresponding compact open subset of $G$. With the above notations, for all $z\in RK^G(P_{\overline E}(G),\tilde B)$ and all $\varepsilon\in(0,\frac{1}{4})$, the following equality holds :
\[(\Psi_B)_*\circ\mu^{\epsilon,\overline E}_{G,\tilde B} (z) = \mu_{X,B}^{\epsilon,E}(\iota^*(z)).\]
\end{thm}

\begin{dem}
%Let $E$ be a compact subset of $G$ such that $\overline \Delta_R \subseteq E$.
By the previous lemma, we only need to check that $(\Psi_A)_*[\mathcal L_{\overline E},0]_{\varepsilon,\overline  E} = [P_{E},0]_{\varepsilon, E} $, which is trivial.\\
\qed
\end{dem}

\begin{rk}
This theorem remains true for the maximal version of the assembly map when $B=\C$. One then has to replace $\hat\mu_{G,\tilde \C}$ and $\hat\mu_{X}$ by $\hat\mu^{max}_{G,\tilde \C}$ and $\hat\mu^{max}_{X}$ respectively.
\end{rk}

This result induces the result of \cite{SkTuYu} in $K$-theory. It also implies interesting consequences for Coarse Geometry. Recall that if the groupoid $G$ satisfies the Baum-Connes conjecture with coefficients, it satisfies the controlled Baum-Connes conjecture. Interesting examples follow from the result of J-L. Tu \cite{TuThese} that a-$T$-menable groupoids satisfy the Baum-Connes conjecture with coefficients. In particular, \\

\begin{itemize}
\item[$\bullet$] amenable groupoids are a-$T$-menable.\\
\item[$\bullet$] Let $X$ be a uniformly discrete metric space with bounded geometry. Then, if $X$ is coarsely embeddable into a separable Hilbert space, $G(X)$ is a-$T$-menable \cite{SkTuYu}. \\
\end{itemize}

\subsection{Fibred coarse embedding}

We now present an application to fibred coarse embedding.

\begin{definition}
Let $X$ be a discrete metric space with bounded geometry and $B$ a $C^*$-algebra. We introduce the following properties.\\
\begin{itemize} 
\item[$\bullet$] $QI_{X,B}(E,E',F,\varepsilon)$ : for any $x\in KK(C_0(P_E(X)), B )$, then $\mu^{\varepsilon,E,F}_{X,B}(x) = 0$ implies $q_E^{E'}(x)=0$ in $KK^G(C_0(P_{E'}(X)),B)$.
\item[$\bullet$] $QS_{X,B}(E,F,F',\varepsilon,\varepsilon')$ : for any $y\in K^{\varepsilon,F}(C^*(X,B))$, there exists $x\in KK(C_0(P_E(G)),B)$ such that $\mu^{\varepsilon',E,F'}_{X,B}(x)=\iota_{\varepsilon,F}^{\varepsilon',F'}(y)$.\\
\end{itemize} 
Let $\lambda \geq 1$ be a positive number. We say that $X$ satisfies the controlled Baum-Connes conjecture with coefficients in $B$ with rescaling $\lambda$ if :
\begin{itemize} 
\item[$\bullet$] for every $\varepsilon \in (0,\frac{1}{4\lambda})$, every $E,F\in\mathcal E$ such that $k_X(\varepsilon).E\subseteq F$, there exists $E'\in \mathcal E$ such that $E \subseteq E'$ and $ QI_{X,B}(E,E',F,\varepsilon)$ holds; 
\item[$\bullet$] for every $\varepsilon \in (0,\frac{1}{4\lambda})$, every $F\in\mathcal E$, there exists $E,F'\in\mathcal E$ such that $k_X(\varepsilon).E \subseteq F'$ and $F\subseteq F'$ and $QS_{X,B}(E,F,F',\varepsilon,\lambda\varepsilon)$ holds. 
\end{itemize} 
If $\hat\mu_{X}$, is replaced by $\hat\mu^{max}_{X}$, we will say that $X$ satisfies the maximal controlled Baum-Connes conjecture with rescaling $\lambda$.\\
\end{definition}

Recall from theorem \ref{propertiesXG} that if $X$ admits a fibred coarse embedding into Hilbert space, then $G(X)_{|\partial \beta X}$ is a-T-menable. For interesting examples of this type, recall the definition of a box space. Let $\Gamma$ be a finitely generated group, and $\mathcal N$ a family of nested normal subgroups with trivial intersection, which have finite index in $\Gamma$. Take the coarse union of the quotients to construct a coarse space $X_{\mathcal N}(\Gamma)= \cup_{H\in \mathcal N } \Gamma/ H$. Then, $X_{\mathcal N}(\Gamma)$ admits a fibred coarse embedding if and only if $\Gamma$ is a-$T$-menable. But if $X_{\mathcal N}$ is an expander, it cannot be coarsely embedded into a Hilbert space, so just take an a-$T$-menable group which has a box space $X$ which is an expander to get a coarse space that is not coarsely embeddable into Hilbert space ($SL(2,\Z)$ for instance), but admits a fibred coarse embedding.\\

The last example gives the following corollary.

\begin{cor}
Let $X$ be a coarse space that admits a fibred coarse embedding into Hilbert space. Then $X$ satisfies the maximal controlled Coarse Baum-Connes conjecture. %$\hat \mu_{X}^{max}$ is a controlled isomorphism, i.e. $X$ satisfies the controlled Coarse Baum-Connes conjecture.
\end{cor}
%
%\begin{dem}
%The maximal crossed product turns restriction of a groupoid to invariant open subsets into exact sequences of $C^*$-algebras, with $F=\partial\beta X$ and $U= F^c$, hence the following diagram commutes
%\[\begin{tikzcd}
%RK^{G_{|U}}(P_F(G_{|U}),l^\infty)\arrow{r}\arrow{d}{\mu_{G_{|U}}^{\varepsilon,E,F}} & RK^G(P_F(G),l^\infty)\arrow{r}\arrow{d}{\mu_{G}^{\varepsilon,E,F}}  & RK^{G_{|F}}(P_F(G_{|F}),l^\infty)\arrow{d}{\mu_{G_{|F}}^{\varepsilon,E,F}}  \\
%K_*^{\varepsilon,E}(l^\infty \rtimes_{max} G_{|U}) \arrow{r} & K_*^{\varepsilon,E}(l^\infty \rtimes_{max} G) \arrow{r} & K_*^{\varepsilon,E}(l^\infty \rtimes_{max} G_{|F}) \\
%\end{tikzcd}.\]
%Now, $G_{|F}$ being a-T-menable and $G_{|U}$ being proper, the two exterior vertical maps are isomorphisms, and the five lemma concludes the proof.%\\
%\qed
%\end{dem}
%%% NEW PROOF
\begin{dem}
By theorem \ref{BCCeq}, it is sufficient to show that $G(X)$ satisfies the maximal controlled Baum-Connes conjecture with coefficients in $l^\infty(X,\mathfrak K)$. We will denote $l^\infty(X,\mathfrak K)$ by $l^\infty$.\\
  
The maximal crossed product turns restriction of a groupoid to invariant open subsets into exact sequences of $C^*$-algebras, hence  
\[0\rightarrow l^\infty \rtimes_{max} G_{|U} \rightarrow l^\infty \rtimes_{max} G \rightarrow l^\infty \rtimes_{max} G_{|Y} \rightarrow 0\]
is an exact sequence, with $Y=\partial\beta X$ and $U= Y^c$. Moreover 
\[l^\infty \rtimes_{max} G_{|U}\cong l^\infty_{|U} \rtimes_{max} G
\quad \text{ and } \quad l^\infty \rtimes_{max} G_{|Y}\cong (l^\infty/l^\infty_{|U}) \rtimes_{max} G,\] 
hence $[\partial_{l^\infty \rtimes G_{|U},l^\infty\rtimes_r G}]=j_G([\partial_{l^\infty_{|U},l^\infty}]) $. 
Recall from Proposition \ref{Kasparov1} that $J_G([\partial_{l^\infty_{|U},l^\infty}])=D_{l^\infty_{|U}\rtimes_r G,l^\infty \rtimes_r G}$, hence there exists a control pair $(\alpha,k)$ such that for every $z\in RK^G(P_E(G),l^\infty / l^\infty_{|U} )$, 
\[\mu_{G}^{\varepsilon,E,F}(z\otimes [\partial_{l^\infty_{|U},l^\infty}] ) 
\sim_{\alpha,k} D_{l^\infty_{|U}\rtimes_r G,l^\infty \rtimes_r G} \circ \mu_{G}^{\varepsilon,E,F}(z )\]

Hence the following diagram commutes :
\[\begin{tikzcd}
RK^G(P_E(G),l^\infty_{|Y}) \arrow{d}{\otimes[\partial_{l^\infty_{|U},l^\infty}]} \arrow{r}{\mu_{G}^{\varepsilon,E,F}} 
			& K_*^{\varepsilon,F}(l^\infty_{|Y} \rtimes_{max} G) \arrow{d}{D_{l^\infty_{|U}\rtimes_r G,l^\infty \rtimes_r G}} \\
RK^G(P_E(G),l^\infty_{|U} )\arrow{d}\arrow{r}{\mu_{G}^{\alpha\varepsilon,E,k(\varepsilon).F}} 
			& K_*^{\alpha\varepsilon,k(\varepsilon).F}(l^\infty_{|U} \rtimes_{max} G) \arrow{d} \\
RK^G(P_E(G),l^\infty)      \arrow{d}\arrow{r}{\mu_{G}^{\alpha\varepsilon,E,k(\varepsilon).F}} 
			& K_*^{\alpha\varepsilon,k(\varepsilon).F}(l^\infty \rtimes_{max} G)      \arrow{d} \\
RK^G(P_E(G),l^\infty_{|Y}) \arrow{d}{\otimes[\partial_{l^\infty_{|U},l^\infty}]}\arrow{r}{\mu_{G}^{\alpha\varepsilon,E,k(\varepsilon).F}} 
			& K_*^{\alpha\varepsilon,k(\varepsilon).F}(l^\infty_{|Y} \rtimes_{max} G) \arrow{d}{D_{l^\infty_{|U}\rtimes_r G,l^\infty \rtimes_r G}} \\
RK^G(P_E(G),l^\infty)               \arrow{r}{\mu_{G}^{\alpha\varepsilon,E,k(\varepsilon).F}} 
			& K_*^{\alpha\varepsilon,k(\varepsilon).F}(l^\infty_{|U} \rtimes_{max} G) \\
\end{tikzcd}.\]
Now, $G_{|Y}$ being a-T-menable and $G_{|U}$ being proper, $\mu_{G_{|Y},B}$ and $\mu_{G_{|U},B}$ are isomorphisms for any $G$-algebra $B$. By theorems \ref{Quant1} and \ref{Quant2}, the families of the four exterior horizontal maps satisfies the controlled Baum-Connes conjecture, and the controlled version of the five lemma concludes the proof.\\
\qed
\end{dem}

%\subsection{Equivariant Novikov conjecture}










