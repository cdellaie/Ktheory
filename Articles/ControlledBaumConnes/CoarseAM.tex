\section{Controlled assembly maps for coarse spaces}

In this section, $X$ will be a discrete metric space with bounded geometry, and $\mathcal E$ is the coarse structure generated by its controlled subsets. We also fix a separable Hilbert space $H$. For $R>0$, $\Delta_R$ is $\{(x,y)\in X\times X\text{ s.t. }d(x,y)<R\}$. We construct a controlled assembly map for coarse space $(X,\mathcal E)$ in the same way as for groupoids.  \\

%Recall first the construction of the Roe algebra of $X$ with coefficients in a $C^*$-algebra $B$, which can be found in \cite{SkTuYu}. 
$H_B$ denotes the standard $B$-Hilbert module $H\otimes B$. Recall that for every $x,y\in X$, and $T\in\mathcal L_B(l^2(X)\otimes H_B)$, we put $T_{xy}\in\mathcal L_B(H_B)$ to be the unique operator such that $\langle T_{xy}\xi,\eta\rangle = \langle T(e_x\otimes \xi),e_y\otimes\eta\rangle $ for every $x,y\in X$ and every $\xi,\eta\in H_B$.\\
%$T_{xy}= \chi_y T\chi_x$ , where $\chi_x,\chi_y$ are the characteristic functions of $\{x\}$ and $\{y\}$, seen as projection operators.\\

%For any positive number $R>0$, define the family of linear subspaces 
%\[C_R[X,B]=\{T\in \mathcal L(l^2(X)\otimes H_B) \text{ s.t. } T_{xy}\in \mathfrak K(H_B) \text{ and }T_{xy}=0 \text{ for }(x,y)\not\in \Delta_R  \}\]
%and $C^*(X,B)$ is the completion of $\cup_{R>0} C_R[X,B]$ for the operator norm in $\mathcal L(l^2(X)\otimes H_B) $. 

Remark that the $C^*$-algebra $C^*(X,B)$ is filtered by $\mathcal E$, and also by $\R_+^*$, seen as a coarse structure. Indeed, the composition law $R\circ R'= R+R'$ provides $\R_+^*$ with a coarse structure and $C_R[X,B] C_{R'} [X,B]\subseteq C_{R+R'}[X,B]$. For the $\mathcal E$-filtration, one can put :
\[C_E[X,B]=\{T\in \mathcal L(l^2(X)\otimes H_B) \text{ s.t. } T_{xy}\in \mathfrak K(H_B) \text{ and }T_{xy}=0 \text{ for }(x,y)\not\in E \}\quad \forall E\in\mathcal E.\]

%%%%%%%%%%%%%%%%%%%%
%For $\phi : A\rightarrow B$, we use the notation $\phi_X $ for the induced $*$-homomorphism $C^*(X,A)\rightarrow C^*(X,B)$. For the reader's convenience, we give details for its construction, which is a standard fact in Coarse Geometry.

%\begin{thm}
%Let $X$ be a discrete metric space with bounded geometry and $\phi : A\rightarrow B$ a $*$-homomorphism. Then there exists a $*$-homomorphism $\phi_X : C^*(X,A)\rightarrow C^*(X,B)$ extending $\phi$. Moreover, $\phi\mapsto \phi_X$ respects composition of $*$-homomorphisms.
%\end{thm}

%\begin{dem}
%Recall that any $*$-morphism $\phi : A\rightarrow B$ induces, for any $A$-Hilbert module $E$, a $*$-morphism $\phi_* : \mathcal L_A(E)\rightarrow \mathcal L_B(E\otimes_A B)$. Now take $E$ to be $l^2(X)\otimes A$. Then $\eta\otimes a\otimes b\mapsto \eta \otimes\phi(a) b $ extends to an isometry $V\in \mathcal L_B(E\otimes_A B,l^2(X)\otimes B)$.\\
%The linear map $T \mapsto V\phi_*(T)V^*$ maps $C_R[X,A]$ into $C_R[X,B]$, and so extends to a $*$-morphism $C^*(X,A)\rightarrow C^*(X,B)$. The composition property is clear from the construction.\\
%\qed
%\end{dem}
%%%%%%%%%%%%%%%%%%%%%%%

To construct the coarse assembly map, we will need the following proposition.\\
 
Let $A$ be a $C^*$-algebra. The theorem \ref{Xfunctor} allows us to take the image of the exact sequence $0 \rightarrow SA \rightarrow CA \rightarrow A \rightarrow 0 $ under the functor $C^*(X,\cdot)$ to get the following filtered semi-split exact sequence 
\[0 \rightarrow C^*(X,SA) \rightarrow C^*(X,CA) \rightarrow C^*(X,A) \rightarrow 0.\] 
Let $D_{X,A} : \hat K_*(C^*(X,A))\rightarrow \hat K_*(C^*(X,SA))$ be the controlled boundary morphism associated to this last extension.

\begin{prop}\label{InverseEven}
Let $A$ be a $C^*$-algebra. 
%The theorem \ref{Xfunctor} allows us to take the image of the exact sequence $0 \rightarrow SA \rightarrow CA \rightarrow A \rightarrow 0 $ under the functor $C^*(X,\cdot)$ to get the following filtered semi-split exact sequence 
%\[0 \rightarrow C^*(X,SA) \rightarrow C^*(X,CA) \rightarrow C^*(X,A) \rightarrow 0.\] 
%Let $D_{X,A} : \hat K_*(C^*(X,A))\rightarrow \hat K_*(C^*(X,SA))$ the controlled boundary morphism associated to this last extension. 
Then there exists a control pair $(\lambda,h)$, independent of $X$ and $A$, such that $D_{X,A}$ is $(\lambda,h)$-invertible.
\end{prop}

\begin{dem}
Recall that $0\rightarrow \mathfrak K(l^2(\N)) \rightarrow \mathcal T_0\rightarrow S\rightarrow 0 $ is the Toeplitz extension. Let $\Psi$ be the obvious $*$-homomorphism $SC^*(X,A)\rightarrow C^*(X,SA) $. The following diagram has exact rows and commutes
\[\begin{tikzcd} 
SC^*(X,A) \arrow{d}{\Psi}\arrow{r} & CC^*(X,A) \arrow{d}\arrow{r} & C^*(X,A) \arrow{d} \\ 
C^*(X,SA) \arrow{r}          & C^*(X,CA) \arrow{r}          & C^*(X,A)
\end{tikzcd}\]
where vertical arrows are obvious inclusions. Remark \ref{rk3.8} implies that 
\[D_{X,A} = \Psi_*\circ D_{C^*(X,A)}.\] 
The following diagram also has exact rows and commutes
\[\begin{tikzcd} 
\mathfrak K(l^2(\N))\otimes C^*(X,A) \arrow{d}\arrow{r} & \mathcal T_0 \otimes C^*(X,A) \arrow{d}\arrow{r} & SC^*(X,A) \arrow{d}{\Psi} \\ 
C^*(X,\mathfrak K(l^2(\N))\otimes A) \arrow{r}          & C^*(X,\mathcal T_0 \otimes A) \arrow{r}          & C^*(X,SA)
\end{tikzcd}\]
where vertical arrows are obvious inclusions. Remark \ref{rk3.8} implies that 
\[D_{\mathfrak K(l^2(\N))\otimes C^*(X,A),\mathcal T_0\otimes C^*(X,A)} = D_{C^*(X,A),C^*(X,\mathcal T_0\otimes A)}\circ\Psi_*.\] 
A simple computation shows that 
\[D_{C^*(X,A),C^*(X,\mathcal T_0\otimes A)}\circ D_{X,A} \sim D_{\mathfrak K(l^2(\N))\otimes C^*(X,A),\mathcal T_0\otimes C^*(X,A)}\circ D_{C^*(X,A)} \sim \mathcal M_{C^*(X,A)}\]
\qed
\end{dem}

\begin{rk}\label{rkInverse} This result induces a similar statement in $K$-theory. %and in $KK$-theory similar statements. Namely : 
Namely, the boundary maps of the extensions $0 \rightarrow C^*(X,SA) \rightarrow C^*(X,CA) \rightarrow C^*(X,A) \rightarrow 0$ and
$0 \rightarrow\mathfrak K(l^2(\N))\otimes C^*(X,A) \rightarrow \mathcal T_0 \otimes C^*(X,A) \rightarrow SC^*(X,A) \rightarrow 0$ are inverse of each other in $K$-theory.
%\begin{itemize}
%\item[$\bullet$] the boundary maps of the extensions $0 \rightarrow C^*(X,SA) \rightarrow C^*(X,CA) \rightarrow C^*(X,A) \rightarrow 0$ and
%$0 \rightarrow\mathfrak K(l^2(\N))\otimes C^*(X,A) \rightarrow \mathcal T_0 \otimes C^*(X,A) \rightarrow SC^*(X,A) \rightarrow 0$ are inverse of each other in $K$-theory,  
%\item[$\bullet$] $[\partial_{K(l^2(\N))\otimes C^*(X,A), T_0 \otimes C^*(X,A)}]$ and $[\partial_{C^*(X,SA),C^*(X,CA)}]$ are $KK$-inverse of each other.
%\end{itemize}
\end{rk}

%%%%%%%%%%%%%%%%%%%%%%%%%%%%%%%%%%%%%%%%
\subsection{Controlled Roe transform}
%%%%%%%%%%%%%%%%%%%%%%%%%%%%%%%%%%%%%%%%

Every $K$-cycle $z\in KK(A,B)$ can be represented as a triplet $(H_B, \pi, T)$ where :
\begin{itemize}
\item[$\bullet$]$\pi : A\rightarrow \mathcal L_B(H_B)$ is a $*$-representation of $A$ on $H_B$.
\item[$\bullet$]$T\in \mathcal L_B(H_B)$ is a self-adjoint operator.
\item[$\bullet$] $T$ and $\pi$ satisfy the $K$-cycle condition, i.e. $[T,\pi(a)]$, $\pi(a)(T^*-T)$ and $\pi(a)(T^2-id_{H_B})$ are compact operators in $\mathfrak K_B(H_B)\cong \mathfrak K \otimes B$ for all $a\in A$.\\
\end{itemize}

We first define a controlled morphism $\hat \sigma_X(z) : \hat K(A)\rightarrow \hat K(B)$ of every $z\in KK(A,B)$, which we name the controlled Roe transform. It induces $-\otimes \sigma_X(z)$ in $K$-theory, and will be needed in the definition of the controlled coarse assembly map. Recall that if $\phi : A \rightarrow B$ is a $*$-homomorphism, we denote by $\phi_X : C^*(X,A)\rightarrow C^*(X,B)$ the induced $*$-homomorphism.

%\subsubsection{Odd case} %%%%%%%%%%%%%%

For $z\in KK_1(A,B)$, represented by $(H_B,\pi,T)\in E(A,B)$, define $P=(\frac{1+T}{2})\in \mathcal L_B(H_B)$ and 
%$P_X\in\mathcal L(H_{C^*(X,B)})$, and  
\[E^{(\pi,T)} = \{(a,P\pi(a)P + y) : a\in A,y\in  B\otimes \mathfrak K\} \]
which is a $C^*$-algebra such that the following sequence :
\[\begin{tikzcd}[column sep = small]0\arrow{r} & B\otimes \mathfrak K \arrow{r} & E^{(\pi,T)}\arrow{r} & A\arrow{r} & 0 \end{tikzcd}.\]
is exact and semi-split by the completely positive section $s : A\rightarrow B\otimes\mathfrak K ; a\mapsto P\pi(a)P$. Define $E_X = C^*(X,E^{(\pi,T)})$. Up to the $*$-isomomorphism $C^*(X,B\otimes\mathfrak K)\cong C^*(X,B)$, the following sequence
\[\begin{tikzcd}[column sep = small]0\arrow{r} & C^*(X,B) \arrow{r} & E_X^{(\pi,T)}\arrow{r} & C^*(X,A)\arrow{r} & 0 \end{tikzcd}.\]
is exact and semi-split by the completely positive section $s_X : C^*(X,A)\rightarrow E_X^{(\pi,T)}$.\\

The same proofs as Propositions \ref{ClassIndepedance}, \ref{Kasparov1} and \ref{Kasparov} yield the following results.

\begin{prop}
The controlled boundary map $D^{(\pi,T)}=D_{C^*(X,B),E_X^{(\pi,T)}}$ of the extension $E_X^{(\pi,T)}$ only depends on the class $z$.
\end{prop}

\begin{definition}
For every $z=[H_B,\pi,T]\in KK_1(A,B)$, we define the Roe transformation $\hat\sigma_X$ as 
\[\hat\sigma_X(z)= D_{C^*(X,B),E_X^{(\pi,T)}}\quad.\]
It is a $(\alpha_D,k_D)$-controlled morphism $\hat K(C^*(X,A))\rightarrow \hat K(C^*(X,B))$ of odd degree.\\

Let $z\in KK(A,B)$ be an even $K$-cycle. Recall that $[\partial_{SB}]\in KK_1(B,SB)$ is the $K$-cycle implementing the boundary of the extension $0\rightarrow SB\rightarrow CB\rightarrow B\rightarrow 0$, and $[\partial]\in KK_1(\C,S)$ is the Bott generator. Recall from proposition \ref{InverseEven} that $D_{X,A}$  and $D_{ C^*(X,A),C^*(X,\mathcal T_0\otimes A) }$ are controlled inverse of each other. We will denote $D_{ C^*(X,A),C^*(X,\mathcal T_0\otimes A) }$ by $T_{X,A}$.\\

As $z\otimes_B [\partial_{SB}]$ is an odd $K$-cycle, we define
\[\hat\sigma_X(z):= T_{X,B}\circ \hat\sigma_X(z\otimes[\partial_{SB}]).\] 
\end{definition}

The controlled Roe transform satisfies the following.

\begin{prop}\label{Roe2}
Let $A$ and $B$ two $C^*$-algebras. For every $z\in KK_*(A,B)$, there exists a control pair $(\alpha_X,k_X)$ and a $(\alpha_X,k_X)$-controlled morphism
\[\hat\sigma_X(z) : \hat K(C^*(X,A))\rightarrow \hat K(C^*(X,B))\]
of the same degree as $z$, such that
\begin{enumerate}
\item[(i)] $\hat\sigma_X(z)$ induces right multiplication by $\sigma_X(z)$ in $K$-theory ;
\item[(ii)] $\hat\sigma_X$ is additive, i.e.
\[\hat\sigma_X(z+z')=\hat\sigma_X(z)+\hat\sigma_X(z').\]
\item[(iii)] For every $*$-homomorphism $f : A_1\rightarrow A_2$,
\[\hat\sigma_X(f^*(z))=\hat\sigma_X(z)\circ f_{X,*}\] for all $z\in KK_*(A_2,B)$.
\item[(iv)] For every $*$-homomorphism $g : B_1\rightarrow B_2$,
\[\hat\sigma_X(g_*(z))= g_{X,*}\circ \hat\sigma_X(z)\] for all $z\in KK_*(A,B_1)$.
\item[(v)] $\hat\sigma_X([id_A]) \sim_{(\alpha_X,k_X)} id_{\hat K(C^*(X,A))}$,
\item[(vi)] Let $0\rightarrow J\rightarrow A\rightarrow A/J\rightarrow 0$ be a semi-split extension of $C^*$-algebras and $[\partial_J]\in KK_1(A/J,J)$ be its boundary element. Then 
\[\hat\sigma_X([\partial_{J,A}])=D_{C^*(X,J),C^*(X,A)}.\] 
\end{enumerate}
\end{prop}

We now show that the Roe transform respects in a quantitative way the Kasparov product. Let us recall the following result from \cite{lafforgue2002k}. It states that every $KK$-element comes from the product of an element coming from a $*$-homomorphism and an element coming from the inverse in $KK$-theory of a $*$-homomorphism. The following lemma is a particular case of decomposition property $d$, defined in \ref{DecompositionPropertyD}.

\begin{lem}[\cite{lafforgue2002k}, lemma $1.6.11$] Let $A$ and $B$ be two $C^*$-algebras and $z\in KK_0(A,B)$. Then, there exists a $C^*$-algebra $A_1$, an element $\alpha \in KK(A,A_1)$ and $*$-homomorphims $\theta : A_1 \rightarrow A$ and $\eta : A_1 \rightarrow B$ such that
$\theta^*(\alpha) = id_{A_1}$, $\theta_*(\alpha) = id_{A}$ and $\theta^*(z) = \eta$.
\end{lem}

\begin{prop} There exists a control pair $(\alpha_R,k_R)$ such that for every $C^*$-algebras $A$, $B$ and $C$, and every $z\in KK(A,B),z'\in KK(B,C)$, the controlled equality
\[\hat\sigma_X(z\otimes_B z') \sim_{\alpha_R,k_R} \hat\sigma_X(z')\circ \hat\sigma_X(z)\]
holds.
\end{prop}

\begin{dem}
Assume $\alpha\in KK_0(A,B)$. By naturality, the previous lemma reduces the proof to the special case of $\alpha$ being the inverse of a $*$-homomorphism $\theta : B\rightarrow A$ in $KK$-theory : $\alpha\otimes_B [\theta]=1_A$. Let $z\in KK(B,C)$ :
\[\begin{array}{rcl}
\hat\sigma_X (\alpha\otimes z) & \sim_{\alpha_J^2,k_J*k_J} &  \hat\sigma_X (\alpha\otimes z)\circ \hat\sigma_X (\alpha\otimes [\theta]) \\
			& \sim & \hat\sigma_X (\alpha\otimes z)\circ \hat\sigma_X (\theta_*(\alpha))\\
			& \sim & \hat\sigma_X (\alpha\otimes z)\circ \theta_{X,*}\circ \hat\sigma_X (\alpha)\\
			& \sim & \hat\sigma_X (\theta^*(\alpha\otimes z))\circ \hat\sigma_X (\alpha)\\
			& \sim & \hat\sigma_X (z)\circ \hat\sigma_X (\alpha) \\
\end{array}\] 
because $\theta^*(\alpha\otimes z)=\theta^*(\alpha)\otimes z=1\otimes z =z$. The control on the propagation of the first line follows from remark \ref{rk2.5} and point $(v)$, the other lines are equal by points $(iii)$ and $(iv)$, hence $(\alpha_R,k_R)$ can be taken to be $(2 \alpha_X^{4},( k_X)^{*2})$. If $z'$ is even, we can apply the same argument.\\

Let $z$ and $z'$ be odd $KK$-elements. Then :
\[\begin{array}{rcl}
\hat\sigma_X (z\otimes z') & = &  \hat\sigma_X (z\otimes_B [\partial_{B}]\otimes_{SB} [\partial_B]^{-1}\otimes_B z') \\
			& \sim & \hat\sigma_X ( [\partial_B]^{-1}\otimes_B z')\circ \hat\sigma_X (z\otimes_B [\partial_{B}])\\
			& \sim & \hat\sigma_X ( [\partial_B]^{-1}\otimes_B z')\circ \hat\sigma_X ( [\partial_B])\circ\hat\sigma_X( [\partial_B]^{-1})\circ \hat\sigma_X (z\otimes_B [\partial_{B}])\\		
			& \sim & \hat\sigma_X (  z')\circ \hat\sigma_X (z),\\
\end{array}\] 
where we used the previous case for the second line, Lemma \ref{Roe2} for the third line, and Proposition \ref{InverseEven} for the last one.\\
\qed
\end{dem}

\subsection{Controlled coarse assembly maps}

%If $(X,\mathcal E_X)$ is a coarse space, and $E\in\mathcal E_X$ a controlled subset, any simplex $\eta$ of the Rips complex $P_E(X) = \{m \in Prob(X)\text{ s.t. supp }m \subseteq E\}$ can be written as $\eta = \sum_{x\in X} \lambda_x(\eta) \delta_x$, where $\delta_x$ si the Dirac probability at $x$, and $\lambda_x : P_E(X)\rightarrow [0,1]$ is a continuous function. Set :
Let $E\in\mathcal E$ be a controlled subset. Then any probability $\eta$ of the Rips complex $P_E(X)$ can be written as $\eta = \sum_{x\in X} \lambda_x(\eta) \delta_x$, where $\delta_x$ si the Dirac probability at $x$, and $\lambda_x : P_E(X)\rightarrow [0,1]$ is a continuous function. Set :
\[ h_E : \left\{\begin{array}{rcl} X \times X & \rightarrow & C_0(P_E(X))\\  (x,y) & \mapsto & \lambda_x^{\frac{1}{2}}\lambda_y^{\frac{1}{2}}\end{array}\right. \]  
Let $(e_x)_{x\in X}$ be the canonical basis of $l^2(X)$, $e$ be a rank-one projection in $H$ and $P_E$ be defined as the extension by linearity and continuity of
\[P_E(e_x\otimes\xi\otimes f)= \sum_{y\in X} e_y\otimes (e\xi)\otimes (h(x,y)f)\] 
for every $x\in X$, $\xi\in H$ and $f\in C_0(P_E(X))$. As $\sum_{x\in X} \lambda_x =1$, $P_E$ is a projection of $\mathfrak K(l^2(X)) \otimes C_0(P_E(X))$ of controlled support : $\text{supp }P_E\subseteq E$. Indeed, $\lambda_x^{\frac{1}{2}}\lambda_y^{\frac{1}{2}} =0$ as soon as $(x,y)\notin E$. Hence $P_E$ defines a class $[P_E,0]_{\varepsilon, E'}\in K_0^{\varepsilon, E'} (C^*(X,C_0(P_E(X)))$ for any $\varepsilon\in (0,\frac{1}{4})$ and any $E'\in\mathcal E$ satisfying $E\subseteq E'$.\\

For every $C^*$-algebra $B$ and every controlled subsets $E,E'\in\mathcal E$ such that $E\subseteq E'$, the canonical inclusion $P_E(X)\hookrightarrow P_{E'}(X)$ induces a $*$-homomorphism $q_E^{E'} : C_0(P_{E'}(X))\rightarrow C_0(P_{E}(X))$, hence a map $(q_E^{E'})^* : KK(C_0(P_E(X)),B)\rightarrow KK(C_0(P_{E'}(X)),B)$ in $KK$-theory. It induces another map $((q_E^{E'})_X)_* : K(C^*(X,C_0(P_{E'}(X))))\rightarrow K(C^*(X,C_0(P_{E}(X))))$ in $K$-theory. The family of projections $P_E$ are compatible with the morphisms $q_E^{E'}$, i.e. $((q_E^{E'})_X)_*[P_{E'},0]_{\varepsilon,E'} = [P_{E},0]_{\varepsilon,E}$, for every $\varepsilon\in (0,\frac{1}{4})$.
%Moreover, the inclusion being an isometry, we have a map $K(C^*(P_E(X), B)) \rightarrow K(C^*(P_E(X), B))$, still denoted $q_E^{E'}$.

\begin{definition}
Let $B$ a $C^*$-algebra, $\varepsilon\in (0,\frac{1}{4})$ and $E,F\in\mathcal E_X$ controlled subsets such that $k_X(\varepsilon).E\subseteq F$. The controlled coarse assembly map $\hat\mu_{X,B}=(\mu_{X,B}^{\varepsilon,E,F})_{\varepsilon,E}$ is defined as the family of maps
\[\hat\mu_{X,B}^{\varepsilon, E,F} :\left\{\begin{array}{rcl} KK(C_0(P_E(X)),B) & \rightarrow & K^{\varepsilon, F}(C^*(X,B)) \\
					z & \mapsto & \iota_{\alpha_X \varepsilon',k_X(\varepsilon').F'}^{\varepsilon,F}\circ\hat\sigma_X(z)[P_{E},0]_{\varepsilon', F'}\end{array}\right.\]
where $\varepsilon'$ and $F'$ satisfy :
\begin{itemize}
\item[$\bullet$] $\varepsilon'\in (0,\frac{1}{4})$ such that $\alpha_X \varepsilon'\leq \varepsilon$,
\item[$\bullet$] and $F'\in\mathcal E$ such that $E\subseteq F'$ and $k_X(\varepsilon').F'\subseteq F$.
\end{itemize}
%are chosen not to exceed $\varepsilon$ and $E$ when composed with the propagation of the controlled morphisms. 
\end{definition}

\begin{rk} The controlled coarse assembly map is compatible with the structure morphisms $q_E^{E'}$. Indeed, for every $E,E'\in \mathcal E$ such that $E\subseteq E'$, by proposition \ref{Roe2}, 
\[\hat\sigma_X((q_E^{E'})^*(z))[P_{E'},0]_{\varepsilon,E'}  = \hat\sigma_X(z)\circ ((q_E^{E'})_X)_*[P_E',0]_{\varepsilon,E'}= \hat\sigma_X(z)[P_E,0]_{\varepsilon,E}.\] 
Hence $\hat\mu_{X,B}^{\varepsilon,E,F}\circ(q_E^{E'})^* =\hat\mu_{X,B}^{\varepsilon,E',F}$.
\end{rk}

\begin{rk} The controlled coarse assembly map is also compatible with the structure morphisms $\iota_{\varepsilon,E}^{\varepsilon',E'}$, i.e. $\iota_{\varepsilon,F}^{\varepsilon',F'}\circ\hat\mu_{X,B}^{\varepsilon,E,F} =\hat\mu_{X,B}^{\varepsilon',E,F'}$ for every $F\subseteq F'$ and $\varepsilon\leq \varepsilon'$ such that this equality is defined. 
\end{rk}

\begin{rk}According to Proposition \ref{Roe2}, $\hat\sigma_X(z)$ induces right-multiplication by $\sigma_X(z)$. Hence, the controlled coarse assembly map $\hat \mu_{X,B}$ induces the coarse assembly map $\mu_{X,B}$ in $K$-theory.
\end{rk}

\begin{rk}
This assembly map is defined for the usual Roe algebra of $X$, but could be defined for any "nice" completion of the algebraic Roe algebra $\cup_{E\in \mathcal E_X} C_E[X]$. In particular, we can define an assembly map with values in the controlled $K$-theory of the maximal Roe algebra $C_{max}
^*(X)$, that we will denote by $\hat \mu_{X}^{max}$.\end{rk}

 

































