\section{Introduction}

Let $X$ be a countable discrete metric space with bounded geometry, i.e. such that, for every $R>0$, 
\[\sup_{x\in X} |B(x,R)|<\infty.\] 
Let $H$ be the separable Hilbert space and 
\[C_R[X] = \{T\in \mathcal L(H\otimes l^2(X)) \text{ s.t. } T_{xy} \in \mathfrak K(H) \text{ and } prop(T) < R \}\]
where $prop(T) = \sup\{d(x,y) \text{ s.t. } T_{xy} \neq 0\}$. Recall that the Roe algebra of $X$ is the $C^*$-algebra defined by
\[C^*(X) = \overline{\cup_{R>0} C_R[X]}\quad ,\] 
where the closure is taken under the operator norm. The coarse Baum-Connes conjecture asserts that
\[\mu_X : \varinjlim KK(C_0(P_d(X),\C) \rightarrow K(C^*(X))\]
is an isomorphism, where:
\begin{itemize}
\item[$\bullet$] $KK(A,B)$ denotes the operator $KK$-theory of G. Kasparov,
\item[$\bullet$] $K(B)$ denotes the operator $K$-theory of the $C^*$-algebra $B$,
\item[$\bullet$] $P_d(X)$ is the Rips complex of $X$.
\end{itemize}

The main application of the coarse Baum-Connes conjecture is the Novikov conjecture on the homotopy invariance of the higher signatures. More precisely, let $\Gamma$ be a countable discrete group endowed with any left-invariant metric. Such a metric is unique up to quasi-isometry, and let us denote by $|\Gamma|$ the coarse class of the underlying metric space. Denote by $B \Gamma$ the classifying space of $\Gamma$, by $[M]\in H_{dim(M)}(M,\mathbb Q)$ the fundamental class of $M$ and by $\mathcal L_M$ the $L$-class of $M$.

\begin{thm}[Descent principle]
If $\mu_{|\Gamma|}$ is an isomorphism, then the Novikov conjecture holds for $\Gamma$, i.e. for any $x\in H^*(B\Gamma,\mathbb Q)$, any map $f:M\rightarrow B  \Gamma $, the higher signature
\[ \sigma(M,f) = \langle \mathcal L_M\cup f^*(x) ,[M]\rangle\]
is homotopy invariant.
\end{thm}  

In \cite{Yu1}, G. Yu proved that if $X$ is of finite asymptotic dimension, then the coarse Baum-Connes conjecture holds for $X$. More generally, G. Yu proved the following theorem.

\begin{thm}[G. Yu \cite{Yu2}]
If $X$ has property $A$, then the coarse Baum-Connes conjecture holds for $X$.
\end{thm}

As property $A$ is implied by finite asymptotic dimension, the last theorem is more powerful. Still, the two results are quite different in their proofs. Whereas property $A$ entails the existence of so called Dirac and Dual-Dirac elements in $KK$-theory, yielding the result, the proof in the setting of finite asymptotic dimension relies on a analog of Mayer-Vietoris decompositions on the Roe algebras. This proof is more geometric in nature, and more elementary. Quantitative $K$-theory was introduced in \cite{OY2} by H. Oyono-Oyono and G. Yu in order to broaden the domain of validity of this strategy.\\ 

In the author's thesis was introduced controlled $K$-theory. This slight generalization relies on a new definition of filtered $C^*$-algebras, which allows to treat more examples of $C^*$-algebras with the controlled $K$-theory. We define controlled assembly maps in the setting of Roe algebras and of crossed products of $C^*$-algebras by étale groupoids. These assembly maps take values in the controlled $K$-theory, and should enjoy more stability properties than the usual assembly maps. The latter is the object of future work. \\

The controlled assembly maps induces the assembly maps in $K$-theory. We study in more details this phenomenon, which gives what we call quantiative statements (theorems \ref{Quant1},\ref{Quant2} \& \ref{UniformQS}). These theorems relate the classical Baum-Connes conjecture for an étale groupoid to its controlled analog. As a byproduct, we prove in lemma \ref{prod} an interesting result on the $K$-homology of a finite $G$-simplicial complex with values in an infinite product of stable $C^*$-algebras.\\

Following the route of \cite{SkTuYu}, we show that these controlled assembly maps are related by the coarse groupoid defined by G. Skandalis, J-L. Tu and G. Yu. More precisely, out of any coarse space $X$, one can construct an étale groupoid $G(X) \rightrightarrows \beta X$, such that the coarse assembly map $\mu_{X,B}$ is equivalent to the assembly map

\[\mu_{G(X),l^\infty_B} : \varinjlim KK(C_0(P_E(G),B) \rightarrow K(l^\infty_B\rtimes_r G(X))\]
where :
\begin{itemize}
\item[$\bullet$] $\beta X$ is the Stone-\v{C}ech compactification of $X$,
\item[$\bullet$] the inductive limit is taken over the compact subsets $E\subseteq G$,
\item[$\bullet$] $l^\infty_B$ is the $G(X)$-algebra $l^\infty (X,B\otimes \mathfrak K)$ and $l^\infty_B\rtimes_r G(X)$ is the associated reduced crossed product,
\item[$\bullet$] $P_E(G)$ is the Rips complex of $G$.
\end{itemize}
These results are implied by their analog in controlled $K$-theory that we prove (theorem \ref{BCCeq}). As a corollary, we prove that any coarse space which admits a fibred coarse embedding into Hilbert space satisfies the controlled maximal coarse Baum-Connes conjecture (theorem \ref{fibred}). This is a stronger version of a result of M. Finn-Sell \cite{FinnSellFibred}. Recall that the notion of fibred embedding into Hilbert space is weaker than embedding into Hilbert space. For instance, some box spaces of $SL(2,\Z)$ are expanders, hence cannot embed into Hilbert space, but admits a fibred embedding.\\ 

The article follows the following plan. The second section present an overview of controlled $K$-theory in the setting of $C^*$-algebras filtered by what we call a coarse structure. In the third and fourth sections, we build assembly maps with values in these controlled $K$-groups, which factorizes the usual assembly maps, in the case of étale groupoids and coarse spaces respectively. The last section is devoted to applications of these results in Coarse Geometry.
%We also quickly show how controlled $K$-theory can be used for $K$-amenability of crossed products by discrete quantum groups.  

\subsection{Preliminaries}

We give a short review of the crossed product construction for étale groupoids. If $p : Y_0\rightarrow X$ and $q : Y_1\rightarrow X$ are two fibrations, we denote by $Y_0\times_{p,q} Y_1 = \{(y,y')\in Y_0\times Y_1 \text{ s.t. } p(y)=q(y')\}$ their fibred product. The non-commutative anologue of fibration are given by $C(X)$-algebras. 

\begin{definition}
A $C(X)$-algebra is given by a $C^*$-algebra $A$ and a non-degenerate $*$-homomorphism $\theta : C_0(X) \rightarrow Z(\mathcal M(A))$, where $\mathcal M(A)$ is the $C^*$-algebra of multipliers of $A$.
\end{definition}

If $(A,\theta)$ is a $C(X)$-algebra and $x\in X$, then the fiber over $X$ is defined as $A_x : = A/ \text{ker }(ev_x)A$. Two $C(X)$-algebras have a balanced tensor product $A\otimes_{C(X)} A'$, such that $(A\otimes_{C(X)} A')_x \cong A_x\otimes A'_x$. Notice that in the case of fibration, we get $C_0(Y_0\times_{p,q}Y_1) \cong C_0(Y_0)\times_{C(X)}C_0(Y_1)$. If $f:X_0 \rightarrow X_1$ is a continuous map, and $A$ is a $C(X_1)$-algebra, $f^* A :=C(X_0)\otimes_{C(X_1)} A$ is naturally a $C(X_0)$-algebra, called the pull-back of $A$ along $f$. For details in $C(X)$-algebras, the reader can consult \cite{blanchard} for instance. 
\\

Recall the following definition.

\begin{definition}
An étale groupoid is given by two topological spaces, the space of arrows $G$ and the space of units $G^{(0)}$ endowed with:
\begin{itemize}
\item[$\bullet$] continuous maps $s,r : G \rightrightarrows G^{(0)}$ which are local homeomorphisms,
\item[$\bullet$] a topological embedding $e: G^{(0)}\rightarrow G$ called the unit map, and a continuous involution $inv : G\rightarrow G; g\mapsto g^{-1}$ called the inverse map,
\item[$\bullet$] a multiplication map $G\times_{s,r}G\rightarrow G; (g,g')\mapsto gg'$ such that $(gg')g'' = g(g'g'')$, $gg^{-1}= e_{r(g)}$, $g^{-1}g= e_{s(g)}$
\end{itemize}
\end{definition}

If $A$ is a $C(G^{(0)})$-algebra, an action of the groupoid $G$ is a isomorphism of $C(G)$-algebras $\alpha : s^* A \rightarrow r^* A$ such that $\alpha_g \circ \alpha_{g'} = \alpha_{gg'}$ for every $(g,g')\in G\times_{s,r} G$. Then, define the space of continuous sections with compact support 
\[C_c(G,A) = \bigcup_{U} C_0(U)\otimes_s A\]
where $U$ runs along all open relatively compact subsets of $G$. Here $C_0(U)\otimes_s A$ denotes $C_0(U)\otimes_{C(G^{(0)})} A$, where $s$ in subscript implies that the $C(G^{(0)})$-algebra structure on $C_0(U)$ is given by $s$.\\

Endowing compact sections with convolution
\[(f_0\ast f_1)(g) = \sum_{h\in G^{r(g)}} f_0(h) \alpha_h(f_1(h^{-1}g)).\]
and involution $\overline f(g)=\alpha_g(f(g^{-1})^*)$ for every $f_0,f_1\in C_c(G,A)$, we get a $*$-algebra. 

The $A$-Hilbert module $L^2(G,A)$ is the completion of $C_c(G,A)$ under the scalar product 
\[\langle \xi ,\eta \rangle_x  = \sum_{g\in G^x} \xi(g)\overline \eta(g) \quad x\in G^{(0)} \]
and $C_c(G,A)$ is represented on $L^2(G,A)$ by $\lambda(f) \xi = f\ast \xi$, for every $ f\in C_c(G,A)$ and $\xi\in L^2(G,A)$.\\

\begin{definition}
The reduced crossed product $A\rtimes_r G$ is the $C^*$-algebra obtained by completion of $C_c(G,A)$ under the norm $||f||_r=||\lambda(f)||$.
\end{definition}

In the following section is presented the setting of controlled $K$-theory. The main example is that the set $\mathcal E$ of compact subsets of $G$ defines a coarse structure, and that $A\rtimes_r G$ is $\mathcal E$-filtered, thus allowing to apply the controlled machinery to crossed product of groupoids. 

%%%%%%%%%%%%%%%%%
%% A ordonner   %
%%%%%%%%%%%%%%%%%

\subsection{To insert into the main corpus}

\begin{rk}\label{isometry}
When we look at a $*$-homomorphism $\phi : A\rightarrow B$, we always have an isometry $V\in \mathcal L_B ( H_A\otimes_\phi B , H_B)$ defined on simple tensors as $(x_j)_j\otimes b \mapsto (\phi(x_j)b)_j$. Indeed, this map extends linearly to $H_A \odot B$, and if $x = (x_j)$ and $x'=(x'_j)$ are in $H_A$ : 
\[\langle V (x\otimes b) , V(x'\otimes b')\rangle = b^* \sum_j \phi(x_j)^* \phi(x'_j) \  b' = b^*\phi(\langle x, x' \rangle)b' = \langle x\otimes b , x'\otimes b' \rangle . \] % Determiner V^* 
This isometry can be used to explicitly describe the projection and the isomorphism appearing in the stabilization theorem in this particular example. Indeed $p = VV^*\in\mathcal L_B(H_B)$ is a projection such that $p H_B \cong H_A\otimes_\phi B $.\\

Moreover, for $T\in \mathcal L_B(H_A\otimes_\phi B)$, $Ad_V(T) = VTV^*$ defines a $*$-homomorphism $Ad_V : \mathcal L_B(H_A\otimes_\phi B)\rightarrow \mathcal L_B(H_B)$ such that $Ad_V(\mathfrak K_B(H_A\otimes_\phi B))\subseteq \mathfrak K_B(H_B)$. Indeed, notice that 
\[V\theta_{\xi,\xi'}V^* = \theta_{V\xi,V\xi'}\]
for every $\xi,\xi'\in H_A\otimes_\phi B$.\\

Composing with $\phi_*$, we get a $*$-homomorphism $\mathcal L_A(H_A)\rightarrow \mathcal L_B(H_B); T\mapsto V(T\otimes_\phi 1)V^*$ respecting compact operators in a natural way : if $\theta = (a_{ij})\in A\otimes\mathfrak K$, then $V(T\otimes_\phi 1)V^* = (\phi(a_{ij}))\in B\otimes\mathfrak K$.  
\end{rk}

\begin{lem}\label{isometryKK}
Let $\phi : B\rightarrow B'$ be a $G$-equivariant homomorphism, and $z=[H_B,\pi, T]\in KK^G(A,B)$. Let $V\in\mathcal L_{B'}(H_B\otimes B', H_{B'})$ be the isometry of the remark \ref{isometry} and $p = VV^*\in\mathcal L_{B'}(H_{B'})$. Define $\pi' : A\rightarrow \mathcal L_{B'}(H_{B'})$ as $\pi'(a) = V\pi(a)V^*$ and $T'= V(T\otimes_\phi 1)V^* + 1-p \in \mathcal L_{B'}(H_B')$. Then $(H_{B'},\pi',T')$ is a $K$-cycle and 
\[g^*(z) = [H_B\otimes B',\pi\otimes_\phi 1,\phi_*(T)]=[H_B', \pi', T']\text{ in } KK^G(A,B').\]
\end{lem} 

More precisely, let us define decomposition property $(d)$.

\begin{definition}\label{DecompositionPropertyD}
Let $d$ be a positive integer. An element $z\in KK^G(A,B)$ is said to satisfy decomposition property $(d)$ if
\begin{itemize}
\item[$\bullet$] there exist $G$-algebras $A_0$, $A_1$, ..., $A_d$ such that $A_0=A$ and $A_d=B$, 
\item[$\bullet$] there exist elements $z_j \in KK^G(A_{j},A_{j+1})$ for $j\in\{0,..,d-1\}$ such that, either $z_j$ is induced by a $G$-morphism $A_j \rightarrow A_{j+1}$, or there exists a $G$-morphism $\phi_j : A_{j+1}\rightarrow A_j$ such that $z_j \otimes_{A_{j+1}} [\phi_j] = 1_{A_j}$ and $ [\phi_j] \otimes_{A_{j}} z_j  = 1_{A_{j+1}}$,
\end{itemize}
such that $z = z_1 \otimes_{A_1}  ... \otimes_{A_{d-1}} z_{d-1} $ holds.
\end{definition}

Then, the following theorem is true for a universal constant $d$, which does not depend on the groupoid. It will be crucial to prove that the controlled Kasparov and Roe transforms, applications to be defined later, respect the Kasparov product. 

\begin{thm}[\cite{LaffOY}]\label{propertyD}
Let $G$ be a locally compact groupoid with Haar system. Then, there exists a universal constant $d$ such that every element $z\in KK^G(A,B)$ has decomposition property $(d)$.
\end{thm}

\begin{definition}
The assembly map for $G$ with coefficients in $B$ is defined as the inductive limit of the maps $\mu_{G,B}^{(Z)} : KK^G(C_0(Z),B)\rightarrow K(B\rtimes_r B)$ given by
\[\mu_{G,B}^{(Z)} (z)=[\mathcal L_Z]\otimes_{C_0(Z)\rtimes G} j_G(z),\]
that is $\mu_{G,B} = \varinjlim \mu_{G,B}^{(Z)}$ (one has to check that theses maps respects the inductive systems, which they do).\\
\end{definition}

\begin{rk}\label{projection}
By lemma \ref{Khomology}, one can restrict to $Z$ of the form $P_E(G)$ for $E\subseteq G$ compact. We will denote by $\mu_{G,B}^E$ the assembly map $\mu_{G,B}^{P_E(G)}$, and $\mu_{G,B} = \varinjlim \mu_{G,B}^{E}$ still holds.
\end{rk}

\begin{thm}\label{Xfunctor}
Let $X$ be a discrete metric space with bounded geometry and $\phi : A\rightarrow B$ a $*$-homomorphism. Then there exists a $*$-homomorphism $\phi_X : C^*(X,A)\rightarrow C^*(X,B)$ extending $\phi$. Moreover, $\phi\mapsto \phi_X$ respects composition of $*$-homomorphisms.
\end{thm}


