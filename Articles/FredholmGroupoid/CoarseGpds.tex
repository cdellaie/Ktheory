\documentclass[a4paper]{amsart}
\usepackage{amssymb,amscd,verbatim,latexsym}
\usepackage{enumerate}
\usepackage{tikz-cd}
\usepackage{tikz}
%\usepackage{mathabx}
\usepackage[all,pdf]{xy}

% Encoding and locale management
\usepackage[T1]{fontenc}
\usepackage[american]{babel}
\usepackage[utf8]{inputenc}
\usepackage{csquotes}

% Better fonts
\usepackage{lmodern}
\usepackage{microtype}

% Better bibliography management
%\usepackage[backend=biber,giveninits=true,sortcites=true,url=false,isbn=false]{biblatex}
%\usepackage{mathscinet}

% Links
\usepackage[ocgcolorlinks]{hyperref}
\hypersetup{
  colorlinks = true,
  allcolors = [rgb]{0.15, 0.37, 0.41},
  linktocpage = true}

%\usepackage{showkeys}
% \usepackage{multicol}
\usepackage[mathscr]{eucal}


%%%%%%%%%%%%%%%%%%%%%%%%%%%%%%%%%%%%%%%%%%%%%%%%%%%%%%%%%%%%%%%
\usepackage{color}
\newcommand\ede{ \, := \, }

%%
\newcommand\blue[1]{\textcolor{blue}{#1}}
\newcommand\red[1]{\textcolor{red}{#1}}
\newcommand\green[1]{\textcolor{green}{#1}}
\newcommand\magenta[1]{\textcolor{magenta}{#1}}
% \newcommand\blue[1]{#1}
% \newcommand\red[1]{}
% \newcommand\green[1]{}
% \newcommand\magenta[1]{\textcolor{magenta}{#1}}
%%%%%%%%%%%%%%%%%%%%%%%%%%%%%%%%%%%%%%%%

\newcommand\lp{`}
\newcommand\rp{'}
\newcommand\alp[1]{\`{#1}}
\newcommand\arp[1]{\'{#1}}
\newcommand\dlp{``}
\newcommand\drp{''}
\newcommand\dpp{"}
\newcommand\umlaut{\"}

%
%\let\labelc\label\renewcommand\label[1]{\mar{#1}\labelc{#1}}
%\setlength{\marginparwidth}{1.12in}
%
\newcommand{\mar}[1]{{\marginpar{\textsf{#1}}}}


%\setcounter{tocdepth}{1}
%
% OPERATORS:
%

\newcommand{\mfk}{\mathfrak}
\newcommand{\mfkg}{\mathfrak g}
\newcommand{\supp}{\operatorname{supp}}
\newcommand{\Irr}{\operatorname{Irr}}
\newcommand{\Ind}{\operatorname{Ind}}
\newcommand{\Spec}{\operatorname{Spec}}
\newcommand\sh[1]{{#1}\sp{\sharp}}
\newcommand\pullback{\sp{\downarrow\downarrow}}
\newcommand{\tto}{\rightrightarrows}
\newcommand{\st}{\rightrightarrows}
\newcommand{\dr}{\rightrightarrows}
\newcommand{\eqdef}{:=}
\newcommand\mathbfPsi{\mathbf \Psi}
\newcommand{\name}{boundary-amenable}
\newcommand{\cname}{almost continuous isotropy}
\newcommand{\Prim}{\operatorname{Prim}}


\newcommand{\pbg}{\downdownarrows}


\newcommand{\CC}{\mathbb C}
\newcommand{\HH}{\mathbb{H}}
\newcommand{\LL}{\mathbb{L}}
\newcommand{\NN}{\mathbb N}
\newcommand{\N}{\mathbb N}
\newcommand{\PP}{\mathbb P}
\newcommand{\RR}{\mathbb R}
\newcommand{\R}{\mathbb R}
\newcommand{\TT}{\mathbb T}
\newcommand{\ZZ}{\mathbb Z}

\newcommand{\CI}{{\mathcal C}^{\infty}}
\newcommand{\CIc}{{\mathcal C}^{\infty}_{\text{c}}}
\newcommand\pa{{\partial}}
\newcommand\Cstar{C\sp{\ast}}
\newcommand\Cstara{$\Cstar$-algebra}
\newcommand\Cstaras{$\Cstar$-algebras}

\newcommand\Cs[1]{C\sp{\ast}(#1)}
\newcommand\rCs[1]{C_r\sp{\ast}(#1)}

\newcommand{\ind}{\operatorname{ind}}
\newcommand{\loc}{\operatorname{loc}}
\newcommand{\ord}{\operatorname{ord}}
\newcommand{\Hom}{\operatorname{Hom}}
\newcommand{\Aut}{\operatorname{Aut}}
\newcommand{\Diff}{\operatorname{Diff}}
\newcommand{\Symb}{\operatorname{Symb}}
\newcommand{\coker}{\operatorname{coker}}
\newcommand{\cl}{\operatorname{cl}}
\newcommand{\reg}{\operatorname{reg}}
\newcommand{\vol}{\operatorname{vol}}
\renewcommand{\div}{\operatorname{div}}
\newcommand\ssub{stratified submersion}

\newcommand{\maA}{\mathcal A}
\newcommand{\maB}{\mathcal B}
\newcommand{\maC}{\mathcal C}
\newcommand{\maD}{\mathcal D}
\newcommand{\maE}{\mathcal E}
\newcommand{\maF}{\mathcal F}
\newcommand{\maG}{\mathcal G}
\newcommand{\maH}{\mathcal H}
\newcommand{\maI}{\mathcal I}
\newcommand{\maJ}{\mathcal J}
\newcommand{\maK}{\mathcal K}
\newcommand{\maL}{\mathcal L}
\newcommand{\maM}{\mathcal M}
\newcommand{\maP}{\mathcal P}
\newcommand{\maR}{\mathcal R}
\newcommand{\maS}{\mathcal S}
\newcommand{\maT}{\mathcal T}
\newcommand{\maV}{\mathcal V}
\newcommand{\maW}{\mathcal W}
\newcommand{\maO}{\mathcal O}

\newcommand\Dir{\ \backslash \! \!\! \! D}

% Remi%%
\newcommand{\A}{\mathcal A}
\newcommand{\B}{\mathcal{B}}
\newcommand{\g}{\mathfrak{g}}
\newcommand{\G}{\mathcal G}
\renewcommand{\S}{\mathbb{S}}
\newcommand{\V}{\mathcal V}
\newcommand{\Man}{\mathbf{Man}}

%\DeclareMathOperator{\Diff}{Diff}


%\newtheorem{assumption}[GlobalTheorems]{Assumption}

\newcommand\m[1]{${#1}$}
\newcommand{\fa}{\mfk{A}}
\newcommand{\faa}{\mfk{a}}

\newcommand\<{\langle}
\renewcommand\>{\rangle}
\newcommand{\ie}{{\em i.\thinspace e.,\ }}

\newcommand\AICI{\red{AICI}}

%I & D macros
%\newcommand{\1}{\textbf{1}}
\newcommand{\Ext}{{\rm Ext}}
%\newcommand{\Ind}{{\rm Ind}}
\newcommand{\Ker}{{\rm Ker}\,}
\newcommand{\de}{{\rm d}}
\def\pa{\partial}
\def\d{\partial}
%
% THEOREM TYPE ENVIRONMENTS:
%
\newtheorem{theorem}{Theorem}[section]
\newtheorem{thm}{Theorem}[section]
\newtheorem{proposition}[theorem]{Proposition}
\newtheorem{corollary}[theorem]{Corollary}
\newtheorem{conjecture}[theorem]{Conjecture}
\newtheorem{assumption}[theorem]{Assumption}
\newtheorem{lemma}[theorem]{Lemma}
\newtheorem{lm}[theorem]{Lemma}
%\theoremstyle{notation}
\newtheorem{notation}[theorem]{Notations}
\theoremstyle{definition}
\newtheorem{definition}[theorem]{Definition}
\newtheorem{defn}[theorem]{Definition}
\theoremstyle{remark}
\newtheorem{rem}[theorem]{Remark}
\newtheorem{remark}[theorem]{Remark}
\newtheorem{example}[theorem]{Example}
\newtheorem{ex}[theorem]{Example}
\newtheorem{examples}[theorem]{Examples}
%-----------------------------------------------------------------
%---Fancy Headings------------------------------------------------
%-----------------------------------------------------------------

%\pagestyle{fancy}
%\renewcommand{\sectionmark}[1]{\markright{\thesection\ #1}}
%\lhead[\thepage]{\upshape\rightmark}
%\rhead[\upshape\leftmark]{\thepage}
%\cfoot{}


%------------------------------------------------------------------
%---Authors/Thanks/Title/Date--------------------------------------
%------------------------------------------------------------------


\author[C. Dell'Aiera]{Cl\'ement Dell'Aiera}
\address{Department of Mathematics, University of Hawaii
2565 McCarthy Mall, Keller 401A
Honolulu, HI, 96822, USA}
\email{dellaiera@math.hawaii.edu }


\author[Yu Qiao]{Yu Qiao}
\address{School of Mathematics and Information Science, Shaanxi Normal University,
Xi'an, 710119, Shaanxi, China} \email{yqiao@snnu.edu.cn}

\thanks{ Qiao was partially supported by the NSFC Grant ... }

\date{\today}
\title[Coarse groupoids and exhaustive families]{Coarse Groupoids and Exhaustive Families of Representations of $C^*$-algebras}

\begin{document}

\maketitle

\begin{abstract}

\end{abstract}



\section{introduction}\label{intro}
give an overview of this article....


\bigskip

\section{Coarse groupoids}

For any discrete metric space $X$ with bounded geometry, Skandalis, Tu, and Yu associate to $X$ a coarse groupoid
$G(X)\tto \beta X$ in \cite{STY}, where $\beta X$ denotes the Stone-\v{C}ech compactification of $X$. The coarse groupoid $G(X)$ is \'etale, Hausdorff, and (\red{topologically??}) amenable, but \blue{not second countable}, and $\beta X$ is totally disconnected.
In particular, if $\Gamma$ is a finitely generated group with word-length (translation-invariant) metric,
at the level of groupoid $C^*$-algebras, we have
\begin{equation*}
l^{\infty}(\Gamma) \rtimes_r \Gamma \cong C_u^*(|\Gamma|)\cong C(\beta \Gamma) \rtimes \Gamma\cong C_r^*(\beta\Gamma \rtimes \Gamma),
\end{equation*}
where $|\Gamma|$ denotes the underlying coarse space, and $G(|\Gamma|)=\beta\Gamma \rtimes \Gamma$.

Let $X$ be a set, and $A, B$ two subsets of $X \times X$. We define
\begin{itemize}
\item $A^{-1}:=\{(y,x) \in X\times X\,| \, (x,y) \in A\}$,
\item $A \circ B:= \{(x,z) \in X \times X \, | \, \exists y \in X \,\,\text{such that} \,\, (x,y) \in A \,\, \text{and} \,\, (y,z) \in B \}$.
\end{itemize}

\begin{defn}
A coarse structure $\maE$ on $X$ is a collection of subsets of $X \times X$ such that
\begin{itemize}
\item If $A$ and $B$ are in $\maE$, so are $A^{-1}$, $A\circ B$, and $A \cup B$;
\item Every finite subset of $X \times X$ is in $X$;
\item If $A \in \maE$ and $B \subset A$, then $B \in \maE$.
\end{itemize}
Any element in $\maE$ is called an {\em entourage}.
\end{defn}

\begin{example}\label{coarsestructure}
Let $(X, d)$ be a proper metric space, i.e, such that any bounded closed ball is compact). For each positive real number $R >0$, we define
$$\Delta_R:= \{(x,y) \in X \times X \, | \, d(x,y) < R \}.$$
Then the set
$$\maE_X:= \{E \in \maP(X\times X) \, | \, \exists R>0, \,\text{such that} \,\, E \subset \Delta_R  \}$$
defines a coarse structure on $E$.
\end{example}

Let $(X,d)$ be a discrete metric space with bounded geometry, i.e., such that for any $R> 0$, $\sup_{x\in X}|B(x,R)|$ is finite, and $\maE_X$ be the coarse structure defined in Example \ref{coarsestructure}.
Recall that $\beta X$ denotes the Stone-\v{C}ech compactification of $X$. For any entourage $E$, denote by $\overline{E}$
the closure of $E$ in $\beta(X\times X)$.

\begin{defn}[\cite{STY}]
The coarse groupoid $G(X)$ associated to $X$ is defined to be
$$G(X):= \bigcup\limits_{E\in \maE_X} \, \overline{E} \tto \beta X.$$
\end{defn}

Some properties of $G(X)$....\\

\bigskip

\section{exhaustive families of representations of a \Cstara}

We recall some concepts of families of representations of a \Cstara \ \cite{Exel14, NP17, Roch03}.

\begin{defn}[Roch \cite{Roch03}]
Let $\maF$ be a family of representations of a unital \Cstara \, $A$.
\begin{enumerate}
\item The family $\maF$ is called {\em strictly spectral} if the following condition is satisfied:
\begin{equation*}
a \in A\,\, \text{is invertible if, and only if,} \,\,\phi(a) \,\, \text{is invertible for all} \,\, \phi \in \maF.
\end{equation*}
\item The family $\maF$ is called {\em strictly norming} if, for any $a\in A$, there exists $\phi \in \maF$ such that
$||a||=||\phi(a)||$.
\end{enumerate}
\end{defn}

\begin{remark}
There is a natural way to extend the above two concepts to non-unital case \cite{NP17}.
\end{remark}

The following theorem gives a characterization of a strictly spectral family \cite[Theorem 3.6]{NP17}.

\begin{thm}
Let $\maF$ be a family of representations of a unital \Cstara \, $A$. Then $\maF$ is strictly spectral if, and only if,
for any $a \in A$,
$$\Spec(a)=\cup_{\phi\in \maF} \Spec(\phi (a)).$$
\end{thm}

The following result is proved in the unital case in \cite{Roch03}, general case in \cite[Theorem 3.4]{NP17}.
\begin{thm}
Suppose that $\maF$ is a set of non-degenerate representation of a \Cstara \, $A$. Then $\maF$ is strictly spectral
if, and only if, it is strictly norming.
\end{thm}

In \cite{NP17}, Nistor and Prudhon introduce the concept of an exhaustive family of representations of a $C^*$-algebra.

\begin{defn}[Nistor and Prudhon]
Let $A$ be a \Cstara\, and $\maF$ be a set of $*$-representations of $A$. We say that $\maF$ is an exhaustive family
if $\Prim(A)=\cup_{\phi\in \maF} \supp(\phi)$, i.e., every irreducible representation of $A$ is weakly contained in some
$\phi \in \maF$.
\end{defn}

Moreover, they prove the following result in \cite[Proposition 3.12]{NP17}.

\begin{proposition}
Let $\maF$ be a family of representations of a \Cstara. If $\maF$ is exhaustive, then it is strictly spectral and hence
strictly norming.
\end{proposition}

Let $\maG \tto M$ be a locally compact groupoid with a Haar system, and denote by $\pi_x$ the regular representation of $\maG$
on $L^2(\maG_x)$. Let $\maR(\maG):=\{\pi_x\,| \, x\in M\}$, i.e., the set of regular representations.

\smallskip
In \cite{Exel14}, Exel proves the following result.

\begin{thm}
Let $\maG \tto M$ be a second countable, Hausdorff, \'etale, amenable groupoid with compact unit space $M$. Then we have that
\begin{equation*}
a \in C^*(\maG)\,\, \text{is invertible if, and only if,} \,\,\pi_x(a) \,\, \text{is invertible for all} \,\, x\in M.
\end{equation*}
In other words, $\maR(\maG)$ is a strictly spectral family for $C^*(\maG)$.
\end{thm}

This result is generalized by Nistor and Prudhon \cite[Theorem 3.18]{NP17}.

\begin{thm}
If $\maG \tto M$ is a second countable, Hausdorff, and (topological) amenable groupoid, then the family $\maR(\maG)$ of regular representations
is exhaustive, hence strictly spectral.
\end{thm}

\begin{defn}[\cite{CNY18}]
We say that $\maG$ has {\em Exel's property} if $\maR(\maG)$ is an exhaustive family of representations for $C^*_r(\maG)$, and
we say that $\maG$ has {\em strong Exel's property} if $\maR(\maG)$ is an exhaustive family of representations for $C^*(\maG)$.
\end{defn}

Denote by $\pi_0$ the vector representation of $C^*(\maG)$. Recall that a topological space is said to be {\em locally Hausdorff} if every point has a Hausdorff neighborhood. We are in position to introduce the notion of Fredholm groupoids.

\begin{defn}[\cite{CNY17,CNY18}]
Let $\maG \tto M$ be a locally compact, second countable, locally Hausdorff groupoid with a continuous Haar system. Then $\maG\tto M$ is called a {\em Fredholm groupoid} if the following two conditions are satisfied:
\begin{enumerate}
\item there is an open, dense, $\maG$-invariant subset $U$ of $M$, such that $\maG\big|_U \simeq U\times U$;
\item given $a\in C^*_r(\maG)$, we have that $1+ \pi_0(a)$ is invertible if and only if all $1+\pi_x(a)$, $x\in M \backslash U$, are invertible.
\end{enumerate}
\end{defn}

We have the following characterization of Fredholm groupoids in \cite{CNY17, CNY18}.
\begin{thm}
Suppose that $\maG \tto M$ is a locally compact, second countable, and locally Hausdorff groupoid with a continuous Haar system. Then $\maG$ is a Fredholm groupoid if and only if the following conditions are verified:
\begin{enumerate}
\item there is an open, dense, $\maG$-invariant subset $U$ of $M$, such that $\maG\big|_U \simeq U\times U$;
\item the vector representation $\pi_0: C^*_r(\maG) \rightarrow \maL(L^2(U))$ is injective;
\item the canonical projection induced an isomorphism $C^*_r(\maG)/ C^*_r(\maG_U) \simeq C^*_r(\maG_F)$;
\item the family $\{\pi_x \, | \, x\in F \}$ of representations is exhaustive for $C^*_r(\maG_F)$.
\end{enumerate}
\end{thm}


\smallskip
\red{Some Questions:
Let us go back to the coarse groupoid. Clearly, $X$ is an open, dense, invariant subset of $\beta X$, and $G(X)\big|_{X} \simeq X \times X$.
Moreover, $\partial\beta X$ is a $G(X)$-invariant and closed subset of $\beta X$, thus $G(X)\big|_{\partial \beta X}$ is a subgroupoid of $G(X)$.
It is natural to ask the following questions:
\begin{enumerate}
\item Does $\maR(G(X))$ form a strictly spectral or exhaustive family (compared to Exel's result since $G(X)$ is Hausdorff, \'etale, and amenable)?
\item Does $\maR(G(X)\big|_{\partial \beta X})$ form a strictly spectral family?
\item Under which condition, does the family $\maR(G(X)\big|_{\partial \beta X})$ have (strong) Exel property
(i.e., the coarse groupoid $G(X)$) is a Fredholm groupoid)?
\end{enumerate}
}
\newpage
\section{Box spaces}

Topological properties of the coarse groupoids encode coarse properties of the underlying metric space. For instance, $X$ has Yu's property (A) iff $G(X)$ is (topologically) amenable. In that case, the following sequence of $C^*$-algebras
\[0 \rightarrow \mathfrak K(l^2 X) \rightarrow C^*_u(X) \rightarrow C_r^*(G_{\partial \beta X}) \rightarrow 0\] 
is exact. As a consequence, $a\in C^*_u(X)$ is Fredholm iff $\lambda_w(a)\in B(l^2 G_w)$ is invertible for every $w\in \partial \beta X$ and $\sup_{\partial \beta X} \| \lambda_w (a)^{-1}\| <\infty$. This result, obvious in that language, was proven by Spakula and Willett in \cite{vspakula2017metric}, avoiding groupoid language as to provide a coarse geometric proof. (The appendix explains how the groupoid approach can shorten it.)\\

The only notions that are not always satisfied by the coarse groupoid to be Fredholm are the last two ones: metric amenability and and exhaustivity of the boundary regular representations. For \textit{metric amenability}, Willett showed in \cite{willett2015non} how to construct non topologically amenable groupoids which are metrically amenable. More precisely, for any residually finite group $\Gamma$ , one can build the so called HLS groupoid $G_{\mathcal N}(\Gamma)$, first considered in \cite{higson2002counterexamples}. Willett shows (\cite{willett2015non}, lemma 2.6) that:
\begin{itemize} 
\item[$\bullet$] $G= G_{\mathcal N}(\Gamma)$ is metrically amenable iff $R(G)$ is norming; 
\item[$\bullet$]$G_{\mathcal N}(\Gamma)$ is topologically amenable iff $\Gamma$ is amenable.
\end{itemize}
Any non-amenable group with property FD of Lubotsky and Shalom gives then such a $G_{\mathcal N}(\Gamma)$. A natural question is to adapt this result to the coarse setting by using the box space associated to $(\Gamma, \mathcal N)$. We can show that exhaustivity of $R(G_{\mathcal N})$ ensures amenability of $\Gamma$ (equivalently topological amenability of $G_{\mathcal N}$. This 

\begin{proposition}
If $R(G)$ is exhaustive for $C^*_{max}(G_{\mathcal N}(\Gamma))$, then $\Gamma$ is amenable (which is equivalent to $G$ being topologically amenable).
\end{proposition}

\begin{proof}
The trace at infinity 
\[\tau(f) = \sum_{g\in G_\infty = \{\infty\} \times \Gamma } f(g)\]
extends to an irreducible representation $C^*_{max}(G_{\mathcal N}(\Gamma)) \rightarrow \mathbb C$. But if $ker \ \lambda_x \subset ker \tau$, $x$ must be $\infty$. Indeed, $N_n \subset Ker \lambda_ n $ for all $n\in \mathbb N$. But $\lambda_\infty = \lambda_\Gamma$, hence the trace extends to $C_r^*\Gamma$, which is equivalent to $\Gamma$ being amenable. 
\end{proof}

\section{Application to inner-exactness}

Recall that, given a decomposition of the base space 
\[G^0 = U \coprod F,\]
with $U$ open (and $F$ closed...) and $G$-invariant, we have a exact sequence of $*$-algebras
\[ 0 \rightarrow C_c(G_{|U})\rightarrow  C_c(G)\rightarrow  C_c(G_{|F})\rightarrow 0. \]
Here, we used the notation $G_{S} = G_S^S = s^{-1}(S)\cap r^{-1}(S)$. While taking the maximal separation-completion preserves exactness (see Renault \cite{}), it might not be so for the reduced one, i.e. the sequence 
\[ 0 \rightarrow C_r^*(G_{|U})\rightarrow  C_r^*(G)\rightarrow  C_r^*(G_{|F})\rightarrow 0. \]
can fail to be exact. Examples were described in \cite{DellAieraWillett} (see theorem ) and include:
\begin{itemize}
\item[$\bullet$] the coarse groupoid associated to an expander $X$, with $U=X \subset \beta X$,
\item[$\bullet$] the HLS groupoid associated to an approximated group $(\Gamma,\mathcal N)$ with property $(\tau)_{\mathcal N}$.
\end{itemize}

In this section, we prove that exhaustivity of $R(G)= \{\lambda_x\}_{x\in G^0} $ prevents this from happening.

\begin{thm}
Let $G$ be a Hausdorff and locally compact groupoid with a Haar system. Suppose $R(G)$ is exhaustive, then $G$ is inner exact.
\end{thm}

This will follow from the proposition below. Here we denote by $R_F$ the family $\{ \lambda_x \}_{x\in F}$.
\begin{proposition} Let $F$ a closed minimal $G$-invriant subset of $G^0$.
If $R_F$ is exhaustive, then \[ 0 \rightarrow C_r^*(G_{|U})\rightarrow  C_r^*(G)\rightarrow  C_r^*(G_{|F})\rightarrow 0 \]
is an exact sequence of $C^*$-algebras.
\end{proposition}

\begin{proof}
Recall that $F$ begin $G$-invariant, $\lambda^{G_{|F}}_x (f) =\lambda^G_x(f), \forall f \in C_c(G_{|F})$. Let us first point out that the sequence above is exact iff
\[ \sup_{x\in F} \|\lambda_x(a) \| = \inf_{y | a-y\in C_c(G_{|U})} \sup_{x\in G^0} \| \lambda_x(y)\| \quad \forall a \in C_c(G_{|F}).  \]
The left hand side is the reduced norm for $C_c(G_{|F})$. Let us call the right hand side $\alpha$, and $C^*_\alpha(G_{|F})$ the separated-completion of $C_c(G_{|F})$ w.r.t $\alpha$. Then $\alpha(a) \leq \| a \|$ holds and the identity of $C_c(G_{|F})$ extends to a surjective $*$-homomorphism
\[C^*_\alpha(G_{|F}) \rightarrow C^*_r(G_{|F}).\]
To build a surjective $*$-homomorphism in the reverse direction, consider the $*$-representation 
\[\phi: C_\alpha^*(G_{|F}) \rightarrow C_r(G) / C_r^*(G_{|U}).\]
It is irreducible by minimality of $F$, hence by exhaustivity of $R_F$, there exists a $x\in F$ such that $ker \ \lambda_x \subset ker \ \phi$, so that the identity of $C_c(G_{|F})$ extends to a surjective $*$-homomorphism
\[C^*_r(G_{|F}) \rightarrow C^*_\alpha(G_{|F}),\]
and $C^*_r(G_{|F}) \cong C^*_\alpha(G_{|F})$.\\
\end{proof}

This gives already two classes of examples where the regular representations are not exhaustive. In particular, if $U$ is minimal in the definition of $G$ being Fredholm, these examples contradicts the hypothesis 4. 

\newpage
\section{Appendix: Induced Representations}

Let $\maG\tto X$ be a second countable, Hausdorff, \'etale groupoid with $X$ compact and $\maH \tto Y$ a closed, \'etale subgroupoid of $\maG$.
Since $\maH$ is closed and $Y=\maH \cap X$, we see that $Y$ is a closed subspace of $X$.

We describe the process of inducing representations from $\Cstar(H)$ to $\Cstar(G)$. To do so, one key step is to introduce space $C_c(\maG_Y)$.
\begin{enumerate}
\item[\text{Step 1:}]
For $\phi$ and $\psi$ in $C_c(\maG_Y)$, define $\left\langle \phi, \psi \right \rangle_{\Cstar(\maH)}$ in $C_c(\maH)$ by
\begin{equation*}
\left\langle \phi, \psi \right \rangle_{\Cstar(\maH)} \, (\gamma):= \sum_{\substack{\gamma_1\gamma_2=\gamma \\ \gamma_1,\gamma_2 \in \maG}} \,
\overline{\phi(\gamma_1^{-1})} \,\psi(\gamma_2), \quad, \forall\, \gamma\in \maH.
\end{equation*}
The above formula is justified as follows: since $d(\gamma_2)=d(\gamma)$ and $r(\gamma_1)=r(\gamma)$, we see that $d(\gamma_2)$ and $r(\gamma_1)$
are both in $Y$, therefore $\gamma_1^{-1}$ and $\gamma_2$ indeed lie in $\maG_Y$, i.e., in the domain of $\phi$ and $\psi$, respectively. One has that $C_c(\maG_Y)$ can be completed to a right $\Cstar(\maH)$-Hilbert module denoted by $M$ with appropriate right action of $\Cstar(\maH)$ on $M$. \blue{(the action should be the convolution)}

\item[\text{Step 2:}]
We next give $C_c(\maG_Y)$ a left $C_c(\maG)$-module structure as follows: for any $f \in C_c(\maG)$ and $\phi \in C_c(\maG_Y)$, define
\begin{equation*}
(f\cdot \phi)(\gamma):=(f * \phi)(\gamma)= \sum_{\substack{\gamma_1\gamma_2=\gamma \\ \gamma_1,\gamma_2 \in \maG}} \,f(\gamma_1) \, \phi(\gamma_2), \quad \forall \, \gamma\in \maG_Y.
\end{equation*}
The left-module structure may be extended to a \Cstara \, module structure
$$\Cstar(\maG) \times M \rightarrow M: \, (a, x)\rightarrow a\cdot x.$$

\item[\text{Step 3:}]
Let $\pi$ be a representation of $\Cstar(\maH)$ on a Hilbert space $H_{\pi}$. The induced representation space $H_{\Ind \pi}= C_c(\maG_Y)\otimes_{\Cstar(\maH)} H_{\pi}$, on which
 $\Cstar(\maG)$ will act, is defined to the completion of
 $$ C_c(\maG_Y) \odot H_\pi,$$
with respect to the inner product
$$\left \langle \phi\odot \xi, \psi \odot \eta \right\rangle := \left\langle \pi (\left\langle \psi, \phi \right\rangle_{\Cstar(\maH)}) \, \xi, \eta \right\rangle_{H_\pi},$$
for all $ \phi, \psi \in C_c(\maG_Y)$ and $ \xi,\eta \in H_{\pi}$, where $\odot$ is actually $\odot_{\Cstar(\maH)}$.

\item[\text{Step 4:}]
Let $f\in C_c(\maG)$, we first define $\Ind_{\Cstar(\maH)}^{\Cstar(\maG)}\pi (f)$ on the dense subspace $C_c(\maG) \odot H_\pi\subset H_{\Ind \pi}$ by the formula
$$\Ind_{\Cstar(\maH)}^{\Cstar(\maG)}\pi (f) (\phi \odot \xi):= (f \cdot \phi) \odot \xi,$$
for any $\phi \in C_c(\maG_Y)$ and $\xi \in H_{\pi}$, and then extend it by continuity to $H_{\Ind \pi}= C_c(\maG_Y)\otimes H_{\pi}$. This provides a $*$-representation of $C_c(\maG)$ on $H_{\Ind \pi}$ which, in turn, can be extended to a representation of $\Cstar(\maG)$.
\end{enumerate}

\begin{proposition}
Let $\lambda_x$ be the left-regular representation of $C^*(\maG_x^x)$ on $l^2(\maG_x^x)$.
Then $\Ind_{\maG_x^x}^{\maG}\lambda_x$ is unitarily equivalent to $\pi_x$ of $C^*(\maG)$ on $l^2(\maG_x)$.
\end{proposition}

\vspace{0.3cm}
\bibliographystyle{plain}
\bibliography{Reference}


\end{document}


