\newpage
\textbf{Notations:} We say that a topological space $Y$ is over a topological space $X$ if a continuous surjective map $Y \rightarrow X$ is given. If $Y_0$ and $Y_1$ are spaces over $X$, with corresponding maps $p_i: Y_i \rightarrow X$, we denote their fibred product by 
\[ Y_0 \times_{p_0,p_1} Y_1 = \{ (y_0,y_1) \in Y_0 \times Y_1 \text{ s.t. } p_0(y_0) = p_1(y_1) \}.\]
We will use extensively the theory of Hilbert $C^*$-modules. The interested reader can consult \cite{Lance} for reference. If $B$ is a $C^*$-algebra, and $E$ a Hilbert $B$-module, $\mathcal L_B(E)$ denotes the $C^*$-algebra of adjoinable $B$-linear continuous operators on $E$. For $\eta,\xi\in E$, $\theta_{\xi,\eta}$ refers to the so-called rank-one operator $x\mapsto \eta\langle \xi, x \rangle $. The closure of the $*$-algebra generated by all rank-one operators in $\mathcal L_B(E)$ is called the algebra of compact operators and denoted by $\mathfrak K_B(E)$. The standard Hilbert $B$-module over $B$ is denoted by 
\[H_B = \{(x_i)\subset B^\N \text{ s.t. } \sum x_i^* x_i \text{ converges in }B \}, \]
with $B$-valued inner-product given by $\langle x, y \rangle = \sum_i x_i^* y_i$. 

%%%%%%%%%%%%%%%%%%%%%%%%
\section{Introduction}
%%%%%%%%%%%%%%%%%%%%%%%%

This article presents a proof of the Künneth formula for crossed products of $C^*$-algebras by actions of a particular class of \'etale groupoids. The technique employed is an extension of the Going-Down principle developed by J. Chabert, S. Echterhoff and H. Oyono-Oyono in \cite{ChabertEOY} in the setting of locally compact groups.   \\

Recall that for $A$ and $B$ two $C^*$-algebras, one can define a homomorphism
\[\alpha_{A,B} : K_*(A)\otimes K_*(B)\rightarrow K_*(A\otimes B) \quad ; \quad (x,y)\mapsto x\otimes   \tau_A(y),\]
where $\tau_A$ is the external Kasparov product.\\

When $A$ and $B$ are unital $C^*$-algebras, if $p$ and $q$ are projections in $M_r(A)$ and $M_s(B)$ and $u$ and $v$ are unitaries in $M_r(A)$ and $M_s(B)$, then :
\[\begin{array}{rl}
\alpha_{A,B}([p]\otimes [q]) & = [p\otimes q], \\
\alpha_{A,B}([u]\otimes [q]) & = [u \otimes q +1\otimes (1-q)], \\
\alpha_{A,B}([p]\otimes [v]) & = [p\otimes v +(1-p)\otimes v] .\\
\end{array}\]

A $C^*$-algebra $A$ satisfies the Künneth formula if, for every $C^*$-algebra $B$, the following sequence 
\[\begin{tikzcd}[column sep = small] 
0 \arrow{r} & K_*(A)\otimes K_*(B)\arrow{r} & K_*(A\otimes B) \arrow{r} & Tor(K_*(A),K_*(B))\arrow{r} & 0 
\end{tikzcd}\]
is exact, where the first arrow is $\alpha_{A,B}$. It was shown in \cite{ChabertEOY} (Prop. $4.2$) that if $\alpha_{A,B}$ is an isomorphism for all $C^*$-algebras $B$ such that $K_*(B)$ is free abelian, then $A$ satisfies the Künneth formula .\\ 

The Künneth formula is known to hold for every $C^*$-algebras in the bootstrap class $\mathcal B$. Recall that $\mathcal B$ is the smallest class of separable nuclear $C^*$-algebras such that:
\begin{itemize}
\item[$\bullet$] $\C \in \mathcal B$,
\item[$\bullet$] $\mathcal B$ is closed under countable inductive limits,  
\item[$\bullet$] $\mathcal B$ is closed under $KK$-equivalence,
\item[$\bullet$] if $0 \rightarrow I \rightarrow A \rightarrow A/I \rightarrow 0$ is a short exact sequence of $C^*$-algebras and two of these are in $\mathcal B$, so is the third. 
\end{itemize}
In the groupoid setting, J-L. Tu proved (\cite{TuThese}, lemma $10.6$) that every a-T-menable groupoid $G$ has its reduced $C^*$-algebra $C^*_r(G)$ in $\mathcal B$. Some $C^*$-algebras are known not to be in $\mathcal B$ and to satisfy the Künneth formula. Indeed, if $A\in \mathcal B$, then $A$ is $KK$-equivalent to a commutative $C^*$-algebra (see \cite{blackadar}, corollary $20.10.3$). Moreover, a result of Skandalis (\cite{SkandalisNotion}) shows that if $\Gamma$ is an infinite hyperbolic property T group, then $C^*_r(\Gamma)$ cannot be KK-equivalent to a commutative $C^*$-algebra (a more recent reference for this result is theorem $6.2.1$ of the notes of N. Higson and E. Guentner \cite{HigsonGuentnerNotes}) so that $C^*_r(\Gamma)$ is not in $\mathcal B$. But corollary $0.2$ of J. Chabert, S. Echterhoff and H. Oyono-Oyono \cite{ChabertEOY} together with V. Lafforgue's result \cite{lafforgue2012conjecture} that hyperbolic group satisfy the Baum-Connes conjecture with coefficients imply that $C^*_r(\Gamma)$ satisfies the Künneth formula.\\

The goal of this work is to extend the results of \cite{ChabertEOY} to the setting of \'etale groupoids. Let us first recall these results, before stating definitions we will need about groupoid crossed-products. The result we are trying to extend is the following:

\begin{thm}[\cite{ChabertEOY}, Th. 0.1 and Cor. 0.2] Let $G$ be a locally compact group and $A$ a $G$-algebra such that:
\begin{itemize}
\item[$\bullet$] $G$ satisfies the Baum-Connes conjecture with coefficients in all $C^*$-algebras $A\otimes B$ for all $C^*$-algebras $B$ with trivial $G$-action,
\item[$\bullet$] for every $C^*$-algebra $B$, considered as a $G$-algebra with trivial action, and every compact subgroup $K$ of $G$, $A\rtimes_r K$ satisfies the Künneth formula.
\end{itemize}
Then $A\rtimes_r G$ satisfies the Künneth formula.
\end{thm}

We will obtain the following result (Th. \ref{Kunneth}). All the relevant definitions are to be found in the subsequent sections.

\begin{thm}
Let $G$ be a $\sigma$-compact étale groupoid and $A$ a $G$-algebra. Suppose that 
\begin{itemize}
\item[$\bullet$] $G$ satisfies the Baum-Connes conjecture with coefficients,
\item[$\bullet$] $G$ is in the class $\mathcal C$,
\item[$\bullet$] for every compact open subgroupoid $H$ of $G$ and every $H$-space $V$ such that the anchor map $p : V \rightarrow H^{(0)}$ is locally injective, $\alpha_{A,B}^{H,V}$ is an isomorphism for every $C^*$-algebra $B$ such that $K_*(B)$ is free.
\end{itemize} 
Then $A\rtimes_r G$ satisfies the Künneth formula.
\end{thm}

We are interested in $C^*$-algebras coming from a crossed product of a $C^*$-algebra by an action of an \'etale groupoid. Recall the following definition.

\begin{definition}
An étale groupoid is given by two locally compact Hausdorff topological spaces, the space of arrows $G$ and the space of units $G^{(0)}$ endowed with:
\begin{itemize}
\item[$\bullet$] continuous maps $s,r : G \rightrightarrows G^{(0)}$ which are local homeomorphisms,
\item[$\bullet$] a topological embedding $e: G^{(0)}\rightarrow G$ called the unit map, and a continuous involution $inv : G\rightarrow G; g\mapsto g^{-1}$ called the inverse map,
\item[$\bullet$] a continuous multiplication map $G\times_{s,r}G\rightarrow G; (g,g')\mapsto gg'$ such that $(gg')g'' = g(g'g'')$, $gg^{-1}= e_{r(g)}$, $g^{-1}g= e_{s(g)}$
\end{itemize}
\end{definition}

We will recall precise definitions in the second part of the article, but here is the general idea. Some $C^*$-algebras can be endowed with a stucture of $C(X)$-algebra, which can be thought of as a fibration of $A$ over a base space $X$. Notice that the standard notation is $C(X)$-algebra, even if the space $X$ is not compact (in which case the structure is given by an action of $C_0(X)$). For every $x\in X$, the fiber $A_x$ is a $C^*$-algebra. A groupoid can act on a $C(G^{(0)})$-algebra by an automorphism $\alpha : s^* A \rightarrow r^* A $ of $C(G)$-algebras. Here $s^* A$ denotes the pull back of along $s$, which is a $C(G)$-algebra with fibers $(s^* A)_g \cong A_{s(g)}$. Then one can construct the $*$-algebra of sections with compact support $C_c(G,A)$, which can be seen as compactly supported sections $f : G \rightarrow A$ such that $f(g)\in A_{r(g)}$. Then, the crossed product $A\rtimes_r G$ is defined as the completion of $C_c(G,A)$ with respect to the norm $||f||_r =\sup \{f \ast \xi : \xi \in C_c(G_x,A_x),||\xi||_2\leq 1,x\in G^{(0)}\}$, where $||.||_2$ is the norm on the Hilbert $A_x$-module $l^2(G_x,A_x)$. When $A= C_0(G^{(0)})$, we get the reduced $C^*$-algebra $C^*_r(G)$.\\
%$C^*$-algebras can be fibred over a base space $X$, which gives the notion of a $C(X)$-algebra. A $C(X)$-algebra $A$ has fibers $A_x$ for every $x\in X$. 
 
Many $C^*$-algebras can be described using the crossed product construction. For instance:
\begin{itemize}
\item[$\bullet$] the irrational rotation algebra, or so called noncommutative torus, can be seen as the reduced $C^*$-algebra of the action groupoid $\mathbb S^1 \rtimes \Z$, i.e.
\[C^*_r(\mathbb S^1 \rtimes \Z)\cong C(\mathbb S^1)\rtimes_r \Z ,\]
\item[$\bullet$] the Toeplitz algebra, Cuntz algebras, and AF-algebras can be seen as reduced $C^*$-algebras of well chosen groupoids, see chapter 2 of \cite{RenaultDynamical}.
\item[$\bullet$] In \cite{SkTuYu}, G. Skandalis, J-L. Tu and G. Yu introduced the coarse groupoid $G(X)$ associated to any discrete metric space with bounded geometry (more generally any coarse space with bounded geometry) $X$. The Roe algebra $C^*(X)$ and the uniform Roe algebra $C_u^*(X)$ are respectively isomorphic to $l^\infty (X,\mathfrak K)\rtimes_r G(X)$ and $C_r^*(G(X))$. 
\end{itemize}

To prove the main theorem, we will use a technique developed by J. Chabert, S. Echterhoff and H. Oyono-Oyono in \cite{ChabertEOY} called the Going Down Principle. It allows one to reduce a statement about the $K$-theory of crossed products to an analog for compact subgroups. We will adapt these techniques to the setting of groupoids. For this, we first need to define induction to $G$ of $H$-equivariant Hilbert modules with respect to a closed and open subgroupoid $H$. Then, we define a class $\mathcal C$ of groupoids, whose proper actions are locally induced from compact open subgroupoids. These groupoids behave well in topological $K$-theory with respect to restriction to compact open subgroupoids, see lemma \ref{Restriction}.

%%%%%%%%%%%%%%%%%%%%%%%%%%%%%%%%%%%%%%%%%%%%%%%%%
\section{Reminder on equivariant $KK^G$ theory}
%%%%%%%%%%%%%%%%%%%%%%%%%%%%%%%%%%%%%%%%%%%%%%%%%

We recall in this section the main definitions and results we need to handle $K$-cycles. For locally compact groups $G$, the $\Z_2$-graded abelian group $KK^G(A,B)$ was introduced by Kasparov in his seminal work on the Novikov conjecture, see \cite{KasparovNovikov} for a reference. This construction was extended to locally compact groupoids with Haar systems by P-Y. Le Gall in his thesis \cite{LeGall}. We use standard notations: if $A$ is any $C^*$-algebra, $M(A)$ is the $C^*$-algebra of multipliers of $A$, and the center of $M(A)$ is $ZM(A)$.\\

If $X$ is a locally compact Hausdorff space, a $C(X)$-algebra is a couple $(A,\theta )$ with $A$ a $C^*$-algebra and $\theta : C_0(X)\rightarrow Z M(A)$ a $*$-morphism such that $\theta(C_0(X)) A  = A$. It endows $A$ with a fibration over $X$: for every $x\in X$, the fiber $A_x$ is defined as $A/ \theta (I_x)A$ where $I_x$ is the ideal of functions that vanishes at $x$. A morphism of $C(X)$-algebras is a $*$-morphism that commutes with the corresponding actions of $C(X)$. Any such morphism $\phi : A\rightarrow B$ between $C(X)$-algebras induces for every $x\in X$ a $*$-morphism $\phi_x: A_x \rightarrow B_x$ between the fibers. If $f: X \rightarrow Y$ is a continuous map of locally compact Hausdorff spaces, then any $C(Y)$-algebra $(A,\theta)$ has a pull-back $C(X)$-algebra $(f^*A, f^* \theta)$, where $f^*\theta = \theta \circ f^*$, where $f^* : C_0(Y)\rightarrow C_b(X)$ is the map induced by $f$. Moreover, any morphism $\phi: A\rightarrow B $ of $C(Y)$-algebras induces a morphism $f^*\phi : f^*A \rightarrow f^*B $ of $C(X)$-algebras. For details on these constructions, see \cite{LeGall}.\\ 

A $G$-algebra is a triple $(A,\theta,\alpha) $ where:
\begin{itemize}
\item $A$ is a $C^*$-algebra, 
\item $\theta : C_0(G^{(0)})\rightarrow Z M(A)$ is a $*$-morphism such that $(A,\theta)$ is a $C(G^{(0)})$-algebra,  
\item $\alpha : s^*A \rightarrow r^* A$ is an isomorphism of $C(G)$-algebras such that 
\[ \alpha_{g}\circ \alpha_{g'} =\alpha_{gg'} \quad \forall (g,g')\in G^{(2)}.\]
\end{itemize}

If $(A,\theta,\alpha)$ is a $G$-algebra and $E$ a $A$-Hilbert module, $r^*E$ denotes the $r^*A$- Hilbert module $E\otimes_A r^*A$, and we similarly denote $s^*E$. The fiber over $x\in X$ is defined as $E_x = E\otimes_{A} A_x$, which inherits a structure of $A_x$-Hilbert module, and any adjoinable operator $T\in \mathcal L_A(E,F)$ between $A$-Hilbert module induces an adjoinable operator $T_x \in \mathcal L_{A_x}(E_x,F_x)$ on the fibers. \\

A $G$-Hilbert $A$-module $(E,V)$ is then defined as a Hilbert $A$-module $E$ given with a unitary $V\in \mathcal L_{s^* B}(s^* E , r^* E)$, which respects 
\begin{gather*} 
V_g V_{g'} = V_{gg'} \quad \forall (g,g')\in G^{(2)}; \\
V_g(\xi b)_{s(g)} = V_g\xi_{s(g)} . \beta_g(b_{s(g)}) \quad \forall g\in G,\xi \in E, b\in B;  
\end{gather*}
the Hilbert $s^*B$-module structure on $r^* E$ being given by the isomorphism $\beta^{-1} : r^* B \rightarrow s^* B$.\\ 

Let $A$ and $B$ be two $G$-algebras, with $G$-actions denoted respectively by $\alpha$ and $\beta$. Recall that elements of $KK^G(A,B)$ are homotopy classes of Kasparov triples. Such a triple $(E,\pi,T)$ is given by:
\begin{itemize}
\item[$\bullet$] a $\Z_2$-graded $G$-Hilbert $B$-module $(E,V)$; 
\item[$\bullet$] a $G$-equivariant $*$-morphism $\pi : A \rightarrow \mathcal L_B(E) $ of degree $0$;
\item[$\bullet$] an adjoinable operator $T\in \mathcal L_B(E)$ of degree $1$ which satisfies the $K$-cycle conditions:
\begin{gather*}
\pi(a)(T^2-1) , \pi(a)(T^*-T), [T, \pi(a)] \in \mathfrak K_B(E) \quad \forall a\in A \\
V\ s^* T \ V^*- r^* T \in \mathfrak K_{r^* E}(r^* E ).
\end{gather*}
\end{itemize} 

Then $KK_0^G(A,B)$ is the abelian group obtained by taking homotopy classes of Kasparov triples, and $KK_1^G(A,B)$ is defined as $KK_0^G(A,B\otimes \C_1)$, where $\C_1$ is the standard trivial $G$-Hilbert module $\C^2$ with grading $\begin{pmatrix} 0 & 1 \\ 1 & 0 \end{pmatrix}$. Finally, for $X$ a Hausdorff topological space endowed with a proper action of $G$, we denote 
\[RK^G(X,B) = \varinjlim_{Z\subseteq X} KK^G(C_0(Z),B)\]
where $B$ is any $G$-algebra, and the inductive limit is taken over all proper $G$-compact subspaces $Z\subseteq X$.\\

In this work, we will study actions of \'etale groupoids $G$ on their Rips complexes, which are finite $G$-simplicial complexes $\Delta$. The proof of the main theorem involves a crucial fact about the $K$-cycles in $RK^G(\Delta,B)$, where $B$ is any $G$-algebra, which is recalled in lemma \ref{JLTform}. The proof, of J-L. Tu, can be found in \cite{TuBC2}. In order to state the result, we need to recall the definitions of $G$-spaces, finite $G$-simplicial complexes, and of the Rips complex. All were defined by J-L. Tu, see \cite{TuBC2} for a reference.

\begin{definition}
A map between two topological spaces $f : X\rightarrow Y$ is said to be locally injective if there exists an open cover $\mathcal U$ of $X$ such that, for all $U\in \mathcal U$, $f_{|U}$ is injective.
\end{definition}

%%%%%
A right action of $G$ on a topological space $Z$ is given by a continuous map $p : Z \rightarrow G^{(0)}$, called the anchor map, and a continuous map $\alpha : Z\times_{p,r} G \rightarrow Z $ such that :
\begin{itemize}
\item[$\bullet$] $\alpha(\alpha(z,g),g') = \alpha(z, gg')$ whenever $(g,g')\in G^{(2)}$ and $p(z)=r(g)$,
\item[$\bullet$] $p(\alpha(z,g))= s(g)$
\item[$\bullet$] $\alpha(z,e_{p(z)})=z$
\end{itemize} We will use the notation $\alpha(g,z) = z.g$ when the action is clear from the context.\\ 

A right action $(Z,p,\alpha)$ will be abbreviated as $G$-space, when no confusion is possible. A $G$-space $Z$ is said to be 
\begin{itemize}
\item[$\bullet$] proper if $(g,z) \mapsto (z,z.g) $ is proper as a continous map,
\item[$\bullet$] free if $\alpha(g,z)=z \Rightarrow g=e_{p(z)} $,
\item[$\bullet$] $G$-compact if the quotient space $Z/G$ is compact.
\end{itemize}

Let $G$ be a locally compact $\sigma$-compact Hausdorff groupoid. A cutoff function for $G$ is a continuous function $c : G^{(0)} \rightarrow \R_+$ such that :
\begin{itemize}
\item[$\bullet$] for all compact subsets $K\subseteq G^{(0)}$, $\text{supp }(c)\cap s(G^K)$ is compact,
\item[$\bullet$] for all $x\in G^{(0)}$, $\int_{g\in G^x} c(s(g)) \lambda^x(dg) = 1$.
\end{itemize}

We recall the following proposition.

\begin{prop}[\cite{TuNovikov}]
A locally compact $\sigma$-compact Hausdorff groupoid is proper iff there exists a cutoff function for $G$. 
\end{prop}
%%%%%

\begin{definition} \label{Gcomplex}
Let $n\in\N$. A $G$-simplicial complex of dimension $\leq n$ is a pair $(X,\Delta)$ where :
\begin{itemize}
\item[$\bullet$] $X$ is a locally compact proper $G$-space, called the space of vertices, such that the anchor map $p : X\rightarrow G^{(0)}$ is locally injective;
\item[$\bullet$] $\Delta$ is a closed $G$-invariant subset of the space of measures on $X$, denoted $M_X$, endowed with the weak $*$-topology. Moreover, $\Delta$ contains only probability measures and satisfies :
\begin{itemize}
\item[$\bullet$] for all $\eta\in\Delta$, there exists $x\in G^ {(0)}$ such that $\text{supp }\eta \subseteq p^{-1}(x)$ and $|\text{supp }\eta|\leq n+1$,
\item[$\bullet$] if $\eta' \in \Delta$ and $\eta\in M_X$ such that $\text{supp }\eta \subseteq \text{supp }\eta'$, then $\eta\in \Delta$.
\end{itemize}
For $\eta\in \Delta$, $\text{supp }\eta$ is called a simplex, or a $j$-simplex when $|\text{supp }\eta | = j$.
\end{itemize}
The complex is said to be typed if there exists a finite space $T$ and a $G$-invariant continuous map $\tau : X\rightarrow T$ such that, for every simplex $S$, $\tau_{|S}$ is injective.  
\end{definition}

For any typed proper $G$-compact $G$-simplicial complex $(X,\Delta)$ of dimension $\leq n$, one can decompose it into its $n$-skeleton $Z_0\subseteq Z_1 \subseteq ... \subseteq Z_n$, where each $Z_j$ is a closed $G$-invariant subset of $\Delta$ such that :
\begin{itemize}
\item[$\bullet$] for all $\eta \in Z_j \setminus Z_{j-1}$, $|\text{supp }\eta|= j$,
\item[$\bullet$] $Z_j \setminus Z_{j-1}$ is $G$-equivariantly homeomorphic to $\mathring\sigma_j \times \Sigma_j$, where $\mathring\sigma_j $ is the interior of the standard simplex of dimension $j$, and $\Sigma_j$ is the subspace of centers of $j$-simplices.
\end{itemize}

\begin{definition}
Let $(X,\Delta)$ be a $G$-simplicial complex of dimension $\leq n$. Its barycentric subdivision is the $G$-simplicial complex $(S,\Delta_S)$ of dimension $\leq n$ defined by :
\[S = \left\{ \frac{1}{|\text{supp } \eta|}\sum_{x\in \text{supp } \eta} \delta_x\ ,\eta\in \Delta \right\} \subseteq \Delta ,\]
i.e. the space of vertices $S$ contains $X$ plus the barycentric center of any simplex of $\Delta$, and $\Delta_S $ satisfies that $\sigma = \{ \eta_0, ...,\eta_k\}$ is a simplex in $\Delta_S$ iff $\{\text{supp }(\eta_0),...,\text{supp }(\eta_k) \}$ is totally ordered for the inclusion. %$\cup_{0 \leq j \leq k}\text{ supp }(\eta_j)$ is a simplex of $\Delta$.
\end{definition}

Up to replacing a $G$-simplicial complex of dimension $\leq n$ by its barycentric subdivision, we can always suppose that it is a typed $G$-simplicial complex of dimension $\leq n$. Now is the time to present a particular example of a finite $G$-simplicial complex when $G$ is \'etale: the Rips complex of $G$.\\

Let $G$ be an \'etale groupoid. For any compact subset $K\subseteq G$, define $P_K(G)$ to be the space of probability measures $\nu $ with support contained in one and only one fiber $G^x$ for some $x\in G^{(0)}$, and such that if $g,g'\in \text{supp }(\nu)$, then $g'g^{-1}\in K$. We endow $P_K(G)$ with the weak-$*$ topology.\\

Every element $\eta\in P_K(G)$ is a finite probability measure on a fiber $G^x$, for some $x\in G^{(0)}$, hence can be written as a finite convex combination $\eta = \sum_{g\in G^{x}}\lambda_g(\eta)\delta_g$, where $\lambda_g(\eta)\in [0,1]$ for every $g$ and $\delta_g$ is the Dirac probability measure at $g\in G^x$.\\ 

Let us define a left action of $G$ on $P_K(G)$. The anchor map $p : P_K(G)\rightarrow G^{(0)}$ is the map associating to $\nu$ the only $x$ such that $\text{supp }(\nu) \subseteq G^x$. The action is defined by left translation, i.e. for every $(g,\eta)\in G\times_{s,p}P_K(G)$ : 
\[(g.\eta)(h) = \eta(g^{-1}h)\quad \forall h\in G^{r(g)}.\]

The main properties of the Rips complex are summed up in the following lemma: 

\begin{lem}[Tu,\cite{TuBC2}]\label{Gspace}]
The action of $G$ on $P_K(G)$ is proper and cocompact. Moreover, for every proper $G$-compact $G$-space $Z$, there exists a compact subset $K\subseteq G$ and a $G$-equivariant continuous map $Z\rightarrow P_K(G)$.
\end{lem}  

\begin{lem}[lemma $3.6$,\cite{TuBC2}]\label{JLTform}
Let $X$ be a $G$-compact proper $G$-space such that the anchor map $p:X\rightarrow G^{(0)}$ is locally injective, and let $B$ be a $G$-algebra. Then for every $z\in RK^G(X,B)$ there exists a $G$-proper $G$-compact space $Z$ and a $K$-cycle $(H_B, \pi, T)\in \mathbb E^G(C_0(Z),B)$ representing $z$ such that :
\begin{itemize}
\item[$\bullet$] $T$ is self-adjoint and $-1 \leq T\leq 1$,
\item[$\bullet$] $T$ is $G$-equivariant, i.e. $r^* T = V s^*T V^*$ ,
\item[$\bullet$] $T$ commutes with the action of $X$, i.e. $[\pi(a),T]= 0$ for all $a\in C_0(Z)$.
\end{itemize}
\end{lem}

Let us recall the definition of the assembly map and state the Baum-Connes conjecture. Recall that:
\begin{itemize}
\item the descent map was introduced for groups by Kasparov and generalized to the groupoid setting by Le Gall in \cite{LeGall}. It is defined as the map
\[j_G : \left\{ \begin{array}{rcl}
KK^G_*(A,B)  & \rightarrow & KK_*(A\rtimes_r G , B \rtimes_r G) \\
\ [ E,\pi, F]    & \mapsto     & [E\rtimes G, \pi\rtimes G, F\rtimes G] 
\end{array}\right.  \] 
where 
$E\rtimes G = E\otimes_{B} (B\rtimes_r G)$, 
$\pi \rtimes G: A\rtimes_r G \rightarrow \mathcal L_{A\rtimes_r G}(E\rtimes G)$ is the induced representation from the $G$-equivariant map $\pi:A \rightarrow \mathcal L_{A}(E)$, and $F\rtimes G = F\otimes id$.
\item for every proper $G$-space $Z$, there exists a projection $\mathcal L_Z \in C_0(Z)\rtimes_r G$ called the Miscenko projection. It is constructed using a cutoff function $c: Z \rightarrow \R_+$ given by properness. Indeed, if $(g,z)\in G\times_{r,p} Z$, $\mathcal L_Z(g,z)= \sqrt{c(z)c(z.g)}$ defines a projection in $C_c(G,C_0(Z))$.
\end{itemize} 
 
The maps 
\[\mu^Z_{G,B} : \left\{ \begin{array}{rcl}
KK^G_*(Y,B)  & \rightarrow & K_*(B\rtimes_r G) \\
\ z   & \mapsto     & [\mathcal L_Z]\otimes j_g(z) 
\end{array}\right.  \] 
are compatible with inductive limits of proper $G$-compact $G$-spaces $Y$. If $Z$ is a proper $G$-space, define $\mu_{G,B}^Z$ to be the inductive limit $\varinjlim_Y \mu^Y_{G,B}$, where $Y$ runs accross all proper $G$-compact subspaces of $Z$.   

The Baum-Connes assembly map for $G$ with coefficients in $B$ is then defined as the inductive limit $\mu_{G,B} = \varinjlim_Z \mu^Z_{G,B}$ over the inductive system formed by all proper $G$-spaces of $G$. Lemma \ref{Gspace} entails that this is equivalent to restricting the inductive limit to $\varinjlim_K \mu^{P_K(G)}_{G,B}$ where $K$ runs along all compact subsets of $G$.\\

The Baum-Connes conjecture for $G$ with coefficients in $B$ is the statement that $\mu_{G,B}$ is an isomorphism. Without any further precision about the coefficients, the Baum-Connes conjecture refers to the case $B=\C$. The Baum-Connes conjecture with coefficients is the statement that $\mu_{G,B}$ is an isomorphism for every $G$-algebra $B$. Already for $B=\C$, the groupoid conjecture is known to be false, see \cite{HigsonLaffSk}. The point is that it offers a framework general enough to incorporate the Baum-Connes conjecture for groups and the coarse Baum-Connes conjecture, see \cite{SkTuYu} for instance. It also allows to study the assembly maps for groups actions or foliations, or any geometrical situation in which on can associate a locally compact groupoid.		 





















