\section{Conclusion}

As a concluding remark, we would like to point out that the class $\mathcal C$ may be too restrictive, i.e. the condition that we require may be too strong. \\

Theorem \ref{Kunneth} is actually a weak version of a theorem proved in the author's thesis \cite{DellAieraThesis}, which says that the Künneth morphism $\alpha_{A\rtimes_r G,B}$ comes from a controlled morphism $\hat \alpha_{A\rtimes_r G,B}$, which is a quantitative isomorphism.

\begin{thm}[Theorem $5.2.13$ \cite{DellAieraThesis}]
Let $G$ be a $\sigma$-compact étale groupoid and $A$ a $G$-algebra. Suppose that 
\begin{itemize}
\item[$\bullet$] $G$ satisfies the Baum-Connes conjecture with coefficients,
\item[$\bullet$] $G$ is in the class $\mathcal C$,
\item[$\bullet$] for every compact open subgroupoid $H$ of $G$ and every $H$-space $V$ such that the anchor map $p : V \rightarrow H^{(0)}$ is locally injective, $\alpha_{A,B}^{H,V}$ is an isomorphism for every $C^*$-algebra $B$ such that $K_*(B)$ is free.
\end{itemize} 
Then $A\rtimes_r G$ satisfies the quantitative Künneth formula.
\end{thm}

Quantitative $K$-theory was developed by H. Oyono-Oyono and G. Yu in \cite{OY2}, and its application to the Künneth formula in \cite{OY4}. The main topic of the author's thesis \cite{DellAieraThesis} was a generalization of operator quantitative $K$-theory, called controlled $K$-theory, which allows to state that crossed products $A\rtimes_r G$ are $C^*$-algebras which are filtered by the set of symmetric compact subsets $K\subseteq G$. One can then study the controlled $K$-theory group $\hat K(A\rtimes_r G)$, which approximate $K(A\rtimes_r G)$ in a precise sense. We refer the reader to \cite{DellAieraThesis} for more details. The proof of the quantitative Künneth formula is essentially the same as the classical one. One just has to use the controlled version of every morphism involved, and has to keep track of the propagation at every steps. 


