\section{The Künneth formula in operator $K$-theory}

In this section, we present a proof of the Künneth formula for some crossed products by an étale groupoid. \\ 

Recall that for $A$ and $B$ two $C^*$-algebras, $\alpha_{A,B}$ is the homomorphism
\[\alpha_{A,B} : K_*(A)\otimes K_*(B)\rightarrow K_*(A\otimes B) \quad ; \quad (x,y)\mapsto x\otimes   \tau_A(y),\]
where $\tau_A$ is the external Kasparov product.\\

\begin{definition}
A $C^*$-algebra $A$ is said to satisfy the Künneth formula if, for every $C^*$-algebra $B$ such that $K_*(B)$ is a free abelian group, $\alpha_{A,B}$ is an isomorphism.
\end{definition}

If $A$ satisfies the Künneth formula, then, for any $C^*$-algebra $B$, one has the following exact sequence
\[\begin{tikzcd}[column sep = small] 0 \arrow{r} & K_*(A)\otimes K_*(B)\arrow{r} & K_*(A\otimes B) \arrow{r} & Tor(K_*(A),K_*(B))\arrow{r} & 0 \end{tikzcd}\]

The remainder of the section is devoted to prove the theorem \ref{Kunneth}. We first recall the induction and restriction machinery in the context of equivariant $KK$-theory. The important result is lemma \ref{Restriction}, which allows to restrict $KK^G$-elements to compact open subgroupoids under some suitable conditions. We then define a class $\mathcal C$ of groupoids which satisfies these suitable conditions, and give instances of this class. The end of the section is devoted to the proof of the main theorem.

%%%%%%%%%%%%%%%%%%%%%%%%%%
%	INDUCTION	%%
%%%%%%%%%%%%%%%%%%%%%%%%%%
%  %%%%%%%%%%%%%%%%%%%%%%%%%%%%%%%%%%%%%%%%%%%%%%%%
\subsection{Induction and Restriction functors}
%%%%%%%%%%%%%%%%%%%%%%%%%%%%%%%%%%%%%%%%%%%%%%%%

We recall the following construction from \cite{LeGall}. Let $G$ and $G'$ two étale groupoids, and $\phi=(Z,p,p')$ a generalized morphism from $G$ to $G'$. \\

For any $G'$-equivariant $B$-Hilbert module $E$, define $\phi^*E=l^2_Z\otimes_{C(G')} E$ where $l^2_Z $ is the completion of the space of sections $C_c(Z,B)$ such that $f(z)\in B_{p'(z)}$. Then $\phi^*E$ is a $G$-equivariant $\phi^*B$-Hilbert module. \\

If $\pi : A\rightarrow \mathcal L_B(E)$ and every $F\in\mathcal L_B(E)$, define $\pi' = \pi\otimes_{C(G')} id_{C_0(G^{(0)})}$, and $F'=F\otimes id_{C_0(G^{(0)})}= $. If $(E,\pi,F)\in E^{G'}(A,B)$, we use the notation $\phi^* (E,\pi,F)=(\phi^*E,\pi',F')$.\\

Le Gall proves \cite{LeGall} that $(E,\pi,F)\mapsto \phi^*(E,\pi,F)$ descends to a homomorphism of abelian groups $KK^{G'}(A,B)\rightarrow KK^{G}(\phi^*A,\phi^*B)$ which respects Kasparov products, i.e. $\phi^*(x\otimes_D y)= \phi^*(x)\otimes_{\phi^*D}\phi^*(y)$ holds for all $x,y\in KK^{G'}(A,B)$.\\

Moreover, the association $(E,\pi)\mapsto (\phi^*E, \pi')$ is functorial at the level of $C^*$-correspondences.

\begin{definition}
$G$ and $G'$ are said to be Morita equivalent when there are two generalized morphisms $Z : G\rightarrow G'$ and $Z' : G'\rightarrow G$ such that their compositions are cohomologous to identities.  
\end{definition}

If $H$ is a subgroupoid of $G$, $G$ naturally carries left and right actions of $G/H\rtimes G$ and $H$, giving rise to two generalized morphisms $\phi : G/H\rtimes G\rightarrow H$ and $\psi : H \rightarrow G/H\rtimes G$, which are inverse of each others ($\phi \psi$ and $\psi\phi$ are cohomologous to the identity). In one sentence, $G/H\rtimes G$ and $H$ are Morita equivalent.\\

If $H$ is compact, one can also see $G$ as a $H$ space, and consider the generalized inclusion  
\[\iota : 
\begin{tikzcd}[column sep = small] 
H \arrow{d} \arrow{d} & \arrow{dl} G \arrow{dr} & G\arrow{d} \arrow{d} \\
 H^{(0)} & & G^{(0)}
\end{tikzcd}\]
and $\eta$ the opposite of $\iota$.  
\begin{definition}
The restriction and induction functors are defined by
\[ \text{Res}_H^G = \iota^*;\quad \text{Ind}_H^G = \eta^*.\]
These are functorial w.r.t. $C^*$-correspondences, and also w.r.t. $KK$-theory.
\end{definition}

The Morita equivalence $G/H\rtimes H \sim H$ allows us to give a very concrete definition of these functors, and actually help to prove that, when $H$ is a compact subgroupoid of $G$ and the action of $G$ is induced by $H$ , $\text{Res}_{H}^G : KK^G(A,B)\rightarrow KK^H( \text{Res}_{H}^G A , \text{Res}_{H}^G B)$ is an isomorphism of abelian groups for any $G$-algebras $A$ and $B$. This is locally satisfied for example when $G$ is ample, i.e. étale with totally disconnected base space.

\begin{definition}
The groupoid $G$ is said to be locally induced if there exists an open cover $\mathcal U$ of $G^{(0)}$ such that for all $U\in\mathcal U$, there exists a compact groupoid $H_U$ acting on $G_{|U}$ and an ismomorphism of groupoids $G_{|U }\simeq G_{|U}/ H_U \rtimes H_U$.  
\end{definition}

Since $G_{|U}/ H_U \rtimes H_U$ is Morita equivalent to $H_U$ and $G$ is Morita equivalent to the pull-back $G[\mathcal U]$, we have an isomorphism 
\[KK^G(A,B) \simeq \prod_{U\in\mathcal U} KK^{H_U}(\phi^* A,\phi^* B).\]

Ample groupoids are example of locally induced groupoid. Recall that an étale groupoid is ample iff the open compact bisections form  a basis of the groupoid. Now take $U$ an open compact bisection, and set $H_U = \{ g\in G\text{ s.t. } gU\cap U \neq \emptyset \}$, which is a compact subgroupoid of $G$ that fulfills the condition.\\

The author presently doesn't kow of any other examples of locally induced groupoids which are not ample. Maybe looking into groupoids with finite asymptotic dimension. 
	%
%%%%%%%%%%%%%%%%%%%%%%%%%%%%%%%%%%%%%%%%%%%%%%%%%
\subsection{Induction and Restriction functors}
%%%%%%%%%%%%%%%%%%%%%%%%%%%%%%%%%%%%%%%%%%%%%%%%

\begin{definition} A subset $H\subseteq G$ is called a subgroupoid if :
\begin{itemize}
\item[$\bullet$] for every $x\in G^{(0)}$, $e_x\in H$,
\item[$\bullet$] for all $h,h'\in H $ such that $s(h') = r(h)$, $h'h \in H$,
\item[$\bullet$] if $h\in H$, $h^{-1}\in H$.
\end{itemize}
Then, the restriction of the multiplication, inverse, unit, target and source maps on $H$ defines a structure of groupoid on $H$ over $H^{(0)} = s(H)$. If $G$ is étale, $H$ is also étale. We will write $H< G$ to indicate that $H$ is a subgroupoid of $G$.
\end{definition}

In this section, we define for all subgroupoids $H < G$ induction and restriction transformations. Let $G$ be an étale groupoid and $H<G$.\\

Let $A$ be a $H$-algebra, with action given by $\alpha : s^*A \rightarrow r^* A$. Put $\phi=r\circ \iota$ where $\iota : H^{(0)}\hookrightarrow G^{(0)}$ is the canonical inclusion, hence $\phi^*A = A\otimes_\phi C_0(G)$ is a $C(G)$-algebra . Define the induced $C(G^{(0)})$-algebra :
%\[\text{Ind}_H^G (A) = \{f \in C_0(G,A) \text{ s.t. } h^{-1} f(gh) = f(g),\forall h\in H\}. \]
\[\text{Ind}_H^G (A) = \{f \in \phi^* A \text{ s.t. } \alpha_{h^{-1}}(f_{hg}) = f_g,\forall h\in H,g\in G^{s(h)}\}. \]
Left translation defines an action of $G$ on $\text{Ind}_H^G A$, so that $\text{Ind}_H^G A$ is a $G$-algebra.\\

Let $E$ be a Hilbert $A$-module, endowed with an action $V\in\mathcal L_A(s^*E,r^*E)$ of $H$. Define the induced Hilbert module :
%\[\text{Ind}_H^G (E) = \{f \in C_0(G,E) \text{ s.t. } h^{-1} f(gh) = f(g),\forall h\in H\}. \]
\[\text{Ind}_H^G (E) = \{f \in \phi^* E \text{ s.t. } \alpha_{h^{-1}}(f_{hg}) = f_g,\forall h\in H,g\in G^{s(h)}\}. \]
Left translation defines an action of $G$ on $\text{Ind}_H^G E$, hence $\text{Ind}_H^G E$ is a $G$-equivariant Hilbert $\text{Ind}_H^G A$-module.

Let $A$ and $B$ be two $H$-algebras. For all $(E,\pi,T)\in \mathbb E^G(A,B)$, define $\text{Ind}_H^G \pi$ and $\text{Ind}_H^G T$ as pointwise evaluation and multiplication by $\pi$ and $T$, i.e. $\text{Ind}_H^G T = T\otimes_\phi 1$ and $\text{Ind}_H^G \pi  = \pi\otimes_\phi id$.

\begin{definition}
For all subgroupoids $H<G$, let $A$, $B$ and $D$ be $G$-algebras and $A'$, $B'$ and $D'$ be $H$-algebras. Then, the map $(E,\pi,T)\mapsto ( \text{Ind}_H^G E, \text{Ind}_H^G\pi ,\text{Ind}_H^G T )$ induces an even homomorphism of $\Z_2$-graded abelian groups
\[\text{Ind}_H^G : KK_*^H(A',B')\rightarrow KK_*^G( \text{Ind}_H^G A', \text{Ind}_H^G B') \] 
called the induction transformation.\\
Moreover, by forgetting the action, we naturally have an even homomorphism of $\Z_2$-graded abelian groups 
\[\text{Res}_H^G : KK_*^G(A,B)\rightarrow KK_*^H( \text{Res}_H^G A, \text{Res}_H^G B) \] 
called the restriction transformation by restricting the action.\\
These two transformations respect the Kasparov product, i.e. 
\[ \text{Ind}_H^G(z\otimes z') = \text{Ind}_H^G(z)\otimes_{\text{Ind}_H^G(D')}\text{Ind}_H^G(z')\quad \forall z\in KK^H(A',D'),z'\in KK^H(D',B') \]
and 
\[ \text{Res}_H^G(z\otimes z') = \text{Res}_H^G(z)\otimes_{\text{Res}_H^G(D)}\text{Res}_H^G(z')\quad \forall z\in KK^G(A,D),z'\in KK^G(D,B) \]
The reader is refered to \cite{LeGall} for a proof.
\end{definition}   

Let $Z$ be a right $H$-space. Define on $Z\times_{p,r} G$ the following equivalence relation :
\[(z,g)\sim_H (z.h, h^{-1}g)\quad \forall z\in Z, h\in H^{p(z)},g\in G^{p(z)}.\]

\begin{definition}
The induced $G$-space of a $H$-space $Z$ is defined as $Z\times_H G = (Z\times G) / \sim_H$. 
\end{definition}

Notice that we have a natural identification between $\text{Ind}_H^G C_0(Z)$ and $C_0(Z\times_H G)$.\\

\begin{lem} \label{Restriction} Let $H$ be an open subgroupoid of $G$, and $V$ a $H$-space such that the anchor map $p : V\rightarrow H^{(0)}$ is locally injective. Then, for every $H$-algebra $A$ and every $G$-algebra $B$, the transformations $Res_H^G$ and $Ind_H^G$ induce an isomorphism of $\Z_2$-graded abelian groups :
\[RK^G( G\times_H V, B) \cong RK^H(V, Res_H^G B).\]
\end{lem}

\begin{dem}
It is clear that induction followed by restriction is the identity. For the converse, let $(E,\pi,T)\in \mathbb E^G(C_0(G\times_H V),B)$. The moment map is locally injective, hence, by lemma \ref{JLTform}, we can suppose that $T$ is self-adjoint $G$-equivariant and commutes with the action of $C_0(G\times_H V)$. As $H$ is open in $G$, $x\mapsto (e_{p(x)},x)$ is a topological embedding and $V$ can be seen as a $H$-invariant open subset in $G\times_H V$. Denote by $E_V$ the Hilbert $H$-invariant $B$-submodule of $E$ generated by 
\[\{\pi(f)\xi \ ,f\in C_0(V), \xi\in E\}.\]
Then, $E = \oplus_{g\in G/H} E_{gV}$. Notice that $E_{V}$ is a $H$-equivariant Hilbert $Res_H^G(B)$-module, such that $E\cong Ind_H^G (E_V)$. Moreover, $\pi$ is $G$-equivariant, hence $\pi(a) = Ind_H^G (\pi(a)_{|E_V} )$. As $[T,\pi(a)]=0$ for every $a\in C_0(G\times_H V)$, $T(E_V)\subseteq E_V$, and by $G$-equivariance, $T_{|E_V}$ determines $T$. Hence $T= Ind_H^G (T_{|E_V})$. Hence, if $z=[E,\pi,T]$ and $z_H =[E_V,\pi(a)_{|E_V},T_{|E_V}]$, we proved that $z = Ind_H^G( z_H)$, hence $Ind_H^G \circ Res_H^G (z)= z$.\\
\qed  
\end{dem}

%\begin{dem} 
%Let $A$ be a $H$-algebra, and $B$ a $G$-algebra.
%Let us first notice that the statement holds for equivariant $*$-homomorphisms. Namely,\\

%\begin{itemize} 
%\item[$\bullet$] if $\phi : Ind_H^G(A) \rightarrow B$ is a $G$-equivariant $*$-homomorphism, then $A = A\otimes C_0(G^{(0)})$ is a $H$-invariant subalgebra of $Ind_H^G(A)$, and the restriction $\phi_H$ of $\phi$ to $A$ is a $H$-equivariant $*$-homomorphism ;   
%\item[$\bullet$] if $\psi : A \rightarrow Res_H^G(B)$ is a $H$-equivariant $*$-homomorphism, then $\psi_G = \psi \otimes id_{C_0(G)}$ is a $H$-equivariant $*$-homomorphism.\\
%\end{itemize}

%And these constructions satisfy $(\phi_H)_G = \phi$ and $(\psi_G)_H = \phi$. Moreover, in $KK$-theory, one has $Res_H^G([\phi]) = [\phi_H]\in KK^H(Ind_H^G(A), B)$ and $Ind_H^G([\psi]) = [\psi_H]\in KK^H( A,Res_H^G (B))$.\\

%Let $z\in KK^H(Ind_H^G(A), B)$. As the restriction and induction transformations respect Kasparov products, by property $(d)$ and naturality of the boundary map, we can suppose that $z$ is the inverse in $KK$-theory of a $H$-equivariant $*$-homomorphism. Hence there exists a $H$-equivariant $*$-homomorphism $\phi : B \rightarrow Ind_H^G(A)$ such that $z\otimes_B [\phi] = 1_{Ind_H^G(A)}$ and $[\phi]\otimes_{Ind_H^G(A)} z  = 1_B$. Taking the induction, we get $Ind_H^G(z) = Ind_H^G([\phi])^{-1} = [\phi_G]^{-1}$, and $Res_H^G \circ Ind_H^G(z) = z$. We can similarly prove that $Ind_H^G \circ Res_H^G (z') = z'$ for every $z\in KK^G(A,Res_H^G(B))$.\\
%\qed
%\end{dem}

%\begin{prop} Let $H$ be an open compact subgroupoid of $G$, $U$ a $H$-space and $B$ a $H$-algebra. Then :
%\[Res_H^G : RK^G(G\times_H U , B)\rightarrow RK^H(U , Res_H^G(B)) \]
%is an isomorphism of $\Z_2$-graded abelian groups.
%\end{prop}

\subsection{Strongly proper groupoids}

We now introduce a property on groupoids that will entail a nice result on induction and restriction transformations at the level of $K$-homology.

\begin{definition}\label{StronglyProper}
A groupoid $G$ is said to be strongly proper if there exists an open cover $\mathcal U$ of $G^{(0)}$ such that, for all $U\in\mathcal U$, there exists a compact open subgroupoid $H_U$ of $G$ and a $H_U$-space $Z_U$ together with a $G$-equivariant homeomorphism
\[\psi_U : U \rightarrow G\times_{H_U} Z_U.\] 
An action of $G$ on a space $Z$ is said to be strongly proper if the groupoid $Z\rtimes G$ is strongly proper. A groupoid is said to be in the class $\mathcal C$ if every proper action of $G$ is strongly proper.
\end{definition}

\begin{rk}
For any strongly proper action of $G$ on a space $Z$, there exists an open cover of $Z$ by subsets of the type $V=G\times_H U$, where $H$ is a compact open subgroupoid and $U$ is a $H$-space. Then, by the previous section, we have an isomorphism
\[RK^G(V,B)\cong RK^H(U, Res_H^G (B))\]
for every $G$-algebra $B$. 
\end{rk}
%\begin{prop}
%Let $G$ be a strongly proper groupoid, and $\mathcal U$ an open cover satisfying the conditions of definition \ref{StronglyProper}. Then, the $G$-map $\psi_U $ extends to an isomorphism of groupoids
%\[\begin{tikzcd}
% G_{|U}\arrow{r}\arrow[shift right]{d}\arrow[shift left]{d} & Z_U\rtimes H_U \arrow[shift right]{d}\arrow[shift left]{d} \\
% U \arrow{r}{\psi_U} & G\times_{H_U} Z_U 
%\end{tikzcd}.\]
% Ce morceau bug sous windows : pourquoi ?
%\end{prop}

Let us give examples of groupoids in class $\mathcal C$. Recall the following definition from \cite{Renault} (page $20$).

\begin{definition}
A topological groupoid is said to be ample if it has a basis $G^a$ of neighborhoods consisting of compact open susbets.
\end{definition}

In \cite{paterson} (page $17$) is stated the following property. An étale groupoid $G$ is ample iff $G^{(0)}$ is totally disconnected. Hence the coarse groupoid of every coarse space $X$ is ample, its basis being $\beta X$.

\begin{prop}
Every ample groupoid is in class $\mathcal C$.
\end{prop}

\begin{dem} The following argument was explained to me by Christian Bönicke.\\ 
Let $G$ be an étale ample groupoid and $Z$ a $G$-space with proper action of $G$. Since $G$ is ample, we can cover $Z$ by compact open subsets. Let $\mathcal U$ be such an open cover. For each $U\in \mathcal U$, there exists $V\subseteq Z$ compact open such that $U= G.V$. Put $H= (r\times s)^{-1}(V,V)$, which is, by properness, a compact open subgroupoid of $Z\rtimes G$. Moreover, $U\cong G\times_H p(V)$ $G$-equivariantly. Hence $Z$ is covered by open subsets of the form $G\times_H V$, with $H$ being compact open subgroupoids and $V$ being $H$-spaces.\\ 
\qed
\end{dem}

\begin{rk}
Let $\Gamma$ be a discrete group. Then every proper action of $\Gamma$ on a space $Z$ is strongly proper by definition.
\end{rk}


%The author presently doesn't know of any other examples of strongly proper groupoids which are not ample.% Maybe looking into groupoids with finite asymptotic dimension. 

%Ample groupoids are example of locally induced groupoid. Recall that an étale groupoid is ample iff the open compact bisections form  a basis of the groupoid. Now take $U$ an open compact bisection, and set $H_U = \{ g\in G\text{ s.t. } gU\cap U \neq \emptyset \}$, which is a compact subgroupoid of $G$ that fulfills the condition.\\

%\begin{definition}
%The groupoid $G$ is said to be locally induced if there exists an open cover $\mathcal U$ of $G^{(0)}$ such that for all $U\in\mathcal U$, there exists a compact groupoid $H_U$ acting on $G_{|U}$ and an ismomorphism of groupoids $G_{|U }\simeq G_{|U}/ H_U \rtimes H_U$.  
%\end{definition}

%Since $G_{|U}/ H_U \rtimes H_U$ is Morita equivalent to $H_U$ and $G$ is Morita equivalent to the pull-back $G[\mathcal U]$, we have an isomorphism 
%\[KK^G(A,B) \simeq \prod_{U\in\mathcal U} KK^{H_U}(\phi^* A,\phi^* B).\]




 %
%%%%%%%%%%%%%%%%%%%%%%%%%%%%%%%%%%%%%%%%%%%%%%%%
\subsection{Induction and Restriction functors}
%%%%%%%%%%%%%%%%%%%%%%%%%%%%%%%%%%%%%%%%%%%%%%%%

We develop a restriction principle in order to apply the "Going Down" technique developed in \cite{ChabertEOY}. A restriction principle for groupoids has been studied in great details by Christian Bönicke in his PhD thesis (unpublished so far).

\begin{definition} A subset $H\subseteq G$ is called a subgroupoid if :
\begin{itemize}
\item[$\bullet$] for every $x\in H^{(0)}$, $e_x\in H$,
\item[$\bullet$] for all $h,h'\in H $ such that $s(h') = r(h)$, $h'h \in H$,
\item[$\bullet$] if $h\in H$, $h^{-1}\in H$.
\end{itemize}
Then, the restriction of the multiplication, inverse, unit, target and source maps on $H$ defines a structure of groupoid on $H$ over $H^{(0)} = s(H)$. If $G$ is étale, $H$ is also étale. We will write $H< G$ to indicate that $H$ is a subgroupoid of $G$.
\end{definition}

In this section, we define for all compact subgroupoids $H < G$ induction and restriction transformations. Let $G$ be an étale groupoid and $H<G$. The action of $G$ on $C_0(G)$ given by 
\[\left\{ \begin{array}{rcl}
C_0(G) & \rightarrow & C_0(G) \\
f & \mapsto  & [x\in G^{s(g)}\mapsto f(gx) ]
\end{array}\right.\] 
is called left translation.\\

Let $A$ be a $H$-algebra, with action given by $\alpha : s^*A \rightarrow r^* A$. Consider the tensor product of $C(H^{(0)})$-algebras $C_0(G)\otimes_{C(H^{(0)})} A$. Left translation provides a structure of $G$-algebra on $C_0(G)\otimes_{C(H^{(0)})} A$. Define the induced $C(G^{(0)})$-algebra by :
\[\text{Ind}_H^G (A) = \{f \in C_0(G)\otimes_{C(H^{(0)})} A \text{ s.t. } \alpha_{h^{-1}}(f_{hg}) = f_g,\forall h\in H,g\in G^{s(h)}\}. \]
Notice that $\text{Ind}_H^G A$ can be identified with the $G$-algebra of $H$-invariant element of $C_0(G)\otimes_{C(H^{(0)})} A$, which we will denote by $(C_0(G)\otimes A )^H$ for the remainder of the section.\\

Let $E$ be a Hilbert $A$-module, endowed with an action $V\in\mathcal L_A(s^*E,r^*E)$ of $H$. Define the induced Hilbert module as follows. The space $C_c(G)\otimes_s E$ is endowed with an action of $H$ by left translation on the $C_c(G)$ factor.  Let $E_0$ be the space of $H$-invariant elements of $C_c(G)\otimes E$. Define on $E_0$ the following inner product :
\[\langle\langle \xi, \eta\rangle\rangle = \int \alpha(s^*\langle \xi,\eta\rangle)d\lambda \quad \forall \xi,\eta\in E_0, \]
which gives fiberwise that 
\[\langle\langle \xi, \eta\rangle\rangle_x = \frac{1}{|H^x|}\int_{G^x} \alpha_g(\langle \xi (g),\eta (g)\rangle_{s(g)})d\lambda^x(g)\quad \forall x\in G^{(0)}.\]
Then, there exists a unique action of $A$ on $E_0$ such that 
\[(\xi .b)_{s(g)} = \xi_{s(g)}.\alpha_{g^-1}(b_{r(g)}) \quad \forall \xi \in E_0, b\in B,g\in G.\]
Indeed, up to taking a particion of unity relative to an open cover of $G$ by compact bisections, it is sufficient to define the action on $C_0(U)\otimes_s E $ for $U$ a compact bisection. Then, if $f\otimes \xi \in C_0(U)\otimes_s E $, the action of $a\in A$ is given by 
\[(f\otimes \xi ) . a = f\otimes (\xi.b ).  \]
This extends to the completion of $E_0$ with respect to the previous inner product $\langle\langle \ , \ \rangle\rangle$, hence the completion is a $A$-Hilbert module. It is endowed with a $G$-action given by left translation on the $G$ factor. Hence this completion is a $G$-equivariant Hilbert $A$-module, called the induced module and denoted by $\text{Ind}_H^G (E)$.\\  

Let $Z$ be a right $H$-space. Define on $Z\times_{p,r} G$ the following equivalence relation :
\[(z,g)\sim_H (z.h, h^{-1}g)\quad \forall z\in Z, h\in H^{p(z)},g\in G^{p(z)}.\]

Let $Z$ be a left $H$-space. Define on $G \times_{s,p} Z $ the following equivalence relation :
\[(g,z)\sim_H (gh^{-1}, h.z)\quad \forall z\in Z, h\in H_{p(z)},g\in G_{p(z)}.\]

\begin{definition}
The induced $G$-space of a left $H$-space $Z$ is defined as $ G\times_H Z = (G \times_{s,p} Z) / \sim_H$. 
\end{definition}

Notice that we have a natural identification between $\text{Ind}_H^G C_0(Z)$ and $C_0(G \times_H Z)$.\\

Let $Z$ be a $H$-space and $B$ a $H$-algebra. For all $(E,\pi,T)\in \mathbb E^H(C_0(Z),B)$, define $\tilde\pi = id\otimes \pi$ and $\tilde T$ as $1\otimes T$. Then $(Ind_H^G (E), \tilde \pi, \tilde T)\in \mathbb E^G(C_0(G\times_H V),B)$.\\

Set
\[Ind_H^G :
\left\{\begin{array}{rcl} 
RK^H(Z,B) & \rightarrow & RK^G(G\times_H Z,B) \\
\ [E,\pi,T ] & \mapsto & [ Ind_H^G (E), \tilde \pi, \tilde T ] \end{array} \right.\]

\begin{lem} \label{Restriction} Let $H$ be a compact open subgroupoid of $G$, and $V$ a $H$-space such that the anchor map $p : V\rightarrow H^{(0)}$ is locally injective. Then, for every $H$-algebra $A$ and every $G$-algebra $B$, we have an isomorphism of $\Z_2$-graded abelian groups :
\[RK^G( G\times_H V, B) \cong RK^H(V, Res_H^G B).\]
\end{lem}

\begin{dem} It is clear that induction followed by restriction is the identity.\\ 

For the converse, let $(E,\pi,T)\in \mathbb E^G(C_0(G\times_H V),B)$. The moment map is locally injective, hence, by lemma \ref{JLTform}, we can suppose that $T$ is self-adjoint $G$-equivariant and commutes with the action of $C_0(G\times_H V)$. As $H$ is open in $G$, $x\mapsto (e_{p(x)},x)$ is a topological embedding and $V$ can be seen as a $H$-invariant open subset in $G\times_H V$. Denote by $E_V$ the Hilbert $H$-invariant $B$-submodule of $E$ generated by 
\[\{\pi(f)\xi \ ,f\in C_0(V), \xi\in E\}.\]

Then $E_{V}$ is a $H$-equivariant Hilbert $Res_H^G(B)$-module, such that $E\cong Ind_H^G (E_V)$. Indeed, let $\{G_i\}_i$ be a cover of $G_{|H^{(0)}}$ by compact open bisections. For each $i$, let $\{H_{ij}\}_j$ be an open cover of $r^{-1}( s(G_i))\cap H$ by compact open bisections $H_{ij} \subseteq H$. Up to taking a subcover, we can suppose $s(G_i) = r(H_{ij})$, for every $i$ and $j$. Put $\tilde G_i = \coprod G_i \circ H_{ij}$, which is a $H$-invariant open subset of $G$, when taking right-translation of $H$ on $G$. Let $\{\phi_i\}_i$ be continuous functions $\phi_i : G^{(0)}\rightarrow [0,1]$ such that 
\[\sum_{i} \phi_i(x) = 1\quad \forall x\in r(G_{|H^{(0)}}) \quad \text{ and } \quad supp \ \phi_i\subseteq r(G_i) .\]
$C_0(\tilde G_i)$ is stable by the action of $H$ by left translation, hence $f\mapsto (\phi_i\circ r) . f$ induces a homomorphism 
\[\Big(C_0(G)\otimes_s E_V\Big)^H \rightarrow \left( C_0(\tilde G_i)\otimes_s E_V\right)^H .\] 
Moreover $C_0(\tilde G_i) = \bigoplus_j C_0(G_i\circ H_{ij})$, and the action $V\in \mathcal L_{s^*B}(s^* E, r^* E)$ induces an isomorphism
\[ E_{s(H_{ij})}=C_0(G_i\circ H_{ij})\otimes_s E \rightarrow C_0(G_i\circ H_{ij})\otimes_r E = E_{r(G_{i})}. \]
Combining these, we get an application 
\[ \Big(C_0(\tilde G_i)\otimes_s E_V\Big)^H \rightarrow (C_0(\tilde G_i)\otimes_r E)^H \cong C_0(\tilde G_i/H)\otimes_r E \]
for every $i$. Composing with the map $f\otimes \xi \mapsto f\xi$ induces
\[\Psi_i : \Big(C_0(\tilde G_i)\otimes_s E_V\Big)^H \rightarrow E_{r(G_i)}\]
for every $i$. 
Let us show that $\Psi_i$ is an isomorphism. For every $x\in r(\tilde G_i)$, there exists $g_x\in \tilde G_i$ such that $r(g_x)=x$. Hence the evaluation map induces an isomorphism 
\[\left( C_0(\tilde G_j / H) \otimes_r E \right)_x \cong_{ev_{g_i^x}} E_x.\] 
Let $x\in H^{(0)}$, and let $g_x \in \tilde G_i$ such that $r(g_x) = x$ as before. We identify elements of $(C_0(\tilde G_i)\otimes_s E_V)^H$ with continuous functions $f : \tilde G_i \rightarrow E_V$ such that $V_h(f(gh)_{s(h)})= f(g)_{r(g)}$ for any $g\in \tilde G_i$ and any $h\in H^{s(g)}$. Then 
\[ (\Psi_i)_x(f_x) = V_{g_x}\left( f(g_{x})_{s(g_x)} \right)\quad \forall f\in \left( C_0(\tilde G_i)\otimes_s E_V\right)^H, \]
which is an isomorphism. Notice that, by the $H$-equivariance of $f$, this last identity does not depend on the $g_x$ chosen, because they all differ by a right translation by an element in $H$. \\

Define $\Psi : Ind_H^G (E_V)\rightarrow E$ by $\sum_{i,g\in G_x}(\phi_i\circ r) \Psi_i$. Then 
\[(\Psi)_x(f_x) =  \sum_{i}\phi_i(x)V_{g_x}(f(g_x)_{s(g_x)}) = V_{g_x}(f(g_x)_{s(g_x)})  \]
\[\begin{array}{rl}
(\Psi)_x(f_x) & = \sum_{i,g}\phi_i(g)V_g(f(g)_{s(g)}) \\
		& = \sum_{i}\phi_i(g_x)V_{g_x}(f(g_x)_{s(g_x)}) \\
		& = \sum_{i, h\in G^x}\phi_i(g_x h)V_{g_x}(f(g_x)_{s(g_x)}) \\
		& = V_{g_x h}((f(g_x)_{s(g_x)})
\end{array}\]
for every $f\in \left( C_0( G)\otimes_{C_0(H^{(0)})} E_V\right)^H$ and any $x\in H^{(0)}$.  Hence $\Psi$ is an isomorphism.\\ %The passage from the second line to the third uses the equivariance condition on $f$. Hence $\Psi$ is an isomorphism.\\ 

%isomorphism $Ind_H^G (E_V) \cong \bigoplus_i E_{r(G_i)} \cong E $.\\
%Then, $E = \oplus_{g\in G/H} E_{gV}$. Notice that $E_{V}$ is a $H$-equivariant Hilbert $Res_H^G(B)$-module, such that $E\cong Ind_H^G (E_V)$.
%Moreover, $\pi$ is $G$-equivariant, hence $\pi(a) = Ind_H^G (\pi(a)_{|E_V} )$. As $[T,\pi(a)]=0$ for every $a\in C_0(G\times_H V)$, $T(E_V)\subseteq E_V$, and by $G$-equivariance, $T_{|E_V}$ determines $T$. Hence $T= Ind_H^G (T_{|E_V})$. Hence, if $z=[E,\pi,T]$ and $z_H =[E_V,\pi(a)_{|E_V},T_{|E_V}]$, we proved that $z = Ind_H^G( z_H)$, hence $Ind_H^G \circ Res_H^G (z)= z$.\\
\qed  
\end{dem}

\subsection{Strongly proper groupoids}

We now introduce a property on groupoids that will entail a nice result on induction and restriction transformations at the level of $K$-homology.

\begin{definition}\label{StronglyProper}
A groupoid $G$ is said to be strongly proper if there exists an open cover $\mathcal U$ of $G^{(0)}$ such that, for all $U\in\mathcal U$, there exists a compact open subgroupoid $H_U$ of $G$ and a $H_U$-space $Z_U$ together with a $G$-equivariant homeomorphism
\[\psi_U : U \rightarrow G\times_{H_U} Z_U.\] 
An action of $G$ on a space $Z$ is said to be strongly proper if the groupoid $Z\rtimes G$ is strongly proper. A groupoid is said to be in the class $\mathcal C$ if every proper action of $G$ is strongly proper.
\end{definition}

\begin{rk}
For any strongly proper action of $G$ on a space $Z$, there exists an open cover of $Z$ by subsets of the type $V=G\times_H U$, where $H$ is a compact open subgroupoid and $U$ is a $H$-space. Then, by the previous section, we have an isomorphism
\[RK^G(V,B)\cong RK^H(U, Res_H^G (B))\]
for every $G$-algebra $B$. 
\end{rk}

Let us give examples of groupoids in class $\mathcal C$. Recall the following definition from \cite{Renault} (page $20$).

\begin{definition}
A topological groupoid is said to be ample if it has a basis $G^a$ of neighborhoods consisting of compact open subsets.
\end{definition}

In \cite{paterson} (page $17$) is stated the following property. An étale groupoid $G$ is ample iff $G^{(0)}$ is totally disconnected. Hence the coarse groupoid of every coarse space $X$ is ample, its basis being $\beta X$.

\begin{prop}
Every ample groupoid is in class $\mathcal C$.
\end{prop}

\begin{dem} The following argument is an adaptation of the lemmas $2.41$ and $2.42$ of \cite{TuNonHaus}.\\% I owe to Christian Bönicke the knowledge of these results.\\ 

Let $G$ be an étale ample groupoid and $Z$ a $G$-space with proper action of $G$. Let $z_0\in Z$ and let $W$ be a compact open neighborhood of $x_0=p(z_0)\in G^{(0)}$. Let $F$ be the stabilizer of $z_0$. By properness, it is a finite group. We can suppose $W$ small enough to satisfy that, for any $g\in F$, there exist bisections $U_g \subseteq G$ such that $W \subseteq s(U_g)$. Denote by $\rho_g : W\rightarrow$ corresponding local sections of $s$ such that $\rho_g(x_0)=g$. We denote by $\alpha_g =r\circ \rho_g$ the corresponding partial homeomorphisms. By continuity of the product, we can suppose $W$ small enough to satisfy 
\[\rho_{g'}(\alpha_g(x))\rho_g(x) = \rho_{g'g}(x) \quad\forall x\in W\]
for all $g$ and $g'$ composable in $F$. Set $W_0 = \cap_{g\in F}\alpha_g(W)$, which is a $F$-invariant neighborhood of $x_0$. It is endowed with an action of $F$ by $g.w=\alpha_g(w)$, and 

\[\phi : \left\{ \begin{array}{rcl}
W_0\rtimes F & \rightarrow & G_{|W_0} \\ 
(w,g) & \mapsto & \alpha_g(w) \end{array}\right.\] 

defines a morphism of groupoids. Then $H= \phi(W_0\rtimes F)$ is a compact open subgroupoid of $G$.\\

There exists a neighborhood of $z_0$ such that $V\cap \alpha_g(V) \neq \emptyset \Rightarrow g\in H$. Indeed, denote by $\alpha : G\times_{s,p} Z\rightarrow Z$ the action and let $C$ be  $(G\times U) \setminus (H\times U)$. Hence $(z_0,z_0)\notin \alpha(C)$, hence there exists a neighborhood $V$ of $z_0$ such that $V\times V \subseteq \alpha(C)$. This $V$ satisfies the previous condition. \\

Let $K\subseteq V$ be a compact $H$-invariant neighborhood of $z_0$. Then   
\[\Psi : \left\{ \begin{array}{rcl}
G\times_H K & \rightarrow & G.K \\ 
(g,y) & \mapsto & g.y \end{array}\right.\]
is well defined, continuous and $G$-equivariant. If $gy =g'y' $, then $g^{-1}g'\in H$, hence $\Psi$ is bijective. As the action on $K$ being proper, $G\times_{s,p} K \rightarrow K\times GK$ is closed. Moreover, $K$ being compact, $pr_2 : K\times GK \rightarrow GK$ is closed. Hence $\Psi$ is closed by composition, hence a $G$-equivariant homeomorphism. \\
\qed
\end{dem}

%\begin{rk}
%Let $\Gamma$ be a discrete group. Then every proper action of $\Gamma$ on a space $Z$ is strongly proper by definition.% hence $\Gamma$ is in $\mathcal C$.
%\end{rk}

%%%%%%%%%%%%%%%%%%%%%%%%%%

%%%%%%%%%%%%%%%%%%%%%%%%%%%%%%%%%%%%%%%%%%%%%%%%%%%%
\section{Baum-Connes and the Künneth formula}
%%%%%%%%%%%%%%%%%%%%%%%%%%%%%%%%%%%%%%%%%%%%%%%%%%%%
The first step in proving theorem \ref{Kunneth} is to define an analytical version of $\alpha_{A,B}$ when an étale groupoid $G$ is given. More precisely, we first construct a homomorphism $\alpha_{A,B}^G : K_*^{top}(G,A)\otimes K_*(B)\rightarrow K_*^{top}(G,A\otimes B )$, inductive limit of $\alpha_{A,B}^{G,Z} : RK^G(Z,A)\otimes K_*(B)\rightarrow RK^G(Z,A\otimes B )$ where $Z$ runs through $G$-proper $G$-compact spaces. Then we show that the assembly map intertwines $\alpha^{G,P_E(G)}_{A,B}$ and $\alpha_{A\rtimes_r G,B}$.\\

If $A$ is a $G$-algebra, and $B$ a $C^*$-algebra, $A\otimes B$ naturally inherits a $G$-algebra structure with trivial action of $G$ on the $B$ factor. Then $(A\rtimes_r G)\otimes B \cong (A\otimes B)\rtimes_r G$.\\

Let $Z$ be a $G$-proper space. Define the homomorphism :
\[\alpha_{A,B}^{G,Z} : RK^G_*(Z,A)\otimes K_*(B)\rightarrow RK_*^G(Z,A\otimes B) \quad ; \quad (x,y)\mapsto x\otimes_{}   \tau_A(y),\]
which respects inductive limits w.r.t. inclusions of $G$-proper spaces, so that it induces
\[\alpha_{A,B}^G : K_*^{top}(G,A)\otimes K_*(B)\rightarrow K_*^{top}(G,A\otimes B ).\]

To prove theorem \ref{Kunneth}, we will need the following result.

\begin{thm}\label{TopologicalKunneth}
Let $G$ be an étale groupoid in the class $\mathcal C$, and let $E\in\mathcal E$ be a controlled subset of $G$ and $P_E(G)$ be the corresponding Rips complex. If, for all compact open subgroupoids $H$ of $G$ and every $H$-space $V$ such that the anchor map $p : V\rightarrow H^{(0)}$ is locally injective, $\alpha_{A,B}^{H,V}$ is an isomorphism, then $\alpha_{A,B}^{G,P_E(G)}$ is an isomorphism for all $C^*$-algebras $B$ such that $K_*(B)$ is a free abelian group.
\end{thm}

\begin{dem}
Let $Z_0\subseteq Z_1\subseteq ... \subseteq Z_n $ be the skeleton decomposition of $P_E(G)$.\\

Let us prove by induction that $\alpha_j=\alpha^{G,Z_j}_{A,B}$ is an isomorphism. By a standard Mayer-Vietoris type argument, %similar to the proof of theorem \ref{prod}, 
it is sufficient to prove the statement for $j=0$.\\

Let $U\subseteq Z_0$ be a $G$-compact $G$-proper space. By strong properness, $U$ can be finitely covered by open subsets of the type $G \times_H V$. By a standard Mayer-Vietoris argument, we can suppose that there exists a compact open subgroupoid $H$ of $G$ and a compact $H$-space $V$ such that $U = G \times_H V$. The following diagram is commutative :
\[\begin{tikzcd}[column sep = small] 
RK_*^G(U,A)\otimes K_*(B) \arrow{r}{\alpha_{A,B}^{G,U}} \arrow{d}{\text{Res}_H^G} & RK^G(U, A\otimes B) \arrow{d}{\text{Res}_H^G}\\
RK_*^H(V, A)\otimes K_*(B) \arrow{r}{\alpha_{ A,B}^{H,V}} & 
	RK_*^H(V,  A\otimes B)\\
\end{tikzcd}\]

The vertical arrows are isomorphisms by Proposition \ref{Restriction}, the last horizontal one is by hypothesis. Hence $\alpha_{A,B}^{G,U}$ is an isomorphism. By taking the inductive limit on $G$-proper $G$-compact spaces $U\subseteq Z_0$, we get that $\alpha_{A,B}^{G,Z_0}$ is an isomorphism.
\end{dem}

\begin{lem}\label{KunnethLemma}
For every étale groupoid $G$, any $G$-algebra $A$, and any $C^*$-algebra $B$, the following diagram commutes :
\[\begin{tikzcd}
RK_*^G(P_E(G),A)\otimes K_*(B) \arrow{r}{\alpha^{G,P_E(G)}_{A,B}}\arrow{d}{\mu_{G,A}\otimes \text{id}_{K_*(B)}} & 
RK_*^G(P_E(G),A\otimes B) \arrow{d}{\mu_{G,A}} \\
K_*(A\rtimes_r G)\otimes K_*(B) \arrow{r}{\alpha_{A\rtimes_r G,B}} & 
K_*((A\otimes B)\rtimes_r G)\\
\end{tikzcd}\] 
for all $E\subseteq G$ compact. 
\end{lem}

\begin{dem}
Let $z\in RK_*^G(P_E(G),A)$ and $y\in K_*(B)$. As the action of $G$ on $B$ is trivial, $(A\rtimes_r G)\otimes B\simeq (A\otimes B)\rtimes_r G$ so that $j_G(z \otimes \tau_A(y))= j_G(z) \otimes \tau_{A\rtimes_r G}(y)$ holds. This entails
\[\alpha_{A\rtimes_r G, B}(\mu_{G,A}(z)\otimes y)= [\mathcal L_E] \otimes j_G(z)\otimes \tau_{A\rtimes_r G}(y) = 
[\mathcal L_E] \otimes j_G( z\otimes \tau_A(y) ) = \mu_{G,A\otimes B}(\alpha^{G,P_E(G)}_{A,B}(z\otimes y)) ,\]
which is just the statement of the lemma.\\
\qed
\end{dem}

We can now prove the main theorem of the section.

\begin{thm}\label{Kunneth}
Let $G$ be a $\sigma$-compact étale groupoid and $A$ a $G$-algebra. Suppose that 
\begin{itemize}
\item[$\bullet$] $G$ satisfies the Baum-Connes conjecture with coefficients,
\item[$\bullet$] $G$ is in the class $\mathcal C$,
\item[$\bullet$] for every compact open subgroupoid $H$ of $G$ and every $H$-space $V$ such that the anchor map $p : V \rightarrow H^{(0)}$ is locally injective, $\alpha_{A,B}^{H,V}$ is an isomorphism for every $C^*$-algebra $B$ such that $K_*(B)$ is free.
\end{itemize} 
Then $A\rtimes_r G$ satisfies the Künneth formula.
\end{thm}

\begin{dem} %\textit{of theorem \ref{Kunneth}}\\

Let $A$ be a $G$-algebra and $B$ a $C^*$-algebra, seen as a $G$-algebra with trivial action, such that $K(B)$ is free. As $(A\otimes B)\rtimes_r G \cong (A\rtimes_r G)\otimes B$, we will identify them in the remaining of the proof. Let us prove that $\alpha_{A\rtimes_r G,B}$ is an isomorphism.\\

Lemma \ref{KunnethLemma} entails that the following diagram commutes
\[\begin{tikzcd}
K_*^{top}(G,A)\otimes K_*(B) \arrow{r}{\alpha^{G,top}_{A,B}}\arrow{d}{\mu_{G,A}\otimes \text{id}_{K_*(B)}} & 
K_*^{top}(G,A\otimes B) \arrow{d}{\mu_{G,A}} \\
K_*(A\rtimes_r G)\otimes K_*(B) \arrow{r}{\alpha_{A\rtimes_r G,B}} & 
K_*((A\otimes B)\rtimes_r G)
\end{tikzcd}\]
with vertical arrows being isomorphism, because $G$ satisfies Baum-Connes with coefficients. Thus, it suffices to show that $\alpha^{G,P_E(G)}_{A,B}$ is an isomorphism for every compact $E\subset G$. But this follows from the last two hypothesis of the theorem together with Theorem \ref{TopologicalKunneth}.\\ 
\qed
\end{dem}

\begin{rk}
For principal and transitive groupoids, the third condition in theorem \ref{Kunneth},
\begin{center} 
``$\alpha_{Res_H^G(A),B}^{H,V}$ is an isomorphism'',
\end{center}
reduces to ``$A$ satisfies the Künneth formula". Indeed, the only non trivial subgroupoid is the trivial one.
\end{rk}

\begin{rk}
We would like to comment on the third condition of theorem \ref{Kunneth}, namely:
\[\text{``for every compact open subgroupoid }H \text{ of } G \text{ and every }H \text{-space }V \]
\[\text{ such that the anchor map }p : V \rightarrow H^{(0)}\text{ is locally injective, }\]
\[\alpha_{A,B}^{H,V}\text{ is an isomorphism for every } C^*\text{-algebra} B \text{ such that $K_*(B)$ is free".}\]
In the case of a group, a locally injective anchor map impose for $V$ to be a disjoint union of points so that the condition reduces to $A\rtimes_r H$ satifies the Künneth formula. \\

We can have a similar statement in the groupoid setting if we suppose that there exists an $H$-equivariant open covering of $V$ such that $p_{|U}$ is injective. The Mayer-Vietoris exact sequence associated to any finite such covering, and then taking the inductive limit, shows that the condition reduces to $A\rtimes_r H$ satisfies the Künneth formula.  
\end{rk}

A natural question is to find an example of $C^*$-algebra for which theorem \ref{Kunneth} ensures that it satisfies the Künneth formula, without this being a consequence of previous known results \cite{RosenbergKunneth},\cite{ChabertEOY},\cite{OY4}.\\

%%%%%%%%%%%%%%%%%%%%%%%%%%%%%%%%%%%%%%%%%%%%%%%
%%%%%%%%%%%%%%%%%%%%%%%%%%%%%%%%%%%%%%%%%%%%%%%
%%%%%%%%%%%%%  HYPEBOLIC  %%%%%%%%%%%%%%%%%%%%%
%%%%%%%%%%%%%%%%%%%%%%%%%%%%%%%%%%%%%%%%%%%%%%%
%%%%%%%%%%%%%%%%%%%%%%%%%%%%%%%%%%%%%%%%%%%%%%%

\subsection{Hyperbolic groupoids}
Recall the following definition from \cite{LaffOY}.

\begin{definition}\label{hyperbolicLaff}
Let $G$ be a Hausdorff \'etale groupoid. $G$ is said to be:
\begin{itemize}
\item[$\bullet$] \textit{hyperbolic} if there exists a positive number $\delta$ and a right invariant length $l$ on $G$ such that $(G_x, l_x)$ is a $\delta$-hyperbolic $\delta$-geodesic space for every $x\in G^{(0)}$,
\item[$\bullet$] \textit{with controlled growth} if there exist positive constants $\alpha $ and $\beta$ such that the balls 
\[B(g,R) = \{ g' \in G_x : d_x(g,g'))\}\] 
for $s(g)=x$ satisfy \[|B(g,R)| \leq \alpha e^{\beta l(g)}.\]
\end{itemize}
\end{definition}

The main result of \cite{LaffOY} is the following theorem:

\begin{thm}[Thm. $4.0.1$, \cite{LaffOY}]
Let $G$ be a Hausdorff \'etale hyperbolic groupoid with controlled growth. If there exists a proper $G$-algebra $A$ and elements 
\[\alpha \in KK^G(C_0(G^{(0)}) , A) \quad \text{and} \quad \beta\in KK^G(A, C_0(G^{(0)}))\]
such that $\gamma = \alpha \otimes_A \beta\in KK^G(C_0(G^{(0)}),C_0(G^{(0)}))$ satisfies:
\begin{itemize}
\item[$\bullet$] there exists a sequence of right invariant length $(l_i)_i$ converging uniformly to zero on every compact subset of $G$ such that 
\[\iota(\gamma) = 1 \in KK^{G,l_i}_{ban}C_0(G^{(0)}),C_0(G^{(0)}))\] 
\item[$\bullet$] for every $G$-compact subset $Z \subseteq \underline E G$ with anchor map $q: Z \rightarrow G^{(0)}$, 
\[q^*(\gamma) = 1 \in KK^{Z\rtimes G}(C_0(Z),C_0(Z)),\]
\end{itemize}
then the assembly map
\[\mu_{G,r} : K^{top}(G)\rightarrow K(C_r^*(G))\]
is an isomorphism.
\end{thm}

In \cite{nekrashevych} is given the definition of a hyperbolic pseudogroup. $S$ is compactly generated if there exists an open relatively compact transversal $T$ of $G(S)$ and a symmetric finite set $F$ such that
\begin{itemize}
\item[$\bullet$] any $[x,\phi]$ such that $x, \phi(x)\in T$ decomposes as a product of elements of $F$,
\item[$\bullet$] every element $\gamma \in F$ is the restriction of some $[x,\phi]$ with the domain and the range of $\gamma $ are relatively compacts whose closure sits inside the domain and the range of $\phi$ respectively. 
\end{itemize}

\begin{definition}
A compactly generated pseudo group is hyperbolic if there exists a open relatively compact transversal $T$ and ...
\end{definition}

If $S$ is hyperbolic in this sense, then $G(S)$ is hyperbolic in the sense of definition \ref{hyperbolicLaff}. Moreover, the space $X$ being the base space of the groupoid, $G(S)$ is ample as soon as $X$ is totally disconnected. Any reduced $C^*$-algebra of a compact groupoid satisfies the Kunneth formula, because it is of type I. The only condition that remains to be proved is that $G(S)$ satifies the Baum-Connes conjecture for any coefficient $B$ a trivial $G$-algebra.  

\subsection{HLS groupoids}

Problem: $\Gamma$ is amenable iff $G_{\mathcal N}$ is. What about a-T-menabilty?

If $\Gamma $ has $T$, then if $\mu_\Gamma$ is injective, $\mu_G  $ is not surjective. So if $\Gamma$ hyperbolic, the HLS groupoid will not satisfy the Baum-Connes conjecture.
























