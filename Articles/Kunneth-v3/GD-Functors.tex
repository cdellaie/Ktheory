\documentclass[reqno,oneside,a4paper,11pt]{amsart}
\usepackage[a4paper, hmargin={2.8cm, 2.8cm}, vmargin={2.5cm, 2.5cm}]{geometry}
\usepackage[ngerman, english]{babel}
\usepackage[utf8]{inputenc}
\usepackage{pdfsync}
\usepackage{verbatim}
\usepackage[onehalfspacing]{setspace}
\usepackage{amsmath}
\usepackage{amsthm}
\usepackage{amssymb}

 \usepackage{paralist}
\theoremstyle{theorem}
\newtheorem{satz}{Theorem}[section]

  \newtheorem{lemma}[satz]{Lemma}
  \newtheorem{kor}[satz]{Corollary}
  \newtheorem{thmx}{Theorem}
   \renewcommand{\thethmx}{\Alph{thmx}}
  \newtheorem{corx}[thmx]{Corollary}
  \newtheorem{prop}[satz]{Proposition}
  \newtheorem{conjecture}[satz]{Vermutung}
  \theoremstyle{definition}
 \newtheorem{defi}[satz]{Definition}
  \newtheorem{bem}[satz]{Remark}
  \newtheorem{aufgabe}[satz]{Aufgabe}
  \newenvironment{beweis}%
    {\begin{proof}[Beweis]}
    {\end{proof}}
  \newtheorem{ex}[satz]{Example}
    \newtheorem{exs}[satz]{Examples}
  \newtheorem{beispiele}[satz]{Beispiele}


\newcommand{\norm}[1]{\lVert#1\rVert}   %Norm{} befehl
\newcommand{\betrag}[1]{\lvert#1\rvert}
\newcommand{\KK}{\mathrm{KK}}
\newcommand{\RKK}{\mathcal{R}\mathrm{KK}}
\newcommand{\Tor}{\mathrm{Tor}}
\newcommand{\K}{\mathrm K}
\newcommand{\EE}{\mathbb E}
\newcommand{\RR}{\mathbb R}
\newcommand{\QQ}{\mathbb Q}
\newcommand{\CC}{\mathbb C}
\newcommand{\NN}{\mathbb N}
\newcommand{\ZZ}{\mathbb Z}
\newcommand{\FF}{\mathbb F}
\newcommand{\TT}{\mathbb T}
\newcommand{\HH}{\mathbb H}
\newcommand{\lk}{\langle}
\newcommand{\rk}{\rangle}
\newcommand{\id}{\text{id}}
\newcommand{\eps}{\varepsilon}
\setcounter{MaxMatrixCols}{19}
\usepackage{tikz}
\usetikzlibrary{matrix,arrows}
\usepackage{hyperref}
\allowdisplaybreaks

\title[Going-Down functors and the K\"unneth-formula]{\texorpdfstring{Going-Down functors and the Künneth-formula for crossed products by ample groupoids}{Going-Down functors and the Künneth-formula for crossed products by ample groupoids}}
\author{Christian B\"onicke}
\address{Mathematisches Institut der WWU M\"unster,
	\newline Einsteinstrasse 62, 48149 M\"unster, Germany}
\email{c.boenicke@wwu.de}

\author{Clément Dell'Aiera}
\address{Department of Mathematics, University of Hawaii
	\newline 2565 McCarthy Mall, Keller 401A Honolulu HI 96822.
}
\email{dellaiera@math.hawaii.edu}

\thanks{Supported by Deutsche Forschungsgemeinschaft (SFB 878).}

\subjclass[2010]{22A22, 46L05, 46L35}
\keywords{Ample groupoid,}
\begin{document}

	\maketitle
	\begin{abstract}
		We study the connection between the Baum-Connes conjecture for an ample groupoid $G$ with coefficient $A$ and the Künneth formula for the $\K$-theory of tensor products by the crossed product $A\rtimes_r G$. To do so we develop the machinery of Going-Down functors as pioneered in the group case by Chabert, Echterhoff and Oyono-Oyono for ample groupoids. Using this machinery we also prove that the topological $\K$-theory of an ample groupoid $G$ is continuous with respect to inductive limits of the coefficient algebra.
	\end{abstract}
	\section{Introduction}
	We say that a $\mathrm{C}^*$-algebra $A$ satisfies the \textit{Künneth formula} if for all $\mathrm{C}^*$-algebras $B$ there exists a canonical short exact sequence
	\begin{equation} \label{KunnethSequence}	
0\longrightarrow \K_*(A)\otimes \K_*(B)\stackrel{\alpha}{\longrightarrow}\K_*(A\otimes B)\stackrel{\beta}{\longrightarrow}\mathrm{Tor}(\K_*(A),\K_*(B))\longrightarrow 0,
	\end{equation}
	where $A\otimes B$ denotes the minimal tensor product of $A$ and $B$ and $\K_*$ denotes $\ZZ /2\ZZ$-graded $\K$-theory. In this paper we study the question of when $A$ satisfies the Künneth formula for the case that $A=C\rtimes_r G$ is a (reduced) crossed product, where $G$ is an ample groupoid and $C$ is a $G$-algebra.
	We follow the strategy of \cite{CEO} and compare existence of the sequence \ref{KunnethSequence} to the existence of a canonical exact sequence

		\begin{equation}\label{MixedSequence}
		0\rightarrow \K_*^{\mathrm{top}}(G;C)\otimes\K_*(B)\stackrel{\alpha_G}{\rightarrow}\K_*^{\mathrm{top}}(G;C\otimes B)\stackrel{\beta_G}{\rightarrow}\mathrm{Tor}(\K_*^{\mathrm{top}}(G;C),\K_*(B))\rightarrow 0
		\end{equation}
	Here $\K_*^{\mathrm{top}}(G;C)$ denotes the topological $\K$-theory of $G$ with coefficient $C$. The link between the sequences \ref{MixedSequence} and \ref{KunnethSequence} is given by the Baum-Connes assembly map $\mu_C:\K_*^{\mathrm{top}}(G;C)\rightarrow K_*(C\rtimes_r G)$.
	Let $\mathcal{N}_G$ denote the class of all separable exact $G$-algebras $C$ for which the canonical exact sequence \ref{MixedSequence} exists. We show that whenever $C$ is in $\mathcal{N_G}$ and $G$ satisfies the Baum-Connes conjecture with coefficients in $C\otimes B$ for all separable $\mathrm{C}^*$-algebras $B$ with respect to the trivial action on $B$, then $A=C\rtimes_r G$ satisfies the Künneth formula.
	We then use the machinery of Going-Down functors to show, that the class $\mathcal{N}_G$ is non-empty, and in fact fairly large by proving the following results.
	\begin{thmx}(see Theorem \ref{Theorem:Kunneth} and Corollary \ref{Cor:Kunneth})
		Let $G$ be an ample groupoid and $A$ a separable and exact $G$-algebra. Suppose that $A_{\mid K}\rtimes K$ satisfies the (ordinary) Künneth formula for all compact open subgroupoids $K\subseteq G$. Then $A\in \mathcal{N}_G$. In particular, if the fibre $A_x$ is type I for all $x\in G^{(0)}$, then $A\in \mathcal{N}_G$.
	\end{thmx}
	
	As an immediate consequence of this and Proposition \ref{Prop:BCandKunneth} we get the following.
	\begin{corx}(see Corollary \ref{Corollary:Kunneth})
		Let $G$ be an ample groupoid and $A$ a separable and exact $G$-algebra. Suppose that $A_{\mid K}\rtimes K$ satisfies the (ordinary) Künneth formula for all compact open subgroupoids $K\subseteq G$. Suppose further that $G$ satisfies the Baum-Connes conjecture for $A\otimes B$ for every $\mathrm{C}^*$-algebra $B$ (with $G$ acting trivially on $B$).
		Then $A\rtimes_r G$ satisfies the Künneth formula. 
	\end{corx}
	
	We also show that the class $\mathcal{N}_G$ enjoys many stability properties. Among these we prove that $\mathcal{N}_G$ is stable under taking inductive limits. To prove this we show that the topological $\K$-theory is continuous with respect to the coefficient algebra, which constitutes another application of the Going-Down principle and is inspired by \cite[§7]{MR1836047}:
	\begin{thmx}(see Theorem \ref{Theorem:Continuity of top. K-theory})
		Let $G$ be an ample groupoid and $(A_n,\varphi_n)_n$ an inductive sequence of $G$-algebras. If we let $A=\lim A_n$, then the maps $\psi_{n,*}:\K_*^{\mathrm{top}}(G;A_n)\rightarrow \K_*^{\mathrm{top}}(G;A)$ induced by the canonical maps $\psi_n:A_n\rightarrow A$, give rise to an isomorphism
		$$\lim_{n\rightarrow\infty}\K_*^{\mathrm{top}}(G;A_n)\cong \K_*^{\mathrm{top}}(G;A).$$
	\end{thmx}
	An immediate consequence is the following permanence property for the Baum-Connes conjecture:
	\begin{corx}(see Corollary \ref{Cor:InductiveLimit})
		Let $G$ be an ample groupoid and $(A_n,\varphi_n)_n$ an inductive sequence of $G$-algebras with $A=\lim_{n\rightarrow\infty}A_n$. Suppose $G$ satisfies the Baum-Connes conjecture with coefficients in $A_n$ for all $n\in\NN$. Assume further, that $G$ is exact, or that all the connecting homomorphisms $\varphi_n$ are injective. Then $G$ satisfies the Baum-Connes conjecture with coefficients in $A$.
	\end{corx}
	
	

	\section{Preliminaries on groupoids and $G$-algebras}
	Recall, that a \textit{groupoid} is a set $G$ together with a distinguished subset $G^{(2)}\subseteq G\times G$, called the set of \textit{composable pairs}, a product map $G^{(2)}\rightarrow G$ denoted by $(g,h)\mapsto gh$, and an inverse map $G\rightarrow G$, written $g\mapsto g^{-1}$, such that:
	\begin{enumerate}
		\item If $(g_1,g_2),(g_2,g_3)\in G^{(2)}$, then so are $(g_1g_2,g_3)$ and $(g_1,g_2g_3)$ and their products coincide, meaning $(g_1g_2)g_3=g_1(g_2g_3)$;
		\item for all $g\in G$ we have $(g,g^{-1})\in G^{(2)}$; and
		\item for any $(g,h)\in G^{(2)}$ we have $g^{-1}(gh)=h$ and $(gh)h^{-1}=g$.
	\end{enumerate}
	Every groupoid comes with a subset
	$$G^{(0)}=\lbrace gg^{-1}\mid g\in G\rbrace=\lbrace g^{-1}g\mid g\in G\rbrace$$
	called the set of \textit{units} of $G$, and two maps $r,d:G\rightarrow G^{(0)}$ given by
	$r(g)=gg^{-1}$ and $d(g)=g^{-1}g$ called \textit{range} and \textit{domain} maps respectively.
	A subgroupoid of $G$ is a subset $H\subseteq G$ which is closed under the product and inversion meaning that $gh\in H$ for all $(g,h)\in G^{(2)}\cap H\times H$ and $g^{-1}\in H$ for all $g\in H$.
	
	When $G$ is endowed with a locally compact Hausdorff topology under which the product and inversion maps are continuous, $G$ is called a locally compact groupoid. A \textit{bisection} is a subset $S\subseteq G$ such that the restrictions of the range and domain maps to $S$ are local homeomorphisms onto open subsets of $G$. We will denote the set of all open bisections by $G^{op}$. A locally compact, Hausdorff groupoid is called \textit{étale} if there is a basis for the topology of $G$ consisting of open bisections. It follows that $G^{(0)}$ is open in $G$. Recall that it is also closed, since $G$ is assumed to be Hausdorff. A topological groupoid is called \emph{ample} if it has a basis of compact open bisections. We will write $G^a$ for the subset of $G^{op}$ consisting of all compact open bisections. If $G$ is a locally compact, Hausdorff and étale groupoid, then $G$ is ample if and only if $G^{(0)}$ is totally disconnected (see \cite[Proposition 4.1]{Exel10}).
	
	For a subset $D\subseteq G^{(0)}$ write
	$$G_D:=\lbrace g\in G\mid d(g)\in D\rbrace,\ G^D:=\lbrace g\in G\mid r(g)\in D\rbrace,\ \text{and } G_D^D:=G_D\cap G^D.$$
	If $D=\lbrace u\rbrace$ consists of a single point $u\in G^{(0)}$ we will omit the braces in our notation and write $G_u:=G_D$, $G^u:=G^D$ and $G_u^u:=G_D^D$.
	
	Recall that if $X$ is a locally compact Hausdorff space and $A$ is a $\mathrm{C}^*$-algebra, then we call $A$ a $C_0(X)-algebra$ if there exists a non-degenerate $\ast$-homomorphism
	$$\Phi:C_0(X)\rightarrow Z(M(A)),$$ where $Z(M(A))$ denotes the center of the multiplier algebra of $A$. For every $x\in X$ there is a closed ideal $I_x$ in $A$ defined by $I_x=\overline{C_0(X\setminus\lbrace x\rbrace)A}$ and we call the quotient $A_x:=A/I_x$ the \textit{fibre} of $A$ over $x$. We write $a(x)$ for the image of $a\in A$ in $A_x$ under the quotient map. Put $\mathcal{A}=\coprod_{x\in X} A_x$. Then $\mathcal{A}$ can be equipped with a topology such that it becomes an upper-semicontinouos $\mathrm{C}^*$-bundle over $X$ and moreover $A\cong \Gamma_0(X,\mathcal{A})$, where $\Gamma_0(X,\mathcal{A})$ denotes the continuous sections of this bundle which vanish at infinity.
	Throughout this work we will freely alternate between the bundle picture and the picture as $C_0(X)$-algebras. For convenience bundles will always be denoted by calligraphic letters.
	The reader unfamiliar with the theory is referred to the expositions in \cite[Appendix C]{Williams} and \cite[Section~3.1]{Goehle}.
	
	recall that a $\ast$-homomorphism $\Phi:A\rightarrow B$ between two $C_0(X)$-algebras $A$ and $B$ is called \textit{$C_0(X)$-linear} if $\Phi(f a)=f \Phi(a)$ for all $f\in C_0(X)$ and all $a\in A$.
	
	If $\Phi:A\rightarrow B$ is a $C_0(X)$-linear homomorphism, it induces $\ast$-homo\-morphisms $\Phi_x:A_x\rightarrow B_x$ on the level of the fibres given by $\Phi_x(a(x))=\Phi(a)(x)$.
	Conveniently, one can check several properties of $\Phi$ on the level of the fibres and vice versa:
	\begin{lemma}\cite[Lemma~2.1]{MR2820377}\label{Lem:IsomorphismCriteriumForC(X)-linearHomomorphisms}
		Let $\Phi:A\rightarrow B$ be a $C_0(X)$-linear homomorphism. Then $\Phi$ is injective (resp. surjective, resp. bijective) if and only if $\Phi_x$ is injective (resp. surjective, resp. bijective) for all $x\in X$.
	\end{lemma}
	
	We will also need the notion of a pullback: If $A$ is a $C_0(X)$-algebra and $f:Y\rightarrow X$ a continuous map, we can define the \textit{pullback} of $A$ along $f$ as follows:
	Let $q:\mathcal{A}\rightarrow X$ denote the upper-semicontinouos $\mathrm{C}^*$-bundle over $X$ associated to $A$. Then we can form the pullback bundle $f^*\mathcal{A}=\lbrace ((y,a)\in Y\times\mathcal{A}\mid f(y)=q(a)\rbrace$. The bundle $f^*\mathcal{A}$ is an upper-semicontinouos $\mathrm{C}^*$-bundle over $Y$ whose fibres $(f^*\mathcal{A})_y$ are canonically isomorphic to $A_{f(y)}$. We let $f^*A:=\Gamma_0(Y,f^*\mathcal{A})$ denote the corresponding $C_0(Y)$-algebra. Note, that we can canonically identify $(f^*A)_y=A_{f(y)}$.
	It is an easy exercise to show that if $A$ is a $C_0(X)$-algebra and $f:Y\rightarrow X$ and $g:Z\rightarrow Y$ are two continuous maps, then the algebras $(f\circ g)^*A$ and $g^*(f^*A)$ are canonically isomorphic as $C_0(Z)$-algebras.
	
	Pullbacks also behave nicely with respect to $C_0(X)$-linear $\ast$-homomorphisms:	
	\begin{lemma}\label{Lem:PullbackOfHomomorphisms}
		Let $A$ and $B$ be two $C_0(X)$-algebras and $f:Y\rightarrow X$ a continuous map. If $\Phi:A\rightarrow B$ is a $C_0(X)$-linear homomorphism, then the map
		$$f^*\Phi:f^*A\rightarrow f^*B$$
		given by $(f^*\Phi)(\psi)(y)=\Phi_{f(y)}(\psi(y))$ is a $C_0(Y)$-linear homomorphism.
		Moreover, the pullback construction is functorial meaning if $\Psi:B\rightarrow C$ is another $C_0(X)$-linear $*$-homo\-morphism into a $C_0(X)$-algebra $C$ then $f^*\Psi\circ f^*\Phi=f^*(\Psi\circ \Phi)$.
	\end{lemma}
	
	Recall that a \textit{groupoid dynamical system} $(A,G,\alpha)$ consists of a locally compact Hausdorff groupoid $G$, a $C_0(G^{(0)})$-algebra $A$ and a family $(\alpha_g)_{g\in G}$ of $*$-isomorphisms $\alpha_g:A_{d(g)}\rightarrow A_{r(g)}$ such that $\alpha_{gh}=\alpha_g\circ \alpha_h$ for all $(g,h)\in G^{(2)}$ and such that $g\cdot a:=\alpha_g(a)$ defines a continuous action of $G$ on the upper-semicontinuous bundle $\mathcal{A}$ associated to $A$.	
	We will often omit the action $\alpha$ in our notation and just say that $A$ is a $G$-algebra.
	Since the topology on an upper-semicontinuous $\mathrm{C}^*$-bundle is notoriously difficult to handle we will rely on the following alternate characterization in this paper:
	\begin{lemma}\cite[Lemma~4.3]{MR2547343}
		Let $(A,G,\alpha)$ be a groupoid dynamical system. Then the mapping $$f\mapsto [g\mapsto \alpha_g(f(g))]$$ defines a $C_0(G)$-linear $\ast$-isomorphism $d^*A\rightarrow r^*A$, also denoted by $\alpha$.
		
		Conversely, if $G$ is a groupoid, $A$ a $C_0(G^{(0)})$-algebra, and $\alpha:d^*A\rightarrow r^*A$ is a $C_0(G)$-linear isomorphism then $\alpha$ induces an isomorphism $\alpha_g:A_{d(g)}\rightarrow A_{r(g)}$ for each $g\in G$. If the equation $\alpha_{gh}=\alpha_g\alpha_h$ holds for all $(g,h)\in G^{(2)}$, then $(A,G,\alpha)$ is a groupoid dynamical system.
	\end{lemma}
	
	Finally, let us briefly recall the definition of a groupoid crossed product	following \cite{MR1900993}.
	Let $G$ be an étale groupoid and $(A,G,\alpha)$ a groupoid dynamical system. Consider the complex vector space $\Gamma_c(G,r^*\mathcal{A})$. It carries a canonical $*$-algebra structure with respect to the following operations:
	$$(f_1\ast f_2)(g)=\sum\limits_{h\in G^{r(g)}} f_1(h)\alpha_h(f_2(h^{-1}g))$$
	and
	$$f^*(g)=\alpha_g(f(g^{-1})^*).$$
	See for example \cite[Proposition~4.4]{MR2547343} for a proof of this fact.
	For $u\in G^{(0)}$ consider the Hilbert $A_u$-module $\ell^2(G^u,A_u)$. It is the completion of the space of finitely supported $A_u$-valued functions on $G^u$, with respect to the inner product 
	$$\lk \xi,\eta\rk=\sum\limits_{h\in G^u}\xi(h)^*\eta(h).$$
	We can then define a $*$-representation $\pi_u:\Gamma_c(G,r^*\mathcal{A})\rightarrow \mathcal{L}(\ell^2(G^u,A_u))$ by
	$$\pi_u(f)\xi(g)=\sum\limits_{h\in G^u}\alpha_g(f(g^{-1}h))\xi(h).$$
	Using this family of representations, we can define a $\mathrm{C}^*$-norm on the convolution algebra $\Gamma_c(G,r^*\mathcal{A})$ by
	$$\norm{f}_r:=\sup\limits_{u\in G^{(0)}}\norm{\pi_u(f)}.$$
	The reduced crossed product $A\rtimes_r G$ is defined to be the completion of $\Gamma_c(G,r^*\mathcal{A})$ with respect to $\norm{\cdot}_r$.
	\section{Inductive limits of $G$-algebras}
	In this section we will show that an inductive limit of $G$-algebras with $G$-equivariant connecting morphisms is again a $G$-algebra in a canonical fashion. These results should be known to the experts but since we could not find a suitable reference and in order to keep the exposition self-contained we elaborate on the details. We start of by considering $C_0(X)$-algebras:
	Let $(A_n,\varphi_n)_{n\in\NN}$ be an inductive sequence of $\mathrm{C}^*$-al\-gebras, where each $A_n$ is a $C_0(X)$-algebra, such that the connecting homomorphisms $\varphi_n$ are $C_0(X)$-linear.
	If $A=\lim_{n\rightarrow\infty}A_n$, then $A$ is a $C_0(X)$-algebra in a canonical way. This is surely well-known to the experts, but we could not find a proper reference, so we include the details.
	
	Let us start by recalling the construction of the limit algebra $A$:
	Consider the algebra $$\widetilde{A}=\lbrace (a_n)_n\in\prod\limits_{n}A_n\mid \exists n_0: a_{n+1}=\varphi_n(a_n)\forall n\geq n_0\rbrace.$$
	Then $A$ is the closure of the image of $\widetilde{A}$ under the quotient map $q:\prod A_n\rightarrow \prod A_n/\bigoplus A_n$.
	Now if $f\in C_0(X)$, then $C_0(X)$-linearity of the $\varphi_n$ implies, that $\widetilde{A}$ is invariant under component-wise multiplication with $f$. It also leaves the ideal $\bigoplus A_n$ invariant. Hence we get a well-defined linear map
	$q(\widetilde{A})\rightarrow q(\widetilde{A})$ by $f\cdot q((a_n)_n):=q((f\cdot a_n)_n)$. Using the equality $\norm{q((a_n)_n)}=lim \norm{a_n}$ (see K-theory script 6.8)
	$\norm{q((f\cdot a_n)_n)}=\lim \norm{f\cdot a_n}\leq \norm{f}\lim \norm{a_n}=\norm{f}\norm{q((a_n)_n)}$. Consequently, $f\cdot$ extends to a bounded linear map $A\rightarrow A$, actually to an element in $Z(M(A))$, where the adjoint is given by $\overline{f}\cdot$. Thus, we have constructed a $*$-homomorphism $\Phi:C_0(X)\rightarrow Z(M(A))$. 
	\begin{lemma}\label{Lem:InductiveLimitsOfC_0(X)-algebras}
		The $*$-homomorphism $\Phi$ from above is non-degener\-ate. Consequently, $A$ is a $C_0(X)$-algebra such that the canonical maps $\psi_n:A_n\rightarrow A$ are $C_0(X)$-linear.
	\end{lemma}
	\begin{proof}
		Let $a\in A$ and $\varepsilon>0$ be given. By construction $\bigcup_n \psi(A_n)$ is dense in $A$, so there exists $b\in A_n$ such that $\norm{\psi_n(b)-a}<\frac{\varepsilon}{2}$. Since the structure homomorphism for $A_n$ is non-degenerate we can also find $f\in C_0(X)$ and $c\in A_n$ such that $\norm{b-fc}< \frac{\varepsilon}{2\norm{\psi_n}}$, and hence $\norm{\psi_n(b)-f\psi_n(c)}<\frac{\varepsilon}{2}$.
		Putting things together we obtain
		$\norm{f\psi_n(c)-a}<\norm{f\psi_n(c)-\psi_n(b)}+\norm{\psi_n(b)-a}<\varepsilon$.
	\end{proof}
	
	We will now identify the fibres of the limit algebra:
	\begin{lemma}
		Let $(A_n,\varphi_n)$ be an inductive sequence of $C_0(X)$-al\-gebras and $A=\lim A_n$. Then, for every $x\in X$, $((A_n)_x,(\varphi_n)_x)$ is an inductive sequence of $\mathrm{C}^*$-algebras and
		$$ \lim\limits_{n\rightarrow\infty} (A_n)_x\cong A_x.$$
	\end{lemma}
	\begin{proof}
		It is immediate, that $((A_n)_x,(\varphi_n)_x)$ is indeed an inductive sequence of $\mathrm{C}^*$-algebras. Hence we only need to identify the limit.
		Let $\pi_{n,x}:A_n\rightarrow (A_n)_x$ denote the quotient maps onto the fibres and $\psi_{n,x}:(A_n)_x\rightarrow \lim\limits_{n}(A_n)_x$ the canonical maps. By the universal property of the limit we obtain a surjective $*$-homo\-morphism $$\pi:A\rightarrow \lim\limits_{n}(A_n)_x.$$
		It remains to show, that the kernel of $\pi$ coincides with the ideal $I_x=\overline{C_0(X\setminus\lbrace x\rbrace)A}$ of $A$.
		If $a=\psi_n(b)$ for some $b\in A_n$ and $f\in C_0(X\setminus\lbrace x\rbrace)$, then $\pi(fa)=\pi(f\psi_n(b))=\pi(\psi_n(fb))=\psi_{n,x}(\pi_{n,x}(fb))=0$. By continuity we get $I_x\subseteq ker(\pi)$.
		
		Suppose conversely that $a\in ker(\pi)$ and $\varepsilon>0$ is given. First we can find $n\in\NN$ and $b\in A_n$ such that $\norm{a-\psi_n(b)}<\frac{\varepsilon}{3}$. Thus
		$\norm{\psi_{n,x}(\pi_{n,x}(b))}=\norm{\pi(\psi_n(b))}=\norm{\pi(\psi_n(b)-a)}\leq \norm{a-\psi_n(b)}<\frac{\varepsilon}{3}$.
		Upon replacing $b$ and $n$ by $\varphi_{m,n}(b)$ for $m$ big enough we can actually assume that $\norm{\pi_{x,n}(b)}<\frac{\varepsilon}{3}$. Then there exists some $b'\in A_n$ such that $\norm{b-b'}<\frac{\varepsilon}{3}$ and $\pi_{n,x}(b')=0$. Hence there must be $b''\in A_n$ and $\varphi\in C_0(X\setminus\lbrace x\rbrace)$ such that $\norm{b'-\varphi b''}<\frac{\varepsilon}{3}$.
		Putting things together we obtain
		$$\norm{a-\varphi\psi_n(b'')}\leq \norm{ a- \psi_n(b)}+\norm{\psi_n(b)-\psi_n(b')}+\norm{\psi_n(b')-\psi_n(\varphi b'')}<\varepsilon$$
		and hence $ker(\pi)\subseteq I_x$, which completes the proof.
	\end{proof}
	Next, we want to show that taking the limit of an inductive sequence commutes with pullbacks: Let $(A_n,\varphi_n)_{n\in\NN}$ be an inductive sequence of $C_0(X)$-algebras and $f:Y\rightarrow X$ a continuous map. Then we get $C_0(Y)$-linear $*$-homo\-morphisms $f^*\varphi_n:f^*A_n\rightarrow f^* A_{n+1}$ by the formula $$(f^*\varphi_n)(\xi)(y)=(\varphi_n)_{f(y)}(\xi(y)).$$
	as in Lemma \ref{Lem:PullbackOfHomomorphisms}. 
	\begin{prop}\label{Prop:LimitsAndPullbacks}
		Let $(A_n,\varphi_n)_{n\in\NN}$ be an inductive sequence of $C_0(X)$-al\-gebras and $f:Y\rightarrow X$ a continuous map.
		Then $(f^*A_n,f^*\varphi_n)_{n}$ is an inductive system of $C_0(Y)$-algebras and $f^*(\lim_n A_n)$ is $C_0(Y)$-linearly isomorphic to $\lim_n f^*(A_n)$.
	\end{prop}
	\begin{proof}
		Let $A=\lim_n A_n$ and $\psi_n:A_n\rightarrow A$ be the canonical $\ast$-homo\-morphisms. Then by Lemma \ref{Lem:PullbackOfHomomorphisms} we obtain $C_0(Y)$-linear $*$-homo\-morphisms $f^*\psi_n:f^*A_n\rightarrow f^*A$ such that $f^*\psi_{n+1}\circ f^*\varphi_n=f^*(\psi_{n+1}\circ \varphi_n)=f^*\psi_n$. Using the universal property of the limit, we obtain a $C_0(Y)$-linear $*$-homo\-morphism $$\Psi:\lim\limits_{n}f^*A_n\rightarrow f^*A.$$
		To show that it is an isomorphism, it is enough to check that $\Psi_y$ is an isomorphism for all $y\in Y$. But under the identifications
		$$(\lim_{n}f^*A_n)_y\cong \lim_{n}(A_n)_{f(y)}\text{ and }(f^*A)_y\cong A_{f(y)}$$ the map $\Psi_y$ coincides with the isomorphism 
		$$\lim\limits_{n}(A_n)_{f(y)}\rightarrow A_{f(y)}$$
		from the previous Lemma.
	\end{proof}
	
	Suppose now that $(A_n,\varphi_n)_n$ is an inductive sequence of $G$-alge\-bras, such that all the connecting homomorphisms are $G$-equivariant. We have already seen in Lemma \ref{Lem:InductiveLimitsOfC_0(X)-algebras}, that $A=\lim_n A_n$ is a $C_0(G^{(0)})$-algebra in a canonical way, such that all the homo\-morphisms $\psi_n:A_n\rightarrow A$ are $C_0(G^{(0)})$-linear. The following Proposition shows how we can use the $G$-actions at each stage of the sequence to obtain a $G$-action on the limit.
	\begin{prop}
		Let $(A_n,\varphi_n)_n$ be an inductive sequence of $G$-algebras, such that $\varphi_n$ is $G$-equivariant for all $n\in\NN$. Let $A:=\lim_n A_n$ and $\psi_n:A_n\rightarrow A$ be the canonical maps. Then there exists a canonical $G$-action on $A$, such that $\psi_n$ is $G$-equivariant for all $n\in\NN$.
	\end{prop}
	\begin{proof}
		For each $n\in\NN$ let $\alpha_n:d^*A_n\rightarrow r^*A_n$ denote the $C_0(G)$-linear isomorphism implementing the action of $G$ on $A_n$. Since $\varphi_n$ is $G$-equivariant for every $n\in\NN$ we have commutative diagrams
		\begin{center}
			\begin{tikzpicture}[description/.style={fill=white,inner sep=2pt}]
			\matrix (m) [matrix of math nodes, row sep=3em,
			column sep=2.5em, text height=1.5ex, text depth=0.25ex]
			{ d^*A_n &  r^*A_n\\
				d^*A_{n+1} &  r^*A_{n+1}\\ };
			\path[->,font=\scriptsize]
			(m-1-1) edge node[auto] {$ \alpha_n $} (m-1-2)
			(m-2-1) edge node[auto] {$ \alpha_{n+1} $} (m-2-2)
			(m-1-1) edge node[auto] { $ d^*\varphi_n $ } (m-2-1)
			(m-1-2) edge node[auto] {$ r^*\varphi_n $} (m-2-2)
			;
			\end{tikzpicture}
		\end{center}
		By the universal property, we obtain a $C_0(G)$-linear $*$-isomorphism between the respective limits. Combining this with Proposition \ref{Prop:LimitsAndPullbacks} we obtain a $C_0(G)$-linear $*$-isomorphism
		$$\alpha:d^*A\rightarrow r^*A.$$
		As each $\alpha_n$ is compatible with the multiplication in $G$, so is the limit homomorphism $\alpha$.
	\end{proof}
	\section{Going-Down Functors}
	Theorem \ref{MainTheorem} can be applied directly in many situations (see sections \ref{Section:AmenabilityAtInfinity} and \ref{Section:TwistedGroupoidAlgebras}) but oftentimes it is not directly a map on $\K_*^{\mathrm{top}}(G;A)$ one is interested in, but a map on a construction involving this group, which still shares the same basic functorial properties.
	Moreover, the map in question must not necessarily be given by taking the Kasparov product. A closer inspection of the proof of Theorem \ref{MainTheorem} reveals, that we only used the naturality of the Kasparov product. Hence, following \cite{CEO} we can use the language of category theory to obtain a more general result.
	To begin with, given a second countable ample groupoid $G$, we denote by $\mathcal{C}(G)$ the category of separable commutative proper $G$-algebras, i.e. algebras of the form $C_0(X)$, where $X$ is a second countable proper $G$-space. Also let $\mathcal{S}(G)$ be the set containing $G$ and all of its compact open subgroupoids.
	\begin{defi}\label{Def:GDfunctor} Let $G$ be an ample groupoid.
		A \textit{Going-Down functor} for $G$ is a collection of $\ZZ$-graded functors $\mathcal{F}=(\mathcal{F}^n_H)_{H\in \mathcal{S}(G)}$, where $\mathcal{F}^n_{H}$ is a covariant additive functor from the category of second countable, proper, locally compact $G$-spaces (with morphisms being the proper, continuous $G$-maps) to the category of abelian groups, such that the following axioms are satisfied:
		\begin{enumerate}
			\item Cohomology axioms: For every $H\in \mathcal{S}(G)$
			\begin{enumerate}
				\item the functor $\mathcal{F}_H^n$ is homotopy invariant;
				\item the functor $\mathcal{F}_H^n$ is half-exact, i.e. for every short exact sequence $$0\longrightarrow I\longrightarrow A\longrightarrow A/I\longrightarrow 0$$
				in $\mathcal{C}(H)$, the sequence
				$$\mathcal{F}_H^n (A/I)\longrightarrow \mathcal{F}_H^n(A)\longrightarrow\mathcal{F}_H^n$$
				is exact in the middle; and 
				\item for each $n\in\ZZ$ there is a natural equivalence between $\mathcal{F}_H^{n+1}$ and the functor $A\mapsto \mathcal{F}_H^n(A\otimes C_0(\RR))$, where $H$ acts trivially on the second tensor factor.
			\end{enumerate}
			\item Induction axiom: For every compact open subgroupoid $H$ of $G$, there are natural equivalences $I_H^G(n)$ between the functors $\mathcal{F}_H^n$ and $\mathcal{F}_G^n\circ Ind_H^G$, compatible with suspension,
			where $Ind_H^G:\mathcal{C}(H)\rightarrow \mathcal{C}(G)$, $A\mapsto Ind_H^{G_{\mid H^{(0)}}} A$ denotes induction from $H$-algebras to $G$-algebras.
		\end{enumerate}
		If $\mathcal{F}$ is a Going-Down functor for $G$, we define $$\mathcal{F}^n(G):=\lim\limits_{X\subseteq \mathcal{E}(G)}\mathcal{F}^n_G(C_0(X)),$$
		where $X$ runs through the $G$-compact subsets of $\mathcal{E}(G)$.
	\end{defi}
	Our main examples of Going-Down functors arise from the topological $\K$-theory of ample groupoids:
	\begin{ex}\label{Example:Going-Down functor}
		Let $G$ be a second countable ample groupoid and $A$ be a fixed $G$-algebra. Define $\mathcal{F}_H^*(C_0(X)):=\mathrm{KK}^H_*(C_0(X),A_{\mid H})$ for $H\in\mathcal{S}(G)$ and $C_0(X)\in\mathcal{C}(H)$. Then $\mathcal{F}$ is a $\ZZ/2\ZZ$-graded Going-Down functor:
		\begin{enumerate}
			\item Cohomology axioms:
			\begin{enumerate}
				\item Homotopy invariance is clear, since groupoid equivariant $\mathrm{KK}$-theory is invariant with respect to equivariant homotopies in the first variable.
				\item Half-exactness follows from \cite[Proposition~7.2 and Lemma~7.7]{Tu99}.
				\item The suspension axiom is clear from the definition of the higher equivariant $\mathrm{KK}$-groups.
			\end{enumerate}
			\item The natural equivalence required in the induction axiom is provided by the compression homomorphism defined prior to Theorem \ref{CompressionIsomorphism} (or rather its inverse, the inflation map). From the definition of the compression homomorphism it is easy to see, that it indeed provides a natural transformation with respect to equivariant $\ast$-homomorphisms.
		\end{enumerate}
	\end{ex}
	
	The following lemma can be proved using standard homotopy techniques (see for example \cite[§21.4]{MR1656031})
	\begin{lemma} Let $\mathcal{F}$ be a Going-Down functor.
		For every short exact sequence $$0\longrightarrow I\longrightarrow A\longrightarrow A/I\longrightarrow 0$$ in $\mathcal{C}(H)$ there are natural maps $\partial_n:\mathcal{F}_H^n(I)\rightarrow\mathcal{F}_H^{n+1}(A/I)$ providing a long exact sequence
		$$\cdots\longrightarrow \mathcal{F}_H^n(A/I)\longrightarrow \mathcal{F}_H^{n}(A)\longrightarrow\mathcal{F}_H^n(I)\stackrel{\partial_n}{\longrightarrow}\mathcal{F}_H^{n+1}(A/I)\longrightarrow\cdots$$
	\end{lemma}
	\begin{defi}
		Let $\mathcal{F}$ and $\mathcal{G}$ be Going-Down functors for the ample groupoid $G$. A \textit{Going-Down transformation} is a collection $\Lambda=(\Lambda_H^n)_{H\in \mathcal{S}(G)}$ of natural transformations between $\mathcal{F}_H^n$ and $\mathcal{G}_H^n$ compatible with suspension, such that
		$I_H^G(n)\circ \Lambda_H^n=\Lambda_G^n\circ I_H^G(n)$.
	\end{defi}
	\begin{ex}
		Let $G$ be a second countable ample groupoid and $A$ and $B$ be separable $G$-algebras. Let $\mathcal{F}$ be the Going-Down functor defined by $\mathcal{F}_H^*(C_0(X))=\mathrm{KK}^H_*(C_0(X),A_{\mid H})$ and let $\mathcal{G}$ be the Going-Down functor defined by $\mathcal{G}_H^*(C_0(X))=\mathrm{KK}^H_*(C_0(X),B_{\mid H})$ as in Example \ref{Example:Going-Down functor}. Suppose that $x\in \mathrm{KK}^G(A,B)$. Then we can define a Going-Down transformation
		$\Lambda$ from $\mathcal{F}$ to $\mathcal{G}$ by letting $\Lambda_H^*(C_0(X))$ be the map $$\mathcal{F}_H^*(C_0(X))=\mathrm{KK}^H_*(C_0(X),A_{\mid H})\stackrel{\cdot \otimes x}{\rightarrow} \mathrm{KK}^H_*(C_0(X),B_{\mid H})=\mathcal{G}_H^*(C_0(X)).$$
		By associativity of the Kasparov product, $\Lambda_H^*$ is a natural transformation, which is clearly compatible with suspension. Compatibility with $I_H^G$ follows from Lemma \ref{Lemma:Compression and Kasparov Product}.	
	\end{ex}
	Using the naturality, a Going-Down transformation $\Lambda$ between two Going-Down functors $\mathcal{F}$ and $\mathcal{G}$ induces morphisms $\Lambda^n(G):\mathcal{F}^n(G)\rightarrow\mathcal{G}^n(G)$ in the limit.
	
	\begin{satz}\label{Theorem:Going-Down Theorem}
		Let $\mathcal{F}$ and $\mathcal{G}$ be two Going-Down functors for an ample groupoid $G$ and let $\Lambda$ be a Going-Down transformation between $\mathcal{F}$ and $\mathcal{G}$. Suppose that $\Lambda_H^n(C(H^{(0)})):\mathcal{F}_H^n(C(H^{(0)}))\rightarrow\mathcal{G}_H^n(C(H^{(0)}))$ is an isomorphism for all compact open subgroupoids $H$ of $G$. Then $\Lambda^n(G):\mathcal{F}^n(G)\rightarrow\mathcal{G}^n(G)$ is an isomorphism.
	\end{satz}
	\begin{proof} The proof is essentially the same as that of Theorem \ref{MainTheorem}, replacing $\mathrm{KK}^H_*(C_0(X),A_{\mid H})$ by $\mathcal{F}^*_H(C_0(X))$ and $\mathrm{KK}^H_*(C_0(X),B_{\mid H})$ by $\mathcal{G}^*_H(C_0(X))$, and the map $\cdot \otimes res_H^G(x)$ by $\Lambda_H^*$, once we note, that all we used in that proof are precisely the properties we ask for in the definition of Going-Down functors and transformations.
	\end{proof}
	
	
	\section{Continuity of Topological K-theory}
	In this section we will show, that the topological $\K$-theory of an ample groupoid is continuous with respect to the coefficient algebra.
	Recall, that an étale groupoid is called \textit{exact}, if for every $G$-equivariant exact sequence
	$$0\rightarrow I\rightarrow A\rightarrow B\rightarrow 0$$
	of $G$-algebras, the corresponding sequence
		$$0\rightarrow I\rtimes_r G\rightarrow A\rtimes_r G\rightarrow B\rtimes_r G \rightarrow 0$$
		of reduced crossed products is exact.
	The following is an analogue of \cite[Lemma~2.5]{MR2010742} for étale groupoids:
	\begin{lemma}\label{Lemma:Proper Groupoids and inductive limits}
		Let $G$ be an étale groupoid and $(A_n,\varphi_n)_n$ an inductive sequence of $G$-algebras with limit $A=\lim_n A_n$. Then $(A_n\rtimes_r G,\varphi_n\rtimes G)_n$ is an inductive sequence of $\mathrm{C}^*$-algebras. Suppose additionally, that either one of the following conditions hold:
		\begin{enumerate}
			\item All the connecting maps $\varphi_n$ are injective.
			\item The groupoid $G$ is exact.
		\end{enumerate}
		Then $A\rtimes_r G=\lim_n A_n\rtimes_r G$ with respect to the connecting homomorphisms $\varphi_n\rtimes G$.
	\end{lemma}
	\begin{proof}
		It is clear that $(A_n\rtimes_r G,\varphi_n\rtimes G)$ is an inductive sequence of $\mathrm{C}^*$-algebras. For the second statement we follow the argument in \cite[Lemma~2.5]{MR2010742}:
		In the case of $(1)$ we may regard each $A_n\rtimes_r G$ as a subalgebra of $A\rtimes_r G$ and hence also the inductive limit $\overline{\bigcup_{n\in\NN}A_n\rtimes_r G}$ is contained in $A\rtimes_r G$. Let us check that $\bigcup_{n\in\NN} \Gamma_c(G,r^*\mathcal{A}_n)\subseteq \overline{\bigcup_{n\in\NN}A_n\rtimes_r G}$ is dense in $A\rtimes_r G$. First, consider elements of the form $f\otimes a\in \Gamma_c(G,r^*\mathcal{A})$ for $f\in C_c(G)$ and $a\in A$. Let $\varepsilon>0$ be given. Then, by $(1)$ we can find $n\in\NN$ and $b\in A_n$ such that $\norm{a-b}<\frac{\varepsilon}{\norm{f}}$. It follows, that $f\otimes b\in \Gamma_c(G,r^*\mathcal{A}_n)$ with $\norm{f\otimes a-f\otimes b}\leq \norm{f}\norm{a-b}<\varepsilon$. Since finite sums of elements of the form $f\otimes a$ are dense in $\Gamma_c(G,r^*\mathcal{A})$ in the inductive limit topology, it follows that $\bigcup_{n\in\NN}\Gamma_c(G,r^*\mathcal{A}_n)$ is dense in $\Gamma_c(G,r^*\mathcal{A})$ with respect to the inductive limit topology and hence also with respect to the reduced norm topology.
		For the proof of $(2)$ we make use of the following general fact:
		If $(B_n,\psi_n)$ is an inductive sequence of $\mathrm{C}^*$-algebras, then so is $(B_n/ker(\psi_n),\widetilde{\psi_n})$, where $\widetilde{\psi}_n$ are the maps induced by $\psi_n$ on the quotients. Then it is easy to check, that all the maps $\widetilde{\psi}_n$ are injective and $B=\lim_n B_n/ker(\psi_n)$.
		Returning to the proof of $(2)$,
		let $I_n=ker(\varphi_n)$. Using the exactness of $G$ now, we see that $I_n\rtimes G$ is precisely the kernel of the map $\varphi_n \rtimes G:A_n\rtimes_r G\rightarrow A\rtimes_r G$.
		By the above remark we have $A=\lim_n A_n/I_n$ and since the connecting maps are all injective we get $A\rtimes_r G=\lim_n (A_n/I_n)\rtimes_r G$ by $(1)$. Using the exactness of $G$ now, we see that $I_n\rtimes G$ is precisely the kernel of the map $\varphi_n \rtimes G:A_n\rtimes_r G\rightarrow A\rtimes_r G$, hence $(A_n/I_n)\rtimes_r G=A_n\rtimes_r G/I_n\rtimes_r G$ and another application of the above mentioned fact together with the identity $I_n\rtimes_r G=ker(\varphi_n\rtimes_r G)$ yields
		$$A\rtimes_r G	=\lim_n A_n/I_n\rtimes_r G=\lim_n A_n\rtimes_r G/I_n\rtimes_r G=\lim_n A_n\rtimes_r G.$$
	\end{proof}
	\begin{satz}\label{Theorem:Continuity of top. K-theory}
		Let $G$ be an ample groupoid and $(A_n,\varphi_n)_n$ an inductive sequence of $G$-algebras. If we let $A=\lim A_n$, then the maps $\psi_{n,*}:\K_*^{\mathrm{top}}(G;A_n)\rightarrow \K_*^{\mathrm{top}}(G;A)$ induced by the canonical maps $\psi_n:A_n\rightarrow A$, give rise to an isomorphism
		$$\lim_{n\rightarrow\infty}\K_*^{\mathrm{top}}(G;A_n)\cong \K_*^{\mathrm{top}}(G;A).$$
	\end{satz}
	\begin{proof}
		Let $\psi^*:\lim_{n\rightarrow\infty}\K_*^{\mathrm{top}}(G;A_n)\rightarrow \K_*^{\mathrm{top}}(G;A)$ be the homomorphism induced by the morphisms $\psi_n: A_n\rightarrow A$. Our aim is to show that $\psi^*$ is an isomorphism. For every proper $G$-space $X$ let $$\psi_X^*: \lim_{n\rightarrow\infty}\mathrm{KK}^G_*(C_0(X),A_n)\rightarrow \mathrm{KK}_*^G(C_0(X),A)$$ be the morphism induced by $\psi_n$ at the level of $X$. Now the structure maps for taking the limit over $X$ are given by left Kasparov products, whereas the structure maps for taking the limit over the $A_n$ is given by right Kasparov products.
		Since the Kasparov product is associative, the limits can be permuted and we get $$\lim_{n\rightarrow\infty}\K_*^{\mathrm{top}}(G;A_n)\cong \lim_X\left( \lim_n \mathrm{KK}^G_*(C_0(X),A_n)\right).$$
		The map $\psi^*$ can then be computed via the maps $\psi_X^*$ by
		$$\lim_X\left( \lim_n \mathrm{KK}^G_*(C_0(X),A_n)\right)\rightarrow \lim_X \mathrm{KK}^G_*(C_0(X),A).$$
		We define a contravariant functor $$\mathcal{F}_H^*(C_0(X)):=\lim_n \mathrm{KK}_*^H(C_0(X),A_{n\mid H}).$$ Then $\mathcal{F}$ is a Going-Down functor. Let $\mathcal{G}$ denote the Going-Down functor $C_0(X)\mapsto \mathrm{KK}^H_*(C_0(X),A_{\mid H})$ from Example \ref{Example:Going-Down functor}. Then the maps $\psi_X$ define a Going-Down transformation $\Psi:\mathcal{F}\rightarrow\mathcal{G}$, such that $\Psi^*(G)=\psi^*$. By Theorem \ref{Theorem:Going-Down Theorem} it is hence enough to prove, that
		$$\lim\limits_n \mathrm{KK}^H_*(C(H^{(0)}),A_{n\mid H})\rightarrow \mathrm{KK}^H_*(C(H^{(0)}),A_{\mid H})$$
		is an isomorphism for all compact open subgroupoids $H$ in $G$.
		For every $n\in\NN$ we have a commutative diagram
		\begin{center}
			\begin{tikzpicture}[description/.style={fill=white,inner sep=2pt}]
			\matrix (m) [matrix of math nodes, row sep=3em,
			column sep=2.5em, text height=1.5ex, text depth=0.25ex]
			{ \mathrm{KK}^H_*(C(H^{(0)}),A_{n\mid H}) & \mathrm{KK}^H_*(C(H^{(0)}),A_{\mid H}) \\
				\K_*(A_{n\mid H}\rtimes H) & \K_*(A_{\mid H}\rtimes H) \\
			};
			\path[->,font=\scriptsize]
			(m-1-1) edge node[auto] {$ (\psi_n)_* $} (m-1-2)
			(m-1-2) edge node[auto] {$ \mu $} (m-2-2)
			(m-2-1) edge node[auto] { $ (\psi_n\rtimes H)_* $ } (m-2-2)
			(m-1-1) edge node[auto] { $ \mu_n $ } (m-2-1)
			;
			\end{tikzpicture},
		\end{center}
		where $\mu$ and $\mu_n$ are the isomorphisms coming from the groupoid version of the Green-Julg theorem (see \cite[Proposition~6.25]{Tu99}). By commutativity of the above diagrams it is hence enough to prove, that the maps $(\psi_n\rtimes H)_*$ induce an isomorphism
		$$\lim\limits_n\K_*(A_{n\mid H}\rtimes H)\rightarrow\K_*(A_{\mid H}\rtimes H).$$
		Using the continuity of $\K$-theory, the result follows from Lemma \ref{Lemma:Proper Groupoids and inductive limits}.
	\end{proof}
	As an immediate consequence, we get the following.
	\begin{kor}\label{Cor:InductiveLimit}
		Let $G$ be an ample groupoid and $(A_n,\varphi_n)_n$ an inductive sequence of $G$-algebras with $A=\lim_{n\rightarrow\infty}A_n$. Suppose $G$ satisfies the Baum-Connes conjecture with coefficients in $A_n$ for all $n\in\NN$. Assume further, that $G$ is exact, or that all the connecting homomorphisms $\varphi_n$ are injective. Then $G$ satisfies the Baum-Connes conjecture with coefficients in $A$.
	\end{kor}
	\section{A Mixed Künneth Formula}
	In this final section we study the $\K$-theory of tensor products by crossed products with ample groupoids in analogy with the results from \cite{CEO}. The main tool is a mixed Künneth formula involving the topological K-theory of the groupoid in question. Under the assumtion that $G$ satisfies the Baum-Connes conjecture with coefficients, one can relate this mixed Künneth formula to the usual Künneth formula for the crossed product.
	Let us recall the usual Künneth formula: We say that a $\mathrm{C}^*$-algebra $A$ satisfies the \textit{Künneth formula} if for all $\mathrm{C}^*$-algebras $B$, there is a canonical short exact sequence
	$$0\longrightarrow \K_*(A)\otimes \K_*(B)\stackrel{\alpha}{\longrightarrow}\K_*(A\otimes B)\stackrel{\beta}{\longrightarrow}\mathrm{Tor}(\K_*(A),\K_*(B))\longrightarrow 0.$$
	The map $\alpha:\K_*(A)\otimes \K_*(B)\rightarrow \K_*(A\otimes B)$ in the above sequence can be obtained using the Kasparov product as the composition
	\begin{center}
		\begin{tikzpicture}[description/.style={fill=white,inner sep=2pt}]
		\matrix (m) [matrix of math nodes, row sep=3em,
		column sep=2em, text height=1.5ex, text depth=0.25ex]
		{ \mathrm{KK}(\CC,A)\otimes \mathrm{KK}_*(\CC,B) &  \mathrm{KK}_*(\CC,A)\otimes \mathrm{KK}_*(A,A\otimes B)\\
			&  \mathrm{KK}_*(\CC,A\otimes B) \\
		};
		\path[->,font=\scriptsize]
		(m-1-1) edge node[auto] {$\id\otimes\sigma_A $} (m-1-2)
		(m-1-2) edge node[auto] {$ \otimes_A $} (m-2-2)
		(m-1-1) edge node[auto] {$ \alpha  $} (m-2-2)
		
		;
		\end{tikzpicture}
	\end{center}
	where $\sigma_A:\KK_*(\CC,B)\rightarrow \KK_*(A,A\otimes B)$ is Kasparov's external tensor product in $\KK$-theory.
	The following result is shown in \cite[Proposition~4.2]{CEO} (extending earlier results by \cite{MR650021}):
	\begin{prop}
		Let $A$ be a separable $\mathrm{C}^*$-algebra. Then $A$ satisfies the Künneth formula if and only if $\alpha:\K_*(A)\otimes \K_*(B)\rightarrow \K_*(A\otimes B)$ is an isomorphism for all separable $\mathrm{C}^*$-algebras $B$ with $\K_*(B)$ free abelian.
	\end{prop}
	The authors in \cite{CEO} then define the class $\mathcal{N}$ to be the class of all separable $\mathrm{C}^*$-algebras such that $\alpha:\K_*(A)\otimes \K_*(B)\rightarrow \K_*(A\otimes B)$ is an isomorphism for all separable $\mathrm{C}^*$-algebras $B$ with $\K_*(B)$ free abelian. It turns out that the class $\mathcal{N}$ is quite large and enjoys many nice permanence properties:
	\begin{enumerate}
		\item The class $\mathcal{N}$ contains the bootstrap class $\mathcal{B}$ (see \cite[Definition~22.3.4]{MR1656031}).
		\item If $A\in\mathcal{N}$ and $B$ is $\mathrm{KK}$-dominated by $A$ (see \cite[Definition~23.10.6]{MR1656031}), then $B\in\mathcal{N}$.
		\item If $0\rightarrow I\rightarrow A\rightarrow A/I\rightarrow 0$ is a semi-split short exact sequence of $\mathrm{C}^*$-algebras such that two of them are in $\mathcal{N}$, then so is the third.
		\item If $A,B\in\mathcal{N}$, then $A\otimes B\in\mathcal{N}$.
		\item If $A=\lim_i A_i$ is an inductive limit, such that each $A_i\in\mathcal{N}$ and, such that all the structure maps are injective, then $A\in\mathcal{N}$.
	\end{enumerate}
	
	Our first goal is, to replace $\K_*(A)$ by the topological $\K$-theory of an ample groupoid with coefficients in a suitable separable $G$-algebra $A$ and define an equivariant version of the map $\alpha$.
	Before we can get into it, we need some preliminary observations on minimal tensor products of $C_0(X)$-algebras:
	
	Recall, that for arbitrary $\mathrm{C}^*$-algebras $A$ and $B$, their minimal tensor product $A\otimes B$ sits as an essential ideal inside $M(A)\otimes M(B)$, and hence, using the universal property of the multiplier algebra, there exists a unique embedding $\iota: M(A)\otimes M(B)\hookrightarrow M(A\otimes B)$, satisfying $\iota(m\otimes n)(a\otimes b)=ma\otimes nb$ and $(a\otimes b)\iota(m\otimes n)=am\otimes bn$. In particular, we have $\iota(ZM(A)\otimes ZM(B))\subseteq ZM(A\otimes B)$. In what follows we will suppress $\iota$ in our notation and view $ZM(A)\otimes ZM(B)$ as a subalgebra of $ZM(A\otimes B)$:
	\begin{prop}\cite[Proposition~3.4]{MR3549520}
		Let $A$ be a $C_0(X)$-algebra with structure map $\Phi_X$ and $B$ a $C_0(Y)$-algebra with structure map $\Phi_Y$. Then $A\otimes B$ is a $C_0(X\times Y)$-algebra with respect to the map $\Phi_X\otimes \Phi_Y$. Moreover, the fibre over $(x,y)\in X\times Y$ is $$(A\otimes B)_{(x,y)}=(A\otimes B)/I_x\otimes B+A\otimes J_y,$$ where $I_x$ and $J_y$ are the ideals corresponding to the fibres $A_x$ and $B_y$ respectively.
	\end{prop}
	In many situations the fibres are much nicer to describe:
	\begin{prop}\label{Prop:FibresMinimalTensorProduct}
		Let $A$ be a $C_0(X)$-algebra, and $B$ be a $C_0(Y)$-algebra. If either $A$ or $B$ is separable and exact, then $$(A\otimes B)_{(x,y)}=A_x\otimes B_y.$$
	\end{prop}
	\begin{proof}
		This is a direct consequence of \cite[IV.3.4.22, Proposition~IV.3.4.23]{MR2188261}.
	\end{proof}
	Now let $A$ and $B$ be $C_0(X)$-algebras over the same space $X$, and let $\Delta:X\rightarrow X\times X$ be the diagonal inclusion. Then we define the minimal balanced tensor product $A\otimes_X B$ of $A$ and $B$ by $\Delta^*(A\otimes B)$. Thus, $A\otimes_X B$ is a $C_0(X)$-algebra by construction. Note, that $A\otimes_X B$ is canonically isomorphic the quotient of $A\otimes B$ by the ideal $\overline{C_0(X\times X\setminus \Delta(X))A\otimes B}$.
	It follows from Proposition \ref{Prop:FibresMinimalTensorProduct} above, that if either $A$ or $B$ is separable and exact, that for all $x\in X$ we have
	$$(A\otimes_X B)_x=A_x\otimes B_x.$$
	With this description of the fibres it is not so hard to see the following:
	\begin{lemma}
		Let $A$ and $B$ be $C_0(X)$-algebras and $f:Y\rightarrow X$ a continuous map. If either $A$ or $B$ is separable and exact, we have $f^*(A\otimes_X B)\cong f^*A\otimes_Y f^*B$.
	\end{lemma}
	\begin{proof}
		Consider the map $f\times f:Y\times Y\rightarrow X\times X$. We will first show, that $f^*A\otimes f^*B$ is canonically isomorphic to $(f\times f)^*(A\otimes B)$ as a $C_0(Y\times Y)$-algebra. Consider the map
		$$\Phi:f^*A\otimes f^*B\rightarrow (f\times f)^*(A\otimes B),$$ 
		which on an elementary tensor $\varphi\otimes \psi\in f^*A\otimes f^*B$ is defined by
		$$\Phi(\varphi\otimes \psi)(y,y')=\varphi(y)\otimes\psi(y')
		\in A_{f(y)}\otimes B_{f(y')}=(f\times f)^*(A\otimes B)_{(y,y')}.$$
		Note, that we use the assumption that either $A$ or $B$ is separable and exact here, to identify the fibres in the last equality. Since
		\begin{align*}\norm{\Phi(\varphi\otimes\psi)}&=\sup\limits_{(y,y')} \norm{\varphi(y)\otimes\psi(y')}\\
		&=\sup\limits_{(y,y')}\norm{\varphi(y)}\norm{\psi(y')}\\
		&\leq\norm{\varphi}\norm{\psi}\\
		&=\norm{\varphi\otimes\psi},
		\end{align*}
		the map $\Phi$ extends to a bounded $C_0(Y\times Y)$-linear $\ast$-homomorphism, which clearly induces an isomorphism on each fibre. Hence $\Phi$ is an isomorphism as desired.
		Observe, that we have $\Delta_X\circ f=(f\times f)\circ \Delta_Y$, where $\Delta_X$ and $\Delta_Y$ denote the diagonal inclusions respectively. Hence we have
		\begin{align*}
		f^*(A\otimes_X B)=(\Delta_X\circ f)^*(A\otimes B)&=((f\times f)\circ \Delta_Y)^*(A\otimes B)\\
		&=\Delta_Y^*((f\times f)^*(A\otimes B))\\
		& \cong\Delta_Y^*(f^*A\otimes f^*B)\\
		&=f^*A\otimes_Y f^*B.
		\end{align*}
	\end{proof}
	Suppose now, that $G$ is an étale Hausdorff groupoid. Suppose further, that $(A,G,\alpha)$ and $(B,G,\beta)$ are groupoid dynamical systems. With the above lemma at hand, it is now easy to define a diagonal action. Suppose that either $A$ or $B$ is separable and exact. Then we define the diagonal action of $G$ on $A\otimes_{G^{(0)}} B$ via the composition
	$$d^*(A\otimes_{G^{(0)}} B)\cong d^*A\otimes_G d^*B\stackrel{\alpha\otimes \beta}{\longrightarrow}r^*A\otimes_G r^*B\cong r^*(A\otimes_{G^{(0)}} B).$$
	Note, that if $(A,G,\alpha)$ is a groupoid dynamical system and $B$ is any $\mathrm{C}^*$-algebra, such that either $A$ or $B$ is separable and exact, then $(A\otimes B,G,\alpha\otimes id)$ is a groupoid dynamical system. The reduced crossed product is compatible with the minimal balanced tensor product in the following way:
	\begin{prop}\cite[Theorem~6.1]{MR3383622}
		There is a natural isomorphism
		$$\Psi:(A\otimes B)\rtimes_{\alpha\otimes id,r} G\rightarrow (A\rtimes_{\alpha,r}G)\otimes B.$$
	\end{prop}
	Before we can proceed, we also need the following:
	\begin{prop}
		Let $A,B$ and $D$ be separable $G$-algebras, such that $D$ is exact. Then there is a homomorphism
		$$\sigma_D:\KK^G(A,B)\rightarrow \KK^G(A\otimes_{G^{(0)}}D,B\otimes_{G^{(0)}}D),$$
		given by associating to an element $(E,\Phi,T)\in\mathbb{E}^G(A,B)$ the triple $(E\otimes_A A\otimes_{G^{(0)}} D, \Phi\otimes \id,T\otimes\id)$.
	\end{prop}
	
	Let us now return to the Künneth formula. Fix a second countable ample Hausdorff groupoid $G$. For ease of notation let us denote its unit space by $X$.
	Let $A$ be a separable exact $G$-algebra and $B$ any $\mathrm{C}^*$-algebra. We wish to define a map
	$$\alpha_G:\K^{\mathrm{top}}_*(G;A) \otimes \K_*(B)\rightarrow \K_*^{\mathrm{top}}(G;A\otimes B).$$
	Consider the trivial group denoted by $1$. Then the canonical groupoid homomorphism $G\rightarrow 1$ induces a homomorphism
	$$\KK_*(\CC,B)\rightarrow \KK^G_*(C_0(X),C_0(X,B)).$$
	Let $\varepsilon$ denote the composition:
	\begin{center}
		\begin{tikzpicture}[description/.style={fill=white,inner sep=2pt}]
		\matrix (m) [matrix of math nodes, row sep=3em,
		column sep=2.5em, text height=1.5ex, text depth=0.25ex]
		{ \K_*(B) & \KK_*(\CC,B) &  \KK^G_*(C_0(X),C_0(X,B))\\
			& &  \KK_*^G(A\otimes_{X}C_0(X),A\otimes_{X}C_0(X,B)) \\
		};
		\path[->,font=\scriptsize]
		(m-1-1) edge node[auto] {$ \cong $} (m-1-2)
		
		(m-1-2) edge node[auto] {$  $} (m-1-3)
		(m-1-3) edge node[auto] {$ \sigma_A $} (m-2-3)
		(m-1-1) edge node[auto] {$ \varepsilon  $} (m-2-3)
		
		;
		\end{tikzpicture}
	\end{center}
	Under the canonical identifications of $G$-algebras $A\otimes_X C_0(X)\cong A$ and $A\otimes_X C_0(X,B)\cong A\otimes B$ we will view $\varepsilon$ as a map
	$$\varepsilon:\K_*(B)\rightarrow \KK_*^G(A,A\otimes B).$$
	Now for any proper and $G$-compact $G$-space $Y\subseteq \mathcal{E}(G)$  we define a map $\alpha_Y$ as the composition
	\begin{center}
		\begin{tikzpicture}[description/.style={fill=white,inner sep=2pt}]
		\matrix (m) [matrix of math nodes, row sep=3em,
		column sep=2.5em, text height=1.5ex, text depth=0.25ex]
		{ \KK_*^G(C_0(Y),A)\otimes \K_*(B) &  \KK_*^G(C_0(Y),A)\otimes \KK_*^G(A,A\otimes B)\\
			&   \KK_*^G(C_0(Y),A\otimes B) \\
		};
		\path[->,font=\scriptsize]
		(m-1-1) edge node[auto] {$\id\otimes\varepsilon $} (m-1-2)
		(m-1-2) edge node[auto] {$ \otimes_A $} (m-2-2)
		(m-1-1) edge node[auto] {$ \alpha_Y  $} (m-2-2)
		
		;
		\end{tikzpicture}
	\end{center}
	Passing to the limit, the maps $\alpha_Y$ induce the desired map
	$$\alpha_G:\K_*^{\mathrm{top}}(G;A)\otimes \K_*(B)\rightarrow \K_*^{\mathrm{top}}(G;A\otimes B).$$
	\begin{defi}
		We denote by $\mathcal{N}_G$ the class of all separable exact $G$-algebras $A$ such that $\alpha_G$ is an isomorphism for all $B$ with $\K_*(B)$ free abelian.
	\end{defi}
	We will now show, that for a $G$-algebra $A$ to be in $\mathcal{N}_G$ also corresponds to satisfying a $G$-equivariant version of the Künneth formula:
	\begin{prop}\label{Prop:Equivariant Kunneth sequence}
		Let $A$ be a separable and exact $G$-algebra. Then $A\in \mathcal{N}_G$ if and only if for every $\mathrm{C}^*$-algebra $B$, there exists a canonical homomorphism $$\beta_G:\K_*^{\mathrm{top}}(G;A\otimes B)\rightarrow \mathrm{Tor}(\K_*^{\mathrm{top}}(G;A),\K_*(B))$$ such that the sequence
		
			$$0\rightarrow \K_*^{\mathrm{top}}(G;A)\otimes\K_*(B)\stackrel{\alpha_G}{\rightarrow}\K_*^{\mathrm{top}}(G;A\otimes B)\stackrel{\beta_G}{\rightarrow}\mathrm{Tor}(\K_*^{\mathrm{top}}(G;A),\K_*(B))\rightarrow 0$$
		is exact.
	\end{prop}
	\begin{proof}
		Let $\mathcal{S}$ denote the category of all separable $\mathrm{C}^*$-algebras with $\ast$-homo\-morphisms as morphisms, and let $\mathbf{Ab}$ denote the category of abelian groups. Consider the functor $F_*:\mathcal{S}\rightarrow\mathbf{Ab}$, given by $F_*(B)=\K_*^{\mathrm{top}}(G;A\otimes B)$ and $F_*(\Phi)=(\id\otimes \Phi)_*$ for a $\ast$-homomorphism $\Phi:B_1\rightarrow B_2$. We will show that $F_*$ is a Künneth functor in the sense of \cite[Definition~3.1]{CEO}, provided that $A\in\mathcal{N}_G$. It is clear, that $F_*$ is stable and homotopy invariant, since the topological $\K$-theory has these properties. To see (K2), combine \cite[Lemma~4.1]{CEO} with \cite[Proposition~5.6]{Tu98}. Item (K3) again follows from the corresponding property of topological K-theory and (K4) is precisely what it means for $A$ to be in the class $\mathcal{N}_G$. Hence an application of \cite[Theorem~3.3]{CEO} completes the proof.	
	\end{proof}
	The class $\mathcal{N}_G$ enjoys many stability properties similar to those of $\mathcal{N}$:
	\begin{lemma}\label{Lemma:Stability Properties of N_G}
		Let $G$ be a second countable ample groupoid. Then the following hold:
		\begin{enumerate}
			\item If $A\in\mathcal{N}_G$ and $B$ is a separable exact $\mathrm{C}^*$-algebra, which is $\KK^G$-dominated by $A$ (i.e. there exist $x\in\KK^G(A,B)$ and $y\in \KK^G(B,A)$ such that $y\otimes x=1_B\in \KK^G(B,B)$), then $B\in\mathcal{N}_G$.
			\item If $0\rightarrow I\rightarrow A\rightarrow A/I\rightarrow 0$ is a semi-split short exact sequence of $G$-algebras such that two of them are in $\mathcal{N}_G$, then so is the third.
			\item If $A\in\mathcal{N}_G$ and $B\in\mathcal{N}$, then $A\otimes B\in\mathcal{N}_G$, where $A\otimes B$ is equipped with the action $\alpha\otimes\id$.
			\item If $(A_n,\varphi_n)_n$ is an inductive sequence of $G$-algebras with injective and $G$-equivariant connecting maps, such that each $A_n\in\mathcal{N}_G$ for all $n\in\NN$, then $A\in\mathcal{N}_G$.
		\end{enumerate}
	\end{lemma}
	\begin{proof}
		For the proof of $(1)$ let $D$ be any $\mathrm{C}^*$-algebra with $\K_*(D)$ free abelian and consider the following commutative diagram:
		\begin{center}
			\begin{tikzpicture}[description/.style={fill=white,inner sep=2pt}]
			\matrix (m) [matrix of math nodes, row sep=3em,
			column sep=3em, text height=1.5ex, text depth=0.25ex]
			{ \K_*^{\mathrm{top}}(G;B)\otimes \K_*(D) & \K_*^{\mathrm{top}}(G;A)\otimes \K_*(D) & \K_*^{\mathrm{top}}(G;B)\otimes \K_*(D)\\
				\K_*^{\mathrm{top}}(G;B\otimes D)& \K_*^{\mathrm{top}}(G;A\otimes D)&\K_*^{\mathrm{top}}(G;B\otimes D) \\
			};
			\path[->,font=\scriptsize]
			(m-1-1) edge node[auto] {$ (\cdot\otimes y)\otimes \id $} (m-1-2)
			(m-1-2) edge node[auto] {$ (\cdot\otimes x)\otimes \id $} (m-1-3)
			(m-2-1) edge node[auto] {$ \otimes \sigma_D(y) $} (m-2-2)
			(m-2-2) edge node[auto] {$ \otimes \sigma_D(x) $} (m-2-3)
			(m-1-1) edge node[auto] {$ \alpha_G  $} (m-2-1)
			(m-1-2) edge node[auto] {$ \alpha_G  $} (m-2-2)
			(m-1-3) edge node[auto] {$ \alpha_G  $} (m-2-3)
			;
			\end{tikzpicture}
		\end{center}
		By assumption, the composition of the horizontal arrows are the identity maps in each row and the middle vertical map is an isomorphism. An easy diagram chase then shows, that the left (and right) vertical arrows must be isomorphisms as well.
		
		For the proof of $(2)$, we first note that exactness passes to ideals (see \cite[Theorem~IV.3.4.3]{MR2188261}), quotients by \cite[Corollary~IV.3.4.19]{MR2188261} and semi-split extensions (see \cite[Theorem~IV.3.4.20]{MR2188261}) by deep results of Kirchberg and Wassermann. By \cite[Lemma~4.1]{CEO} the sequence
		$0\rightarrow I\otimes B\rightarrow A\otimes B\rightarrow A/I\otimes B\rightarrow 0$ is a semi-split short exact sequence as well, and hence $(2)$ follows from an easy application of the Five Lemma.
		
		For $(3)$ let us first observe, that if $A$ and $B$ are separable and exact $\mathrm{C}^*$-algebras, then so is their minimal tensor product $A\otimes B$ by associativity of the minimal tensor product. Now suppose that $A\in\mathcal{N}_G$ and $B\in\mathcal{N}$. Let $D$ be any $\mathrm{C}^*$-algebra with $\K_*(B)$ free abelian. As in the proof of \cite[Lemma~4.4(iii)]{CEO} we can use this fact to make the canonical identification 
		$$\Tor(\K_*^{\mathrm{top}}(G;A),\K_*(B)\otimes \K_*(D))\cong \Tor(\K_*^{\mathrm{top}}(G;A),\K_*(B))\otimes \K_*(D).$$
		Now consider the following commutative diagram:
		\begin{center}
			\begin{tikzpicture}[description/.style={fill=white,inner sep=2pt}]
			\matrix (m) [matrix of math nodes, row sep=3em,
			column sep=3.0em, text height=1.5ex, text depth=0.25ex]
			{ 	0 & 0\\
				\K_*^{\mathrm{top}}(G;A)\otimes \K_*(B)\otimes \K_*(D) & \K_*^{\mathrm{top}}(G;A)\otimes \K_*(B\otimes D)\\
				\K_*^{\mathrm{top}}(G;A\otimes B)\otimes \K_*(D)& \K_*^{\mathrm{top}}(G;A\otimes B\otimes D) \\
				\Tor(\K_*^{\mathrm{top}}(G;A),\K_*(B)\otimes \K_*(D)) & \Tor(\K_*^{\mathrm{top}}(G;A), \K_*(B\otimes D))\\
				0&0\\
			};
			\path[->,font=\scriptsize]
			(m-2-1) edge node[auto] {$ \id\otimes\alpha $} (m-2-2)
			(m-3-1) edge node[auto] {$ \alpha_G $} (m-3-2)
			(m-4-1) edge node[auto] {$ \Tor(\id,\alpha) $} (m-4-2)
			(m-1-1) edge node[auto] {$   $} (m-2-1)
			(m-1-2) edge node[auto] {$   $} (m-2-2)
			(m-2-1) edge node[auto] {$   $} (m-3-1)
			(m-2-2) edge node[auto] {$   $} (m-3-2)
			(m-3-1) edge node[auto] {$   $} (m-4-1)
			(m-3-2) edge node[auto] {$   $} (m-4-2)
			(m-4-1) edge node[auto] {$   $} (m-5-1)
			(m-4-2) edge node[auto] {$   $} (m-5-2)
			;
			\end{tikzpicture}
		\end{center}
		Under the identification of the $\Tor$ groups mentioned above, the first column is the equivariant Künneth sequence for $(A,B)$ tensored with $\K_*(D)$. Thus, using our assumption, that $A\in\mathcal{N}_G$, it is exact by Proposition \ref{Prop:Equivariant Kunneth sequence}. Similarly, the second column is the equivariant Künneth sequence for $(A,B\otimes D)$, and hence exact, too.
		Finally, the top and bottom arrows are isomorphisms, since $B$ was assumed to be in $\mathcal{N}$. By the Five Lemma, the middle vertical map $\alpha_G$ must be an isomorphism as well.
		
		Finally, for item $(4)$ note, that separability clearly passes to sequential inductive limits and exactness passes to inductive limits with injective connecting maps (see \cite[Proposition~IV.3.4.4]{MR2188261}). Hence the result follows from Theorem \ref{Theorem:Continuity of top. K-theory}.
	\end{proof}
	Using the Baum-Connes assembly map we can relate the map $\alpha_G$ to the map $\alpha$ for the crossed product as follows:
	\begin{prop}
		Let $A$ be a separable exact $G$-algebra and $B$ be any $\mathrm{C}^*$-algebra. Then the diagram
		\begin{center}
			\begin{tikzpicture}[description/.style={fill=white,inner sep=2pt}]
			\matrix (m) [matrix of math nodes, row sep=3em,
			column sep=2.5em, text height=1.5ex, text depth=0.25ex]
			{ \K_*^{\mathrm{top}}(G;A)\otimes \K_*(B) & \K_*(A\rtimes_r G)\otimes \K_*(B)\\
				\K_*^{\mathrm{top}}(G;A\otimes B)& \K_*((A\otimes B)\rtimes_r G) \\
			};
			\path[->,font=\scriptsize]
			(m-1-1) edge node[auto] {$ \mu_A\otimes \id $} (m-1-2)
			(m-2-1) edge node[auto] {$ \mu_{A\otimes B} $} (m-2-2)
			(m-1-1) edge node[auto] {$ \alpha_G  $} (m-2-1)
			(m-1-2) edge node[auto] {$ \alpha  $} (m-2-2)
			;
			\end{tikzpicture}
		\end{center}
		commutes. In particular, if $\mu_{A\otimes B}$ is an isomorphism for all $\mathrm{C}^*$-algebras $B$, then $A\in\mathcal{N}_G$ if and only if $A\rtimes_r G\in\mathcal{N}$.
	\end{prop}
	\begin{proof}
		First, note that for all $x\in \K_*(B)$ we have $j_G(\varepsilon(x))=\sigma_{A\rtimes_r G}(x)$.
		Using this, we can easily check commutativity of the above diagram on the level of each $G$-compact subspace $Y\subseteq \mathcal{E}(G)$ as follows: For $y\in \KK^G_*(C_0(Y),A)$ and $x\in\K_*(B)$ we compute
		\begin{align*}
		\mu_{Y,A\otimes B}(\alpha_Y(y\otimes x)) & = [p_Y]\otimes_{C_0(Y)\rtimes_r G} j_G(\alpha_Y(y\otimes x))\\
		& = [p_Y]\otimes_{C_0(Y)\rtimes_r G} j_G(y\otimes_A \varepsilon(x))\\
		& = [p_Y]\otimes_{C_0(Y)\rtimes_r G} (j_G(y)\otimes_{A\rtimes_r G} \sigma_{A\rtimes_r G}(x))\\
		& = \mu_{Y,A}(y)\otimes \sigma_{A\rtimes_r G}(x)\\
		& = \alpha(\mu_{Y,A}(y)\otimes x).
		\end{align*}
		The second statement then follows directly from the commutativity of the diagram.
	\end{proof}
	We are now ready for the main result of this section:
	\begin{satz}\label{Theorem:Kunneth}
		Let $G$ be a second countable ample groupoid and $A$ a separable and exact $G$-algebra. Suppose that $A_{\mid K}\rtimes K\in\mathcal{N}$ for all compact open subgroupoids $K\subseteq G$. Then $A\in\mathcal{N}_G$.
	\end{satz}
	\begin{proof}
		Let $B$ be a fixed $\mathrm{C}^*$-algebra with $\K_*(B)$ free abelian. For each $H\in \mathcal{S}(G)$ define contravariant functors
		$\mathcal{F}_H:\mathcal{C}(H)\rightarrow\mathbf{Ab}$ and $\mathcal{G}_H:\mathcal{C}(H)\rightarrow\mathbf{Ab}$ by
		$$\mathcal{F}_H(C_0(Y)):=\KK^H_*(C_0(Y),A)\otimes \K_*(B),$$
		$$\mathcal{G}_H(C_0(Y)):=\KK^H_*(C_0(Y),A\otimes B).$$
		Both $(\mathcal{F}_H)_{H\in\mathcal{S}(G)}$ and $(\mathcal{G}_H)_{H\in\mathcal{S}(G)}$ define Going-Down functors in the sense of Definition \ref{Def:GDfunctor}.
		
		Moreover, for each $H\in \mathcal{S}(G)$ and every proper $H$-space $Y$ the maps $\alpha_Y$ determine natural transformations $\Lambda_H:\mathcal{F}_H\rightarrow\mathcal{G}_H$, which form a Going-Down transformation $\Lambda$.
		Our assumptions then translate to the fact that $\Lambda_K:\mathcal{F}_K(C(K^{(0)}))\rightarrow \mathcal{G}_K(C(K^{(0)}))$ is an isomorphism for every compact open subgroupoid $K$ of $G$. Hence, by Theorem \ref{Theorem:Going-Down Theorem} the result follows.
	\end{proof}
	The following corollary gives many examples, when the hypothesis of Theorem \ref{Theorem:Kunneth} are satisfied and thus provides many examples of $G$-algebras in class $\mathcal{N}_G$.
	\begin{kor}\label{Cor:Kunneth}
		Let $G$ be a second countable ample groupoid and $A$ be a separable exact $G$-algebra, such that $A_u$ is type I for all $u\in G^{(0)}$. Then $A\in\mathcal{N}_G$.
	\end{kor}
	\begin{proof}
		It follows from \cite[Proposition~10.3]{Tu98}, that $A_{\mid K}\rtimes K$ is a type I $\mathrm{C}^*$-algebra for all compact subgroupoids $K\subseteq G$, and hence it is contained in the bootstrap class $\mathcal{B}\subseteq \mathcal{N}$. The result then follows from Theorem \ref{Theorem:Kunneth}.
	\end{proof}
	We conclude this section by pointing out the connections between Theorem \ref{Theorem:Kunneth} and the Baum-Connes conjecture:
	\begin{prop}\label{Prop:BCandKunneth}
		Let $G$ be a second countable ample groupoid and $A\in\mathcal{N}_G$. Consider the following properties:
		\begin{enumerate}
			\item $G$ satisfies the Baum-Connes conjecture with coefficients in $A\otimes B$ for all separable $\mathrm{C}^*$-algebras $B$ (with respect to the trivial action on the second factor).
			\item $A\rtimes_r G\in\mathcal{N}$.
		\end{enumerate}
		Then $(1)$ implies $(2)$ and the converse holds, provided that $G$ satisfies the Baum-Connes conjecture with coefficients in $A$.
	\end{prop}
	\begin{proof}
		Consider the commutative diagram
		\begin{center}
			\begin{tikzpicture}[description/.style={fill=white,inner sep=2pt}]
			\matrix (m) [matrix of math nodes, row sep=3em,
			column sep=3em, text height=1.5ex, text depth=0.25ex]
			{ 	0 & 0\\
				\K_*^{\mathrm{top}}(G;A)\otimes \K_*(B) & \K_*(A\rtimes_r G)\otimes \K_*(B)\\
				\K_*^{\mathrm{top}}(G;A\otimes B)& \K_*((A\rtimes_r G)\otimes B) \\
				\Tor(\K_*^{\mathrm{top}}(G;A),\K_*(B)) & \Tor(\K_*(A\rtimes_r G), \K_*(B))\\
				0&0\\
			};
			\path[->,font=\scriptsize]
			(m-2-1) edge node[auto] {$ \mu_A\otimes\id $} (m-2-2)
			(m-3-1) edge node[auto] {$ \mu_{A\otimes B} $} (m-3-2)
			(m-4-1) edge node[auto] {$ \Tor(\mu_A,\id) $} (m-4-2)
			(m-1-1) edge node[auto] {$   $} (m-2-1)
			(m-1-2) edge node[auto] {$   $} (m-2-2)
			(m-2-1) edge node[auto] {$ \alpha_G  $} (m-3-1)
			(m-2-2) edge node[auto] {$  \alpha $} (m-3-2)
			(m-3-1) edge node[auto] {$  \beta_G $} (m-4-1)
			(m-3-2) edge node[auto] {$ \beta  $} (m-4-2)
			(m-4-1) edge node[auto] {$   $} (m-5-1)
			(m-4-2) edge node[auto] {$   $} (m-5-2)
			;
			\end{tikzpicture}
		\end{center}
		Since $A\in\mathcal{N}_G$ the left column is exact by Proposition \ref{Prop:Equivariant Kunneth sequence}. Now in the situation of $(1)$, all the horizontal arrows are isomorphisms. Consequently, the right column is also exact, which establishes $(2)$.
		If conversely $A\rtimes_r G\in\mathcal{N}$ and moreover $G$ satisfies the Baum-Connes conjecture with coefficients in $A$, then both columns in the above diagram are exact by Proposition \ref{Prop:Equivariant Kunneth sequence} and \cite[Proposition~4.2]{CEO} respectively. Moreover, the top and bottom horizontal maps are isomorphisms and an application of the Five Lemma completes the proof.
		
	\end{proof}
	Combining Theorem \ref{Theorem:Kunneth} and the preceding proposition we have the following corollary.
	\begin{kor}\label{Corollary:Kunneth}
		Let $G$ be a second countable ample groupoid and $A$ a separable exact $G$-algebra such that $A_{\mid K}\rtimes K\in\mathcal{N}$ for all compact open subgroupoids $K\subseteq G$. If $G$ satisfies the Baum-Connes conjecture with coefficients in $A\otimes B$ for all separable $\mathrm{C}^*$-algebras $B$ (with respect to the trivial action on the second factor), then $A\rtimes_r G\in\mathcal{N}$.		
		In particular, $C_r^*(G)\in\mathcal{N}$, provided that $G$ satisfies the Baum-Connes conjecture with coefficients in $C_0(G^{(0)},B)$ for all separable $\mathrm{C}^*$-algebras $B$ (equipped with the trivial action).
	\end{kor}
	
	\section{Example: Uniform Roe-algebras}
	
	\begin{prop}
		Let $X$ be a discrete metric space with bounded geometry, which admits a coarse embedding into a Hilbert space. Then $C_u^*(X)$ satisfies the Künneth formula.
	\end{prop}
	\begin{proof}
		Since $C_u^*(X)=C_r^*(G(X))$ not separable we cannot apply the results above directly. However, by \cite[Theorem~5.4]{STY02} there exists a second countable a-T-menable groupoid $G'$ such that $G(X)=G'\ltimes \beta X$. Let $\Psi$ denote the canonical isomorphism $C_r^*(G(X))\cong C_r^*(G')$. Then, for every $\mathrm{C}^*$-algebra $B$ we obtain a commutative diagram, where the vertical maps are isomorphisms.
		\begin{center}
			\begin{tikzpicture}[description/.style={fill=white,inner sep=2pt}]
			\matrix (m) [matrix of math nodes, row sep=3em,
			column sep=2.5em, text height=1.5ex, text depth=0.25ex]
			{ \K_*(C_r^*(G(X)))\otimes K_*(B) & \K_*(C_r^*(G(X))\otimes B) \\
				\K_*(C_r^*(G'))\otimes K_*(B) & \K_*(C_r^*(G')\otimes B)  \\
			};
			\path[->,font=\scriptsize]
			(m-1-1) edge node[auto] {$ \alpha $} (m-1-2)
			(m-1-2) edge node[auto] {$ (\Psi\otimes\id_B)_* $} (m-2-2)
			(m-2-1) edge node[auto] { $ \alpha' $ } (m-2-2)
			(m-1-1) edge node[auto] { $ \Psi_*\otimes \id$ } (m-2-1)
			;
			\end{tikzpicture},
		\end{center}
		Since $G'$ is second countable and a-T-menable it satisfies the Baum-Connes conjecture with coefficients. It follows from Corollary \ref{Corollary:Kunneth} that $\alpha'$ is an isomorphism, provided that for every compact open subgroupoid $K$ of $G'$ the $\mathrm{C}^*$-algebra $C_r^*(K)$ satisfies the Künneth formula. But $C^*(K)=C_r^*(K)$ is type I by \cite[Proposition~10.3]{Tu98} and hence satisfies the Künneth formula. Using the commutative diagram above we conclude that $\alpha$ is an isomorphism as well.
	\end{proof}
	The same proof works for $A\rtimes_r G(X)$ whenever $A$ is type I.
	
	\ \newline
	{\bf Acknowledgments}. The content of this paper covers some results from the authors doctoral thesis. He would like to thank his supervisor Siegfried Echterhoff for his support and advice.
	
\bibliographystyle{amsalpha}
\bibliography{Literatur}

\end{document}