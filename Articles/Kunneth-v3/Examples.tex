A natural question is to find examples of $C^*$-algebras for which theorem \ref{Kunneth} ensures the Künneth formula, without this being a consequence of previous known results \cite{RosenbergKunneth},\cite{ChabertEOY},\cite{OY4}. This is actually not an easy question (in the second author's opinion). The first class of examples we give are actually consequences of a combination of previous result, but they are obtained in a straitforward way. The author's thought they were interesting in themselves, and allow to illustrate the power of the Going-Down principle for grupoids (or at least to compare it to other methods). A positive answer will be provided by Uniform Roe algebras of spaces which admits a coarse embedding into Hilbert space. This is, to our knowledge, a new result, and an easy application of the Going-Down principle.\\ %Unfortunately, the author was not able to come up with such an example. We will nevertheless list several applications, as it shows the interest of the Going-Down principle extended to groupoids, even if the results are derivable from the existing litterature.\\

\subsection{Some examples}

Let $\Gamma$ an infinite hyperbolic property T group, and $\Omega$ a Cantor space equipped with an action of $\Gamma$ by homeomorphisms. Then the action groupoid $X\rtimes \Gamma$ is ample, thus in the class $\mathcal C$ by proposition \ref{ampleC}. By Lafforgue's work \cite{lafforgue2012conjecture}, $\Gamma$ satisfies the Baum-Connes conjecture with coefficients so the isomorphism \[A\rtimes_r (\Omega\rtimes \Gamma) \cong (A\otimes C(\Omega))\rtimes_r \Gamma \]
ensures that $\Omega\rtimes \Gamma$ satisfies the Baum-Connes conjecture with coefficients. The last condition will be automatic if $A\rtimes_r H$ satisfies the Künneth formula for $H$ a compact open sugroupoid, which is of the form $U\rtimes F $ with $U $ a $\Gamma$-invariant compact open subset of $\Omega$ and $F$ a finite subgroup $F$ of $\Gamma$. Thus theorem \ref{Kunneth} ensures the following.\\

\textbf{Application 1: } If $\Gamma$ is an infinite hyperbolic property T group, $\Omega$ a Cantor $\Gamma$-space, and $A$ is a $C^*$-algebra such that $(A\otimes C(U))\rtimes_r F$ satisfies the Künneth formula for every $\Gamma$-invariant compact open subset $U$ of $\Omega$ and any finite subgroup $F$ of $\Gamma$, then $A \rtimes_r (X\rtimes \Gamma)$ satisfies the Künneth formula. \\ 

Here again, this result is not new and is a consequence of the Going Down principle (\cite{ChabertEOY}) applied to $\Gamma$. We can also blend the two last examples together: if $\Gamma$ is a discrete group, then it has only one coarse class of left invariant metric, and we denote by $|\Gamma|$ the associated coarse metric space. It is shown in \cite{SkTuYu} that the coarse groupoid is an action groupoid, more precisely \[G(|\Gamma|) \cong \beta |\Gamma| \rtimes \Gamma.\]
This groupoid is again in the class $\mathcal C$ and satisfies Baum-Connes with coefficients. The third condition follows as earlier.\\

\textbf{Application 2:} If $\Gamma$ is a hyperbolic group and $A$ is a $C^*$-algebra such that $(A\otimes C(U))\rtimes_r F$ satisfies the Künneth formula for every $\Gamma$-invariant compact open subset $U$ of $\beta |\Gamma|$ and any finite subgroup $F$ of $\Gamma$, then $A\rtimes_r G(X)$ satisfies the Künneth formula. In particular the uniform Roe algebra \[l^\infty(\Gamma)\rtimes_r \Gamma\cong C_u^* (\Gamma)\] satisfies the Künneth formula.  \\

Recall the following construction from \cite{HLS}. Let $\Gamma$ be a finitely generated residually finite group, and $\mathcal N = {N_i}_i$ a decreasing family of nested finite index normal subgroups, i.e. $N_{i+1} < N_i $, $[\Gamma, N_i]<\infty$ and $\cap N_i =\{e_\Gamma\}$. Following R. Willett, we call the HLS groupoid associated to $(\Gamma,\mathcal N)$ the bundle of groups over the one point compactification $\overline{\N}=\N\cup \{\infty\}$ defined as follows:
\begin{itemize}
\item[$\bullet$] if $n\in\N$, $G_n= \Gamma / N_n $,
\item[$\bullet$] $G_\infty=\Gamma$,
\item[$\bullet$] each fibers is endowed with the discrete topology, and a basis of neighboorhood at $\infty$ is given by   
\[ \mathcal V_{N} = \{(n,g_n) : n\geq N, \pi_n(g_n) = g\}.\]
\end{itemize}
This defines an ample groupoid, and the exact sequence
\[ 0 \rightarrow \oplus \C[\Gamma_n] \rightarrow C_c(G) \rightarrow \C [\Gamma]  \rightarrow 0\]      
induces the following exact sequence of $C^*$-algebras
\[ 0 \rightarrow \oplus \C[\Gamma_n] \rightarrow C_r^*(G) \rightarrow C^*_{\mathcal N} (\Gamma)  \rightarrow 0\]   
where $C^*_{\mathcal N}(\Gamma)$ is the completion of $\C[\Gamma]$ w.r.t. to the norm
\[||x||_{\mathcal N} = \sup_{N\in \mathcal N} ||\lambda_{N} (x)||\quad x\in \C[\Gamma] \]
induced by the quasi-regular representations $\lambda_{N} : C_{max}^*(\Gamma) \rightarrow \mathcal L(l^2(\Gamma/ N))$.  \\ 

Now this exact sequence intertwines the Baum-Connes assembly maps, and the Baum-Connes conjecture for $G_{\mathcal N}(\Gamma)$ is equivalent to $\mu_{\Gamma,\mathcal N}$ being an isomorphism. \\

\textbf{Application 3:} 
\begin{itemize}
\item[$\bullet$] If $\Gamma= \mathbb F_2$ and 
\[N_n = \cap ker \phi \]
for $\phi$ running accross all group homomorphisms from $\Gamma$ to a finite group of cardinality less than $n$, then $C_{\mathcal N}^*(\Gamma) \cong C_{max}^*(\Gamma)$ and $G$ satifies the Baum-Connes conjecture, is ample and satisfies the restriction condition. So we get that $C_r^*(G)$ satisfies the Künneth formula. It is still a result that one can get using the fact that $\Gamma$ being a-T-menable, it is $K$-amenable. Hence $C^*_{max}(\Gamma)$ and $C_r^*(\Gamma)$ are $KK$-equivalent and bootstrap, so that $C_r^*(G)$ also is by extension stability of bootstrapness. A remark of R. Willett is worth mentioning: $\mathbb F_2$ being the fundamental group of the wedge of two circles, it is $KK$-equivalent to $C(\mathbb S^1 \wedge \mathbb S^1)$.\\
\item[$\bullet$] One can artificially try to get rid of bootstrapiness by spatially tensoring this exact sequence by $C_r^*(\Lambda)$ for a infinite hyperbolic property T group. One then get the extension
\[ 0 \rightarrow \oplus \C[\Gamma_n] \otimes_{min} C_r^*(\Lambda) \rightarrow C_r^*(G\times \Lambda) \rightarrow C^*_{\mathcal N} (\Gamma)\otimes_{min} C_r^*(\Lambda)   \rightarrow 0.\]
The restriction principle applies for the groupoid $G_{\mathcal N}(\Gamma)\times\Lambda$, and induces that its reduced $C^*$-algebra satisfies the Künneth formula. But then again, one can deduce this from a previous result, namely the restriction principle for groups. Indeed, apply it to $\Lambda$ with coefficient on the trivial bootstrap $\Lambda$-algebra $C_r^*(G)$.	\\ 
\end{itemize} 

%%%%
%%%%

\subsection{Uniform Roe algebras}

Let $X$ be a countable discrete metric space with bounded geometry. In \cite{SkTuYu} is constructed the coarse groupoid $G(X)$ associated to $X$ where it is shown to be locally compact, ample, Hausdorff and principal. Thus $G(X)$ is in the class $\mathcal C$. Moreover, if $X$ admits a coarse embedding into a Hilbert space iff $G(X)$ is a-T-menable, which implies (\cite{TuThese}) the Baum-Connes conjecture with coefficients for $G(X)$. The coarse groupoid is principal, and has only two invariant subsets and thus only has only three subgroupoids: $G^{(0)}= \beta X$, $G(X)_{|\partial X}$ and $G(X)_{|X}$. As soon as $X$ is not bounded, none of the last two are compact. So the last condition of theorem \ref{Kunneth} reduces to the usual Künneth formula for $A$, and we get the following application. \\

\textbf{Application:} If $X$ admits a coarse embedding into a Hilbert space and $A$ is a $C^*$-algebra satisfying the Künneth formula, then $A\rtimes_r G(X)$ satisfies the Künneth formula. In particular the uniform Roe algebra $C_r^*(G(X))\cong C_u^* (X) $ satisfies the Künneth formula.  \\

Of course, this is not new for complex coefficients: $G(X)$ being a-T-menable implies that $C_r^*(G(X))$ is in the bootstrap class. Notice that this fails for the Roe algebra $C^*(X) \cong l^\infty (X, \mathfrak K(H)) \rtimes_r G(X)$, as $l^\infty (X, \mathfrak K(H))$ fails to satisfy the Künneth formula. \\ 

\begin{prop}
		Let $X$ be a discrete metric space with bounded geometry, which admits a coarse embedding into a Hilbert space. Then $C_u^*(X)$ satisfies the Künneth formula.
	\end{prop}
	\begin{proof}
		Since $C_u^*(X)=C_r^*(G(X))$ not separable we cannot apply the results above directly. However, by \cite[Theorem~5.4]{STY02} there exists a second countable a-T-menable groupoid $G'$ such that $G(X)=G'\ltimes \beta X$. Let $\Psi$ denote the canonical isomorphism $C_r^*(G(X))\cong C_r^*(G')$. Then, for every $\mathrm{C}^*$-algebra $B$ we obtain a commutative diagram, where the vertical maps are isomorphisms.
		\begin{center}
			\begin{tikzpicture}[description/.style={fill=white,inner sep=2pt}]
			\matrix (m) [matrix of math nodes, row sep=3em,
			column sep=2.5em, text height=1.5ex, text depth=0.25ex]
			{ \K_*(C_r^*(G(X)))\otimes K_*(B) & \K_*(C_r^*(G(X))\otimes B) \\
				\K_*(C_r^*(G'))\otimes K_*(B) & \K_*(C_r^*(G')\otimes B)  \\
			};
			\path[->,font=\scriptsize]
			(m-1-1) edge node[auto] {$ \alpha $} (m-1-2)
			(m-1-2) edge node[auto] {$ (\Psi\otimes\id_B)_* $} (m-2-2)
			(m-2-1) edge node[auto] { $ \alpha' $ } (m-2-2)
			(m-1-1) edge node[auto] { $ \Psi_*\otimes \id$ } (m-2-1)
			;
			\end{tikzpicture},
		\end{center}
		Since $G'$ is second countable and a-T-menable it satisfies the Baum-Connes conjecture with coefficients. It follows from Corollary \ref{Corollary:Kunneth} that $\alpha'$ is an isomorphism, provided that for every compact open subgroupoid $K$ of $G'$ the $\mathrm{C}^*$-algebra $C_r^*(K)$ satisfies the Künneth formula. But $C^*(K)=C_r^*(K)$ is type I by \cite[Proposition~10.3]{Tu98} and hence satisfies the Künneth formula. Using the commutative diagram above we conclude that $\alpha$ is an isomorphism as well.
	\end{proof}
	The same proof works for $A\rtimes_r G(X)$ whenever $A$ is type I.