A natural question is to find examples of $C^*$-algebras for which theorem \ref{Kunneth} ensures the Künneth formula, without this being a consequence of previous known results \cite{RosenbergKunneth},\cite{ChabertEOY},\cite{OY4}. This is actually not an easy question (in the second author's opinion). The first class of examples we give are actually also consequences of a combination of previous result, but they are obtained here in a straitforward way using the Going-Down principle for groupoids. The author's thought they were interesting in themselves, and allow to illustrate the power of our results (or at least to compare it to other methods). A genuine new application will be provided by Uniform Roe algebras of spaces which admits a coarse embedding into Hilbert space. This is, to our knowledge, a new result, and an easy application of the Going-Down principle.\\ %Unfortunately, the author was not able to come up with such an example. We will nevertheless list several applications, as it shows the interest of the Going-Down principle extended to groupoids, even if the results are derivable from the existing litterature.\\

\subsection{Some examples}

Let $\Gamma$ an infinite hyperbolic property T group, and $\Omega$ a Cantor space equipped with an action of $\Gamma$ by homeomorphisms. Then the action groupoid $X\rtimes \Gamma$ is ample, thus in the class $\mathcal C$ by proposition \ref{ampleC}. By Lafforgue's work \cite{lafforgue2012conjecture}, $\Gamma$ satisfies the Baum-Connes conjecture with coefficients so the isomorphism \[A\rtimes_r (\Omega\rtimes \Gamma) \cong (A\otimes C(\Omega))\rtimes_r \Gamma \]
ensures that $\Omega\rtimes \Gamma$ satisfies the Baum-Connes conjecture with coefficients. The last condition will be automatic if $A\rtimes_r H$ satisfies the Künneth formula for $H$ a compact open sugroupoid, which is of the form $U\rtimes F $ with $U $ a $\Gamma$-invariant compact open subset of $\Omega$ and $F$ a finite subgroup $F$ of $\Gamma$. Thus theorem \ref{Kunneth} ensures the following.\\

\textbf{Application 1: } If $\Gamma$ is an infinite hyperbolic property T group, $\Omega$ a Cantor $\Gamma$-space, and $A$ is a $C^*$-algebra such that $(A\otimes C(U))\rtimes_r F$ satisfies the Künneth formula for every $\Gamma$-invariant compact open subset $U$ of $\Omega$ and any finite subgroup $F$ of $\Gamma$, then $A \rtimes_r (X\rtimes \Gamma)$ satisfies the Künneth formula. \\ 

Here again, this result is not new and is a consequence of the Going Down principle (\cite{ChabertEOY}) applied to $\Gamma$. We can also blend the two last examples together: if $\Gamma$ is a discrete group, then it has only one coarse class of left invariant metric, and we denote by $|\Gamma|$ the associated coarse metric space. It is shown in \cite{SkTuYu} that the coarse groupoid is an action groupoid, more precisely \[G(|\Gamma|) \cong \beta |\Gamma| \rtimes \Gamma.\]
This groupoid is again in the class $\mathcal C$ and satisfies Baum-Connes with coefficients. The third condition follows as earlier.\\

\textbf{Application 2:} If $\Gamma$ is a hyperbolic group and $A$ is a $C^*$-algebra such that $(A\otimes C(U))\rtimes_r F$ satisfies the Künneth formula for every $\Gamma$-invariant compact open subset $U$ of $\beta |\Gamma|$ and any finite subgroup $F$ of $\Gamma$, then $A\rtimes_r G(X)$ satisfies the Künneth formula. In particular the uniform Roe algebra \[l^\infty(\Gamma)\rtimes_r \Gamma\cong C_u^* (\Gamma)\] satisfies the Künneth formula.  \\

Recall the following construction from \cite{HLS}. Let $\Gamma$ be a finitely generated residually finite group, and $\mathcal N = {N_i}_i$ a decreasing family of nested finite index normal subgroups, i.e. $N_{i+1} < N_i $, $[\Gamma, N_i]<\infty$ and $\cap N_i =\{e_\Gamma\}$. Following R. Willett, we call the HLS groupoid associated to $(\Gamma,\mathcal N)$ the bundle of groups over the one point compactification $\overline{\N}=\N\cup \{\infty\}$ defined as follows:
\begin{itemize}
\item[$\bullet$] if $n\in\N$, $G_n= \Gamma / N_n $,
\item[$\bullet$] $G_\infty=\Gamma$,
\item[$\bullet$] each fibers is endowed with the discrete topology, and a basis of neighboorhood at $\infty$ is given by   
\[ \mathcal V_{N} = \{(n,g_n) : n\geq N, \pi_n(g_n) = g\}.\]
\end{itemize}
This defines an ample groupoid, and the exact sequence
\[ 0 \rightarrow \oplus \C[\Gamma_n] \rightarrow C_c(G) \rightarrow \C [\Gamma]  \rightarrow 0\]      
induces the following exact sequence of $C^*$-algebras
\[ 0 \rightarrow \oplus \C[\Gamma_n] \rightarrow C_r^*(G) \rightarrow C^*_{\mathcal N} (\Gamma)  \rightarrow 0\]   
where $C^*_{\mathcal N}(\Gamma)$ is the completion of $\C[\Gamma]$ w.r.t. to the norm
\[||x||_{\mathcal N} = \sup_{N\in \mathcal N} ||\lambda_{N} (x)||\quad x\in \C[\Gamma] \]
induced by the quasi-regular representations $\lambda_{N} : C_{max}^*(\Gamma) \rightarrow \mathcal L(l^2(\Gamma/ N))$.  \\ 

Now this exact sequence intertwines the Baum-Connes assembly maps, and the Baum-Connes conjecture for $G_{\mathcal N}(\Gamma)$ is equivalent to $\mu_{\Gamma,\mathcal N}$ being an isomorphism. \\

\textbf{Application 3:} 
\begin{itemize}
\item[$\bullet$] If $\Gamma= \mathbb F_2$ and 
\[N_n = \cap ker \phi \]
for $\phi$ running accross all group homomorphisms from $\Gamma$ to a finite group of cardinality less than $n$, then $C_{\mathcal N}^*(\Gamma) \cong C_{max}^*(\Gamma)$ and $G$ satifies the Baum-Connes conjecture, is ample and satisfies the restriction condition. So we get that $C_r^*(G)$ satisfies the Künneth formula. It is still a result that one can get using the fact that $\Gamma$ being a-T-menable, it is $K$-amenable. Hence $C^*_{max}(\Gamma)$ and $C_r^*(\Gamma)$ are $KK$-equivalent and bootstrap, so that $C_r^*(G)$ also is by extension stability of bootstrapness. A remark of R. Willett is worth mentioning: $\mathbb F_2$ being the fundamental group of the wedge of two circles, it is $KK$-equivalent to $C(\mathbb S^1 \wedge \mathbb S^1)$.\\
\item[$\bullet$] One can artificially try to get rid of bootstrapiness by spatially tensoring this exact sequence by $C_r^*(\Lambda)$ for a infinite hyperbolic property T group. One then get the extension
\[ 0 \rightarrow \oplus \C[\Gamma_n] \otimes_{min} C_r^*(\Lambda) \rightarrow C_r^*(G\times \Lambda) \rightarrow C^*_{\mathcal N} (\Gamma)\otimes_{min} C_r^*(\Lambda)   \rightarrow 0.\]
The restriction principle applies for the groupoid $G_{\mathcal N}(\Gamma)\times\Lambda$, and induces that its reduced $C^*$-algebra satisfies the Künneth formula. But then again, one can deduce this from a previous result, namely the restriction principle for groups. Indeed, apply it to $\Lambda$ with coefficient on the trivial bootstrap $\Lambda$-algebra $C_r^*(G)$.	\\ 
\end{itemize} 

%%%%
%%%%

\subsection{Uniform and maximal Roe algebras}

Let $X$ be a countable discrete metric space with bounded geometry. In \cite{SkTuYu} is constructed the coarse groupoid $G(X)$ associated to $X$ where it is shown to be locally compact, ample, Hausdorff and principal. Thus $G(X)$ is in the class $\mathcal C$. Moreover, if $X$ admits a coarse embedding into a Hilbert space iff $G(X)$ is a-T-menable, which implies (\cite{TuThese}) the Baum-Connes conjecture with coefficients for $G(X)$.\\

%\textbf{Application:} If $X$ admits a coarse embedding into a Hilbert space and $A$ is a $C^*$-algebra satisfying the Künneth formula, then $A\rtimes_r G(X)$ satisfies the Künneth formula. In particular the uniform Roe algebra $C_r^*(G(X))\cong C_u^* (X) $ satisfies the Künneth formula.  \\

%Of course, this is not new for complex coefficients: $G(X)$ being a-T-menable implies that $C_r^*(G(X))$ is in the bootstrap class. Notice that this fails for the Roe algebra $C^*(X) \cong l^\infty (X, \mathfrak K(H)) \rtimes_r G(X)$, as $l^\infty (X, \mathfrak K(H))$ fails to satisfy the Künneth formula. \\ 

Let $X$ be a discrete bounded geometry metric space, i.e. means that its $R$-balls have uniformly bounded cardinality, or
\[\sup_{x\in X} |B(x,R)| < \infty \quad \forall R>0.\]
Motivated by index theory in the setting of non-compact Riemanian manifolds, J. Roe introduced a $C^*$-algebra $C^*(X)$, now called the Roe algebra of $X$. It comes in different versions, which are all completions of the $*$-algebra of locally compact finite propagation operators, of which we now recall the definition. If $H$ is an auxiliary separable Hilbert space and $R>0$, define 
\[ \C_R[X] = \{T\in\mathcal B(l^2(X)\otimes H)\text{ s.t. } T_{xy}\in \mathfrak K(H) \text{ and } T_{xy} = 0 \text{ if } d(x,y)>R \} \]
as a closed subspace of $\mathcal B(l^2(X)\otimes H)$. Then 
\[\C[X] = \cup_{R>0} \C_R[X]\]
is a $*$-algebra, and
\begin{itemize}
\item[$\bullet$] if $H=\C$, the completion of $\C[X]$ inside $\mathcal B(l^2(X))$ is defined to be the \textit{uniform Roe algebra} $C_u^*(X)$ of $X$,
\item[$\bullet$] if $H=l^2(\N)$, the completion of $\C[X]$ inside $\mathcal B(l^2(X)\otimes H)$ is defined to be the \textit{Roe algebra} $C^*(X)$  of $X$,
\item[$\bullet$] the envelopping $C^*$-algebra of $\C[X]$ is called the \textit{maximal Roe algebra} $C_{max}^*(X)$  of $X$.
\end{itemize}
In \cite{SkTuYu}, G. Skandalis, J.-L. Tu and G. Yu define a locally compact Hausdorff ample groupoid $G(X)$, called the \textit{coarse groupoid} of $X$, such that the convolution algebra $C_c(G(X))$ is $*$-isomorphic to $\C[X]$, and 
\begin{itemize}
\item[$\bullet$] $C^*_r(G(X)) \cong C_u^*(X)$,
\item[$\bullet$] $l^\infty_X=l^\infty(X, \mathfrak K(H))$ is a $G(X)$-algebra and $C^*(X)\cong l^\infty_X \rtimes_r G(X)$,
\item[$\bullet$] $C_{max}^*(G(X)) \cong C_{max}^*(X)$.
\end{itemize}
An interesting feature of the coarse groupoid is that its dynamical properties reflect the coarse properties of $X$. Indeed,
\begin{itemize}
\item[$\bullet$] $X$ has property A iff $G(X)$ is amenable,
\item[$\bullet$] $X$ embeds into Hilbert space iff $G(X)$ is a-T-menable,
\item[$\bullet$] $X$ admits a fibred coarse embedding into Hilbert space iff $G(X)_{|\partial \beta X }$ is a-T-menable (see \cite{FinnSellFibred}). Here, $\partial \beta X$ is the complement of $X$ in $\beta X$, hence is a closed $G(X)$-invariant subset of $\beta X$.
\end{itemize}
Fibred coarse embeddings were introduced by Chen, Wang and Yu in \cite{ChenWangYu}.\\

One would wish to apply the results of the present work directly, but the problem is that $G(X)$ is not second-countable, so that the $C^*$-algebras above are not separable. However, by \cite[Theorem~5.4]{STY02} there exists a second countable a-T-menable groupoid $G'$ such that $G(X)=G'\ltimes \beta X$. Moreover, we may write $\beta X$ as an inverse limit $\lim\limits_{\longleftarrow} Y_i$, such that:
\begin{itemize}
\item Each $Y_i$ is a metrizable quotient of $\beta X$.
\item The action of $G'$ on $\beta X$ factors through an action of $G'$ on $Y_i$, making the quotient map $G'$-equivariant.
\item For each $i\leq j$ the canonical map $Y_j\rightarrow Y_i$ is $G'$-equivariant and surjective.
\end{itemize}
We hence get that $G(X)$ can be written as a projective limit of second-countable ample groupoid. 

\begin{satz}
		Let $X$ be a discrete metric space with bounded geometry, which admits a coarse embedding into a Hilbert space. Then $C_u^*(X)$ satisfies the Künneth formula.
	\end{satz}
	\begin{proof}
%However, by \cite[Theorem~5.4]{STY02} there exists a second countable a-T-menable groupoid $G'$ such that $G(X)=G'\ltimes \beta X$. Moreover, we may write $\beta X$ as an inverse limit $\lim\limits_{\longleftarrow} Y_i$, such that:
%		\begin{itemize}
%			\item Each $Y_i$ is a metrizable quotient of $\beta X$.
%			\item The action of $G'$ on $\beta X$ factors through an action of $G'$ on $Y_i$, making the quotient map $G'$-equivariant.
%			\item For each $i\leq j$ the canonical map $Y_j\rightarrow Y_i$ is $G'$-equivariant and surjective.
%		\end{itemize}
		With the notations above, we get an inductive system $(C(Y_i))_i$ of separable $G'$-algebras with injective and $G'$-equivariant connecting homomorphisms. Hence $(C(Y_i)\rtimes_r G')_i$ is an inductive system with injective connecting maps as well and we can apply Lemma \ref{Lemma:Proper Groupoids and inductive limits} to obtain
		$$C_u^*(X)=C_r^*(G(X))=C(\beta X)\rtimes_r G'=\lim\limits_{\longrightarrow} C(Y_i)\rtimes_r G'.$$
		Now $G'$ is a second countable ample groupoid and satisfies the Baum-Connes conjecture for all coefficients by \cite{Tu98}. If $K\subseteq G'$ is a compact open subgroupoid, then $C(Y_i)_{\mid K}\rtimes K=C_r^*(K\ltimes(Y_i)_{\mid K})$ is the $\mathrm{C}^*$-algebra of a compact groupoid, and hence contained in the class $\mathcal{N}$. By Corollary \ref{Corollary:Kunneth} it follows that $C(Y_i)\rtimes_r G'$ satisfies the Künneth formula for all $i$, i.e. for every $\mathrm{C}^*$-algebra $B$ we obtain canonical short exact sequences
		$$0\rightarrow \K_*(C(Y_i)\rtimes_r G')\otimes \K_*(B)\rightarrow \K_*((C(Y_i)\rtimes_r G')\otimes B)\rightarrow \Tor(\K_*(C(Y_i)\rtimes_r G'),\K_*(B))\rightarrow 0.$$
		Since the connecting maps of the directed system $(C(Y_i)\rtimes_r G')_i$ are all injective by construction, we can apply \cite[II.9.6.6]{MR2188261} to get $$\lim\limits_{\longrightarrow}(C(Y_i)\rtimes_r G')\otimes B=C_u^*(X)\otimes B.$$
		As the (algebraic) tensor product functor and the Tor functor commute with inductive limits, and $\mathrm{K}$-theory is continuous, we obtain a short exact sequence
		$$0\rightarrow \K_*(C_u^*(X))\otimes \K_*(B)\rightarrow \K_*(C_u^*(X)\otimes B)\rightarrow \Tor(\K_*(C_u^*(X)),\K_*(B))\rightarrow 0,$$
		in the limit, as desired.
	\end{proof}
	The same proof works for $A\rtimes_r G(X)$ whenever $A$ is type I.

\begin{satz}
		Let $X$ be a discrete metric space with bounded geometry, which admits a fibred coarse embedding into a Hilbert space. Then $C_{max}^*(X)$ satisfies the Künneth formula.
\end{satz}
\begin{proof}
The closed saturated subset $\partial \beta X$ gives rise to the following exact sequence of $C^*$-algebras
\[0 \rightarrow C_{max}^*(X\times X) \rightarrow C^*_{max}(G(X)) \rightarrow C^*_{max}(G(X)_{|\partial \beta X}) \rightarrow 0.\] 
The groupoid $X\times X$ being proper, the $C^*$-algebra on the left side is of type I and satisfies the Künneth formula. By the above, 
\[G(X)_{|\partial \beta X} = \partial \beta X \rtimes G' =\lim\limits_{\longleftarrow} X_i\cap G', \]
where $X_i$ is the image of $\partial \beta X$ under the $G(X)$-equivariant quotient map $\beta X \rightarrow Y_i$.\\

By hypothesis, $X$ admits a fibred embedding into Hilbert space, and $G(X)_{|\partial \beta X}$ is a-T-menable. The same argument as in the previous result ensures that $C^*_{r}(G(X)_{|\partial \beta X})$ satisfies the Künneth formula. But a-T-menable groupoids have $K$-amenable $C^*$-algebras by \cite{TuThese}, so that $C^*_{max}(G(X)_{|\partial \beta X})$ satisfies the Künneth formula.%The latter is isomorphic to $C^*_{max}(G(X)_{|\partial \beta X})$ as  
The Kûnneth formula is stable by extension, which concludes the proof.
\end{proof}