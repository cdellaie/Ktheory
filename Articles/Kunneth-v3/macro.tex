%% Christian's macro %%%
\usepackage[a4paper, hmargin={2.8cm, 2.8cm}, vmargin={2.5cm, 2.5cm}]{geometry}
\usepackage[ngerman, english]{babel}
\usepackage[utf8]{inputenc}
\usepackage{pdfsync}
\usepackage{verbatim}
\usepackage[onehalfspacing]{setspace}
\usepackage{amsmath}
\usepackage{amsthm}
\usepackage{amssymb}
\usepackage{amsfonts}

 \usepackage{paralist}
\theoremstyle{theorem}
\newtheorem{satz}{Theorem}[section]

  \newtheorem{lemma}[satz]{Lemma}
  \newtheorem{kor}[satz]{Corollary}
  \newtheorem{thmx}{Theorem}
   \renewcommand{\thethmx}{\Alph{thmx}}
  \newtheorem{corx}[thmx]{Corollary}
  \newtheorem{prop}[satz]{Proposition}
  \newtheorem{conjecture}[satz]{Vermutung}
  \theoremstyle{definition}
 \newtheorem{defi}[satz]{Definition}
  \newtheorem{bem}[satz]{Remark}
  \newtheorem{aufgabe}[satz]{Aufgabe}
  \newenvironment{beweis}%
    {\begin{proof}[Beweis]}
    {\end{proof}}
  \newtheorem{ex}[satz]{Example}
    \newtheorem{exs}[satz]{Examples}
  \newtheorem{beispiele}[satz]{Beispiele}


\newcommand{\norm}[1]{\lVert#1\rVert}   %Norm{} befehl
\newcommand{\betrag}[1]{\lvert#1\rvert}
\newcommand{\KK}{\mathrm{KK}}
\newcommand{\RKK}{\mathcal{R}\mathrm{KK}}
\newcommand{\Tor}{\mathrm{Tor}}
\newcommand{\K}{\mathrm K}
\newcommand{\EE}{\mathbb E}
\newcommand{\RR}{\mathbb R}
\newcommand{\QQ}{\mathbb Q}
\newcommand{\CC}{\mathbb C}
\newcommand{\NN}{\mathbb N}
\newcommand{\ZZ}{\mathbb Z}
\newcommand{\FF}{\mathbb F}
\newcommand{\TT}{\mathbb T}
\newcommand{\HH}{\mathbb H}
\newcommand{\lk}{\langle}
\newcommand{\rk}{\rangle}
\newcommand{\id}{\text{id}}
\newcommand{\eps}{\varepsilon}
\setcounter{MaxMatrixCols}{19}
\usepackage{tikz}
\usetikzlibrary{matrix,arrows}
\usepackage{hyperref}
\allowdisplaybreaks

%%%%%%%%%
%%%%%%%%%

%% Clement's macro %%%

%\usepackage[frenchb,british]{babel}
\usepackage{amsfonts}
\usepackage{amsmath}
\usepackage{amssymb}
%\usepackage[T1]{fontenc}
\usepackage[utf8]{inputenc}
\usepackage{amsthm}
\usepackage{graphicx}
\usepackage{tikz}
\usepackage{tikz-cd}
\usepackage{hyperref}
\usepackage{amssymb}

\hypersetup{                    % parametrage des hyperliens
    colorlinks=true,                % colorise les liens
    breaklinks=true,                % permet les retours à la ligne pour les liens trop longs
    urlcolor= blue,                 % couleur des hyperliens
    linkcolor= blue,                % couleur des liens internes aux documents (index, figures, tableaux, equations,...)
    citecolor= blue               % couleur des liens vers les references bibliographiques
    }

%Commandes

\theoremstyle{definition}
\newtheorem{definition}{Definition}[section]
\newtheorem{thm}[definition]{Theorem}
%\newtheorem{ex}{Exercice}
\newtheorem{lem}[definition]{Lemma}
\newtheorem*{dem}{Proof}
%\newtheorem{prop}[definition]{Proposition}
\newtheorem{cor}[definition]{Corollary}
\newtheorem{conj}[definition]{Conjecture}
\newtheorem{Res}{Result}
\newtheorem{Expl}[definition]{Example}
%\newtheorem{rk}[definition]{Remark}

% French style
\newtheorem*{definitionfr}{Définition}
\newtheorem*{propfr}{Proposition}
\newtheorem*{thmfr}{Théorème}
\newtheorem*{corfr}{Corollaire}

\newcommand{\N}{\mathbb N}
\newcommand{\Z}{\mathbb Z}
\newcommand{\R}{\mathbb R}
\newcommand{\C}{\mathbb C}
\newcommand{\Hil}{\mathcal H}
\newcommand{\Mn}{\mathcal M _n (\mathbb C)}
%\newcommand{\K}{\mathbb K}
\newcommand{\B}{\mathbb B}
\newcommand{\Cat}{\mathbb B / \mathbb K}
\newcommand{\G}{\mathcal G }

% Style
% Package Fancyhdr, documentation at
% http://www.xm1math.net/doculatex/entetepied.html

\setlength\parindent{0pt}

\usepackage{geometry}
\geometry{hmargin=2.5cm,vmargin=2.5cm}
%\usepackage[inner=2.5cm,outer=1.5cm,bottom=2cm]{geometry}
%\pagestyle{headings}

\usepackage{fancyhdr}
%\pagestyle{fancy}

\renewcommand{\headrulewidth}{1pt}
\fancyhead[LE]{\nouppercase{\leftmark}}
\fancyhead[RO]{\nouppercase{\rightmark}}
%\fancyhead[LE]{Chapter \thechapter \leftmark}
%\fancyhead[RO]{ \thesection \rightmark}
\fancyhead[RE,LO]{}

%\fancyhead[R]{\thepage}
%\fancyhead[C]{\leftmark}

\renewcommand{\footrulewidth}{0pt}






















