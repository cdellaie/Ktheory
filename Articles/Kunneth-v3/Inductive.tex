In this section we will show that an inductive limit of $G$-algebras with $G$-equivariant connecting morphisms is again a $G$-algebra in a canonical fashion. These results should be known to the experts but since we could not find a suitable reference and in order to keep the exposition self-contained we elaborate on the details. We start of by considering $C_0(X)$-algebras:
	Let $(A_i,\varphi_{i,j})$ be an inductive system of $\mathrm{C}^*$-al\-gebras, where each $A_i$ is a $C_0(X)$-algebra, such that the connecting homomorphisms $\varphi_i$ are $C_0(X)$-linear.
	If $A=\lim\limits_{\rightarrow}A_i$, then $A$ is a $C_0(X)$-algebra in a canonical way. This is surely well-known to the experts, but we could not find a proper reference, so we include the details.
	
	Let us start by recalling the construction of the limit algebra $A$:
	Consider the algebra $$\widetilde{A}=\lbrace (a_i)_i\in\prod\limits_{i\in I}A_i\mid \exists i_0: a_{i}=\varphi_{i,i_0}(a_{i_0})\forall i\geq i_0\rbrace.$$
	Then $A$ is the closure of the image of $\widetilde{A}$ under the quotient map $q:\prod A_i\rightarrow \prod A_i/\bigoplus A_i$.
	Now if $f\in C_0(X)$, then $C_0(X)$-linearity of the $\varphi_{i,j}$ implies, that $\widetilde{A}$ is invariant under component-wise multiplication with $f$. It also leaves the ideal $\bigoplus A_i$ invariant. Hence we get a well-defined linear map
	$q(\widetilde{A})\rightarrow q(\widetilde{A})$ by $f\cdot q((a_i)_i):=q((f\cdot a_i)_i)$. Using the equality $\norm{q((a_i)_i)}=\lim \norm{a_i}$ we get
	$\norm{q((f\cdot a_i)_i)}=\lim \norm{f\cdot a_i}\leq \norm{f}\lim \norm{a_i}=\norm{f}\norm{q((a_i)_i)}$. Consequently, $f\cdot$ extends to a bounded linear map $A\rightarrow A$, actually to an element in $Z(M(A))$, where the adjoint is given by $\overline{f}\cdot$. Thus, we have constructed a $*$-homomorphism $\Phi:C_0(X)\rightarrow Z(M(A))$. 
	\begin{lemma}\label{Lem:InductiveLimitsOfC_0(X)-algebras}
		The $*$-homomorphism $\Phi$ from above is non-degener\-ate. Consequently, $A$ is a $C_0(X)$-algebra such that the canonical maps $\psi_{i}:A_i\rightarrow A$ are $C_0(X)$-linear.
	\end{lemma}
	\begin{proof}
		Let $a\in A$ and $\varepsilon>0$ be given. By construction of the inductive limit $\bigcup_{i\in I} \psi_i(A_i)$ is dense in $A$, so there exists an $i\in I$ and $b\in A_i$ such that $\norm{\psi_i(b)-a}<\frac{\varepsilon}{2}$. Since the structure homomorphism for $A_i$ is non-degenerate we can also find $f\in C_0(X)$ and $c\in A_i$ such that $\norm{b-fc}< \frac{\varepsilon}{2\norm{\psi_i}}$, and hence $\norm{\psi_i(b)-f\psi_i(c)}<\frac{\varepsilon}{2}$.
		Combining the above inequalites we obtain
		$\norm{f\psi_i(c)-a}<\norm{f\psi_i(c)-\psi_i(b)}+\norm{\psi_i(b)-a}<\varepsilon$.
	\end{proof}
	
	We will now identify the fibres of the limit algebra:
	\begin{lemma}
		Let $(A_i,\varphi_{i,j})$ be an inductive system of $C_0(X)$-al\-gebras and $A=\lim_{i} A_i$. Then, for every $x\in X$, $((A_i)_x,(\varphi_{i,j})_x)$ is an inductive system of $\mathrm{C}^*$-algebras and
		$$ \lim\limits_{i} (A_i)_x\cong A_x.$$
	\end{lemma}
	\begin{proof}
		It is immediate, that $((A_i)_x,(\varphi_{i,j})_x)$ is indeed an inductive sequence of $\mathrm{C}^*$-algebras. Hence we only need to identify the limit.
		Let $\pi_{i,x}:A_i\rightarrow (A_i)_x$ denote the quotient maps onto the fibres and $\psi_{i,x}:(A_i)_x\rightarrow \lim\limits_{i}(A_i)_x$ the canonical maps. By the universal property of the limit we obtain a surjective $*$-homo\-morphism $$\pi:A\rightarrow \lim\limits_{i}(A_i)_x.$$
		It remains to show that the kernel of $\pi$ coincides with the ideal $I_x=\overline{C_0(X\setminus\lbrace x\rbrace)A}$ of $A$.
		If $a=\psi_i(b)$ for some $b\in A_i$ and $f\in C_0(X\setminus\lbrace x\rbrace)$, then $\pi(fa)=\pi(f\psi_i(b))=\pi(\psi_i(fb))=\psi_{i,x}(\pi_{i,x}(fb))=0$. By continuity we get $I_x\subseteq ker(\pi)$.
		
		Suppose conversely that $a\in ker(\pi)$ and $\varepsilon>0$ is given. First we can find $i\in I$ and $b\in A_i$ such that $\norm{a-\psi_i(b)}<\frac{\varepsilon}{3}$. Thus,
		$\norm{\psi_{i,x}(\pi_{i,x}(b))}=\norm{\pi(\psi_i(b))}=\norm{\pi(\psi_i(b)-a)}\leq \norm{a-\psi_i(b)}<\frac{\varepsilon}{3}$.
		Upon replacing $b$ and $i$ by $\varphi_{j,i}(b)$ for $j\geq i$ big enough we can actually assume that $\norm{\pi_{x,i}(b)}<\frac{\varepsilon}{3}$. Then there exists some $b'\in A_i$ such that $\norm{b-b'}<\frac{\varepsilon}{3}$ and $\pi_{i,x}(b')=0$. Hence there must be $b''\in A_i$ and $\varphi\in C_0(X\setminus\lbrace x\rbrace)$ such that $\norm{b'-\varphi b''}<\frac{\varepsilon}{3}$.
		Putting things together we obtain
		$$\norm{a-\varphi\psi_i(b'')}\leq \norm{ a- \psi_i(b)}+\norm{\psi_i(b)-\psi_i(b')}+\norm{\psi_i(b')-\psi_i(\varphi b'')}<\varepsilon$$
		and hence $ker(\pi)\subseteq I_x$, which completes the proof.
	\end{proof}
	Next, we want to show that taking the limit of an inductive system commutes with pullbacks: Let $(A_i,\varphi_{i,j})$ be an inductive system of $C_0(X)$-algebras and $f:Y\rightarrow X$ a continuous map. Then we get $C_0(Y)$-linear $*$-homo\-morphisms $f^*\varphi_{i,j}:f^*A_j\rightarrow f^* A_{i}$ by the formula $$(f^*\varphi_{i,j})(\xi)(y)=(\varphi_{i,j})_{f(y)}(\xi(y)).$$
	as in Lemma \ref{Lem:PullbackOfHomomorphisms}. 
	\begin{prop}\label{Prop:LimitsAndPullbacks}
		Let $(A_i,\varphi_{i,j})$ be an inductive system of $C_0(X)$-al\-gebras and $f:Y\rightarrow X$ a continuous map.
		Then $(f^*A_i,f^*\varphi_{i,j})$ is an inductive system of $C_0(Y)$-algebras and $f^*(\lim_i A_i)$ is $C_0(Y)$-linearly isomorphic to $\lim_i f^*(A_i)$.
	\end{prop}
	\begin{proof}
		Let $A=\lim_i A_i$ and $\psi_i:A_i\rightarrow A$ be the canonical $\ast$-homo\-morphisms. Then by Lemma \ref{Lem:PullbackOfHomomorphisms} we obtain $C_0(Y)$-linear $*$-homo\-morphisms $f^*\psi_i:f^*A_i\rightarrow f^*A$ such that $f^*\psi_{i}\circ f^*\varphi_{i,j}=f^*(\psi_{i}\circ \varphi_{i,j})=f^*\psi_j$. Using the universal property of the limit, we obtain a $C_0(Y)$-linear $*$-homo\-morphism $$\Psi:\lim\limits_{i}f^*A_i\rightarrow f^*A.$$
		To show that it is an isomorphism, it is enough to check that $\Psi_y$ is an isomorphism for all $y\in Y$. But under the identifications
		$$(\lim_{i}f^*A_i)_y\cong \lim_{i}(A_i)_{f(y)}\text{ and }(f^*A)_y\cong A_{f(y)}$$ the map $\Psi_y$ coincides with the isomorphism 
		$$\lim\limits_{i}(A_i)_{f(y)}\rightarrow A_{f(y)}$$
		from the previous Lemma.
	\end{proof}
	
	Suppose now that $(A_i,\varphi_{i,j})$ is an inductive system of $G$-alge\-bras, such that all the connecting homomorphisms are $G$-equivariant. We have already seen in Lemma \ref{Lem:InductiveLimitsOfC_0(X)-algebras}, that $A=\lim_i A_i$ is a $C_0(G^{(0)})$-algebra in a canonical way, such that all the homo\-morphisms $\psi_i:A_i\rightarrow A$ are $C_0(G^{(0)})$-linear. The following Proposition shows how we can use the $G$-actions at each stage of the sequence to obtain a $G$-action on the limit.
	\begin{prop}
		Let $(A_i,\varphi_{i,j})$ be an inductive system of $G$-algebras, such that $\varphi_{i,j}$ is $G$-equivariant for all $i,j\in I$ with $i\geq j$. Let $A:=\lim_i A_i$ and $\psi_i:A_i\rightarrow A$ be the canonical maps. Then there exists a canonical $G$-action on $A$, such that $\psi_i$ is $G$-equivariant for all $i\in I$.
	\end{prop}
	\begin{proof}
		For each $i\in I$ let $\alpha_i:d^*A_i\rightarrow r^*A_i$ denote the $C_0(G)$-linear isomorphism implementing the action of $G$ on $A_i$. Since $\varphi_{i,j}$ is $G$-equivariant for all $i,j\in I$ with $i\geq j$ we have commutative diagrams
		\begin{center}
			\begin{tikzpicture}[description/.style={fill=white,inner sep=2pt}]
			\matrix (m) [matrix of math nodes, row sep=3em,
			column sep=2.5em, text height=1.5ex, text depth=0.25ex]
			{ d^*A_j &  r^*A_j\\
				d^*A_{i} &  r^*A_{i}\\ };
			\path[->,font=\scriptsize]
			(m-1-1) edge node[auto] {$ \alpha_j $} (m-1-2)
			(m-2-1) edge node[auto] {$ \alpha_{i} $} (m-2-2)
			(m-1-1) edge node[auto] { $ d^*\varphi_{i,j} $ } (m-2-1)
			(m-1-2) edge node[auto] {$ r^*\varphi_{i,j} $} (m-2-2)
			;
			\end{tikzpicture}
		\end{center}
		By the universal property, we obtain a $C_0(G)$-linear $*$-isomorphism between the respective limits. Combining this with Proposition \ref{Prop:LimitsAndPullbacks} we obtain a $C_0(G)$-linear $*$-isomorphism
		$$\alpha:d^*A\rightarrow r^*A.$$
		As each $\alpha_i$ is compatible with the multiplication in $G$, so is the limit homomorphism $\alpha$.
	\end{proof}