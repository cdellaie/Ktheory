\section*{Remerciements}

Je voudrais remercier mon directeur de thèse, Hervé Oyono-Oyono, de m'avoir proposé ce sujet de thèse et d'avoir accepté de m'encadrer pendant ces trois années. C'est avec plaisir que je lui exprime ma gratitude. La générosité et la disponibilité dont il a fait preuve à mon égard m'ont permis d'apprendre plus de mathématiques que ce que j'espérais. Il a su me guider avec patience, tout en me laissant assez de liberté pour m'épanouir dans mon travail, sans oublier les nombreuses destinations vers lesquelles il m'a envoyé travailler. \\

Je voudrais aussi remercier mon co-directeur de thèse, Andrzej Zuk, pour son soutient, et son enseignement que j'ai suivi dès le master. J'ai pu régulièrement profiter de discussions très originales sur des sujets atypiques.\\

Je voudrais remercier Jacek Brodzki et Guoliang Yu d'avoir rapporté et relu ma thèse avec attention. \\

Je tiens à exprimer ma reconnaisance à Emmanuel Germain, Jean Renault, d'avoir accepté d'être membres du jury.\\   

Un grand merci à l'équipe du septième étage de Sophie Germain, d'Orsay et aux participants du groupe de travail sur la $KK$-théorie, pour leurs conseils et discussions : Pierre Fima, Stéphane Vassout, Maria Paola Gomez Aparicio. C'est à Paris 7 que j'ai rencontré Vito Zenobi, qui m'a beaucoup aidé lors de ma première année \\

Une autre équipe a compté dans lors de ma formation à la recherche. Je voudrais remercier Siegfried Echterhoff et son étudiant Christian Bönicke, pour les différentes invitations à Münster et leur accueil chaleureux.\\

C'est avec regret que je vais quitter le laboratoire de l'IECL. Je voudrais remercier Victor Nistor, Nicolas Prudhon et Salah Mehdi. Leur bonne humeur et leur disponibilité m'ont été d'un grand secours. \\

Passons aux remerciements plus personnels. Merci à Matthieu Brachet et Benjamin Alvarez. La thèse aurait été beaucoup plus rude sans la légereté et l'humour qu'ils ont apporté au laboratoire. Je voudrais remercier mon ami de longue date, Pierre Cagne, qui a eu un rôle plus que déterminant dans mon orientation. Sa passion pour les mathématiques et les sciences en général m'a profondément influencé. \\

Enfin, un grand merci à Ariane Constantin. Ses efforts resteront à jamais méconnus. S'il m'a fallu de la patience pour finir cette thèse, ce n'est rien comparé aux longues heures qu'elle a passé à m'écouter en parler.   


