\section*{Remerciements}

Je voudrais remercier mon directeur de thèse, Hervé Oyono-Oyono, de m'avoir proposé ce sujet de thèse et d'avoir accepté de m'encadrer pendant ces trois années. C'est avec plaisir que je lui exprime ma gratitude. La générosité et la disponibilité dont il a fait preuve à mon égard m'ont permis d'apprendre beaucoup de mathématiques.\\% Il a su me guider avec patience, tout en me laissant assez de liberté pour m'épanouir dans mon travail, sans oublier les nombreuses destinations vers lesquelles il m'a envoyé travailler. \\

Je voudrais aussi remercier mon co-directeur de thèse, Andrzej Zuk, pour son soutien, et son enseignement que j'ai suivi dès le master. J'ai régulièrement eu des discussions très enrichissantes avec lui.\\

Je remercie Jacek Brodzki et Guoliang Yu d'avoir accepté de rapporter cette thèse, ainsi que les autres membres du jury, Paulo Carillo-Rouse, Emmanuel Germain, Maria-Paula Gomez-Aparicio, Jean Renault, et Jean Louis-Tu.\\   

Un grand merci à l'équipe du septième étage de Sophie Germain, d'Orsay et aux participants du groupe de travail sur la $KK$-théorie, pour leurs conseils et discussions, en particulier Pierre Fima, Maria Paula Gomez Aparicio, et Stéphane Vassout. J'ai pu aussi apprendre beaucoup au contact des doctorants, Kevin Boucher, Aurélien Sagnier et Vito Zenobi. \\

Je voudrais remercier Siegfried Echterhoff et son étudiant Christian Bönicke, pour les différentes invitations à Münster et leur accueil chaleureux.\\

C'est avec regret que je vais quitter le laboratoire de l'IECL. Je voudrais remercier Victor Nistor, Nicolas Prudhon et Salah Mehdi. Leur bonne humeur et leur disponibilité m'ont été d'un grand secours. Merci à Matthieu Brachet et Benjamin Alvarez. La thèse aurait été beaucoup plus rude sans la légereté et l'humour qu'ils ont apporté au laboratoire. \\

Je voudrais remercier mon ami de longue date, Pierre Cagne, qui m'a profondément influencé. Discuter avec lui de mathématiques et de sciences en général se révèle toujours un plaisir. Un grand merci à Rubén Martos et à son enthousiasme constant depuis maintenant quatre années. \\

Enfin, un grand merci à Ariane Constantin. Ses efforts resteront à jamais méconnus. S'il m'a fallu de la patience pour finir cette thèse, ce n'est rien comparé aux longues heures qu'elle a passé à m'écouter en parler.\\   


