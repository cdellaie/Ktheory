%\section{The Novikov conjecture for groups with finite decomposition complexity (maybe)}

During the past years, there has been a growing interest on the links between several conjectures involving assembly maps. This report will focus on the link between the coarse Baum-Connes conjecture and the Novikov conjecture. If $\Gamma$ is a finitely generated group, the descent principle assures that if the coarse Baum-Connes assembly map for $\Gamma$ as a metric space with the word length is an isomorphism, then the Baum-Connes assembly map for $\Gamma$ is rationnaly injective, thus the Novikov conjecture holds for $\Gamma$.\\

Following ideas of M. Gromov, G. Yu introduced new coarse concepts in the study of these assembly maps. He was able to prove the coarse Baum-Connes conjecture for proper metric spaces with finite asymptotic dimension \cite{Yu1}, which is a coarse analogue of the topological covering dimension. Later on, in a paper with Guenter and R. Tessera \cite{GTY}, they defined decomposition complexity for metric spaces, which is a broad generalization of asymptotic dimension. In particular, proper metric spaces with finite asymptotic dimension are of finite decomposition complexity. At the end of \cite{GTY}, as concluding remarks, the authors point out that one should be able to derive a new proof of the coarse Baum-Connes conjecture for spaces with finite decomposition complexity. We should emphasize that this is already known : a space which is finitely decomposable has property (A), hence verifies the coarse Baum-Connes conjecture by the work of G. Yu. \cite{Yu2} But the techniques of this proof is highly analytical, it uses a Dirac-Dual Dirac type construction, which involves infinite dimensional analysis. The suggestion of \cite{GTY} is to give a geometrical proof, using a coarse Mayer-Vietoris argument in the spirit of the proof of the Baum-Connes conjecture for spaces with finite asymptotic dimension.\\

Such a proof was given in the setting of algebraic $K$-theory in a paper of D. A. Ramras, R. Tessera and G. Yu where they established the integral Novikov conjecture for algebraic $K$-theory of group rings $R[\Gamma]$ when the group $\Gamma$ has FDC (finite decomposition complexity). Their proof uses the continuously controlled algebraic $K$-theory groups very intensively : their key lemma is a vanishing theorem of these groups. In a series of papers \cite{OY2}\cite{OY3}, H. Oyono-Oyono and G. Yu developed an analogue of this controlled $K$-theory for operator algebras, which they named quantitative $K$-theory. It consists of a family of groups $\hat K (A) = (K^{\epsilon,r}(A))$ for $r\geq 0,\epsilon \in (0,\frac{1}{4})$ and $A$ a filtered $C^*$-algebra, which we shall describe later. They were able to define quantitative assembly maps that factorize the usual ones, and to give equivalence between isomorphisms of the assembly map and quantitative statements.\\

Following the route of these articles \cite{OY2}\cite{OY3}, we will define quantitative assembly maps for étale groupoids with a proper length. These assembly maps are equivalent to the coarse quantitative assembly maps for proper metric spaces $X$ defined in \cite{OY3} if one takes $G= G(X)$, the coarse groupoid of $X$. We give also quantitative statements equivalent to a certain isomorphism. \textbf{(rerédiger ce paragraphe de façon plus précise une fois les résultats écrits)}\\









































