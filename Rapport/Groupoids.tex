\section{Preliminaries on topological groupoids}

\begin{definition}
A groupoid is the data of 
\begin{itemize}
\item[$\bullet$] two topological spaces $G$ and $G^{(0)}$, called respectively the space of arrows, and the unit space,
\item[$\bullet$] source and range maps $r,s : G\rightarrow G^{(0)}$ which are continuous and surjective,
\item[$\bullet$] a continuous multiplication map $m : G^{(2)}=G\times_{s,r}G\rightarrow G; (g,g')\mapsto gg'$ such that $(gg')g''=g(g'g'')$ for all triple $g,g',g''$ that are composable,
\item[$\bullet$] unit and inverse map $e : G^{(0)}\rightarrow G; x\mapsto e_x$ and $\text{inv} : G\rightarrow G: g\mapsto g^{-1}$ which are continous and such that $\forall g\in G, gg^{-1}=e_{s(g)} \text{ and }g^{-1}g=e_{r(g)}$.
\end{itemize} 
\end{definition}

\begin{definition}
Let $G$ be a locally compact groupoid.\\
$G$ is said to be étale if the range map $r:G\rightarrow G^{(0)}$ is a local homeomorphism.\\
$G$ is said to be principal if $r\times s : G\rightarrow G^{(0)}\times G^{(0)}$ is injective, transitive if $r\times s $ is surjective, and proper if $r\times s $ is a proper map. 
\end{definition}

If $G$ is étale, then any $g\in G$ has an open neighborhood $U$ such that $r$ and $s$ are injective on $U$. The element $g$ defines in this way a partial homeomorphism $\alpha_U = r\circ (s_{|U})^{-1} : s(U)\rightarrow r(U)$. Such open sets $U$ where $r$ and $s$ are injective are called bisections. The set of bisections is naturally a invers semi-group.\\

For $U,V\subset G^{(0)}$, we will us the notations $G_U=s^{-1}(U)$, $G^V=r^{-1}(V)$ and $G_U^V=s^{-1}(U)\cap r^{-1}(V)$. If one of the subsets is reduced to a single point, we will use $G_x=G_{\{x\}}$, etc.

\subsection{Equivariant $KK$-theory}

In \cite{LeGall}, Le Gall defined, for any locally compact groupoid $G$, and $G$-algebras $A$ and $B$, $\Z_2$-graded abelian groups $KK^G(A,B)$ generalizing Kasparov's equivariant $KK$-theory for a locally compact group. Elements of $KK^G(A,B)$ are defined as equivalence classes of triples $(E,\pi,T)$ that we will call $K$-cycles, where :
\begin{itemize}
\item[$\bullet$] $E$ is a right $\Z_2$-graded $B$-Hilbert module, equipped with a unitary action of $G$, i.e. an even unitary $V\in\mathcal L_B(s^*E,r^*E)$ such that $V_g V_{g'} = V_{gg'}$ for all $(g,g')\in G^{(2)}$,
\item[$\bullet$] $\pi : A\rightarrow \mathcal L_B(E)$ is a $*$-homomorphism with image containe in even operators, and we will often write $a$ instead of $\pi(a)$,
\item[$\bullet$] $T\in\mathcal L_B(E)$ is an odd bounded $B$-linear operator satifying the $K$-cycle relations i.e. $[a, T]$, $a(T-T^*)$, $a(T^2-1)$, $a(T_{r(g)}-V_gT_{s(g)}V_g^*)$ are compact operators in $\mathfrak K_B(E)$.
\end{itemize} 

The set of such $K$-cycles is denoted by $\mathbb E^G(A,B)$.

\begin{definition}
Two $K$-cycles $(E,\pi,T)$ and $ (E',\pi',T')$ in $\mathbb E^G(A,B)$ are homotopic if there exists $ (\tilde E,\tilde \pi,\tilde T) \in\mathbb E^G(A,B[0,1])$ such that $(\tilde E_0,\tilde \pi_0,\tilde T_0)\simeq (E,\pi,T)$ and $(\tilde E_1,\tilde \pi_1,\tilde T_1)\simeq (E',\pi',T')$.
\end{definition}

We will use the following result.

\begin{prop}
Let $z\in KK^G_1(A,B)$. The induced homomorphism $-\otimes_A z : K_*(A)\rightarrow K_*(B)$ can be realized, up to Morita equivalence, as the boundary of the following semi-split $G$-equivariant extension of $G$-algebras
\[\begin{tikzcd}[column sep = small] 
0 \arrow{r}& \mathfrak K \otimes B \arrow{r} & E^{\pi,T} \arrow{r} & A \arrow{r} & 0\end{tikzcd}\]
where $E^{\pi,T}=\{(a,P\pi(a)P +y) : a\in A , y\in \mathfrak K \otimes B\}$, the arrows are the obvious inclusion and projection, and the completely positive section is $a\mapsto (a, P \pi(a) P)$. Here, we implicitely choose a $K$-cycle $(H_B,\pi,T)$ representing $z$, and $P=\frac{1+T}{2}$. The fact that this choice does not affect the boundary will be proven in the upcoming sections. 
\end{prop}









































