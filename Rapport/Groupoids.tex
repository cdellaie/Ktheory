\section{Preliminaries on topological groupoids}

\begin{definition}
A groupoid is the data of 
\begin{itemize}
\item[$\bullet$] two topological spaces $G$ and $G^{(0)}$, called respectively the space of arrows, and the unit space,
\item[$\bullet$] a topological embedding $e: G^{(0)}\rightarrow G; x\mapsto e_x$ called the unit map,
\item[$\bullet$] source and range maps $r,s : G\rightarrow G^{(0)}$ which are continuous, surjective, and satisfy $r\circ e = s\circ e = Id_{G^{(0)}}$,
\item[$\bullet$] a continuous multiplication map $m : G^{(2)}=G\times_{s,r}G\rightarrow G; (g,g')\mapsto gg'$ such that $(gg')g''=g(g'g'')$ for all triple $g,g',g''$ that are composable,
\item[$\bullet$] an inverse map $\text{inv} : G\rightarrow G: g\mapsto g^{-1}$ which is continous and such that $\forall g\in G, gg^{-1}=e_{s(g)} \text{ and }g^{-1}g=e_{r(g)}$, and $g e_{s(g)} = e_{r(g)} g =g$.
\end{itemize} 
\end{definition}

\begin{definition}
Let $G$ be a locally compact groupoid.\\
$G$ is said to be étale if the range map $r:G\rightarrow G^{(0)}$ is a local homeomorphism.\\
$G$ is said to be principal if $r\times s : G\rightarrow G^{(0)}\times G^{(0)}$ is injective, transitive if $r\times s $ is surjective, and proper if $r\times s $ is a proper map. 
\end{definition}

If $G$ is étale, then any $g\in G$ has an open neighborhood $U$ such that $r$ and $s$ are injective on $U$. The element $g$ defines in this way a partial homeomorphism $\alpha_U = r\circ (s_{|U})^{-1} : s(U)\rightarrow r(U)$. Such open sets $U$ where $r$ and $s$ are injective are called bisections. The set of bisections is naturally a invers semi-group.\\

For $U,V\subset G^{(0)}$, we will us the notations $G_U=s^{-1}(U)$, $G^V=r^{-1}(V)$ and $G_U^V=s^{-1}(U)\cap r^{-1}(V)$. If one of the subsets is reduced to a single point, we will use $G_x=G_{\{x\}}$, etc. \\

A Haar system on $G$ is a family of borelian measures $(\lambda^x)_{x\in G^{(0)}}$ such that 
\begin{itemize}
\item[$\bullet$] the support of $\lambda^x$ is $G^x$, 
\item[$\bullet$] the map $G^{(0)} \rightarrow \C; x\mapsto \int_{G^x} f d\lambda^x$ is continuous for all $f\in C_c(G)$,
\item[$\bullet$] for all $f\in C_c(G)$ and $g\in G$, $\int_{G^{r(g)}} f(h) d\lambda^{r(g)}(dh) \int_{G^{s(g)}} f(gh) \lambda^{s(g)}(dh)$ holds.
\end{itemize}

\subsection{Actions of groupoids}

\begin{definition}
An left action of $G$ on a topological space $Z$ is given by a continuous map $p : Z \rightarrow G^{(0)}$, called the anchor map, and a map $\alpha : G\times_{s,p} Z \rightarrow Z $ such that $\alpha(g',\alpha(g,z)) = \alpha(g'g,z)$ whenever $(g,g')\in G^{(2)}$ and $p(z)=s(g)$. We will use the notation $\alpha(g,z) = g.z$ when the action is clear from the context.\\

An right action of $G$ on a topological space $Z$ is given by a continuous map $p : Z \rightarrow G^{(0)}$, again called the anchor map, and a map $\alpha : Z\times_{p,r} G \rightarrow Z $ such that $\alpha(\alpha(z,g),g') = \alpha(z, gg')$ whenever $(g,g')\in G^{(2)}$ and $p(z)=r(g)$. We will use the notation $\alpha(g,z) = z.g$ when the action is clear from the context.\\ 
\end{definition}

An action $(Z,p,\alpha)$ is said to be proper if $\alpha\times id_Z : $ is proper as a continous map.\\

\subsection{$G$-algebras}

\begin{definition}
Let $X$ be a topological space. A $X$-algebra is a $C^*$-algebra together with a $*$-homomorphism $\theta : C_0(X)\rightarrow \mathcal M(Z(A))$ such that $\theta (C_0(X)) A = A$. The couple $(A,\theta)$ is then called a $X$-structure.
\end{definition} 

If $f : X\rightarrow Y$ is a continuous map, and $A$ a $Y$-algebra, then $f$ induces a $X$ structure on $A$, and we define $f^* A = A \otimes_f C_0(X)$ the resulting $X$-algebra. For details on tensor products of $X$-algebras, see \cite{LeGall} for instance. As a particular case, one gets the notion of a fiber of a $X$-algebra : for $x\in X$, $A_x = ev_x^* A$ is called the fiber over $x$, where $ev_x$ is the inclusion of $x$ in $X$.\\

A $*$-homomorphism between two $X$-algebras $\alpha : A\rightarrow A'$ is a $*$-homomorphism which commutes with the action of $C_0(X)$, i.e. $\alpha(\theta(f)a)=\theta'(f)\alpha(a)$ for all $a\in A,f\in C_0(X)$. Its fibers are defined unambiguously by $\alpha_x = id_\C \otimes_{ev_x} \alpha$.\\ 

\begin{definition}
An action of a groupoid $G$ on a $C^*$-algebra $A$ is a triple $(A,\theta,\alpha)$ where $(A,\theta)$ is a $G^{(0)}$-structure and $\alpha : s^* A \rightarrow r^* A$ is a morphism of $G^{(0)}$-algebras such that $\alpha_{gg'}= \alpha_g\circ \alpha_{g'}$.
\end{definition} 

\subsection{Crossed products}

If $A$ is a $G$-algebra, and $U \subset G^{(0)}$ is an open subset, recall \cite{LeGall} that the restriction $A_U$ is defined as $\eta^* A = C_0(U) \otimes_\eta A$ where $\eta : C(U) \hookrightarrow C_0(G^{(0)})$ is the canonical inclusion. Let $F= G^{(0)}\backslash U$. As $A_U$ is naturally embedded into $A$, define $A_F$ to be the complementary of $A_U$. \\

Let $E$ be a compact subset of $G$. The space of continuous sections with support in $E$ is defined as $A_E$. Then 
\[C_c(G,A) = \cup_{E\in \mathcal E} A_E\]
is called the space of continuous sections with compact support. It is naturally a $*$-algebra with the convolution product
\[\phi\ast \psi (g) = \int_{G^{r(g)}} \phi(h)\psi(h^{-1}g)\lambda^{r(g)}(h).\] 

If $f\in C_c(G,A)$, define $\lambda(f)\in \mathcal L(L^2(G,A))$ as the convolution operator 
\[\lambda(f)\eta = f\ast \eta\]
and the reduced norm as $||f||_r= ||\lambda(f)||_{\mathcal L(L^2(G,A))}$.
\begin{definition}
The reduced crosssed product $A\rtimes_r G$ is defined as the completion of $C_c(G,A)$ under the reduced norm.
\end{definition}

\subsection{Equivariant $KK$-theory}

In \cite{LeGall}, Le Gall defined, for any locally compact groupoid $G$, and $G$-algebras $A$ and $B$, $\Z_2$-graded abelian groups $KK^G(A,B)$ generalizing Kasparov's equivariant $KK$-theory for a locally compact group. Elements of $KK^G(A,B)$ are defined as equivalence classes of triples $(E,\pi,T)$ that we will call $K$-cycles, where :
\begin{itemize}
\item[$\bullet$] $E$ is a right $\Z_2$-graded $B$-Hilbert module, equipped with a unitary action of $G$, i.e. an even unitary $V\in\mathcal L_B(s^*E,r^*E)$ such that $V_g V_{g'} = V_{gg'}$ for all $(g,g')\in G^{(2)}$,
\item[$\bullet$] $\pi : A\rightarrow \mathcal L_B(E)$ is a $*$-homomorphism with image containe in even operators, and we will often write $a$ instead of $\pi(a)$,
\item[$\bullet$] $T\in\mathcal L_B(E)$ is an odd bounded $B$-linear operator satifying the $K$-cycle relations i.e. $[a, T]$, $a(T-T^*)$, $a(T^2-1)$, $a(T_{r(g)}-V_gT_{s(g)}V_g^*)$ are compact operators in $\mathfrak K_B(E)$.
\end{itemize} 

The set of such $K$-cycles is denoted by $\mathbb E^G(A,B)$.

\begin{definition}
Two $K$-cycles $(E,\pi,T)$ and $ (E',\pi',T')$ in $\mathbb E^G(A,B)$ are homotopic if there exists $ (\tilde E,\tilde \pi,\tilde T) \in\mathbb E^G(A,B[0,1])$ such that $(\tilde E_0,\tilde \pi_0,\tilde T_0)\simeq (E,\pi,T)$ and $(\tilde E_1,\tilde \pi_1,\tilde T_1)\simeq (E',\pi',T')$.
\end{definition}

We will use the following result.

\begin{prop}
Let $z\in KK^G_1(A,B)$. The induced homomorphism $-\otimes_A z : K_*(A)\rightarrow K_*(B)$ can be realized, up to Morita equivalence, as the boundary of the following semi-split $G$-equivariant extension of $G$-algebras
\[\begin{tikzcd}[column sep = small] 
0 \arrow{r}& \mathfrak K \otimes B \arrow{r} & E^{\pi,T} \arrow{r} & A \arrow{r} & 0\end{tikzcd}\]
where $E^{\pi,T}=\{(a,P\pi(a)P +y) : a\in A , y\in \mathfrak K \otimes B\}$, the arrows are the obvious inclusion and projection, and the completely positive section is $a\mapsto (a, P \pi(a) P)$. Here, we implicitely choose a $K$-cycle $(H_B,\pi,T)$ representing $z$, and $P=\frac{1+T}{2}$. The fact that this choice does not affect the boundary will be proven in the upcoming sections. 
\end{prop}

The following result will be constantly used in the upcoming sections. In short, it explains how to construct the boundary map associated to any extension $0\rightarrow J\rightarrow A \rightarrow A/J \rightarrow 0$ as the Kasparov product of an element $[\partial_{J,A}]\in KK^G_1(A/J,J)$. For references, see \cite{blackadar}.\\

Recall that, for any $G$-equivariant $*$-homomorphism $\phi : A \rightarrow B$, we can define its cone denoted $C_\phi$ by 
\[C_\phi = \{(a,f)\in A\times CB \text{ s.t. } \phi(a)=f(0)\}.\]
If any $G$-algebra $A$ is given, $0\rightarrow SA\rightarrow CA\rightarrow A \rightarrow 0$ is a $G$-equivariant semi-split extension, and defines an element $[\partial_A]=\tau_A([\partial_\C])\in KK^G_1(A,SA)$ which is invertible. Now, give yourself a semi-split extension 
\[\begin{tikzcd}[column sep = small]0 \arrow{r} & J \arrow{r}{\iota} & A \arrow{r}{q} & A/J \arrow{r}& 0 \end{tikzcd}\] 
and define $\alpha : J\rightarrow C_q ; x\mapsto (x,0)$ and the canonical inclusion $\beta : S (A/J)\hookrightarrow C_q$. It turns out that $[\alpha]\in KK_0^G(J,C_q)$ is invertible.

\begin{prop}
The boundary map $\partial_{J,A} : K_*(A/J)\rightarrow K_{*+1}(J)$ is given by the Kasparov product by an element $[\partial_{J,A}]\in KK^G_1(J,A/J)$ which satisfies 
\[[\partial_{J,A}] = [\partial_{A/J}]\otimes_{} \beta^*([\alpha]^{-1}).\]
\end{prop}








































