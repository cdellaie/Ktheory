\section{Discrete and Compact Quantum groups}

We recall in this section basic facts about the theory of compact quantum groups. This theory was developed by Woronowicz in the $1980$'s. The reader can find more details in \cite{Wo}. Our goal is to present reduced cross products by discrete quantum group. We define discrete quantum groups as duals of compact quantum groups. The ultimate goal of this section is to notice that reduced cross products by discrete quantum group are filtered, in a sense to be defined later, by a subset of the set of equivalence classes of unitary representations of the compact quantum group.\\ 

Our presentation is strongly inspired by the article \cite{VerVaes}. We recall that $\otimes$ denotes the minimal or spatial tensor product of $C^*$-algebras. If $x\in A\otimes A$, we will use the notation $a_{ij}$ for the obvious element in $A\otimes A\otimes A$, e.g. $a_{12} = a\otimes 1$. For every Hilbert space $H$ and every $\xi,\eta \in H$, $\omega_{\xi,\eta}$ denotes the linear form on $\mathcal L(H)$ defined by $\omega_{\xi,\eta}(T) = \langle T\eta,\xi\rangle$.

\begin{definition}
A compact quantum group is a pair $\mathbb G = (C(\mathbb G),\Delta)$ where $C(\mathbb G)$ is a unital $C^*$-algebra and $\Delta : C(\mathbb G) \rightarrow C(\mathbb G)\otimes C(\mathbb G)$ is a unital $*$-homomorphism satisfying :
\begin{itemize}
\item[$\bullet$] the coassociativity condition, i.e. $(\Delta \otimes id)\Delta=(id\otimes \Delta )\Delta$ ;
\item[$\bullet$] the left and right cancellation property, i.e. the linear spans of $\Delta(C(\mathbb G))(1\otimes C(\mathbb G))$ and $\Delta(C(\mathbb G))(C(\mathbb G)\otimes 1)$ are dense in $C(\mathbb G)\otimes C(\mathbb G)$.
\end{itemize} 
\end{definition}

\begin{Expl} Let $G$ be a compact group, and define $\Delta : C(G)\rightarrow C(G)\otimes C(G)$ by $(\Delta f)(g,g') = f(gg')$, where $C(G)\otimes C(G)$ is identified with $C(G\times G)$. Then $(C(G),\Delta ) $ is a compact quantum group.
\end{Expl}

The notation $C(\mathbb G)$ was adopted to suggest an analogy with this example. In general though, the $C^*$-algebra is noncommutative, and there is no topological space underlying the quantum group. Still, one often refer to $C(\mathbb G)$ as "the algebra of continous functions" on $\mathbb G$.

\begin{Expl}
Let $\Gamma$ be a discrete group and define the reduced $C^*$-algebra $C_r^*(\Gamma)$ of $\Gamma$. Recall that it is the $C^*$-algebra generated in $\mathcal L(l^2(\Gamma))$ by the left translation operators $\lambda_\gamma$, for every $\gamma\in\Gamma$. The map $ \lambda_\gamma \mapsto \lambda_\gamma\otimes \lambda_\gamma$ extends to a unital $*$-homomorphism $\Delta :C_r^*(\Gamma)\rightarrow C_r^*(\Gamma)\otimes C_r^*(\Gamma) $. Then $(C_r^*(\Gamma),\Delta)$ is a compact quantum group. It can be seen as a candidate for what should be the continuous functions on the reduced dual $\hat\Gamma$. By Fourier transform, it is exactly the case when $\Gamma$ is commutative.
\end{Expl}

\begin{definition} A unitary representation $(H,U)$ of $\mathbb G$ on a Hilbert space $H$ is a unitary element $U \in \mathcal M(\mathfrak K(H)\otimes C(\mathbb G))$ such that $(id\otimes\Delta)(U) = U_{12}U_{13}$. Let $(H,U)$ and $(H',U')$ be two unitary representations of $\mathbb G$. The space of intertwiners between $U$ and $U'$ is defined as 
\[Hom(U,U') = \{T\in \mathcal L(H',H) \text{ s.t. }U(T\otimes 1) = (T\otimes 1) U' \}.\]
We denote by $End(U)$ the intertwiners of $Hom(U,U)$. A unitary representation is called irreducible if $End(U) = \C 1$. The trivial representation is denoted by $\varepsilon$. If $(H,U)$ is a unitary representation of $\mathbb G$, and $\eta,\xi\in H$, then $U_{\xi,\eta} = (\omega_{\xi,\eta}\otimes 1)( U )$ is called a coefficient of $U$.
\end{definition}

\begin{Expl} Let $\Gamma$ be a discrete group and $\mathbb G = (C_r^*(\Gamma),\Delta)$ be the corresponding compact quantum group. Then a finite dimensional unitary representation $U\in \mathcal M(\mathfrak K(H)\otimes C(\mathbb G))\cong\mathfrak M_n(C^*_r(\Gamma))$ of $\mathbb G$ is given by a family of orthogonal projections $P_\gamma\in \mathfrak M_n(\C)$ for every $\gamma\in \Gamma$ such that 
\[P_{\gamma}P_{\gamma'} = \delta_{\gamma,\gamma'} P_{\gamma}\quad\text{ and }\quad \sum_{\gamma\in\Gamma} P_{\gamma} = I_n.\]
\end{Expl}

We define sum and tensor product of representations as follows. Let $(H,U)$ and $(H',U')$ be two unitary representations. Then the sum is defined as $(H\oplus H', U\oplus U')$. The tensor product is defined as $(H\otimes H',W)$ where $W = U_{13} U'_{23}\in \mathcal M(\mathfrak K(H\otimes H')\otimes C(\mathbb G))$. 

\begin{definition}
As in the case of compact groups, we define an equivalence relation on unitary representations of $\mathbb G$ by 
\[(H,U)\sim (H',U') \text{ iff } Hom(U,U') \text{ contains a non trivial unitary operator}.\]
The set of equivalence classes of irreducible unitary representations is denoted by $Irrep(\mathbb G)$. We choose for each $\pi\in Irrep (\mathbb G)$ a unitary representation $(H_\pi, U^{\pi})$. 
\end{definition}

The following theorem is crucial, and is a generalization of the classical case. 

\begin{thm}[\cite{VerVaes},  section $1$] Every irreducible representation of a compact quantum group is finite dimensional. Every unitary representation of a compact quantum group is equivalent to a direct sum of irreducible unitary representations. 
\end{thm}
 
We also have the analog of the Haar measure, and of the regular representations.

\begin{prop}[\cite{VerVaes}, section $1$]
For every compact quantum group $\mathbb G$, there exists a unique state $h$ on $C(\mathbb G)$, called the Haar state on $\mathbb G$, satisfying $(id\otimes h)\Delta(a) = h(a)1 = (h\otimes id)\Delta(a)$ for every $a\in C(\mathbb G)$. 
\end{prop}

Let us denote by $L^2(\mathbb G)$ the Hilbert space obtained as the GNS construction with respect to the Haar state, i.e. $L^2(\mathbb G) = L^2(C(\mathbb G),h)$. We denote by $\xi_h\in L^2(\mathbb G)$ and $\pi_h : C(\mathbb G)\rightarrow \mathcal L(L^2(\mathbb G))$ the corresponding cyclic vector and representation. According to \cite{Wo}, there exists a unique unitary $W\in \mathcal L(L^2(\mathbb G) \otimes L^2(\mathbb G) )$ satisfying 
\[W^*(\xi\otimes \pi_h(a)\xi_h ) = (\pi_h\otimes\pi_h)\circ \Delta(a)(\xi\otimes \xi_h)\quad \forall \xi\in L^2(\mathbb G),a\in C(\mathbb G).\] 

\begin{definition}
The reduced form $C_r^*(\mathbb G)$ of $\mathbb G$ is defined as the $C^*$-algebra $\pi_h(C(\mathbb G))$. The unitary $W$ defines an element of $W\in\mathcal M(C_r^*(\mathbb G)\otimes \mathfrak K(L^2(\mathbb G))$ which is a unitary representation  called the regular representation of $\mathbb G$.
\end{definition}

According to \cite{Wo} (section $6$), for every $\pi\in Irrep(\mathbb G)$, there exists a unique $\pi^*\in Irrep(\mathbb G)$ such that $Hom(\pi\otimes \pi^*,\varepsilon)\neq 0 \neq Hom(\varepsilon,\pi\otimes \pi^*) $. For a representative $(H_\pi,U^\pi)$, $(H_{\pi^*},U^{\pi^*})$ is called the contragredient representation.\\
 %Define :
%\[L^2(\mathbb G) = \bigoplus_{\pi\in Irrep(\mathbb G)} H_\pi \otimes H_{\pi^*}. \]
%Then the right regular representation is defined by
%\[\rho\left\{\begin{array}{rcl}
%C(\mathbb G) & \rightarrow & \mathcal L(L^2(\mathbb G)) \\
%??	& \mapsto & ??\\
%\end{array}\right.\]

We now turn to discrete quantum group. We will define it as dual of a compact quantum group. An intrinsic definition (without reference to a given compact quantum group) exists and can be found in \cite{vandaele}.\\

Define the following $C^*$-algebras :
\[\begin{array}{c}
c_0(\hat{\mathbb G})= \bigoplus_{\pi\in Irred(\mathbb G)} \mathcal L(H_\pi),\\
l^\infty(\hat{\mathbb G})= \prod_{\pi\in Irred(\mathbb G)} \mathcal L(H_\pi).
\end{array}\]
Up to our choice of representations $\{(H_\pi,U^\pi)\}_{\pi\in Irrep(\mathbb G)}$, we can define a unitary $U = \bigoplus_{\pi\in Irrep(\mathbb G)} U^\pi\in \mathcal M(c_0(\hat{\mathbb G}) \otimes C(\mathbb G))$ which is a unitary representation of $\mathbb G$. 

\begin{prop}[\cite{PodlesWo}, Theorem $2.1$] There exists a unique $*$-homomorphism $\hat \Delta : c_0(\hat{\mathbb G}) \rightarrow \mathcal M(c_0(\hat{\mathbb G})\otimes c_0(\hat{\mathbb G}))$ such that \[(\hat\Delta\otimes id ) U = U_{13}U_{23}.\]
Moreover, $\hat\Delta$ is coassociative, i.e. $(\hat\Delta \otimes id) \hat\Delta = (id\otimes \hat\Delta) \hat\Delta$.
%for every $\pi\in Irrep(\mathbb G)$ and every $T\in\mathcal L(H_\pi)$,
%\[\hat\Delta(T)\circ \phi = \phi \circ T \quad \forall y,z\in Irrep(\mathbb G), \forall \phi \in Hom(\pi,y\otimes z).\]
\end{prop}
%This allows to define a unital $*$-homomorphism $\hat \Delta : l^\infty(\hat{\mathbb G}) \rightarrow l^\infty(\hat{\mathbb G})\otimes l^\infty(\hat{\mathbb G})$ by 
%\[\hat\Delta ( x) = U(x \otimes 1)U^*\quad \forall x\in l^\infty(\hat{\mathbb G})).\]
%This $*$-homomorphism satisfies $\hat\Delta(c_0(\hat{\mathbb G}))\subseteq \mathcal M(c_0(\hat{\mathbb G})\otimes c_0(\hat{\mathbb G}))$ and $(\hat\Delta \otimes id) \hat\Delta = (id\otimes \hat\Delta)\hat\Delta$. %\textbf{OTHER CONDITION}.

\begin{definition}
Let $\mathbb G$ be a compact quantum group. With the above notations, its dual is defined as the discrete quantum group $\hat{\mathbb G} = (c_0(\hat{\mathbb G}),\hat\Delta)$.
\end{definition}

\begin{Expl} Let $\Gamma$ be a discrete group and consider the compact quantum group $\mathbb G=(C^*(\Gamma),\Delta)$. Then, $\gamma\mapsto \lambda_\gamma$ gives a bijection $Irrep(\mathbb G)\cong \Gamma$, and $\hat{\mathbb G}$ is $c_0(\Gamma)$ with comultiplication $\hat\Delta (f)(\gamma,\gamma') = f(\gamma\gamma')$ for every $f\in c_0(\Gamma)$ and $\gamma,\gamma'\in\Gamma$.
\end{Expl}

The dual of a compact quantum group is also represented on $L^2(\mathbb G)$ via the regular representations. 
\begin{prop}
For every compact quantum group $\mathbb G$, there exists a unique state $h$ on $c_0(\hat{\mathbb G})$, called the left Haar state on $\hat{\mathbb G}$, satisfying $(id\otimes \hat h)\hat\Delta(a) = \hat h(a)1 $ for every $a\in c_0(\hat{\mathbb G})$. The GNS Hilbert space is canonically isomorphic to $L^2(\mathbb G)$. We will denote by $\hat \pi_h : c_0(\hat{\mathbb G})\rightarrow \mathcal L(L^2(\mathbb G))$ and $\hat \xi_h$ the corresponding representation and cyclic vector respectively.  
\end{prop}

%Let us define the dual left regular representation by :
%\[\hat\lambda\left\{\begin{array}{rcl}
%l^\infty(\hat{\mathbb G}) & \rightarrow & \mathcal L(L^2(\mathbb G)) \\
%a	& \mapsto & \hat\lambda (a) : \xi \mapsto \{(a_\pi\otimes 1)\xi_\pi\}_\pi\\
%\end{array}\right.\]

\begin{definition} An action of a discrete quantum group $\hat{\mathbb G}$ on a $C^*$-algebra $A$ is a nondegenerate $*$-homomorphism $\alpha : A\rightarrow \mathcal M(A\otimes c_0(\hat{\mathbb G}))$ such that :
\begin{itemize}
\item[$\bullet$] $(\alpha \otimes id )\alpha = (id\otimes \hat\Delta) \alpha$
\item[$\bullet$] for every $a\in A$, $(\hat\varepsilon \otimes id)\alpha(a) = a$.
\end{itemize}
Such a pair $(A,\alpha)$ is called a $\hat{\mathbb G}$-algebra.
\end{definition}

Let $\alpha : A\rightarrow \mathcal M(A\otimes c_0(\hat{\mathbb G}))$ be an action of $\hat{\mathbb G}$, and $\pi\in Irrep(\mathbb G)$ a unitary representation of $\mathbb G$. Let $p_\pi\in c_0(\hat{\mathbb G}) $ be the orthogonal projection on $H_\pi$. We define the coefficients of $\alpha$ as the elements $\alpha^\pi_{\xi,\eta}(a) = ( id_A \otimes \omega_{\xi,\eta})(\alpha(a)( id _A\otimes p_\pi)) \in A$ for any $\xi,\eta\in H_\pi$.\\  

Let $(A,\alpha)$ be a $\hat{\mathbb G}$-algebra. Define $W= id_A \otimes (id\otimes\pi_h)(U) \in \mathcal M (A\otimes c_0(\hat{\mathbb G})\otimes C^*_r(\mathbb G))  $. Let $\pi\in Irrep(\mathbb G)$, and $\xi,\eta \in H_\pi$. Then the coefficient 
$W^\pi_{\xi,\eta}:=(id\otimes \omega_{\xi,\eta}\otimes id_{C^*_r(\mathbb G)})(W)$ is an element of $\mathcal M(A\otimes C_r^*(\mathbb G)) \subseteq \mathcal L_A(A\otimes L^2(\mathbb G))$. 

\begin{definition} Let $(A,\alpha)$ be a $\hat{\mathbb G}$-algebra. The reduced crossed product $A\rtimes_r \hat{\mathbb G}$ is the $C^*$-subalgebra of $\mathcal L_A(A\otimes L^2(\mathbb G))$ generated by 
\[\{\ \theta(a)W^\pi_{\xi,\eta}\ , a\in A , \pi\in Irrep(\mathbb G),\xi,\eta\in H_\pi\},\]
where $\theta = ( id \otimes  \hat\pi_h)\alpha$.  
\end{definition}

\begin{rk}
There is also a notion of maximal (or full) crossed product for $\hat{\mathbb G}$-algebras.
\end{rk}

\begin{rk}
As in the classical case, the reduced and the maximal crossed products respect the semi-split exact sequences of $\hat{\mathbb G}$-algebras.
\end{rk}

Denote by $\mathcal E_{\mathbb G}$ the set of finite dimensional unitary representations $\pi$ of $\mathbb G$ such that $\pi$ and $\pi^*$ are unitarily equivalent. We endow $\mathcal E_{\mathbb G}$ with a semigroup structure with the following product 
\[\pi\circ \pi' = (\pi\otimes\pi')\oplus (\pi'\otimes \pi)\quad \forall \pi,\pi'\in \mathcal E_{\mathbb G}.\]
For every $\pi\in \mathcal E_{\mathbb G}$, denote by $C_\pi(\hat{\mathbb G}, A)$ the closure of the linear spans of 
\[\{\ \theta(a) W_{\xi,\eta}^\pi \ ,a\in A, \xi,\eta\in H_\pi\}.\]

\begin{prop} \label{filtrationQG}
For every $\hat{\mathbb G}$-algebra $A$, $A\rtimes_r \hat{\mathbb G}$ is the closure of $\cup_{\pi\in \mathcal E_{\mathbb G}} C_\pi(\hat{\mathbb G}, A)$. Moreover,
\[C_\pi(\hat{\mathbb G}, A).C_{\pi'}(\hat{\mathbb G}, A) \subseteq C_{\pi \circ \pi'}(\hat{\mathbb G}, A).\]
\end{prop}

\begin{dem}
Let $\{\xi_1,...,\xi_{dim\ \pi}\}$ be an orthonormal basis of $H_\pi$ for every $\pi\in\mathcal E_{\mathbb G}$. Then the coefficients $W_{\xi,\eta}^\pi$ are uniquely determined by the coefficients $W_{i,j}^\pi := W_{\xi_i,\xi_j}^\pi$. Moreover, the following relation holds
\[W_{i,j}^\pi \theta(a)= \sum_k \theta(\alpha_{i,k}^\pi(a)) W_{k,j}^\pi \quad \forall a\in A,\]
hence, for $\pi$ and $\pi'$ in $\mathcal E_{\mathbb G}$, and every $a,a'\in A$,
\[\theta(a) W_{i,j}^\pi \theta(a')W_{p,q}^{\pi'} = \sum_k \theta(a \alpha_{i,k}^{\pi}(a')) W_{k,j}^\pi W_{p,q}^{\pi'} 
= \sum_k \theta(a \alpha_{i,k}^{\pi}(a')) W_{kp,jq}^{\pi\otimes \pi'}, \]
where, if $\{\xi_k\}$ and $\{\xi'_p\}$ are orthonormal basis of $H_\pi$ and $H_{\pi'}$ respectively, $W_{kp,jq}^{\pi\otimes \pi'}$ denotes the coefficient $W_{\xi_k\otimes \xi'_p,\xi_j \otimes \xi'_q}^{\pi\otimes \pi'}$.\\
\qed
\end{dem}

As in the case of locally compact groups or locally compact $\sigma$-compact groupoids with Haar system, equivariant $KK$-theory was developped by S. Baaj and G. Skandalis in \cite{BaajSk} and R. Vergnioux in \cite{vergnioux}. Details are beyond the scope of this thesis, and we refer to \cite{vergnioux} for details.\\

\begin{prop}[\cite{vergnioux}]
Let $A$ and $B$ be two $\hat{\mathbb G}$-algebras. Then there exists a $\Z_2$-graded abelian group $KK^{\hat{\mathbb G}}(A,B) $ such that :
\begin{itemize}
\item[$\bullet$] every $\hat{\mathbb G}$-equivariant $*$-homomorphism $\phi : A\rightarrow B$ defines a class $[\phi]\in KK^{\hat{\mathbb G}}(A,B)$,
\item[$\bullet$] $KK^{\hat{\mathbb G}}(A,B)$ is a contravariant functor in the $A$ variable w.r.t. $\hat{\mathbb G}$-equivariant $*$-homomorphisms,
\item[$\bullet$] $KK^{\hat{\mathbb G}}(A,B)$ is a covariant functor in the $B$ variable w.r.t. $\hat{\mathbb G}$-equivariant $*$-homomorphisms,
\item[$\bullet$] for every $\hat{\mathbb G}$-algebra $D$, and if $A$ is separable, there exists a bilinear map called the Kasparov product 
\[\otimes_D : KK^{\hat{\mathbb G}}(A,D)\times KK^{\hat{\mathbb G}}(D,B)  \rightarrow  KK^{\hat{\mathbb G}}(A,B)\]
such that 
\[ [\phi]\otimes_D z' = \phi^*(z') \text{ and } z\otimes_D[\phi'] = \phi'_*(z)\]
for every $z\in KK^{\hat{\mathbb G}}(A,D)$, $z\in KK^{\hat{\mathbb G}}(D,B)$, and every $\hat{\mathbb G}$-equivariant $*$-homomorphisms $\phi : A \rightarrow D$ and $\phi' : D\rightarrow B$.
\end{itemize}
Moreover, one can construct a descent morphism
\[j_{\hat{\mathbb G}} : KK^{\hat{\mathbb G}}(A,B)\rightarrow KK(A\rtimes_r \hat{\mathbb G},B\rtimes_r \hat{\mathbb G})\]
compatible with Kasparov products, i.e. 
\[j_{\hat{\mathbb G}}(z\otimes_D z') = j_{\hat{\mathbb G}}(z)\otimes_{D\rtimes \hat{\mathbb G}} j_{\hat{\mathbb G}}(z')\quad \forall z\in KK^{\hat{\mathbb G}}(A,D),z'\in KK^{\hat{\mathbb G}}(D,B) \]
\end{prop}

Notice that every $KK^{\hat{\mathbb G}}$-element satisfies property $(d)$ for some universal $d>0$ independent of $\mathbb G$ as well (see the definition \ref{DecompositionPropertyD}). Indeed, the proof of \cite{LaffOY} can be carried out without modification in the setting of discrete quantum groups.































