\section{Discrete and Compact Quantum groups}

We recall in this section basic facts about the theory of compact quantum groups. This theory was developed by Woronowicz in the $1980$'s. The reader can find more details in \cite{Wo}. Our goal is to present reduced cross products by discrete quantum group. Our presentation ows a lot to the article \cite{VerVaes}. We recall that $\otimes$ denotes the minimal or spatial tensor product of $C^*$-algebras.

\begin{definition}
A compact quantum group is a pair $\mathbb G = (C(\mathbb G),\Delta)$ where $C(\mathbb G)$ is a unital $C^*$-algebra and $\Delta : C(\mathbb G) \rightarrow C(\mathbb G)\otimes C(\mathbb G)$ is a unital $*$-homomorphism satisfying :
\begin{itemize}
\item[$\bullet$] the coassociativity condition, i.e. $(\Delta \otimes id)\Delta=(id\otimes \Delta )\Delta$ ;
\item[$\bullet$] the left and right cancellation property, i.e. the linear spans of $\Delta(C(\mathbb G))(1\otimes C(\mathbb G))$ and $\Delta(C(\mathbb G))(C(\mathbb G)\otimes C(\mathbb G))$ are dense in $C(\mathbb G)\otimes C(\mathbb G)$.
\end{itemize} 
\end{definition}

\begin{Expl} Let $G$ be a compact group, and define $\Delta : C(G)\rightarrow C(G)\otimes C(G)$ by $(\Delta f)(g,g') = f(gg')$, where $C(G)\otimes C(G)$ is identified with $C(G\times G)$. Then $(C(G),\Delta ) $ is a compact quantum group.
\end{Expl}

The notation $C(\mathbb G)$ was adopted to suggest an analogy with this example. In general though, the $C^*$-algebra is noncommutative, and there is no topological space underlying the quantum group. Still, one often talks of $C(\mathbb G)$ as "the algebra of continous functions" on $\mathbb G$.

\begin{Expl}
Let $\Gamma$ be a discrete group and define the reduced $C^*$-algebra $C_r^*(\Gamma)$ of $\Gamma$. It coincides with $\C\rtimes_r \Gamma$ and is unital. The map $ \lambda_\gamma \mapsto \lambda_\gamma\otimes \lambda_\gamma$ extends to a unital $*$-homomorphism $\Delta :C_r^*(\Gamma)\rightarrow C_r^*(\Gamma)\otimes C_r^*(\Gamma) $. Then $(C_r^*(\Gamma),\Delta)$ is a compact quantum group, which can be seen as a candidate for what should be the continuous functions on the reduced dual $\hat\Gamma$. By Fourier transform, it is exactly the case when $\Gamma$ is commutative.
\end{Expl}

\begin{definition} A unitary representation $(H,U)$ of $\mathbb G$ on a Hilbert space $H$ is a unitary element $U \in \mathcal M(\mathfrak K(H)\otimes C(\mathbb G))$ such that $(id\otimes\Delta)(U) = U_{12}U_{13}$. Let $(H,U)$ and $(H',U')$ be two unitary representations of $\mathbb G$. The space of intertwiners between $U$ and $U'$ is defined as 
\[Hom(U,U') = \{T\in \mathcal L(H',H) \text{ s.t. }U(T\otimes 1) = (T\otimes 1) U' \}.\]
We denote by $End(U)$ the intertwiners of $Hom(U,U)$. A unitary representation is called irreducible if $End(U) = \C 1$. The trivial representation is denoted $\varepsilon$.
\end{definition}

We define sum and tensor product of representations as follows. Let $(H,U)$ and $(H',U')$ be two unitary representations. Then the sum is defined as $(H\oplus H', U\oplus U')$. The tensor product is defined as $(H\otimes H',W)$ where $W = U_{13} U'_{23}\in \mathcal M(\mathfrak K(H\otimes H')\otimes C(\mathbb G))$. 

\begin{definition}
As in the case of compact groups, we define an equivalence relation on unitary representations of $\mathbb G$ by 
\[(H,U)\sim (H',U') \text{ iff } Hom(U,U') \text{ contains a non trivial unitary operator}.\]
The set of equivalence classes of irreducible unitary representations is denoted by $Irrep(\mathbb G)$. We choose for each $\pi\in Irrep (\mathbb G)$ a unitary representation $(H_\pi, U^{\pi})$. 
\end{definition}

The following theorem is crucial, and is a generalization of the classical case. 

\begin{thm} Every irreducible representation of a compact quantum group is finite dimensional. Every finite dimensional representation of a compact quantum group is equivalent to a sum of irreducible unitary representations. 
\end{thm}
 
We also have the analog of the Haar measure, and of the regular representations.

\begin{prop}
For every compact quantum group $\mathbb G$, there exists a unique state $h$ on $C(\mathbb G)$, called the Haar state on $\mathbb G$, satisfying $(id\otimes h)\Delta(a) = h(a)1$ for every $a\in C(\mathbb G)$. 
\end{prop}

For every $\pi\in Irrep(\mathbb G)$, there exists a unique $\pi^*\in Irrep(\mathbb G)$ such that $Hom(\pi\otimes \pi^*,\varepsilon)\neq 0 \neq Hom(\varepsilon,\pi\otimes \pi^*) $. For a representative $(H_\pi,U^\pi)$, $(H_{\pi^*},U^{\pi^*})$ is called the contragredient representation. Define :
\[L^2(\mathbb G) = \bigoplus_{\pi\in Irrep(\mathbb G)} H_\pi \otimes h_{\pi^*}. \]
Then the right regular representation is defined by
\[\rho\left\{\begin{array}{rcl}
C(\mathbb G) & \rightarrow & \mathcal L(L^2(\mathbb G)) \\
??	& \mapsto & ??\\
\end{array}\right.\]

We now turn to discrete quantum group. We will define it as dual of a compact quantum group. An intrinsic definition (without reference to a given compact quantum group) exists and can be found in \cite{vandaele}.\\

Define the following $C^*$-algebras :
\[\begin{array}{c}
c_0(\hat{\mathbb G})= \bigoplus_{\pi\in Irred(\mathbb G)} \mathcal L(H_\pi),\\
l^\infty(\hat{\mathbb G})= \prod_{\pi\in Irred(\mathbb G)} \mathcal L(H_\pi).
\end{array}\]
Up to our choice of representations $\{(H_\pi,U^\pi)\}_{\pi\in Irrep(\mathbb G)}$, we can define a unitary $U = \bigoplus_{\pi\in Irrep(\mathbb G)} U^\pi\in \mathcal M(c_0(\hat{\mathbb G}) \otimes C(\mathbb G))$. This allows to define a unital $*$-homomorphism $\hat \Delta : l^\infty(\hat{\mathbb G}) \rightarrow l^\infty(\hat{\mathbb G})\otimes l^\infty(\hat{\mathbb G})$ by 
\[\hat\Delta ( x) = U(x \otimes 1)U^*\quad \forall x\in l^\infty(\hat{\mathbb G})).\]
This $*$-homomorphism satisfies $\hat\Delta(c_0(\hat{\mathbb G}))\subseteq \mathcal M(c_0(\hat{\mathbb G})\otimes c_0(\hat{\mathbb G}))$ and \textbf{OTHER CONDITION}.
\begin{definition}
Let $\mathbb G$ be a compact quantum group. With the above notations, its dual is defined as the discrete quantum group $\hat{\mathbb G} = (c_0(\mathbb G),\hat\Delta)$.
\end{definition}

\begin{Expl} Let $\Gamma$ be a discrete group and consider the compact quantum group $\mathbb G=(C^*(\Gamma),\Delta)$. Then, $\gamma\mapsto \lambda_\gamma$ gives a bijection $Irrep(\mathbb G)\cong \Gamma$, and $\hat{\mathbb G}$ is $c_0(\Gamma)$ with comultiplication $\hat\Delta (f)(\gamma,\gamma') = f(\gamma\gamma')$ for every $f\in c_0(\Gamma)$ and $\gamma,\gamma'\in\Gamma$.
\end{Expl}

The dual of a compact quantum group is also represented on $L^2(\mathbb G)$ via the regular repsentations. Let us define the dual left regular representation by :
\[\hat\lambda\left\{\begin{array}{rcl}
l^\infty(\hat{\mathbb G}) & \rightarrow & \mathcal L(L^2(\mathbb G)) \\
a	& \mapsto & \hat\lambda (a) : \xi \mapsto \{(a_\pi\otimes 1)\xi_\pi\}_\pi\\
\end{array}\right.\]


\begin{definition} An action of a discrete quantum group $\hat{\mathbb G}$ on a $C^*$-algebra $A$ is a nondegenerate $*$-homomorphism $A\rightarrow \mathcal M(A\otimes c_0(\hat{\mathbb G}))$ such that :
\begin{itemize}
\item[$\bullet$] $(\alpha \otimes id )\alpha = (id\otimes \hat\Delta) \alpha$
\item[$\bullet$] for every $a\in A$, $(\hat\varepsilon \otimes id)\alpha(a) = a$.
\end{itemize}
Such a pair $(A\alpha)$ is called a $\hat{\mathbb G}$-algebra.
\end{definition}

\begin{definition} Let $(A,\alpha)$ be a $\hat{\mathbb G}$-algebra. The reduced crossed product $A\rtimes_r \hat{\mathbb G}$ is the $C^*$-algebra generated by 
\[\{\theta(a)(w\otimes id)(\hat W_{12}), a\in A , w\in l^\infty(\hat{\mathbb G})_* \},\]
where $\theta = (\hat\lambda \otimes id)\alpha$ and $\hat W = (id\otimes \rho)(U)$.  
\end{definition}

\begin{rk}
There is also a notion of maximal (or full) crossed product for $\hat{\mathbb G}$-algebras.
\end{rk}

\begin{rk}
As in the classical case, the reduced and the maximal crossed products respect the semi-split exact sequences of $\hat{\mathbb G}$-algebras.
\end{rk}

































