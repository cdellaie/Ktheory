\section{Decomposition complexity}

Let $G$ be a locally compact étale groupoid with a proper length $l$. Fix $s>0$, and $G_s=\{g\in G : l^{r(g)}(g)\leq s\}$. For each subset $U$ of $X$, we can define the $s$-neiborghood of $U$ :
\[U_s = \{x\in X : \exists g\in G_s \text{ s.t. } r(g)\in U \text{ and } s(g)=x \}.\]

\begin{definition}
Let $G$ be an étale groupoid with compact base space $X$, and $\mathcal F$ a family of groupoids . We say that $G$ decomposes over $\mathcal F$ if, for every open relatively compact subset $K$ of $G$, every $R>0$, there exists a decomposition $X=X^{(1)}\cup X^{(2)}$ of $K^{(0)}:= r(K)\cup s(K)$ into (open ?) subspaces such that each subspace $X^{(i)}$ is a disjoint union $\sqcup_j X^{(i)}_j$ satisfying the following conditions :
\begin{itemize}
\item[$\bullet$] if $j\neq j'$,  $X^{(i)}_{j,R}\cap X^{(i)}_{j'}=\varnothing$
\item[$\bullet$] there exists $s>0$ such that for every $i,j$, the groupoid $G_{j,s}^{(i)}$ generated by $G_{|X^{(i)}_j}\cap G_s$ is relatively compact and is in $\mathcal F$.
\end{itemize}
\end{definition}

\begin{definition}
Let $R>0$. An étale groupoid is said to be $R$-decomposable over $\mathcal F$ if for every open relatively compact subset $K$ of $G$, every $R>0$, there exists an open cover of $K^{(0)}:= r(K)\cup s(K)\subset V^{(0)}\cup V^{(1)}$ such that each subspace $X^{(i)}$ is a disjoint union $\sqcup_j X^{(i)}_j$ satisfying the following conditions :
\begin{itemize}
\item[$\bullet$] if $j\neq j'$,  $X^{(i)}_{j,R}\cap X^{(i)}_{j'}=\varnothing$
\item[$\bullet$] there exists $s>0$ such that for every $i,j$, the groupoid $G_{j,s}^{(i)}$ generated by $G_{|X^{(i)}_j}\cap G_s$ is in $\mathcal F$.
\end{itemize}
We write $\begin{tikzcd}\mathcal G \arrow{r}{R} & \mathcal F\end{tikzcd}$.
\end{definition}

Let $\mathcal A$ and $\mathcal B$ be two families of $C^*$-algebras.

\begin{definition}
We say that $\mathcal A$ is decomposable over $\mathcal B$ if for every $A$ in $\mathcal A$ and every $\epsilon>0$ and every finite subset $F$ of $A$, there exist two $C^*$-algebras $B^{(0)}$ and $B^{(1)}$ in $\mathcal B$ and ccp maps $\Psi$ and $\Phi$ 
\[\begin{tikzcd}
 \ & B^{(0)}\oplus B^{(1)} \arrow{dr}{\Phi} & \ \\
A \arrow{ur}{\Psi} & \ & A
\end{tikzcd}\]
such that $\Phi_{|B^{(i)}}$ is of order zero and 
\[||\Phi\circ \Psi (a)-a||<\epsilon,\forall a\in F. \]
\end{definition}

\subsection{Examples and applications}

\begin{itemize}
\item[$\bullet$] \textbf{Motivating example.} Let $G=G(X)$ the coarse groupoid associated to a uniformly discrete metric space $X$ with bounded geometry. Then $G$ has FDC iff $X$ has FDC.\\
\item[$\bullet$] \textbf{Dynamic asymptotic dimension.} Recall that an étale groupoid is said to have dynamic asymptotic dimension $d$ if it is the smallest integer such that for every relatively compact subset $K$ of $G$, there exist open subsets $U_0$, ...,$U_d$ of $G^{(0)}$ such that they cover $r(K)\cup s(K)$ and for each $j$, $G_{|U_j}\cap K$ generates a relatively compact subgroupoids of $G$. Then if $G$ has finite dynamic asymptotic dimension, it has FDC. 

\item[$\bullet$] \textbf{Nuclear dimension :} Obtaining a bound on $\dim_{nuc}(A\rtimes G)$ w.r.t. $\dim_{nuc} (A\rtimes G_1)$ and $\dim_{nuc} (A\rtimes G_2)$, maybe $\dim_{cov}(G^{(0)})$.
\item[$\bullet$] \textbf{Mayer Vietoris :} $A\rtimes G_1$ and $A\rtimes G_1$ are a coercive pair, and we have quantitative Mayer-Vietoris. 
\end{itemize}




