\section{Controlled assembly maps for coarse spaces}

We will first construct a controlled assembly map for coarse space $(X,\mathcal E)$. In this section, $X$ will be a discrete metric space with bounded geometry, and $\mathcal E$ is the coarse structure generated by its controlled subsets. We also fix a separable Hilbert space $H$. For $R>0$, $\Delta_R$ is $\{(x,y)\in X\times X\text{ s.t. }d(x,y)<R\}$. \\

Recall first the construction of the Roe algebra of $X$ with coefficients in a $C^*$-algebra $B$, which can be found in \cite{SkTuYu}. $H_B$ denotes the standard $B$-Hilbert module $H\otimes B$. Recall that for every $x,y\in X$, and $T\in\mathcal L_B(l^2(X)\otimes H_B)$, we put $T_{xy}\in\mathcal L_B(H_B)$ to be the unique operator such that $\langle T_{xy}\xi,\eta\rangle = \langle T(e_x\otimes \xi),e_y\otimes\eta\rangle $ for every $x,y\in X$ and every $\xi,\eta\in H_B$.
%$T_{xy}= \chi_y T\chi_x$ , where $\chi_x,\chi_y$ are the characteristic functions of $\{x\}$ and $\{y\}$, seen as projection operators.\\

For any positive number $R>0$, define the family of linear subspaces 
\[C_R[X,B]=\{T\in \mathcal L(l^2(X)\otimes H_B) \text{ s.t. } T_{xy}\in \mathfrak K(H_B) \text{ and }T_{xy}=0 \text{ for }(x,y)\not\in \Delta_R  \}\]
and $C^*(X,B)$ is the completion of $\cup_{R>0} C_R[X,B]$ for the operator norm in $\mathcal L(l^2(X)\otimes H_B) $. Remark that the $C^*$-algebra $C^*(X,B)$ is filtered by $\mathcal E$, and also by $\R_+^*$, seen as a coarse structure. Indeed, the composition law $R\circ R'= R+R'$ provides $\R_+^*$ with a coarse structure and $C_R[X,B] C_{R'} [X,B]\subseteq C_{R+R'}[X,B]$. For the $\mathcal E$-filtration, one can put :
\[C_E[X,B]=\{T\in \mathcal L(l^2(X)\otimes H_B) \text{ s.t. } T_{xy}\in \mathfrak K(H_B) \text{ and }T_{xy}=0 \text{ for }(x,y)\not\in E \}\quad \forall E\in\mathcal E.\]

%%%%%%%%%%%%%%%%%%%%
%For $\phi : A\rightarrow B$, we use the notation $\phi_X $ for the induced $*$-homomorphism $C^*(X,A)\rightarrow C^*(X,B)$. For the reader's convenience, we give details for its construction, which is a standard fact in Coarse Geometry.

%\begin{thm}
%Let $X$ be a discrete metric space with bounded geometry and $\phi : A\rightarrow B$ a $*$-homomorphism. Then there exists a $*$-homomorphism $\phi_X : C^*(X,A)\rightarrow C^*(X,B)$ extending $\phi$. Moreover, $\phi\mapsto \phi_X$ respects composition of $*$-homomorphisms.
%\end{thm}

%\begin{dem}
%Recall that any $*$-morphism $\phi : A\rightarrow B$ induces, for any $A$-Hilbert module $E$, a $*$-morphism $\phi_* : \mathcal L_A(E)\rightarrow \mathcal L_B(E\otimes_A B)$. Now take $E$ to be $l^2(X)\otimes A$. Then $\eta\otimes a\otimes b\mapsto \eta \otimes\phi(a) b $ extends to an isometry $V\in \mathcal L_B(E\otimes_A B,l^2(X)\otimes B)$.\\
%The linear map $T \mapsto V\phi_*(T)V^*$ maps $C_R[X,A]$ into $C_R[X,B]$, and so extends to a $*$-morphism $C^*(X,A)\rightarrow C^*(X,B)$. The composition property is clear from the construction.\\
%\qed
%\end{dem}
%%%%%%%%%%%%%%%%%%%%%%%

To construct the coarse assembly map, we will need the following proposition.\\
 
Let $A$ be a $C^*$-algebra. The theorem \ref{Xfunctor} allows us to take the image of the exact sequence $0 \rightarrow SA \rightarrow CA \rightarrow A \rightarrow 0 $ under the functor $C^*(X,\cdot)$ to get the following filtered semi-split exact sequence 
\[0 \rightarrow C^*(X,SA) \rightarrow C^*(X,CA) \rightarrow C^*(X,A) \rightarrow 0.\] 
Let $D_{X,A} : \hat K_*(C^*(X,A))\rightarrow \hat K_*(C^*(X,SA))$ be the controlled boundary morphism associated to this last extension.

\begin{prop}\label{InverseEven}
Let $A$ be a $C^*$-algebra. 
%The theorem \ref{Xfunctor} allows us to take the image of the exact sequence $0 \rightarrow SA \rightarrow CA \rightarrow A \rightarrow 0 $ under the functor $C^*(X,\cdot)$ to get the following filtered semi-split exact sequence 
%\[0 \rightarrow C^*(X,SA) \rightarrow C^*(X,CA) \rightarrow C^*(X,A) \rightarrow 0.\] 
%Let $D_{X,A} : \hat K_*(C^*(X,A))\rightarrow \hat K_*(C^*(X,SA))$ the controlled boundary morphism associated to this last extension. 
Then there exists a control pair $(\lambda,h)$, independent of $X$ and $A$, such that $D_{X,A}$ is $(\lambda,h)$-invertible.
\end{prop}

\begin{dem}
Recall that $0\rightarrow \mathfrak K(l^2(\N)) \rightarrow \mathcal T_0\rightarrow S\rightarrow 0 $ is the Toeplitz extension. The following diagram has exact rows and commutes
\[\begin{tikzcd} 
SC^*(X,A) \arrow{d}\arrow{r} & CC^*(X,A) \arrow{d}\arrow{r} & C^*(X,A) \arrow{d} \\ 
C^*(X,SA) \arrow{r}          & C^*(X,CA) \arrow{r}          & C^*(X,A)
\end{tikzcd}\]
so that remark \ref{rk3.8} implies that $D_{X,A} = \Psi_*\circ D_{C^*(X,A)}$. The following diagram also has exact rows and commutes
\[\begin{tikzcd} 
\mathfrak K(l^2(\N))\otimes C^*(X,A) \arrow{d}\arrow{r} & \mathcal T_0 \otimes C^*(X,A) \arrow{d}\arrow{r} & SC^*(X,A) \arrow{d} \\ 
C^*(X,\mathfrak K(l^2(\N))\otimes A) \arrow{r}          & C^*(X,\mathcal T_0 \otimes A) \arrow{r}          & C^*(X,SA)
\end{tikzcd}\]
so that remark \ref{rk3.8} implies that $D_{\mathfrak K(l^2(\N))\otimes C^*(X,A),\mathcal T_0\otimes C^*(X,A)} = D_{C^*(X,A),C^*(X,\mathcal T_0\otimes A)}\circ\Psi_*$. A simple computation shows that 
\[D_{C^*(X,A),C^*(X,\mathcal T_0\otimes A)}\circ D_{X,A} \sim D_{\mathfrak K(l^2(\N))\otimes C^*(X,A),\mathcal T_0\otimes C^*(X,A)}\circ D_{C^*(X,A)} \sim \mathcal M_{C^*(X,A)}\]
\qed
\end{dem}

\begin{rk}\label{rkInverse} This result induces a similar statement in $K$-theory. %and in $KK$-theory similar statements. Namely : 
Namely, the boundary maps of the extensions $0 \rightarrow C^*(X,SA) \rightarrow C^*(X,CA) \rightarrow C^*(X,A) \rightarrow 0$ and
$0 \rightarrow\mathfrak K(l^2(\N))\otimes C^*(X,A) \rightarrow \mathcal T_0 \otimes C^*(X,A) \rightarrow SC^*(X,A) \rightarrow 0$ are inverse of each other in $K$-theory.
%\begin{itemize}
%\item[$\bullet$] the boundary maps of the extensions $0 \rightarrow C^*(X,SA) \rightarrow C^*(X,CA) \rightarrow C^*(X,A) \rightarrow 0$ and
%$0 \rightarrow\mathfrak K(l^2(\N))\otimes C^*(X,A) \rightarrow \mathcal T_0 \otimes C^*(X,A) \rightarrow SC^*(X,A) \rightarrow 0$ are inverse of each other in $K$-theory,  
%\item[$\bullet$] $[\partial_{K(l^2(\N))\otimes C^*(X,A), T_0 \otimes C^*(X,A)}]$ and $[\partial_{C^*(X,SA),C^*(X,CA)}]$ are $KK$-inverse of each other.
%\end{itemize}
\end{rk}

%%%%%%%%%%%%%%%%%%%%%%%%%%%%%%%%%%%%%%%%
\subsection{Controlled descent functor}
%%%%%%%%%%%%%%%%%%%%%%%%%%%%%%%%%%%%%%%%

Every $K$-cycle $z\in KK(A,B)$ can be represented as a triplet $(H_B, \pi, T)$ where :
\begin{itemize}
\item[$\bullet$]$\pi : A\rightarrow \mathcal L_B(H_B)$ is a $*$-representation of $A$ on $H_B$.
\item[$\bullet$]$T\in \mathcal L_B(H_B)$ is a self-adjoint operator.
\item[$\bullet$] $T$ and $\pi$ satisfy the $K$-cycle condition, i.e. $[T,\pi(a)]$, $\pi(a)(T^*-T)$ and $\pi(a)(T^2-id_{H_B})$ are compact operators in $\mathfrak K_B(H_B)\cong \mathfrak K \otimes B$ for all $a\in A$.\\
\end{itemize}

We first define a controlled morphism $\hat \sigma_X(z) : \hat K(A)\rightarrow \hat K(B)$ of odd degree for all $z\in KK(A,B)$, which we name the controlled Roe transform. It induces $-\otimes \sigma_X(z)$ in $K$-theory, and will be needed in the definition of the Controlled Coarse Assembly map. Recall that if $\phi : A \rightarrow B$ is a $*$-homomorphism, we denote by $\phi_X : C^*(X,A)\rightarrow C^*(X,B)$ the induced $*$-homomorphism.

\subsubsection{Odd case} %%%%%%%%%%%%%%

For $z\in KK_1(A,B)$, represented by $(H_B,\pi,T)\in E(A,B)$, define $P=(\frac{1+T}{2})\in \mathcal L_B(H_B)$ and 
%$P_X\in\mathcal L(H_{C^*(X,B)})$, and  
\[E^{(\pi,T)} = \{(a,P\pi(a)P + y) : a\in A,y\in  B\otimes \mathfrak K\} \]
which is a $C^*$-algebra such that the following sequence :
\[\begin{tikzcd}[column sep = small]0\arrow{r} & B\otimes \mathfrak K \arrow{r} & E^{(\pi,T)}\arrow{r} & A\arrow{r} & 0 \end{tikzcd}.\]
is exact and semi-split by the completely positive section $s : A\rightarrow B\otimes\mathfrak K ; a\mapsto P\pi(a)P$. Define $E_X = C^*(X,E^{(\pi,T)})$. Up to the $*$-isomomorphism $C^*(X,B\otimes\mathfrak K)\cong C^*(X,B)$, the following sequence
\[\begin{tikzcd}[column sep = small]0\arrow{r} & C^*(X,B) \arrow{r} & E_X^{(\pi,T)}\arrow{r} & C^*(X,A)\arrow{r} & 0 \end{tikzcd}.\]
is exact and semi-split by the completely positive section $s_X : C^*(X,A)\rightarrow E_X^{(\pi,T)}$.\\

\begin{prop}
The controlled boundary map $D^{(\pi,T)}=D_{C^*(X,B),E_X^{(\pi,T)}}$ of the extension $E_X^{(\pi,T)}$ only depends on the class $z$.
\end{prop}

\begin{dem}
Let $(H_B, \pi_j,T_j), j=0,1$ two $K$-cycles which are homotopic via $(H_{B[0,1]},\pi,T)$. We denote by $e_t$ the evaluation at $t\in[0,1]$ for an element of $B[0,1]$, and set $y_t=(e_t)_X(y)$ for every $y\in C^*(X,B[0,1])$. The $*$-homomorphism
\[\phi : \left\{\begin{array}{lll}E_X^{(\pi,T)} & \rightarrow & E_X^{(\pi_t,T_t)} \\ (x,y) & \mapsto & (x, y_t)\end{array}\right.\]
satisfies $\phi(C^*(X,B[0,1]))\subseteq C^*(X,B)$ and makes the following diagram commute
\[\begin{tikzcd}[column sep = small]
0\arrow{r} & C^*(X,B[0,1])\arrow{r}\arrow{d}{\phi_{|C^*(X,B[0,1])}} & E_X^{(\pi,T)} \arrow{r}\arrow{d}{\phi} & C^*(X,A)\arrow{r}\arrow{d}{=}& 0 \\
0\arrow{r} & C^*(X,B)\arrow{r} &  E_X^{(\pi_t,T_t)} \arrow{r} & C^*(X,A) \arrow{r} & 0
\end{tikzcd}.\]

According to remark \ref{rk3.8}, the following holds
\[D_{C^*(X,B),E_X^{(\pi_t,T_t)}} = \phi_* \circ D_{C^*(X,B[0,1]),E_X^{(\pi,T)}}.\]
As $id \otimes (e_t)_X$ gives a homotopy between $id\otimes (e_0)_X$ and $id\otimes (e_1)_X$, and as if two $*$-homomorphisms are homotopic, then they are equal in controlled $K$-theory, 
\[D_{C^*(X,B), E_X^{(\pi_0,T_0)}}=D_{C^*(X,B), E_X^{(\pi_1,T_1)}}\]
holds, and the boundary of the extension $E_X^{(\pi,T)}$ depends only on $z$.\\
\qed
\end{dem}

\begin{definition}
For every $z=[H_B,\pi,T]\in KK_1(A,B)$, we define the Roe transformation $\hat\sigma_X$ as 
\[\hat\sigma_X(z)= D_{C^*(X,B),E_X^{(\pi,T)}}\quad.\]
It is a $(\alpha_D,k_D)$-controlled morphism $\hat K(C^*(X,A))\rightarrow \hat K(C^*(X,B))$ of odd degree.
%, because the Morita equivalence preserves the filtration.
\end{definition}

\begin{prop}\label{Roe1}
Let $A$ and $B$ two $C^*$-algebras. There exists a control pair $(\alpha_X,k_X)$ such that for every $z\in KK_1(A,B)$, there exists a $(\alpha_X,k_X)$-controlled morphism
\[\hat\sigma_{X}(z) : \hat K_*(C^*(X,A))\rightarrow \hat K_{*+1}(C^*(X,B))\]
such that
\begin{enumerate}
\item[(i)] $\hat\sigma_X(z)$ induces right multiplication by $\sigma_{X}(z)$ in $K$-theory ;
\item[(ii)] $\hat\sigma_X$ is additive, i.e.
\[\hat\sigma_X(z+z')=\hat\sigma_X(z)+\hat\sigma_X(z').\]
\item[(iii)] For every $*$-homomorphism $f : A_1\rightarrow A_2$,
\[\hat\sigma_X(f^*(z))=\hat\sigma_X(z)\circ f_{X,*}\] for every $z\in KK_1(A_2,B)$.
\item[(iv)] For every $*$-homomorphism $g : B_1\rightarrow B_2$,
\[\hat\sigma_X(g_*(z))= g_{X,*}\circ \hat\sigma_X(z)\] for every $z\in KK_1(A,B_1)$.
\item[(v)] Let $0\rightarrow J\rightarrow A\rightarrow A/J\rightarrow 0$ be a semi-split extension of $C^*$-algebras and $[\partial_J]\in KK_1(A/J,J)$ be its boundary element. Then 
\[\hat\sigma_X([\partial_{J,A}])=D_{C^*(X,J),C^*(X,A)}.\] 
\end{enumerate}
\end{prop}

\begin{dem}
\begin{enumerate}

\item[(i)]The $K$-cycle $[\partial_{C^*(X,B),E_X^{(\pi,T)}}]\in KK_1(C^*(X,A), C^*(X,B))$ implementing the boundary of the extension $E^{(\pi,T)}$ induces the map $\sigma_X(z)$ by definition which immediately gives the first point.

\item[(ii)] If $z,z'$ are elements of $KK_1(A,B)$, represented by two $K$-cycles $(H_B,\pi_j,T_j)$, and if $(H_B,\pi,T)$ is a $K$-cycle representing the sum $z+z'$, then $E_X^{(\pi,T)}$ is naturally isomorphic to the extension sum of the $E_X^{(j)}:=E_X^{(\pi_j,T_j)}$, namely
\[\begin{tikzcd}[column sep = small]
0\arrow{r} & \mathfrak M_2( C^*(X,B)) \arrow{r} & D \arrow{r} & C^*(X,A) \arrow{r} & 0
\end{tikzcd}\]
where 
\[D=\left\{\begin{pmatrix}x_1 & k_{12}\\ k_{21} & x_2\end{pmatrix} : x_j\in E_X^{(j)} , p_1(x_1)=p_2(x_2), k_{ij}\in C^*(X,B)\right\},\]
with $p_j : E_X^{(j)}\rightarrow C^*(X,A)$ the $*$-homomorphisms of the extensions. Naturality of the controlled boundary maps \ref{rk3.8} ensures that the boundary of the sum of two extensions is the sum of the boundary of each, thus the result.

\item[(iii)] Let $z\in KK_1(A_2,B)$, represented by a cycle $(H_B,\pi,T)$. Representing $f^*(z)$ is $(H_B,f^*\pi,T)$ with $f^*\pi=\pi \circ f$. The map 
\[\phi : \left\{\begin{array}{lll} E_X^{f^*(\pi,T)} & \rightarrow & E_X^{(\pi,T)} \\
( x, P_X(f^*\pi)(x)P_X+y) & \rightarrow & ( f_X(x), P_X(f^*\pi)_X(x)P_X+y) \end{array}\right. \]
satisfies
\begin{enumerate}
\item[$\bullet$] $\phi(C^*(X,B))\subseteq C^*(X,B)$, and makes the following diagram commutes
\[\begin{tikzcd}[column sep = small]
0\arrow{r} & C^*(X,B)\arrow{r}\arrow{d}{=} & E^{f^*(\pi,T)} \arrow{r}\arrow{d}{\phi}& C^*(X,A_1)\arrow{r}\arrow{d}{f_X} & 0\\
0\arrow{r} & C^*(X,B)\arrow{r} & E^{(\pi,T)} \arrow{r}& C^*(X,A_2)\arrow{r} & 0
\end{tikzcd}.\]
\item[$\bullet$] It intertwines the sections of the two extensions.
\end{enumerate}
Remark \ref{rk3.8} ensures that \[D_{C^*(X,B), E_X^{f^*(\pi,T)} } =  D_{C^*(X,B), E_X^{(\pi,T)} }\circ f_{X,*},\] 
and the claim is clear. % from composition by $\mathcal M_{C^*(X,B)}^{-1}$.

%\item[(iv)] Let $\mathcal E = H_{B_1}\otimes_g B_2$, which is a countably generated Hilbert $B_2$-module. The homomorphism $g:B_1\rightarrow B_2$ gives rise to $g_* : \mathcal L_{B_1}(H_{B_1})\rightarrow \mathcal L_{B_2}(\mathcal E)$, which preserves compact operators : $g_*(K_{B_1})\subseteq K(\mathcal E)$. We have a similar statement for $g_X : C^*(X,B_1)\rightarrow C^*(X,B_2)$. We denote $\mathcal E_X$ the Hilbert $C^*(X,B_2)$-module $\mathcal C^*(X,E)\simeq H_{C^*(X,B_1)}\otimes_g (C^*(X,B_2))$.\\

%Let $z\in KK(A,B_1)$ be represented by the $K$-cycle $(H_{B_1},\pi,T)$. Then $(H_{B_1}\otimes_g B_2,g_*\circ\pi, g_*(T))=(\mathcal E, \tilde\pi,\tilde T)$ represents $g_*(z)$.\\

%The map $(x,y)\mapsto (x, (g_X)_*(y))$ induces $\Psi :E^{(\pi,T)}\rightarrow  E^{g_*(\pi,T)} $ such that
%\[\Psi(x,P_X \pi_X(x) P_X +y)\mapsto (x,\tilde P_X \tilde\pi_X(x) \tilde P_X+(g_X)_*(y)).\]
%Indeed, the functor $A\mapsto C^*(X,A)$ commutes with pull-back by $*$-homomorphisms, and $(g_X)*\circ\pi_X=(g_*\circ\pi)_X=\tilde \pi_X$ and $(g_X)_*(P_X) = g_*(P)_X=\tilde P_X$ so that 
%\[(g_X)_*(P_X \pi_X(x) P_X)=\tilde P_X \tilde\pi_X(x) \tilde P_X. \]
%Now, by the stabilisation lemma, we know that the countably generated Hilbert module $\mathcal C^*(X,E)$ sits as a complemented module of $H_{C^*(X,B_2)}$, and there exists a projection $p\in L(H_{C^*(X,B_2)})$ such that $pH_{C^*(X,B_2)}\simeq \mathcal E_X$ and $pK_{C^*(X,B_2)}p\simeq K(\mathcal E_X)$. Let $\psi$ be the composition $K_{C^*(X,B_1)}\rightarrow_{(g_X)_*} K(\mathcal E_X)\rightarrow K_{C^*(X,B_2)}$. In this particular case, we can give an explicit description of $\psi$. The map defined on basic tensor products $(x_j)_{j}\otimes b\mapsto (g(x_j)b)_j $ extends to an isometric embedding $\mathcal E_X \rightarrow H_{C^*(X,B_2)}$, under which $ b\theta_{e_i,e_j}$ is mapped to $g(b)\theta_{u_i,u_j}$, where $\{e_j\}$ and $\{u_j\}$ are respectively the canonical orthogonal basis of $H_{C^*(X,B_1)}$ and $H_{C^*(X,B_2)}$. This gives a commutative diagram 
%\[\begin{tikzcd}[column sep = small]
%0\arrow{r} & K_{C^*(X,B_1)}\arrow{r}\arrow{d}{\psi} & E^{(\pi,T)} \arrow{r}\arrow{d}{\Psi}& C^*(X,A)\arrow{r}\arrow{d}{=} & 0\\
%0\arrow{r} & K_{C^*(X,B_2)}\arrow{r} & E^{g_*(\pi,T)} \arrow{r}& C^*(X,A)\arrow{r} & 0
%\end{tikzcd}.\]
%and $\Psi$ intertwines the two filtered sections by the previous relation. Moreover, $\Psi_{|K_{C^*(X,B_1)}}\subseteq K_{C^*(X,B_2)}$, so that we can again apply the remark $3.7$ of \cite{OY2} to state
%\[ D_{K_{C^*(X,B_2)},E^{g_*(\pi,T)}}=\psi_*\circ D_{K_{C^*(X,B_1)},E^{(\pi,T)}},\]
%which we compose by the Morita equivalence on the left $M_{C^*(X,B_2)}^{-1}$
%\[\hat\sigma_X(g_*(z)) = M_{C^*(X,B_2)}^{-1}\circ g_{X,*}\circ D_{K_{C^*(X,B_1)},E^{(\pi,T)}}.\]
%The homomorphisms inducing the Morita equivalence make the following diagram commutes,
%\[\begin{tikzcd}C^*(X,B_1)\arrow{r}{g_X}\arrow{d} & C^*(X,B_2)\arrow{d} \\ K_{C^*(X,B_1)} \arrow{r}{\psi}& K_{C^*(X,B_2)}\end{tikzcd},\]
%and $\hat\sigma_X(g_*(z))= g_{X,*}\circ M_{C^*(X,B_1)}^{-1}\circ D_{K_{C^*(X,B_1)},E^{(\pi,T)}}=g_{X,*}\circ \hat\sigma(z)$.\\

%%%%%%%%%%%%%%%%%%%%%%%%
%%%  NOUVELLE PREUVE  %%
%%%%%%%%%%%%%%%%%%%%%%%%

\item[(iv)]
Let $z \in KK(A,B_1)$ be represented by the $K$-cycle $(H_{B_1},\pi,T)$. Let $V\in \mathcal L_{B_2}(H_{B_1}\otimes_g B_2)$ be the isometry of remark \ref{isometry}. According to Lemma \ref{isometryKK}, 
\[g_*(z)=[H_{B_1}\otimes_g B_2, \pi\otimes_g 1, T\otimes_g 1]\in KK^G(A,B_2)\] 
is also represented by $[H_{B_2}, \pi',T' ]$ where $\pi' = Ad_{V}\circ (\pi\otimes_g 1)$ and $T' = V(T\otimes_g 1)V^* +1-VV^*$. Let $\psi$ be given by the composition $Ad_{V_X}\circ g_X$.\\
The map $\Psi :(x,y)\mapsto (x, \psi(y))$ defines a $*$-homomorphism $E_X^{(\pi,T)} \rightarrow E_X^{(\pi',T')}$ such that 
\[\Psi(x,P_X\pi_X(x)P_X +y)= (x, P'_X  \pi_X'(a)P'_X + \psi(y)) ).\] 
Indeed, the functor $C^*(X,-)$ commutes with pull-back by $*$-homomorphisms, and $Ad_{V_X}\circ g_X\circ\pi_X= (Ad_V\circ g \circ \pi)_X = \pi'_X$ and $\psi(P_X)= V_X (P_X\otimes_{g_X} 1)V^*_X = (V(P\otimes_g 1 ) V^*)_X = (P')_X$ so that 
\[\psi(P_X \pi_X(x) P_X)=P'_X \pi'_X(x) P'_X. \]
This gives a commutative diagram 
\[\begin{tikzcd}[column sep = small]
0\arrow{r} & C^*(X,B_1)\arrow{r}\arrow{d}{\psi} & E_X^{(\pi,T)} \arrow{r}\arrow{d}{\Psi}& C^*(X,A)\arrow{r}\arrow{d}{=} & 0\\
0\arrow{r} & C^*(X,B_2)\arrow{r} & E_X^{(\pi',T')} \arrow{r}& C^*(X,A)\arrow{r} & 0
\end{tikzcd}.\]
and $\Psi$ intertwines the two filtered sections by the previous relation. Moreover, $\Psi (C^*(X,B_1))\subseteq C^*(X,B_2)$, so that we can again apply the remark \ref{rk3.8} to state
\[ D_{C^*(X,B_2),E_X^{(\pi',T')}}=\psi_*\circ D_{C^*(X,B_1),E_X^{(\pi,T)}}\ .\]
%which we compose by the Morita equivalence on the left $M_{C^*(X,B_2)}^{-1}$
Under the identification $C^*(X,B\otimes\mathfrak K) \cong C^*(X,B)$, $\psi = g_X$, hence
\[\hat\sigma_X(g_*(z)) = g_{X,*}\circ D_{C^*(X,B_1),E_X^{(\pi,T)}}.\]
The homomorphisms inducing the Morita equivalence make the following diagram commutes,
\[\begin{tikzcd}C^*(X,B_1)\arrow{r}{g_X}\arrow{d} & C^*(X,B_2)\arrow{d} \\ K_{C^*(X,B_1) } \arrow{r}{\psi}& K_{C^*(X,B_2)}\end{tikzcd},\]
and $\sigma_X(g_*(z))= g_{X,*}\circ M_{C^*(X,B_1)}^{-1}\circ D_{K_{C^*(X,B_1)},E^{(\pi,T)}}=g_{X,*}\circ \sigma_X(z)$.\\

%% NEW NEW PROOF
\item[(v)] We can suppose $A$ unital. Let $0 \rightarrow J \rightarrow A \rightarrow A /J \rightarrow 0$ be a semi-split extension of $C^*$-algebras with $q:A\rightarrow A/J$ the quotient map.. Let us denote by $s : A/J \rightarrow A $ the completely positive cross section.  According to Kasparov-Stinespring theorem \ref{KasparovStinespring}, there exists a $A$-Hilbert module $E$ and a $*$-homomorphism $\pi : A/J \rightarrow \mathcal L_{A}(A\oplus E)$ such that $s(x) = P \pi(x) P$, where $P \in \mathcal L_{A}(A\oplus E)$ is the projection on the $A$ factor. Consider the $J$-Hilbert module $E' = (A\oplus E)\otimes_J J \cong J\oplus (E\otimes_J J)$, and the natural map $\tilde\pi =\pi\otimes_J 1: A/ J \rightarrow \mathcal L_{J}(E')$. Put $\tilde T= (2P-1)\otimes_J id_J$. By the stabilization theorem, we can suppose that $E'$ is a standard $J$-Hilbert module. 
Put 
\[\psi_0(x)(y \oplus \xi ) =  (xy) \oplus \xi\quad \forall \xi \in E\otimes J,\forall y\in J, \] 
for every $x\in A$. This defines a $*$-homomorphism $\psi_0 : A \rightarrow \mathcal L_J(E')$ such that $\psi_0(x)\in  \mathfrak K _J (E')$. Put $\psi ( a ) = (q(a), \tilde \pi (a))$. As $P\pi (q(a))P -\psi_0(a) = \psi_0( s(q(a)) - a )\in \mathfrak K_J(E')$, $\psi(a)\in E^{(\tilde\pi,\tilde T)}$ holds, and the following diagram is commutative with exact rows:
\[\begin{tikzcd}[column sep = small]
0\arrow{r} &J\arrow{r}\arrow{d}{\psi_0} & A \arrow{r}\arrow{d}{\psi}& A/J\arrow{r}\arrow{d}{=} & 0\\
0\arrow{r} & \mathfrak K_J(E')\arrow{r} & E^{(\tilde \pi, \tilde T)} \arrow{r}& A/J\arrow{r} & 0
\end{tikzcd},\]
Hence, by functoriality and semi-split exactness of the Roe algebra, the following diagram commutes
\[\begin{tikzcd}[column sep = small]
0\arrow{r} & C^*(X,J)\arrow{r}\arrow{d}{(\psi_0)_X} & C^*(X,A) \arrow{r}\arrow{d}{\psi_X}& C^*(X,A/J)\arrow{r}\arrow{d}{=} & 0\\
0\arrow{r} & C^*(X,J)\arrow{r} & E_X^{(\tilde \pi, \tilde T)} \arrow{r}& C^*(X,A/J)\arrow{r} & 0
\end{tikzcd},\]
and remark \ref{rk3.8} ensures that  $\hat\sigma_X([\partial_J]) = D_{C^*(X,J),C^*(X,A)}$.

%% END NEW NEW PROOF

%% NEW PROOF
%\item[(v)] We can suppose $A$ unital. Let $0 \rightarrow J \rightarrow A \rightarrow A /J \rightarrow 0$ be a semi-split extension of $C^*$-algebras with $q:A\rightarrow A/J$ the quotient map.. Let us denote by $s : A/J \rightarrow A $ the completely positive cross section.  According to Kasparov-Stinespring theorem \ref{KasparovStinespring}, there exists a $A$-Hilbert module $E$ and a $*$-homomorphism $\pi : A/J \rightarrow \mathcal L_{A}(A\oplus E)$ such that $s(x) = P \pi(x) P$, where $P \in \mathcal L_{A}(A\oplus E)$ is the projection on the $A$ factor. Consider the $J$-Hilbert module $E' = (A\oplus E)\otimes_J J \cong J\oplus (E\otimes_J J)$, and the natural map $\tilde\pi =\pi\otimes_J 1: A/ J \rightarrow \mathcal L_{J}(E')$. Put $T= (2P-1)\otimes_J id_J$. Then, according to \cite{SkandalisExtension}, 
%\[[E',\tilde \pi , T] = [\partial_J]\]
%in $KK_1(A/J,J)$. Up to adding a degenerate $K$-cycle, we can suppose that $E'$ is standard. Put $\psi ( a ) = (q(a), \tilde \pi (a))$, hence the following diagram is commutative with exact rows:
%\[\begin{tikzcd}[column sep = small]
%0\arrow{r} &J\arrow{r}\arrow{d}{\psi_0} & A \arrow{r}\arrow{d}{\psi}& A/J\arrow{r}\arrow{d}{=} & 0\\
%0\arrow{r} & \mathfrak K_J(E')\arrow{r} & A/J \oplus \mathcal L_J (E') \arrow{r}& A/J\arrow{r} & 0
%\end{tikzcd},\]
%where $J \rightarrow  \mathfrak K_J(E') $ is defined by $\psi_0(x)(\xi\oplus y) = \xi\oplus (xy) $ for every $\xi\in E$ and $x,y\in J$. 
%This entails that the boundary of $E^{(\pi,T)}$ and right multiplication by $[\partial_J]$ coincides. Hence, by functoriality and semi-split exactness of the Roe algebra, remark \ref{rk3.8} ensures that  $\hat\sigma_X([\partial_J]) = D_{C^*(X,J),C^*(X,A)}$.
%\item[(v)] Let $q:A\rightarrow A/J$ be the quotient map and $(H_J, \pi, T)$ be a cycle representing $[\partial_{J,A}]$ and $P=\frac{1+T}{2}$. Put $\mathcal C=A\oplus \mathcal L_J(H_J)$ and consider the following commutative diagram :
%\[\begin{tikzcd}[column sep = small]
%0\arrow{r} &J\arrow{r}\arrow{d} & A \arrow{r}\arrow{d}{\phi}& A/J\arrow{r}\arrow{d}{=} & 0\\
%0\arrow{r} & J\otimes \mathfrak K\arrow{r} & \mathcal C \arrow{r}& A/J\arrow{r} & 0\end{tikzcd},\]
%where $\phi(a) = (q(a),\pi(q(a)))$ and $J\rightarrow J\otimes \mathfrak K; j\mapsto j\otimes e;$ with $e$ a rank one projection. We apply $C^*(X,-)$ to get the commutative diagram
%\[\begin{tikzcd}[column sep = small]
%0\arrow{r} & C^*(X,J)\arrow{r}\arrow{d} & C^*(X,A) \arrow{r}\arrow{d}{\phi_X}& C^*(X,A/J)\arrow{r}\arrow{d}{=} & 0\\
%0\arrow{r} & C^*(X,J)\arrow{r} & C^*(X,\mathcal C) \arrow{r}& C^*(X,A/J)\arrow{r} & 0\end{tikzcd},\]
%where the lines are semi-split filtered and exact. Remark \ref{rk3.8} applies and $D_{C^*(X,J),C^*(X,A)} = D_{C^*(X,J),C^*(X,\mathcal C)}$. But $E^{(\pi,T)}_X$ is a $C^*$-subalgebra of $C^*(X,\mathcal C)$ which makes the following diagram commutative :
%\[\begin{tikzcd}[column sep = small]
%0\arrow{r} & C^*(X,J)\arrow{r}\arrow{d}{=} & E_X^{(\pi,T)} \arrow{r}\arrow{d}{\phi_X}& C^*(X,A/J)\arrow{r}\arrow{d}{=} & 0\\
%0\arrow{r} & C^*(X,J)\arrow{r} & C^*(X,\mathcal C) \arrow{r}& C^*(X,A/J)\arrow{r} & 0\end{tikzcd},\]
%hence another application of remark \ref{rk3.8} yields $D_{J,A} = D_{C^*(X,J),E_X^{(\pi,T)}}$. 
%where the first vertical arrow is the canonical mapping that induces the Morita equivalence. \\

\end{enumerate}
\qed
\end{dem}

\subsubsection{Even case} %%%%%%%%%%%%%%%

We can now define $\hat\sigma_X$ for even $K$-cycles. Let $A$ and $B$ be two $C^*$-algebras and $z\in KK(A,B)$. Recall that $[\partial_{SB}]\in KK_1(B,SB)$ is the $K$-cycle implementing the boundary of the extension $0\rightarrow SB\rightarrow CB\rightarrow B\rightarrow 0$, and $[\partial]\in KK_1(\C,S)$ is the Bott generator. Recall from proposition \ref{InverseEven} that $D_{X,A}$  and $D_{ C^*(X,A),C^*(X,\mathcal T_0\otimes A) }$ are controlled inverse of each other. We will denote $D_{ C^*(X,A),C^*(X,\mathcal T_0\otimes A) }$ by $T_{X,A}$.\\

As $z\otimes_B [\partial_{SB}]$ is an odd $K$-cycle, we can define
\[\hat\sigma_X(z):= T_{X,B}\circ \hat\sigma_X(z\otimes[\partial_{SB}]).\] 

%Here $\tau_D$ refers to the $(\alpha_\tau,k_\tau)$-controlled map $\hat K (A_1\otimes D )\rightarrow \hat K(A_2\otimes D)$, that H. Oyono-Oyono and G. Yu constructed in \cite{OY2} for any $C^*$-algebras $D,A_1,A_2$ and $z\in KK_*(A_1,A_2)$. It enjoys many natural properties, and induces right multiplication by $\tau_D(z)\in KK(A_1\otimes D,A_2\otimes D)$ in $K$-theory. We can see that, if we set $\alpha_J=\alpha_\tau \alpha_D$ and $k_J=k_\tau * k_D$, $\hat\sigma(z)$ is $(\alpha_X,k_X)$-controlled.\\

\begin{prop}\label{Roe2}
Let $A$ and $B$ two $C^*$-algebras. For every $z\in KK_*(A,B)$, there exists a control pair $(\alpha_X,k_X)$ and a $(\alpha_X,k_X)$-controlled morphism
\[\hat\sigma_X(z) : \hat K(C^*(X,A))\rightarrow \hat K(C^*(X,B))\]
of the same degree as $z$, such that
\begin{enumerate}
\item[(i)] $\hat\sigma_X(z)$ induces right multiplication by $\sigma_X(z)$ in $K$-theory ;
\item[(ii)] $\hat\sigma_X$ is additive, i.e.
\[\hat\sigma_X(z+z')=\hat\sigma_X(z)+\hat\sigma_X(z').\]
\item[(iii)] For every $*$-homomorphism $f : A_1\rightarrow A_2$,
\[\hat\sigma_X(f^*(z))=\hat\sigma_X(z)\circ f_{X,*}\] for all $z\in KK_*(A_2,B)$.
\item[(iv)] For every $*$-homomorphism $g : B_1\rightarrow B_2$,
\[\hat\sigma_X(g_*(z))= g_{X,*}\circ \hat\sigma_X(z)\] for all $z\in KK_*(A,B_1)$.
\item[(v)] $\hat\sigma_X([id_A]) \sim_{(\alpha_X,k_X)} id_{\hat K(C^*(X,A))}$
\end{enumerate}
\end{prop}

\begin{dem} The work is already done for odd $KK$-elements. Let $z\in KK_0(A,B)$.
\begin{enumerate}
\item[(i)] Proposition \ref{Roe1} ensures that $\hat \sigma_X(z\otimes [\partial_{SB}])$ induces right multiplication by $\sigma_X(z\otimes [\partial_{SB}])$ in $K$-theory. Proposition \ref{InverseEven} implies that $T_{X,B}$ induces multiplication by $[\partial_{C^*(X,B),C^*(X,B\otimes\mathcal T_0)}]^{-1}$ in $K$-theory. Hence, according to theorem \ref{sigma}, their composition induces $ \sigma_X([\partial_{A,\mathcal T_0 \otimes  A}]^{-1})\circ\sigma_X([\partial_{SB}])\circ\sigma_X(z)$ where we identify $KK$-elements with the maps they induce in $K$-theory. But this element is just $\sigma_X(z)$, because, by remark \ref{rkInverse}, $\sigma_X([\partial_{A,\mathcal T_0 \otimes  A}]^{-1})\circ\sigma_X([\partial_{SB}])= id_{K(C^*(X,B))}$.
\item[(ii)] This is clear by \ref{Roe1}.
\item[(iii)] It is a consequence of the previous proposition \ref{Roe1}, and of the equality $f^*(x)\otimes_D y = f^*(x\otimes_D y)$ for every $*$-homomorphism $f : A'\rightarrow A$ and for every $x\in KK(A,D),y\in KK(D,B)$.
\item[(iv)] By naturality of the boundary element, we have $g_*(z)\otimes_{B_2} [\partial_{SB_2}] = z\otimes_{B_1} g^{*}[\partial _{SB_2}] = (Sg)_{*}(z\otimes_{B_1} [\partial _{SB_1}])$, hence, by proposition \ref{Roe1}, 
\[\hat\sigma_X(g_*(z)) = T_{X,B_2}\circ ((Sg)_X)_* \circ \hat\sigma_{X}(z\otimes_{B_1} [\partial_{SB_1}]).\]
Controlled boundaries are natural, hence $T_{X,B_2}\circ ((Sg)_X)_* = (g_X)_* \circ T_{X,B_1}$, and $\hat\sigma_X(g_*(z)) =(g_X)_* \circ \hat\sigma_X(z)$.
\item[(v)] The same kind of argument we used for the first point concludes. By definition, $\hat\sigma_X([id_A]) = T_{X,A}\circ \hat\sigma_X([\partial_{SA}])$, and, by Proposition \ref{InverseEven}, $T_{X,A}$ is a controlled inverse of $D_{C^*(X,SA),C^*(X,CA)} = D_{X,A}$ which is equal to $\hat\sigma_X([\partial_{SA}])$ by point $(v)$ of \ref{Roe1}.
\end{enumerate}
\qed
\end{dem}

We now show that the Roe transform respects in a quantitative way the Kasparov product. Let us recall the following result from \cite{lafforgue2002k}. It states that every $KK$-element comes from the product of an element coming from a $*$-homomorphism and an element coming from the inverse in $KK$-theory of a $*$-homomorphism. Recall property $(d)$ from \ref{propertyD}, which is just this statement for groupoids. In other words, when the groupoid is a locally compact group, property $d$ is property $2$.

\begin{lem}[\cite{lafforgue2002k}, lemma $1.6.11$] Let $A$ and $B$ be two $C^*$-algebras and $z\in KK_0(A,B)$. Then, there exists a $C^*$-algebra $A_1$, an element $\alpha \in KK(A,A_1)$ and $*$-homomorphims $\theta : A_1 \rightarrow A$ and $\eta : A_1 \rightarrow B$ such that
$\theta^*(\alpha) = id_{A_1}$, $\theta_*(\alpha) = id_{A}$ and $\theta^*(z) = \eta$.
\end{lem}

\begin{prop} There exists a control pair $(\alpha_R,k_R)$ such that for every $C^*$-algebras $A$, $B$ and $C$, and every $z\in KK(A,B),z'\in KK(B,C)$, the controlled equality
\[\hat\sigma_X(z\otimes_B z') \sim_{\alpha_R,k_R} \hat\sigma_X(z')\circ \hat\sigma_X(z)\]
holds.
\end{prop}

\begin{dem}
Assume $\alpha\in KK_0(A,B)$. By naturality, the previous lemma reduces the proof to the special case of $\alpha$ being the inverse of a $*$-homomorphism $\theta : B\rightarrow A$ in $KK$-theory : $\alpha\otimes_B [\theta]=1_A$. Let $z\in KK(B,C)$ :
\[\begin{array}{rcl}
\hat\sigma_X (\alpha\otimes z) & \sim_{\alpha_J^2,k_J*k_J} &  \hat\sigma_X (\alpha\otimes z)\circ \hat\sigma_X (\alpha\otimes [\theta]) \\
			& \sim & \hat\sigma_X (\alpha\otimes z)\circ \hat\sigma_X (\theta_*(\alpha))\\
			& \sim & \hat\sigma_X (\alpha\otimes z)\circ \theta_{X,*}\circ \hat\sigma_X (\alpha)\\
			& \sim & \hat\sigma_X (\theta^*(\alpha\otimes z))\circ \hat\sigma_X (\alpha)\\
			& \sim & \hat\sigma_X (z)\circ \hat\sigma_X (\alpha) \\
\end{array}\] 
because $\theta^*(\alpha\otimes z)=\theta^*(\alpha)\otimes z=1\otimes z =z$. The control on the propagation of the first line follows from remark \ref{rk2.5} and point $(v)$, the other lines are equal by points $(iii)$ and $(iv)$, hence $(\alpha_R,k_R)$ can be taken to be $(2 \alpha_X^{4},( k_X)^{*2})$. If $z'$ is even, we can apply the same argument.\\

Let $z$ and $z'$ be odd $KK$-elements. Then :
\[\begin{array}{rcl}
\hat\sigma_X (z\otimes z') & = &  \hat\sigma_X (z\otimes_B [\partial_{B}]\otimes_{SB} [\partial_B]^{-1}\otimes_B z') \\
			& \sim & \hat\sigma_X ( [\partial_B]^{-1}\otimes_B z')\circ \hat\sigma_X (z\otimes_B [\partial_{B}])\\
			& \sim & \hat\sigma_X ( [\partial_B]^{-1}\otimes_B z')\circ \hat\sigma_X ( [\partial_B])\circ\hat\sigma_X( [\partial_B]^{-1})\circ \hat\sigma_X (z\otimes_B [\partial_{B}])\\		
			& \sim & \hat\sigma_X (  z')\circ \hat\sigma_X (z),\\
\end{array}\] 
where we used the previous case for the second line, Lemma \ref{Roe1} for the third line, and Proposition \ref{InverseEven} for the last one.\\
\qed
\end{dem}

\subsection{Controlled coarse assembly maps}

%If $(X,\mathcal E_X)$ is a coarse space, and $E\in\mathcal E_X$ a controlled subset, any simplex $\eta$ of the Rips complex $P_E(X) = \{m \in Prob(X)\text{ s.t. supp }m \subseteq E\}$ can be written as $\eta = \sum_{x\in X} \lambda_x(\eta) \delta_x$, where $\delta_x$ si the Dirac probability at $x$, and $\lambda_x : P_E(X)\rightarrow [0,1]$ is a continuous function. Set :
Let $E\in\mathcal E$ be a controlled subset. Then any probability $\eta$ of the Rips complex $P_E(X)$ can be written as $\eta = \sum_{x\in X} \lambda_x(\eta) \delta_x$, where $\delta_x$ si the Dirac probability at $x$, and $\lambda_x : P_E(X)\rightarrow [0,1]$ is a continuous function. Set :
\[ h_E : \left\{\begin{array}{rcl} X \times X & \rightarrow & C_0(P_E(X))\\  (x,y) & \mapsto & \lambda_x^{\frac{1}{2}}\lambda_y^{\frac{1}{2}}\end{array}\right. \]  
Let $(e_x)_{x\in X}$ be the canonical basis of $l^2(X)$, $e$ be a rank-one projection in $H$ and $P_E$ be defined as the extension by linearity and continuity of
\[P_E(e_x\otimes\xi\otimes f)= \sum_{y\in X} e_y\otimes (e\xi)\otimes (h(x,y)f)\] 
for every $x\in X$, $\xi\in H$ and $f\in C_0(P_E(X))$. As $\sum_{x\in X} \lambda_x =1$, $P_E$ is a projection of $\mathfrak K(l^2(X)) \otimes C_0(P_E(X))$ of controlled support : $\text{supp }P_E\subseteq E$. Indeed, $\lambda_x^{\frac{1}{2}}\lambda_y^{\frac{1}{2}} =0$ as soon as $(x,y)\notin E$. Hence $P_E$ defines a class $[P_E,0]_{\varepsilon, E'}\in K_0^{\varepsilon, E'} (C^*(X,C_0(P_E(X)))$ for any $\varepsilon\in (0,\frac{1}{4})$ and any $E'\in\mathcal E$ satisfying $E\subseteq E'$.\\

For every $C^*$-algebra $B$ and every controlled subsets $E,E'\in\mathcal E$ such that $E\subseteq E'$, the canonical inclusion $P_E(X)\hookrightarrow P_{E'}(X)$ induces a $*$-homomorphism $q_E^{E'} : C_0(P_{E'}(X))\rightarrow C_0(P_{E}(X))$, hence a map $(q_E^{E'})^* : KK(C_0(P_E(X)),B)\rightarrow KK(C_0(P_{E'}(X)),B)$ in $KK$-theory. It induces another map $((q_E^{E'})_X)_* : K(C^*(X,C_0(P_{E'}(X))))\rightarrow K(C^*(X,C_0(P_{E}(X))))$ in $K$-theory. The family of projections $P_E$ are compatible with the morphisms $q_E^{E'}$, i.e. $((q_E^{E'})_X)_*[P_{E'},0]_{\varepsilon,E'} = [P_{E},0]_{\varepsilon,E}$, for every $\varepsilon\in (0,\frac{1}{4})$.
%Moreover, the inclusion being an isometry, we have a map $K(C^*(P_E(X), B)) \rightarrow K(C^*(P_E(X), B))$, still denoted $q_E^{E'}$.

\begin{definition}
Let $B$ a $C^*$-algebra, $\varepsilon\in (0,\frac{1}{4})$ and $E,F\in\mathcal E_X$ controlled subsets such that $k_X(\varepsilon).E\subseteq F$. The controlled coarse assembly map $\hat\mu_{X,B}=(\mu_{X,B}^{\varepsilon,E,F})_{\varepsilon,E}$ is defined as the family of maps
\[\hat\mu_{X,B}^{\varepsilon, E,F} :\left\{\begin{array}{rcl} KK(C_0(P_E(X)),B) & \rightarrow & K^{\varepsilon, F}(C^*(X,B)) \\
					z & \mapsto & \iota_{\alpha_X \varepsilon',k_X(\varepsilon').F'}^{\varepsilon,F}\circ\hat\sigma_X(z)[P_{E},0]_{\varepsilon', F'}\end{array}\right.\]
where $\varepsilon'$ and $F'$ satisfy :
\begin{itemize}
\item[$\bullet$] $\varepsilon'\in (0,\frac{1}{4})$ such that $\alpha_X \varepsilon'\leq \varepsilon$,
\item[$\bullet$] and $F'\in\mathcal E$ such that $E\subseteq F'$ and $k_X(\varepsilon').F'\subseteq F$.
\end{itemize}
%are chosen not to exceed $\varepsilon$ and $E$ when composed with the propagation of the controlled morphisms. 
\end{definition}

\begin{rk} The controlled coarse assembly map is compatible with the structure morphisms $q_E^{E'}$. Indeed, for every $E,E'\in \mathcal E$ such that $E\subseteq E'$, by proposition \ref{Roe2}, 
\[\hat\sigma_X((q_E^{E'})^*(z))[P_{E'},0]_{\varepsilon,E'}  = \hat\sigma_X(z)\circ ((q_E^{E'})_X)_*[P_E',0]_{\varepsilon,E'}= \hat\sigma_X(z)[P_E,0]_{\varepsilon,E}.\] 
Hence $\hat\mu_{X,B}^{\varepsilon,E,F}\circ(q_E^{E'})^* =\hat\mu_{X,B}^{\varepsilon,E',F}$.
\end{rk}

\begin{rk} The controlled coarse assembly map is also compatible with the structure morphisms $\iota_{\varepsilon,E}^{\varepsilon',E'}$, i.e. $\iota_{\varepsilon,F}^{\varepsilon',F'}\circ\hat\mu_{X,B}^{\varepsilon,E,F} =\hat\mu_{X,B}^{\varepsilon',E,F'}$ for every $F\subseteq F'$ and $\varepsilon\leq \varepsilon'$ such that this equality is defined. 
\end{rk}

\begin{rk}According to Proposition \ref{Roe2}, $\hat\sigma_X(z)$ induces right-multiplication by $\sigma_X(z)$. Hence, the controlled coarse assembly map $\hat \mu_{X,B}$ induces the coarse assembly map $\mu_{X,B}$ in $K$-theory.
\end{rk}

\begin{rk}
This assembly map is defined for the usual Roe algebra of $X$, but could be defined for any "nice" completion of the algebraic Roe algebra $\cup_{E\in \mathcal E_X} C_E[X]$. In particular, we can define an assembly map with values in the controlled $K$-theory of the maximal Roe algebra $C_{max}
^*(X)$, that we will denote by $\hat \mu_{X}^{max}$.\end{rk}

 

































