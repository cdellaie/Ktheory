\section{Controlled assembly maps for coarse spaces}

We will first construct a controlled assembly map for coarse space $(X,\mathcal E)$. In this section, $X$ will be a discrete metric space with bounded geometry, and $\mathcal E$ is the coarse structure generated by its controlled subsets. We also fix a separable Hilbert space $H$. For $R>0$, $\Delta_R$ is $\{(x,y)\in X\times X\text{ s.t. }d(x,y)<R\}$.\\

Recall first the construction of the Roe algebra of $X$ with coefficients in a $C^*$-algebra $B$, which can be found in \cite{SkTuYu}. $H_B$ denotes the standard $B$-Hilbert module $H\otimes B$.\\

For any positive number $R>0$, define the family of linear subspaces 
\[C_R[X,B]=\{T\in \mathcal L(l^2(X)\otimes H_B) \text{ s.t. } T_{xy}\in \mathfrak K(H_B) \text{ and }T_{xy}=0 \text{ for }(x,y)\not\in \Delta_R  \}\]
and $C^*(X,B)$ is the completion of $\cup_{R>0} C_R[X,B]$ for the operator norm in $\mathcal L(l^2(X)\otimes H_B) $. Remark that the $C^*$-algebra $C^*(X,B)$ is filtered by $\mathcal E$, and also by $\R_+^*$, seen as a coarse structure.\\

For $\phi : A\rightarrow B$, we use the notation $\phi_X $ for the induced $*$-homomorphism $C^*(X,A)\rightarrow C^*(X,B)$. We give details for its construction, which is a standard fact in Coarse Geometry, but the author could not find a written proof.

\begin{thm}
Let $X$ be a discrete metric space with bounded geometry and $\phi : A\rightarrow B$ a $*$-morphism. Then there exists a $*$-morphism $\phi_X : C^*(X,A)\rightarrow C^*(X,B)$ extending $\phi$. Moreover, $\phi\mapsto \phi_X$ respects composition of $*$-morphisms.
\end{thm}

\begin{proof}
Recall that any $*$-morphism $\phi : A\rightarrow B$ induces, for any $A$-Hilbert module $E$, a $*$-morphism $\phi_* : \mathcal L_A(E)\rightarrow \mathcal L_B(E\otimes_A B)$. Now take $E$ to be $l^2(X)\otimes A$. Then $\eta\otimes a\otimes b\mapsto \eta \otimes\phi(a) b $ extends to an isometry $V\in \mathcal L_B(E\otimes_A B,l^2(X)\otimes B)$.\\
The linear map $T \mapsto V\phi_*(T)V^*$ maps $C_R[X,A]$ into $C_R[X,B]$, and so extends to a $*$-morphism $C^*(X,A)\rightarrow C^*(X,B)$. The compostion property is clear from the construction.
\end{proof}

\subsection{Controlled descent functor}

Every $K$-cycle $z\in KK(A,B)$ can be represented as a triplet $(H_B, \pi, T)$ where :
\begin{itemize}
\item[$\bullet$]$\pi : A\rightarrow \mathcal L_B(H_B)$ is a $*$-representation of $A$ on $H_B$.
\item[$\bullet$]$T\in \mathcal L_B(H_B)$ is a self-adjoint operator.
\item[$\bullet$] $T$ and $\pi$ satisfy the $K$-cycle condition, i.e. $[T,\pi(a)]$, and $\pi(a)(T^2-id_{H_B})$ are compact operators over $H_B$ for all $a\in A$.\\
\end{itemize}

We first define a controlled morphism $\hat \sigma_X(z) : \hat K(A)\rightarrow \hat K(B)$ of odd degree for all $z\in KK(A,B)$, which we name the Roe transform. It induces $-\otimes \sigma_X(z)$ in $K$-theory, and will be needed in the definition of the Controlled Coarse Assembly map.

\subsubsection{Odd case}

For $z\in KK_1(A,B)$, represented by $(H_B,\pi,T)\in E(A,B)$, define $P=(\frac{1+T}{2})\in \mathcal L(H_B)$ and $P_X\in\mathcal L(H_{C^*(X,B)})$, and  
\[E^{(\pi,T)} = \{(a,P\pi(a)P + y) : a\in C^*(X,A),y\in C^*(X,B)\otimes \mathfrak K\} \]
which is a $C^*$-algebra which make the following sequence exact :
\[\begin{tikzcd}[column sep = small]0\arrow{r} & C^*(X,B)\otimes \mathfrak K \arrow{r} & E^{(\pi,T)}\arrow{r} & C^*(X,A)\arrow{r} & 0 \end{tikzcd}.\]

Let us show that the controlled boundary map $D^{(\pi,T)}=D_{C^*(X,B)\otimes \mathfrak K,E^{(\pi,T)}}$ only depends on the class $z$.

Let $(H_B, \pi_j,T_j), j=0,1$ two $K$-cycles which are homotopic via $(H_{B[0,1]},\pi,T)$. We denote $e_t$ the evaluation at $t\in[0,1]$ for an element of $B[0,1]$, and set $y_t=e_t(y)$ for such a $y$. The $*$-morphism
\[\phi : \left\{\begin{array}{lll}E^{(\pi,T)} & \rightarrow & E^{(\pi_t,T_t)} \\ (x,y) & \mapsto & (x, y_t)\end{array}\right.\]
satisfies $\phi(C^*(X,K_{B[0,1]}))\subset K_{C^*(X,B)}$ and makes the following diagram commute
\[\begin{tikzcd}[column sep = small]
0\arrow{r} & C^*(X,K_{B[0,1]})\arrow{r}\arrow{d}{\phi_{|C^*(X,K_{B[0,1]})}} & E^{(\pi,T)} \arrow{r}\arrow{d}{\phi} & C^*(X,A)\arrow{r}\arrow{d}{=}& 0 \\
0\arrow{r} & C^*(X,B)\otimes\mathfrak K\arrow{r} &  E^{(\pi_t,T_t)} \arrow{r} & C^*(X,A) \arrow{r} & 0
\end{tikzcd}.\]

According to \cite{OY2}, remark $3.7.$, the following holds
\[D_{C^*(X,B)\otimes\mathfrak K,E^{(\pi_t,T_t)}} = \phi_* \circ D_{C^*(X,B[0,1])\otimes\mathfrak K,E^{(\pi,T)}}.\]
As $id \otimes e_t$ gives a homotopy between $id\otimes e_0$ and $id\otimes e_1$, and as if two $*$-morphisms are homotopic, then they are equal in controlled $K$-theory, 
\[D_{C^*(X,B)\otimes\mathfrak K, E^{(\pi_0,T_0)}}=D_{C^*(X,B)\otimes\mathfrak K, E^{(\pi_1,T_1)}}\]
holds, and the boundary of the extension $E^{(\pi,T)}$ depends only on $z$.\\

\begin{definition}
We define the Roe transformation $\hat\sigma_X$ as 
\[\hat\sigma_X(z)= \mathcal M_{C^*(X,B)\otimes \mathfrak K}^{-1} \circ D_{C^*(X,B)\otimes \mathfrak K,E^{(\pi,T)}}
\quad, \forall z\in KK_1(A,B)\]
which is a $(\alpha_D,k_D)$-controlled morphism $\hat K(C^*(X,A))\rightarrow \hat K(C^*(X,B))$, because the Morita equivalence preserves the filtration.
\end{definition}

\begin{prop}\label{Roe1}
Let $A$ and $B$ two $C^*$-algebras. There exists a control pair $(\alpha_X,k_X)$ such that for every $z\in KK_1(A,B)$, there exists a $(\alpha_X,k_X)$-controlled morphism
\[\hat\sigma_{X}(z) : \hat K_*(C^*(X,A))\rightarrow \hat K_{*+1}(C^*(X,B))\]
such that
\begin{enumerate}
\item[(i)] $\hat\sigma_X(z)$ induces right multiplication by $\sigma_{X}(z)$ in $K$-theory ;
\item[(ii)] $\hat\sigma_X$ is additive, i.e.
\[\hat\sigma_X(z+z')=\hat\sigma_X(z)+\hat\sigma_X(z').\]
\item[(iii)] For every $*$-homomorphism $f : A_1\rightarrow A_2$,
\[\hat\sigma_X(f^*(z))=\hat\sigma_X(z)\circ f_{X,*}\] for all $z\in KK_1(A_2,B)$.
\item[(iv)] For every $*$-homomorphism $g : B_1\rightarrow B_2$,
\[\hat\sigma_X(g_*(z))= g_{X,*}\circ \hat\sigma_X(z)\] for all $z\in KK_1(A,B_1)$.
\item[(v)] Let $0\rightarrow J\rightarrow A\rightarrow A/J\rightarrow 0$ be a semi-split extension of $C^*$-algebras and $[\partial_J]\in KK_1(A/J,J)$ be its boundary element. Then 
\[\hat\sigma_X([\partial_J])=D_{C^*(X,J),C^*(X,A)}.\] 
\end{enumerate}
\end{prop}

\begin{dem}
\begin{enumerate}

\item[(i)]The $K$-cycle $[\partial_{K_{C^*(X,B)},E^{(\pi,T)}}]\in KK_1(C^*(X,A), C^*(X,B))$ implementing the boundary of the extension $E^{(\pi,T)}$ induces the map $\sigma_X$ by definition, and modulo Morita equivalence, which immediately gives the first point.

\item[(ii)] If $z,z'$ are elements of $KK_1(A,B)$, represented by two $K$-cycles $(H_B,\pi_j,T_j)$, and if $(H_B,\pi,T)$ is a $K$-cycle representing the sum $z+z'$, then $E^{(\pi,T)}$ is naturally isomorphic to the extension sum of the $E_j:=E^{(\pi_j,T_j)}$, namely
\[\begin{tikzcd}[column sep = small]
0\arrow{r} & K_{C^*(X,B)} \arrow{r} & D \arrow{r} & C^*(X,A) \arrow{r} & 0
\end{tikzcd}\]
where 
\[D=\left\{\begin{pmatrix}x_1 & k_{12}\\ k_{21} & x_2\end{pmatrix} : x_j\in E_j , p_1(x_1)=p_2(x_2), k_{ij}\in K(E_j,E_i)\right\}.\]
Naturality of the controlled boundary maps \cite{OY2} ensures that the boundary of the sum of two extensions is the sum of the boundary of each, thus the result.
\item[(iii)] Let $z\in KK_1(A_2,B)$, represented by a cycle $(H_B,\pi,T)$. Representing $f^*(z)$ is $(H_B,f^*\pi,T)$ with off course $f^*\pi=\pi \circ f$. The map 
\[\phi : \left\{\begin{array}{lll} E^{f^*(\pi,T)} & \rightarrow & E^{(\pi,T)} \\
( x, P_X(f^*\pi)(x)P_X+y) & \rightarrow & ( f_X(x), P_X(f^*\pi)(x)P_X+y) \end{array}\right. \]
satisfies
\begin{enumerate}
\item[$\bullet$] $\phi(K_{C^*(X,B)})\subset K_{C^*(X,B)}$, and makes the following diagram commute
\[\begin{tikzcd}[column sep = small]
0\arrow{r} & K_{C^*(X,B)}\arrow{r}\arrow{d}{=} & E^{f^*(\pi,T)} \arrow{r}\arrow{d}{\phi}& C^*(X,A_1)\arrow{r}\arrow{d}{f_X} & 0\\
0\arrow{r} & K_{C^*(X,B)}\arrow{r} & E^{(\pi,T)} \arrow{r}& C^*(X,A_2)\arrow{r} & 0
\end{tikzcd}.\]
\item[$\bullet$] It intertwines the sections of the two extensions.
\end{enumerate}
Remark $3.7$ of \cite{OY2} assures that \[D_{K_{C^*(X,B)}, E^{f^*(\pi,T)} } =  D_{K_{C^*(X,B)}, E^{(\pi,T)} }\circ f_{X,*}\], and the claim is clear from composition by $\mathcal M_{C^*(X,B)}^{-1}$.

\item[(iv)] Let $\mathcal E = H_{B_1}\otimes_g B_2$, which is a countably generated Hilbert $B_2$-module. The homomorphism $g:B_1\rightarrow B_2$ gives rise to $g_* : \mathcal L_{B_1}(H_{B_1})\rightarrow \mathcal L_{B_2}(\mathcal E)$, which preserves compact operators : $g_*(K_{B_1})\subset K(\mathcal E)$. We have a similar statement for $g_X : C^*(X,B_1)\rightarrow C^*(X,B_2)$. We denote $\mathcal E_X$ the Hilbert $C^*(X,B_2)$-module $\mathcal C^*(X,E)\simeq H_{C^*(X,B_1)}\otimes_g (C^*(X,B_2))$.\\

Let $z\in KK(A,B_1)$ be represented by the $K$-cycle $(H_{B_1},\pi,T)$. Then $(H_{B_1}\otimes_g B_2,g_*\circ\pi, g_*(T))=(\mathcal E, \tilde\pi,\tilde T)$ represents $g_*(z)$.\\

The map $(x,y)\mapsto (x, (g_X)_*(y))$ induces $\Psi :E^{(\pi,T)}\rightarrow  E^{g_*(\pi,T)} $ such that
\[\Psi(x,P_X \pi_X(x) P_X +y)\mapsto (x,\tilde P_X \tilde\pi_X(x) \tilde P_X+(g_X)_*(y)).\]
Indeed, the functor $A\mapsto C^*(X,A)$ commutes with pull-back by $*$-homomorphisms, and $(g_X)*\circ\pi_X=(g_*\circ\pi)_X=\tilde \pi_X$ and $(g_X)_*(P_X) = g_*(P)_X=\tilde P_X$ so that 
\[(g_X)_*(P_X \pi_X(x) P_X)=\tilde P_X \tilde\pi_X(x) \tilde P_X. \]
Now, by the stabilisation lemma, we know that the countably generated Hilbert module $\mathcal C^*(X,E)$ sits as a complemented module of $H_{C^*(X,B_2)}$, and there exists a projection $p\in L(H_{C^*(X,B_2)})$ such that $pH_{C^*(X,B_2)}\simeq \mathcal E_X$ and $pK_{C^*(X,B_2)}p\simeq K(\mathcal E_X)$. Let $\psi$ be the composition $K_{C^*(X,B_1)}\rightarrow_{(g_X)_*} K(\mathcal E_X)\rightarrow K_{C^*(X,B_2)}$. In this particular case, we can give an explicit description of $\psi$. The map defined on basic tensor products $(x_j)_{j}\otimes b\mapsto (g(x_j)b)_j $ extends to an isometric embedding $\mathcal E_X \rightarrow H_{C^*(X,B_2)}$, under which $ b\theta_{e_i,e_j}$ is mapped to $g(b)\theta_{u_i,u_j}$, where $\{e_j\}$ and $\{u_j\}$ are respectively the canonical orthogonal basis of $H_{C^*(X,B_1)}$ and $H_{C^*(X,B_2)}$. This gives a commutative diagram 
\[\begin{tikzcd}[column sep = small]
0\arrow{r} & K_{C^*(X,B_1)}\arrow{r}\arrow{d}{\psi} & E^{(\pi,T)} \arrow{r}\arrow{d}{\Psi}& C^*(X,A)\arrow{r}\arrow{d}{=} & 0\\
0\arrow{r} & K_{C^*(X,B_2)}\arrow{r} & E^{g_*(\pi,T)} \arrow{r}& C^*(X,A)\arrow{r} & 0
\end{tikzcd}.\]
and $\Psi$ intertwines the two filtered sections by the previous relation. Moreover, $\Psi_{|K_{C^*(X,B_1)}}\subset K_{C^*(X,B_2)}$, so that we can again apply the remark $3.7$ of \cite{OY2} to state
\[ D_{K_{C^*(X,B_2)},E^{g_*(\pi,T)}}=\psi_*\circ D_{K_{C^*(X,B_1)},E^{(\pi,T)}},\]
which we compose by the Morita equivalence on the left $M_{C^*(X,B_2)}^{-1}$
\[\hat\sigma_X(g_*(z)) = M_{C^*(X,B_2)}^{-1}\circ g_{X,*}\circ D_{K_{C^*(X,B_1)},E^{(\pi,T)}}.\]
The homomorphisms inducing the Morita equivalence make the following diagram commutes,
\[\begin{tikzcd}C^*(X,B_1)\arrow{r}{g_X}\arrow{d} & C^*(X,B_2)\arrow{d} \\ K_{C^*(X,B_1)} \arrow{r}{\psi}& K_{C^*(X,B_2)}\end{tikzcd},\]
and $\hat\sigma_X(g_*(z))= g_{X,*}\circ M_{C^*(X,B_1)}^{-1}\circ D_{K_{C^*(X,B_1)},E^{(\pi,T)}}=g_{X,*}\circ \hat\sigma(z)$.\\

%%%%%%%%%%%%%%%%%%%%%%%%
%%%  NOUVELLE PREUVE  %%
%%%%%%%%%%%%%%%%%%%%%%%%

\item[(iv)] New proof.\\
Let $z \in KK(A,B_1)$ be represented by the $K$-cycle $(H_{B_1},\pi,T)$. Let $V\in \mathcal L_{B_2}(H_{B_1}\otimes_g B_2)$ be the isometry of remark \ref{isometry}. According to Lemma \ref{isometryKK}, 
\[g_*(z)=[H_{B_1}\otimes_g B_2, \pi\otimes_g 1, T\otimes_g 1]\in KK^G(A,B_2)\] 
is also represented by $[H_{B_2}, \pi',T' ]$ where $\pi' = Ad_{V}\circ (\pi\otimes_g 1)$ and $T' = V(T\otimes_g 1)V^* +1-VV^*$. Let $\psi$ be given by the composition $Ad_{V_X}\circ (g_X)_*$.\\
The map $\Psi :(x,y)\mapsto (x, \psi(y))$ defines a $*$-homomorphism $E^{(\pi,T)} \rightarrow E^{(\pi',T')}$ such that 
\[\Psi(x,P_X\pi_G(x)P_X +y)= (x, P'_X  \pi_X'(a)P'_X + \psi(y)) ).\] 
Indeed, the crossed-product functor commutes with pull-back by $*$-homomorphisms, and $Ad_{V_X}\circ(g_X)_*\circ\pi_X= (Ad_V\circ g_* \circ \pi)_X = \pi'_X$ and $\psi(P_X)= V_X (P_X\otimes_{g_X} 1)V^*_X = (V(P\otimes_g 1 ) V^*)_X = (P')_X$ so that 
\[\psi(P_X \pi_X(x) P_X)=P'_X \pi'_X(x) P'_X. \]
This gives a commutative diagram 
\[\begin{tikzcd}[column sep = small]
0\arrow{r} & K_{C^*(X,B_1)}\arrow{r}\arrow{d}{\psi} & E^{(\pi,T)} \arrow{r}\arrow{d}{\Psi}& C^*(X,A)\arrow{r}\arrow{d}{=} & 0\\
0\arrow{r} & K_{C^*(X,B_2)}\arrow{r} & E^{(\pi',T')} \arrow{r}& C^*(X,A)\arrow{r} & 0
\end{tikzcd}.\]
and $\Psi$ intertwines the two filtered sections by the previous relation. Moreover, $\Psi_{|K_{C^*(X,B_1)}}\subset K_{C^*(X,B_2)}$, so that we can again apply the remark $3.8$ of \cite{OY2} to state
\[ D_{K_{C^*(X,B_2)},E^{(\pi',T')}}=\psi_*\circ D_{K_{C^*(X,B_1)},E^{(\pi,T)}},\]
which we compose by the Morita equivalence on the left $M_{C^*(X,B_2)}^{-1}$
\[\sigma_X(g_*(z)) = M_{C^*(X,B_2)}^{-1}\circ g_{X,*}\circ D_{K_{C^*(X,B_1)},E^{(\pi,T)}}.\]
The homomorphisms inducing the Morita equivalence make the following diagram commutes,
\[\begin{tikzcd}C^*(X,B_1)\arrow{r}{g_X}\arrow{d} & C^*(X,B_2)\arrow{d} \\ K_{C^*(X,B_1) } \arrow{r}{\psi}& K_{C^*(X,B_2)}\end{tikzcd},\]
and $\sigma_X(g_*(z))= g_{X,*}\circ M_{C^*(X,B_1)}^{-1}\circ D_{K_{C^*(X,B_1)},E^{(\pi,T)}}=g_{X,*}\circ \sigma_X(z)$.\\

\item[(v)] Let $q:A\rightarrow A/J$ be the quotient map and $(H_J, \pi, T)$ be a cycle representing $[\partial_J]$. Then we apply remark $3.7$ of \cite{OY2} to the commutative diagram
\[\begin{tikzcd}[column sep = small]
0\arrow{r} & C^*(X,J)\arrow{r}\arrow{d} & C^*(X,A) \arrow{r}\arrow{d}{s\circ q_X}& C^*(X,A/J)\arrow{r}\arrow{d}{=} & 0\\
0\arrow{r} & K_{C^*(X,J)}\arrow{r} & E^{(\pi,T)} \arrow{r}& C^*(X,A/J)\arrow{r} & 0
\end{tikzcd},\]
where the first vertical arrow is the canonical mapping that induces the Morita equivalence. \\
\qed
\end{enumerate}
\end{dem}

\subsubsection{Even case}

We can now define $\hat\sigma_X$ for even $K$-cycles. Let $A$ and $B$ be two $C^*$-algebras. Let $[\partial_{SB}]\in KK_1(B,SB)$ be the $K$-cycle implementing the boundary of the extension $0\rightarrow SB\rightarrow CB\rightarrow B\rightarrow 0$, and $[\partial]\in KK_1(\C,S)$ be the Bott generator. As $z\otimes_B [\partial_{SB}]$ is an odd $K$-cycle, we can define
\[\hat\sigma_X(z):= \tau_{C^*(X,B)}([\partial]^{-1})\circ \hat\sigma_X(z\otimes[\partial_{SB}]).\] 

Here $\tau_D$ refers to the $(\alpha_\tau,k_\tau)$-controlled map $\hat K (A_1\otimes D )\rightarrow \hat K(A_2\otimes D)$, that H. Oyono-Oyono and G. Yu constructed in \cite{OY2} for any $C^*$-algebras $D,A_1,A_2$ and $z\in KK_*(A_1,A_2)$. It enjoys many natural properties, and induces right multiplication by $\tau_D(z)\in KK(A_1\otimes D,A_2\otimes D)$ in $K$-theory. We can see that, if we set $\alpha_J=\alpha_\tau \alpha_D$ and $k_J=k_\tau * k_D$, $\hat\sigma(z)$ is $(\alpha_X,k_X)$-controlled.\\

\begin{prop}
Let $A$ and $B$ two $C^*$-algebras. For every $z\in KK_*(A,B)$, there exists a control pair $(\alpha_X,k_X)$ and a $(\alpha_X,k_X)$-controlled morphism
\[\hat\sigma_X(z) : \hat K(C^*(X,A))\rightarrow \hat K(C^*(X,B))\]
of the same degree as $z$, such that
\begin{enumerate}
\item[(i)] $\hat\sigma_X(z)$ induces right multiplication by $\sigma_X(z)$ in $K$-theory ;
\item[(ii)] $\hat\sigma_X$ is additive, i.e.
\[\hat\sigma_X(z+z')=\hat\sigma_X(z)+\hat\sigma_X(z').\]
\item[(iii)] For every $*$-homomorphism $f : A_1\rightarrow A_2$,
\[\hat\sigma_X(f^*(z))=\hat\sigma_X(z)\circ f_{X,*}\] for all $z\in KK_*(A_2,B)$.
\item[(iv)] For every $*$-homomorphism $g : B_1\rightarrow B_2$,
\[\hat\sigma_X(g_*(z))= g_{X,*}\circ \hat\sigma_X(z)\] for all $z\in KK_*(A,B_1)$.
\item[(v)] $\hat\sigma_X([id_A]) \sim_{(\alpha_X,k_X)} id_{\hat K(C^*(X,A))}$
\end{enumerate}
\end{prop}

\begin{dem}
The point $(iii)$ is a consequence of the previous proposition \ref{Kasparov1}, and of the equality $f^*(x)\otimes y = f^*(x\otimes y)$.\\
\qed
\end{dem}

We now show that the Roe transform respects in a quantitative way the Kasparov product.

\begin{prop} There exists a control pair $(\alpha_R,k_R)$ such that for every $C^*$-algebras $A$, $B$ and $C$, and every $z\in KK(A,B),z'\in KK(B,C)$, the controlled equality
\[\hat\sigma_X(z\otimes_B z') \sim_{\alpha_R,k_R} \hat\sigma_X(z')\circ \hat\sigma_X(z)\]
holds.
\end{prop}

\subsection{Controlled coarse assembly maps}

If $(X,\mathcal E_X)$ is a coarse space, and $E\in\mathcal E_X$ a controlled subset, any simplex $\eta$ of the Rips complex $P_E(X) = \{m \in Prob(X)\text{ s.t. supp }m \subset E\}$ can be written as $\eta = \sum_{x\in X} \lambda_x(\eta) \delta_x$, where $\delta_x$ si the Dirac probability at $x$, and $\lambda_x : P_E(X)\rightarrow [0,1]$ is a continuous function. Set :
\[ P_E : \left\{\begin{array}{rcl} X \times X & \rightarrow & C_0(P_E(X))\\  (x,y) & \mapsto & \lambda_x^{\frac{1}{2}}\lambda_y^{\frac{1}{2}}\end{array}\right. \]  
As $\sum \lambda_x =1$, $P_E$ is a projection of $\mathfrak K(l^2(X)) \otimes C_0(P_E(X))$ of controlled support in $E$, which gives you class in $[P_E,0]_{\varepsilon, E'}\in K_0^{\varepsilon, E'} (C^*(X,C_0(P_E(X)))$ for any $\varepsilon\in (0,\frac{1}{4})$ and any $E'$ containing $E$.\\

\begin{definition}
Let $B$ a $C^*$-algebra, $\varepsilon\in (0,\frac{1}{4})$ and $E\in\mathcal E_X$ a controlled subset. The controlled coarse assembly map $\hat\mu_{X,B}=(\mu_{X,B}^{\varepsilon,E,F})$ is defined as the family of maps
\[\hat\mu_{X,B}^{\varepsilon, E,F} :\left\{\begin{array}{rcl} RK(P_F(X),B) & \rightarrow & K^{\varepsilon, E}(C^*(X,B)) \\
					z & \mapsto & \hat\sigma_X(z)[P_{E'},0]_{\varepsilon', E'}\end{array}\right.\]
where $\varepsilon'$ and $E'$ are chosen not to exceed $\varepsilon$ and $E$ when composed with the propagation of the controlled morphisms. 
\end{definition}

\textit{Remarks :}
\begin{enumerate}
\item This assembly map is defined for the usual Roe algebra of $X$, but could be defined for any "nice" completion of the algebraic Roe algebra $\cup_{E\in \mathcal E_X} C_E[X]$
\item The controlled coarse assembly map $\hat \mu_{X,B}$ induces the coarse assembly map $\mu_{X,B}$ in $K$-theory.
\end{enumerate}

 

































