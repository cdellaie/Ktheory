\usepackage[frenchb]{babel}
\usepackage{amsfonts}
\usepackage{amsmath}
\usepackage{amssymb}
%\usepackage[T1]{fontenc}
\usepackage[utf8]{inputenc}
\usepackage{amsthm}
\usepackage{graphicx}
\usepackage{tikz}
\usepackage{tikz-cd}
\usepackage{hyperref}
\usepackage{amssymb}

\hypersetup{                    % parametrage des hyperliens
    colorlinks=true,                % colorise les liens
    breaklinks=true,                % permet les retours à la ligne pour les liens trop longs
    urlcolor= blue,                 % couleur des hyperliens
    linkcolor= blue,                % couleur des liens internes aux documents (index, figures, tableaux, equations,...)
    citecolor= cyan               % couleur des liens vers les references bibliographiques
    }

%Commandes

\theoremstyle{definition}
\newtheorem{definition}{Definition}[section]
\newtheorem{thm}[definition]{Theorem}
\newtheorem{ex}{Exercice}
\newtheorem{lem}[definition]{Lemma}
\newtheorem*{dem}{Proof}
\newtheorem{prop}[definition]{Proposition}
\newtheorem{cor}[definition]{Corollary}
\newtheorem{conj}[definition]{Conjecture}
\newtheorem{Res}{Result}
\newtheorem{Expl}[definition]{Example}
\newtheorem{rk}[definition]{Remark}

\newcommand{\N}{\mathbb N}
\newcommand{\Z}{\mathbb Z}
\newcommand{\R}{\mathbb R}
\newcommand{\C}{\mathbb C}
\newcommand{\Hil}{\mathcal H}
\newcommand{\Mn}{\mathcal M _n (\mathbb C)}
\newcommand{\K}{\mathbb K}
\newcommand{\B}{\mathbb B}
\newcommand{\Cat}{\mathbb B / \mathbb K}
\newcommand{\G}{\mathcal G }

% Style
% Package Fancyhdr, documentation at
% http://www.xm1math.net/doculatex/entetepied.html

\setlength\parindent{0pt}

\usepackage{geometry}
\geometry{hmargin=2.5cm,vmargin=2.5cm}
%\usepackage[inner=2.5cm,outer=1.5cm,bottom=2cm]{geometry}
%\pagestyle{headings}

\usepackage{fancyhdr}
\pagestyle{fancy}

\renewcommand{\headrulewidth}{1pt}
\fancyhead[L]{Chapter \thechapter}
\fancyhead[R]{}
%\fancyhead[R]{\thepage}
%\fancyhead[C]{\leftmark}

\renewcommand{\footrulewidth}{0pt}






















