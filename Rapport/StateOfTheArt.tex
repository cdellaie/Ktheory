\section{Introduction}

This section is devoted to summarize the extent my work at the end of the second year of my PhD.\\

The aim of this dissertation is to study relationship between several worlds : coarse geometry, groupoids and $C^*$-algebras. The idea is to transfer ideas that were very profitable to coarse geometry into groupoids, and to try and exploit them to compute $K$-theory groups of certain groupoids. In this particular section, we will consider several functors between categories which were defined in the litterature, and will recall their construction.\\

\[\begin{tikzcd}
\  & \textbf{Coarse} \quad (X,\mathcal E) \arrow[bend left]{dr}{G(X)} \arrow[bend right]{dl}{\beta X} \arrow{dd}{C^*X,C^*_uX}& \ \\
 \textbf{Top} \quad (Z,\mathcal T) \arrow[bend right]{dr}{C_0(Z)} & \  & \textbf{Grpds}\quad G \arrow[bend left]{dl}{C^*_r G} \\
 \ & \textbf{C}^* \quad A \arrow{d} & \ \\
\ & \textbf{KK} & \ \\
\end{tikzcd}\] 

\begin{definition}
Let $G$ be a topological groupoid, and $\mathcal E$ a family of subsets of $G$. $\mathcal E$ is said to be a coarse structure on $G$ if 
\begin{itemize}
\item[$\bullet$] if $E$ and $F$ are in $\mathcal E$, then $EF\in \mathcal E$, and $E\cup F\in \mathcal E$, $E^{-1}\in \mathcal E$
\item[$\bullet$] if $E\in \mathcal E$ and $F\subset E$, then $F\in \mathcal E$,
\item[$\bullet$] any finite subset is in $\mathcal E$ (?) and if $G^{(0)}\in \mathcal E$, $\mathcal E$ is said to be unital.
\end{itemize}
\end{definition}

\subsection{From topological spaces to $C^*$-algebras}

This first section recalls some easy facts, and is mainly directed at enlightening the parallel and the analogy of all the constructions that will be done in the following sections.\\

\begin{definition}
Let $Y$ be a set, and $\mathcal T$ a subset of $\mathcal P(Y)$. $\mathcal T$ is called a topology if 
\begin{itemize}
\item[$\bullet$] $\emptyset \in \mathcal T$,
\item[$\bullet$] if $U\in\mathcal U$, then $U^c\in \mathcal U$,
\item[$\bullet$] if $\mathcal U\subset\mathcal T$ , then $\cup_{U\in \mathcal U} U \in \mathcal U$.
\end{itemize}
Elements of $\mathcal U$ are called open sets, and if $x\in X$, we  say of any set containing both an open set and $x$ is a neighborhood of $x$.\\
A topological space is said to be locally compact iff every point has a relatively compact neighborhood. \\
A map between two topological spaces $h : (X,\mathcal T_X)\rightarrow (Y,\mathcal T_Y)$ is continuous if it respects the topology i.e. if $\forall V\in \mathcal T_Y, h^{-1}(V)\in \mathcal T_X$.\\
\end{definition}

The natural $C^*$-algebra associated to a locally compact space $Y$ is $C_0(Y)=\{ f :Y \rightarrow \C \text{ continuous s.t. }\lim_{\infty}f=0\}$, endowed with the supremom norm $||f||=\sup_{y\in Y} |f(y)|$.\\

A continuous map $h : X\rightarrow Y $ induces a $*$-homomorphism $C_0(Y)\rightarrow C_0(X)$ iff $h$ is proper, that is the inverse image of any compact set is compact. A simple reformulation leads to the following.

\begin{prop}
If \textbf{Top} is the category of locally compact spaces with morphisms continuous proper maps, and $C^*$ is the category of $C^*$-algebras with morphisms $*$-homomorphisms, $Y\mapsto C_0(Y)$ is a contravariant functor from \textbf{Top} to $\textbf{C}^*$. 
\end{prop}

There is a notion of dimension in the \textbf{Top}, the covering dimension.

\begin{definition}
The covering dimension of $Y$ is less than $d$ if, for every covering $\mathcal V$ of $Y$, there exists a covering $\mathcal U$ that refines $\mathcal V$ and that decomposes into $d+1$ pieces $\mathcal U = \mathcal U_0 \sqcup ... \sqcup \mathcal U_d$ with the property that for $U,V\in \mathcal U^{(j)}$, then $U\cap V=\emptyset$.
\end{definition}

\subsection{From coarse spaces to $C^*$-algebras}

\begin{definition}
A coarse space is a couple $(X,\mathcal E)$, where $X$ is a set and $\mathcal E$ a coarse structure on the pair groupoid $X\times X$. A coarse map is a map respecting the coarse structure i.e. $h : (X,\mathcal E)\rightarrow (Y,\mathcal F)$ such that $\forall F\in\mathcal F,(h\times h)^{-1}(F)\in \mathcal E$.
\end{definition}

A metric space $(X,d)$ naturally inherits a coarse structure from its bounded subsets :
\[\mathcal E_X = \{E\subset X\times X : \sup_{E} d(x,y)<\infty\}.\]

\begin{definition}
Let $(X,\mathcal E)$ be a coarse space. A $X$-module is a pair $(H_X,\phi )$ where 
\begin{itemize}
\item[$\bullet$] $H_X$ is a Hilbert space, 
\item[$\bullet$] $\phi : C_0(X)\rightarrow \mathcal L(H_X)$ is a $*$-homomorphism.
\item[$\bullet$] The module is said to be non-degenerate if $\{\phi(f)\eta : f\in C_0(X),\eta\in H_X\}$ is dense in $H_X$, and standard if no non-zero function of $C_0X)$ acts as a compact operator of $H_X$.
\end{itemize}
\end{definition}

\textbf{Examples} If $X$ is a discrete metric space with bounded geometry, $l^2(X)\otimes H$ with action by multiplication on the first factor defines a s.n.d. $X$-module.\\
If $X$ is a compact metric space endowed with a measure $\mu$ without atoms, then $L^(X,\mu)$ is a s.n.d. $X$-module.\\

\begin{definition}
Let $T\in \mathcal L(H_X, H_Y)$ be an operator.
\begin{itemize}
\item[$\bullet$] $T$ is said to be locally compact if $\phi(g)T$ and $T\phi(f)$ are compact operators for all $f\in C_0(X),g\in C_0(Y)$.
\item[$\bullet$] The support of $T$ is the complement of the set of points $(x,y)\in X\times X$ such that there exist $f_x,f_y\in C_0(X),C_0(Y)$ such that $f_x(x)\neq 0,f_{y}(y)\neq 0$ and $\phi(f_{y}) T \phi(f_x)=0$.
\item[$\bullet$] The propagation of $T$ is the smallest $E\in \mathcal E$ such that supp $T \subset E$.\textbf{ADAPTER}
\end{itemize}
\end{definition}

Let us define the Roe algebra of $X$ form a fixed s.n.d. $X$-module $H_X$. It will be shown that, up to unnatural isomorphism, it does not depend on the choice of the s.n.d. $X$-module.\\

\begin{definition}
For any $E\in \mathcal E$, define the subspace of locally compact operators with propagation $E$-controlled :
\[C_E[X,H_X] = \{T\in\mathcal L(H_X) : T \text{ is locally compact and supp }T\subset E\}.\]
The Roe algebra is the $C^*$-algebra 
\[C^*(X,H_X) = \overline{\cup_{E\in\mathcal E} C_E[X,H_X]}\]
the closure being taken with respect to the operator norm of $\mathcal L(H_X)$.
\end{definition}

\begin{prop}
Let $(X,\mathcal E),(Y,\mathcal F)$ be two coarse spaces, $H_X,H_Y$ two s.n.d. modules over $X$ and $Y$ repsectively, and $h :X\rightarrow Y$ a coarse map. Then, for any $F\in \mathcal F$, there exists an isometry $V\in \mathcal L(H_X,H_Y)$ such that
\[\text{supp }V \subset (h\times id_Y)^{-1}(E).\]
\end{prop}

\begin{dem}
Extend the representation $\phi$ to $\tilde \phi : L^\infty (X)\rightarrow \mathcal L(H_X)$. \\

Let $F\in \mathcal F$ and $\mathcal U$ be a Borel partition of $Y$ such that $\sqcup_{U\in \mathcal U} U\times U \subset F$ and every $U\in \mathcal U$ is of non-empty interior. If $\chi_A$ denotes the characteristic function of $A$, and because the modules are s.n.d., we can find an isometry \[V^{(U)} :\chi_{h^{-1}(U)}H_X \rightarrow \chi_U H_Y\] 
and by standardness $H_X =\bigoplus \chi(h^{-1}(U)) H_X$ and $H_Y =\bigoplus \chi_U H_Y$, so $V = \oplus V^{(U)}$ fits. Indeed, 
\[\text{supp }V\subset \sqcup h^{-1}(U)\times U \subset (h\times id_Y)^{-1}(F).\] 
\qed
\end{dem}

Now, if $H_X$ and $H'_X$ are two s.n.d. $X$-modules, apply the preceding lemma to the identity map to have an isometry $V: H_X\rightarrow H'_X$ which is supported as close as you want of the diagonal. This induces $Ad_V : C^*(X,H_X)\rightarrow C^*(X,H'_X)$ by $Ad_V(T) = VTV^*$, and gives our isomorphism, which is canonical in $K$-theory. \\

If we decide to fix a s.n.d. module for every coarse space, we can speak of "the" Roe algebra of $X$, and we saw that a coarse map between two coarse spaces $h : X\rightarrow Y$ induces a $*$-homomorphism $h_* : C^*(X,H_X)\rightarrow C^*(Y,H_Y)$. \\

We will now define a dimension on coarse spaces, after an idea of Gromov \textbf{à vérifier}, and that was extensively used in coarse geometry.\\

\begin{definition}
A coarse space $(X,\mathcal E)$ is said to have asymptotic dimension less than $d$ if, for every $E\in \mathcal E$, there is a controlled set $F\in\mathcal F$ and a family $\mathcal U$ of susbsets such that 
\begin{itemize}
\item[$\bullet$] $\mathcal U$ covers $X$,
\item[$\bullet$] every $U\in \mathcal U$ is $F$-controlled, i.e. $\sqcup_{U \in \mathcal U} U\times U \subset F$,
\item[$\bullet$] we have a decomposition $\mathcal U = \mathcal U_0 \sqcup ... \sqcup \mathcal U_d$ such that every pair of subsets of a $\mathcal U^{(j)}$ are $E$-separated.
\end{itemize}
\end{definition}

\subsection{From groupoids to $C^*$-algebras}

We first precise what we consider as morphisms in the category of topological groupoids \textbf{Grpds}.\\

\begin{definition}
Let $G$ and $G'$ be two topological groupoids.\\ 
A generalized morphism between from $G$ to $G'$ is a triple $(Z,p,p')$ where $Z$ is a locally compact space endowed with a right action of $G$ with respect to $p$ and a right action of $G'$ with respect to $p'$ such that :
\begin{itemize}
\item[$\bullet$] the two actions commute : $(g.z).g'=g.(zg'),\forall z\in Z,g\in G_{p(z)},g'\in G'^{p'(z)}$,
\item[$\bullet$] the action of $G'$ is free and proper,
\item[$\bullet$] $Z\times_{p',s} G'\rightarrow Z\times Z $ induces a homeomorphism $Z\rtimes G' \rightarrow Z\times_{G^{(0)}} Z$.\\
\end{itemize}
\end{definition}

Let $G \rightrightarrows G^{(0)}$ be a locally compact groupoid, endowed with a Haar system $\lambda = (\lambda^x)_{x\in G^{(0)}}$.\\

\begin{definition}
Let $X$ be a locally compact $\sigma$-compact space.\\
\begin{itemize}
\item[$\bullet$] A $C_0(X)$-algebra is a pair $(A,\theta)$ where $A$ is a $C^*$-algebra and $\theta : C_0(X)\rightarrow Z(\mathcal M (A))$ is a $*$-homomorphism such that $\theta (C_0(X))A = A$.\\
\item[$\bullet$] A $G$-algebra is a triple $(A,\theta,\alpha)$ where the two first elements form a $C_0(G^{(0)})$-algebra, and $\alpha : s^* A \rightarrow r^* A $ is an action of $G$, i.e. an isomorphism of $C_0(G)$-algebras such that $\alpha_g \circ \alpha_{g'} = \alpha_{gg'}$ whenever $(g,g')\in G^{(2)}$.\\ 
\item[$\bullet$] if $(A,\alpha)$ is a $G$-algebra, a $G$-module $E$ is a pair $(E,V)$ where $E$ is a $A$-Hilbert-module and $V\in \mathcal L(s^*E,r^*E)$ is an unitary such that $V_g\circ V_{g'} = V_{gg'},\forall (g,g')\in G^{(2)}$.
\end{itemize}
\end{definition}

An important example of $G$-module is $L^2(G,A)$. Define the $C_0(X,A)$-valued scalar product on $C_c(G,A)$
\[ \langle \eta,\xi \rangle(x) = \int_{G^x} \eta(g)^* \xi(g) d\lambda^x(g).\]
The $A$-Hilbert module $L^2(G,A)$ is the completion of $C_c(G,A)$ under this scalar-product, and the action of $G$ is defined by left-translation :
\[\forall \xi\in L^2(G_{s(g)},A),(V_g \xi)(h) = \xi(g^{-1}h), \]
which is an isomorphism $s^*L^2(G,A)\rightarrow r^*L^2(G,A)$.\\

As in the group case, the data of a $G$-algebra $A$ allows one to construct the crossed-product $A\rtimes_r G$, which is a $C^*$-algebra. Just note that $C_c(G,A)$ is a $*$-algebra for the convolution product 
\[\xi \ast \eta  (g)=\int_{G^{r(g)}} \xi(h)\alpha_{h}(\eta(h^{-1}g))\lambda^{r(g)}(h)\]
and that it acts on $L^2(G,A)$ by the left-regular representation :
\[\forall f\in C_c(G,A), \xi\in L^2(G,A),\lambda (f)\xi = f\ast \xi.\]
This induces an operator norm, which is often called the reduced norm, on $C_c(G,A)$, $||f||_{r}:= ||\lambda(f)||_{\mathcal L(L^2(G,A))}$.
\begin{definition}
$A\rtimes_r G$ is defined as the $*$-completion of $C_c(G,A)$ for the reduced norm. \\
When $A=\C$ is the $C^*$-algebra of complex numbers with the trivial action of $G$, the reduced cross-product is called the reduced $C^*$-algebra and is denoted by $C^*_r(G)$.\\
\end{definition}

The reduced cross-product is functorial in $A$ with respect to $*$-homomorphisms but does not respect equivariant exact sequences of $G$-algebras (unlike the maximal crossed-product, which does). Going to $KK$-theory, the reduced crossed-product is also functorial $KK^G\rightarrow KK$.\\

\begin{prop}[LeGall \cite{LeGall}]
There exists a homomorphism $j_G : KK^{G}(A,B)\rightarrow KK(A\rtimes_r G,B\rtimes_r G)$ which is natural with respect to the Kaparov product, i.e. $\forall x\in KK^G(A,B),\forall y\in KK^G(B,C)$, one has :
\[j_G(x\otimes_B y ) = j_G(x)\otimes_{B\rtimes_r G} j_G(y).\]
\end{prop}

An important remark that we will use later is that $- \rtimes G$ preserves semi-split exact sequences. More precisely, let 
$\begin{tikzcd}[column sep = small]
0\arrow{r} &  A'\arrow{r} &  A \arrow{r} &  A'' \arrow{r} & 0
\end{tikzcd}$
be an equivariant exact sequence of $G$-algebras which has a completely positive $G$-equivariant section, then
$\begin{tikzcd}[column sep = small]
0\arrow{r} &  A'\rtimes_r G \arrow{r} &  A\rtimes_r G \arrow{r} &  A''\rtimes_r G \arrow{r} & 0
\end{tikzcd}$ is an exact sequence of $C^*$-algebras.\\

What about functoriality in the $G$-variable ? If we reduce our considerations to reduced $C^*$-algebras, the work of J-L. Tu \cite{TuNonHaus} allows to state the following results.\\

\begin{definition}
A generalised morphism $(Z,p,p')$ from $G$ to $G'$ is said to be proper if 
\begin{itemize}
\item[$\bullet$] it is locally proper : the action of $G$ is proper,
\item[$\bullet$] for every compact subset $K\subset G'^{(0)}$, $p'^{-1}(K)$ is $G$-compact.
\end{itemize}
\end{definition}

\begin{prop}[Tu, \cite{TuNonHaus}]
Let $G$ and $G'$ be two locally compact groupoids with Haar systems, and $(Z,p,p')$ a generalized morphism from $G$ to $G'$. \\
If $(Z,p,p')$ is locally proper, the one can construct a $C^*$-correspondence $(E,\pi)$ from $C_r^*(G_1)$ to $C^*_r(G_2)$. \\
Moreover, if the generalized morphism is proper, $\pi : C^*_r(G')\rightarrow \mathcal L(E)$ maps to the compact operators, so that $(Z,p,p')$ defines an element of $KK(C_r^*(G'),C_r^*(G))$.\\
These construction are functorial with respect to composition of generalized morphisms and $C^*$-correspondences.\\ 
\end{prop}

\textbf{Some remarks.} The result of J-L. Tu applies even if the groupoids are non-Hausdorff. If we actually consider a strict morphism $\phi: G_1 \rightarrow G_2$, the generalized morphism is proper iff $\phi$ restricted to $(G_1)^K_K$ is proper, for every compact subset $K\in G_1^{(0)}$.\\ 

\subsection{From coarse spaces to groupoids}

This section gives details on he construction of the coarse groupoid from \cite{SkTuYu}.\\

Let $(X,\mathcal E)$ be a coarse space. The idea is to construct an étale groupoid $G(X)$ extending the pair groupoid $X\times X$. As a topological space, 
\[G(X)= \cup_{E\in \mathcal E} \overline{E},\]
the closure being taken in $\beta (X\times X)$, so that it is Hausdorff and locally compact. The base space is $G(X)^{(0)}=\beta X$. Now, the firt and second projections can be extended by universal property of the Stone-Cech compactification to give the source and the range map $s,r : G(X)\rightrightarrows \beta X$ respectively, as shown in this commutative diagramm :  
\[\begin{tikzcd}
X\times X \arrow{r}\arrow{d}{\iota_{X\times X}} &  X \arrow{r}{\iota_X} & \beta X \\
\beta (X\times X) \arrow[dotted]{urr} & & \\ 
\end{tikzcd}.\]

The only non trivial part is to extend the multiplication of the pair groupoid. This is done by extending the inclusion $E \rightarrow X\times X$ to $\overline{E} \rightarrow \beta X\times \beta X$. The corollary 10.31 of \cite{RoeCoarse} assures that 
\begin{lem}
The map $r\times s : \overline E \rightarrow \beta X\times \beta X$ is a topological embedding.
\end{lem}
Using this lemma, we can embed $G(X)$ in $\beta X\times \beta X$ and use the pair multiplication in this groupoid, the point being that such a multiplication $G(X)\times_{s,r} G(X) \rightarrow G(X)$ is continuous and extend the pair multiplication of $X\times X$.\\

The following proposition (Proposition $3.5$ from \cite{SkTuYu}) assures that $X\mapsto G(X)$ is a functor from uniformly locally finite coarse spaces to groupoids with generalized morphisms as arrows.\\

\begin{prop}
Let $(X,\mathcal E_X)$ and $(Y,\mathcal E_Y)$ be two coarse spaces with uniformly locally finite coarse structures, then a coarse map $h :X\rightarrow Y$ induces a generalized morphism from $G(X)$ to $G(Y)$ :
\[\begin{tikzcd}
G(X)\arrow[xshift=-0.7ex]{d}\arrow[xshift=0.7ex]{d} \ & \arrow{dl} Z \arrow{dr} & G(Y)\arrow[xshift=-0.7ex]{d}\arrow[xshift=0.7ex]{d} \\
\beta X & &\beta Y \\
\end{tikzcd}\] 
\end{prop}

We now state a result from \cite{SkTuYu} (lemma $4.4$), and give, for the reader's convenience, a more detailed proof than that of the paper.\\

\begin{prop}
Let $(X,\mathcal E)$ be a uniformly locally finite coarse space, then we have an isomorphism of $C^*$-algebras
\[C^*(X) \simeq l^\infty(X,\mathfrak K) \rtimes_r G(X).\]
\end{prop}

\begin{proof}
Let $D=l^\infty(X,\mathfrak K)$ and $G=G(X)$. The $C^*$-algebra $D\rtimes_r G$ is generated by continuous functions $f : \overline E \rightarrow D$ such that $f(g)\in D_{s(g)}\simeq \mathfrak K$ for some $E\in \mathcal E$. The crossed product is obtained as the closure in the norm operator defined by the actions of such functions by convolution on $\mathcal E = L^2(G,D)$, which defines by definition a faithful map $D\rtimes_r G \rightarrow \mathcal L(\mathcal E)$.\\

Now take the $G$-invariant ideal $J= C_0(X,\mathfrak K)$. As $D\rightarrow \mathcal M(J)$ is faithful, $\mathcal L(\mathcal E)\rightarrow \mathcal L(\mathcal E\otimes_{D} J)$ is isometric, and we obtain a faithful map by composition $D\rtimes_r G \rightarrow \mathcal L(\mathcal E\otimes_D J)$, and $\mathcal E\otimes_D J \simeq L^2(G,J)$.\\

But $C_0(X)$ acts faithfully by multiplication on $C_0(X,\mathfrak K)$, hence on $H_X=L^(G,J)$, which makes it a n.d.s. $X$-module and induces a faithful map $C^*(X,H_X) \rightarrow \mathcal L(H_X)$.\\

If $T\in C_E[X,H_X]$, define $f(g)=T_{s(g),r(g)}$ when $g\in X\times X$, and extend $f$ by continuity to get $f : \overline E \rightarrow D$, and $f(g)\in D_{s(g)}$.\\

If $f:\overline E \rightarrow D$ is a continuous function such that $f(g)\in D_{s(g)}$, define $T\in C_E[X,H_X]$ by $T\xi = f\xi$. This defines a $*$-homomorphism $D\rtimes_r G \rightarrow C^*(X,H_X)$. The previous construction shows it is surjective. But the action of $D\rtimes_r G$ on $H_X$ being faithful, it is also injective, which concludes the proof.\\
\end{proof}



































