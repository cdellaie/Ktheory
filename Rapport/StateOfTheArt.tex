\section{Introduction}

This section is devoted to summarize the extent my work at the end of the second year of my PhD.\\

The aim of my subject is to study relationship between several worlds : coarse geometry, groupoids and $C^*$-algebras. The idea is to transfer ideas that were very profitable to coarse geometry into groupoids, and to try and exploit them to compute $K$-theory groups of certain groupoids. More precisely, we will consider several functors between categories which we will define later.

\[\begin{tikzcd}
\  & \textbf{Coarse} \quad (X,\mathcal E) \arrow[bend left]{dr}{G(X)} \arrow[bend right]{dl}{\beta X} \arrow{dd}{C^*X,C^*_uX}& \ \\
 \textbf{Top} \quad (Z,\mathcal T) \arrow[bend right]{dr}{C_0(Z)} & \  & \textbf{Grpds}\quad G \arrow[bend left]{dl}{C^*_r G} \\
 \ & \textbf{C}^* \quad A \arrow{d} & \ \\
\ & \textbf{KK} & \ \\
\end{tikzcd}\] 

\begin{definition}
Let $G$ be a topological groupoid, and $\mathcal E$ a family of subsets of $G$. $\mathcal E$ is said to be a coarse structure on $G$ if 
\begin{itemize}
\item[$\bullet$] if $E$ and $F$ are in $\mathcal E$, then $EF\in \mathcal E$, and $E\cup F\in \mathcal E$, $E^{-1}\in \mathcal E$
\item[$\bullet$] if $E\in \mathcal E$ and $F\subset E$, then $F\in \mathcal E$,
\item[$\bullet$] any finite subset is in $\mathcal E$ (?) and if $G^{(0)}\in \mathcal E$, $\mathcal E$ is said to be unital.
\end{itemize}
\end{definition}

\subsection{From coarse spaces to $C^*$-algebras}

\begin{definition}
A coarse space is a couple $(X,\mathcal E)$, where $X$ is a set and $\mathcal E$ a coarse structure on the pair groupoid $X\times X$. A coarse map is a map respecting the coarse structure i.e. $h : (X,\mathcal E)\rightarrow (Y,\mathcal F)$ such that $\forall F\in\mathcal F,(h\times h)^{-1}(F)\in \mathcal E$.
\end{definition}

A metric space $(X,d)$ naturally inherits a coarse structure from its bounded subsets :
\[\mathcal E_X = \{E\subset X\times X : \sup_{E} d(x,y)<\infty\}.\]

\begin{definition}
Let $(X,\mathcal E)$ be a coarse space. A $X$-module is a pair $(H_X,\phi )$ where 
\begin{itemize}
\item[$\bullet$] $H_X$ is a Hilbert space, 
\item[$\bullet$] $\phi : C_0(X)\rightarrow \mathcal L(H_X)$ is a $*$-homomorphism.
\item[$\bullet$] The module is said to be non-degenerate if $\{\phi(f)\eta : f\in C_0(X),\eta\in H_X\}$ is dense in $H_X$, and standard if no non-zero function of $C_0X)$ acts as a compact operator of $H_X$.
\end{itemize}
\end{definition}

\textbf{Examples} If $X$ is a discrete metric space with bounded geometry, $l^2(X)\otimes H$ with action by multiplication on the first factor defines a s.n.d. $X$-module.\\
If $X$ is a compact metric space endowed with a measure $\mu$ without atoms, then $L^(X,\mu)$ is a s.n.d. $X$-module.\\

\begin{definition}
Let $T\in \mathcal L(H_X, H_Y)$ be an operator.
\begin{itemize}
\item[$\bullet$] $T$ is said to be locally compact if $\phi(g)T$ and $T\phi(f)$ are compact operators for all $f\in C_0(X),g\in C_0(Y)$.
\item[$\bullet$] The support of $T$ is the complement of the set of points $(x,y)\in X\times X$ such that there exist $f_x,f_y\in C_0(X),C_0(Y)$ such that $f_x(x)\neq 0,f_{y}(y)\neq 0$ and $\phi(f_{y}) T \phi(f_x)=0$.
\item[$\bullet$] The propagation of $T$ is the smallest $E\in \mathcal E$ such that supp $T \subset E$.\textbf{ADAPTER}
\end{itemize}
\end{definition}

Let us define the Roe algebra of $X$ form a fixed s.n.d. $X$-module $H_X$. It will be shown that, up to unnatural isomorphism, it does not depend on the choice of the s.n.d. $X$-module.\\

\begin{definition}
For any $E\in \mathcal E$, define the subspace of locally compact operators with propagation $E$-controlled :
\[C_E[X,H_X] = \{T\in\mathcal L(H_X) : T \text{ is locally compact and supp }T\subset E\}.\]
The Roe algebra is the $C^*$-algebra 
\[C^*(X,H_X) = \overline{\cup_{E\in\mathcal E} C_E[X,H_X]}\]
the closure being taken with respect to the operator norm of $\mathcal L(H_X)$.
\end{definition}

\begin{prop}
Let $(X,\mathcal E),(Y,\mathcal F)$ be two coarse spaces, $H_X,H_Y$ two s.n.d. modules over $X$ and $Y$ repsectively, and $h :X\rightarrow Y$ a coarse map. Then, for any $F\in \mathcal F$, there exists an isometry $V\in \mathcal L(H_X,H_Y)$ such that
\[\text{supp }V \subset (h\times id_Y)^{-1}(E).\]
\end{prop}

\begin{dem}
Extend the representation $\phi$ to $\tilde \phi : L^\infty (X)\rightarrow \mathcal L(H_X)$. \\

Let $F\in \mathcal F$ and $\mathcal U$ be a Borel partition of $Y$ such that $\sqcup_{U\in \mathcal U} U\times U \subset F$ and every $U\in \mathcal U$ is of non-empty interior. If $\chi_A$ denotes the characteristic function of $A$, and because the modules are s.n.d., we can find an isometry \[V^{(U)} :\chi_{h^{-1}(U)}H_X \rightarrow \chi_U H_Y\] 
and by standardness $H_X =\bigoplus \chi(h^{-1}(U)) H_X$ and $H_Y =\bigoplus \chi_U H_Y$, so $V = \oplus V^{(U)}$ fits. Indeed, 
\[\text{supp }V\subset \sqcup h^{-1}(U)\times U \subset (h\times id_Y)^{-1}(F).\] 
\qed
\end{dem}

Now, if $H_X$ and $H'_X$ are two s.n.d. $X$-modules, apply the preceding lemma to the identity map to have an isometry $V: H_X\rightarrow H'_X$ which is supported as close as you want of the diagonal. This induces $Ad_V : C^*(X,H_X)\rightarrow C^*(X,H'_X)$ by $Ad_V(T) = VTV^*$, and gives our isomorphism, which is canonical in $K$-theory. \\

If we decide to fix a s.n.d. module for every coarse space, we can speak of "the" Roe algebra of $X$, and we saw that a coarse map between two coarse spaces $h : X\rightarrow Y$ induces a $*$-homomorphism $h_* : C^*(X,H_X)\rightarrow C^*(Y,H_Y)$. 

\subsection{From groupoids to $C^*$-algebras}
\subsection{From coarse spaces to groupoids}








































