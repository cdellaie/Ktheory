\section{$C^*$-algebras and $K$-theory}

We recall in this section basic result on $C^*$-algebras and their $K$-theory. Most of the proofs are omitted, and the reader can find them in many textbooks \cite{Murphy}\cite{WeggeOlsen}.\\

An involutive algebra is a $\C$-algebra $A$ endowed with an involution $* : A\rightarrow A$ such that 
\[ (x+y)^*=x^*+y^* \quad (x^*)^* =x \quad (xy)^* = y^* x^* \quad (\lambda x ) = \overline{\lambda}x^*,\]
for all $x,y\in A$ and $\lambda\in\C$.
 
\begin{definition}
A $C^*$-algebra $A$ is an involutive algebra which has a Banach norm with respect to which it is complete and which satisfies the following equality :
\[ || x^* x || = || x ||^2\quad \forall x\in A. \]
\end{definition}

Actually, the norm on a $C^*$-algebra is unique, and is entirely determined by the relation
\[ ||x||^2 = \sup \{t\in \R_+ : x^*x-t^2 \notin A^{\times}\} \]
where $A^\times$ denotes the set of invertible elements of $A$.

\begin{Expl}
For any complex Hilbert space $H$, the set of bounded linear operators $\mathcal L(H)$ with involution given by the adjoint, and norm given by the operator norm, is a $C^*$-algebra. When $H$ is of finite dimension $d$, we get $\mathfrak M_d(\C)$ as a particular case.\\

All finite dimensional (as $\C$-vector spaces) $C^*$-algebras are of the form
\[A \simeq \bigoplus_{j=1}^N \mathfrak M_{d_j}(\C),\]
for some integers $d_j$.
\end{Expl}

\begin{Expl}
If $X$ is a locally compact Hausdorff topological space, the set of continuous functions $C_0(X)$ from $X$ to $\C$ is a $C^*$-algebra, with $||f||=\sup\{|f(x)| : x\in X\}$ and $f^*(x)=\overline f(x)$.\\

The Gelfand-Naimark theorem asserts that all commutative $C^*$-algebras are this form. More specifically, the Gelfand-Naimark correspondence associates to any $C^*$-algebra $A$ the set $X=Spec(A)$ of its maximal ideals. Topologized with, it becomes a locally compact Hausdorff space, which can be identified with the set of algebra homomorphisms $\phi : A\rightarrow \C$. The Gelfand transform $A\rightarrow C_0(X) ; a \mapsto [\phi\mapsto \phi(a)]$ is a $*$-isomorphism.
\end{Expl}

\begin{Expl}

\end{Expl}
