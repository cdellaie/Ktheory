\section{$C^*$-algebras} % and $K$-theory}

We recall in this section basic results on $C^*$-algebras and their $K$-theory. Most of the proofs are omitted, and the reader can find them in many textbooks \cite{Murphy}\cite{WeggeOlsen}.\\

An involutive algebra is a $\C$-algebra $A$ endowed with an involution $* : A\rightarrow A$ such that 
\[ (x+y)^*=x^*+y^* \quad (x^*)^* =x \quad (xy)^* = y^* x^* \quad (\lambda x ) = \overline{\lambda}x^*,\]
for all $x,y\in A$ and $\lambda\in\C$.
 
\begin{definition}
A $C^*$-algebra $A$ is an involutive algebra which has a Banach norm with respect to which it is complete and which satisfies the following equality :
\[ || x^* x || = || x ||^2\quad \forall x\in A. \]
\end{definition}

Actually, the norm on a $C^*$-algebra is unique, and is entirely determined by the relation
\[ ||x||^2 = \sup \{t\in \R_+ : x^*x-t^2 \notin A^{\times}\} \]
where $A^\times$ denotes the set of invertible elements of $A$.

\begin{Expl}
For any complex Hilbert space $H$, the set of bounded linear operators $\mathcal L(H)$ with involution given by the adjoint, and norm given by the operator norm, is a $C^*$-algebra. When $H$ is of finite dimension $d$, we get $\mathfrak M_d(\C)$ as a particular case.\\

All finite dimensional (as $\C$-vector spaces) $C^*$-algebras are of the form
\[A \cong \bigoplus_{j=1}^N \mathfrak M_{d_j}(\C),\]
for some integers $N$ and $d_1,...,d_N$.
\end{Expl}

\begin{Expl}
If $X$ is a locally compact Hausdorff topological space, the set of continuous functions vanishing at infinity $C_0(X)$ from $X$ to $\C$ is a $C^*$-algebra, with $||f||=\sup\{|f(x)| : x\in X\}$ and $f^*(x)=\overline f(x)$.\\

The Gelfand-Naimark theorem asserts that all commutative $C^*$-algebras are this form. More specifically, the Gelfand-Naimark correspondence associates to any commutative $C^*$-algebra $A$ the set $X=Spec(A)$ of its maximal ideals. It is a locally compact Hausdorff space, which can be identified with the set of algebra homomorphisms $\phi : A\rightarrow \C$ with the weak-$*$ topology (seen as a subspace of the dual $A^*$). The Gelfand transform $A\rightarrow C_0(X) ; a \mapsto [\phi\mapsto \phi(a)]$ is a $*$-isomorphism.
\end{Expl}

%\begin{Expl}
%Graph $C^*$-algebras.
%\end{Expl}

\begin{definition} Let $A$ and $B$ be two $C^*$-algebras and $\phi : A\rightarrow B$ a linear map.
\begin{itemize}
\item[$\bullet$] $\phi$ is said to be a $*$-homomorphism if $\phi(a^*)=\phi(a)^*$ and $\phi(ab) = \phi(a)\phi(b)$, for every $a,b\in A$,
\item[$\bullet$] $\phi$ is said to be a positive map if it sends positive elements of $A$ to positive elements of $B$, 
\item[$\bullet$] $\phi$ is said to be a completely positive map if, for every $n\in \N^*$, the map $\phi_n : \mathfrak M_n(A) \rightarrow \mathfrak M_n(A) $ defined by $\phi_n([a_{ij}]_{ij})= [\phi(a_{ij})]_{ij}$ is positive.
\end{itemize}
\end{definition}

We recall some useful constructions. Let $A$ and $B$ be $C^*$-algebras. Then : 

\begin{itemize}
\item[$\bullet$] $SA$ denotes its suspension defined by $A(0,1) = A\otimes C_0(0,1) = C_0((0,1),A)$,
\item[$\bullet$] $CA$ denotes its cone defined by $A[0,1) = A\otimes C_0[0,1) = C_0([0,1),A)$.
\item[$\bullet$] More generally, let $\phi : A \rightarrow B$ be a $*$-homomorphism. The cone of $\phi$, denoted $C_\phi$, is defined as 
\[C_\phi = \{(f,a)\in CB\oplus A \text{ s.t. } f(0) = \phi(a)\}.\] 
\end{itemize}
