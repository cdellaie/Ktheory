\section*{Notations and conventions}

In this paper, $G$ will in general denote a locally compact $\sigma$-compact groupoid, $A,B,D$ will be reserved for $C^*$-algebras or $G$-algebras. In general $\Gamma$ will be a discrete countable group.\\

For any topological space $X$, $\beta X$ denotes its Stone-\v{C}ech compactification. If a continuous surjective map $p : Z\rightarrow X $ is given, $Z_x = p^{-1}(x)$ is called the fiber over $x\in X$. Let $p : Z\rightarrow X $ and $p' : Z'\rightarrow X $ be two such maps. Then $Z\times_{p,p'} Z'$ denotes the fibred product $\{(z,z')\in Z\times Z' \text{ s.t. } p(z)=p'(z')\}$. \\ 

$H$ is the separable Hilbert space. For any $C^*$-algebra $A$, $\mathcal M(A)$ denotes the multiplier algebra of $A$ and $\tilde A$ its smallest unitalization, and for non unital $A$, $\rho_A : \tilde A \rightarrow \C; a+\lambda 1 \mapsto \lambda$ is the evaluation homomorphism. Note that $\tilde A=A$ if $A$ is already unital. If $B$ is a $C^*$-algebra, $H_B$ is the standard right Hilbert module over $B$, i.e. \[H_B= H\otimes B=\{(b_j)\in B^\N \text{ s.t. }\sum b_j^* b_j \text{ converges in B}\}\] with the inner product $\langle b, b'\rangle = \sum b_j^* b_j'$. For any right $B$-Hilbert module $E$, $\mathcal L_B(E)$ is the $C^*$-algebra of adjointable operators and $\mathfrak K_B(E)$ the $C^*$-algebra of compact operators of $E$ in the sense of Hilbert modules.\\

We will use the equivariant $KK$-theory of Le Gall, which are $\Z_2$-graded abelian groups $KK^G_*(A,B)$.\\

For any $G$-algebra $A$, $A\rtimes_r G$ denotes the reduced crossed-product of $A$ by $G$, $A\rtimes_{max} G$ the maximal one. \\

We use the notation $\mathbb G$ for compact quantum groups, and $\hat{\mathbb G}$ for discrete quantum groups. 
