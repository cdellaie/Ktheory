\section{Coarse geometry, Roe algebras and étale groupoids}

\subsection{From coarse spaces to $C^*$-algebras}

\begin{definition}
A coarse space is a couple $(X,\mathcal E)$, where $X$ is a set and $\mathcal E$ a coarse structure on the pair groupoid $X\times X$. A coarse map is a map respecting the coarse structure i.e. $h : (X,\mathcal E)\rightarrow (Y,\mathcal F)$ such that $\forall F\in\mathcal F,(h\times h)^{-1}(F)\in \mathcal E$.
\end{definition}

A metric space $(X,d)$ naturally inherits a coarse structure from its bounded subsets :
\[\mathcal E_X = \{E\subset X\times X : \sup_{E} d(x,y)<\infty\}.\]

\begin{definition}
Let $(X,\mathcal E)$ be a coarse space. A $X$-module is a pair $(H_X,\phi )$ where 
\begin{itemize}
\item[$\bullet$] $H_X$ is a Hilbert space, 
\item[$\bullet$] $\phi : C_0(X)\rightarrow \mathcal L(H_X)$ is a $*$-homomorphism.
\item[$\bullet$] The module is said to be non-degenerate if $\{\phi(f)\eta : f\in C_0(X),\eta\in H_X\}$ is dense in $H_X$, and standard if no non-zero function of $C_0X)$ acts as a compact operator of $H_X$.
\end{itemize}
\end{definition}

\textbf{Examples} If $X$ is a discrete metric space with bounded geometry, $l^2(X)\otimes H$ with action by multiplication on the first factor defines a s.n.d. $X$-module.\\
If $X$ is a compact metric space endowed with a measure $\mu$ without atoms, then $L^(X,\mu)$ is a s.n.d. $X$-module.\\

\begin{definition}
Let $T\in \mathcal L(H_X, H_Y)$ be an operator.
\begin{itemize}
\item[$\bullet$] $T$ is said to be locally compact if $\phi(g)T$ and $T\phi(f)$ are compact operators for all $f\in C_0(X),g\in C_0(Y)$.
\item[$\bullet$] The support of $T$ is the complement of the set of points $(x,y)\in X\times X$ such that there exist $f_x,f_y\in C_0(X),C_0(Y)$ such that $f_x(x)\neq 0,f_{y}(y)\neq 0$ and $\phi(f_{y}) T \phi(f_x)=0$.
\item[$\bullet$] The propagation of $T$ is the smallest $E\in \mathcal E$ such that supp $T \subset E$.\textbf{ADAPTER}
\end{itemize}
\end{definition}

Let us define the Roe algebra of $X$ form a fixed s.n.d. $X$-module $H_X$. It will be shown that, up to unnatural isomorphism, it does not depend on the choice of the s.n.d. $X$-module.\\

\begin{definition}
For any $E\in \mathcal E$, define the subspace of locally compact operators with propagation $E$-controlled :
\[C_E[X,H_X] = \{T\in\mathcal L(H_X) : T \text{ is locally compact and supp }T\subset E\}.\]
The Roe algebra is the $C^*$-algebra 
\[C^*(X,H_X) = \overline{\cup_{E\in\mathcal E} C_E[X,H_X]}\]
the closure being taken with respect to the operator norm of $\mathcal L(H_X)$.
\end{definition}

\begin{prop}
Let $(X,\mathcal E),(Y,\mathcal F)$ be two coarse spaces, $H_X,H_Y$ two s.n.d. modules over $X$ and $Y$ repsectively, and $h :X\rightarrow Y$ a coarse map. Then, for any $F\in \mathcal F$, there exists an isometry $V\in \mathcal L(H_X,H_Y)$ such that
\[\text{supp }V \subset (h\times id_Y)^{-1}(E).\]
\end{prop}

\begin{dem}
Extend the representation $\phi$ to $\tilde \phi : L^\infty (X)\rightarrow \mathcal L(H_X)$. \\

Let $F\in \mathcal F$ and $\mathcal U$ be a Borel partition of $Y$ such that $\sqcup_{U\in \mathcal U} U\times U \subset F$ and every $U\in \mathcal U$ is of non-empty interior. If $\chi_A$ denotes the characteristic function of $A$, and because the modules are s.n.d., we can find an isometry \[V^{(U)} :\chi_{h^{-1}(U)}H_X \rightarrow \chi_U H_Y\] 
and by standardness $H_X =\bigoplus \chi(h^{-1}(U)) H_X$ and $H_Y =\bigoplus \chi_U H_Y$, so $V = \oplus V^{(U)}$ fits. Indeed, 
\[\text{supp }V\subset \sqcup h^{-1}(U)\times U \subset (h\times id_Y)^{-1}(F).\] 
\qed
\end{dem}

Now, if $H_X$ and $H'_X$ are two s.n.d. $X$-modules, apply the preceding lemma to the identity map to have an isometry $V: H_X\rightarrow H'_X$ which is supported as close as you want of the diagonal. This induces $Ad_V : C^*(X,H_X)\rightarrow C^*(X,H'_X)$ by $Ad_V(T) = VTV^*$, and gives our isomorphism, which is canonical in $K$-theory. \\

If we decide to fix a s.n.d. module for every coarse space, we can speak of "the" Roe algebra of $X$, and we saw that a coarse map between two coarse spaces $h : X\rightarrow Y$ induces a $*$-homomorphism $h_* : C^*(X,H_X)\rightarrow C^*(Y,H_Y)$. \\

We will now define a dimension on coarse spaces, after an idea of Gromov \textbf{à vérifier}, and that was extensively used in coarse geometry.\\

\begin{definition}
A coarse space $(X,\mathcal E)$ is said to have asymptotic dimension less than $d$ if, for every $E\in \mathcal E$, there is a controlled set $F\in\mathcal F$ and a family $\mathcal U$ of susbsets such that 
\begin{itemize}
\item[$\bullet$] $\mathcal U$ covers $X$,
\item[$\bullet$] every $U\in \mathcal U$ is $F$-controlled, i.e. $\sqcup_{U \in \mathcal U} U\times U \subset F$,
\item[$\bullet$] we have a decomposition $\mathcal U = \mathcal U_0 \sqcup ... \sqcup \mathcal U_d$ such that every pair of subsets of a $\mathcal U^{(j)}$ are $E$-separated.
\end{itemize}
\end{definition}

\subsection{From coarse spaces to groupoids}

This section gives details on he construction of the coarse groupoid from \cite{SkTuYu}.\\

Let $(X,\mathcal E)$ be a coarse space. The idea is to construct an étale groupoid $G(X)$ extending the pair groupoid $X\times X$. As a topological space, 
\[G(X)= \cup_{E\in \mathcal E} \overline{E},\]
the closure being taken in $\beta (X\times X)$, so that it is Hausdorff and locally compact. The base space is $G(X)^{(0)}=\beta X$. Now, the firt and second projections can be extended by universal property of the Stone-Cech compactification to give the source and the range map $s,r : G(X)\rightrightarrows \beta X$ respectively, as shown in this commutative diagramm :  
\[\begin{tikzcd}
X\times X \arrow{r}\arrow{d}{\iota_{X\times X}} &  X \arrow{r}{\iota_X} & \beta X \\
\beta (X\times X) \arrow[dotted]{urr} & & \\ 
\end{tikzcd}.\]

The only non trivial part is to extend the multiplication of the pair groupoid. This is done by extending the inclusion $E \rightarrow X\times X$ to $\overline{E} \rightarrow \beta X\times \beta X$. The corollary 10.31 of \cite{RoeCoarse} assures that 
\begin{lem}
The map $r\times s : \overline E \rightarrow \beta X\times \beta X$ is a topological embedding.
\end{lem}
Using this lemma, we can embed $G(X)$ in $\beta X\times \beta X$ and use the pair multiplication in this groupoid, the point being that such a multiplication $G(X)\times_{s,r} G(X) \rightarrow G(X)$ is continuous and extend the pair multiplication of $X\times X$.\\

The following proposition (Proposition $3.5$ from \cite{SkTuYu}) assures that $X\mapsto G(X)$ is a functor from uniformly locally finite coarse spaces to groupoids with generalized morphisms as arrows.\\

\begin{prop}
Let $(X,\mathcal E_X)$ and $(Y,\mathcal E_Y)$ be two coarse spaces with uniformly locally finite coarse structures, then a coarse map $h :X\rightarrow Y$ induces a generalized morphism from $G(X)$ to $G(Y)$ :
\[\begin{tikzcd}
G(X)\arrow[xshift=-0.7ex]{d}\arrow[xshift=0.7ex]{d} \ & \arrow{dl} Z \arrow{dr} & G(Y)\arrow[xshift=-0.7ex]{d}\arrow[xshift=0.7ex]{d} \\
\beta X & &\beta Y \\
\end{tikzcd}\] 
\end{prop}

We now state a result from \cite{SkTuYu} (lemma $4.4$), and give, for the reader's convenience, a more detailed proof than that of the paper.\\

\begin{prop}
Let $(X,\mathcal E)$ be a uniformly locally finite coarse space, then we have an isomorphism of $C^*$-algebras
\[C^*(X) \simeq l^\infty(X,\mathfrak K) \rtimes_r G(X).\]
\end{prop}

\begin{proof}
Let $D=l^\infty(X,\mathfrak K)$ and $G=G(X)$. The $C^*$-algebra $D\rtimes_r G$ is generated by continuous functions $f : \overline E \rightarrow D$ such that $f(g)\in D_{s(g)}\simeq \mathfrak K$ for some $E\in \mathcal E$. The crossed product is obtained as the closure in the norm operator defined by the actions of such functions by convolution on $\mathcal E = L^2(G,D)$, which defines by definition a faithful map $D\rtimes_r G \rightarrow \mathcal L(\mathcal E)$.\\

Now take the $G$-invariant ideal $J= C_0(X,\mathfrak K)$. As $D\rightarrow \mathcal M(J)$ is faithful, $\mathcal L(\mathcal E)\rightarrow \mathcal L(\mathcal E\otimes_{D} J)$ is isometric, and we obtain a faithful map by composition $D\rtimes_r G \rightarrow \mathcal L(\mathcal E\otimes_D J)$, and $\mathcal E\otimes_D J \simeq L^2(G,J)$.\\

But $C_0(X)$ acts faithfully by multiplication on $C_0(X,\mathfrak K)$, hence on $H_X=L^(G,J)$, which makes it a n.d.s. $X$-module and induces a faithful map $C^*(X,H_X) \rightarrow \mathcal L(H_X)$.\\

If $T\in C_E[X,H_X]$, define $f(g)=T_{s(g),r(g)}$ when $g\in X\times X$, and extend $f$ by continuity to get $f : \overline E \rightarrow D$, and $f(g)\in D_{s(g)}$.\\

If $f:\overline E \rightarrow D$ is a continuous function such that $f(g)\in D_{s(g)}$, define $T\in C_E[X,H_X]$ by $T\xi = f\xi$. This defines a $*$-homomorphism $D\rtimes_r G \rightarrow C^*(X,H_X)$. The previous construction shows it is surjective. But the action of $D\rtimes_r G$ on $H_X$ being faithful, it is also injective, which concludes the proof.\\
\end{proof}












