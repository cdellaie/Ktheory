\section{A quantitative Künneth formula}

In this section, we present an application of the previous results to a Künneth formula for crossed products by an étale groupoid. Controlled Künneth formulas are more stable than Künneth formulas in $K$-theory. In \cite{OY4}, it was proved that they are stable by controlled Mayer-Vietoris decomposition. The reader is sent to the conclusion for details (in particular Definition \ref{MVpair}). \\ 

Recall that for $A$ and $B$ two $C^*$-algebras, one can define a homomorphism
\[\alpha_{A,B} : K_*(A)\otimes K_*(B)\rightarrow K_*(A\otimes B) \quad ; \quad (x,y)\mapsto x\otimes   \tau_A(y),\]
where $\tau_A$ is the external Kasparov product.\\

Recall that, when $A$ and $B$ are unital $C^*$-algebras, if $p$ and $q$ are projections in $\mathfrak M_r(A)$ and $\mathfrak M_s(B)$ and $u$ and $v$ are unitaries in $\mathfrak M_r(A)$ and $\mathfrak M_s(B)$, then :
\[\begin{array}{rl}
\alpha_{A,B}([p]\otimes [q]) & = [p\otimes q], \\
\alpha_{A,B}([u]\otimes [q]) & = [u \otimes q +1\otimes (1-q)], \\
\alpha_{A,B}([p]\otimes [v]) & = [p\otimes v +(1-p)\otimes v] .\\
\end{array}\]

\begin{definition}
A $C^*$-algebra $A$ is said to satisfy the Künneth formula if, for every $C^*$-algebra $B$ such that $K_*(B)$ is a free abelian group, $\alpha_{A,B}$ is an isomorphism.
\end{definition}

If $A$ satifies the Künneth formula, then, for any $C^*$-algebra $B$, one has the following exact sequence
\[\begin{tikzcd}[column sep = small] 0 \arrow{r} & K_*(A)\otimes K_*(B)\arrow{r} & K_*(A\otimes B) \arrow{r} & Tor(K_*(A),K_*(B))\arrow{r} & 0 \end{tikzcd}\]

If $(A,\mathcal E)$ is now a $\mathcal E$-filtered $C^*$-algebra, $(A\otimes B,\mathcal E)$ is also filtered, and one can define a controlled morphism
\[\hat\alpha_{A,B} : \hat K_*(A)\otimes K_*(B)\rightarrow \hat K_*(A\otimes B) \quad ; \quad (x,y)\mapsto \hat\tau_A(y)(x),\]
which induces $\alpha_{A,B}$ in $K$-theory. \\

\begin{definition}
A filtered $C^*$-algebra $(A,\mathcal E)$ is said to satisfy the quantitative Künneth formula if, for every $C^*$-algebra $B$ such that $K_*(B)$ is a free abelian group, $\hat\alpha_{A,B}$ is a controlled isomorphism.
\end{definition}

The remainder of the section is devoted to prove the following theorem. We will define a class $\mathcal C$ of groupoids such that the following is true.

\begin{thm}\label{Kunneth}
Let $G$ be a $\sigma$-compact étale groupoid and $A$ a $G$-algebra. Suppose that 
\begin{itemize}
\item[$\bullet$] $G$ satisfies the Baum-Connes conjecture with coefficients,
\item[$\bullet$] $G$ is in class $\mathcal C$,
\item[$\bullet$] for every compact open subgroupoid $H$ of $G$, $A\rtimes_r H$ satisfies the Künneth formula.
\end{itemize} 
Then $A\rtimes_r G$ satisfies the quantitative Künneth formula.
\end{thm}

%The following corollary is obvious from the theorem.\cite{FinnSellFibred}

%\begin{cor}
%Let $X$ be a proper discrete metric space with bounded geometry which admits a fibered coarse embedding into a Hilbert space and $A$ a $C^*$-algebra. If $A$ satisfies the Künneth formula, then $A\otimes K(l^2(X))$ satifies the quantitative Künneth formula.
%\end{cor}

%%%%%%%%%%%%%%%%%%%%%%%%%%
%	INDUCTION	%%
%%%%%%%%%%%%%%%%%%%%%%%%%%
%  %%%%%%%%%%%%%%%%%%%%%%%%%%%%%%%%%%%%%%%%%%%%%%%%
\subsection{Induction and Restriction functors}
%%%%%%%%%%%%%%%%%%%%%%%%%%%%%%%%%%%%%%%%%%%%%%%%

We recall the following construction from \cite{LeGall}. Let $G$ and $G'$ two étale groupoids, and $\phi=(Z,p,p')$ a generalized morphism from $G$ to $G'$. \\

For any $G'$-equivariant $B$-Hilbert module $E$, define $\phi^*E=l^2_Z\otimes_{C(G')} E$ where $l^2_Z $ is the completion of the space of sections $C_c(Z,B)$ such that $f(z)\in B_{p'(z)}$. Then $\phi^*E$ is a $G$-equivariant $\phi^*B$-Hilbert module. \\

If $\pi : A\rightarrow \mathcal L_B(E)$ and every $F\in\mathcal L_B(E)$, define $\pi' = \pi\otimes_{C(G')} id_{C_0(G^{(0)})}$, and $F'=F\otimes id_{C_0(G^{(0)})}= $. If $(E,\pi,F)\in E^{G'}(A,B)$, we use the notation $\phi^* (E,\pi,F)=(\phi^*E,\pi',F')$.\\

Le Gall proves \cite{LeGall} that $(E,\pi,F)\mapsto \phi^*(E,\pi,F)$ descends to a homomorphism of abelian groups $KK^{G'}(A,B)\rightarrow KK^{G}(\phi^*A,\phi^*B)$ which respects Kasparov products, i.e. $\phi^*(x\otimes_D y)= \phi^*(x)\otimes_{\phi^*D}\phi^*(y)$ holds for all $x,y\in KK^{G'}(A,B)$.\\

Moreover, the association $(E,\pi)\mapsto (\phi^*E, \pi')$ is functorial at the level of $C^*$-correspondences.

\begin{definition}
$G$ and $G'$ are said to be Morita equivalent when there are two generalized morphisms $Z : G\rightarrow G'$ and $Z' : G'\rightarrow G$ such that their compositions are cohomologous to identities.  
\end{definition}

If $H$ is a subgroupoid of $G$, $G$ naturally carries left and right actions of $G/H\rtimes G$ and $H$, giving rise to two generalized morphisms $\phi : G/H\rtimes G\rightarrow H$ and $\psi : H \rightarrow G/H\rtimes G$, which are inverse of each others ($\phi \psi$ and $\psi\phi$ are cohomologous to the identity). In one sentence, $G/H\rtimes G$ and $H$ are Morita equivalent.\\

If $H$ is compact, one can also see $G$ as a $H$ space, and consider the generalized inclusion  
\[\iota : 
\begin{tikzcd}[column sep = small] 
H \arrow{d} \arrow{d} & \arrow{dl} G \arrow{dr} & G\arrow{d} \arrow{d} \\
 H^{(0)} & & G^{(0)}
\end{tikzcd}\]
and $\eta$ the opposite of $\iota$.  
\begin{definition}
The restriction and induction functors are defined by
\[ \text{Res}_H^G = \iota^*;\quad \text{Ind}_H^G = \eta^*.\]
These are functorial w.r.t. $C^*$-correspondences, and also w.r.t. $KK$-theory.
\end{definition}

The Morita equivalence $G/H\rtimes H \sim H$ allows us to give a very concrete definition of these functors, and actually help to prove that, when $H$ is a compact subgroupoid of $G$ and the action of $G$ is induced by $H$ , $\text{Res}_{H}^G : KK^G(A,B)\rightarrow KK^H( \text{Res}_{H}^G A , \text{Res}_{H}^G B)$ is an isomorphism of abelian groups for any $G$-algebras $A$ and $B$. This is locally satisfied for example when $G$ is ample, i.e. étale with totally disconnected base space.

\begin{definition}
The groupoid $G$ is said to be locally induced if there exists an open cover $\mathcal U$ of $G^{(0)}$ such that for all $U\in\mathcal U$, there exists a compact groupoid $H_U$ acting on $G_{|U}$ and an ismomorphism of groupoids $G_{|U }\simeq G_{|U}/ H_U \rtimes H_U$.  
\end{definition}

Since $G_{|U}/ H_U \rtimes H_U$ is Morita equivalent to $H_U$ and $G$ is Morita equivalent to the pull-back $G[\mathcal U]$, we have an isomorphism 
\[KK^G(A,B) \simeq \prod_{U\in\mathcal U} KK^{H_U}(\phi^* A,\phi^* B).\]

Ample groupoids are example of locally induced groupoid. Recall that an étale groupoid is ample iff the open compact bisections form  a basis of the groupoid. Now take $U$ an open compact bisection, and set $H_U = \{ g\in G\text{ s.t. } gU\cap U \neq \emptyset \}$, which is a compact subgroupoid of $G$ that fulfills the condition.\\

The author presently doesn't kow of any other examples of locally induced groupoids which are not ample. Maybe looking into groupoids with finite asymptotic dimension. 
	%
%%%%%%%%%%%%%%%%%%%%%%%%%%%%%%%%%%%%%%%%%%%%%%%%
\subsection{Induction and Restriction functors}
%%%%%%%%%%%%%%%%%%%%%%%%%%%%%%%%%%%%%%%%%%%%%%%%

\begin{definition} A subset $H\subseteq G$ is called a subgroupoid if :
\begin{itemize}
\item[$\bullet$] for every $x\in G^{(0)}$, $e_x\in H$,
\item[$\bullet$] for all $h,h'\in H $ such that $s(h') = r(h)$, $h'h \in H$,
\item[$\bullet$] if $h\in H$, $h^{-1}\in H$.
\end{itemize}
Then, the restriction of the multiplication, inverse, unit, target and source maps on $H$ defines a structure of groupoid on $H$ over $H^{(0)} = s(H)$. If $G$ is étale, $H$ is also étale. We will write $H< G$ to indicate that $H$ is a subgroupoid of $G$.
\end{definition}

In this section, we define for all subgroupoids $H < G$ induction and restriction transformations. Let $G$ be an étale groupoid and $H<G$.\\

Let $A$ be a $H$-algebra, with action given by $\alpha : s^*A \rightarrow r^* A$. Put $\phi=r\circ \iota$ where $\iota : H^{(0)}\hookrightarrow G^{(0)}$ is the canonical inclusion, hence $\phi^*A = A\otimes_\phi C_0(G)$ is a $C(G)$-algebra . Define the induced $C(G^{(0)})$-algebra :
%\[\text{Ind}_H^G (A) = \{f \in C_0(G,A) \text{ s.t. } h^{-1} f(gh) = f(g),\forall h\in H\}. \]
\[\text{Ind}_H^G (A) = \{f \in \phi^* A \text{ s.t. } \alpha_{h^{-1}}(f_{hg}) = f_g,\forall h\in H,g\in G^{s(h)}\}. \]
Left translation defines an action of $G$ on $\text{Ind}_H^G A$, so that $\text{Ind}_H^G A$ is a $G$-algebra.\\

Let $E$ be a Hilbert $A$-module, endowed with an action $V\in\mathcal L_A(s^*E,r^*E)$ of $H$. Define the induced Hilbert module :
%\[\text{Ind}_H^G (E) = \{f \in C_0(G,E) \text{ s.t. } h^{-1} f(gh) = f(g),\forall h\in H\}. \]
\[\text{Ind}_H^G (E) = \{f \in \phi^* E \text{ s.t. } \alpha_{h^{-1}}(f_{hg}) = f_g,\forall h\in H,g\in G^{s(h)}\}. \]
Left translation defines an action of $G$ on $\text{Ind}_H^G E$, hence $\text{Ind}_H^G E$ is a $G$-equivariant Hilbert $\text{Ind}_H^G A$-module.

Let $A$ and $B$ be two $H$-algebras. For all $(E,\pi,T)\in \mathbb E^G(A,B)$, define $\text{Ind}_H^G \pi$ and $\text{Ind}_H^G T$ as pointwise evaluation and multiplication by $\pi$ and $T$, i.e. $\text{Ind}_H^G T = T\otimes_\phi 1$ and $\text{Ind}_H^G \pi  = \pi\otimes_\phi id$.

\begin{definition}
For all subgroupoids $H<G$, let $A$, $B$ and $D$ be $G$-algebras and $A'$, $B'$ and $D'$ be $H$-algebras. Then, the map $(E,\pi,T)\mapsto ( \text{Ind}_H^G E, \text{Ind}_H^G\pi ,\text{Ind}_H^G T )$ induces an even homomorphism of $\Z_2$-graded abelian groups
\[\text{Ind}_H^G : KK_*^H(A',B')\rightarrow KK_*^G( \text{Ind}_H^G A', \text{Ind}_H^G B') \] 
called the induction transformation.\\
Moreover, by forgetting the action, we naturally have an even homomorphism of $\Z_2$-graded abelian groups 
\[\text{Res}_H^G : KK_*^G(A,B)\rightarrow KK_*^H( \text{Res}_H^G A, \text{Res}_H^G B) \] 
called the restriction transformation by restricting the action.\\
These two transformations respect the Kasparov product, i.e. 
\[ \text{Ind}_H^G(z\otimes z') = \text{Ind}_H^G(z)\otimes_{\text{Ind}_H^G(D')}\text{Ind}_H^G(z')\quad \forall z\in KK^H(A',D'),z'\in KK^H(D',B') \]
and 
\[ \text{Res}_H^G(z\otimes z') = \text{Res}_H^G(z)\otimes_{\text{Res}_H^G(D)}\text{Res}_H^G(z')\quad \forall z\in KK^G(A,D),z'\in KK^G(D,B) \]
The reader is refered to \cite{LeGall} for a proof.
\end{definition}   

Let $Z$ be a right $H$-space. Define on $Z\times_{p,r} G$ the following equivalence relation :
\[(z,g)\sim_H (z.h, h^{-1}g)\quad \forall z\in Z, h\in H^{p(z)},g\in G^{p(z)}.\]

\begin{definition}
The induced $G$-space of a $H$-space $Z$ is defined as $Z\times_H G = (Z\times G) / \sim_H$. 
\end{definition}

Notice that we have a natural identification between $\text{Ind}_H^G C_0(Z)$ and $C_0(Z\times_H G)$.\\

\begin{lem} \label{Restriction} Let $H$ be an open subgroupoid of $G$, and $V$ a $H$-space such that the anchor map $p : V\rightarrow H^{(0)}$ is locally injective. Then, for every $H$-algebra $A$ and every $G$-algebra $B$, the transformations $Res_H^G$ and $Ind_H^G$ induce an isomorphism of $\Z_2$-graded abelian groups :
\[RK^G( G\times_H V, B) \cong RK^H(V, Res_H^G B).\]
\end{lem}

\begin{dem}
It is clear that induction followed by restriction is the identity. For the converse, let $(E,\pi,T)\in \mathbb E^G(C_0(G\times_H V),B)$. The moment map is locally injective, hence, by lemma \ref{JLTform}, we can suppose that $T$ is self-adjoint $G$-equivariant and commutes with the action of $C_0(G\times_H V)$. As $H$ is open in $G$, $x\mapsto (e_{p(x)},x)$ is a topological embedding and $V$ can be seen as a $H$-invariant open subset in $G\times_H V$. Denote by $E_V$ the Hilbert $H$-invariant $B$-submodule of $E$ generated by 
\[\{\pi(f)\xi \ ,f\in C_0(V), \xi\in E\}.\]
Then, $E = \oplus_{g\in G/H} E_{gV}$. Notice that $E_{V}$ is a $H$-equivariant Hilbert $Res_H^G(B)$-module, such that $E\cong Ind_H^G (E_V)$. Moreover, $\pi$ is $G$-equivariant, hence $\pi(a) = Ind_H^G (\pi(a)_{|E_V} )$. As $[T,\pi(a)]=0$ for every $a\in C_0(G\times_H V)$, $T(E_V)\subseteq E_V$, and by $G$-equivariance, $T_{|E_V}$ determines $T$. Hence $T= Ind_H^G (T_{|E_V})$. Hence, if $z=[E,\pi,T]$ and $z_H =[E_V,\pi(a)_{|E_V},T_{|E_V}]$, we proved that $z = Ind_H^G( z_H)$, hence $Ind_H^G \circ Res_H^G (z)= z$.\\
\qed  
\end{dem}

%\begin{dem} 
%Let $A$ be a $H$-algebra, and $B$ a $G$-algebra.
%Let us first notice that the statement holds for equivariant $*$-homomorphisms. Namely,\\

%\begin{itemize} 
%\item[$\bullet$] if $\phi : Ind_H^G(A) \rightarrow B$ is a $G$-equivariant $*$-homomorphism, then $A = A\otimes C_0(G^{(0)})$ is a $H$-invariant subalgebra of $Ind_H^G(A)$, and the restriction $\phi_H$ of $\phi$ to $A$ is a $H$-equivariant $*$-homomorphism ;   
%\item[$\bullet$] if $\psi : A \rightarrow Res_H^G(B)$ is a $H$-equivariant $*$-homomorphism, then $\psi_G = \psi \otimes id_{C_0(G)}$ is a $H$-equivariant $*$-homomorphism.\\
%\end{itemize}

%And these constructions satisfy $(\phi_H)_G = \phi$ and $(\psi_G)_H = \phi$. Moreover, in $KK$-theory, one has $Res_H^G([\phi]) = [\phi_H]\in KK^H(Ind_H^G(A), B)$ and $Ind_H^G([\psi]) = [\psi_H]\in KK^H( A,Res_H^G (B))$.\\

%Let $z\in KK^H(Ind_H^G(A), B)$. As the restriction and induction transformations respect Kasparov products, by property $(d)$ and naturality of the boundary map, we can suppose that $z$ is the inverse in $KK$-theory of a $H$-equivariant $*$-homomorphism. Hence there exists a $H$-equivariant $*$-homomorphism $\phi : B \rightarrow Ind_H^G(A)$ such that $z\otimes_B [\phi] = 1_{Ind_H^G(A)}$ and $[\phi]\otimes_{Ind_H^G(A)} z  = 1_B$. Taking the induction, we get $Ind_H^G(z) = Ind_H^G([\phi])^{-1} = [\phi_G]^{-1}$, and $Res_H^G \circ Ind_H^G(z) = z$. We can similarly prove that $Ind_H^G \circ Res_H^G (z') = z'$ for every $z\in KK^G(A,Res_H^G(B))$.\\
%\qed
%\end{dem}

%\begin{prop} Let $H$ be an open compact subgroupoid of $G$, $U$ a $H$-space and $B$ a $H$-algebra. Then :
%\[Res_H^G : RK^G(G\times_H U , B)\rightarrow RK^H(U , Res_H^G(B)) \]
%is an isomorphism of $\Z_2$-graded abelian groups.
%\end{prop}

\subsection{Strongly proper groupoids}

We now introduce a property on groupoids that will entail a nice result on induction and restriction transformations at the level of $K$-homology.

\begin{definition}\label{StronglyProper}
A groupoid $G$ is said to be strongly proper if there exists an open cover $\mathcal U$ of $G^{(0)}$ such that, for all $U\in\mathcal U$, there exists a compact open subgroupoid $H_U$ of $G$ and a $H_U$-space $Z_U$ together with a $G$-equivariant homeomorphism
\[\psi_U : U \rightarrow G\times_{H_U} Z_U.\] 
An action of $G$ on a space $Z$ is said to be strongly proper if the groupoid $Z\rtimes G$ is strongly proper. A groupoid is said to be in the class $\mathcal C$ if every proper action of $G$ is strongly proper.
\end{definition}

\begin{rk}
For any strongly proper action of $G$ on a space $Z$, there exists an open cover of $Z$ by subsets of the type $V=G\times_H U$, where $H$ is a compact open subgroupoid and $U$ is a $H$-space. Then, by the previous section, we have an isomorphism
\[RK^G(V,B)\cong RK^H(U, Res_H^G (B))\]
for every $G$-algebra $B$. 
\end{rk}
%\begin{prop}
%Let $G$ be a strongly proper groupoid, and $\mathcal U$ an open cover satisfying the conditions of definition \ref{StronglyProper}. Then, the $G$-map $\psi_U $ extends to an isomorphism of groupoids
%\[\begin{tikzcd}
% G_{|U}\arrow{r}\arrow[shift right]{d}\arrow[shift left]{d} & Z_U\rtimes H_U \arrow[shift right]{d}\arrow[shift left]{d} \\
% U \arrow{r}{\psi_U} & G\times_{H_U} Z_U 
%\end{tikzcd}.\]
% Ce morceau bug sous windows : pourquoi ?
%\end{prop}

Let us give examples of groupoids in class $\mathcal C$. Recall the following definition from \cite{Renault} (page $20$).

\begin{definition}
A topological groupoid is said to be ample if it has a basis $G^a$ of neighborhoods consisting of compact open susbets.
\end{definition}

In \cite{paterson} (page $17$) is stated the following property. An étale groupoid $G$ is ample iff $G^{(0)}$ is totally disconnected. Hence the coarse groupoid of every coarse space $X$ is ample, its basis being $\beta X$.

\begin{prop}
Every ample groupoid is in class $\mathcal C$.
\end{prop}

\begin{dem} The following argument was explained to me by Christian Bönicke.\\ 
Let $G$ be an étale ample groupoid and $Z$ a $G$-space with proper action of $G$. Since $G$ is ample, we can cover $Z$ by compact open subsets. Let $\mathcal U$ be such an open cover. For each $U\in \mathcal U$, there exists $V\subseteq Z$ compact open such that $U= G.V$. Put $H= (r\times s)^{-1}(V,V)$, which is, by properness, a compact open subgroupoid of $Z\rtimes G$. Moreover, $U\cong G\times_H p(V)$ $G$-equivariantly. Hence $Z$ is covered by open subsets of the form $G\times_H V$, with $H$ being compact open subgroupoids and $V$ being $H$-spaces.\\ 
\qed
\end{dem}

\begin{rk}
Let $\Gamma$ be a discrete group. Then every proper action of $\Gamma$ on a space $Z$ is strongly proper by definition.
\end{rk}


%The author presently doesn't know of any other examples of strongly proper groupoids which are not ample.% Maybe looking into groupoids with finite asymptotic dimension. 

%Ample groupoids are example of locally induced groupoid. Recall that an étale groupoid is ample iff the open compact bisections form  a basis of the groupoid. Now take $U$ an open compact bisection, and set $H_U = \{ g\in G\text{ s.t. } gU\cap U \neq \emptyset \}$, which is a compact subgroupoid of $G$ that fulfills the condition.\\

%\begin{definition}
%The groupoid $G$ is said to be locally induced if there exists an open cover $\mathcal U$ of $G^{(0)}$ such that for all $U\in\mathcal U$, there exists a compact groupoid $H_U$ acting on $G_{|U}$ and an ismomorphism of groupoids $G_{|U }\simeq G_{|U}/ H_U \rtimes H_U$.  
%\end{definition}

%Since $G_{|U}/ H_U \rtimes H_U$ is Morita equivalent to $H_U$ and $G$ is Morita equivalent to the pull-back $G[\mathcal U]$, we have an isomorphism 
%\[KK^G(A,B) \simeq \prod_{U\in\mathcal U} KK^{H_U}(\phi^* A,\phi^* B).\]




 %
%%%%%%%%%%%%%%%%%%%%%%%%%%

%%%%%%%%%%%%%%%%%%%%%%%%%%%%%%%%%%%%%%%%%%%%%%%%%%%%
\subsection{Baum-Connes and the Künneth formula}
%%%%%%%%%%%%%%%%%%%%%%%%%%%%%%%%%%%%%%%%%%%%%%%%%%%%
The first step in proving theorem \ref{Kunneth} is to define a analytical version of $\alpha_{A,B}$ when an étale groupoid $G$ is given. More precisely, we first construct a homomorphism $\alpha_{A,B}^G : K_*^{top}(G,A)\otimes K_*(B)\rightarrow K_*^{top}(G,A\otimes B )$, inductive limit of $\alpha_{A,B}^{G,Z} : RK^G(Z,A)\otimes K_*(B)\rightarrow RK^G(Z,A\otimes B )$ where $Z$ runs through $G$-proper $G$-compact spaces. Then we show that the controlled assembly map intertwines $\alpha^{G,P_E(G)}_{A,B}$ and $\alpha_{A\rtimes_r G,B}$ uniformly.\\

If $A$ is a $G$-algebra, and $B$ a $C^*$-algebra, $A\otimes B$ naturally inherits a $G$-algebra structure with trivial action of $G$ on the $B$ factor. Then $(A\rtimes_r G)\otimes B \cong (A\otimes B)\rtimes_r G$.\\

Let $Z$ be a $G$-proper space. Define the homomorphism :
\[\alpha_{A,B}^{G,Z} : RK^G_*(Z,A)\otimes K_*(B)\rightarrow RK_*^G(Z,A\otimes B) \quad ; \quad (x,y)\mapsto x\otimes_{}   \tau_A(y),\]
which respects inductive limits w.r.t. inclusions of $G$-proper spaces, so that it induces
\[\alpha_{A,B}^G : K_*^{top}(G,A)\otimes K_*(B)\rightarrow K_*^{top}(G,A\otimes B ).\]

To prove theorem \ref{Kunneth}, we will need the following result.

\begin{thm}\label{TopologicalKunneth}
Let $G$ be an étale groupoid in the class $\mathcal C$, and let $E\in\mathcal E$ be a controlled subset of $G$ and $P_E(G)$ be the corresponding Rips complex. If for all compact open subgroupoids $H$ of $G$, $A\rtimes_r H$ satisfies the Künneth formula, then $\alpha_{A,B}^{G,P_E(G)}$ is an isomorphism for all $C^*$-algebras $B$ such that $K_*(B)$ is a free abelian group.
\end{thm}

\begin{dem}
Let $Z_0\subseteq Z_1\subseteq ... \subseteq Z_n $ be the skeleton decomposition of $P_E(G)$.\\

Let us prove by induction that $\alpha_j=\alpha^{G,Z_j}_{A,B}$ is an isomorphism. By a standard argument similar to the proof of theorem \ref{prod}, it is sufficient to prove the statement for $j=0$.\\

Let $U\subseteq Z_0$ be a $G$-compact $G$-proper space. By strong properness, $U$ can be finitely covered by open subsets of the type $G \times_H V$. By a standard Mayer-Vietoris argument, we can suppose that there exists a compact open subgroupoid $H$ of $G$ and a compact $H$-space $V$ such that $U = G \times_H V$. The following diagram is commutative :
\[\begin{tikzcd}[column sep = small] RK_*^G(U,A)\otimes K_*(B) \arrow{r}{\alpha_{A,B}^{G,U}} \arrow{d}{\text{Res}_H^G} & RK^G(U, A\otimes B) \arrow{d}{\text{Res}_H^G}\\
RK_*^H(V, \text{Res}_H^G (A))\otimes K_*(B) \arrow{r}{\alpha_{\text{Res}_H^G A,B}^{H,V}} \arrow{d}{\mu_{H,\text{Res}_H^G A \otimes id}} & 
	RK_*^H(V, \text{Res}_H^G (A)\otimes B)\arrow{d}{\mu_{H,\text{Res}_H^G A }} \\
K_*(\text{Res}_H^G (A)\rtimes_r H)\otimes K_*(B)\arrow{r}{\alpha_{\text{Res}_H^G A,B}} & K_*\left((\text{Res}_H^G A\otimes B)\rtimes_r H\right) 
\end{tikzcd}\]
The first vertical arrows are isomorphisms by Proposition \ref{Restriction}, the last horizontal one is by hypothesis. The last vertical ones are by compactness of $H$, hence $\alpha_{A,B}^{G,U}$ is an isomorphism. By taking the inductive limit on $G$-proper $G$-compact spaces $U\subseteq Z_0$, we get that $\alpha_{A,B}^{G,Z_0}$ is an isomorphism.
\end{dem}

\begin{lem}\label{KunnethLemma}
There exists a positive number $\kappa$ and a function $c : (0,\frac{1}{4\kappa})\times\mathcal E\rightarrow \mathcal E$, decreasing in $E$ and non increasing in $\varepsilon$ such that, for any étale groupoid $G$, any filtered $G$-algebra $(A,\mathcal E)$, and any $C^*$-algebra $B$, the following diagram commutes :
\[\begin{tikzcd}
RK_*^G(P_E(G),A)\otimes K_*(B) \arrow{r}{\alpha^{G,P_E(G)}_{A,B}}\arrow{d}{\mu_{G,A}^{\varepsilon,E,F}\otimes \text{id}_{K_*(B)}} & 
RK_*^G(P_E(G),A\otimes B) \arrow{d}{\mu_{G,A}^{\alpha_ \tau\varepsilon,E,h_\varepsilon F}} \\
K_*^{\varepsilon,F}(A\rtimes_r G)\otimes K_*(B) \arrow{r}{\alpha_{A\rtimes_r G,B}^{\varepsilon,F}} & 
K_*^{\alpha_\tau\varepsilon,h_{\varepsilon}F}((A\otimes B)\rtimes_r G)\\
\end{tikzcd}\] 
for all $\varepsilon\in(0,\frac{1}{4\kappa})$ and $E,F\in\mathcal E$ such that $F\geq c(\varepsilon,E)$. \\
\qed
\end{lem}

\begin{dem}
Let $z\in RK_*^G(P_E(G),A)$ and $y\in K_*(B)$. As the action of $G$ on $B$ is trivial, $(A\rtimes_r G)\otimes B\simeq (A\otimes B)\rtimes_r G$ so that $\hat\tau_A(y)\circ \hat J_{G}(z) \sim \hat J_{G}(z)\circ \hat\tau_A(y) $ holds. This entails
\[\hat J_G^{\alpha_\tau\varepsilon, h_{\tau,\varepsilon}E}(z)\left([\mathcal L_E,0]_{\varepsilon, E} \right)\otimes \tau_{A\rtimes_r G}(y) = 
\left( \hat\tau(y)\circ \hat J_G(z) \right)^{\varepsilon, E}\left([\mathcal L_E,0]_{\varepsilon, E}\right),\]
which is just the statement of the lemma.\\
\qed
\end{dem}

%\textbf{SHOW THAT $\hat J_G(z)\circ \hat\tau(y)=$}

\begin{dem} \textit{of theorem \ref{Kunneth}}\\
Let $\kappa>0$ and $c : (0,\frac{1}{\kappa})\times\mathcal E\rightarrow \mathcal E$ as in lemma \ref{KunnethLemma}.\\

Let $A$ be a $G$-algebra and $B$ a $C^*$-algebra, seen as a $G$-algebra with trivial action. As $(A\otimes B)\rtimes_r G \simeq (A\rtimes_r G)\otimes B$, we will identify them in the remaining of the proof. Let us prove that $\hat \alpha_{A\rtimes_r G,B}$ is a controlled isomorphism.\\

Let $F\in\mathcal E$ and $\varepsilon\in (0,\frac{1}{4})$. Choose $E \in\mathcal E$ such that $k_J(\varepsilon).E\subseteq F$. By controlled surjectivity, there exists $F'\in \mathcal E$ such that $QS_{G,A\otimes B}(\varepsilon,\lambda\varepsilon, F,F',E)$ holds.\\

Let $y\in K^{\varepsilon,F}_*((A\otimes B)\rtimes_r G)$. By controlled surjectivity of $\hat\mu_{G,A\otimes B}$, there exists $z\in RK_*^G(P_E(G), A\otimes B) $ such that $\mu_{G,A\otimes B}^{\kappa\varepsilon,F'}(z)=\iota_{\varepsilon,F}^{\kappa\varepsilon,F'}(y)$. Theorem \ref{TopologicalKunneth} entails that there exists $z_1 \in RK_*^G(P_E(G),A)\otimes K_*(B) $ such that $\alpha_{A,B}^{G,P_E(G)}(z_1)=z$, and lemma \ref{KunnethLemma} shows that, if $z_2=(\mu_{G,A}^{\varepsilon,E,F} \otimes id_{\hat K_*(B)})(z_1)$, then $\alpha_{A\rtimes_r G,B}^{\varepsilon,F} (z_2)=\iota_{\varepsilon,F}^{\alpha_\tau\varepsilon,h_\varepsilon F}(y)$.\\

Let us show the controlled injectivity. Let $\varepsilon\in(0,\frac{1}{4})$ and $F\in\mathcal E$.\\ 

Let $x\in K^{\varepsilon,F}_*(A\rtimes_r G)\otimes K_*(B)$ such that $\alpha_{A\rtimes_r G,B}^{\varepsilon,F}(x)=0$ in $K^{\alpha_\tau,h_{\tau,\varepsilon} F}((A\otimes B)\rtimes_r G)$. By controlled surjectivity of $\hat\mu_{G,A}$, there exists $z\in RK(P_E(G),A)\otimes K_*(B)$ such that $\mu_{G,A}^{\alpha_\tau,h_{\tau,\alpha_\tau\varepsilon} F}(z)=(\iota_{\varepsilon,F}^{\varepsilon',F'}\otimes id) (x)$. If $z' = \alpha_{A,B}^{G,P_E(G)}(z)$, lemma \ref{KunnethLemma} entails that $\mu^{E}_{G,A\otimes B}(z)=0$, which by controlled injectivity implies that $q_E^{E'}(z')=0$ in $RK^G(P_{E'}(G),A\otimes B)$. As $\alpha_{A,B}^{G,P_E(G)}$ is an isomorphism, $q_E^{E'}(z)=0$ in $RK^G(P_{E'}(G),A)\otimes K(B)$.\\
\qed
\end{dem}




























