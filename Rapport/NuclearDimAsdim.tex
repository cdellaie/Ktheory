\section{Nuclear dimension}

In their paper \textbf{reference !}, E. Guentner, R. Willett and G. Yu defined asymptotic dimension for étale groupoids, and the theorem which we are intersted in is the following.

\begin{thm}
Let $G$ be an étale groupoid with finite asymptotic dimension, and with base space $G^{(0)}$ of finite covering dimension. If $G$ is free, then the following inequality holds :
\[\text{dim}^{+1}_{nuc}(C^*_rG)\leq \text{dim}^{+1}_{cov}(G^{(0)}).\text{asdim}^{+1}(G).\] 
\end{thm}

The article points out that freeness is mainly technical, and that one could somehow could get rid of it. That is what we entail to do. \\

The point of the proof is to construct, out of any compact subset $K\subset G$, an almost invariant partition of unity subordinate to a covering fulfilling the definition of asymptotic dimension. This PDU is use to construct a completely positive factorisation of the identity of $C^*_rG$ through the reduced $C^*$-algebra of the open relatively compact subgroupoids generated by restiction to the cover and intersection with $K$. Here freeness does not intervene. \textbf{VERIFY} \\

Freeness is only used for this paticular result.

\begin{prop}
Let $G$ a free étale groupoid, and $H$ an open relatively compact subgroupoid of $G$. Then 
\[\text{dim}_{nuc} C^*_r H = \text{dim}_{cov} H^{(0)}.\]
\end{prop}

Of course, the result should not hold without any assumption on $G$, more precisely, I think we should ask something about the isotropy bundle of $G$.\\

\begin{definition}
The isotropy bundle of $G$ is the closed subgroupoid $\mathcal J = (r\times s)^{-1}(\Delta)$, i.e. the group bundle of the stabilizers $\mathcal J = \cup G_x^x$. Here $\Delta\subset G^{(0)}\times G^{(0)}$ is the diagonal of the base space. 
\end{definition}

\textbf{Remark :} If $G$ is a (locally compact) transitive groupoid, then P. Muhly, J. Renault and D. Williams \textbf{ref !} have shown that $C^*_rG$ is isomorphic to $C^*_r H \otimes \mathfrak K(L^2(\mu))$ for $H$ any of the group statbilizers, which are all isomorphic, and $\mu$ a measure on $G^{(0)}$, so that
\[\text{dim}_{nuc}^{+1}(C_r^*G)\leq \text{dim}_{nuc}^{+1}(C_r^*H).\text{dim}_{nuc}^{+1}(\mathfrak K(L^2(\mu))).\]
by Prop. $2.3$ of \textbf{WZ}.

Let $H^{(0)}/H$ be the base space quotiented by the equivalence relation induced by $H$, and let $[x]=r(s^{-1}(x))$ denotes the equivalence class of $x\in H^{(0)}$, $\pi : H^{(0)}\rightarrow H^{(0)}/H$ the canonical projection map.\\  

\begin{prop}
Let $H$ be a compact étale groupoid. Then $X=H^{(0)}/H$ is compact and Hausdorff, moreover there exists a structure of $C(X)$-algebra on $C^*_rH$ with fibers isomorphic to $C^*_r(H_x^x)\otimes \mathfrak K(l^2([x]))$ for any $x\in H^{(0)}$.
\end{prop}

\begin{proof}
Define
\[\left\{\begin{array}{rcl} C(X) &\rightarrow & Z(\mathcal M(C^*_r H)) \\ f &\mapsto & f\circ \pi \end{array}\right. ,\]
which defines a $C(X)$-structure on $C^*_rH$.\\
Its fiber over $t\in X$ is given by the quotient by the ideal $C(H^{(0)}- t) C^*_r H $ i.e. $C^* H(t)$, as $H^{(0)}-t$ is an open $H$-invariant subset. But $H(t)$ is a principal groupoid so we can apply the remark to get $C^*_r H(t)\simeq C^*_r H_x^x \otimes \mathfrak K(l^2(t))$ where $x$ is any point of $t$.\\
\end{proof}

\begin{prop}
Let $H$ be a compact étale groupoid. \\
If, for all $x\in H^{(0)}$, $C^*_rH_x^x$ is a nuclear $C^*$-algebra, then $C^*_r H$ is nuclear.\\
If $\sup_{x\in H^{(0)}}\text{dim}_{nuc}(C^*_rH_x^x)\leq d$ then
\[\text{dim}_{nuc}(C^*_r H)\leq (\text{dim}_{cov}(H^{(0)})+1)(d+1)-1. \]
\end{prop}

\begin{proof}
\textbf{LATER}
\end{proof}







