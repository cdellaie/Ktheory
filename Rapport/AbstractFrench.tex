\section*{Résumé français de la thèse}

Cette thèse porte sur la $K$-théorie contrôlée, ou $K$-théorie quantitative, ainsi que ses applications en géométrie non-commutative, en particulier sur la conjecture de Baum-Connes.\\

La $K$-théorie contrôlée, introduite par H. Oyono-Oyono et G. Yu dans \cite{OY2}, est un raffinement de la $K$-théorie opératorielle. Elle permet de prendre en compte les phénomènes de propagation en $K$-théorie. Détaillons un peu notre propos.\\

Dans le cadre des $C^*$-algèbres, on dispose d'un foncteur à valeurs dans la catégorie des groupes abéliens $A\mapsto K_*(A)$, qui à une $C^*$-algèbre $A$ associe ses groupes de $K$-théorie $K_0(A)$ et $K_1(A)$. Calculer la $K$-théorie d'une $C^*$-algèbre est un problème important mais souvent difficile. Dans les années 90, P. Baum, A. Connes et N. Higson proposèrent dans \cite{BaumConnesHigson} de calculer la $K$-théorie de la $C^*$-algèbre réduite d'un groupe localement compact $G$ grâce à une application d'assemblage $\mu_G : K^{top}(G)\rightarrow K(C_r^*(G))$. Le membre de gauche est appelé la $K$-théorie topologique de $G$, et se calcule par des méthodes de topologie algébrique classiques. La conjecture de Baum-Connes affirme que l'application d'assemblage est un isomorphisme.\\

S'inspirant de cette conjecture, une application d'assemblage fut définie dans le cadre de la géométrie coarse. La géométrie coarse s'intéresse à une certaine classe d'espaces métriques, dont elle cherche à determiner les propriétés à grande échelle. Si l'on se donne un espace coarse $X$, on peut construire la $C^*$-algèbre de Roe $C^*(X)$. La $K$-théorie de $C^*(X)$ est le réceptacle pour les hautes signatures associées à des variétés complètes non nécessairement compactes quasi-isométriques à $X$. Pour la calculer, il est alors possible de définir une application d'assemblage coarse $\mu_X : KX(X)\rightarrow K(C^*(X))$. Le membre de gauche est appelé $K$-homologie coarse de $X$. La conjecture de Baum-Connes coarse affirme que cette application est un isomorphisme pour tout espace coarse à géometrie bornée.\\

Bien qu'ouverte dans le cas général, la conjecture de Baum-Connes a été démontrée pour une large classe de groupes. Un contre-exemple a été donné à la conjecture de Baum-Connes coarse, mais elle est vérifiée pour une grande famille d'espaces coarses. L'idée sous-jacente à la $K$-théorie contrôlée est déjà contenue dans la preuve de la conjecture de Baum-Connes coarse que donne G. Yu dans \cite{Yu1} pour les espaces coarses de dimension asymptotique finie. L'un des intérêts de cette démonstration est qu'elle évite des arguments analytiques de type Dirac-Dual-Dirac. Un objectif de la $K$-théorie contrôlée est d'extraire de cette preuve des méthodes générales qui permettent de calculer la $K$-théorie de $C^*$-algèbres autres que les algèbres de Roe d'espaces coarses de dimension asymptotique finie.\\

Par exemple, de larges classes de $C^*$-algèbres sont construites à partir :
\begin{itemize}
\item[$\bullet$] d'espaces métrique $(X,d)$ : les algèbres de Roe $C^*(X)$,
\item[$\bullet$] de groupes localement compacts $G$ : les $C^*$-algèbres réduite $C^*_r(G)$ et maximale $C^*_{max}(G)$, et les produits croisés $A\rtimes G$. Plus généralement, les mêmes notions existent pour les groupoïdes localement compacts avec système de Haar,
\item[$\bullet$] de groupes quantiques compacts $\mathbb G$ et de leur duals, les groupes quantiques discrets $\hat{\mathbb G}$ : à nouveau leurs $C^*$-algèbres réduites, maximales et leurs produits croisés.\\
\end{itemize}

Or, dans la plupart des cas, ces $C^*$-algèbres possèdent une structure supplémentaire appelée filtration. La première étape de notre travail a consisté en une généralisation de la notion de filtration, ce qui permet d'appliquer les techniques développées dans \cite{OY2} et \cite{OY3} à de nouveaux exemples. Dans ces articles, les auteurs définissent une $C^*$-algèbre filtrée de la manière suivante.

\begin{definitionfr}
Une $C^*$-algèbre $A$ est filtrée s'il existe une famille de sous-espaces vectoriels fermés auto-adjoints $\{A_R\}_{R>0}$ de $A$ telle que :
\begin{itemize}
\item[$\bullet$] si $R\leq R'$, alors $A_R\subseteq A_{R'}$,
\item[$\bullet$] pour tout $R,R'>0$, $A_R.A_{R'}\subseteq A_{R+R'}$,
\item[$\bullet$] $A$ est l'adhérence de l'union des sous-espaces $A_R$, i.e. $\overline{\cup_{R>0}A_R} = A$.
\item[$\bullet$] si $A$ est unitale, on a de plus $1\in A_R,\forall R>0$.
\end{itemize}
\end{definitionfr}

Nous proposons alors d'étendre cette définition en considérant des familles des sous-espaces vectoriels indexés par des structures plus générales que nous appelons structures coarses. 

\begin{definitionfr}
Une structure coarse $\mathcal E$ est un semi-groupe abélien qui est aussi un treillis. %is a lattice which is an abelian semi-group. %such that $\forall E,E'\in \mathcal E$, $E\leq E'E$. 
Rappelons qu'un treillis est un ensemble partiellement ordonné tel que toute paire $(E,E')$ admette un supremum $E\vee E'$ et un infimum $E\wedge E'$.
\end{definitionfr}

Si une structure coarse $\mathcal E$ est donnée, une $C^*$-algèbre $\mathcal E$-filtrée est définie comme dans la définition précédente, en remplaçant les nombres positifs par des éléments de la structure coarse, et l'addition par la composition. On montre alors comment la $K$-théorie contrôlée de \cite{OY2} s'adapte à ce cadre plus général.\\

De nouvelles $C^*$-algèbres peuvent alors être vues comme des $C^*$-algèbres filtrées. Voici des exemples qui sont détaillés par la suite.
\begin{itemize} 
\item[$\bullet$] Soit $(X,d)$ un espace métrique discret à géométrie bornée. Alors les ensembles $E\subseteq X\times X$ symmétriques tels que $\sup d_{|E}<\infty$, sont munis d'un ordre partiel $\subseteq$ et d'une loi de composition donnée par 
\[E\circ E = EE'\cup E'E,\]
où $EE' = \{(x,y)\in X\times X \text{ t.q. }\exists z\in X / (x,z)\in E \text{ et }(z,y)\in E'\}$. Cela définit une structure coarse $\mathcal E_X$.
\item[$\bullet$] Soit $G$ un groupoïde étale $\sigma$-compact. Alors les ensembles $E\subseteq G$ symmétriques compacts, sont munis d'un ordre partiel $\subseteq$ et d'une loi de composition donnée par 
\[E\circ E = EE'\cup E'E,\]
où $EE' = \{gg' ; (g,g')\in G^{(2)}\}$. Cela définit une structure coarse $\mathcal E_G$.
\item[$\bullet$] Soit $\mathbb G$ un groupe quantique compact. Alors l'ensemble des classes d'équivalence de représentations unitaires symétriques de dimension finie de $\mathbb G$ est muni de l'ordre suivant : $\pi\leq \pi'$ ssi $\pi$ est unitairement équivalente à une sous-représentation de $\pi$. De plus, le produit tensoriel symétrisé fournit une loi de composition, et cela définit une structure coarse $\mathcal E_{\mathbb G}$.\\
\end{itemize} 

Il s'avère que les algèbres de Roe $C^*(X,B)$, les produits croisés par $G$, et les produits croisés par $\hat{\mathbb G}$ sont filtrés par $\mathcal E_X$, $\mathcal E_G$ et $\mathcal E_{\mathbb G}$ respectivement.\\

%La première partie de la thèse rappelle les notions nécessaire à la compréhension du texte. Dans la seconde partie, nous développons ce que nous appelons %%%%%%%%%%%%%%%%%%%%%%%%%%%%%%%%%%%%%%%%%%

Dans un premier temps, nous traitons la construction d'applications d'assemblage à valeurs dans la $K$-théorie contrôlée, dans le cas des espaces coarses et dans celui des groupoïdes étales. Une propriété importante est que ces applications factorisent les applications d'assemblage classiques. Cela permet de relier les conjectures de Baum-Connes et de Baum-Connes coarse aux propriétés de nos applications. Ces dernières sont définies en deux étapes. La première est de définir ce que l'on appelle un foncteur de descente. Puis, on définit le complexe de Rips associé à $X$ et à $G$, et un certain projecteur canonique associé. \\

Plus précisément, soit $(X,d)$ un espace métrique dénombrable discret, que l'on suppose à géométrie bornée; i.e. $\forall R>0,\sup | \{ (x,y) \in X\times X \text{ t.q. }d(x,y)<R\} |<\infty$. Nous construisons d'abord la transformation de Roe contrôlée $\hat\sigma_X(z)$ pour tout élément $z\in KK(A,B)$. Les propositions \ref{Roe1} et \ref{Roe2}, qui décrivent ses propriétés, peuvent être résumées dans la proposition suivante :

\begin{propfr}
Soient $A$ et $B$ deux $C^*$-algèbres. Pour tout $z\in KK_*(A,B)$, il existe une paire de contrôle $(\alpha_X,k_X)$ et un morphisme $(\alpha_X,k_X)$-contrôlé
\[\hat\sigma_X(z) : \hat K(C^*(X,A))\rightarrow \hat K(C^*(X,B))\]
de même degré que $z$, tels que :
\begin{enumerate}
\item[(i)] $\hat\sigma_X(z)$ induit la multiplication à droite par $\sigma_X(z)$ en $K$-théorie ;
\item[(ii)] $\hat\sigma_X$ est additif, i.e.
\[\hat\sigma_X(z+z')=\hat\sigma_X(z)+\hat\sigma_X(z').\]
\item[(iii)] Pour tout $*$-homomorphisme $f : A_1\rightarrow A_2$,
\[\hat\sigma_X(f^*(z))=\hat\sigma_X(z)\circ f_{X,*}\] pour tout $z\in KK_*(A_2,B)$.
\item[(iv)] Pour tout $*$-homomorphisme $g : B_1\rightarrow B_2$,
\[\hat\sigma_X(g_*(z))= g_{X,*}\circ \hat\sigma_X(z)\] pour tout $z\in KK_*(A,B_1)$.
\item[(v)] Soient $0\rightarrow J\rightarrow A\rightarrow A/J\rightarrow 0$ une extension semi-scindée de $C^*$-algèbres et $[\partial_J]\in KK_1(A/J,J)$ son bord en $KK$-théorie. Alors : 
\[\hat\sigma_X([\partial_{J,A}])=D_{C^*(X,J),C^*(X,A)}.\] 
\item[(vi)] $\hat\sigma_X([id_A]) \sim_{(\alpha_X,k_X)} id_{\hat K(C^*(X,A))}$
\end{enumerate}
\end{propfr}

Une fois la transformation de Roe contrôlée construite, l'étape suivante naturelle est la construction de l'application d'assemblage contrôlée $\hat\mu_{X,B}$ , pour toute $C^*$-algèbre $B$. Pour cela, pour tout entourage $E$, on définit une projection canonique $P_E\in C^*(X,C_0(P_E(X)))$ de propagation finie. Elle définit alors une classe paire de $K$-théorie contrôlée.

\begin{definitionfr}
Soient $B$ une $C^*$-algèbre, $\varepsilon\in (0,\frac{1}{4})$ et $E,F\in\mathcal E_X$ des entourages tels que $k_X(\varepsilon).E\subseteq F$. L'application d'assemblage contrôlée $\hat\mu_{X,B}=(\mu_{X,B}^{\varepsilon,E,F})_{\varepsilon,E}$ est définie comme la famille d'applications
\[\hat\mu_{X,B}^{\varepsilon, E,F} :\left\{\begin{array}{rcl} KK(C_0(P_E(X)),B) & \rightarrow & K^{\varepsilon, F}(C^*(X,B)) \\
					z & \mapsto & \iota_{\alpha_X \varepsilon',k_X(\varepsilon').F'}^{\varepsilon,F}\circ\hat\sigma_X(z)[P_{E},0]_{\varepsilon', F'}\end{array}\right.\]
où $\varepsilon'$ et $F'$ vérifient :
\begin{itemize}
\item[$\bullet$] $\varepsilon'\in (0,\frac{1}{4})$ tel que $\alpha_X \varepsilon'\leq \varepsilon$,
\item[$\bullet$] et $F'\in\mathcal E$ tel que $E\subseteq F'$ et $k_X(\varepsilon').F'\subseteq F$.
\end{itemize}
\end{definitionfr}

Nous définissons des applications analogues pour les groupoïdes étales. Plus précisément, soit $G$ un groupoïde étale $\sigma$-compact. Nous montrons que les ensembles $E\subseteq G$ compacts symmétriques forment une structure coarse $\mathcal E$, et que les produits croisés réduits de $G$ sont filtrés par $\mathcal E$. La première étape est à nouveau de construire un analogue du foncteur de descente en $K$-théorie contrôlée. Les propositions \ref{Kasparov1} et \ref{Kasparov} peuvent être résumées dans la proposition suivante. 

\begin{propfr}
Soient $A$ et $B$ deux $G$-algèbres. Pour tout $z\in KK^G_*(A,B)$, il existe une paire de contrôle $(\alpha_J,k_J)$ et un morphisme $(\alpha_J,k_J)$-contrôlé
\[J_{red,G}(z) : \hat K(A\rtimes_r G)\rightarrow \hat K(B\rtimes_r G)\]
de même degré que $z$, tels que
\begin{enumerate}
\item[(i)] $J_{red,G}(z)$ induit la multiplication à droite par $j_{red,G}(z)$ en $K$-théorie ;
\item[(ii)] $J_{red,G}$ est additif, i.e.
\[J_{red,G}(z+z')=J_{red,G}(z)+J_{red,G}(z').\]
\item[(iii)] Pour tout $G$-morphisme $f : A_1\rightarrow A_2$,
\[J_{red,G}(f^*(z))=J_{red,G}(z)\circ f_{G,red,*}\] pour tout $z\in KK_*^G(A_2,B)$.
\item[(iv)] Pour tout $G$-morphisme $g : B_1\rightarrow B_2$,
\[J_{red,G}(g_*(z))= g_{G,red,*}\circ J_{red,G}(z)\] pour tout $z\in KK_*^G(A,B_1)$.
\item[(v)] Soient $0\rightarrow J\rightarrow A\rightarrow A/J\rightarrow 0$ une extension semi-scindée $G$-équivariante de $G$-algèbres et  $[\partial_J]\in KK_1^G(A/J,J)$ son bord en $KK^G$-théorie. Alors 
\[J_G([\partial_J])=D_{J\rtimes_r G,A\rtimes_rG}.\] 
\item[(vi)] $J_{red,G}([id_A]) \sim_{(\alpha_J,k_J)} id_{\hat K(A\rtimes G)}$
\end{enumerate}
\end{propfr}

Cette transformation, que nous appelons transformation de Kasparov contrôlée, permet de construire l'application d'assemblage contrôlée $\hat\mu_{G,B}$ pour toute $G$-algèbre $B$. Pour tout ensemble contrôlé $E\subseteq G$, on dispose d'un projecteur $\mathcal L_E\in C_0(P_E(G))\rtimes_r G$ qui est de propagation finie. Il définit donc une classe de $K$-théorie contrôlée.

\begin{definitionfr}
Soient $B$ une $G$-algèbre, $\varepsilon\in (0,\frac{1}{4})$, et $E\in\mathcal E$. Soit $F\in \mathcal E$ tel que $k_J(\varepsilon).E \subseteq F$. L'application d'assemblage contrôlée pour $G$ est définie comme la famille d'applications :
\[\mu_{G,B}^{\varepsilon,E,F}\left\{
\begin{array}{rcl}
RK^G(P_E(G), B) & \rightarrow & K_*^{\varepsilon, F}(B\rtimes_r G)\\
z & \mapsto & \iota_{\alpha_J\varepsilon', k_J(\varepsilon').F'}^{\varepsilon,F} \circ J_G^{\varepsilon', F'}(z)([\mathcal L_E,0]_{\varepsilon' , F'})
\end{array}\right.\]
où $\varepsilon'$ et $F'$ vérifient :
\begin{itemize}
\item[$\bullet$] $\varepsilon'\in (0,\frac{1}{4})$ tel que $\alpha_J \varepsilon'\leq \varepsilon$,
\item[$\bullet$] et $F'\in\mathcal E$ tel que $E\subseteq F'$ et $k_J(\varepsilon').F'\subseteq F$.
\end{itemize}
\end{definitionfr}

L'application d'assemblage admet aussi une version maximale à valeurs dans $\hat K(A\rtimes_{max} G) = \{K^{\varepsilon,E}(A\rtimes_{max} G)\}_{\varepsilon \in (0,\frac{1}{4}),E\in\mathcal E}$.\\

La deuxième étape du travail consiste à formuler une version controlée de la conjecture de Baum-Connes. Nous montrons ensuite comment relier l'application d'assemblage contrôlée à l'application d'assemblage classique. Les théorèmes qui réalisent ce programme sont les théorèmes \ref{Quant1} et \ref{Quant2}, que nous appelons énoncés quantitatifs. Pour cela, il nous faut d'abord introduire les propriétés suivantes :\\
\begin{itemize}
\item[$\bullet$] $QI_{G,B}(E,E',F,\varepsilon)$ : pour tout $x\in RK^G(P_E(G), B )$, alors $\mu^{\varepsilon,E,F}_{G,B}(x) = 0$ implique que $q_E^{E'}(x)=0$ dans $RK^G(P_{E'}(G),B)$.
\item[$\bullet$] $QS_{G,B}(E,F,F',\varepsilon,\varepsilon')$ : pour tout $y\in K^{\varepsilon,F}(B\rtimes G)$, il existe $x\in RK^G(P_E(G),B)$ tel que $\mu^{\varepsilon',E,F'}_{G,B}(x)=\iota_{\varepsilon,F}^{\varepsilon',F'}(y)$.\\
\end{itemize} 

Les énoncés quantitatifs forment le résultat central de la thèse. Ils prennent la forme suivante. Soit $G$ un groupoïde étale $\sigma$-compact à base compacte.

\begin{thmfr}
Soient $B$ une $G$-algèbre, et $\tilde B = l^\infty(\N,B\otimes\mathfrak K)$. Alors $\mu_{G,\tilde B}$ est injective ssi pour tout $E\in\mathcal E,\varepsilon\in(0,\frac{1}{4})$ et $F$ tel que $k_J(\varepsilon).E\subseteq F$, il existe $E' \in\mathcal E$ tel que $E\subseteq E'$ et $QI_{G,B}(E,E',\varepsilon,F)$ soit vérifiée.
\end{thmfr}

\begin{thmfr}
Soient $B$ une $G$-algèbre, et $\tilde B = l^\infty(\N,B\otimes\mathfrak K)$. Alors il existe $\lambda \geq 1$ tel que $\mu_{G,\tilde B}$ est surjective ssi pour tout $0<\varepsilon<\frac{1}{4\lambda}$ et $F\in\mathcal E$, il existe $E\in\mathcal E$ et $F'$ tel que $k_J .E \subseteq F$ tel que $QS_{G,B}(E,F,F',\varepsilon,\lambda\varepsilon)$ soit vérifiée.
\end{thmfr}

Ils admettent une version uniforme :
\begin{thmfr} Soit $G$ un groupoïde étale $\sigma$-compact avec une base compacte. \\
\begin{itemize}
\item[$\bullet$] Supposons que pour toute $G$-algèbre $B$, $\mu_{G,B}$ soit injective. Alors, pour tout $\varepsilon\in (0,\frac{1}{4})$ et tout $E,F\in\mathcal E$ tels que $k_J(\varepsilon). E\subseteq F$, il existe $E'\in\mathcal E$ tel que $E\subseteq E'$ et tel que $QI_{G,B}(E,E',\varepsilon,F)$ soit satisfait pout toute $G$-algèbre $B$.\\
\item[$\bullet$] Supposons que pour toute $G$-algèbre $B$, $\mu_{G,B}$ soit surjective. Alors, pour un certain $\lambda \geq 1$ et pour tout $\varepsilon\in (0,\frac{1}{4\lambda})$ et tout $F\in\mathcal E$, il existe $E,F'\in\mathcal E$ tels que $k_J(\varepsilon). E\subseteq F'$ et $F\subseteq F'$ tel que, pour toute $G$-algèbre $B$, $QS_{G,B}(E, F,F',\varepsilon,\lambda \varepsilon)$ soit satisfait.
\end{itemize}
\end{thmfr}

Le reste du chapitre présente deux applications. La première concerne un résultat dit d'Approximation Persistante, et le second donne une version controlée d'un résultat de $K$-moyennabilité de groupes quantiques discrets. \\

\begin{itemize}

\item[$\bullet$] Rappelons la définition de la propriété d'approximation persistante (PAP), donnée dans \cite{OY3}.\\

\begin{definitionfr}
Soit $(B,\mathcal E)$ une $C^*$-algèbre filtrée, $\lambda>0$ et soient $\varepsilon,\varepsilon'$ des nombres positifs tels que $0<\varepsilon <\varepsilon' <\frac{1}{4}$ et $F,F'\in\mathcal E$ tels que $F\subseteq F'$. 
\begin{itemize}
\item[$\bullet$] $PA_B(\varepsilon,\varepsilon',F,F')$ : pour tout $x\in K_*^{\varepsilon,F}(B)$ tel que $\iota_{\varepsilon,F}(x)=0$ dans $K_*(B)$, alors $\iota_{\varepsilon,F}^{\varepsilon',F'}(x)=0$ dans $K_*^{\varepsilon',F'}(B)$.
\item[$\bullet$] $B$ satisfait la propriété d'approximation persistante $(PAP)_\lambda$ si pour tout $F\in\mathcal E$ et $\varepsilon\in (0,\frac{1}{4})$, il existe $F'\in\mathcal E$ tel que $PA_B(\varepsilon,\lambda\varepsilon,F,F')$ soit vérifiée.\\
\end{itemize}
\end{definitionfr}

La propriété (PAP) est satisfaite lorsque la $K$-théorie est bien approximée, de manière uniforme, par la $K$-théorie controlée. Le théorème \ref{PAPG} permet de relier l'application d'assemblage et la propriété d'approximation persistante. \\

\begin{thmfr} 
Soit $G$ un groupoïde étale à base compacte qui admet un classifiant des actions propres cocompact, et $A$ une $G$-algèbre. Si $\mu_{G,l^\infty(\N, K_A)}$ est surjective et $\mu_{G,A}$ est injective, alors, pour une constante universelle $\lambda_{PA}\geq 1$, pour tout $\varepsilon \in(0,\frac{1}{4\lambda_{PA}})$ et $F\in\mathcal E$, il existe $F'\in\mathcal E$ tel que $F\subseteq F'$ et tel que $PA_{A\rtimes G}(\varepsilon,\lambda_{PA}\varepsilon,F,F')$ soit vérifiée.\\
\end{thmfr}

\item[$\bullet$] En remarque, nous expliquons comment la tranformation de Kasparov contrôlée $J_G$ peut se définir pour les produits croisés réduits et maximaux de groupes quantiques discrets. Il suffit en effet de procéder de la même manière que dans le cas des groupoïdes étales. Pour ce faire, il suffit de remarquer que les propriétés importantes que vérifie le produit croisé, et qui permettent de contruire $J_G$, restent vraies dans le cas quantique. Les preuves écrites dans le cadre des groupoïdes étales se transposent alors sans modification.\\

Nous donnons une application à la $K$-moyennabilité des groupes quantiques discrets, sous la forme de la proposition \ref{QGK}. Notons, pour tout $\hat{\mathbb G}$-algèbre $A$, $\lambda_A$ l'application canonique $A\rtimes_{max} \hat{\mathbb G} \rightarrow A \rtimes_r \hat{\mathbb G}$.\\

\begin{propfr}
Soit $\hat{\mathbb G}$ un groupe quantique discret $K$-moyennable. Alors, il existe une paire de contrôle $\rho$ telle que, pour toute $\hat{\mathbb G}$-algèbre $A$,
\[(\lambda_A)_* : \hat K(A\rtimes_{max} \hat{\mathbb G}) \rightarrow \hat K(A\rtimes_r \hat{\mathbb G}) \]
est un isomorphisme $\rho$-contrôlé.\\
\end{propfr}
\end{itemize}

La dernière partie de la thèse se concentre sur deux applications. \\

\begin{itemize}

\item[$\bullet$] La première application concerne la géométrie coarse. Un moyen pour ce faire est d'utiliser le groupoïde coarse $G(X)$, introduit dans \cite{SkTuYu}, qui est un groupoïde étale associé à tout espace coarse. Notamment, nous démontrons que l'application d'assemblage coarse contrôlée pour $X$ à valeurs dans $B$ est équivalente à l'application d'assemblage contrôlée pour $G(X)$ à valeurs dans $l^\infty(X,B\otimes\mathfrak K)$. Le théorème \ref{BCCeq} est le suivant.\\

\begin{thmfr}
Soient $B$ une $C^*$-algèbre, $E\in\mathcal E_X$ un entourage et $\tilde B$ la $G$-algèbre $l^\infty(X,B\otimes \mathfrak K)$. Alors, pour tout $z\in RK^G(P_{\overline E}(G),\tilde B)$ et tout $\varepsilon\in(0,\frac{1}{4})$, l'égalité suivante est vérifiée :
\[(\Psi_B)_*\circ\mu^{\epsilon,\overline E}_{G,\tilde B} (z) = \mu_{X,B}^{\epsilon,E}(\iota^*(z)).\]
\end{thmfr}

Ce théorème induit en $K$-théorie un résultat de G. Skandalis, J-L. Tu et G. Yu \cite{SkTuYu}, qui établit l'équivalence entre la conjecture de Baum-Connes coarse pour $X$ à coefficients dans $B$ et la conjecture de Baum-Connes pour $G(X)$ à coefficients dans $l^\infty(X,B\otimes \mathfrak K)$. Il permet notamment de donner une version contrôlée d'un résultat de M. Finn-Sell \cite{FinnSellFibred}. \\

\begin{corfr}
Soit $X$ un espace coarse qui admet un plongement fibré dans l'espace de Hilbert. Alors $\hat \mu_{X}^{max}$ est un isomorphisme contrôlé, i.e. $X$ vérifie la version maximale de la conjecture de Baum-Connes coarse contrôlée.\\
\end{corfr}

\item[$\bullet$] La dernière application concerne une formule de Künneth en $K$-théorie quantitative pour les $C^*$-algèbres de groupoïdes, qui constitue le théorème \ref{Kunneth}. Le cas des groupes discrets a été couvert dans un article de H. Oyono-Oyono et G. Yu \cite{OY4}. Pour établir la formule de Künneth, nous montrons qu'un certain morphisme $\alpha_{A\rtimes_r G,B}$ est un isomorphisme. La stratégie consiste en une généralisation des techniques dites de "Going Down Principle" développées par J. Chabert, S. Echterhoff et H. Oyono-Oyono dans \cite{ChabertEOY}. Elle peut se résumer ainsi : nous définissons une version topologique $\alpha_{A,B}^{G,Z} : RK^G(Z,A)\otimes K_*(B)\rightarrow RK^G(Z,A\otimes B )$ de $\alpha_{A\rtimes_r G,B}$, puis démontrons que l'application d'assemblage les entrelace. Pour cela, nous avons besoin de définir le foncteur $Ind_H^G$ d'induction pour un sous-groupoïde compact ouvert $H$ de $G$ en $KK$-théorie équivariante, ainsi que d'introduire une propriété sur les actions de groupoïdes, appelée propreté forte. Nous définissons ensuite une classe $\mathcal C$ de groupoïdes, dont toutes les actions propres sont fortement propres, i.e. localement induites par des sous-groupoïdes compacts ouverts. Nous prouvons que les groupoïdes amples sont dans la classe $\mathcal C$, ce qui donne une large classe d'exemples. Le résultat suivant permet alors de relier la classe $\mathcal C$ à la formule de Künneth. \\

%La dernière application concerne une formule de Künneth en $K$-théorie quantitative pour les $C^*$-algèbres de groupoïdes, qui constitue le théorème \ref{Kunneth}. Le cas des groupes discrets a été couvert dans un article de H. Oyono-Oyono et G. Yu \cite{OY4}. Pour établir la formule de Künneth, nous montrons qu'un certain morphisme $\alpha_{A\rtimes_r G,B}$ est un isomorphisme. La stratégie consiste en une généralisation des techniques dites de "Going Down Principle" développées par J. Chabert, S. Echterhoff et H. Oyono-Oyono dans \cite{ChabertEOY}. Pour cela, nous avons besoin de définir le foncteur $Res_H^G$ de restriction à un sous-groupoïde compact $H$ de $G$ en $KK$-théorie équivariante, ainsi qu'une propriété sur les actions de groupoïdes, appelée propreté forte. Nous définissons ensuite une classe $\mathcal C$ de groupoïdes, dont toutes les actions propres sont fortement propres, i.e. localement induites par des sous-groupoïdes compacts ouverts. Nous prouvons que les groupoïdes amples sont dans la classe $\mathcal C$, ce qui donne une large classe d'exemples. Le principe de la preuve est le suivant : nous définissons une version topologique $\alpha_{A,B}^{G,Z} : RK^G(Z,A)\otimes K_*(B)\rightarrow RK^G(Z,A\otimes B )$ de $\alpha_{A\rtimes_r G,B}$, puis démontrons que l'application d'assemblage les entrelace. Le résultat suivant permet alors de relier la classe $\mathcal C$ à la formule de Künneth. \\

\begin{thmfr}
Soit $G$ un groupoïde étale de la classe $\mathcal C$, et soient $E\in\mathcal E$ un ensemble contrôlé de $G$ et $P_E(G)$ le complexe de Rips associé. Si, pour tout sous-groupoïde compact ouvert $H$ de $G$ et tout $H$-espace $V$ tel que l'application moment $p : V\rightarrow H^{(0)}$ soit localement injective, $\alpha_{Res_H^G(A),B}^{H,V}$ est un isomorphisme, alors $\alpha_{A,B}^{G,P_E(G)}$ est un isomorphisme pour toute $C^*$-algèbre $B$ telle que $K_*(B)$ est un groupe abélien libre.\\
\end{thmfr}

Voici le contenu du théorème principal.\\

\begin{thmfr}
Soit $G$ un groupoïde $\sigma$-compact étale et $A$ une $G$-algèbre. Si 
\begin{itemize}
\item[$\bullet$] $G$ vérifie la conjecture de Baum-Connes à coefficients,
\item[$\bullet$] $G$ est un groupoïde fortement propre,
\item[$\bullet$] pour tout sous-groupoïde compact ouvert $H$ de $G$ et tout $H$-espace $V$ tels que l'application moment $p : V\rightarrow H^{(0)}$ soit localement injective, $\alpha_{A,B}^{G,V}$ est un isomorphisme.
\end{itemize} 
Alors $A\rtimes_r G$ vérifie la formule de Künneth contrôlée.
\end{thmfr}

\end{itemize}

































