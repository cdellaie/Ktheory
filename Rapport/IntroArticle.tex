In this article, we develop a controlled $K$-theory suited for étale groupoids, which is a generalization of the controlled or quantitative $K$-theory developed by my advisor H. Oyono-Oyono and G. Yu in \cite{OY2}. \\

One of the reasons operator $K$-theory is studied is that it is the receptacle for higher indices of elliptic differential operators. Of particular relevance to us are the following class of $C^*$-algebras :
\begin{itemize}
\item[$\bullet$] Roe algebras $C^*(X)$ associated to Coarse Spaces, whose $K$-theory encodes the higher indices of elliptic operators on noncompact complete Riemannian manifolds, 
\item[$\bullet$] Reduced crossed-products $A\rtimes_r G$ of $C^*$-algebras by an action of an étale groupoid $G$, similarly used in the context of foliations and longitudinally elliptic operators, higher indices of equivariant elliptic operators on covering spaces.\\
\end{itemize}

These examples culminate in the statement of the Baum-Connes conjecture \cite{BaumConnesHigson}, which proposes an algorithm to compute these $K$-theory groups using an assembly map, classically dentoted $\mu_{G,A} : K^{top}_*(G,A)\rightarrow K_*(A\rtimes_r G)$. The Baum-Connes conjecture asserts that $\mu_{G,A}$ is an isomorphism, and it is known to hold, for example, for groups with the Haagerup property \cite{HigsonKasparov}. This celebrated result is proved using an argument very analytical in nature, called the Dirac-Dual-Dirac method, which prevents to the author's knowledge to adapt it in other situations. For example, the injectivity of the Baum-Connes assembly map $\mu_{\Gamma,\C}$ for a discrete group implies the Novikov conjecture for higher signatures for $\Gamma$. Algebraic versions of this conjecture exists, and it is not known if the Dirac-Dual-Dirac method can be adapted in this setting.  \\

Controlled $K$-theory aims at tackling this issue, i.e. to propose other ways to prove isomorphisms of assembly maps by decomposing the object under study in simpler parts and gluying everything by Mayer-Vietoris type arguments. It is based on the celebrated proof of G. Yu \cite{Yu1} of the Novikov conjecture for groups with finite asympotitc dimension. In the course of the proof appeared some approximations of the operator $K$-theory groups. Controlled, or quantitative, $K$-theory is a nice framework to formalize this construction. It consists of a family of $\Z_2$-graded abelian groups $(K^{\varepsilon, R}(A))_{\varepsilon\in(0,\frac{1}{4},R>0}$ defined for any filtered $C^*$-algebra in the sense of \cite{OY2}. \\

The point of departure of this work is to use the coarse groupoid $G(X)$ associated to any Coarse Space $X$, defined in \cite{SkTuYu}, to try and understand how one can generalize the controlled $K$-theory to étale groupoids. The basic observation is the following : in $G(X)$, what encodes the propagation $R$ appearing in the controlled $K$-theory of Roe algebras are a family of compact open denoted by $\overline \Delta_R$. The crucial property is the stability of this family with respect to composition. Reversing this line of thought, one can redefine filtered $C^*$-algebras and controlled $K$-theory with respect to a groupoid using a family of compact subsets of $G$, which are thought as controlling the propagation in $G$. It turns out that, when specialized to the coarse groupoid $G(X)$, this new controlled theory gives back the results of \cite{OY2}, giving a correct generalization. We also prove controlled versions of the results of \cite{SkTuYu}, which gives a large class of examples satisfying the controlled Baum-Connes conjecture. The ultimate goal of this construction is to show that the controlled assembly maps respects stability properties, using controlled Mayer-Vietoris sequence.\\

This article is organized as follows. In the third section, we define controlled $K$-theory 

   
