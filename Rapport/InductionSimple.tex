%%%%%%%%%%%%%%%%%%%%%%%%%%%%%%%%%%%%%%%%%%%%%%%%
\subsection{Induction and Restriction functors}
%%%%%%%%%%%%%%%%%%%%%%%%%%%%%%%%%%%%%%%%%%%%%%%%

\begin{definition} A subset $H\subseteq G$ is called a subgroupoid if :
\begin{itemize}
\item[$\bullet$] for all $h,h'\in H $ such that $s(h') = r(h)$, $h'h \in H$,
\item[$\bullet$] if $h\in H$, $h^{-1}\in H$.
\end{itemize}
Then, the restriction of the multiplication, inverse, unit, target and source maps on $H$ defines a structure of groupoid on $H$ over $H^{(0)} = s(H)$. If $G$ is étale, $H$ is also étale. We will write $H< G$ to indicate that $H$ is a subgroupoid of $G$.
\end{definition}

In this section, we define for all subgroupoids $H < G$ induction and restriction transformations. Let $G$ be an étale groupoid and $H<G$.\\

Let $A$ be a $H$-algebra, with action $\alpha : s^*A \rightarrow r^* A$. Put $\phi=r\circ \iota$ where $\iota : H^{(0)}\hookrightarrow G^{(0)}$ is the canonical inclusion, hence $\phi^*A = A\otimes_\phi C_0(G)$ is a $C(G)$-algebra . Define the induced $C(G^{(0)})$-algebra :
%\[\text{Ind}_H^G (A) = \{f \in C_0(G,A) \text{ s.t. } h^{-1} f(gh) = f(g),\forall h\in H\}. \]
\[\text{Ind}_H^G (A) = \{f \in \phi^* A \text{ s.t. } \alpha_{h^{-1}}(f_{hg}) = f_g,\forall h\in H\}. \]
Left translation defines an action of $G$ on $\text{Ind}_H^G A$, so that $\text{Ind}_H^G A$ is a $G$-algebra.\\

Let $E$ be a Hilbert $A$-module, endowed with an action $V\in\mathcal L_A(s^*E,r^*E)$ of $H$. Define the induced Hilbert module :
%\[\text{Ind}_H^G (E) = \{f \in C_0(G,E) \text{ s.t. } h^{-1} f(gh) = f(g),\forall h\in H\}. \]
\[\text{Ind}_H^G (E) = \{f \in \phi^* E \text{ s.t. } \alpha_{h^{-1}}(f_{hg}) = f_g,\forall h\in H\}. \]
Left translation defines an action of $G$ on $\text{Ind}_H^G E$, hence $\text{Ind}_H^G E$ is a $G$-equivariant Hilbert $\text{Ind}_H^G A$-module.

Let $A$ and $B$ be two $H$-algebras. For all $(E,\pi,T)\in \mathbb E^G(A,B)$, define $\text{Ind}_H^G \pi$ and $\text{Ind}_H^G T$ as pointwise evaluation and multiplication by $\pi$ and $T$, i.e. $\text{Ind}_H^G T = T\otimes_\phi 1$ and $\text{Ind}_H^G \pi  = 1\otimes_\phi \pi $.

\begin{definition}
For all subgroupoids $H<G$, let $A$, $B$ and $D$ be $G$-algebras and $A'$, $B'$ and $D'$ be $H$-algebras. Then, the map $(E,\pi,T)\mapsto ( \text{Ind}_H^G E, \text{Ind}_H^G\pi ,\text{Ind}_H^G T )$ induces an even homomorphism of $\Z_2$-graded abelian groups
\[\text{Ind}_H^G : KK_*^H(A',B')\rightarrow KK_*^G( \text{Ind}_H^G A', \text{Ind}_H^G B') \] 
called the induction transformation.\\
Moreover, we naturally have an even homomorphism of $\Z_2$-graded abelian groups 
\[\text{Res}_H^G : KK_*^G(A,B)\rightarrow KK_*^H( \text{Res}_H^G A, \text{Res}_H^G B) \] 
called the restriction transformation by restricting the action.\\
These two transformations respect the Kasparov product, i.e. 
\[ \text{Ind}_H^G(z\otimes z') = \text{Ind}_H^G(z)\otimes_{\text{Ind}_H^G(D')}\text{Ind}_H^G(z')\quad \forall z\in KK^H(A',D'),z'\in KK^H(D',B') \]
and 
\[ \text{Res}_H^G(z\otimes z') = \text{Res}_H^G(z)\otimes_{\text{Res}_H^G(D)}\text{Res}_H^G(z')\quad \forall z\in KK^G(A,D),z'\in KK^G(D,B) \]
The reader is refered to \cite{LeGall} for a proof.
\end{definition}   

Let $Z$ be a right $H$-space. Define on $Z\times G$ the following equivalence relation :
\[(z,g)\sim_H (z.h, h^{-1}g)\quad \forall h\in H.\]

\begin{definition}
The induced $G$-space of a $H$-space $Z$ is defined as $Z\times_H G = (Z\times G) / \sim_H$. 
\end{definition}

Notice that we have a natural identification between $\text{Ind}_H^G C_0(Z)$ and $C_0(Z\times_H G)$.

\subsection{Strongly proper groupoids}

We now introduce a property on groupoids that will entail a nice result on induction and restriction transformations at the level of $K$-homology.

\begin{definition}\label{StronglyProper}
A groupoid $G$ is said to be strongly proper if there exists an open cover $\mathcal U$ such that, for all $U\in\mathcal U$, there exists a compact subgroupoid $H_U$ of $G$ and a $H_U$-space $Z_U$ together with a $G$-equivariant homeomorphism
\[\psi_U : U \rightarrow G\times_{H_U} Z_U.\] 
\end{definition}

\begin{prop}
Let $G$ be a strongly proper groupoid, and $\mathcal U$ an open cover satisfying the conditions of definition \ref{StronglyProper}. Then, the $G$-map $\psi_U $ extends to an isomorphism of groupoids
%\[\begin{tikzcd}
% G_{|U}\arrow{r}\arrow[shift right]{d}\arrow[shift left]{d} & Z_U\rtimes H_U \arrow[shift right]{d}\arrow[shift left]{d} \\
% U \arrow{r}{\psi_U} & G\times_{H_U} Z_U 
%\end{tikzcd}.\]
% Ce morceau bug sous windows : pourquoi ?
\end{prop}

\begin{Expl}
Recall that $G$ is ample if $G$ is étale and $G^{[0)}$ is totally disconnected. An ample groupoid is strongly proper.
\end{Expl}

\begin{definition}
The groupoid $G$ is said to be locally induced if there exists an open cover $\mathcal U$ of $G^{(0)}$ such that for all $U\in\mathcal U$, there exists a compact groupoid $H_U$ acting on $G_{|U}$ and an ismomorphism of groupoids $G_{|U }\simeq G_{|U}/ H_U \rtimes H_U$.  
\end{definition}

Since $G_{|U}/ H_U \rtimes H_U$ is Morita equivalent to $H_U$ and $G$ is Morita equivalent to the pull-back $G[\mathcal U]$, we have an isomorphism 
\[KK^G(A,B) \simeq \prod_{U\in\mathcal U} KK^{H_U}(\phi^* A,\phi^* B).\]

Ample groupoids are example of locally induced groupoid. Recall that an étale groupoid is ample iff the open compact bisections form  a basis of the groupoid. Now take $U$ an open compact bisection, and set $H_U = \{ g\in G\text{ s.t. } gU\cap U \neq \emptyset \}$, which is a compact subgroupoid of $G$ that fulfills the condition.\\

The author presently doesn't know of any other examples of locally induced groupoids which are not ample. Maybe looking into groupoids with finite asymptotic dimension. 



