\section{Assembly Maps}

In this section, all groupoids $G\rightrightarrows G^{(0)}$ are étale. 

\subsection{Proper groupoids and proper actions}

\begin{definition}
A topological groupoid $G\rightrightarrows G^{(0)}$ is proper if the map $G\times_s G^{(0)}\rightarrow G^{(0)}\times G^{(0)}$ defined by $(g,z)\mapsto (z,gz)$ is a proper map.\\
A $G$-space $Z$ is said to be proper if the crossed-product groupoid $Z\rtimes G$ is proper. If the space of orbits $Z/G$ is compact, $Z$ is called $G$-compact.\\
\end{definition}

\begin{definition}
A $G$-space $\underline E G$ is called universal if, given any proper $G$-compact $G$-space $Z$, there exists a continuous $G$-equivariant map $Z\rightarrow \underline E G$.
\end{definition}

We now construct a model for universal spaces of $G$, based on J.-L.Tu's work \cite{TuBC2}. For any compact subset $K\subset G$, define $P_K(G)$ to be the space of probability measures $\nu $ with support contained in one and only one fiber $G^x$ for some $x\in G^{(0)}$, and such that if $g,g'\in \text{supp }(\nu)$, then $g'g^{-1}\in K$. We endow $P_K(G)$ with the weak-$*$ topology.\\

The action of $G$ is defined by translation. The momentum map $P_K(G)\rightarrow G^{(0)}$ is just the map associating to $\nu$ the only $x$ such that $\text{supp }(\nu) \subset G^x$. As the fibers are discrete, any $\nu\in P_K(G)$ can be represented as as sum $\nu = \sum_{g\in G^x} \lambda_g(\nu)\delta_g$, where $\delta_g$ is the Dirac measure at $g\in G $. The continuous functions $\lambda_g$ are called coordinate functions and satisfy $\sum_{g\in G^x} \lambda_g(\nu) =1$ for every $x\in G^{(0)}$. The action of $G$ is given by $g\lambda_h = \lambda_{g^{-1}h}$.\\

\begin{lem}[Tu,\cite{TuBC2}]
The action of $G$ on $P_K(G)$ is proper and cocompact.
\end{lem}  

\begin{lem}[Tu,\cite{TuBC2}]
Let $Z$ be a proper $G$-compact $G$-space. Then there exists a compact subset $K\subset G$ and a $G$-equivariant continuous map $Z\rightarrow P_K(G)$.
\end{lem}  

\subsection{Equivariant $K$-homology}

We will use the equivariant $KK$-theory developped by Le Gall in his thesis \cite{LeGall}, which is an extension of the usual equivariant $KK$-theory of Kasparov. Recall that, if $A$ and $B$ are two $G$-algebras, elements of $KK^G(A,B)$ are homotopy classes of triple $(E,\pi,T)$ where :\\

\begin{itemize}
\item[$\bullet$] $E$ is a $G$-module,
\item[$\bullet$] $\pi : A\rightarrow \mathcal L_B(E)$ is a $*$-homomorphism,
\item[$\bullet$] $T\in \mathcal L_B(E)$ is an adjoinable operator such that the triple satisfies the condition of $K$-cycle : $\pi(a)(T^2-T),\pi(a)(T^*-T)[\pi(a),T]$, and $\pi(a)(T-g.T)$ are compact operators if $E$, for all $a\in A, g\in G$.\\
\end{itemize}

\begin{definition}
Let $Y$ be a proper $G$-space and $B$ a $G$-algebra. Then the analytic $K$-homology of $Y$ with coefficients in $B$ is defined by 
\[RK^G(Y,B) = \varinjlim_{Z\subset Y} KK^G(C_0(Z),B), \]
the inductive limit being taken on proper $G$-compact $G$-subspaces $Z$ of $Y$.
\end{definition}

The previous lemmas assures that this inductive limit can be somehow restricted when it comes to the analytic $>K$-homology of any universal space $\underline E G$ :
\[RK^G(\underline E G,B) = \varinjlim_{K\subset G \text{ compact }} KK^G(C_0(P_K(G)),B).\]

\subsection{Descent functor}

There exists a natural transformation $KK^G(A,B)\rightarrow KK(A\rtimes_r G, B\rtimes_r G)$ respecting the Kasparov product, which is called the descent functor. The same statement remains true for maximal crossed products. We recall its construction, which was first stated in \cite{LeGall}.\\
 
Let $A$ and $B$ be two $G$-algebras and $(E,\pi,T)\in E^G(A,B)$ be a $K$-cycle. Define :
\begin{itemize}
\item[$\bullet$] $E_G$ to be the completion of $C_c(G,E)$ with respect to the norm $||f||= \sup_{x\in G^{(0)}} \sum_{G^x} |f(g)|^2$,
\item[$\bullet$] $\pi_G$ to be the image of $\pi:A \rightarrow \mathcal L_B(E)$ under the reduced crossed-product functor,
\item[$\bullet$] $T_G$ to be the image of $T\in\mathcal L(E)$ under the reduced crossed-product functor.
\end{itemize}
Then $(E_G,\pi_G,T_G)\in E(A\rtimes_r G,B\rtimes_r G)$ and the map $(E,\pi,T)\mapsto (E_G,\pi_G,T_G) $ induces a homorphism of abelian groups
\[j_G : KK^G(A,B)\rightarrow KK(A\rtimes_r G,B\rtimes_r G)\]
satisfying $j_G(z\otimes_D z')=j_G(z)\otimes_{D\rtimes_r G} j_G(z')$ for any $z\in KK^G(A,D), z'\in KK^G(D,B)$.\\ 

\subsection{The assembly map}

If $Z$ is a proper $G$-compact $G$-space, $Z\rtimes G$ is a proper groupoid by definition, hence there exists a cutoff function $c : Z\rightarrow [0,1]$ such that $\int_G^{p(z)} c(zg)d\lambda^x(g) = 1$. The function $g\mapsto c(r(g))^{\frac{1}{2}}c(s(g))^{\frac{1}{2}} $ defines a projection in $C_0(Z)\rtimes_r G$ which we denote by $\mathcal L_Z$. If $Z=P_K(G)$, then $\mathcal L_Z = \mathcal L_K$.

\begin{definition}
The assembly map for $G$ with coefficients in $B$ is defined as the inductive limit of the maps $\mu_{G,B}^{(Z)} : KK^G(C_0(Z),B)\rightarrow K(B\rtimes_r B)$ given by
\[\mu_{G,B}^{(Z)} (z)=[\mathcal L_Z]\otimes_{C_0(Z)\rtimes G} j_G(z),\]
that is $\mu_{G,B} = \varinjlim \mu_{G,B}^{(Z)}$ (one has to check that theses maps respects the inductive systems, which they do).
\end{definition}

Another definition : with $Ad$...

\subsection{The Baum-Connes conjecture}

%\begin{conjecture}
Let $G$ be an étale groupoid, and $A$ a $G$-algebra.\\
The Baum-Connes conjecture for $G$ with coefficients in $A$ is the following claim : $\mu_{G,A}$ is an isomorphism of abelian groups.\\
The Baum-Connes conjecture with coefficients is : for all $G$-algebras $A$, $\mu_{G,A}$ is an isomorphism of abelian groups.
%\end{conjecture}

The conjecture was first stated in \cite{BaumConnes} and in \cite{BaumConnesHigson}. The statement is a descendent of the Connes-Kasparov conjecture, which is simply, when one knows the Baum-Connes conjecture, the latter for almost connected groups (i.e. locally compact groups $G$ such that $G/G_0$ is compact, where $G_0$ is the connected component of the identity).\\

Here is the status of the conjecture :
\begin{itemize}
\item[$\bullet$] The Connes-Kasparov conjecture was established by J. Chabert, S. Echterhoff and R. Nest, who proved that $\mu_{G}$ is an isomorphism for every secound countable almost connected group $G$, and every group of $k$-rational points of a linear algebraic group over a local field of characteristic $0$.\\
\item[$\bullet$] N. Higson and G. Kasparov proved that the conjecture with coefficients holds for groups having Haagerup property. 
\item[$\bullet$] V. Lafforgue proved the conjecture with coefficients for hyperbolic groups.
\item[$\bullet$] V. Lafforgue showed the conjecture for every semi-simple Lie group, real or $p$-adic reductive, and discrete cocompact subgroup of real rank $1$ Lie groups or of $SL(3,k)$ for any local field $k$. 
\end{itemize}

\subsection{The Coarse Assembly map}

The construction of the Coarse Assembly map is very similar to the Assembly map for groupoids, but relies on different functors.\\

In \cite{SkTuYu} is defined the Roe algebra $C^*(X,B)$ of a discrete metric space with bounded geometry $X$ with coefficients in an arbitrary $C^*$-algebra $B$. For $R>0$, define $C_R[X,B]$ to be the involutive subspace of operators $T\in \mathcal L_B(l^2(X)\otimes H\otimes B)$ such that $T_{xy}\in \mathfrak K(H\otimes B)$ and $T_{xy}=0$ as soon as $d(x,y)>R$, and $\C[X,B]= \cup_{R>0} C_R[X,B]$. 

\begin{definition}
The Roe algebra $C^*(X,B)$ is defined as the Hibert $B$-module obtained after the completion of $\C[X,B]$ with respect to the operator norm in $\mathcal L_B(l^2(X)\otimes H\otimes B)$.
\end{definition} 

This construction is functorial : any $*$-homomorphism $\phi: A\rightarrow B$ gives rise to a $*$-homomorphism $\phi_X : C^*(X,A)\rightarrow C^*(X,B)$. Moreover this functoriality extends to $KK$-theory : there exists a natural transformation 
\[\sigma_X : KK(A,B)\rightarrow KK(C^*(X,A),C^*(X,B))\]
which respects the Kasparov product, i.e. $\sigma(z\otimes_D z')=\sigma_X(z)\otimes_{C^*(X,D)} \sigma_X(z')\quad,\forall z\in KK(A,D),z'\in KK(D,B)$.\\

Let $X$ be a coarse space with bounded geometry. The Rips complex is the inductive system of finite dimensional simplicial complexes
\[P_E(X)=\{\eta\in \text{Prob}(X)\text{ s.t. supp }\eta\subset E\}\quad,\forall E\in \mathcal E_X,\]
endowed with the $*$-weak topology. It is an inductive system with respect to inclusion of entourages.\\
 
The coarse homology group of $X$ with coefficients in an arbitrary $C^*$-algebra $B$ is defined as 
\[KX_*(X,B)=\varinjlim_{E\in \mathcal E_X} RK(P_E(X),B)\]
where the limit is taken along the inductive system $\{P_E(X)\}_{E\in\mathcal E_X}$. \\

Let $E\in \mathcal E_X$. Any simplex $\eta\in P_E(X)$ can be written as a finite sum $\eta=\sum_x \lambda_x(\eta)\delta_x$ by boundedness of the geometry. Here, $\delta_x$ is the Dirac measure at $x\in X$. The functions $\lambda_x :\eta\mapsto \lambda_x(\eta)$ are continuous and satifisfy $\sum\lambda_x(\eta) = 1,\forall\eta\in P_E(X)$, so that
\[V_0\left\{\begin{array}{rcl} 
 C_0(P_E(X)) & \rightarrow 	& l^2(X)\otimes C_0(P_E(X)) 		\\ 
 f           & \mapsto 		& (\lambda_x^{\frac{1}{2}}f)_{x\in X} 
\end{array}\right.\] 
is an isometry of $C_0(P_E(X))$-modules, and $V_0V_0^*\otimes id_H$ defines a projection in $l^2(X)\otimes H\otimes C_0(P_E(X))$ with finite propagation, so class $[\mathcal L_E]\in K_0(C^*(X,C_0(P_E(X)))$.\\

\begin{definition}
The Coarse Assembly map $\mu:KX_*(X,B)\rightarrow K_*(C^*(X,B))$ for $X$ with coefficients in $B$ is defined as 
\[\forall z\in RK(P_E(X), B), \mu_{X,B}(z)=[\mathcal L_E]\otimes_{C^*(X,C_0(P_E(X)))} \sigma_X(z).\]
\end{definition}

The Coarse Baum-Connes conjecture is the following claim.\\

%\begin{conjecture}
For any coarse space $X$ with bounded geometry, the Coarse Assembly map $\mu_{X,\C}$ is an isomorphism.\\
%\end{conjecture}

Here are some remarks about the status of the conjecture :
\begin{itemize}
\item[$\bullet$] the conjecture is known to hold for any coarse space that admits a coarse embedding into a  separable Hilbert space\cite{Yu2}.
\item[$\bullet$] couterexamples have been constructed \cite{HigsonLaffSk},
\item[$\bullet$] Let $\Gamma$ be a finitely generated group, and $|\Gamma|$ the coarse space associated to the word-length metric (any two left invariant metric on $\Gamma$ are coarsely equivalent). Then the Coarse Baum-Connes conjecture for $|\Gamma|$ implies the Novikov conjecture for higher signatures for $\Gamma$.
\end{itemize}




















