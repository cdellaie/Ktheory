\section{Assembly Maps}

In this section, all groupoids $G\rightrightarrows G^{(0)}$ are étale. 

\subsection{Proper groupoids and proper actions}

\begin{definition}
A topological groupoid $G\rightrightarrows G^{(0)}$ is proper if the map $G\times_s G^{(0)}\rightarrow G^{(0)}\times G^{(0)}$ defined by $(g,z)\mapsto (z,gz)$ is a proper map.\\
A $G$-space $Z$ is said to be proper if the crossed-product groupoid $Z\rtimes G$ is proper. If the space of orbits $Z/G$ is compact, $Z$ is called $G$-compact.\\
\end{definition}

\begin{definition}
A $G$-space $\underline E G$ is called universal if, given any proper $G$-compact $G$-space $Z$, there exists a continuous $G$-equivariant map $Z\rightarrow \underline E G$.
\end{definition}

We now construct a model for universal spaces of $G$, based on J.-L.Tu's work \ref{TuBC2}. For any compact subset $K\subset G$, define $P_K(G)$ to be the space of probability measures $\nu $ with support contained in one and only one fiber $G^x$ for some $x\in G^{(0)}$, and such that if $g,g'\in \text{supp }(\nu)$, then $g'g^{-1}\in K$. We endow $P_K(G)$ with the weak-$*$ topology.\\

The action of $G$ is defined by translation. The momentum map $P_K(G)\rightarrow G^{(0)}$ is just the map associating to $\nu$ the only $x$ such that $\text{supp }(\nu) \subset G^x$. As the fibers are discrete, any $\nu\in P_K(G)$ can be represented as as sum $\nu = \sum_{g\in G^x} \lambda_g(\nu)\delta_g$, where $\delta_g$ is the Dirac measure at $g\in G $. The continuous functions $\lambda_g$ are called coordinate functions and satisfy $\sum_{g\in G^x} \lambda_g(\nu) =1$ for every $x\in G^{(0)}$. The action of $G$ is given by $g\lambda_h = \lambda_{g^{-1}h}$.\\

\begin{lem}[Tu,\ref{TuBC2}]
The action of $G$ on $P_K(G)$ is proper and cocompact.
\end{lem}  

\begin{lem}[Tu,\ref{TuBC2}]
Let $Z$ be a proper $G$-compact $G$-space. Then there exists a compact subset $K\subset G$ and a $G$-equivariant continuous map $Z\rightarrow P_K(G)$.
\end{lem}  

\subsection{Equivariant $K$-homology}

We will use the equivariant $KK$-theory developped by Le Gall in his thesis \ref{LeGall}, which is an extension of the usual equivariant $KK$-theory of Kasparov. Recall that, if $A$ and $B$ are two $G$-algebras, elements of $KK^G(A,B)$ are homotopy classes of triple $(E,\pi,T)$ where :\\

\begin{itemize}
\item[$\bullet$] $E$ is a $G$-module,
\item[$\bullet$] $\pi : A\rightarrow \mathcal L_B(E)$ is a $*$-homomorphism,
\item[$\bullet$] $T\in \mathcal L_B(E)$ is an adjoinable operator such that the triple satisfies the condition of $K$-cycle : $\pi(a)(T^2-T),\pi(a)(T^*-T)[\pi(a),T]$, and $\pi(a)(T-g.T)$ are compact operators if $E$, for all $a\in A, g\in G$.\\
\end{itemize}

\begin{definition}
Let $Y$ be a proper $G$-space and $B$ a $G$-algebra. Then the analytic $K$-homology of $Y$ with coefficients in $B$ is defined by 
\[RK^G(Y,B) = \varinjlim_{Z\subset Y} KK^G(C_0(Z),B), \]
the inductive limit being taken on proper $G$-compact $G$-subspaces $Z$ of $Y$.
\end{definition}

The previous lemmas assures that this inductive limit can be somehow restricted when it comes to the analytic $>K$-homology of any universal space $\underline E G$ :
\[RK^G(\underline E G,B) = \varinjlim_{K\subset G \text{ compact }} KK^G(C_0(P_K(G)),B).\]


\subsection{The assembly map}

If $Z$ is a proper $G$-compact $G$-space, $Z\rtimes G$ is a proper groupoid by definition, hence there exists a cutoff function $c : Z\rightarrow [0,1]$ such that $\int_G^{p(z)} c(zg)d\lambda^x(g) = 1$. The function $g\mapsto c(r(g))^{\frac{1}{2}}c(s(g))^{\frac{1}{2}} $ defines a projection in $C_0(Z)\rtimes_r G$ which we denote by $\mathcal L_Z$. If $Z=P_K(G)$, then $\mathcal L_Z = \mathcal L_K$.

\begin{definition}
The assembly map for $G$ with coefficients in $B$ is defined as the inductive limit of the maps $\mu_{G,B}^{(Z)} : KK^G(C_0(Z),B)\rightarrow K(B\rtimes_r B)$ given by
\[\mu_{G,B}^{(Z)} (z)=[\mathcal L_Z]\otimes_{C_0(Z)\rtimes G} j_G(z),\]
that is $\mu_{G,B} = \varinjlim \mu_{G,B}^{(Z)}$ (one has to check that theses maps respects the inductive systems, which they do).
\end{definition}

Another definition : with $Ad$...

\subsection{The Baum-Connes conjecture}

\begin{conjecture}
Let $G$ be an étale groupoid, and $A$ a $G$-algebra.\\
The Baum-Connes conjecture for $G$ with coefficients in $A$ is the following claim : $\mu_{G,A}$ is an isomorphism of abelian groups.\\
The Baum-Connes conjecture with coefficients is : for all $G$-algebras $A$, $\mu_{G,A}$ is an isomorphism of abelian groups.
\end{conjecture}

The conjecture was first stated in \ref{BaumConnes} and in \ref{BaumConnesHigson}. The statement is a descendent of the Connes-Kasparov conjecture, which is simply, when one knows the Baum-Connes conjecture, the latter for almost connected groups (i.e. locally compact groups $G$ such that $G/G_0$ is compact, where $G_0$ is the connected component of the identity).\\

Here is the status of the conjecture :
\begin{itemize}
\item[$\bullet$] The Connes-Kasparov conjecture was established by J. Chabert, S. Echterhoff and R. Nest, who proved that $\mu_{G}$ is an isomorphism for every secound countable almost connected group $G$, and every group of $k$-rational points of a linear algebraic group over a local field of characteristic $0$.\\
\item[$\bullet$] N. Higson and G. Kasparov proved that the conjecture with coefficients holds for groups having Haagerup property. 
\item[$\bullet$] V. Lafforgue proved the conjecture with coefficients for hyperbolic groups.
\item[$\bullet$] V. Lafforgue showed the conjecture for every semi-simple Lie group, real or $p$-adic reductive, and discrete cocompact subgroup of real rank $1$ Lie groups or of $SL(3,k)$ for any local field $k$. 
\end{itemize}


























