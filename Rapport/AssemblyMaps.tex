\section{Assembly Maps}

In this section, all groupoids $G\rightrightarrows G^{(0)}$ are étale. 

\subsection{Proper groupoids and proper actions}

\begin{definition}
A topological groupoid $G\rightrightarrows G^{(0)}$ is proper if the map $G\times_s G^{(0)}\rightarrow G^{(0)}\times G^{(0)}$ defined by $(g,z)\mapsto (z,gz)$ is a proper map.\\
A $G$-space $Z$ is said to be proper if the crossed-product groupoid $Z\rtimes G$ is proper. If the space of orbits $Z/G$ is compact, $Z$ is called $G$-compact.
\end{definition}

\begin{definition}
A $G$-space $\underline E G$ is called universal if, given any proper $G$-space $Z$, there exists a continuous $G$-equivariant map $Z\rightarrow \underline E G$.
\end{definition}

We now construct a model for universal spaces of $G$, based on J.-L.Tu's work \cite{TuBC2}. For any compact subset $K\subseteq G$, define $P_K(G)$ to be the space of probability measures $\nu $ with support contained in one and only one fiber $G^x$ for some $x\in G^{(0)}$, and such that if $g,g'\in \text{supp }(\nu)$, then $g'g^{-1}\in K$. We endow $P_K(G)$ with the weak-$*$ topology.\\

The action of $G$ is defined by translation. The momentum map $p : P_K(G)\rightarrow G^{(0)}$ is the map associating to $\nu$ the only $x$ such that $\text{supp }(\nu) \subseteq G^x$. Let us show that $p$ is continuous. Let $U\subseteq G^{(0)}$ be an open set. Then, 
\[p^{-1}(U) = \cup_{g\in G}\{\eta\in P_K(G) \text{ s.t. supp}( \eta ) \subseteq r^{-1}(U) \cap gK\}.\]
As for all $g\in G$, $r^{-1}(U)\cap gK$ is compact, it is covered by a finite number of open bisections $S_1$, ... , $S_r$. Put $W_j= r^{-1}(U)\cap gK\cap S_j$ and choose open subsets $V_j\subseteq W_j$. Let us take, for each $j\in\{1,...,r\}$, a continuous function $\phi_j : G\rightarrow [0,1]$ such that $\phi$ is equal to $1$ on $V_j$ and $0$ outside of $W_j$. Then $\cup_j \{\eta\in P_K(G)\text{ s.t. } |\langle \eta, \phi_j\rangle -1|<\frac{1}{4} \}\subseteq \{\eta\in P_K(G) \text{ s.t. supp}( \eta ) \subseteq r^{-1}(U) \cap gK\}$, hence $p^{-1}(U)$ is open for the weak-$*$ topology.\\

As the fibers are discrete, any $\nu\in P_K(G)$ can be represented as as sum $\nu = \sum_{g\in G^x} \lambda_g(\nu)\delta_g$, where $\delta_g$ is the Dirac measure at $g\in G $. The continuous functions $\lambda_g$ are called coordinate functions and satisfy $\sum_{g\in G^x} \lambda_g(\nu) =1$ for every $x\in G^{(0)}$. The action of $G$ is given by $g\lambda_h = \lambda_{g^{-1}h}$.

\begin{lem}[Tu,\cite{TuBC2}]
The action of $G$ on $P_K(G)$ is proper and cocompact.
\end{lem}  

\begin{lem}[Tu,\cite{TuBC2}]\label{Gspace}
Let $Z$ be a proper $G$-compact $G$-space. Then there exists a compact subset $K\subseteq G$ and a $G$-equivariant continuous map $Z\rightarrow P_K(G)$.
\end{lem}  

\subsection{Equivariant $K$-homology}

We will use the equivariant $KK$-theory developped by Le Gall in his thesis \cite{LeGall}, which is an extension of the usual equivariant $KK$-theory of Kasparov. Recall that, if $A$ and $B$ are two $G$-algebras, elements of $KK^G(A,B)$ are homotopy classes of triple $(E,\pi,T)$ where :\\

\begin{itemize}
\item[$\bullet$] $E$ is a $G$-module, the action of $G$ being implemented by a unitary $V\in\mathcal L_{s^* B}(s^* E, r^* E)$,
\item[$\bullet$] $\pi : A\rightarrow \mathcal L_B(E)$ is a $G$-equivariant $*$-homomorphism,
\item[$\bullet$] $T\in \mathcal L_B(E)$ is an adjointable operator such that the triple satisfies the $K$-cycle conditions : $\pi(a)(T^2-T),\pi(a)(T^*-T),[\pi(a),T]$ are compact operators in $\mathfrak K(E)$, and $\pi(a)(r^*T-V s^*T V^*)$ are compact operators in $\mathfrak K_{r^* B}(r^* E)$ for all $a\in A, g\in G$.\\
\end{itemize}

\begin{definition}
Let $Y$ be a proper $G$-space and $B$ a $G$-algebra. Then the analytic $K$-homology of $Y$ with coefficients in $B$ is defined by 
\[RK^G(Y,B) = \varinjlim_{Z\subseteq Y} KK^G(C_0(Z),B), \]
the inductive limit being taken on proper $G$-compact $G$-subspaces $Z$ of $Y$. The analytic $K$-homology of $G$ is defined as the analytic $K$-homology of $\underline E G$. In the litterature, the notation $K^{top}(G,B)$ is common to denote $RK^G(\underline E G,B)$.
\end{definition}

The previous lemmas ensures that the analytic $K$-homology of $G$ can be computed as an inductive limit of the $K$-homology of the spaces $P_K(G)$ :
%somehow restricted when it comes to the analytic $K$-homology of any universal space $\underline E G$ :
\begin{lem} Let $B$ be a $G$-algebra. Then :
\[RK^G(\underline E G,B) = \varinjlim_{K\subseteq G \text{ compact }} KK^G(C_0(P_K(G)),B).\]
\end{lem}

\begin{dem}
Let $Z$ and $Z'$ be two proper $G$-compact $G$-spaces such that $Z\subseteq Z'$. By lemma \ref{Gspace}, there exists compact subsets $K$ and $K'$ of $G$ and $G$-equivariant continuous maps $h : Z\rightarrow P_K(G)$ and $h : Z'\rightarrow P_{K'}(G)$. As $h'_{|Z} = h $, the following diagram commutes
\[\begin{tikzcd}
KK^G(C_0(Z),B)\arrow{r}\arrow{d}{h^*} & KK^G(C_0(Z'),B)\arrow{d}{(h')^*} \\
RK^G(P_{K}(G),B)\arrow{r} & RK^G(P_{K'}(G),B) \\
\end{tikzcd},\]
hence $\varinjlim_{K\subseteq G \text{ compact }} KK^G(C_0(P_K(G)),B)$ and $\varinjlim_{Z\subseteq Y} KK^G(C_0(Z),B)$ are equal.\\
\qed
\end{dem}

\subsection{Descent functor}

There exists a natural transformation $KK^G(A,B)\rightarrow KK(A\rtimes_r G, B\rtimes_r G)$ respecting the Kasparov product, which is called the descent functor. The same statement remains true for maximal crossed products. We recall its construction, which was first stated in \cite{LeGall}.\\
 
Let $A$ and $B$ be two $G$-algebras and $(E,\pi,T)\in E^G(A,B)$ be a $K$-cycle. Define :
\begin{itemize}
\item[$\bullet$] $E_G$ to be the $B\rtimes_r G$-Hilbert module $E\otimes_B (B\rtimes_r G)$,
\item[$\bullet$] $\pi_G = \pi\otimes_{A}id_{A\rtimes_r G} $ to be the image of $\pi:A \rightarrow \mathcal L_B(E)$ under the reduced crossed-product functor,
\item[$\bullet$] $T_G = T\otimes_{B}1$ to be the image of $T\in\mathcal L(E)$ under the reduced crossed-product functor.
\end{itemize}
Then $(E_G,\pi_G,T_G)\in E(A\rtimes_r G,B\rtimes_r G)$ and the map $(E,\pi,T)\mapsto (E_G,\pi_G,T_G) $ induces a homorphism of abelian groups
\[j_G : KK^G(A,B)\rightarrow KK(A\rtimes_r G,B\rtimes_r G)\]
satisfying $j_G(z\otimes_D z')=j_G(z)\otimes_{D\rtimes_r G} j_G(z')$ for any $z\in KK^G(A,D), z'\in KK^G(D,B)$.\\ 

\subsection{The assembly map}

If $Z$ is a proper $G$-compact $G$-space, $Z\rtimes G$ is a proper groupoid by definition, hence there exists a cutoff function $c : Z\rightarrow [0,1]$ such that $\sum_{g\in G^{p(z)}} c(zg) = 1$. The function $g\mapsto c(r(g))^{\frac{1}{2}}c(s(g))^{\frac{1}{2}} $ defines a projection in $C_0(Z)\rtimes_r G$ which we denote by $\mathcal L_Z$. If $Z=P_K(G)$, then $\mathcal L_Z = \mathcal L_K$.

\begin{definition}
The assembly map for $G$ with coefficients in $B$ is defined as the inductive limit of the maps $\mu_{G,B}^{(Z)} : KK^G(C_0(Z),B)\rightarrow K(B\rtimes_r B)$ given by
\[\mu_{G,B}^{(Z)} (z)=[\mathcal L_Z]\otimes_{C_0(Z)\rtimes G} j_G(z),\]
that is $\mu_{G,B} = \varinjlim \mu_{G,B}^{(Z)}$ (one has to check that theses maps respects the inductive systems, which they do).\\
One can restrict to $Z$ of the form $P_E(G)$ for $E\subseteq G$ compact. We will denote by $\mu_{G,B}^E$ the assembly map $\mu_{G,B}^{P_E(G)}$, and $\mu_{G,B} = \varinjlim \mu_{G,B}^{E}$ still holds.
\end{definition}

%Another definition : with $Ad$...

\subsection{The Baum-Connes conjecture}

\begin{conj}[Baum-Connes conjecture]
Let $G$ be an étale groupoid, and $A$ a $G$-algebra.\\
The Baum-Connes conjecture for $G$ with coefficients in $A$ is the following claim : \\
\center{$\textbf{BC}_{G,A}$ : $\mu_{G,A}$ is an isomorphism of abelian groups.}\\
The Baum-Connes conjecture with coefficients is : for all $G$-algebras $A$, $\mu_{G,A}$ is an isomorphism of abelian groups.
\end{conj}

The conjecture was first stated for groups in \cite{BaumConnes} and in \cite{BaumConnesHigson}. The statement is a generalization of the Connes-Kasparov conjecture, which is simply, when one knows the Baum-Connes conjecture, the latter for almost connected groups (i.e. locally compact groups $G$ such that $G/G_0$ is compact, where $G_0$ is the connected component of the identity). J.-L. Tu stated the conjecture for general locally compact, $\sigma$-compact, Hausdorff groupoids with Haar systems.\cite{TuBC} \\

Here is the status of the conjecture :
\begin{itemize}
\item[$\bullet$] N. Higson and G. Kasparov proved that the conjecture with coefficients holds for groups having Haagerup property \cite{higsonkasparov}. 
\item[$\bullet$] V. Lafforgue proved the conjecture with coefficients for hyperbolic groups \cite{lafforgue2012conjecture}.
\item[$\bullet$] V. Lafforgue showed the conjecture for every semi-simple Lie group, real or $p$-adic reductive, and discrete cocompact subgroup of real rank $1$ Lie groups or of $SL(3,k)$ for any local field $k$ \cite{lafforgueT}. 
\item[$\bullet$] The Connes-Kasparov conjecture was established by J. Chabert, S. Echterhoff and R. Nest, who proved that $\mu_{G}$ is an isomorphism for every secound countable almost connected group $G$, and every group of $k$-rational points of a linear algebraic group over a local field of characteristic $0$ \cite{chabertEN}.\\
\end{itemize}

Recall that a function $h : G\rightarrow \R$ is said to be of negative type if :
\begin{itemize}
\item[$\bullet$] $h_{|G^{(0)}}=0$,
\item[$\bullet$] $h(g)=h(g^{-1})$ for all $g\in G$,
\item[$\bullet$] for all $t_i\geq 0, i=1,..,n$ such that $\sum t_i=0$ and all $g_1,...,g_n\in G$ having the same range, $\sum_i t_i t_j h(g_i^{-1} g_j )\leq 0$.
\end{itemize}
The function $h$ is said to be locally proper if, for every compact $K \subseteq G^{(0)}$, $h_{|G_K^K}$ is proper.\\

\begin{definition}
A topological groupoid $G$ is said to be a-T-menable if it satisfies one of the following equivalent property :
\begin{itemize}
\item[$\bullet$] there exists a continuous field of Hilbert spaces over $G^{(0)}$ with a proper affine action of $G$,
\item[$\bullet$] there exists a locally proper negative type function on $G$.
\end{itemize} 
\end{definition}

For nice topological groupoids, the conjecture is known to be false in full generality \cite{HigsonLaffSk}, but the following result was shown by J-L. Tu in \cite{TuThese}
\begin{thm}\label{Tu}
Let $G$ be a locally compact $\sigma$-compact Hausdorff groupoid. Then if $G$ is a-T-menable, the Baum-Connes conjecture with coefficients holds for $G$.
\end{thm}

%%%%%%%%%%%%%%%%%%%%%%%%%%%%%%%%%%%%%%%%%
\subsection{The Coarse Assembly map}
%%%%%%%%%%%%%%%%%%%%%%%%%%%%%%%%%%%%%%%%%

The construction of the coarse assembly map is very similar to the assembly map for groupoids, but relies on different functors.\\

The metric space $X$ is discrete, hence $l^2(X)$ is generated by $e_x = (\delta_{xy})_{y\in X}\in l^2(X)$. For $x,y\in X$ and $T\in \mathcal L_B(l^2(X)\otimes H_B)$, $T_{x,y}$ is defined as the unique operator in $\mathcal L_B(H_B)$ satisfying $\langle T_{xy}\xi,\eta\rangle = \langle T (e_x\otimes \xi),e_y\otimes \eta \rangle$ for all $\xi,\eta\in H_B$.\\

Recall how to define the Roe algebra $C^*(X,B)$ of a discrete metric space with bounded geometry $X$ with coefficients in an arbitrary $C^*$-algebra $B$. For $R>0$, define $C_R[X,B]$ to be the involutive subspace of operators $T\in \mathcal L_B(l^2(X)\otimes H\otimes B)$ such that $T_{xy}\in \mathfrak K_B(H\otimes B)$ and $T_{xy}=0$ as soon as $d(x,y)>R$, and $\C[X,B]= \cup_{R>0} C_R[X,B]$. 

\begin{definition}
The Roe algebra $C^*(X,B)$ is defined as the Hibert $B$-module obtained after the completion of $\C[X,B]$ with respect to the operator norm in $\mathcal L_B(l^2(X)\otimes H\otimes B)$.
\end{definition} 
%%%%%%%%%%%%%%%%%%%%%%%%%%%%%%%
This construction is functorial : any $*$-homomorphism $\phi: A\rightarrow B$ gives rise to a $*$-homomorphism $\phi_X : C^*(X,A)\rightarrow C^*(X,B)$. For the reader's convenience, we give details for the construction of $\phi\mapsto \phi_X$, which is a standard fact in Coarse Geometry.

\begin{thm}\label{Xfunctor}
Let $X$ be a discrete metric space with bounded geometry and $\phi : A\rightarrow B$ a $*$-homomorphism. Then there exists a $*$-homomorphism $\phi_X : C^*(X,A)\rightarrow C^*(X,B)$ extending $\phi$. Moreover, $\phi\mapsto \phi_X$ respects composition of $*$-homomorphisms.
\end{thm}

\begin{dem}
Recall that any $*$-morphism $\phi : A\rightarrow B$ induces, for any $A$-Hilbert module $E$, a $*$-morphism $\phi_* : \mathcal L_A(E)\rightarrow \mathcal L_B(E\otimes_A B)$. Now take $E$ to be $l^2(X)\otimes A$. Then $\eta\otimes a\otimes b\mapsto \eta \otimes\phi(a) b $ extends to an isometry $V\in \mathcal L_B(E\otimes_A B,l^2(X)\otimes B)$.\\
The linear map $T \mapsto V\phi_*(T)V^*$ maps $C_R[X,A]$ into $C_R[X,B]$, and so extends to a $*$-morphism $C^*(X,A)\rightarrow C^*(X,B)$. The composition property is clear from the construction.\\
\qed
\end{dem}
%%%%%%%%%%%%%%%%%%%%%%%
This functoriality extends to $KK$-theory.

\begin{thm}\label{sigma} There exists a natural transformation 
\[\sigma_X : KK(A,B)\rightarrow KK(C^*(X,A),C^*(X,B))\]
which transforms the Kasparov product into composition, i.e. 
\[\sigma(z\otimes_D z')= \sigma_X(z')\circ \sigma_X(z) \quad \forall z\in KK(A,D),z'\in KK(D,B),\]
where the $KK$-elements $\sigma(z\otimes_D z')$, $ \sigma_X(z)$ and $\sigma_X(z')$ are seen as the maps they induce in $K$-theory. 
\end{thm}

\begin{rk}
This relation has to be thought as if $\sigma_X$ respects the Kasparov product. The problem is that $C^*(X,A)$ is not a separable $C^*$-algebra in general, a condition which is required for the domain of the left element in the Kasparov product. As $K(A) \cong KK(\C,A)$ for every $C^*$-algebra $A$, if $x\in K(C^*(X,A))$, $x\otimes_{C^*(X,A)}\sigma_X(z)\in KK(\C,C^*(X,B))\cong K(C^*(X,B))$ is well defined. The map $x\mapsto x\otimes_{C^*(X,A)}\sigma_X(z)$ is what we called the map induced by $\sigma_X(z)$ in $K$-theory.
\end{rk}

Let us recall the construction of $\sigma_X$. Let us first notice that $C^*(X;B\otimes\mathfrak K)\cong C^*(X,B)$. Indeed, $H$ being the separable Hilbert space, any linear isomorphism induces a $*$-isomorphism $\mathfrak K \otimes \mathfrak K \cong \mathfrak K$. An element $T\in C_R[X,B\otimes\mathfrak K]$ is then an operator such that $T_{xy}\in \mathfrak K_{B\otimes\mathfrak K}= \mathfrak K\otimes \mathfrak K \otimes B \cong \mathfrak K\otimes B$ for every $x,y\in X$. As the propagation is on the $l^2(X)$ factor, this induces the claimed isomorphism. \\

Let $(H_B,\pi,T)\in\mathbb E(A,B)$ be a $K$-cycle. Put $E = l^2(X)\otimes H\otimes A$. We can construct the internal tensor product  $E\otimes_\pi H_B$ with the help of the $*$-homomorphism $\pi : A \rightarrow \mathcal L_B(H_B)$. We also dispose of a $*$-homomorphism $\mathcal L_A(E)\rightarrow \mathcal L_B(E\otimes_\pi H_B)$. But the map 
\[(\xi\otimes a)\otimes_\pi \eta \in E\otimes_\pi H_B \mapsto \xi\otimes (\pi(a)\eta)\in l^2(X)\otimes H\otimes H_B \] %\cong l^2(X)\otimes H_B
extends to an isometry $E\otimes_\pi H_B \rightarrow l^2(X)\otimes H\otimes  H_B$. As in remark \ref{isometry}, the conjugation by this isometry sends compact operators to compact operators. Moreover, it does not alter propagation, hence $\mathcal L_A(E)\rightarrow \mathcal L_B(E\otimes_\pi H_B)\subseteq \mathcal L_B(l^2(X)\otimes H\otimes H_B)$ induces a map $\tilde \pi : C^*(X,A)\rightarrow \mathcal \mathcal M(C^*(X,B\otimes\mathfrak K))\cong \mathcal L_{C^*(X,B)}(C^*(X,B))$. Define $\tilde T$ as $1\otimes T$ acting on $l^2(X)\otimes H \otimes H_B$. Then $(l^2(X)\otimes H\otimes B,\tilde \pi,\tilde T)\in \mathbb E(C^*(X,A),C^*(X,B))$ and we put $\sigma_X([H_B,\pi,T])=[l^2(X)\otimes H\otimes B,\tilde \pi,\tilde T]\in KK(C^*(X,A),C^*(X,B))$.\\

Let $X$ be a coarse space with bounded geometry. The Rips complex is the inductive system of finite dimensional simplicial complexes
\[P_E(X)=\{\eta\in \text{Prob}(X)\text{ s.t. supp }\eta\subseteq E\}\quad,\forall E\in \mathcal E_X,\]
endowed with the $*$-weak topology. It is an inductive system with respect to inclusion of entourages.\\
 
Let us recall the definiton of the left-hand side of the coarse assembly map. The coarse homology group of $X$ with coefficients in an arbitrary $C^*$-algebra $B$ is defined as 
\[KX_*(X,B)=\varinjlim_{E\in \mathcal E_X} KK(C_0(P_E(X)),B)\]
where the limit is taken along the inductive system $\{P_E(X)\}_{E\in\mathcal E_X}$. \\

Let $E\in \mathcal E_X$. Any simplex $\eta\in P_E(X)$ can be written as a finite sum $\eta=\sum_x \lambda_x(\eta)\delta_x$ by boundedness of the geometry. Here, $\delta_x$ is the Dirac measure at $x\in X$. The functions $\lambda_x :\eta\mapsto \lambda_x(\eta)$ are continuous and satifisfy $\sum\lambda_x(\eta) = 1,\forall\eta\in P_E(X)$, so that
\[V_0\left\{\begin{array}{rcl} 
 C_0(P_E(X)) & \rightarrow 	& l^2(X)\otimes C_0(P_E(X)) 		\\ 
 f           & \mapsto 		& (\lambda_x^{\frac{1}{2}}f)_{x\in X} 
\end{array}\right.\] 
is an isometry of $C_0(P_E(X))$-modules, and $V_0V_0^*\otimes id_H$ defines a projection in $l^2(X)\otimes H\otimes C_0(P_E(X))$ with finite propagation (less than $R= \sup_E d$, because $\text{supp }(V_0 V_0^*)\subseteq E$), so a class $[\mathcal L_E]\in K_0(C^*(X,C_0(P_E(X)))$.\\

\begin{definition}
The coarse assembly map $\mu:KX_*(X,B)\rightarrow K_*(C^*(X,B))$ for $X$ with coefficients in $B$ is defined as 
\[\forall z\in RK(P_E(X), B), \mu_{X,B}(z)=[\mathcal L_E]\otimes_{C^*(X,C_0(P_E(X)))} \sigma_X(z).\]
\end{definition}

The coarse Baum-Connes conjecture is the following claim.\\

\begin{conj}[Coarse Baum-Connes conjecture]
For any coarse space $X$ with bounded geometry, the coarse assembly map $\mu_{X,\C}$ is an isomorphism.\\
\end{conj}

%%% Descent principle

One of the interesting properties of the coarse Baum-Connes conjecture is the following application to topology, which is called the descent principle. The reader is referred to J. Roe's book \cite{RoeIndex} for details. Recall that if $\Gamma$ is a finitely generated group, $|\Gamma|$ denotes the metric space obtained by choosing a set of generators and endowing $\Gamma$ with the  associated word-length metric. Also, $B\Gamma$ denotes the classifying space of $\Gamma$, which is defined in any standard textbook on algebraic topology, see \cite{May} for instance. Definitions ans statements of the Novikov conjectures can be found in \cite{ferrynovikov}.

\begin{thm}[Descent principle]
Let $\Gamma$ be a finitely generated group such that the classifying space $B\Gamma$ has the homotopy type of a finite CW-complex. Then, the coarse Baum-Connes conjecture for $|\Gamma |$ implies the strong Novikov conjecture for $\Gamma$. 
\end{thm}

Here are some remarks about the status of the conjecture :
\begin{itemize}
\item[$\bullet$] the conjecture is known to hold for any coarse space that admits a coarse embedding into a  separable Hilbert space\cite{Yu2}.
\item[$\bullet$] counterexamples have been constructed \cite{HigsonLaffSk}.\\
%\item[$\bullet$] Let $\Gamma$ be a finitely generated group, and $|\Gamma|$ the coarse space associated to the word-length metric (any two left invariant metric on $\Gamma$ are coarsely equivalent). Then the Coarse Baum-Connes conjecture for $|\Gamma|$ implies the Novikov conjecture for higher signatures for $\Gamma$.\\
\end{itemize}

Here is another description of the coarse assembly map which will be of some interest for us.\\

Let $E\in \mathcal E_X$ be an entourage. Take a cycle $(H,\pi, T)\in E(C_0(P_E(X)), \C)$, and equip the finite dimensional simplicial complex $P_E(X)$ with a metric which restricts to the spherical metric on simplices, so that $H$ is a $P_E(X)$-module that we can suppose standard non-degenerate. Then 
\[T'=\sum \lambda_x^{\frac{1}{2}} T \lambda_x^{\frac{1}{2}} \] 
is a compact perturbation of $T$ in $\mathcal L(H)$ with bounded propagation, and is invertible modulo $C^*(P_E(X),H)$, so defines a class $[T']\in K_0(C^*(P_E(X),H))$. 
%But $P_E(X)$ and $X$ are coarsely equivalent, the barycentric map $P_E(X)\rightarrow X$ is a coarse equivalence which is covered by $V_0$, so that 
The same formula for $V_0$, i.e. $V_0f = (\lambda_x^{\frac{1}{2}}f)_{x}$ defines an isometry of Hilbert spaces, still denoted $V_0$, from $l^2(P_E(X))$ to $l^2(X)\otimes l^2(P_E(X))$. Multiplication by $C_0(P_E(X))$ on the $l^2(P_E(X))$ factor endows these Hilbert spaces with a structure of $P_E(X)$-module, with respect to which the propagation of $V_0$ is bounded by $s = \sup_E d$. Put $V = 1\otimes V_0 $ which defines a $*$-homomorphism
\[Ad_{V} :\left\{\begin{array}{rcl} 
\mathcal L(H\otimes l^2(X)\otimes l^2(P_E(X)) ) & \rightarrow & \mathcal L(H\otimes l^2(X)\otimes l^2(P_E(X)) ) \\
T &\mapsto & VTV^* \\
\end{array}\right.\]
which sends operators of propagation $R$ to operators of propagation $R+2s$ and respects local compactness. Hence, it induces a $*$-homomorphism $C^*(P_E(X))\rightarrow C^*(X)$ and a homomorphism
\[(Ad_{V})_* : K_*(C^*(P_E(X)))\rightarrow K_*(C^*(X)).\]
%is an isomorphism, which does only depend on the coarse class of the barycentric map.\\

With that in mind, one can show that $\mu_{X}([H,\pi,T])=(Ad_{V_0})_*[T']$.\\

Indeed, with the same notations and if $z=[H,\pi,T]\in KK(C_0(P_E(X)),\C)$, we want to compute the Kasparov product $[\mathcal L_E]\otimes \sigma_X(z)$. According to Proposition $8.7.2$ of \cite{HigsonRoe}, this is equal to the Fredholm index of $P(1\otimes T)P$ where $P=VV^*$. A simple computation shows that $V^*(1\otimes T)V=T'$ so that $P(1\otimes T)P = Ad_V(T')$, hence the equality.\\

The point of the article \cite{SkTuYu} is to show that the coarse assembly map is equivalent to the assembly map for the coarse groupoid $G=G(X)$ with coefficients in the $G$-algebra $l^\infty=l^\infty(X,\mathfrak K)$. \\

More precisely, let $x\in X$. Then $Z=P_{\overline E}(G)$ is a proper cocompact $G$-space with fiber $Z_x\simeq P_E(X)$, and $l^\infty_x \simeq \mathfrak K$, so that the inclusion of groupoid $\{x\}\hookrightarrow G $ induces a morphism $\iota_Z : RK^G(Z,l^\infty)\rightarrow KK(Z_x,\mathfrak K)$ which is actually an isomorphism. Recall that one can construct a $*$-isomorphism $\Psi_X: l^\infty \rtimes_r G\rightarrow C^*(X)$, and $\mu_X^{(Z_x)}\circ \iota_Z = (\Psi_X)_*\circ \mu_{G,l^\infty}^{(Z)}$ holds and respects inductive limits over $Z$ to give the following commutative diagram :
\[\begin{tikzcd}
K_*^{top}(\underline E G, l^\infty) \arrow{r}{\mu_{G,l^\infty}}\arrow{d}{\varinjlim \iota_Z} & K_*(l^\infty\rtimes_r G)\arrow{d}{(\Psi_X)_*} \\
KX_*(X)\arrow{r}{\mu_X} & K_*(C^*(X))
\end{tikzcd}\]  
with vertical arrows being isomorphism. We will give another proof of this result, based on controlled $K$-theory. Indeed, we actually prove a stronger result in theorem \ref{BCCeq}, which induces the previous result. \\

A key point is to use how analytical properties of $G$ translate coarse properties of $X$. In particular, the results of theorem \ref{propertiesXG}, combined with theorem \ref{Tu}, give examples of coarse spaces satisfying the coarse Baum-Connes conjecture and the boundary coarse Baum-Connes conjecture.

%\begin{itemize}
%\item[$\bullet$] in \cite{SkTuYu} is proved that $X$ admits a coarse embedding into Hilbert space iff $G(X)$ is a-T-menable, 
%\item[$\bullet$] in \cite{SkTuYu} is proved that $X$ has property A iff $G(X)$ is amenable, 
%\item[$\bullet$] in \cite{FinnSellFibred}, M. Finn-Sell shows that if $X$ admits a coarse fibered embedding, then $G_\partial = G(X)_{|\partial \beta X}$ is a-T-menable. 
%\end{itemize}
















