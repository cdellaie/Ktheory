\section{Controlled $K$-theory}

In this section, we define controlled $K$-theory in a little more generality than the functor defined in \cite{OY2}, so that the propagation can be indexed by compact subsets of an étale groupoid instead of positive real numbers.\\

\begin{definition}
A $C^*$-algebra $A$ is said to be filtered if there exists a partially ordered semigroup $\mathcal E$ and, for every $E\in \mathcal E$, linear subspaces $A_E$ of $A$ such that :\\
\begin{itemize}
\item[$\bullet$] if $E \leq E'$, then $A_E\subset A_{E'}$, and the inclusion $\phi_E^{E'}: A_E\hookrightarrow A_{E'}$ induces an inductive system of linear spaces,
\item[$\bullet$] $A_E$ is stable by involution,
\item[$\bullet$] for all $E,E'\in\mathcal E$, $A_E.A_{E'}\subset A_{EE'}$,
\item[$\bullet$] the union of subspaces is dense in $A$, i.e. $\overline{\cup_{E\in\mathcal E}A_E} = \varinjlim A_E = A$.
\item[$\bullet$] if $A$ is unital, we impose that $1\in A_E,\forall E\in\mathcal E$.\\
\end{itemize}
\end{definition}

\subsection{Almost unitaries and almost projections}

\begin{definition}
Let $(A,\mathcal E)$ be a unital filtered $C^*$-algebra. The set of $\varepsilon$-$E$-unitaries is the set 
\[U^{\varepsilon, E}(A)= \{u\in A_E \text{ s.t. } ||u^*u-1||<\varepsilon\text{ and }||uu^*-1||<\varepsilon \}\]
and the set $\varepsilon$-$E$-projections is the set 
\[P^{\varepsilon, E}(A)= \{p\in A_E \text{ s.t. } p=p^*\text{ and }||p^2-p||<\varepsilon \}.\]
We will use the notation $P_n^{\varepsilon, E}(A)$ for $P^{\varepsilon, E}(M_n(A))$, and $U_n^{\varepsilon, E}(A)$ for $U^{\varepsilon, E}(M_n(A))$. Also, $P_\infty^{\varepsilon, E}(A)$ is the algebraic inductive limit of the $P_n^{\varepsilon, E}(A)$ under the natural inclusions
\[\left\{\begin{array}{rcl}
	P^{\varepsilon,E}_n(A) 		& \rightarrow	& P^{\varepsilon,E}_{n+1}(A)\\ 
	p 		& \mapsto 	& \begin{pmatrix}p& 0 \\ 0&0 \end{pmatrix}
\end{array}\right.\]
and $U_\infty^{\varepsilon, E}(A)$ is the algebraic inductive limit of the $U_n^{\varepsilon, E}(A)$ under the natural inclusions
\[\left\{\begin{array}{rcl}
	U^{\varepsilon,E}_n(A) 		& \rightarrow	& U^{\varepsilon,E}_{n+1}(A)\\ 
	u 		& \mapsto 	& \begin{pmatrix}u & 0 \\ 0& 1 \end{pmatrix}
\end{array}\right. .\]
\end{definition}

In order to define controlled $K$-groups, define the following equivalence relations on $P^{\varepsilon, E}_\infty(A)\times \N$ and $U^{\varepsilon,E}_n(A)$.
\begin{itemize}

\item[$\bullet$] $(p,l) \sim (q,l')$ if there exists a homotopy of almost projections $h\in P^{\varepsilon, E}_\infty(A[0,1])$ and an integer $k$ such that 
\[h(0)=\begin{pmatrix} p & 0 \\ 0 & 1_{k+l'} \end{pmatrix} \text{ and }
h(1)=\begin{pmatrix} q & 0 \\ 0 & 1_{k+l} \end{pmatrix}\]
\item[$\bullet$] $u \sim v$ if there exists a homotopy of almost unitaries $h\in U^{3\varepsilon, E\circ E}_\infty(A[0,1])$ and an integer $k$ such that $h(0)= u \text{ and }h(1)=v$.\\
\end{itemize}

Denote $[(p,l)]_{\varepsilon,E}$ and $[u]_{\varepsilon,E}$ for the equivalence classes of almost-projections and almost-unitaries. Then, the same proof as \cite{OY2} shows that $[p,l]_{\varepsilon,E}+[q,l']_{\varepsilon,E}=[diag(p,q),l+l']_{\varepsilon,E}$ and $[u]_{\varepsilon,E}+[v]_{\varepsilon,E}=[diag(u,v)]_{\varepsilon,E}$ induces a group law on the equivalence classes, that we denote $K_0^{\varepsilon,E}(A) = P^{\varepsilon, E}_\infty(A)\times \N / \sim$ and $K_1^{\varepsilon,E}(A) = U^{\varepsilon, E}_\infty(A) / \sim$.\\

\begin{definition}
The controlled $K$-theory of a filtered $C^*$-algebra $(A,\mathcal E)$ is the family of abelian groups $\hat K_0(A) = (K_0^{\varepsilon,E}(A))_{\varepsilon\in (0,\frac{1}{4}),E\in\mathcal E}$ and $\hat K_1(A) = (K_1^{\varepsilon,E}(A))_{\varepsilon\in (0,\frac{1}{4}),E\in\mathcal E}$ defined above.\\
\end{definition}

We defined canonical inclusion morphisms : if $\varepsilon, \varepsilon'\in (0,\frac{1}{4})$ and $E,E'\in\mathcal E$ such that $\varepsilon < \varepsilon'$ and $E \leq E'$, the natural inclusion $K_*^{\varepsilon,E}(A)\hookrightarrow K_*^{\varepsilon',E'}(A)$ is denoted by $\iota_{\varepsilon,E}^{\varepsilon',E'}$. Notive that $\iota_{\varepsilon',E'}^{\varepsilon'',E''}\circ\iota_{\varepsilon,E}^{\varepsilon',E'}=\iota_{\varepsilon,E}^{\varepsilon'',E''}$ when this expression makes sense.\\

One has also forgetful inclusions $\iota_{\varepsilon,E} : K_*^{\varepsilon,E}\rightarrow K_*(A)$, and 
$\iota_{\varepsilon',E'}\circ\iota_{\varepsilon',E'}^{\varepsilon'',E''}=\iota_{\varepsilon, E}$ holds.\\

