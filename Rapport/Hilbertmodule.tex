\section{Hilbert modules}

The $KK$-theory of Kasparov can be defined using the theory of Hilbert module. We present in this section the basic facts about Hilbert modules that we will need.

\begin{definition}
Let $A$ be a $C^*$-algebra. A pre-Hilbert $A$-module is a right $A$-module $E$ endowed with a sesquilinear function $\langle\cdot, \cdot \rangle : E\times E \rightarrow A$ satisfying :
\begin{itemize}
\item[$\bullet$] $\langle x, ya\rangle = \langle x,y\rangle a $ for all $x,y\in E$ and $a\in A$,
\item[$\bullet$] $\langle x,y\rangle^* = \langle y,x\rangle$ for all $x,y\in E$,
\item[$\bullet$] $\langle x ,x\rangle \geq 0 $ and if $\langle x,x\rangle = 0$, then $x=0$, for all $x\in E$.
\end{itemize}
For $x\in E$, we define $||x|| = ||\langle x,x\rangle ||^{\frac{1}{2}}$. It is a norm on $E$.\\

A $A$-Hilbert module $(E,\langle \cdot,\cdot \rangle)$ is a pre-Hilbert module which is complete for the norm $||\cdot ||$ defined above. $E$ is called a full Hilbert module if the linear span of $\{ <x,y> : x,y\in E\}$ is dense in $A$.
\end{definition}

The same definition exists for left-Hilbert modules. Usually, we will omit to precise right or left, as most of our modules will be right modules. By abuse of language, we will call the sesquilinear function $\langle \cdot,\cdot \rangle$ an inner product.

\begin{Expl} If $J$ is any closed right ideal in $A$, $J$ is naturally a $A$-Hilbert module with respect to the inner product $\langle a, b \rangle = a^* b$. In particular, $A$ is a $A$-Hilbert module.
\end{Expl}

\begin{Expl}
If $\{E_j\}_{j\in J}$ is a finite of countable family of $A$-Hilbert module, define the direct sum as
\[\bigoplus_{j\in J} E_j = \{(x_j) : x_j\in E_j \text{ s.t. } \sum_j ||x_j||_{E_j} \text{ converges in } A\}\]
with the inner product $\langle x,y \rangle = \sum_{j\in J} \langle x_j, y_j\rangle_{E_j} $. It is a $A$-Hilbert module. As a particular case, we get a central example $H_A$ called the \textbf{standard $A$-Hilbert module}. It is obtained as the direct sum of the family $\{A\}_{j\in \N}$ where each $A$ is seen as a $A$-Hilbert module.
\end{Expl}

\begin{definition}
Let $E$ and $F$ be two $A$-Hilbert modules. The set of adjoinable operators $\mathcal L_A(E,F)$ is defined as the set of maps $T : E\rightarrow F $ such that there exists a map $T^* : F\rightarrow E$ satisfying 
\[\langle Tx, y \rangle = \langle x ,T^* y\rangle\quad \forall x\in E,y\in F.\]
The operator $T^*$ is called the adjoint of $T$.
\end{definition}

\begin{rk}
Notice that any adjoinable operator is automatically a module map, and a linear map. Indeed, $\langle T(xa),y\rangle = \langle xa,T^* y \rangle = a^* \langle x, T^* y\rangle =a^*\langle Tx,  y\rangle =  \langle a(Tx),y \rangle$.
\end{rk}

