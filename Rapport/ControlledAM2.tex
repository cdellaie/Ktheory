\section{Controlled assembly maps for étale groupoids}

In this section, we will always use a coarse structure $\mathcal E$ of a locally compact $\sigma$-compact étale groupoid generated by a countable subset of its compact symmetric subsets such that for every compact subset $K\subseteq G$, there exists a $E\in\mathcal E$ such that $K\subseteq E$.

%%%%%%%%%%%%%%%%%%
\subsection{Kasparov transform} %%
%%%%%%%%%%%%%%%%%%

%Let $A$ and $B$ be two $G$-$C^*$-algebras, and $H$ a separable Hilbert space, $l^2(\Z)$ for instance, and $H_G= H\otimes L^2(G,\lambda)$. The standard Hilbert module over $B$ is denoted by $H_B=H_G\otimes B$, and $K_B$ is the algebra of compact operators for $H_B$, i.e. $K(H)\otimes L^2(G,\lambda)\otimes B$. \\

Let $A$ and $B$ be two $G$-$C^*$-algebras. If $\beta : s^* B \rightarrow r^* B$ is the action of $G$, notice that the canonical action on the standard Hilbert module $H_B$ is given by $V=id_H\otimes \beta : s^* H_B \cong H_{s^* B} \rightarrow r^* H_B \cong H_{r^* B}$. \\

Recall that every $K$-cycle $z\in KK^G(A,B)$ can be represented as a triple $(H_B, \pi, T)$ where :
\begin{itemize}
\item[$\bullet$]$\pi : A\rightarrow \mathcal L_B(H_B)$ is a $*$-representation of $A$ on $H_B$.
\item[$\bullet$]$T\in \mathcal L_B(H_B)$ is a self-adjoint operator.
\item[$\bullet$] $T$ and $\pi$ satisfy the $K$-cycle condition, i.e. $[T,\pi(a)]$, $\pi(a)(T^2-id_{H_B})$ are compact operatros in $\mathfrak K_B(H_B)$ and $\pi(a)(r^*T-Vs^*T V^*)$ are compact operators in $\mathfrak K_{r^* B}(r^* H_B)\cong \mathfrak K_{r^* B}(H_{r^* B})$ for all $a\in A, g\in G$.\\
\end{itemize}

Set $T_G= T\otimes id_{B\rtimes_r G}\in \mathcal L_{B\rtimes_r G}(H_B\otimes (B\rtimes_r G))\simeq \mathcal L_{B\rtimes_r G}(H_{B\rtimes_r G})$, and $\pi_G: A\rtimes_r G \rightarrow L_{B\rtimes_r G}(H_{B\rtimes_r G})$. Then, according to Le Gall \cite{LeGall}, $(H_{B\rtimes_r G}, \pi_G, T_G)$ represents the $K$-cycle $j_G(z)\in KK(A\rtimes_r G,B\rtimes_r G)$. Let us construct a controlled morphism associated to $z$,
\[J_G(z) : \hat K(A\rtimes_r G)\rightarrow \hat K(B\rtimes_r G), \]
which induces right multiplication by $j_G(z)$ in $K$-theory.

\subsubsection{Odd case}

Let us first do the work for $z\in KK_1^G(A,B)$. Let $(H_B,\pi,T)$ be a $K$-cycle representing $z$. Set $P=\frac{1+T}{2}$ and $P_G=P\otimes id_{B\rtimes_r G}$. We define
\[E^{(\pi,T)}=\{(x,P_G\pi_G(x)P_G + y) : x\in A\rtimes_r G, y\in K_{B\rtimes_r G}\}\]
a $C^*$-algebra which is filtered by
\[E_U^{(\pi,T)}=\{(x,P_G\pi_G(x)P_G + y) : x\in (A\rtimes G)_U, y\in K\otimes (B\rtimes G)_U\}\]
for all $U\in\mathcal E$. This $C^*$-algebra fits into the filtered extension
\[\begin{tikzcd}[column sep = small]
0\arrow{r} & K_{B\rtimes_r G}\arrow{r} & E^{(\pi,T)} \arrow{r} & A\rtimes_r G \arrow{r}& 0
\end{tikzcd}\]
which is semi split by  $s :\left\{\begin{array}{lll}A\rtimes_r G & \rightarrow & E^{(\pi,T)} \\ x & \mapsto & (x, P_G \pi_G(x)P_G)\end{array}\right.$.\\

Let us show that the controlled boundary map of this extension does not depend on the representant chosen, but only on the class $z$.

\begin{lem} With the above notations, the controlled boundary map $D_{K_{B\rtimes_r G},E^{(\pi,T)}}$ only depends on the class $z$.
\end{lem}

\begin{dem}
Let $(H_B, \pi_j,T_j), j=0,1$ two $K$-cycles which are homotopic via $(H_{B[0,1]},\pi,T)$. We denote $e_t$ the evaluation at $t\in[0,1]$ for an element of $B[0,1]$, and set $y_t=e_t(y)$ for such a $y$. The $*$-morphism
\[\phi : \left\{\begin{array}{lll}E^{(\pi,T)} & \rightarrow & E^{(\pi_t,T_t)} \\ (x,y) & \mapsto & (x, y_t)\end{array}\right.\]
satisfies $\phi(K_{B[0,1] \rtimes_r G})\subseteq K_{B \rtimes_r G}$ and makes the following diagram commute
\[\begin{tikzcd}[column sep = small]
0\arrow{r} & K_{B[0,1] \rtimes_r G}\arrow{r}\arrow{d}{\phi_{|K_{B[0,1] \rtimes_r G}}} & E^{(\pi,T)} \arrow{r}\arrow{d}{\phi} & A\rtimes_r G \arrow{r}\arrow{d}{=}& 0 \\
0\arrow{r} & K_{B \rtimes_r G}\arrow{r} &  E^{(\pi_t,T_t)} \arrow{r} & A\rtimes_r G \arrow{r} & 0
\end{tikzcd}.\]

According to \cite{OY2}, remark $3.7.$, the following holds
\[D_{K_{B\rtimes_r G},E^{(\pi_t,T_t)}} = \phi_* \circ D_{K_{B[0,1]\rtimes_r G},E^{(\pi,T)}}.\]
As $id \otimes e_t$ gives a homotopy between $id\otimes e_0$ and $id\otimes e_1$, and as if two $*$-morphisms are homotopic, then they are equal in controlled $K$-theory, 
\[D_{K_{B\rtimes_r G}, E^{(\pi_0,T_0)}}=D_{K_{B\rtimes_r G}, E^{(\pi_1,T_1)}}\]
holds, and the boundary of the extension $E^{(\pi,T)}$ depends only on $z$.\\
\qed
\end{dem}

\begin{definition}
The controlled Kasparov transform of an element $z\in KK_1^G(A,B)$ is defined as the composition
\[J_{red,G}(z)=\mathcal M_{B\rtimes_r G}^{-1}\circ D_{K_{B\rtimes_r G}, E^{(\pi,T)}}.\]
\end{definition}

As the boundary map is a $(\alpha_D,k_D)$-controlled morphism and the Morita equivalence preserves the filtration, $J_{red,G(z)}$ is  $(\alpha_D,k_D)$-controlled. 

\begin{prop}\label{Kasparov1}
Let $A$ and $B$ two $G$-$C^*$-algebras. There exists a control pair $(\alpha_J,k_J)$ such that for every $z\in KK^G_1(A,B)$, there exists a $(\alpha_J,k_J)$-controlled morphism
\[J_{red,G}(z) : \hat K_*(A\rtimes_r G)\rightarrow \hat K_{*+1}(B\rtimes_r G)\]
such that
\begin{enumerate}
\item[(i)] $J_{red,G}(z)$ induces right multiplication by $j_{red,G}(z)$ in $K$-theory ;
\item[(ii)] $J_{red,G}$ is additive, i.e.
\[J_{red,G}(z+z')=J_{red,G}(z)+J_{red,G}(z').\]
\item[(iii)] For every $G$-morphism $f : A_1\rightarrow A_2$,
\[J_{red,G}(f^*(z))=J_{red,G}(z)\circ f_{G,red,*}\] for all $z\in KK_1^G(A_2,B)$.
\item[(iv)] For every $G$-morphism $g : B_1\rightarrow B_2$,
\[J_{red,G}(g_*(z))= g_{G,red,*}\circ J_{red,G}(z)\] for all $z\in KK_1^G(A,B_1)$.
\item[(v)] Let $0\rightarrow J\rightarrow A\rightarrow A/J\rightarrow 0$ be a semi-split equivariant extension of $G$-algebras and $[\partial_J]\in KK_1^G(A/J,J)$ be its boundary element. Then 
\[J_G([\partial_J])=D_{J\rtimes_r G,A\rtimes_rG}.\] 
\end{enumerate}
\end{prop}

\begin{dem}
\begin{enumerate}

\item[(i)]The $K$-cycle $[\partial_{K_{B\rtimes_r G},E^{(\pi,T)}}]\in KK_1(A\rtimes_r G, B\rtimes_r G)$ implementing the boundary of the extension $E^{(\pi,T)}$ induces the map $j_{red,G}$ by definition, and modulo Morita equivalence, which immediately gives the first point.

\item[(ii)] If $z,z'$ are elements of $KK_1^G(A,B)$, represented by two $K$-cycles $(H_B,\pi_j,T_j)$, and if $(H_B,\pi,T)$ is a $K$-cycle representing the sum $z+z'$, then $E^{(\pi,T)}$ is naturally isomorphic to the extension sum of the $E_j:=E^{(\pi_j,T_j)}$, namely
\[\begin{tikzcd}[column sep = small]
0\arrow{r} & K_{B\rtimes_r G} \arrow{r} & D \arrow{r} & A\rtimes_r G \arrow{r} & 0
\end{tikzcd}\]
where 
\[D=\left\{\begin{pmatrix}x_1 & k_{12}\\ k_{21} & x_2\end{pmatrix} : x_j\in E_j , p_1(x_1)=p_2(x_2), k_{ij}\in K(E_j,E_i)\right\}.\]
Naturality of the controlled boundary maps \cite{OY2} ensures that the boundary of the sum of two extensions is the sum of the boundary of each, thus the result.
\item[(iii)] Let $z\in KK_1^G(A_2,B)$, represented by a cycle $(H_B,\pi,T)$. Representing $f^*(z)$ is $(H_B,f^*\pi,T)$ with $f^*\pi=\pi \circ f$. The map 
\[\phi : \left\{\begin{array}{lll} E^{f^*(\pi,T)} & \rightarrow & E^{(\pi,T)} \\
( x, P_G(f^*\pi)(x)P_G+y) & \rightarrow & ( f_G(x), P_G(f^*\pi)(x)P_G+y) \end{array}\right. \]
satisfies
\begin{enumerate}
\item[$\bullet$] $\phi(K_{B\rtimes_r G})\subseteq K_{B\rtimes_r G}$, and makes the following diagram commute
\[\begin{tikzcd}[column sep = small]
0\arrow{r} & K_{B\rtimes_r G}\arrow{r}\arrow{d}{=} & E^{f^*(\pi,T)} \arrow{r}\arrow{d}{\phi}& A_1\rtimes_r G\arrow{r}\arrow{d}{f_G} & 0\\
0\arrow{r} & K_{B\rtimes_r G}\arrow{r} & E^{(\pi,T)} \arrow{r}& A_2\rtimes_r G\arrow{r} & 0
\end{tikzcd}.\]
\item[$\bullet$] It intertwines the sections of the two extensions.
\end{enumerate}
Remark \ref{rk3.8} ensures that \[D_{K_{B\rtimes_r G}, E^{f^*(\pi,T)} } =  D_{K_{B\rtimes_r G}, E^{(\pi,T)} }\circ f_{G,*},\] and the claim is clear from composition by $\mathcal M_{B\rtimes_r G}^{-1}$.

%\item[(iv)] Let $\mathcal E = H_{B_1}\otimes_g B_2$, which is a countably generated Hilbert $B_2$-module. The homomorphism $g:B_1\rightarrow B_2$ gives rise to $g_* : \mathcal L_{B_1}(H_{B_1})\rightarrow \mathcal L_{B_2}(\mathcal E)$, which preserves compact operators : $g_*(K_{B_1})\subseteqK(\mathcal E)$. We have a similar statement for $g_G : B_1\rtimes G\rightarrow B_2\rtimes G$. We denote $\mathcal E_G$ the Hilbert $B_2\rtimes G$-module $\mathcal E\rtimes G\simeq H_{B_1\rtimes G}\otimes_g (B_2\rtimes G)$.\\

%Let $z\in KK^G(A,B_1)$ be represented by the $K$-cycle $(H_{B_1},\pi,T)$. Then $(H_{B_1}\otimes_g B_2,g_*\circ\pi, g_*(T))=(\mathcal E, \tilde\pi,\tilde T)$ represents $g_*(z)$.\\

%The map $(x,y)\mapsto (x, (g_G)_*(y))$ induces $\Psi :E^{(\pi,T)}\rightarrow  E^{g_*(\pi,T)} $ such that
%\[\Psi(x,P_G \pi_G(x) P_G +y)\mapsto (x,\tilde P_G \tilde\pi_G(x) \tilde P_G+(g_G)_*(y)).\]
%Indeed, the crossed-product functor commutes with pull-back by $G$-morphisms, and $(g_G)_*\circ\pi_G=(g_*\circ\pi)_G=\tilde \pi_G$ and $(g_G)_*(P_G) = g_*(P)_G=\tilde P_G$ so that 
%\[(g_G)_*(P_G \pi_G(x) P_G)=\tilde P_G \tilde\pi_G(x) \tilde P_G. \]
%Now, by the stabilisation lemma of Le Gall \cite{LeGall}, we know that the countably generated Hilbert module $\mathcal E_G$ sits as a complemented module of $H_{B_2\rtimes G}$, and there exists a projection $p\in L(H_{B_2\rtimes G})$ such that $pH_{B_2\rtimes G}\simeq \mathcal E_G$ and $pK_{B_2\rtimes G}p\simeq K(\mathcal E_G)$. Let $\psi$ be the composition $K_{B_1\rtimes G}\rightarrow_{(g_G)_*} K(\mathcal E_G)\rightarrow K_{B_2\rtimes G}$. In this particular case, we can give an explicit description of $\psi$. The map defined on basic tensor products $(x_j)_{j}\otimes b\mapsto (g(x_j)b)_j $ extends to an isometric embedding $\mathcal E_G \rightarrow H_{B_2\rtimes G}$, under which $ b\theta_{e_i,e_j}$ is mapped to $g(b)\theta_{u_i,u_j}$, where $\{e_j\}$ and $\{u_j\}$ are respectively the canonical orthogonal basis of $H_{B_1 \rtimes G}$ and $H_{B_2 \rtimes G}$. This gives a commutative diagram 
%\[\begin{tikzcd}[column sep = small]
%0\arrow{r} & K_{B_1\rtimes G}\arrow{r}\arrow{d}{\psi} & E^{(\pi,T)} \arrow{r}\arrow{d}{\Psi}& A\rtimes_r G\arrow{r}\arrow{d}{=} & 0\\
%0\arrow{r} & K_{B_2\rtimes G}\arrow{r} & E^{g_*(\pi,T)} \arrow{r}& A\rtimes G\arrow{r} & 0
%\end{tikzcd}.\]
%and $\Psi$ intertwines the two filtered sections by the previous relation. Moreover, $\Psi_{|K_{B_1\rtimes G}}\subseteqK_{B_2\rtimes G}$, so that we can again apply the remark $3.7$ of \cite{OY2} to state
%\[ D_{K_{B_2\rtimes G},E^{g_*(\pi,T)}}=\psi_*\circ D_{K_{B_1\rtimes G},E^{(\pi,T)}},\]
%which we compose by the Morita equivalence on the left $M_{B_2\rtimes G}^{-1}$
%\[J_G(g_*(z)) = M_{B_2\rtimes G}^{-1}\circ g_{G,*}\circ D_{K_{B_1\rtimes G},E^{(\pi,T)}}.\]
%The homomorphisms inducing the Morita equivalence make the following diagram commutes,
%\[\begin{tikzcd}B_1\rtimes G\arrow{r}{g_G}\arrow{d} & B_2\rtimes G\arrow{d} \\ K_{B_1\rtimes G } \arrow{r}{\psi}& K_{B_2\rtimes G }\end{tikzcd},\]
%and $J_G(g_*(z))= g_{G,*}\circ M_{B_1\rtimes G}^{-1}\circ D_{K_{B_1\rtimes G},E^{(\pi,T)}}=g_{G,*}\circ J_G(z)$.\\

%%%%%%%%%%%%%%%%%%%%%%%%%
%%%%% NOUVELLE PREUVE %%%
%%%%%%%%%%%%%%%%%%%%%%%%%

\item[(iv)] %New proof.\\
Let $z \in KK^G(A,B_1)$ be represented by the $K$-cycle $(H_{B_1},\pi,T)$. Let $V\in \mathcal L_{B_2}(H_{B_1}\otimes_g B_2,H_{B_2})$ be the isometry of remark \ref{isometry}. Notice that $V$ intertwines the actions of $G$ on $ H_{B_1}\otimes B_2 $ and $H_{B_2}$. According to Lemma \ref{isometryKK}, 
\[g_*(z)=[H_{B_1}\otimes_g B_2, \pi\otimes_g 1, T\otimes_g 1]\in KK^G(A,B_2)\] 
is also represented by $[H_{B_2}, \pi',T' ]$ where $\pi' = Ad_{V}\circ (\pi\otimes_g 1)$ and $T' = V(T\otimes_g 1)V^* +1-VV^*$. Let $\psi$ be given by the composition $Ad_{V_G}\circ g_G$.\\
The map $\Psi :(x,y)\mapsto (x, \psi(y))$ defines a $*$-homomorphism $E^{(\pi,T)} \rightarrow E^{(\pi',T')}$ such that 
\[\Psi(x,P_G\pi_G(x)P_G +y)= (x, P'_G  \pi_G'(a)P'_G + \psi(y)) ).\] 
Indeed, the crossed-product functor commutes with pull-back by $G$-morphisms, and $Ad_{V_G}\circ g_G \circ\pi_G= (Ad_V\circ g_* \circ \pi)_G = \pi'_G$ and $\psi(P_G)= V_G (P_G\otimes_{g_G} 1)V^*_G = (V(P\otimes_g 1 ) V^*)_G = (P')_G$ so that 
\[\psi(P_G \pi_G(x) P_G)=P'_G \pi'_G(x) P'_G. \]
This gives a commutative diagram 
\[\begin{tikzcd}[column sep = small]
0\arrow{r} & K_{B_1\rtimes G}\arrow{r}\arrow{d}{\psi} & E^{(\pi,T)} \arrow{r}\arrow{d}{\Psi}& A\rtimes_r G\arrow{r}\arrow{d}{=} & 0\\
0\arrow{r} & K_{B_2\rtimes G}\arrow{r} & E^{(\pi',T')} \arrow{r}& A\rtimes G\arrow{r} & 0
\end{tikzcd}.\]
and $\Psi$ intertwines the two filtered sections by the previous relation. Moreover, $\Psi_{|K_{B_1\rtimes G}}\subseteq K_{B_2\rtimes G}$, so that we can again apply the remark $3.8$ of \cite{OY2} to state
\[ D_{K_{B_2\rtimes G},E^{(\pi',T')}}=\psi_*\circ D_{K_{B_1\rtimes G},E^{(\pi,T)}},\]
which we compose by the Morita equivalence on the left $M_{B_2\rtimes G}^{-1}$
\[J_G(g_*(z)) = M_{B_2\rtimes G}^{-1}\circ g_{G,*}\circ D_{K_{B_1\rtimes G},E^{(\pi,T)}}.\]
The homomorphisms inducing the Morita equivalence make the following diagram commutes,
\[\begin{tikzcd}B_1\rtimes G\arrow{r}{g_G}\arrow{d} & B_2\rtimes G\arrow{d} \\ K_{B_1\rtimes G } \arrow{r}{\psi}& K_{B_2\rtimes G }\end{tikzcd},\]
and $J_G(g_*(z))= g_{G,*}\circ M_{B_1\rtimes G}^{-1}\circ D_{K_{B_1\rtimes G},E^{(\pi,T)}}=g_{G,*}\circ J_G(z)$.\\

\item[(v)] Let $q:A\rightarrow A/J$ be the quotient map and $(H_J, \pi, T)$ be a cycle representing $[\partial_J]$. Then we apply remark $3.7$ of \cite{OY2} to the commutative diagram
\[\begin{tikzcd}[column sep = small]
0\arrow{r} & J\rtimes G\arrow{r}\arrow{d} & A\rtimes G \arrow{r}\arrow{d}{s\circ q_G}& A/J\rtimes_r G\arrow{r}\arrow{d}{=} & 0\\
0\arrow{r} & K_{J\rtimes G}\arrow{r} & E^{(\pi,T)} \arrow{r}& A/J\rtimes G\arrow{r} & 0
\end{tikzcd},\]
where the first vertical arrow is the canonical mapping that induces the Morita equivalence. \\
\qed
\end{enumerate}
\end{dem}

\subsubsection{Even case}

We can now define $J_G$ for even $K$-cycles. Let $A$ and $B$ be two $G$-algebras. Let $[\partial_{SB}]\in KK_1(B,SB)$ be the $K$-cycle implementing the boundary of the extension $0\rightarrow SB\rightarrow CB\rightarrow B\rightarrow 0$, and $[\partial]\in KK_1(\C,S)$ be the Bott generator. As $z\otimes_B [\partial_{SB}]$ is an odd $K$-cycle, we can define
\[J_G(z):= \hat\tau_{B\rtimes G}([\partial]^{-1})\circ J_G(z\otimes[\partial_{SB}]).\] 

Here $\hat\tau_D$ refers, for any $C^*$-algebras $D,A_1,A_2$ and $z\in KK_*(A_1,A_2)$, to the $(\alpha_\tau,k_\tau)$-controlled map $\hat K (A_1\otimes D )\rightarrow \hat K(A_2\otimes D)$ of theorem \ref{tensorization}. We can see that, if we set $\alpha_J=\alpha_\tau \alpha_D$ and $k_J=k_\tau * k_D$, $J_G(z)$ is $(\alpha_J,k_J)$-controlled.\\

\begin{prop}\label{Kasparov}
Let $A$ and $B$ two $G$-$C^*$-algebras. For every $z\in KK^G_*(A,B)$, there exists a control pair $(\alpha_J,k_J)$ and a $(\alpha_J,k_J)$-controlled morphism
\[J_{red,G}(z) : \hat K(A\rtimes_r G)\rightarrow \hat K(B\rtimes_r G)\]
of the same degree as $z$, such that
\begin{enumerate}
\item[(i)] $J_{red,G}(z)$ induces right multiplication by $j_{red,G}(z)$ in $K$-theory ;
\item[(ii)] $J_{red,G}$ is additive, i.e.
\[J_{red,G}(z+z')=J_{red,G}(z)+J_{red,G}(z').\]
\item[(iii)] For every $G$-morphism $f : A_1\rightarrow A_2$,
\[J_{red,G}(f^*(z))=J_{red,G}(z)\circ f_{G,red,*}\] for all $z\in KK_*^G(A_2,B)$.
\item[(iv)] For every $G$-morphism $g : B_1\rightarrow B_2$,
\[J_{red,G}(g_*(z))= g_{G,red,*}\circ J_{red,G}(z)\] for all $z\in KK_*^G(A,B_1)$.
\item[(v)] $J_{red,G}([id_A]) \sim_{(\alpha_J,k_J)} id_{\hat K(A\rtimes G)}$
\end{enumerate}
\end{prop}

\begin{dem} Let $z\in KK_0^G(A,B)$.
\begin{itemize}
\item[$(i)$] $J_{red,G}(z)$ induces in $K$-theory right multiplication by $j_{red,G}(z\otimes [\partial_{SB}])\otimes \tau_{B\rtimes_r G}([\partial]^{-1})$. But $j_{red, G}$ respects Kasparov products, and by \ref{}, $j_{red,G}([\partial_{SB}]) = [\partial_{SB\rtimes_r G}]$ and $\tau_{B\rtimes_r G}([\partial]^{-1}) = [\partial_{SB\rtimes_r G }]^{-1}$ are inverse of each others.
\item[$(ii)$] Additivity follows from additivity of the Kasparov product and of proposition \ref{Kasparov1}.
\item[$(iii)$] Recall the equality $f^*(x)\otimes y = f^*(x\otimes y)$. As a consequence of the previous proposition \ref{Kasparov1},
\[\begin{array}{lcl} J_{red,G}(f^*(z)) & = &  \hat\tau_{B\rtimes_r G}([\partial]^{-1}) \circ J_{red,G}(f^*(z)\otimes [\partial_{SB}]) \\
		& = &  \hat\tau_{B\rtimes_r G}([\partial]^{-1}) \circ J_{red,G}(f^*(z\otimes [\partial_{SB}]))\\
		& = &  \hat\tau_{B\rtimes_r G}([\partial]^{-1}) \circ J_{red,G}(z\otimes [\partial_{SB}])\circ f_{G,red,*}\\
		& = & J_{red,G}(z) \circ f_{G,red,*}
\end{array}\]
\item[$(iv)$] We have $g_*(z)\otimes_{B'} [\partial_{SB'}] = z\otimes_B g^{*}[\partial _{SB'}] = (Sg)_{*}(z\otimes_B [\partial _{SB}])$, hence, by proposition \ref{Kasparov1}, 
\[J_{red,G}(g_*(z)) = \hat\tau_{B'\rtimes_r G}([\partial]^{-1}) \circ (Sg)_{G,red,*} \circ J_{red,G}(z\otimes_B [\partial_{SB}]),\]
but, using properties of $\hat\tau_{SB}$, see \ref{tensorization}, $\tau_{B'\rtimes_r G}([\partial]^{-1})\circ (Sg)_{*,red,G} = g_{G,red,*}\circ\tau_{B\rtimes_r G}(([\partial]^{-1})$ 
\end{itemize}
\qed
\end{dem}

We now show that the controlled Kasparov transform respects in a quantitative way the Kasparov product.

\begin{prop} There exists a control pair $(\alpha_K,k_K)$ such that for every $G$-$C^*$-algebras $A$, $B$ and $C$, and every $z\in KK^G(A,B),z'\in KK^G(B,C)$, the controlled equality
\[J_G(z\otimes_B z') \sim_{\alpha_K,k_K} J_G(z')\circ J_G(z)\]
holds.
\end{prop}
\begin{dem}
%We will use the following fact : there exists a positive integer $d$ such that every cycle $z\in KK^G(A,B)$ has decomposition property $(d)$. For more details, we send to the appendice of the article of V. Lafforgue \cite{LaffOY} where H. Oyono-Oyono shows that claim. We just need to know that $z$ satisfies the decomposition property $(d)$ if there exist $d+1$ $G$-$C^*$-algebras $A_j$  and $d$ cycles $\alpha_j\in KK^G(A_{j-1},A_j), j=1,d$ such that $A_0=A$, $A_d=B$ and each $\alpha_j$ is either coming from a $*$-morphism $A_{j-1}\rightarrow A_j$, or there is a $*$-morphism $\theta_j: A_j\rightarrow A_{j-1}$ such that $\alpha_j \otimes_{A_j} [\theta_j]=1$ in $KK^G(A_{j-1},A_{j-1})$.\\
Recall from \ref{propertyD} that every $\alpha\in KK^G(A,B)$ satisfies property $(d)$. This property reduces the proof to the special case of $\alpha\in KK^G(A,B)$ being the inverse of a $*$-homomorphism $\theta : B\rightarrow A$ in $KK^G$-theory : $\alpha\otimes_B [\theta]=1_A$. Let $z\in KK^G(B,C)$ :
\[\begin{array}{rcl}
J_G (\alpha\otimes z) & \sim_{\alpha_J^2,k_J*k_J} &  J_G(\alpha\otimes z)\circ J_G(\alpha\otimes [\theta]) \\
			& \sim & J_G(\alpha\otimes z)\circ J_G(\theta_*(\alpha))\\
			& \sim & J_G(\alpha\otimes z)\circ \theta_{G,*}\circ J_G(\alpha)\\
			& \sim & J_G(\theta^*(\alpha\otimes z))\circ J_G(\alpha)\\
			& \sim & J_G(z)\circ J_G(\alpha) \\
\end{array}\] 
because $\theta^*(\alpha\otimes z)=\theta^*(\alpha)\otimes z=1\otimes z =z$. The control on the propagation of the first line follows from remark $2.5$ of \cite{OY2} and point $(v)$, the other lines are equal by points $(iii)$ and $(iv)$. As $d$ is uniform for all locally compact groupoids with Haar systems, a simple induction concludes, and $(\alpha_K,k_K)$ can be taken to be $(d \alpha_J^{2d},( k_J*k_J)^{*d})$.
\qed
\end{dem}

%%%%%%%%%%%%%%%%%%%%%%%%%%%%%%%%%%%%%%%
%%%%%%%%%%%%%%%%%%%%%%%%%%%%%%%%%%%%%%%
\subsection{Quantitative assembly maps}
%%%%%%%%%%%%%%%%%%%%%%%%%%%%%%%%%%%%%%%
%%%%%%%%%%%%%%%%%%%%%%%%%%%%%%%%%%%%%%%

Let $E\in\mathcal E$. There exists continuous functions $\lambda_g : P_E(G)\rightarrow [0,1]$ for all $g\in G$ such that $\eta = \sum_{g\in G^x}\lambda_g(\eta) \delta_g$ for any $\eta\in P_E(G)$. Define $h(x)=\lambda_{e_x}$ and $\phi = \sqrt h$. Notice that $g.h = \lambda_g$ for all $g\in G$, and as $\sum_{g\in G^x}\lambda_g = 1 ,\forall x\in G^{(0)}$, 
\[\mathcal L_E =\sum_{g\in G^x} \phi(g.\phi)\]
defines a projection of $C_0(P_E(G))\rtimes_r G$ with bounded propagation, and defines a $K$-theory class $[\mathcal L_E,0]_{\varepsilon,F}$ for any $\varepsilon\in (0,\frac{1}{4})$ and any $F\in \mathcal E$ such that $E\subseteq F$. Moreover, for such controlled subsets, if $q_E^F : C_0(P_E(G))\rightarrow C_0(P_F(G))$, then $(q_E^F)_*[\mathcal L_E,0]_{\varepsilon,E} = [\mathcal L_F,0]_{\varepsilon,F}$.\\

\begin{definition}
Let $B$ be a $G$-algebra, and $\varepsilon\in (0,\frac{1}{4}),E\in\mathcal E$. Let $F\in \mathcal E$ such that $k_J.E \subseteq F$. The controlled assembly map for $G$ is defined as the composition of $J_G$ with the evaluation at $[\mathcal L_E,0]_{\varepsilon,E}$ :
\[\mu_{G,B}^{\varepsilon,E,F}\left\{
\begin{array}{rcl}
RK^G(P_E(G), B) & \rightarrow & K_*^{\varepsilon, F}(B\rtimes_r G)\\
z & \mapsto & \iota_{k_j.E}^{F} \circ J_G^{\varepsilon, E}(z)([\mathcal L_E,0]_{\varepsilon , E})
\end{array}\right.\]
\end{definition}


%%%%%%%%%%%%
%%%%%%%%%%%%
%%%%%%%%%%%%

\begin{rk}
The assembly map is defined for any reasonnable crossed-products by $G$. In particular for the reduced one and the maximal one, so that we have two different assembly maps, which we shall distinguish writing $\mu_{G,r}$ and $\mu_{G,max}$ if necessary.
\end{rk}

\begin{rk} 
The family of assembly maps $\mu_{G,B}^{\varepsilon,E,F}$ induces the Baum-Connes assembly map for $G$ in $K$-theory : the following diagram commutes
\[\begin{tikzcd}
RK^G(P_E(G),B) \arrow{r}{\mu_{G,B}^{\varepsilon,E,F}}\arrow{dr}{\mu_{G,B}^E} & K_*^{\varepsilon, F}(B\rtimes_r G)\arrow{d}{\iota_{\varepsilon,F}}\\ 
		&  K_*(B\rtimes_r G)
\end{tikzcd}\]
because $J_G(z)$ induces the right multiplication by $j_G(z)$ and also $\mu_G^E(z)=[\mathcal L_E]\otimes j_G(z)$. But, as $(q_E^F)_*[\mathcal L_E,0]_{\varepsilon,E} = [\mathcal L_F,0]_{\varepsilon,F}$ as soon as $E\subseteq F$, this diagram commutes with inductive limit over $E$.\\
\end{rk}

%In \cite{OY3}, H. Oyono-Oyono and G. Yu defined a bunch of local quantitative coarse assembly maps for a metric space $X$. For the sake of simplicity, we take $X$ to be discrete and uniformly bounded. Let $\mathcal C$ be its coarse structure, that is the set of all its controlled subsets. Then, for any $C^*$-algebras $A$ and $B$ and a $K$-cycle $z\in KK(A,B)$, they construct a controlled morphism
%\[\sigma_X(z) : \hat K(C^*(X,A))\rightarrow \hat K(C^*(X,B)).\]
%There exists a projection $P_X$ with finite propagation, and the local quantitative assembly map is defined as 
%\[A_{X,B}^{\varepsilon,r,d}(z)=\sigma_X^{\varepsilon,r}(z)([P_X]_{\varepsilon,r})\] for $z\in KK(C_0(P_d(X)),B)$, where $P_d(X)$ is the classical Rips complex of $X$. This bunch of assembly maps induce the usual coarse assembly map of $X$ 
%\[A_{X,B} : KX_*(X,B)\rightarrow K_*(C^*(X,B)\]
%in $K$-theory. Now let $G$ be the coarse groupoid of $X$. It is an étale groupoid with compact base space $G^{(0)}=\beta X$, the Stone-Cech compactification of $X$ defined as
%\[G := \cup_{E \in \mathcal C} \overline E,\] 
%where $\overline E$ is the closure of $E$ in $\beta (X \times X)$. 

%Recall that $X$ is a discrete metric space with bounded geometry, and $G=G(X)$ is its coarse groupoid. A classical result of G. Skandalis, J.-L. Tu and G. Yu \cite{SkTuYu} claims that the coarse Baum-Connes conjecture for $X$ with coefficients in $B$ is equivalent to the Baum-Connes conjecture for the groupoid $G$ with coefficient in $l^\infty(X,K_B)$. More precisely, there is an isomorphism of $C^*$-algebras $\Psi_B : l^\infty(X,K_B)\rtimes_r G \simeq C^*(X,B)$ and the following diagram commutes :
%\[\begin{tikzcd}
%\mu_{G,l^\infty(X,K_B)}^d : KK_*^G(C_0(P_d(G),l^\infty(X,K_B)) \arrow{r}\arrow{d}{\iota^*}& K_*(l^\infty(X, K_B)\rtimes_r G)\arrow{d}{(\Psi_B)_*}\\
%A_{X,B}^d : KK_*(C_0(P_d(X),B) \arrow{r} & K_*(C^*(X,B))
%\end{tikzcd}\]
%where the left vertical arrow comes from the inclusion of groupoid $\iota :\{x\}\rightarrow G$ for any $x\in X$. We claim that we can prove a controlled analogue of this result which induces it in $K$-theory.

%To prove this, we shall describe $\Psi$ more precisely. For any $C^*$-algebra $B$, let $\tilde B = l^\infty (X,K_B)$. It is naturaly a $G$-algebra, and the fiber over any $x\in\beta X$ is easily seen to be $\tilde B_x = B$. Now, if $f\in C_c(G, \tilde B)$, as $\overline E$ are the compact-open of $G$, $f$ is continuous over a $\overline E$, so it is just a bounded function over $E$.\\
%Define for $g=(x,y)\in X\times X\subseteqG$,
%$\Psi_B(f)_{xy}=f(g)(x)\in \tilde B$,
%so that $\Psi_B(f)=(\Psi_B(f)_{xy})_{x,y\in X}$ is a locally compact operator of finite propagation (its support is in $E$). This is a $*$-morphism which extends to the annouced isomorphism. Moreover, $\tilde B$ is naturally a $C^*$-subalgebra of both $\tilde B\rtimes_r G$ and $C^*(X,B)$, and the two inclusion commute modulo $\Psi_B$. We have a diagram :
%\[\begin{tikzcd} 
%  \  & B \arrow[bend left]{rdd}{\iota_3^B}& \\
%  \ &\tilde B \arrow{u}{ev_x}\arrow[hookrightarrow]{ld}{\iota_1^B}\arrow[hookrightarrow]{rd}{\iota_2^B} &  \\ 
%\tilde B\rtimes_rG \arrow{rr}{\Psi_B} &  &  C^*(X,B) 
%\end{tikzcd}\] 
%where the lower triangle is commutative.\\

%Off course, $\Psi_B$ induces $\Psi_{B*} : \mathcal L_{\tilde B\rtimes_r G}(H_{B\rtimes_r G})\rightarrow \mathcal L_{C^*(X,B)}(\mathcal E)$ where $\mathcal E = H_{B\rtimes_r G}\otimes_{\Psi_B} C^*(X,B)$. \\
%Let $A$ and $B$ be two $C^*$-algebra and $z\in KK_1^G(\tilde A,\tilde B)$, represented by $(H_{\tilde B}, \psi, T)$. As $T_G = (\iota_1)_*(T)$, we have $(\Psi_B)_*(T_G)=(\iota_2)_*(T)=(T_x)_X$. Also, the relations $(\iota_1^A)_*\circ\psi = \psi_G\circ \iota_1^A$ and $(\iota_2^A)_*\circ\psi_x = (\psi_x)_X\circ \iota_2^A$ are easy to derive, which lead to $(\Psi_B)_*\circ \psi_G \circ \iota_1^A= (\iota_2^B)_*\circ \psi_x = (\psi_x)_X\circ \Psi_A\circ \iota_1^A$. By extending $G$-equivariantly to $\tilde A  \rtimes_r G$, we have $(\Psi_B)_*(\psi_G(a))=(\psi_x)_X(\Psi_A(a))$. The map $(x,y)\mapsto (\Psi_A(x), (\Psi_B)_*(y))$ induces a morphism $\Psi_E : E^{(\psi,T)}  \rightarrow  E^{((\psi_x,T_x))}$ which sends 
%$(x,P_G \psi_G(x)P_G + y)$ to $(\Psi_A(x), (P_x)_X(\psi_x)_X(\Psi_A(x))(P_x)_X+(\Psi_B)_*(y))$ by the previous computations. This map makes the following diagram commute
%\[
%\begin{tikzcd}[column sep = small]
%0\arrow{r} & K_{\tilde B\rtimes G}\arrow{r}\arrow{d}{(\Psi_B)_*} & E^{(\psi,T)} \arrow{r}\arrow{d}{\Psi_E}& \tilde A\rtimes_r G\arrow{r}\arrow{d}{\Psi_A} & 0\\
%0\arrow{r} & K_{C^*(X,B)}\arrow{r} & E^{(\psi_x,T_x)} \arrow{r}& C^*(X,A)\arrow{r} & 0
%\end{tikzcd}.
%\]
%Now the remark $3.7$ of \cite{OY2} gives $((\Psi_B)_*)_*\circ D_{\tilde A\rtimes_rG}^{K_{\tilde B\rtimes_G}} = D_{C^*(X,A)}^{K_{C^*(X,B)}}\circ (\Psi_A)_*$, and if we compose by the Morita equivalence, we get 
%\[\sigma(\iota^*(z)) \circ (\Psi_A)_* = (\Psi_B)_*\circ J_G(z),\]
%where $\iota^*(z)$ is indeed the class of $(H_B, \psi_x,T_x)$.\\

%As $C_0(P_d(G))$ is a $G$-algebra whose fiber over any $w\in\beta X$ is isomorphic to $C_0(P_d(X))$, if $A=C_0(P_d(G))$, then $(\Psi_A)_*[\mathcal L_d]\in K^{\varepsilon , R}_0(C^*(X,C_0(P_d(X)))$ which is equal to $[P_X]$, and gives the result.
%\[(\Psi_B)_*\circ\mu^{\varepsilon,R}_G (z) = A^{\varepsilon,R}(\iota^*(z)).\]
%This, passing to $K$-theory, implies the result of \cite{SkTuYu}.\\

%%%%%%%%%%%%%%%%%%%%%%%%%%%%%%%%%%%%
\subsection{Quantitative statements}
%%%%%%%%%%%%%%%%%%%%%%%%%%%%%%%%%%%%

\begin{prop} 
Let A be a $G$-algebra.\\
If the following statement is true :\\

$\bullet$(Quantitative Injectivity) $\forall E\in\mathcal E$, there exists $\varepsilon\in (0,\frac{1}{4})$ such that, for all $F$ such that $k_J . E\subseteq F$, there exists $E'$ such that $E\subseteq E'$ satisfying : if $x\in RK_*^G(P_E(G),A)$ satisfies $\mu_{G,A}^{\varepsilon,E,F}(x)=0\in K^{\varepsilon,F}(A\rtimes_r G)$, then $x=0$ in $RK^G(P_{E'}(G),A)$;\\

then $\mu_{G,A}$ is injective.\\

On the other hand, if this statement is true : \\

$\bullet$(Quantitative Surjectivity) there exists $\varepsilon\in (0,\frac{1}{4})$ such that for any $E$ and $F$ controlled subsets such that $k_J.E\subseteq F$,$\exists \varepsilon',F'$ such that $\varepsilon'\leq \varepsilon<\frac{1}{4}$ and $k_J E\subseteq F\subseteq F'$, such that for all $y\in K_*^{\varepsilon,F}(A\rtimes_r G),\exists x \in RK_*^G(P_E(G),A)$ such that $\mu_{G,A}^{\varepsilon',E,F'}(x)=\iota_{\varepsilon,F}^{\varepsilon',F'}(y)$;\\

then $\mu_{G,A}$ is surjective.
\end{prop}

\begin{dem}
Let $x\in RK(P_E(G)),A)$ which satisfies $\mu_{G,A}(x)=0$, then $\iota_{\varepsilon,F}\circ\mu_{G,A}^{\varepsilon,E,F}(x)=0$. By remark $1.18$ of \cite{OY2}, there exists a universal $\lambda>0$ and a certain $F''>0$ such that
\[\begin{array}{lll}0 &  =  & \iota_{\varepsilon,F}^{\lambda\varepsilon,F'}\circ \mu_{G,A}^{\varepsilon,E,F}(x) \\
			& = & \iota_{\varepsilon,F}^{\lambda\varepsilon,F'} (J_{G}^{\varepsilon,F}(x)([\mathcal L_E,0]_{\varepsilon,F})) \\
			& = & J_{G}^{\lambda\varepsilon,F'}(x)([\mathcal L_E,0]_{\lambda\varepsilon,F'}) \\
			& = & \mu_{G,A}^{\lambda\varepsilon,E,F'}(x).
\end{array}\]
But then the quantitative injectivity condition ensures that $x=0$ in $RK^G(P_{E'}(G),A)$ and $x=0$ in the inductive limit over $E$ $K^{top}(G,A)$.\\
The second point is immediate. \\
\qed
\end{dem}

This kind of statement leads us to define the following proprieties, following \cite{OY3}.\\
\begin{itemize}
\item[$\bullet$] $QI_{G,B}(E,E',F,\varepsilon)$ : for any $x\in RK^G(P_E(G), B )$, $\mu^{\varepsilon,E,F}_G(x) = 0$ implies $\iota_E^{E'}(x)=0$ in $RK^G(P_{E'}(G),B)$.
\item[$\bullet$] $QS_{G,B}(E,F,F',\varepsilon,\varepsilon')$ : for any $y\in K^{\varepsilon,F}(B\rtimes G)$, there exists $x\in KK^G(P_E(G),B)$ such that $\mu^{\varepsilon',F'}_G(x)=\iota_{\varepsilon,F}^{\varepsilon',F'}(y)$.
\end{itemize} 

\begin{thm}\label{Quant1}
Let $B$ a $G$-algebra, and $\tilde B = l^\infty(X,K_B)$. Then $\mu_{G,\tilde B}$ is injective if and only if for all $E\in\mathcal E,\varepsilon\in(0,\frac{1}{4})$ and $F$ such that $k_J.E\subseteq F$, there exists $E' \in\mathcal E$ such that $E\subseteq E'$ and $QI_{G,B}(E,E',\varepsilon,F)$. 
\end{thm}

To prove the theorem, we will need a serie of lemmas. The strategy is very similar to the proofs of the part $3$ of \cite{TuBC2}.

\begin{lem}[lemma $3.6$,\cite{TuBC2}]\label{JLTform}
Let $X$ be a $G$-compact proper $G$-space such that the anchor map $p:X\rightarrow G^{(0)}$ is locally injective, and let $B$ be a $G$-algebra. Then for every $z\in RK^G(X,B)$ there exists a $G$-proper $G$-compact space $Z$ and a $K$-cycle $(H_B, \pi, T)\in \mathbb E^G(C_0(Z),B)$ representing $z$ such that :
\begin{itemize}
\item[$\bullet$] $T$ is self-adjoint and $-1 \leq T\leq 1$,
\item[$\bullet$] $T$ is $G$-equivariant, i.e. $r^* T = V s^*T V^*$ ,
\item[$\bullet$] $T$ commutes with the action of $X$, i.e. $[\pi(a),T]= 0$ for all $a\in C_0(Z)$.
\end{itemize}
\end{lem}

\begin{dem}
Let $K$ be a compact fundamental domain for the action of $G$ on $Z$. By local injectivity of $p$, let $\mathcal U$ be a finite open cover of $K$ such that $p_{|U}$ is injective for all $U\in \mathcal U$. There exists compactly supported continuous functions $\phi_U : Z\rightarrow [0,1]$ such that  \[\text{supp }\phi_U \subseteq U \quad\text{ and }\quad K\subseteq \cup_{U\in\mathcal U} \phi_U^{-1}(0,+\infty).\]
Up to replacing $\phi_U$ by $\phi_U(z) / \sum_{V,g} \phi_V(z.g)$, we can assume $\sum_{U\in\mathcal U,g\in G^{p(z)}} \phi_U (z.g) = 1,\forall z\in Z$. For $x\in G^{(0)}$, define 
\[F'_x = \sum_{U\in\mathcal U, g\in G^x} V_g\pi(\phi_U^{\frac{1}{2}}) F_{s(g)}\pi(\phi_U^{\frac{1}{2}})V_g^*.\] 
The operator $F'$ is  $G$-invariant and commutes with the action of $C_0(Z)$. Indeed, by local injectivity, for all $g\in G$, there exists $z_g\in Z$ such that $Z_{s(g)}\cap U = \{z_g\}$, and for all $f\in C_0(Z_{s(g)})$, $[ \phi_U^{\frac{1}{2}} T_{s(g)} \phi_U^{\frac{1}{2}},f ] = [ \phi_U^{\frac{1}{2}} T_{s(g)} \phi_U^{\frac{1}{2}},f(z_g) \chi_{z_g} ]=0 $.\\

Moreover, $F'$ is a compact perturbation of $F$ as the following computation shows.\\
\[\begin{array}{rl}
F_x -F'_x 	& = (\sum_{g\in G^x,U\in\mathcal U} V_g \phi_U V_g^*)F_x -F'_x \\
		& = \sum_{g,U} V_g\phi_U^{\frac{1}{2}} \ (\phi_U^\frac{1}{2} V_g^* F_x V_g- F_{s(g)}\phi_U^\frac{1}{2}) \ V_g^*\\
		& = \sum_{g,U} V_g\phi_U^{\frac{1}{2}} \ 
			\left(\phi_U^\frac{1}{2} (V_g^* F_x V_g- F_{s(g)}) + [F_{s(g)},\phi_U^\frac{1}{2}]\right) \ 
				V_g^*	  	
\end{array}\]
For every $x$, the sum is finite, and each of the summand is compact, hence $[H_B, \pi,T]=[H_B,\pi,T']$.\\ 
\qed
\end{dem}

\begin{lem}
Let $X$ be a $G$-compact proper $G$-space such that the anchor map $p:X\rightarrow G^{(0)}$ is locally injective, and let $(B_j)_j$ be a countable family of $G$-algebras. Then the projection $\prod_j B_j \otimes K \rightarrow B_j\otimes K$ induces an isomorphism
\[\Theta : RK^G(X,\prod_j B_j\otimes K)\rightarrow \prod_j RK^G(X,B_j\otimes K)\cong \prod_j RK^G(X,B_j).\]
\label{LocalInjectivity}
\end{lem}

%\begin{dem}
%Let $B_\infty = \prod B_j\otimes K$ and $p_k : B_\infty \rightarrow B_k$ the projection.\\
%Let $(\mathcal E ,\varphi,F)\in E^G(C_0(Z), B_\infty)$ be a cycle such that every 
%\[(\mathcal E_k , \varphi_k,F_k)=(p_k)_*(\mathcal E ,\varphi,F)\] 
%is homotopic to $0$. 
%Up to replace $F_k$ with $\frac{F_k+F_k^*}{2}$, we can suppose that $F_k$ is self adjoint.
%According to \cite{OY3}, we can choose a homotopy which is $C$-Lipschitz on the Calkin algebra for a universal constant $C>0$, hence (\cite{WeggeOlsen}, Lemma $17.3.3$) we can find a family of compact operators $T_{s,t}\in K(\mathcal E)$ such that $||F_s-F_t+T_{s,t}||\leq C|s-t|$. But $t\mapsto F'_t= F_{t}+T_{0,t}$ is a compact perturbation of $s\mapsto F_s$ in $\mathcal L(\mathcal E)$ which is $C$-Lipschitzian. Up to replace $(F_s)_s$ with $(F'_t)$, we can suppose the homotopies are uniformly Lipschitzian, and $\tilde F=\prod F_j$ defines a bounded operator.\\
%We now use an idea of \cite{TuBC2}, lemma $3.6$. Namely, using the local injectivity of $p$, we show that $F$ can be supposed to commute with $\varphi$ and $G$. For the sake of completness, we recall the proof. First, choose a finite open cover $(U_j)_j$ of a compact fundamental domain $K$ for the action of $G$ such that $p_{|U_j}$ is injective, and take compactly supported continuous functions $\phi_j : Z\rightarrow \R_+$ such that $\text{supp }\phi_j \subseteqU_j$ and $K\subseteq\cup \phi_j^{-1}(0,+\infty)$. We can suppose $\sum_{j,g\in G^{p(z)}} \phi_j (zg) = 1,\forall z\in Z$. Now define $F'_x = \sum_{j, g\in G^x} \alpha_g (\phi^{\frac{1}{2}} F_{s(g)}\phi^{\frac{1}{2}})$. It is an $G$-invariant operator which commutes with the action of $C_0(Z)$.\\
%Now we can see that $(\prod_j \mathcal E , \prod \varphi_j , \prod_j F_j)$ defines a cycle as $[\varphi(a),\tilde F]=0$ and $\varphi(a)(\alpha_g(F_{s(g)}-\tilde F_{r(g)})=0$. Moreover it is unitarly equivalent to $(\mathcal E,\varphi, \tilde F )$, and homotopic to $0$.\\
%For the surjectivity, just take $[(\prod \mathcal E_j,\prod \varphi_j,\prod F_j)]$ as a preimage of $\prod_j [(\mathcal E_j, \varphi_j,F_j)]$, using the previous construction.\\
%\qed  
%\end{dem}

% NEW PROOF
\begin{dem}
Put $B_\infty = \prod_j B_j\otimes \mathfrak K $. Let us define a $Z_2$-graded homomorphism 
\[\eta :  \prod_j RK^G(X,B_j\otimes K) \rightarrow RK^G(X,\prod_j B_j\otimes \mathfrak K).\] 
Let $Z\subseteq X$ be a $G$-proper $G$-compact subspace, and, for all $j$, let $z_j\in KK^G(C_0(Z),B_j)$ be represented by a standard $K$-cycle $(H_{B_j},\pi_j ,T_j)$. By lemma \ref{JLTform}, we can suppose that $T_j$ is $G$-equivariant, commutes with the action of $C_0(Z)$ and $-1\leq T_j \leq 1$. These conditions ensure that $T = (T_j)$ defines an operator in $\prod_j \mathcal L_{B_j\otimes \mathfrak K}(H_{B_j\otimes \mathfrak K})\subseteq \mathcal L_{B_\infty}(H_{B_\infty})$.\\

Define $\pi(a) = (\pi_j(a))_j\in  \mathcal L_{B_\infty}(H_{B_\infty})$, as $\sup ||\pi_j(a)||\leq ||a||$. Then, $[\pi(a),T]=0$ and $r^*T = Vs^*T V^*$. Moreover, $0\leq T_j^2-1\leq 1$ and $-2\leq T_j^*-T_j\leq 2$, hence $T^2-1$ and $T^*-T$ are in $\prod \mathfrak K_{B_j\otimes\mathfrak K}\subseteq \mathfrak K_{B_\infty}(H_{B_\infty})$. This ensures that $[H_{B_\infty},\pi,T]\in KK^G(C_0(Z),B_\infty)$. Let $\eta((z_j)_j) = [H_{B_\infty},\pi,T] $. It is clear that $\Theta\circ \eta = id_{\prod_j RK^G(X,B_j\otimes K) }$, and $\Theta $ is onto.\\

Let $z\in KK^G(C_0(Z),B_\infty)$ such that $\Theta(z) = 0$. Let us denote $\Theta(z) = (z_j)_j$ where each $z_j\in KK^G(C_0(Z),B_j)$ is represented by $K$-cycles $(H_{B_j}, \pi_j,T_j)$ homotopic to $0$ by an operator homotopy 
\[\left\{\begin{array}{rcl} [0,1] & \rightarrow & \mathbb E^G(C_0(Z),B_\infty)\\ s & \mapsto & (H_{B_j}, \pi_j,T_j(s))\end{array}\right.\] 
Then $s\mapsto T_j(s)$ descends to an homotopy of unitaries in the Calkin algebra $\mathcal L_{B_j}(H_{B_j})/ {\mathfrak K}_{B_j}(H_{B_j}) $ and, according to Proposition \ref{Lip}, we can choose the homotopy to be $L$-Lipschitz for a universal constant $L>0$, hence (\cite{WeggeOlsen}, Lemma $17.3.3$) we can find a family of compact operators $F_j(s,t)\in \mathfrak K_{B_j}(H_{B_j})$ such that $||T_j(s)-T_j(t)+F_j(s,t)||\leq L|s-t|$ for all $s,t\in[0,1]$. But $s\mapsto T'_j(s)= T_j(s)+F_j(0,s)$ is a compact perturbation of $s\mapsto T_j(s)$ in $\mathcal L_{B_j}(\mathcal H_{B_j})$ which is $L$-Lipschitz. Up to replace $(T_j(s))_s$ by $(T'_j(s))_s$, we can suppose the homotopies $s\mapsto T_j(s)$ are $L$-Lipschitz for all $j$. Hence $s\mapsto (T_j(s))_j$ is in $\prod\mathcal L_{B_j\otimes \mathfrak K}(H_{B_j\otimes \mathfrak K}) $. Define $T : s\mapsto (T_j(s))_j$ so that :
\[\left\{\begin{array}{rcl} [0,1] & \rightarrow & \mathbb E^G(C_0(Z),B_\infty)\\ s & \mapsto & (H_{B_\infty}, \pi,T(s))\end{array}\right.\]
defines a $L$-Lipschitz operator homotopy between $0$ and $(H_{B_\infty},\pi,T) = \eta()$. 
\qed
\end{dem}
%END NEW PROOF

\begin{lem}\label{prod}
Let $G$ be a locally compact, $\sigma$-compact étale groupoid, $\{B_j\}_{j\geq  0}$ a family of $G$-algebras and $\mathfrak K$ the algebra of compact operators over a separable Hilbert space. For $E\in\mathcal E$, set $\Delta=P_E(G)$. Then, we have an $\Z_2$-graded isomorphism of abelian groups
\[RK^G(\Delta,\prod_j B_j\otimes \mathfrak K)\simeq \prod_j RK^G(\Delta,B_j)\]
\end{lem}

\begin{dem}
For all $j$ and any locally compact $G$-space $X$, the projection $\prod_j B_j\otimes K\rightarrow B_j \otimes K$ induces a morphism
\[\Theta^X : KK^G(C_0(X),\prod_j B_j\otimes K )\rightarrow \prod_j  KK^G(C_0(X),B_j\otimes K ).\]
Let $X_0\subseteq X_1 \subseteq...\subseteq X_n$ be the $n$-skeleton decomposition associated to the simplicial structure of the Rips complex $\Delta$ and let $Z_j = C_0(X_j)$, $Z^j_{j-1} = C_0(X_j \setminus X_{j-1})$ and $\Theta_j = \Theta^{X_j}$.
We will show the claim by induction on the dimension of $\Delta$.\\

By naturality of the boundary element, the extension of $G$-algebras $0\rightarrow Z^j_{j-1} \rightarrow Z_j \rightarrow Z_{j-1}\rightarrow 0$ gives a commutative diagram with exact rows :
%lines :
%\[\begin{tikzcd}
%KK_*(Z^j_{j-1},\prod_j B_j\otimes K)\arrow{r}{\delta}\arrow{d}{\Theta^j_{j-1}} & KK_*(Z_{j-1},\prod_j B_j\otimes K)\arrow{r}\arrow{d}{\Theta_{j-1}} & KK_*(Z_j,\prod_j B_j\otimes K)\arrow{r}\arrow{d}{\Theta_j} & KK_*(Z^j_{j-1},\prod_j B_j\otimes K)\arrow{r}{\delta} \arrow{d}{\Theta^j_{j-1}} & KK_*(Z_{j-1},\prod_j B_j\otimes K)\arrow{d}{\Theta_{j-1}}\\
%\prod_j KK_*(\tilde Z^j_{j-1},B_j \otimes K)\arrow{r}{\delta} & \prod_j KK_*(\tilde Z_{j-1},B_j \otimes K)\arrow{r} & \prod_j KK_*(\tilde Z_j,B_j \otimes K)\arrow{r} & \prod_j KK_*(\tilde Z^j_{j-1},B_j \otimes K)\arrow{r}{\delta} & \prod_j KK_*(\tilde Z_{j-1},B_j \otimes K)\\
%\end{tikzcd}\]
\[\begin{tikzcd}
RK^G_*(Z^j_{j-1},\prod_j B_j\otimes K)\arrow{d}{\delta}\arrow{r}{\Theta^j_{j-1}} & \prod_j RK^G_*( Z^j_{j-1},B_j \otimes K)\arrow{d}{\delta}  \\
RK^G_*(Z_{j-1},\prod_j B_j\otimes K)\arrow{d}\arrow{r}{\Theta_{j-1}}  & \prod_j RK^G_*( Z_{j-1},B_j \otimes K)\arrow{d} \\
RK^G_*(Z_j,\prod_j B_j\otimes K)\arrow{d}\arrow{r}{\Theta_j} & \prod_j RK^G_*( Z_j,B_j \otimes K)\arrow{d} \\
RK^G_*(Z^j_{j-1},\prod_j B_j\otimes K)\arrow{d}{\delta}\arrow{r}{\Theta^j_{j-1}} & \prod_j RK^G_*( Z^j_{j-1},B_j \otimes K)\arrow{d}{\delta}\\
RK^G_*(Z_{j-1},\prod_j B_j\otimes K)\arrow{r}{\Theta_{j-1}} & \prod_j RK^G_*( Z_{j-1},B_j \otimes K)
\end{tikzcd}\]

The five lemma ensures that if $\Theta_{j-1}$ and $\Theta^j_{j-1}$ are isomorphisms, then so is $\Theta_j$. Moreover, because $\Delta$ is a typed $G$-simplicial complex (see \ref{Gcomplex}), $X_j\setminus X_{j-1}$ is $G$-equivariantly homeomorphic to $\mathring \sigma_j \times \Sigma_j$, where  $\mathring \sigma _ j $ denotes the interior of the standard simplex, and   $\Sigma_j$ is the set of centers of $j$-simplices of $X_j$. Bott periodicity ensures then that, if $\Theta_{j-1}$ is an isomorphism, then so is $\Theta^j_{j-1}$. By induction, proving that $\Theta_0$ is an isomorphism concludes the proof, which is essentially the content of lemma \ref{LocalInjectivity} : $X_0$ is a $G$-compact proper $G$-space, and its anchor map is just the target map $r:G\rightarrow G^{(0)}$, which is supposed to be étale, so locally injective.\\
\qed
\end{dem}

We can now prove the theorem \ref{Quant1}.\\

\begin{dem}
Let $x\in KK^G(P_E(G),\tilde B)$ such that $\mu_{G,\tilde B}^E(x)=0$. Then, as the quantitative assembly maps factorize $\mu_{G,\tilde B}$, there exist $\varepsilon>0$ and $F$ such that $k_J. E \subseteq F$, satisfying $\mu_{G,\tilde B}^{\varepsilon,E,F}(x)=0$. Using the isomorphism of lemma \ref{prod} and the Morita equivalence, we can identify $x$ with $(x_j)_j$ under $RK^G(P_E(G),\tilde A)\simeq\prod_j RK^G(P_E(G),A)$. Now let $F'$ such that $F\subseteq F'$ and $QI_{A}(E,E',\varepsilon,F)$ holds. That ensures that $x_j=0$ in $RK^G(P_{E'}(G),B)$, and $x=0$.\\

For the converse, suppose one can find $E,\varepsilon,F$ such that $QI_{G,A}(E,E',\varepsilon,F)$ is NOT true for all $F'$ such that $F\subseteq F'$. Then one can extract a increasing sequence $E_j$ such that $\cup E_j =G$ and $x_j\in RK^G(P_E(G),A)$ such that $\mu_{G,\tilde B}^{\varepsilon,E,F}(x_j)=0$ and $x_j\neq 0$ in $RK^G(P_{E_j},A)$. Let $x\in RK^G(P_E(G),\tilde A)$ be the image of $(x_j)\in \prod RK^G(P_E(G),A)$. We have $\mu_{G,\tilde A}(x)=0$, and $x\neq 0$ in $RK^G(P_{E'}(G),\tilde A)$ for all $F'$ such that $F\subseteq F'$, so $\mu_{G,\tilde A}$ is not injective. \\
\qed   
\end{dem}

We also have a theorem relating quantitative surjectivity for $\hat\mu_{G,B}$ and surjectivity of $\mu_{G,\tilde B}$.

\begin{thm}
Let $B$ a $G$-algebra, and $\tilde B = l^\infty(X,K_B)$. Then there exists $\lambda>1$ such that $\mu_{G,\tilde B}$ is onto if and only if for any $0<\varepsilon<\frac{1}{4\lambda}$ and nonempty $F\in\mathcal E$, there exist $E\in\mathcal E$ and $F'$ such that  $k_J .E \subseteq F$ such that $QS_{B,G}(E,F,F',\varepsilon,\lambda\varepsilon)$ holds.
\end{thm}

\begin{dem}
Let $\lambda>0$ the universal constant of remark $1.18$ of \cite{OY2} : for any $C^*$-algebra and $x,y\in K^{\varepsilon, F}(A)$ such that $\iota_{\varepsilon,F} x =\iota_{\varepsilon,F}y$, there exists $F'$ such that $F\subseteq F'$ and $\iota_{\varepsilon,F}^{\lambda\varepsilon,F'} x =\iota_{\varepsilon,F}^{\lambda\varepsilon,F'}y$.\\

Let $y\in K_*(\tilde B\rtimes G)$, and take $z\in K^{\varepsilon,F}(\tilde B\rtimes G)$, where $F$ is nonempty and $0<\varepsilon<\frac{1}{4\lambda}$, such that $\iota_{\varepsilon,F}(z) = y$. The projection on the $j^{th}$ component $\tilde B \rightarrow K_B $ used in $K$-theory then composed with Morita equivalence gives a map $K^{\varepsilon,F}(\tilde B\rtimes G)\rightarrow K^{\varepsilon ,F}(B\rtimes G)$, and $z_j$ denotes the image of $z$ under this map. We can pick $E$ and $F'$ such that $k_J.E\subseteq F$ and $QS(E,F,F',\varepsilon,\lambda\varepsilon)$ : for every $j$, there exists $x_j\in RK^G(P_E(G),B) $ such that $\mu_{G,B}^{\lambda\varepsilon,E,F'}(x_j)=\iota_{\varepsilon,RF}^{\lambda\varepsilon,F'}(z_j)$. 
Let $x$ be the element of $RK^G(P_E(G),\tilde B)$ corresponding to $(x_j)\in RK^G(P_E(G),\tilde B)\simeq \prod_j KK^G(P_E(G),B)$ under the isomorphism $RK^G(P_E(G),\tilde B)\simeq \prod_j KK^G(P_E(G),B)$. Naturality of the assembly maps, and compatibility of quantitative assembly maps with the usual one ensures that $\mu_{G,\tilde B}(x)=z$, whereby $\mu_{G,\tilde B}$ is onto.\\

Suppose that there exist $0<\varepsilon<\frac{1}{4\lambda}$ and nonempty $F$ such that for every $E\in\mathcal E$ and $F'$ such that $k_J. E \subseteq F$, $QS(E,F,F',\varepsilon,\lambda\varepsilon)$ does not hold. Let $(E_j)$ and $(F_j)$ be unbounded increasing sequences of controlled subsets and $y_j\in K^{\varepsilon,F}(B\rtimes G)$ such that $\iota_{\varepsilon,F}^{\lambda\varepsilon,F_j}(y_j)$ is not in the range of $\mu_{G,B}^{E_j,\lambda\varepsilon,F_j}$. Let $y\in K^{\varepsilon,F}(\tilde B\rtimes G)$ be an element such that its image with the previous map coincides with $y_j$. If there exists $x\in RK^G(P_{S},\tilde B)$ for a $S$ such that $E\subseteq S$ and $\iota_{\varepsilon,F}( y) =\mu_{G,\tilde B}^S(x)$ then there would exists a $F'$ such that $F\subseteq F'$ and
\[\iota_{\varepsilon,F}^{\lambda\varepsilon, F'}( y) =\mu_{G,\tilde B}^{s,\lambda\varepsilon,F'}(x) = \iota_{\varepsilon,F}^{\lambda\varepsilon, F'}\circ\mu_{G,\tilde B}^{\varepsilon,S,F}(x).\]
Now choose $j$ such that $S\subseteq E_j $ and $F'\subseteq F_j$, and compose the previous equality with $\iota_{\lambda\varepsilon,F'}^{\lambda\varepsilon,F_j}$ and $q_S^{E_j}$ to obtain $\iota_{\varepsilon,F}^{\lambda\varepsilon,F_j}(y_j)=\mu_{G,B}^{d_j,\lambda\varepsilon,F_j}(x)$ which contradicts our assumption.\\
\qed
\end{dem}

%Recall that if the groupoid $G$ satisfies the Baum-Connes conjecture with coefficients, it satisfies the quantitative Baum-Connes conjecture. Interesting examples follow from the result of J-L. Tu \cite{TuThese} that a-$T$-menable groupoids satisfy the Baum-Connes conjecture with coefficients. In particular, \\

%\begin{itemize}
%\item[$\bullet$] amenable groupoids are a-$T$-menable.\\
%\item[$\bullet$] Let $X$ be a uniformly discrete metric space with bounded geometry. Then, if $X$ is coarsely embeddable into a separable Hilbert space, $G(X)$ is a-$T$-menable.\cite{SkTuYu} \\
%\item[$\bullet$] If $X$ admits a fibred coarse embedding into Hilbert space, then $G(X)_{|\partial \beta X}$ is a-$t$-menable.\cite{FinnSellFibred} For interesting examples of this type, recall the definition of a box space. Let $Gamma$ be a finitely generated group, and $\mathcal N$ a family of nested normal subgroups with trivial intersection, which have finite index in $Gamma$. Take the coarse union of the quotients to construct a coarse space $X_{\mathcal N}(Gamma)= \cup_{H\in \mathcal N } Gamma/ H$. Then, $X_{\mathcal N}(Gamma)$ admits a fibred coarse embedding if and only if $Gamma$ is a-$T$-menable. But if $X_{\mathcal N}$ is an expander, it cannot be coarsely embedded into a Hilbert space, so just take an a-$T$-menable group which has a box space $X$ which is an expander to get a coarse space that is not coarsely embeddable into Hilbert space ($SL(2,\Z)$ works), but admits a fibred coarse embedding.\\
%\end{itemize}

%The last example gives easily the following corollary.

%\begin{cor}
%Let $X$ be a coarse space that admits a fibred coarse embedding into Hilbert space. Then $\hat \mu_{X}^{max}$ is a controlled isomorphism, i.e. $X$ satisfies the controlled Coarse Baum-Connes conjecture.
%\end{cor}

%\begin{dem}
%Just use that the maximal crossed product turns restriction of a groupoid to invariant open subsets into exact sequences of $C^*$-algebras, with $F=\partial\beta X$ and $U= F^c$, hence the following diagramm commutes
%\[\begin{tikzcd}
%RK^{G_{|U}}(P_F(G_{|U}),l^\infty)\arrow{r}\arrow{d}{\mu_{G_{|U}}^{\varepsilon,E,F}} & RK^G(P_F(G),l^\infty)\arrow{r}\arrow{d}{\mu_{G}^{\varepsilon,E,F}}  & RK^{G_{|F}}(P_F(G_{|F}),l^\infty)\arrow{d}{\mu_{G_{|F}}^{\varepsilon,E,F}}  \\
%K_*^{\varepsilon,E}(l^\infty \rtimes_{max} G_{|U}) \arrow{r} & K_*^{\varepsilon,E}(l^\infty \rtimes_{max} G) \arrow{r} & K_*^{\varepsilon,E}(l^%\infty \rtimes_{max} G_{|F}) \\
%\end{tikzcd}.\]
%Now, $G_{|F}$ being a-T-menable and $G_{|U}$ being proper, the two exterior vertical maps are isomorphisms, and the five lemma concludes the proof.\%\
%\qed
%\end{dem}

\subsection{Persistence approximation property}

We recall the following definition from \cite{OY3}.

\begin{definition}
Let $B$ be a filtered $C^*$-algebra, posiive numbers $\varepsilon,\varepsilon'$ such that $0<\varepsilon <\varepsilon' <\frac{1}{4}$ and nonempty $F,F'\in\mathcal E$ such that $F\subseteq F'$. 
\begin{itemize}
\item[$\bullet$] $PA_B(\varepsilon,\varepsilon',F,F')$ : for every $x\in K_*^{\varepsilon,F}(B)$ such that $\iota_{\varepsilon,F}(x)=0$ in $K_*(B)$, then $\iota_{\varepsilon,F}^{\varepsilon',F'}(x)=0$ in $K_*^{\varepsilon',F'}(B)$.
\item[$\bullet$] $B$ is said to satisfy the Persistance Approximation Property (PAP) if for every nonempty $F$ and $\varepsilon\in (0,\frac{1}{4})$, there exists nonempty $F'$ and $\varepsilon'\in [\varepsilon,\frac{1}{4})$ such that $PA(\varepsilon,\varepsilon',F,F')$ holds.
\end{itemize}
\end{definition}

\begin{thm} 
Up to some hypothesis, if $\mu_{G,l^\infty(\N, K_A)}$ is onto and $\mu_{G,A}$ is one-to-one, then, for a universal constant $\lambda_{PA}$, for all $\varepsilon \in(0,\frac{1}{4\lambda_{PA}})$ and nonempty $F$, there exists $F'$ such that $F\subseteq F'$ and $PA_{A\rtimes G}(\varepsilon,\lambda_{PA}\varepsilon,F,F')$ holds.
\end{thm}

\begin{dem}
We denote $l^\infty(\N,K_A)$ by $\tilde A$.\\
Assume the statement does not holds : there exists $\varepsilon$ and $F$ such that $PA(\varepsilon,\varepsilon',F,F')$ is not true for every $F \subseteq F'$. Then we can extract a increasing unbounded sequence of positive numbers $F_j$ and elements $x_j\in K_*^{\varepsilon,F}(A\rtimes G)$ such that $\iota_{\varepsilon,F}(x_j)=0$ and $\iota_{\varepsilon,F}^{\lambda_{PA},F_j}(x)\neq 0$. \\
According to the lemma \ref{LocalInjectivity}, there exists $x\in K_*^{\alpha\varepsilon,h_\varepsilon F}(\tilde A\rtimes G)$ such that $p_j(x)=x_j$ where $p_j$ is the composition of the projection on the $j^{th}$ factor $\tilde A \rtimes G \rightarrow K_A \rtimes G$ and the Morita equivalence in K-theory.
By naturality of the quantitative assembly maps, the following diagram commutes
\[\begin{tikzcd}
RK^G(P_E(G),\tilde A) \arrow{r}{\mu_{G,\tilde A}^{\alpha\varepsilon,E, h_\varepsilon F}} \arrow{rd}{\mu_{G,\tilde A}^d} & K^{\alpha\varepsilon,h_\varepsilon F}(\tilde A\rtimes G) \arrow{d}{\iota_{\alpha\varepsilon, h_\varepsilon F}}\\
                                                      \                          & K(\tilde A\rtimes G)
\end{tikzcd}\]

If $\iota_{\alpha\varepsilon,h_\varepsilon F}(x)$ is in the range of $\mu_{G,\tilde A}$, there exists a nonempty $E$ and $z\in RK^G(P_E(G),\tilde A)$ such that $\mu_{G,\tilde A}(z)=\iota_{\alpha\varepsilon,h_\varepsilon F}(x)$. Denote the image of $z$ under the isomorphism $RK^G(P_E(G),\tilde A) \simeq \prod RK^G(P_E(G),A)$ by $(z_j)_j$. By naturality, $\mu_{G,\tilde A}^E(z_j)=\iota_{\alpha\varepsilon,h_\varepsilon F}^{\lambda\varepsilon,F_j}(x_j)\neq 0$ and $\mu_{G,A}(z)=0$ so that $\mu_{G,A}$ is not injective.\\
\qed
\end{dem}

