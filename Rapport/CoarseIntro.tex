\section{Coarse Geometry}

Some notations are in order. Let $X$ be a set. For any subsets $A$ and $B$ of $X\times X$, 
\begin{itemize}
\item[$\bullet$] let $A\circ B = \{(x,z)\in X\times X \text{ s.t. } \exists y\in X\text{ s.t. } (x,y)\in A, (y,z)\in B\}$,
\item[$\bullet$] and $A^{-1} = \{(x,y)\in X\times X \text{ s.t. } (y,x)\in A\}$.
\end{itemize}

\subsection{Definitions and examples}

\begin{definition}\cite{RoeCoarse}
A coarse structure on a set $X$ is a subset $\mathcal E$ of $\mathcal P(X\times X)$ such that :
\begin{itemize}
\item[$\bullet$] if $A$ and $B$ are in $\mathcal E$, so are $A^{-1}$, $A\cup B$ and $A\circ B$,
\item[$\bullet$] every finite subset of $X\times X$ is in $\mathcal E$,
\item[$\bullet$] if $A\in\mathcal E$ and $B\subseteq A$, then $B\in \mathcal E$.
\end{itemize}
\end{definition} 

\begin{Expl}
When $X$ is a discrete proper metric space, we define, for all positive number $R>0$, the $R$-diagonal as 
\[\Delta_R= \{(x,y)\in X\times X \text{ s.t. } d(x,y)\leq R\}.\]
A set $E\subseteq X\times X$ is called an entourage if there exists $R>0$ such that $E\subseteq \Delta_R$. The set of entourages $\mathcal E_X$ is a coarse structure on $X$.
\end{Expl}

\begin{Expl} Let $\Gamma$ be a finitely generated group, and $S$ be a symmetric (i.e. $S=S^{-1}$) generating subset. Define the word length associated to $S$ as 
\[l(g) = \inf\{ k\in \N : \exists s_1,\cdots,s_k\in S \text{ s.t. } g= s_1 \cdots s_k\}.\]
This defines a proper length on $\Gamma$ which does depend on $S$, but the coarse structure associated to the left invariant metric $d(\gamma_1,\gamma_2)= l(\gamma_1^{-1}\gamma_2)$ does not. It is common in the litterature to denote $|\Gamma|$ the  associated coarse space.
\end{Expl}

%--------------
% Expanders
%--------------

In order to give another example, we recall the definition of the coarse disjoint union.

\begin{definition}
Let $(X_j,d_j)$ be a countable family of metric spaces. The coarse disjoint union $\coprod X_j$ is the coarse space obtained as the usual disjoint union with a metric $d$ such that 
\begin{itemize}
\item[$\bullet$] the restriction of $d$ to any $X_j$ coincides with $d_j$, 
\item[$\bullet$] $\lim d(X_i,X_j)= + \infty$ when $|i+j|$ goes to $+\infty$.
\end{itemize}
Any such metric defines the same coarse structure on, so that the coarse equivalence class of $\coprod X_j$ is well defined. 
\end{definition}

\begin{Expl}
Let $\Gamma$ be a finitely generated group which is residually finite w.r.t. $\mathcal N=\{\Gamma_j\}$, a nested family of normal subgroups $\Gamma_0 > \Gamma_1>...$ with trivial intersection. Notice that the image of $S$ in the group $\Gamma / \Gamma_j$ is a generating subset. The box space of $\Gamma$ w.r.t. $\mathcal N$ is the coarse disjoint union of the metric spaces $\Gamma/\Gamma_j$,
\[X_{\mathcal N}(\Gamma)=\coprod_j |\Gamma/\Gamma_j|.\]
\end{Expl}

This last example gives, under suitable conditions, an example of coarse space which is not coarsely embeddable into Hilbert space.

\begin{definition}
A metric space $(X,d_X)$ coarsely embeds into Hilbert space if there exists two increasing functions $\rho_{+/-}: \mathbb R_+\rightarrow \mathbb R_+$ such that $\lim_{\infty} \rho(R) = +\infty$, and a map $\phi : X\rightarrow H$ such that :
\[\rho_-(d_X(x,y))\leq ||\phi(x)-\phi(y)||_H \leq \rho_+(d_X(x,y))\quad,\forall x,y\in X,\]
where $H$ is the separable Hilbert space.\\
\end{definition}

Recall that a graph is a a couple $(V,E)$, $V$ being the set of vertices, and $E\subseteq V\times V$. Define, for $v\in V$, the set of neighbours of $v$ as $N_v=\{w\in V / (v,w)\in E\}$ and the degree of $x$ as $deg(x)=|N_x|$. The length of a path in a graph is defined as the number of edges it contains. A graph is naturally a metric space with the path distance. The Laplacian of the graph is the positive operator $\Delta \in\mathcal L(l^2(V))$ defined by :
\[(\Delta f) (x) = f(x) - \frac{1}{deg(x)}\sum_{y\in N_x} f(y) .\] 
Let $X_j = (V_j,E_j)$ be a countable family of finite graphs. The Laplacian of the coarse disjoint union $X=\coprod X_j$ is defined as $\Delta_X = \bigoplus \Delta_j \in \mathcal L(l^2(X))$.
 
\begin{definition}
An expander is a family $X=(X_j)_j$ of finite graphs $(X_j,E_j)$ such that 
\begin{itemize}
\item[$\bullet$] $\lim_{j}|X_j|=\infty$
\item[$\bullet$] the degree of the graphs is constant : $\exists k, deg(X_j)=k,\forall j$,
\item[$\bullet$] the second eigenvalue of the Laplacian is bounded above : $sp(\Delta_j)\subset \{0\}\cup [\epsilon,1]$.
\end{itemize}
\end{definition}
When speaking of expanders, we often confuse them with the metric space consisting of the coarse disjoint union of all the graphs, which is just a metric space with the distance induced by the length on the graph when restricted to one of the graph, and such that $\lim_{j+k\rightarrow \infty}d(X_j,X_k)$.

\begin{prop}\cite{NowakYu}
Let $(X_j)$ be a countable family of metric spaces such that $\coprod X_j$ is an expander. Then $\coprod X_j$ does not coarsely embeds into Hilbert space.
\end{prop}

We can construct expanders using groups satisfying some representation theoretic properties that we now recall. If $G$ is a locally compact group, let $\hat G $ be its dual, i.e. the topological space of equivalence classes of all unitary continuous representations $(\pi,H)$ of $G$ on a Hilbert space $H$. For such a representation $(\pi,H)$, denote by $[\pi,H]$ its equivalence class. Let $(\pi,H)$ be a unitary representation of $G$. For all $\varepsilon>0$, compact subset $K\subseteq G$, and unit vector $v\in H$,
\[\mathcal V_{\varepsilon,K,v} = 
		\{ [\pi',H'] \in \hat G\text{ s.t. } \exists v'\in H', ||v'||=1 \text{ and }
		| \langle v,\pi(g)v\rangle - \langle v',\pi'(g)v' \rangle |<\varepsilon , \forall g\in K\}\]
defines a basis of neighborhoods for $[\pi,H]$. The topology generated by these subsets is called the Fell topology on $\hat G$.\\

Property (T) was introduced by Kazhdan in 1967 in \cite{kazhdan}. Property $\tau$ is a weaker version of property (T). Margulis was the first to construct expanders in \cite{MargulisExplicit} using the following fact : property (T) implies property $\tau$. Off course, this was not stated with this terminolgy at the time.

\begin{definition}
Let $G$ be a locally compact group, and let $\hat G_0$ denote the space of equivalence classes of unitary continuous representations of $G$ without invariant vectors in $\hat G$. Then, $G$ has property (T) if the trivial representation is isolated from $\hat G_0$ in $\hat G$.
\end{definition} 

\begin{definition}
Let $\Gamma$ be a finitely generated group and $\mathcal N = \{\Gamma_j\}$ be a family of finite index normal subgroups of $\Gamma$. Let $\hat{\mathcal N}$ be the set of equivalence classes of unitary continuous representations of $\Gamma$ without invariant vectors which factorizes through $\Gamma / \Gamma_j$ for some $j$. Then, $\Gamma$ has property $\tau$ with respect to $\mathcal N$ if the trivial representation is isolated from $\hat{\mathcal N}$ in $\hat G$. 
\end{definition}  

\begin{thm}\cite{NowakYu}
If $\Gamma$ has property $\tau$ w.r.t. $\mathcal N$, then $X_{\mathcal N}(\Gamma)$ is an expander. In particular, Kazhdan's property (T) implies property $\tau$, so that property (T) groups satisfy this obstruction. For example, you can take $SL(n,\mathbb Z)$ for $n\geq 3$. 
\end{thm} 

The following property was introduced by X. Chen, Q. Wang and G. Yu in \cite{ChenWangYu}. It is a generalization of coarse embeddability.

\begin{definition} A discrete metric space $X$ is said to admit a fibred coarse embedding into Hilbert space if there exist
\begin{itemize}
\item[$\bullet$] a field of Hilbert spaces $\{H_x\}_{x\in X}$ over $X$,
\item[$\bullet$] a section $s : X\rightarrow \coprod H_x$, i.e. $s(x)\in H_x$,
\item[$\bullet$] two non-decreasing functions $\rho_{+/-}: \mathbb R_+\rightarrow \mathbb R_+$ such that $\lim_{\infty} \rho(R) = +\infty$,
\item[$\bullet$] a reference Hilbert space $H$
\end{itemize}
such that, for any $R>0$, there exists a bounded subset $E_R\subseteq X$ and a trivialization 
\[t_C : \coprod_{x\in C} H_x \rightarrow C\times H\]
for all $C\subseteq X \setminus E_R$ of diameter less than $R$. Moreover, for all $x\in X$, the map $t_C(x)$ is an affine isometry satisfying 
\begin{itemize}
\item[$\bullet$] for all $x,y\in C$, $\rho_1(d(x,y))\leq ||t_C(x)(s(x))-t_c(y)(s(y))|| \leq \rho_2(d(x,y))$
\item[$\bullet$] for all $C_1,C_2$ in $X - E_R$ of diameter less than $R$ with nonempty intersection $C_1\cap C_2$, there exists an affine isometry $t_{C_1,C_2} : H\rightarrow H $ such that $t_{C_1}(x)t_{C_2}(x)^{-1} = t_{C_1,C_2}(x)$ for all $x\in C_1\cap C_2$.
\end{itemize}
\end{definition}

\begin{rk} If $X$ admits a coarse embedding into Hilbert space, then it admits a fibred coarse embedding into Hilbert space. However, spaces that admit a fibred coarse embedding into Hilbert space which do not coarsely embed into Hilbert space exists. Indeed, let $\Gamma$ be a finitely generated group residually finite w.r.t. a nested family $\mathcal N$ of normal subgroups with trivial intersection. Then $X_{\mathcal N}(\Gamma)$ admits a fibred coarse embedding into Hilbert space iff $\Gamma$ is a-T-menable, see \cite{FinnSellFibred} for reference. For example, let $\Gamma = SL(2,\Z)$ and $\mathcal N= \{SL(2,p^k\Z)\}_{k>1}$ for a prime $p$. Then $\Gamma$ satisfies property $\tau$, so that $X_{\mathcal N}(\Gamma)$ is an expander and does not embbed into Hilbert space. But $\Gamma$ is a-T-menable.
\end{rk}
%%%%
%%%%

%----------
% X module
%----------

\subsection{Geometric modules and Roe algebras}

From now on, $(X,d)$ will be a discrete metric space with bounded geometry, i.e. such that, for all $R>0$, $\sup_{x\in X} |B(x,R)|$ is finite. In this section, we define the Roe algebras of $X$. 

\begin{definition} A $X$-module is a pair $(H_X,\phi)$ where :
\begin{itemize}
\item[$\bullet$] $H_X$ is a Hilbert space,
\item[$\bullet$] $\phi : C_0(X)\rightarrow \mathcal L(H_X)$ is a $*$-homomorphism. 
\end{itemize}
The module is said to be non-degenerate if the closure of the subspace generated by $\{\phi(f)\xi : f\in C_0(X), \xi\in H_X\}$ is dense in $H_X$, and it is called standard if no non-zero function acts as a compact operator, i.e. $\phi(f) \in\mathfrak K(H_X)$ implies $f=0$.
\end{definition}

We will sometime allow ourself a slight misuse of notation by writing $f\xi$ for $\phi(f)\xi$ when $f\in C_0(X)$ and $\xi \in H_X$.

\begin{Expl}
Let $H$ denotes the separable Hilbert space. Continuous functions $C_0(X)$ act on $H_X =l^2(X)\otimes H$ as bounded operators, which gives $H_X$ the structure of a $X$-module, called the standard $X$-module. It is standard non-degenerate (s.n.d. for short).  
\end{Expl}

\begin{definition}
Let $X$ and $Y$ be two discrete proper metric spaces with bounded geometry, $H_X$ and $H_Y$ $X$ and $Y$-modules respectively, and $T\in \mathcal L(H_X, H_Y)$ be a bounded operator.
\begin{itemize}
\item[$\bullet$] $T$ is said to be locally compact if $\phi(g)T$ and $T\phi(f)$ are compact operators for all $f\in C_0(X),g\in C_0(Y)$.
\item[$\bullet$] The support of $T$ is the complement of the set of points $(x,y)\in X\times X$ such that there exist $f_x,f_y\in C_0(X),C_0(Y)$ such that $f_x(x)\neq 0,f_{y}(y)\neq 0$ and $\phi(f_{y}) T \phi(f_x)=0$.
\item[$\bullet$] When $X=Y$, the propagation of $T$, denoted prop$(T)$, is the smallest $R>0$ such that supp $T \subseteq \Delta_R$.
\end{itemize}
\end{definition}

The following proposition says that there is actually one s.n.d. $X$-module, up to isomorphism. For $V\in\mathcal L(E,E')$, we denote $Ad_V : \mathcal L(E) \rightarrow \mathcal L(E') ; T\mapsto VTV^*$.

\begin{prop}\label{SND}
Let $H_X$ and $H'_X$ two s.n.d. $X$-modules. Then, for any $\varepsilon>0$, there exists an unitary $V\in\mathcal L(H_X,H'_X)$ such that $\text{supp }V \subseteq \Delta_\varepsilon$.
\end{prop}

\begin{dem}
By bounded functional calculus, extend the representations $\phi : C_0(X)\rightarrow \mathcal L(H_X)$ and $\phi' : C_0(X)\rightarrow \mathcal L(H'_X)$ to $\tilde \phi : l^\infty(X)\rightarrow \mathcal L(H_X) $ and $\tilde \phi' : l^\infty(X)\rightarrow \mathcal L(H'_X) $. Choose a Borel partition $\mathcal U$ of $X$ such that each Borel subset $U\in\mathcal U$ satifies diam$(U)\leq \varepsilon $ and has non-empty interior. If $\chi_U$ denotes the characteristic function of $U$, $p_U = \tilde\phi(\chi_U)$ and $p'_{U}=\tilde\phi'(\chi_U)$ define projections in $H_X$ and $H'_X$ respectively. Hence, if $H_U = p_U H_X$ and $H'_U= p'_U H'_X$, we have an orthogonal decomposition 
\[H_X = \bigoplus_{U_in\mathcal U} H_U \quad  \text{ and } \quad H'_X = \bigoplus_{U_in\mathcal U} H'_U.\]
But, the representations being standard, $H_U$ and $H'_U$ are separable infinite dimensional Hilbert spaces, hence there exists an isometry $V_U : H_U\rightarrow H'_U$. Define 
\[V = \bigoplus_{U\in\mathcal U} V_U : H_X = \bigoplus_{U_in\mathcal U} H_U \rightarrow  \bigoplus_{U_in\mathcal U} H'_U = H'_X.  \]
$V$ is an isometry, and supp$(V)\subseteq \coprod U\times U \subseteq \Delta_\varepsilon$.\\
\qed
\end{dem}

%--------------
% Roe algebras
%--------------

We now define the Roe algebra $C^*(X, H_X)$ of $X$ when we have fixed a standard non-degenerate $X$-module $H_X$. We will prove that it is unique up to unnatural isomorphism, and that this $*$-isomorphism induces a natural isomorphism in $K$-theory.\\

Define $C_R[X,H_X]$ as the following subspace of $\mathcal L(H_X)$ :
\[C_R[X,H_X] = \{T\in \mathcal L(H_X) \text{ locally compact  s.t. supp }T\subseteq \Delta_R \},\]
namely, $C_R[X,H_X]$ consists of locally compact operators with propagation less than $R$. It is a closed self-adjoint subspace of $\mathcal L(H_X)$ which satisfies $C_R[X,H_X].C_S[X,H_X]\subseteq C_{R+S}[X,H_X]$ for any positive numbers $R>0,S>0$.\\

Let $C[X,H_X]$ be the subspace 
\[\bigcup_{R>0} C_R[X,H_X]\subseteq \mathcal L(H_X).\] 
It is actually an involutive algebra naturally represented in a Hilbert space. Completion with respect to operator norm will naturally give $C^*$-algebras, which are called Roe algebras of $X$. They are the object of the following definition, but we need first to define another norm in order to define the maximal Roe algebra of $X$.\\

\begin{lem}
Let $(\pi, E_\pi)$ be a $*$-representation $\pi : C[X,H_X]\rightarrow \mathcal L(E_\pi)$ where $E_\pi$ is a Hilbert space. Let $T\in C_R[H_X]$, and define $N_R =\sup_{x\in X} |B(x,R)|<\infty$ and $T_{xy} = \chi_y T \chi_x$. Then $M=\sup_{x,y\in X} |T_{xy}|$ is finite and the following inequality holds :
\[||\pi(T)||\leq N_R M.\] 
\end{lem}
\begin{dem}
$X$ being discrete, $T=\sum_{x,y\in X} T_{xy}$, and 
\[||\pi(T)||\leq \sup_{x\in X} ||\sum_{y\in X} T_{xy}||\leq M \sup_{x\in X} |B(x,R)|.\]
\qed
\end{dem}

The maximal norm is defined, for any $T\in C[X,H_X]$, as 
\[||T||_{max} = \sup_{\pi\in\mathcal F} ||\pi(T)||\]
where $\pi$ runs in the family $\mathcal F$ of $*$-representations $\pi : C[X,H_X]\rightarrow \mathcal L(E_\pi)$. By the previous lemma, it is finite. 

\begin{definition}
Let $X$ be a discrete metric space with bounded geometry and fix a s.n.d. $X$-module. Let $C[X,H_X] $ be the subspace 
\[\bigcup_{R>0} C_R[X,H_X]\subseteq \mathcal L(H_X).\] 
\begin{itemize}
\item[$\bullet$] The Roe algebra of $X$, denoted $C^*(X)$, is defined as the closure of $C[X,H_X]$ under the operator norm.
\item[$\bullet$] The uniform Roe algebra of $X$, denoted $C_u^*(X)$, is defined as the closure of $C[X,l^2(X)]$ in the operator norm.
\item[$\bullet$] The maximal Roe algebra of $X$, denoted $C_{max}^*(X)$, is defined as the closure of $C[X,H_X]$ under the maximal norm. 
\end{itemize}
\end{definition}

Let us show that the definition of $C^*(X)$ does not depend on the choice of the s.n.d. $X$-module. Before doing so, we denote $C^*(X,H_X)$ the completion of $C[X,H_X]$ in order to keep track of the s.n.d. $X$-module.

\begin{prop}
Let $H_X$ and $H'_X$ be two s.n.d. $X$-modules. Then there exists an unnatural $*$-isomorphism $C^*(X,H_X)\cong C^*(X,H'_X)$ which is natural in $K$-theory.
\end{prop}
\begin{dem}
Let $H_X$ and $H'_X$ be two s.n.d. $X$-modules and $\varepsilon>0$. By proposition \ref{SND}, there exists an isometry $V\in\mathcal L(H_X,H'_X)$ with propagation less than $\varepsilon$. Hence $Ad_V(T) = V T V^*$ defines a bounded linear map $C_R[X,H_X]\rightarrow C_{R+2\varepsilon}[X,H_X']$ which extends to a $*$-isomorphism $Ad_V : C^*(X,H_X)\rightarrow C^*(X,H'_X)$.\\

Let $V_1$ and $V_2$ two isometries in $\mathcal L(H_X,H'_X)$ with propagation less than $\varepsilon$. Define $P : H_X\rightarrow H_X\oplus H'_X ; \xi \mapsto \xi \oplus 0$ and, for $t\in [0,\frac{\pi}{2}]$, $W_t : H_X \rightarrow H_X\oplus H'_X$ by 
\[W_t = 
\begin{pmatrix} \cos(t) & \sin (t)\\ -\sin (t) & \cos (t)\end{pmatrix}
\begin{pmatrix} V_1 & 0 \\ 0 & V_2 \end{pmatrix}
\begin{pmatrix} \cos(t) & \sin (t)\\ -\sin (t) & \cos (t)\end{pmatrix}^* P.\] 
Then $Ad_{W_t}$ is a homotopy between $Ad_{V_1}$ and $Ad_{V_2}$, so that 
\[(Ad_{V_1})_* = (Ad_{V_2})_* : K_*(C^*(X,H'_X))\rightarrow K_*(C^*(X,H'_X)).\]
\qed
\end{dem}

Let us define the Roe algebra of $X$ with coefficients in a $C^*$-algebra $B$. Define $C_R[X,B]$ as the following subspace of $\mathcal L_B(H\otimes l^2(X)\otimes B)$ :
\[C_R[X,B] = \{T\in \mathcal L_B(H\otimes l^2(X)\otimes B) \text{ locally compact  s.t. supp }T\subseteq \Delta_R \}.\]
It is a subspace of $\mathcal L_B(H\otimes l^2(X)\otimes B)$ which satisfies $C_R[X,B].C_S[X,B]\subseteq C_{R+S}[X,B]$. It is easy to see that 
\[C[X,B] = \bigcup_{R>0} C_R[X,B]\] 
is an involutive sub-algebra of $\mathcal L_B(H\otimes l^2(X)\otimes B)$.

\begin{definition}
Let $B$ be a $C^*$-algebra. The Roe algebra of $X$ with coefficients in $B$ is the completion of $C[X,B]$ under the operator norm of $\mathcal L_B(H\otimes l^2(X)\otimes B)$. 
\end{definition} 

In the next section, we will define the coarse groupoid $G(X)$, which is an étale groupoid, and express $C^*(X,B)$ as the reduced crossed-product of a well chosen $G(X)$-algebra.

%%%%%%%%%%%%%%%%%%%%%%%%%%%%%%%%
%%%%%%%%%%%%%%%%%%%%%%%%%%%%%%%%

\section{Coarse Groupoid}  %%

%%%%%%%%%%%%%%%%%%%%%%%%%%%%%%%%
%%%%%%%%%%%%%%%%%%%%%%%%%%%%%%%%

Recall that $\beta X$ denotes the Stone-Cech compactification of $X$. Let us first define the coarse groupoid of $X$. It is defined as the smallest topological groupoid $G(X)$ with unit space $\beta X$ extending the pair groupoid $X\times X$ over $X$. It was defined by G. Skandalis, J.L. Tu and G. Yu in \cite{SkTuYu}. It allows to translate coarse properties of $X$ into topological or dynamical properties of $G(X)$. Moreover, it turns out that Roe algebras with coefficients of $X$ can be expressed as crossed products of $G(X)$, and, as we will see in another chapter, that the coarse Baum-Connes conjecture for $X$ coincides with the Baum-Connes conjecture for this crossed-product.  \\

For any entourage $E$, let $\overline E$ denotes the closure of $E$ in $\beta (X\times X)$. Define $G(X) = \cup_{E\in\mathcal E_X} \overline E$. By universal property of the Stone-Cech compactification, the first and second projections $X\times X\rightarrow X$ extend to continuous maps $G(X)\rightarrow \beta X$ denoted $s$ and $r$ respectively, as shown in this commutative diagramm :  
\[\begin{tikzcd}
X\times X \arrow{r}\arrow{d}{\iota_{X\times X}} &  X \arrow{r}{\iota_X} & \beta X \\
\beta (X\times X) \arrow[dotted]{urr} & & 
\end{tikzcd}.\]
The same remains true for the inverse map $(x,y)\mapsto (y,x)$ and the unit map $x\mapsto (x,x)$.The following lemma defines the groupoid structure on $G(X)$. The reader can check \cite{RoeCoarse} for a proof.

\begin{lem}\cite{RoeCoarse}
For any entourage, the map $(s,r) : E\rightarrow X\times X$ extends to a topological embedding $\overline E \hookrightarrow \beta X\times \beta X$.
\end{lem}

Using the lemma, we get a topological embedding $G(X)\hookrightarrow \beta X\times \beta X$, and the multiplication map is defined as conjugation by this topological embedding of the multiplication map on the pair groupoid $\beta X\times \beta X$.\\

If $E$ is an entourage, $E$ is contained in a finite union of entourage $E_1\cup \cdots \cup E_n$ such that $s: X\times X \rightarrow X$ and $r: X\times X \rightarrow X$ are injective when restricted to each $E_k$ \cite{RoeCoarse}. We call such entourages \textbf{partial translations}. Every partial translation $E$ acts as a partial bijection $s(E)\rightarrow r(E)$ on $X$ in the following way : $E.x = r\circ (s_{|E})^{-1}(x)$ if $x\in s(E)$. The composition of two partial translations remains a partial translation, and if $E$ and $E'$ are partial translations, then $E'.(E.x) = (E'\circ E).x$ for all $x\in s(E)$. In other words, partial translations of $X$ are a semigroup which acts on $X$.

\begin{prop}\cite{SkTuYu} $G(X)\rightrightarrows \beta X$ is an étale groupoid such that $G(X)_{|X}$ is the pair groupoid $X\times X\rightrightarrows X$.  
\end{prop}

\begin{Expl} If $\Gamma$ is a finitely generated group, then the action of $\Gamma$ on $|\Gamma|$ extend to $\beta | \Gamma |$. In \cite{SkTuYu}, the reader can find a proof that $G( | \Gamma | )\cong \beta | \Gamma | \rtimes \Gamma $.
\end{Expl}

\begin{Expl} Let $\Gamma$ be a finitely generated group, residually finite w.r.t. a nested family $\mathcal N$ of normal subgroups with trivial intersection, and $X=X_{\mathcal N}(\Gamma)$. Then the restriction $G(X)_{|\partial \beta X}$ of the coarse groupoid is isomorphic to $\partial \beta X\rtimes \Gamma$. See \cite{FinnCoarse} for a proof.
\end{Expl}

We compile in the next theorem a list of some kown relations between topological properties of $G(X)$ and coarse properties of $X$.

\begin{thm}\label{propertiesXG}
Let $X$ be a discrete metric space with bounded geometry. Then :
\begin{itemize}
\item[$\bullet$] $X$ has property $A$ iff $G(X)$ is amenable \cite{SkTuYu};
\item[$\bullet$] $X$ admits a coarse embedding into Hilbert space iff $G(X)$ is a-T-menable \cite{SkTuYu};
\item[$\bullet$] $\partial \beta X$ is a $G$-invariant closed subset of $\beta X$ hence $G(X)_{|\partial\beta X}$ is a subgroupoid of $G(X)$. Moreover $X$ admits a fibred coarse embedding into Hilbert space iff $G(X)_{|\partial\beta X}$ is a-T-menable \cite{FinnSellFibred}.
\end{itemize}
\end{thm}

The remaining of the section is devoted to prove that the Roe algebra of $X$ is isomorphic to a reduced cross-product of a well chosen $C^*$-algebra by $G(X)$. For the reader's convenience, we will abreviate $G(X)$ as $G$.\\

For any $C^*$-algebra $B$, let $\tilde B$ denotes the $C^*$-algebra $l^\infty(X,B\otimes\mathfrak K)$. Multiplication by $l^\infty(X)\cong C(\beta X)$ provides a $C(\beta X)$-structure on $\tilde B$. Recall that $s^* \tilde B = C_0(G)\otimes_s \tilde B$ and $r^* \tilde B = C_0(G)\otimes_r \tilde B$. As $C_0(G)\subseteq C(\beta(X\times X))\cong l^\infty(X\times X)$, these algebras are generated by elementary tensors $\chi_E\otimes_s f$ and $\chi_E\otimes_r f$ where $f\in\tilde B$ and $E$ is a partial translation. Let us first define the action of $G$ on $\tilde B$, which is an isomorphism of $C(\beta X)$-algebras $V :s^* \tilde B\rightarrow r^* \tilde B$.\\

Suppose $E$ is a partial translation and $f\in\tilde B$. Define 
\[E.f(x) = \left\{\begin{array}{ll} f(E^{-1}.x) & \text{ if }x\in r(E)\\ 0 & \text{otherwise.}\end{array}\right.\]
Off course $E(E'f)=(E\circ E')f$. Define the action on elementary tensors as follows :
\[V(\chi_E \otimes_s f) = \chi_E \otimes_r (E.f).\]

\begin{lem}
$V$ defines an action of $G$ on $\tilde B$.
\end{lem}

\begin{dem}
Let us show that $V$ extends to a isomorphism of $C(\beta X)$-algebras from $s^* \tilde B$ to $r^*\tilde B$.\\

$G$ is totally disconnected, and the set of the closure in $\beta (X\times X)$ of controlled subsets is a basis for the topology. Hence $C_0(G)$ is the closure of the algebra of locally constant functions with finite value set. We can suppose these functions are of the form $f = \sum_{j=1}^k \lambda_j\chi_{E_j}$, the $E_j$ being mutually disjoint partial translations, and $\lambda_j\in \C$. The $C(\beta X)$-algebras $s^* \tilde B = C_0(G) \otimes_s \tilde B$ and $r^* \tilde B = C_0(G) \otimes_r \tilde B$ are similarly generated by functions of the form $h = \sum_{j=1}^k \lambda_j\chi_{E_j}\otimes f_j$ with the $E_j$ and $\lambda_j$ as before, and $f_j\in\tilde B$. Let $h$ be such a function. Then the expression $\sum_{j=1}^k \lambda_j\chi_{E_j}\otimes_r (E_j.f_j)$ does not depend on the decomposition. Indeed, if $E_1$ and $E_2$ are partial translations such that $E_1\subseteq E_2$, then $\chi_{E_1} \otimes_r E_2.f = \chi_{E_1} \otimes_r E_1.f $. Consider two decompositions $\sum_{i} \alpha_i\chi_{E_i}\otimes f_i= \sum_{j} \beta_j\chi_{F_j}\otimes g_j$. Notice that $\alpha_i \chi_{E_i\cap F_j }\otimes f_i = \beta_j \chi_{E_i\cap F_j }\otimes g_j$. Then :
\[ \begin{array}{rl}
  \sum_{i} \alpha_i\chi_{E_i}\otimes (E_i.f_i)  & = \sum_{i,j} \alpha_j\chi_{E_i\cap F_j}\otimes (E_i.f_i)         \\
						& = \sum_{i,j} \alpha_j\chi_{E_i\cap F_j}\otimes (E_i\cap F_j).f_i \\
						& = \sum_{i,j} \beta_j\chi_{E_i\cap F_j}\otimes (E_i\cap F_j).g_j  \\
						& = \sum_{j} \beta_j\chi_{F_j}\otimes (F_j.g_j)
\end{array}\]

Moreover :
\[ \begin{array}{rl} 
||\sum_j \lambda_j\chi_{E_j}\otimes(E_j.f)|| 
 \ & \leq \sum_j |\lambda_j|\ ||f_j|| \\ 
 \ & \leq \sup \{ \sum_j |\lambda'_j| \times  ||f'_j||\text{ s.t. }\sum \lambda'_j \chi_{E'_j}\otimes f'_j  =h\}, \\ 
\ & \leq ||h||, \\
\end{array}\]

%\[ \begin{array}{rl} 
%||(\sum_j \lambda_j\chi_{E_j}, (E_j.f))|| 
% \ & \leq (\sup_j |\lambda_j|) \times  ||f|| \\ 
 %\ & \leq \sup \{ \sum_j |\lambda'_j| \times  ||f'_j||\text{ s.t. }\sum \lambda'_j \chi_{E'_j}\otimes f'_j  =h\}=||h||, \\ 
%\ & \leq ||\sum_j \lambda_j \chi_{E_j}|| \times  ||f||, \\
%\end{array}\]
so $V$ extends by continuity to a continuous linear map $ C_0(G)\otimes\tilde B \rightarrow r^* \tilde B$. This map being $C(\beta X)$-linear, $V$ extends to a $C(\beta X)$-algebra morphism $V :s^*\tilde B \rightarrow r^* \tilde B$. The rule $\chi_E \otimes_r f\mapsto \chi_E \otimes_r (E^{-1}.f)$ defines similarly a $C(\beta X)$-algebra morphism between $r^*\tilde B \rightarrow s^*\tilde B$. It is the inverse of $V$, hence $V$ is a isomorphism of $C(\beta X)$-algebras.\\

Let us show $V$ is an action. For $g\in G$, denote by $V_g : (\tilde B)_{s(g)}\rightarrow (\tilde B)_{r(g)}$ the induced map between the fibers of $s^*\tilde B$ and $r^*\tilde B$. Let $(g,g')\in G^{(2)}$, and choose partial translations $E, E'\subseteq X\times X$ such that $g\in \overline E$ and $g'\in \overline{E'}$. Then $g'\circ g\in \overline{E'\circ E}$ and $E'\circ E$ is a partial translation. Then if $f\in s^* \tilde B$, $V_g(f_{s(g)}) = (E.f)_{r(g)}$. Then :
\[ \begin{array}{rl}
V_{g'g}(f_{s(g)}) & = ((E'\circ E).f)_{r(g')}         \\
		& = V_{g'}((E.f)_{s(g')})) \\
		& = V_{g'} ((E.f)_{r(g)})  \\
		& = V_{g'} V_{g} (f_{s(g)}) \\
\end{array}\]
so that $V_{g'g}= V_{g'} V_g$, and $V$ defines an action of $G$ on $\tilde B$.\\
\qed
%If $r(\overline E)\neq s(\overline {E'})$, we can take $U=r(\overline E)\cap s(\overline {E'})$ to get partial translations $F = (r_{|E})^{-1}(U)$ and $F' = (s_{|E'})^{-1}(U)$ which satisfy the same hypothesis. Let us assume then that $r(\overline E) = s(\overline {E'})$, hence $\chi_E\otimes_r f = \chi_E'\otimes_s f = $ and $\chi_E'\otimes_r f = \chi_{E'\circ E} \otimes_r f = $ for any $f\in \tilde B$. Then, if $f\in \tilde B$ :
%\[\begin{array}{rl} V_{g'}(V_g(f_{s(g))})) =  & V_{g'}\left( \varinjlim_{F : g\in \overline F} \chi_E \otimes_r E.f\right) \\
%				=&  \varinjlim_{F' : g'\in \overline{F'}\subseteq \overline{E'}}\varinjlim_{F : g\in \overline F\subseteq \overline E} \chi_{E'} \otimes_r (E'\circ E).f \\
%				=& \varinjlim_{F' : g'\in \overline{F'}\subseteq \overline{E'}}\varinjlim_{F : g\in \overline F\subseteq \overline E} \chi_{E'\circ E} \otimes_r (E'\circ E).f \\
%				=& \varinjlim_{F : g'g\in \overline{F}\subseteq \overline{E'\circ E}} \chi_{E'\circ E} \otimes_r (E'\circ E).f \\
%				=& V_{g'g}(f_{s(g)})\\ 
%\end{array} \] 
\end{dem}

\begin{thm}\label{IsomCoarseGroupoid}
Let $X$ be a discrete metric space with bounded geometry, and $B$ a $C^*$-algebra. There exists a natural isomorphism 
\[\begin{tikzcd}\Psi_B :  \ l^\infty(X,B\otimes\mathfrak K)\rtimes_r G(X) \arrow{r}{\cong} &   C^*(X,B)\end{tikzcd}.\] 
\end{thm}

\begin{dem}
%A FAIRE
The proof is a detailed version of the proof of \cite{SkTuYu}. The idea is to represent faithfully the two $C^*$-algebras on the same Hilbert module, and to show that they are equal in this representation.\\

By definition, $\tilde B\rtimes_r G$ is faithfully represented on $L^2(G,\tilde B)$. Consider the $G$-invariant ideal $J= C_0(X,B\otimes\mathfrak K)$ and the Hilbert $\tilde B$-module $L^2(G,J) = L^2(G,\tilde B)\otimes J$. The $*$-homomorphism $\mathcal L_{\tilde B}(L^2(G,\tilde B)) \rightarrow \mathcal L_{\tilde B}(L^2(G,J))$ ; $T\mapsto T\otimes 1 $ is isometric, so we obtain a faithful representation $\tilde B\rtimes_r G \rightarrow \mathcal L_{\tilde B}(L^2(G,J))$.\\

But $L^2(G,J)  \cong C_0(X) \otimes L^2(G)\otimes B\otimes \mathfrak K$, hence $C_0(X)\otimes l^2(X) \otimes B \otimes \mathfrak K$ is a submodule of $L^2(G,J)$. Notice that, as a Hilbert space, $\mathfrak K \cong H$, hence $C^*(X,B)$ is faithfully represented on $L^2(G,J)$, acting trivially on the $C_0(X)$ factor. \\

Recall that $\tilde B\rtimes_r G$ is generated by elementary tensors $\chi_E\otimes f$, where $E$ is a partial translation and $f\in\tilde B$. To such a $\chi_E\otimes f$, associate the following operator $T$ on $l^2(X)\otimes H\otimes B$ : 
\[T_{xy} =\left\{\begin{array}{ll} f(x) & \text{ if }(x,y)\in E\\ 0 & \text{otherwise.}\end{array}\right. \]
Then, if $R=\sup_E d$, $T\in C_R[X,B]$, and the images of $\chi_E\otimes f$ and $T$ coincide under the faitfull representations described above, so that $\chi_E\otimes f\rightarrow T$ extends to an isomorphism $\Psi_B$.\\

If $\phi : A\rightarrow B$ is a $*$-homomorphism, it induces $\tilde \phi : \tilde A\rightarrow \tilde B$ and $\phi_X : C^*(X,A)\rightarrow C^*(X,B)$. Naturality of $\Psi$ follows then from functoriality of the Hilbert module tensor product and functoriality of $l^\infty(X, - \otimes \mathfrak K)$ and $C^*(X, - )$.\\   
\qed
\end{dem}

\begin{rk} A minor modification of this proof yields that $C_r^*(G(X))\cong C^*_u(X)$. Let $\Gamma$ be a finitely generated group. Then $C_u^*(| \Gamma |)\cong l^\infty( \Gamma)\rtimes_r \Gamma  $.
\end{rk}
