\subsection{Coarse Geometry}

Some notations are in order. Let $X$ be a set. For any subsets $A$ and $B$ of $X\times X$, 
\begin{itemize}
\item[$\bullet$] let $A\circ B = \{(x,z)\in X\times X \text{ s.t. } \exists y\in X\text{ s.t. } (x,y)\in A, (y,z)\in B\}$,
\item[$\bullet$] and $A^{-1} = \{(x,y)\in X\times X \text{ s.t. } (y,x)\in A\}$.\\
\end{itemize}

Let $(X,d)$ be a proper discrete metric space with bounded geometry. In this section, we define the Roe algebras of $X$.\\

\begin{definition}\cite{RoeCoarse}
A coarse structure on a set $X$ is a subset $\mathcal E$ of $\mathcal P(X\times X)$ such that :
\begin{itemize}
\item[$\bullet$] if $A$ and $B$ are in $\mathcal E$, so are $A^{-1}$, $A\cup B$ and $A\circ B$,
\item[$\bullet$] every finite subset of $X\times X$ is in $\mathcal E$,
\item[$\bullet$] if $A\in\mathcal E$ and $B\subset A$, then $B\in \mathcal E$.
\end{itemize}
\end{definition} 

\begin{Expl}
When $X$ is a discrete proper metric space, we define, for all positive number $R>0$, the $R$-diagonal as 
\[\Delta_R= \{(x,y)\in X\times X \text{ s.t. } d(x,y)\leq R\}.\]
A set $E\subset X\times X$ is called an entourage if there exists $R>0$ such that $E\subset \Delta_R$. The set of entourage $\mathcal E_X$ is a coarse structure on $X$.
\end{Expl}

For now on, $X$ will be a discrete proper metric space with bounded geometry, i.e. such that, for all $R>0$, $\sup_{x\in X} |B(x,R)|$ is finite. 

\begin{definition} A $X$-module is a pair $(H_X,\phi)$ where :
\begin{itemize}
\item[$\bullet$] $H_X$ is a Hilbert space,
\item[$\bullet$] $\phi : C_0(X)\rightarrow \mathcal L(H_X)$ is a $*$-homomorphism. 
\end{itemize}
The module is said to be non-degenerate if the closure of the subspace generated by $\{\phi(f)\xi : f\in C_0(X), \xi\in H_X\}$ is dense in $H_X$, and it is called standard if no non-zero function acts as a compact operator, i.e. $\phi(f) \in\mathfrak K(H_X)$ implies $f=0$.
\end{definition}

We will sometime allow ourself a slight abuse of notation by writing $f\xi$ for $\phi(f)\xi$ when $f\in C_0(X)$ and $\xi \in H_X$.

\begin{Expl}
Let $H$ denotes the separable Hilbert space. Continuous functions $C_0(X)$ act on $H_X =l^2(X)\otimes H$ as bounded operators, which gives $H_X$ the structure of a $X$-module, called the standard $X$-module. It is standard non-degenerate ( s.n.d. for short).  
\end{Expl}

The following proposition says that there is actually one s.n.d. $X$-module, up to isomorphism.

\begin{prop}
Let $H_X$ and $H'_X$ two s.n.d. $X$-module. Then, for any $\varepsilon>0$, there exists an isometry $V\in\mathcal L(H_X,H'_X)$ such that $\text{supp }V \subset \Delta_\varepsilon$. 
\end{prop}

\begin{dem}
% A FAIRE 
\qed
\end{dem}

\begin{definition}
Let $X$ and $Y$ be two discrete proper metric spaces with bounded geometry, $H_X$ and $H_Y$ $X$ and $Y$-modules respectively, and $T\in \mathcal L(H_X, H_Y)$ be a bounded operator.
\begin{itemize}
\item[$\bullet$] $T$ is said to be locally compact if $\phi(g)T$ and $T\phi(f)$ are compact operators for all $f\in C_0(X),g\in C_0(Y)$.
\item[$\bullet$] The support of $T$ is the complement of the set of points $(x,y)\in X\times X$ such that there exist $f_x,f_y\in C_0(X),C_0(Y)$ such that $f_x(x)\neq 0,f_{y}(y)\neq 0$ and $\phi(f_{y}) T \phi(f_x)=0$.
\item[$\bullet$] When $X=Y$, the propagation of $T$ is the smallest $R>0$ such that supp $T \subset \Delta_R$.
\end{itemize}
\end{definition}

We now define the Roe algebra $C^*(X, H_X)$ of $X$ when we have fixed a standard non-degenerate $X$-module $H_X$. We will prove that it is unique up to unnatural isomorphism, and that this $*$-isomorphism induces a natural isomorphism in $K$-theory.\\

Define $C_R[X,H_X]$ as the following subspace of $\mathcal L(H_X)$ :
\[C_R[X,H_X] = \{T\in \mathcal L(H_X) \text{ locally compact  s.t. supp }T\subset \Delta_R \}.\]

Define $C_R[X,B]$ as the following subspace of $\mathcal L_B(H\otimes l^2(X)\otimes B)$ :
\[C_R[X,B] = \{T\in \mathcal L_B(H\otimes l^2(X)\otimes B) \text{ locally compact  s.t. supp }T\subset \Delta_R \}.\]

% A FINIR

\subsection{Coarse groupoid}

Recall that $\beta X$ denotes the Stone-Cech compactification of $X$. Let us first define the coarse groupoid of $X$. It is defined as the smallest topological groupoid $G(X)$ with unit space $\beta X$ extending the pair groupoid $X\times X$ over $X$.\\

For any entourage $E$, let $\overline E$ denotes the closure of $E$ in $\beta (X\times X)$. Define $G(X) = \cup_{E\in\mathcal E_X} \overline E$. By universal property of the Stone-Cech compactification, the first and second projections $X\times X\rightarrow X$ extend to continuous maps $G(X)\rightarrow \beta X$ denoted $s$ and $r$ respectively. The same remains true for the inverse map $(x,y)\mapsto (y,x)$ and the unit map $x\mapsto (x,x)$.The following lemma defines the groupoid structure on $G(X)$. The reader can check \cite{RoeCoarse} for a proof.

\begin{lem}\cite{RoeCoarse}
For any entourage, the map $(s,r) : E\rightarrow X\times X$ extends to a topological embedding $\overline E \hookrightarrow \beta X\times \beta X$.
\end{lem}

Using the lemma, we get a topological embedding $G(X)\hookrightarrow \beta X\times \beta X$, and the multiplication map is defined as conjugation by this topological embedding of the multiplication map on the pair groupoid $\beta X\times \beta X$.\\

The remaining of the section is devoted to prove that the Roe algebra of $X$ is isomorphic to a reduced cross-product of a well chosen $C^*$-algebra by $G(X)$. For the reader's convenience, we will abreviate $G(X)$ as $G$.\\

For any $C^*$-algebra $B$, let $\tilde B$ denotes the $C^*$-algebra $l^\infty(X,B\otimes\mathfrak K)$. Multiplication by $l^\infty(X)\cong C(\beta X)$ provides a $C(\beta X)$-structure on $\tilde B$. Recall that $s^* \tilde B = C_0(G)\otimes_s \tilde B$ and $r^* \tilde B = C_0(G)\otimes_r \tilde B$. Let us first define the action of $G$ on $\tilde B$, which is an isomorphism of $C(\beta X)$-algebras $V :s^* \tilde B\rightarrow r^* \tilde B$.\\

If $E$ is an entourage, $E$ is contained in a finite union of entourage $E_1\cup \cdots E_n$ such that $s: X\times X \rightarrow X$ and $r: X\times X \rightarrow X$ are injective when restricted to each $E_k$ \cite{RoeCoarse}. We call such entourages \textbf{partial translations}. Every partial translation $E$ acts as a partial bijection on $X$ in the following way : $E.x = r\circ (s_{|E})^{-1}(x)$ if $x\in s(E)$. The composition of two partial translation remains a partial translation, and if $E$ and $E'$ are partial translations, then $E'.(E.x) = (E'\circ E).x$ for all $x\in s(E)$.\\

Suppose $E$ is a partial translation and $f\in\tilde B$. Define 
\[E.f(x) = \left\{\begin{array}{ll} f(E.x) & \text{ if }x\in s(E)\\ 0 & \text{otherwise.}\end{array}\right.\]
Both $s^* \tilde B$ and $r^* \tilde B$ are generated by elementary tensors of the form $\chi_E \otimes_s f$ and $\chi_E \otimes_r f$. Define the action on elementary tensors as follows :
\[V(\chi_E \otimes_s f) = \chi_E \otimes_r (E.f).\]

\begin{lem}
$V$ defines an action of $G$ on $\tilde B$.
\end{lem}

\begin{dem}
% A FINIR
\qed
\end{dem}

\begin{thm}
Let $X$ be a discrete metric space with bounded geometry, and $B$ a $C^*$-algebra. There exists a natural isomorphism 
\[\Psi_B : l^\infty(X,B\otimes\mathfrak K)\rtimes_r G(X) \rightarrow C^*(X,B).\] 
\end{thm}

\begin{dem}
%A FAIRE
\qed
\end{dem}
















 