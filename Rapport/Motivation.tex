\section*{Background and historical perspective}

After the pioneering work of Grothendieck on the Riemann-Roch theorem, $K$-theory was generalized to the topological setting by Atiyah and Hirzebruch. This construction was crucial for the proof of the index theorem by Atiyah and Singer. From another perspective, $C^*$-algebras define a natural extension of classical topology to the noncommutative setting. The Serre-Swan theorem allows to define $K$-theory for $C^*$-algebras, which generalizes the topological $K$-theory. This operator $K$-theory led to a huge amount of work and interest, based on the wide range of its applications. From classification of $C^*$-algebras to index theorems, even appearing in some areas of theoretical physics, $K$-theory has been the subject of increasing interest. Instead of detailing precisely all of these applications, which would be almost impossible, we propose to adopt the following axiom.\\   
 
\textbf{AXIOM 1} Computing the $K$-theory groups of a $C^*$-algebra is a difficult and interesting problem.\\ 

During the eighties, Alain Connes and Paul Baum conjectured that one could describe the $K$-theory of the reduced $C^*$-algebra of a discrete group with a geometrical object, what is now called the analytical $K$-homology of the classifying space for proper action. They gave a relation between the two objects, embodied in the assembly map. Since then, the assembly map has been defined for locally compact group, for actions of groups by automorphisms on $C^*$-algebras, for topological groupoids, for coarse spaces and, in some cases, for quantum groups. The Baum-Connes conjecture, and its variants, state that the assembly map is an isomorphism, thus providing an algorithm to compute the $K$-theory of the $C^*$-algebra under consideration.\\

Great success was achieved in the study of the conjecture. It is known to hold :
\begin{itemize}
\item[$\bullet$] for groups with the Haagerup property, by work of N. Higson and G. Kasparov \cite{higsonkasparov},
\item[$\bullet$] for some groups which may satisfy Kazhdan's property T, by work of V. Lafforgue \cite{Lafforgue}, 
\item[$\bullet$] for almost connected group, by work of J. Chabert, S. Echterhoff and R. Nest \cite{chabertEN},
\item[$\bullet$] for groupoids which admits a proper action by isometries on a filed of Hilbert space, by work of J-L. Tu \cite{TuThese},
\item[$\bullet$] for coarse space which satisfy property A, by work of G. Yu \cite{Yu2}.
\end{itemize}  
Even if these class of examples is very large, counterexamples have been exhibited by N. Higson, V. Lafforgue, and G. Skandalis in \cite{HigsonLaffSk}. One of the interest of the Baum-Connes conjectures is the way they relate to other conjectures. For instance, the injectivity of the Baum-Connes assembly map for a discrete group ensures that, by a principle descent, the group satisfies the Novikov conjecture for higher signatures. The different assembly maps also interelates between each other. In \cite{SkTuYu}, G. Skandalis, J-L. Tu and G. Yu proved that the coarse assembly map is equivalent to the assembly map associated to a topological groupoid canonically defined from the coarse space. \\

We are interested by G. Yu's celebrated proof of the Baum-Connes conjecture for coarse spaces with finite asymptotic dimension \cite{Yu1}. It is worth noticing that any coarse space with finite asymptotic dimension satisfies property A, hence this proof is less general that the one in \cite{Yu2}. Nonetheless, the idea is very different. To simplify, we could see this proof as a "controlled cutting and pasting" : the finite asymtpotic-dimensional coarse space is cut into smaller pieces, which are shown to satisfy the coarse Baum-Connes conjecture. One need then to paste the different pieces in a way compatible with the assembly maps. It is in this part of the proof that appeared some modified $K$-theory groups that were essential to the proof. \\

In order to apply this strategy to more general situations, controlled $K$-theory was developed in \cite{OY2} by H. Oyono-Oyono and G. Yu. Controlled $K$-theory is defined in the setting of filtered $C^*$-algebras. 
