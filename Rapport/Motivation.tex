\section*{Background and historical perspective}

After the pioneering work of Grothendieck on the Riemann-Roch theorem, $K$-theory was generalized to the topological setting by Atiyah and Hirzebruch. This construction was crucial for the proof of the index theorem by Atiyah and Singer. From another perspective, $C^*$-algebras define a natural extension of classical topology to the noncommutative setting. The Serre-Swan theorem allows to define $K$-theory for $C^*$-algebras, which generalizes the topological $K$-theory. This operator $K$-theory led to a huge amount of work and interest, based on the wide range of its applications. From classification of $C^*$-algebras to index theorems, even appearing in some areas of theoretical physics, $K$-theory has been the subject of increasing interest. Instead of detailing precisely all of these applications, which would be almost impossible, we propose to adopt the following axiom.\\   
 
\textbf{AXIOM 1} Computing the $K$-theory groups of a $C^*$-algebra is a difficult and interesting problem.\\ 

During the eighties, Alain Connes and Paul Baum conjectured that one could describe the $K$-theory of the reduced $C^*$-algebra of a discrete group with a geometrical object, what is now called the analytical $K$-homology of the classifying space for proper action. They gave a relation between the two objects, embodied in the assembly map. Since then, the assembly map has been defined for locally compact group, for actions of groups by automorphisms on $C^*$-algebras, for topological groupoids, for coarse spaces and, in some cases, for quantum groups. The Baum-Connes conjecture, and its variants, state that the assembly map is an isomorphism, thus providing an algorithm to compute the $K$-theory of the $C^*$-algebra under consideration.\\

More precisely, for any locally compact group $G$, there exists a homomorphism of $\Z_2$-graded abelian groups $\mu_G : K^{top}(G)\rightarrow K(C_r^*(G))$ where :\\
\begin{itemize}
\item[$\bullet$] $K^{top}(G)$ is the analytical $K$-homology group of $G$,
\item[$\bullet$] $C^*_r(G)$ is the reduced $C^*$-algebra of $G$ and $K(C_r^*(G))$ is its (operator) $K$-theory groups.\\
\end{itemize}
%%%% BAUM CONNES
The Baum-Connes conjecture is the following claim.

\begin{conj}[Baum-Connes]
For every locally compact group $G$, $\mu_G$ is an isomorphism.
\end{conj}

There exists more general versions of the conjecture : first, one can take $G$ to be a locally compact $\sigma$-compact groupoid with a Haar system. Let $A$ be a $G$-algebra, which is a $C^*$-algebra endowed with an action of $G$. It is possible to built a $C^*$-algebra $A\rtimes_r G$ out of the action, called the reduced crossed product, such that the reduced crossed product with $\C$ (endowed with the trivial $G$-action) coincides with $C^*_r(G)$. Then, we can consider a coefficient version of the assembly map $\mu_{G,A} : K^{top}(G,A)\rightarrow K(A\rtimes_r G)$. The Baum-Connes conjecture for $G$ with coefficients in $A$ is the claim that $\mu_{G,A}$ is an isomorphism. We say that $G$ satisfies the Baum-Connes conjecture if the Baum-Connes conjecture for $G$ with coefficients in $A$ holds, for every $G$-algebra $A$.\\
%%%%% COARSE BAUM CONNES
Inspired by these ideas, a coarse assembly map was defined. For every coarse space $X$ and every $C^*$-algebra $A$, there exists a homomorphism of $\Z_2$-graded abelian groups $\mu_X : KX(X,A)\rightarrow K(C^*(X,A))$ where :\\
\begin{itemize}
\item[$\bullet$] $KX(X,A)$ is the coarse $K$-homology group of $X$ with coefficients in $A$,
\item[$\bullet$] $C^*(X,A)$ is the Roe algebra of $X$ with coefficients in $A$.\\
\end{itemize}

\begin{conj}[Coarse Baum-Connes]
For every coarse space $X$ with bounded geometry, $\mu_{X,\C}$ is an isomorphism.
\end{conj}

The coarse Baum-Connes conjecture admits version with coefficients as in the group setting.\\ 

Great success was achieved in the study of the conjecture. It is known to hold :\\
\begin{itemize}
\item[$\bullet$] for groups with the Haagerup property, by work of N. Higson and G. Kasparov \cite{higsonkasparov},
\item[$\bullet$] for some groups which may satisfy Kazhdan's property T, by work of V. Lafforgue \cite{Lafforgue}, 
\item[$\bullet$] for almost connected group, by work of J. Chabert, S. Echterhoff and R. Nest \cite{chabertEN},
\item[$\bullet$] for groupoids which admits a proper action by isometries on a field of Hilbert space, by work of J-L. Tu \cite{TuThese},
\item[$\bullet$] for coarse space which satisfy property A, by work of G. Yu \cite{Yu2}.\\
\end{itemize}  
Even if these class of examples is very large, counterexamples have been exhibited by N. Higson, V. Lafforgue, and G. Skandalis in \cite{HigsonLaffSk}. One of the interest of the Baum-Connes conjectures is the way they relate to other conjectures. For instance, the injectivity of the Baum-Connes assembly map for a discrete group ensures that, by a principle descent, the group satisfies the Novikov conjecture for higher signatures. The different assembly maps also interelates between each other. In \cite{SkTuYu}, G. Skandalis, J-L. Tu and G. Yu proved that the coarse assembly map is equivalent to the assembly map associated to a topological groupoid canonically defined from the coarse space. We now give some details concerning these relations.\\

Let us remind the reader of the statement of the Novikov conjecture. The reader is referred to \cite{kreckluck} for details. Let $\Gamma$ be a discrete group. Then there exists a unique, up to homotopy, connected CW-complex $B\Gamma$ such that $\pi_1(B \Gamma) \cong \Gamma$ and such that its universal covering $\tilde{B\Gamma}$ is contractible \cite{May}. This space $B\Gamma$ is called the classifying space. For any smooth oriented closed manifold $M$, denote by \[\mathcal L_M \in \bigoplus_{k\geq 0} H^{4k}(M,\mathbb Q).\] the $L$-class of $M$. Recall (see \cite{kreckluck} for precise definitions) that each components $(\mathcal L_M)_k\in H^{4k}(M,\mathbb Q) $ is a homogeneous polynomial of degree $k$ in the rational Pontrjagin classes. For any map $f: M\rightarrow B\Gamma$, define the higher signature 
\[\sigma_x(M,f) = \langle \mathcal L_M \cup f^*(x),[M] \rangle \quad \forall x\in H(B\Gamma,\mathbb Q).\]
Here $[M]\in H_{\text{dim }M}(M,\mathbb Q)$ is the fundamental class of the closed oriented manifold $M$, and $\langle \ , \ \rangle $ is the pairing between cohomology and homology, given by Kronecker product. The Novikov conjecture claims that higher signatures are homotopy invariant.
\begin{conj}[Novikov]\label{Novikov}
Let $M$ and $N$ two smooth oriented closed manifolds, $f: N\rightarrow B\Gamma$ a map and $\phi : M\rightarrow N$ an orientation preserving homotopy equivalence. Then 
\[\sigma_x(N,f)= \sigma_x(M,\phi\circ f)\quad \forall x\in H(B\Gamma,\mathbb Q).\]
\end{conj}

The relation between the Novikov conjecture and the Baum-Connes conjecture is contained in the following result (\cite{kreckluck}, Corollary $23.15$).
\begin{thm}
Let $\Gamma$ be a discrete group. Then the Baum-Connes conjecture for $\Gamma$, or more generally the rational injectivity of the Baum-Connes assembly map $\mu_\Gamma$, implies the Novikov conjecture for $\Gamma$.
\end{thm}
The study of the relations between the Novikov conjecture and operator algebras was initiated by Lusztig \cite{lusztig}. Another connection was revealed in the article of G. Skandalis, JL. Tu and G.Yu \cite{SkTuYu}. They built, out of every coarse space with bounded geometry $X$, an étale groupoid $G(X)$ such that :\\
\begin{itemize}
\item[$\bullet$] $l^\infty_B \rtimes_r G(X)$ and $C^*(X,B)$ are isomorphic as $C^*$-algebras,
\item[$\bullet$] $\mu_{X,B}$ and $\mu_{G(X),l^\infty_B}$ are equivalent,\\
\end{itemize}
where $l^\infty_B$ is the $G$-algebra $l^\infty(X,B\otimes\mathfrak K)$.\\

%%%%%%%%%%%%%% Explication de la these : origine de la K theorie controllee
We are interested by G. Yu's celebrated proof of the Baum-Connes conjecture for coarse spaces with finite asymptotic dimension \cite{Yu1}. It is worth noticing that any coarse space with finite asymptotic dimension satisfies property A, hence this proof is less general that the one in \cite{Yu2}. Nonetheless, the idea is very different. To simplify, we could see this proof as a "controlled cutting and pasting" : the finite asymtpotic-dimensional coarse space is cut into smaller pieces, which are shown to satisfy the coarse Baum-Connes conjecture. One need then to paste the different pieces in a way compatible with the assembly maps. It is in this part of the proof that appeared some modified $K$-theory groups that were essential for the "cutting and pasting". \\

In order to apply this strategy to more general situations, controlled $K$-theory was developed in \cite{OY2} by H. Oyono-Oyono and G. Yu. Controlled $K$-theory is defined in the setting of filtered $C^*$-algebras. Roe algebras of metric spaces, crossed products of $C^*$-algebras by discrete groups, or by groupoids endowed with a proper length, are examples of filtered $C^*$-algebras. The filtration encodes propagation of operators. An good analogy to understand propagation is the case of differential operators. A differential operator is said to be of bounded propagation if, applied to a function with bounded support, the output remains a function with bounded support.\\

In \cite{OY2} and \cite{OY3}, a controlled $K$-theory was developed in this general setting. It is a refinement of operator $K$-theory, which keeps track of the filtration by replacing the $K$-groups $K_*(A)$ of a $C^*$-algebra $A$ by a family of groups $\hat K_*(A)= \{K_*^{\varepsilon,R}(A)\}_{\varepsilon\in (0\frac{1}{4}),R>0}$. These controlled groups are naturally related to the usual $K$-theory, and are compatible when one increases the propagation. A new parameter appears, $\varepsilon\in (0 , \frac{1}{4})$, which is a technical adjustment to make sure controlled $K$-theory are genuine groups. The bound $\frac{1}{4}$ is not crucial, it is the greater bound that allows one to get $K$-theory element out of any controlled $K$-theory class. One could in principle take any lesser bound.\\ 

%%%%% PLAN DE LA THESE %%%%%

The idea behind this thesis is, first, to extend the setting of controlled $K$-theory, and then to define controlled assembly maps. We develop a formulation general enough to include the study of the Roe algebra of coarse spaces and crossed products of $C^*$-algebras by étale groupoids and discrete quantum groups. Let us detail the plan.\\ 

The first part is devoted to remind the reader of the definitions and results we will need. After a quick introduction to $C^*$-algebras and to Hilbert modules, the reader is guided through étale groupoids, coarse spaces and their relations. We give details on an important construction of \cite{SkTuYu}, the (étale) coarse groupoid $G(X)$ associated to a coarse space $X$. We also give a detailed proof of the $*$-isomorphism between the Roe algebra of $X$ with coefficients in a $C^*$-algebra $B$ and the reduced crossed product of $l^\infty (X,B\otimes\mathfrak K)$ with the natural action of $G(X)$. We end this section with a quick reminder of the construction of crossed products in the setting of discrete quantum groups.\\
 
In the second part, the first chapter begins with a definition of a filtered $C^*$-algebra with respect to what we call a coarse structure. An exposition of the controlled $K$-theory in this setting follows. All the results obtained by H. Oyono-Oyono and G. Yu in \cite{OY2} and \cite{OY3} are true in this setting. The interesting part is the definition of a coarse structure and the examples which cannot fit into the former formalism adopted in \cite{OY2}. These examples are given by the coarse structure $\mathcal E_X$ generated by symmetric entourages of a coarse space $X$, the coarse structure $\mathcal E_G$ generated by symmetric compact subsets of an étale groupoid $G$, and the coarse structure $\mathcal E_{\mathbb G}$ generated by symmetric unitary finite dimensional representations of a discrete quantum group $\mathbb G$. We show that these structures define filtration on the Roe algebras of $X$, and crossed products of $C^*$-algebras endowed with an action of $G$ and $\hat{\mathbb G}$ respectively.\\

The next chapter is divided into three parts. The first contains definitions of the usual assembly maps for étale groupoids and coarse spaces. It is followed by two very similar parts, so as to underline the similarity between the two situations, in which the controlled assembly maps are built. The steps are the following. First, define a descent transformation, called the controlled Roe transformation in the coarse setting, and the controlled Kasparov transformation in the groupoid setting. These are transformations that map $KK$-elements from $A$ to $B$ to controlled morphism from $\hat K(A)$ to $\hat K(B)$. We always begin with odd $KK$-elements, and realize them as boundaries of a well chosen semi-split extension of $C^*$-algebras. Taking the reduced crossed product (or the Roe algebra, depending on the setting) preserves these properties, and the controlled transformation is defined as the boundary of the resulting extension composed by some invertible controlled transformation. The odd case is obtaine using Bott periodicity. We then study properties of the descent transformation, the most crucial being that it is natural and respects the Kasparov product. The controlled assembly maps are defined from canonical projections $P_E$ and $\mathcal L_E$. At the end of the third part, we states what we called quantitative statements, relating the behaviour of usual assembly maps to controlled assembly maps. Doing so requires some interesting results on the $K$-homology groups with respect to infinite products of stable $C^*$-algebras, i.e. the content of the lemma \ref{prod}. We end the chapter by stating that the descent transformation can be defined in a similar way for quantum groups and give an application to $K$-amenability for discrete quantum groups.\\

The following chapter gives two applications of the controlled assembly maps. The first application deals with a controlled version of the main result of \cite{SkTuYu} : the coarse Baum-Connes conjecture for $X$ with coefficients in $B$ and the Baum-Connes conjecture for $G(X)$ with coefficients in $l^\infty (X, B\otimes\mathfrak K)$ are equivalent. We are able to prove in theorem \ref{BCCeq} that the two corresponding controlled assembly maps are equivalent. This result induces the classical one, and gives another proof. We apply the equivalence between the two controlled assembly maps to obtain some result about spaces which admit fibred coarse embedding into Hilbert space. The following application deals with a controlled Künneth formula for étale groupoids which satisfy a certain property. We call this class of groupoids the class $\mathcal C$, and prove that ample groupoids are in this class. This class seems worth studying. Notably, induction and restriction are interesting to study. \\  












































