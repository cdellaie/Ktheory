\chapter{Conclusion and Perspectives}

This thesis contains a generalization of controlled $K$-theory and constructions of controlled assembly maps, and some applications to computability in $K$-theory. We give now some natural questions that arise for future research.\\ 

One should look at the real advantage of controlled $K$-theory and controlled assembly maps : stability. Indeed, one should expect the controlled assembly map to respect weak notions of decomposition. The original goal of H. Oyono-Oyono and G. Yu was to extend the proof of the coarse Baum-Connes conjecture in the finite asymptotic dimension case to the finite decomposition complexity setting. We suggest the following program.\\
 
%First, give an analog of finite decomposition complexity for groupoids which gives six terms exact sequence in controlled $K$-theory. Indeed, a definition for asymtptotic dimension has been given by E. Guentner, R. Willett and G. Yu in \cite{GWY}.\\ % Finite decomposition complexity coincides in the coarse setting with an inductive definition of asymptotic dimension. \\ %This should provide an algorithm to compute the $K$-theory of crossed products by the groupoids. \\

In a first step, study controlled Mayer-Vietoris decompositions associated to étale groupoids. The idea is to consider a decomposition of the base space into open subsets, which are not supposed to be invariant. This leads to a Mayer-Vietoris exact sequence in $K$-homology for the groupoid, but also to a controlled Mayer-Vietoris exact sequence in controlled $K$-theory. \\

Then, one could develop a theory of decomposition complexity for étale groupoids, and, if properly defined, the technique using Mayer-Vietoris decomposition should really work for this notion. More precisely, one should be able to prove that groupoids with finite decomposition complexity satisfy the Baum-Connes conjecture. The work would include showing explicit examples of groupoids which are not of finite dynamic asymptotic dimension, but are of finite decomposition complexity. The point of using both this kind of decomposition and controlled $K$-theory is that it is more algebraic in nature and could be adapted to other assembly conjectures, such as the Farell-Jones conjecture or the algebraic Novikov conjecture. \cite{RTY} \\

This leads to stability results of the controlled Baum-Connes Conjecture, which could be used to show the Baum-Connes conjecture for some kind of groupoids. The highlight of this method is that a proof actually gives you an algorithm to compute the $K$-theory of the crossed-product $C^*$-algebra of the groupoid, which is far to be obvious in the usual cases, e.g. for a-T-menable groupoids. %For an example, take the topological groupoid associated to a manifold with corners, which is amenable, hence satisfies the Baum-Connes conjecture with coefficients, but for which only some examples of actual computations are known. 
This could be applied to a new proof of the Baum-Connes conjecture for étale groupoids with finite dynamic asymptotic dimension as defined by E. Guentner, R. Willett, and G. Yu \cite{GWY}. This last step should be the ultimate generalization of the "controlled cutting and pasting" of G. Yu's proof \cite{Yu1} in the setting of étale groupoids.\\

The key notion for these techniques was introduced by H. Oyono-Oyono and G. Yu in \cite{OY4} : Mayer-Vietoris pairs in a filtered $C^*$-algebra. Let us give more details. \\

We introduce the notation for a controlled neighborhood as follows. Let $(A,\mathcal E)$ be a filtered $C^*$-algebra, $E\in\mathcal E$ and $\Delta$ be a closed linear subspace of $A_E$ stable by involution. Then the $E$-controlled neighborhood of $\Delta$, denoted by $C^* N_{\Delta}^{(E,E')}$, is defined as the $C^*$-algebra generated by $N_{\Delta}^{(E,E')} = \Delta + A_{E'}.\Delta +\Delta. A_{E'} + A_{E'}.\Delta.A_{E'} $ for every $E'\in\mathcal E$.

\begin{definition} \label{MVpair}
Let $(A,\mathcal E)$ be a filtered $C^*$-algebra and $E \in\mathcal E$. A $E$-controlled weak Mayer-Vietoris pair for $A$ is a quadruple $(\Delta_0,\Delta_1,A_0,A_1)$ such that for some constant $c>0$:\\

\begin{itemize}
\item[$\bullet$] $\Delta_j$ are closed linear subspaces of $A$, 
\item[$\bullet$] $A_j$ are sub-$C^*$-algebras of $A$ such that $A_j$ is $\mathcal E$-filtered by $(A_{E'}\cap A_j)_{E'\in\mathcal E}$ and $C^* N_{\Delta_j}^{(E,5.E)}\subseteq A_j$,
\item[$\bullet$] if $x\in M_n(A_{E'})$ for $E'\in\mathcal E$ such that $E'\subseteq E$ and any $n>0$ , there exists $x_j\in M_n(\Delta_j\cap A_{E'})$ such that $x=x_0+x_1$ and $||x_j||\leq c||x||$,
%\item[$\bullet$] $A_j$ is filtered by $(A_j\cap A_{E'})_{E'\in\mathcal E}$ and $C^* N_{\Delta_j}^{(E,5.E)}\subseteq A_j$
\item[$\bullet$] for every $\varepsilon >0$, if $x\in M_n(A_{0,E'})$, $y\in M_n(A_{1,E'})$ such that $||x-y||<\varepsilon$,  there exists $z\in M_n(A_{0,E'}\cap A_{1,E'})$ with $||z-x||<\varepsilon$ and $||z-y||<\varepsilon$. \\
\end{itemize}
\end{definition}

\begin{Expl} Let $G$ be an étale groupoid with compact base space and let $V\subseteq G^{(0)}$ be an open subset. For $E\in\mathcal E$, put :
\begin{itemize}
\item[$\bullet$] $V^E= \{r(g) : g\in G_{V}\cap E\}$,
\item[$\bullet$] $\Delta_V = C_0(G_{V}\cap E)$,
\item[$\bullet$] $G^{V,(E)}_V$ the subgroupoid generated by $G_V^V \cap E$.
\end{itemize} 
Let $V_0$ and $V_1$ be open subsets such that $G^{(0)} = V_0\ \cup \ V_1$. Denote $G_{V_j^E}^{V_j^E , (E)}$ by $G_j$. Then $(\Delta_{V_0},\Delta_{V_1}, C_r^*(G_0),C_r^*(G_1))$ is a $E$-controlled Mayer-Vietoris pair for $C^*_r (G)$. 
\end{Expl}

For example, take $G$ to be an étale groupoïd with proper length $l$, with compact base space $X$. The convolution algebra $C^*_r G$ is filtered by $(C_c(G_R))_{R>0}$, $G_R=l^{-1}[0,R)$. Fix $R>5r$. If $V$ is an open subset of $X$, set \\

\begin{itemize}
\item[$\bullet$] $V^R= \{r(g) : g\in G_V\cap G_R\}$,
\item[$\bullet$] $\Delta_V = C_0(G_V\cap G_R)$
\item[$\bullet$] $G_{V,R}^V = \langle G_V^V \cap G_R \rangle$\\
\end{itemize} 

then $(\Delta_{V_0},\Delta_{V_1}, C_r^*(G_0),C_r^*(G_1))$ is a $r$-controlled Mayer-Vietoris pair for $C^*_r G$ when $X=V_0\cup V_1$, and $G_j = G_{V_j^R}^{V_j^R , (R)}$.\\

The existence of a controlled Mayer-Vietoris pair is nice, because even if the $C^*$-algebra is simple, it can possess such a decomposition, and the following result gives a way to compute the $K$-theory analogous to the situation of a classical Mayer-Vietoris decomposition :

\begin{thm}
For every positive $c$, there exists a control pair $\rho_{MV}$ such that for any $\mathcal E$-filtered $C^*$-algebra $A$ and any $E\in\mathcal E$, if there exists a $E$-controlled Mayer-Vietoris pair $(\Delta_0,\Delta_1,A_0,A_1)$, then there exists a controlled morphism $D : \hat K_*(A)\rightarrow \hat K(A_0\cap A_1)$ such that the following sequence 
\[\begin{tikzcd}
\hat K_0(A_0\cap A_1 ) \arrow{r}& \hat K_0(A_0)\arrow{r}\oplus \hat K_0(A_1) & \hat K_0(A)\arrow{d}{D} \\
\arrow{u}{D}\hat K_1(A) &\arrow{l} \hat K_1(A_0)\oplus \hat K_1(A_1) & \arrow{l}\hat K_1(A_0\cap A_1) 
\end{tikzcd}\]
is $\rho$-exact at order $E$.
%where $D$ is a controlled Mayer-Vietoris boundary.\\
\end{thm}

To go back to our example, if $G^{(0)} = V_0\ \cup \ V_1 $, and with the same notations, we simulteanously have two sequences : \\
\begin{itemize}
\item[$\bullet$] the exact sequence in $K$-homology arising from the decomposition of the Rips complex into two open sets 
\[P_E(G) = P_E(G^{V_0^E,(E)}_{V_0^E}) \cup P_E(G^{V_1^E,(E)}_{V_1^E}) ,\] 
\item[$\bullet$] the controlled exact sequence in controlled $K$-theory arising from the controlled Mayer Vietoris pair $(\Delta_{V_0},\Delta_{V_1}, C_r^*(G_0),C_r^*(G_1))$. \\
\end{itemize}
The next step is to show that the controlled assembly map intertwines these two sequences. \\

%On another level, one could develop the range of applications of controlled $K$-theory. Indeed, the formalism is easily generalized to more abstract situations. This could in principle be used to give a filtration on other $C^*$-algebras coming from general topological groupoids, locally compact quantum groups or reduced $C^*$-algebras twisted by finite dimensional representations, which are Banach algebras. Another very promising application of controlled $K$-theory is classification of $C^*$-algebras. \\

As a conclusion, a lot of questions stem from the use of controlled $K$-theory for the study of groupoids. Controlled $K$-theory for filtered Banach algebras, developed in \cite{Chung2} by Yeong Chyuan Chung, would also hopefully lead to interesting ideas. One can off course always ask why the $K$-theory of the reduced $C^*$-algebra is chosen to be the range of the assembly map. Some situations require to consider other Banach algebras, possibly lacking the $C^*$-property. In this area, $L^p$-version of the Baum-Connes conjecture are heavily studied, and a controlled $K$-theory could be of strong interest, as shown by the work of Yeong Chyuan Chung in \cite{Chung1}.\\



















 

      
