\chapter{Conclusion and Perspectives}

This thesis contains constructions of controlled assembly maps, and some applications. We give now some natural questions that arise for future research.\\ 

First one should look at the real advantage of controlled $K$-theory and controlled assembly maps : stability. Indeed, one should expect the controlled assembly map to respect weak notions of decomposition. The orignal goal of H. Oyono-Oyono and G. Yu was to extend the proof of the coarse Baum-Connes conjecture in the finite asymptotic dimension case to the finite decomposition complexity setting. We suggest the following program : first, give a analog of finite decomposition complexity for groupoids which gives six terms exact sequence in controlled $K$-theory. This should be the ultimate generalization of the "controlled cutting and pasting" of G. Yu's proof in \cite{Yu1}.\\

The first step should be to study controlled Mayer-Vietoris decompositions associated to étale groupoids. The idea is to consider a decomposition of the base space into open subsets, which are not supposed to be invariant. This leads to a Mayer-Vietoris exact sequence in $K$-homology for the groupoid, but also to a controlled Mayer-Vietoris exact sequence in controlled $K$-theory. This leads to stability results of the controlled Baum-Connes Conjecture, which could be used to show the Baum-Connes conjecture for some kind of groupoids. The highlight of this method is that a proof actually gives you an algorithm to compute the $K$-theory of the crossed-product $C^*$-algebra of the groupoid, which is far to be obvious in the usual cases, e.g. for a-T-menable groupoids. %For an example, take the topological groupoid associated to a manifold with corners, which is amenable, hence satisfies the Baum-Connes conjecture with coefficients, but for which only some examples of actual computations are known. 
This could be applied to a new proof of the Baum-Connes conjecture for étale groupoids with finite dynamic asymptotic dimension as defined by E. Guentner, R. Willett, and G. Yu \cite{GWY}.\\

Then, one could develop a theory of decomposition complexity for étale groupoids, and, if properly defined, the technique using Mayer-Vietoris decomposition should really work for this notion. More precisely, one should be able to prove that groupoids with finite decomposition complexity satisfy the Baum-Connes conjecture. The work would include showing explicit examples of groupoids which are not of finite dynamic asymptotic dimension, but are of finite decomposition complexity. The point of using both this kind of decomposition and controlled $K$-theory is that it is more algebraic in nature and could be adapted to other assembly conjectures, such as the Farell-Jones conjecture or the algebraic Novikov conjecture. \cite{RTY} \\

On another level, one could develop the range of applications of controlled $K$-theory. Indeed, the formalism is easily generalized to more abstract situations. This could in principle be used to give a filtration on other $C^*$-algebras coming from general topological groupoids, quantum groups or reduced $C^*$-algebras twisted by finite dimensional representations, which are Banach algebras. Another very promising application of controlled $K$-theory is classification of $C^*$-algebras. \\

Controlled $K$-theory for filtered Banach algebras would also hopefully lead to interesting ideas. One can off course always ask why the $K$-theory of the reduced $C^*$-algebra is chosen to be the range of the assembly map. Some situations require to consider other Banach algebras, possibly lacking the $C^*$-property. In this area, $L^p$-version of the Baum-Connes conjecture are heavily studied, and a controlled $K$-theory could be of strong interest.\\

\section{Mayer-Vietoris sequence in $K$-homology}

We first recall how to construct a Mayer-Vietoris element out of any pull back diagram of $C^*$-algebras. We fix an étale $\sigma$-compact groupoid.\\

Let $X$ be a locally compact space and let us decompose $X$ into two open sets $U_0$ and $U_1$. Consider the $C^*$-algebras $A=C_0(X)$, $A_j=C_0(U_j)$ where $U_{01}=U_0\cap U_1$, and $C$ the pull-back $\{(f_0,f_1) \in CA_0\oplus CA_1 \text{ s.t. } f_0(0)=0,f_1(1)=0,\text{ and } f_0(\frac{1}{2})=f_1(\frac{1}{2})\in A_{01}\}$, with the canonical morphisms $\alpha :C \rightarrow A$ and $\beta : C \rightarrow SA_{01}$. The latter is an homotopy equivalence, and the Mayer-Vietoris boundary is defined as the dotted arrow in the following commutative diagram
\[\begin{tikzcd}
KK^G_*(C,B) \arrow{r}{\beta^*}\arrow{d}{ \alpha^*}& KK^G_*(SA_{01},B)\arrow{d}{[\partial_{A_{01}}]\otimes -}\\
KK^G_*(A,B) \arrow[dotted]{r}{ [\partial_{MV}] \otimes}& KK^G_{*+1}(A_{01},B)\\
\end{tikzcd}\]
where $[\partial_{MV}]=[\partial_{A_{01}}]\otimes_{SA_{01}} [\beta]^{-1}\otimes_C [\alpha] = [\partial_{A_{01}}]\otimes_{SA_{01}} \alpha^*([\beta]^{-1}])\in KK^G_1(A_{01},A)$ and where the right vertical arrow is the Bott element $[\partial_{A_{01}}]\in KK_1^G(A_{01},SA_{01})$.\\

In proving the compatibility of the controlled assembly map with the Mayer-Vietoris sequence, the following lemma shows how to compute the image of 
the boundary of a $K$-cycle in term of its image.\\

Let $f : X\rightarrow [0,1]$ be a continuous function which is identically equal to $0$ on $X\backslash U_1 $ and identically equal to $1$ on $X\backslash U_1$. Let $A$ be a $G$-algebra and $E\in\mathcal E$. For any $p\in P_n^{\varepsilon,E}(A\rtimes_r G) $, define 
\begin{itemize}
\item[$\bullet$] $U_{f,p}(x) = e^{2i\pi f(x)}p(x) +1_n -p(x)$ if $x\in U_0\cap U_1$
\item[$\bullet$] $U_{f,p}(x) = 1_n$ elsewhere.
\end{itemize}
so that $U_{f,p}$ defines an element of $ U^{\varepsilon,E}_n(A\rtimes_r G)$.\\

\begin{lem}
Let $p\in P_n^{\varepsilon,E}(A\rtimes_r G)$ and $U_{f,p}\in U_n^{\varepsilon,E}(A_{01}\rtimes_r G)$ as above. Then
\[J^{\varepsilon, E}_G([\partial_{MV}]\otimes z ) [p,0]_{\varepsilon, E} \sim J^{\varepsilon, E}_G(z)[U_{f,p}]_{\varepsilon, E}\]
for any $G$-algebras $B$ and $B'$ and every $z\in KK^G(B,B')$.
\end{lem}

\begin{dem}
From $f$, define the following function $F$ on $U_0\times[0,\frac{1}{2}]\cup U_1\times [\frac{1}{2},1]\subseteq X\times[0,1]$: 
\begin{itemize}
\item[$\bullet$] $F(x,t)= 2t f(x)$ if $t\in [0,\frac{1}{2}]$,
\item[$\bullet$] $F(x,t)= 2t (f(x)+t-tf(x))-1$ if $t\in [\frac{1}{2},1]$.
\end{itemize} 
This allows to define $W_{f,p}(x,t)=e^{2i\pi F(x,t)}p(x)+1-p(x)$ which is a lift of $U_{f,p}$ in $U_n^{\varepsilon,E}( C \rtimes_r G)$ via $\alpha$.
But $s\in [0,1]\mapsto e^{2i\pi (sF(x,t)+(1-s)t)}p(x)+1-p(x)$ yields a homotopy of almost unitaries between $W_{f,p}$ and $e^{2i\pi t}p + 1-p$, so that $(\beta_G)_*[W_{f,p}]_{\varepsilon,E}=D_{A_{01}\rtimes_r G}^{SA_{01}\rtimes_r G} [p,0]_{\varepsilon,E}$.\\

Applying proposition \ref{Kasparov} gives : 
\[J_G([\partial_{MV}]\otimes z])[p,0]_{\varepsilon,E} \sim J_G(z)\circ J_G([\alpha])\circ J_G([\beta]^{-1}) \circ D_{A_{01}\rtimes_r G}^{SA_{01}\rtimes_r G} [p,0]  \]
and the previous line of thought entails that the right member is equal to $J_G(z)\circ J_G([\alpha]) \circ [W_{f,p}]_{\varepsilon,E} = J_G(z)[U_{f,p}]_{\varepsilon,E} $ because $(\alpha_G)_* [W_{f,p}]_{\varepsilon,E}=[U_{f,p}]_{\varepsilon,E} $.\\
\qed
\end{dem}

%%%%%%
%%%%%%
%%%%%%
At the level of controlled $K$-theory, Oyono-Oyono and Yu introduced (unpublished) a notion of Mayer-Vietoris pair in a filtered $C^*$-algebra $A$ weaker than that of a pull-back diagram. We introduce the notation for a controlled neighborhood as follows. Let $(A,\mathcal E)$ be a filtered $C^*$-algebra, $E\in\mathcal E$ and $\Delta$ be a closed linear subspace of $A_E$ stable by involution. Then the $E$-controlled neighborhood of $\Delta$, denoted by $C^* N_{\Delta}^{(E,E')}$, is defined as the $C^*$-algebra generated by $N_{\Delta}^{(E,E')} = \Delta + A_{E'}.\Delta +\Delta. A_{E'} + A_{E'}.\Delta.A_{E'} $ for every $E'\in\mathcal E$.

\begin{definition} \label{MVpair}
Let $(A,\mathcal E)$ be a filtered $C^*$-algebra and $E \in\mathcal E$. A $E$-controlled weak Mayer-Vietoris pair for $A$ is a quadruple $(\Delta_0,\Delta_1,A_0,A_1)$ such that for some constant $c>0$:\\

\begin{itemize}
\item[$\bullet$] $\Delta_j$ are closed linear subspaces of $A$, 
\item[$\bullet$] $A_j$ are sub-$C^*$-algebras of $A$ such that $A_j$ is $\mathcal E$-filtered by $(A_{E'}\cap A_j)_{E'\in\mathcal E}$ and $C^* N_{\Delta_j}^{(E,5.E)}\subseteq A_j$,
\item[$\bullet$] if $x\in M_n(A_{E'})$ for $E'\in\mathcal E$ such that $E'\subseteq E$ and any $n>0$ , there exists $x_j\in M_n(\Delta_j\cap A_{E'})$ such that $x=x_0+x_1$ and $||x_j||\leq c||x||$,
%\item[$\bullet$] $A_j$ is filtered by $(A_j\cap A_{E'})_{E'\in\mathcal E}$ and $C^* N_{\Delta_j}^{(E,5.E)}\subseteq A_j$
\item[$\bullet$] for every $\varepsilon >0$, if $x\in M_n(A_{0,E'})$, $y\in M_n(A_{1,E'})$ such that $||x-y||<\varepsilon$,  there exists $z\in M_n(A_{0,E'}\cap A_{1,E'})$ with $||z-x||<\varepsilon$ and $||z-y||<\varepsilon$. \\
\end{itemize}
\end{definition}

\begin{Expl} Let $G$ be an étale groupoid with compact base space and let $V\subseteq G^{(0)}$ be an open subset. For $E\in\mathcal E$, put :
\begin{itemize}
\item[$\bullet$] $V^E= \{r(g) : g\in G_{V}\cap E\}$,
\item[$\bullet$] $\Delta_V = C_0(G_{V}\cap E)$,
\item[$\bullet$] $G^{V,(E)}_V$ the subgroupoid generated by $G_V^V \cap E$.
\end{itemize} 
Let $V_0$ and $V_1$ be open subsets such that $G^{(0)} = V_0\ \cup \ V_1$. Denote $G_{V_j^E}^{V_j^E , (E)}$ by $G_j$. Then $(\Delta_{V_0},\Delta_{V_1}, C_r^*(G_0),C_r^*(G_1))$ is a $E$-controlled Mayer-Vietoris pair for $C^*_r (G)$. 
\end{Expl}

For example, take $G$ to be an étale groupoïd with proper length $l$, with compact base space $X$. The convolution algebra $C^*_r G$ is filtered by $(C_c(G_R))_{R>0}$, $G_R=l^{-1}[0,R)$. Fix $R>5r$. If $V$ is an open subset of $X$, set \\

\begin{itemize}
\item[$\bullet$] $V^R= \{r(g) : g\in G_V\cap G_R\}$,
\item[$\bullet$] $\Delta_V = C_0(G_V\cap G_R)$
\item[$\bullet$] $G_{V,R}^V = \langle G_V^V \cap G_R \rangle$\\
\end{itemize} 

then $(\Delta_{V_0},\Delta_{V_1}, C_r^*(G_0),C_r^*(G_1))$ is a $r$-controlled Mayer-Vietoris pair for $C^*_r G$ when $X=V_0\cup V_1$, and $G_j = G_{V_j^R}^{V_j^R , (R)}$.\\

The existence of a controlled Mayer-Vietoris pair is nice, because even if the $C^*$-algebra is simple, it can possess such a decomposition, and the following result gives a way to compute the $K$-theory analogous to the situation of a classical Mayer-Vietoris decomposition :

\begin{thm}
For every positive $c$, there exists a control pair $\rho_{MV}$ such that for any $\mathcal E$-filtered $C^*$-algebra $A$ and any $E\in\mathcal E$, if there exists a $E$-controlled Mayer-Vietoris pair $(\Delta_0,\Delta_1,A_0,A_1)$, then there exists a controlled morphism $D : \hat K_*(A)\rightarrow \hat K(A_0\cap A_1)$ such that the following sequence 
\[\begin{tikzcd}
\hat K_0(A_0\cap A_1 ) \arrow{r}& \hat K_0(A_0)\arrow{r}\oplus \hat K_0(A_1) & \hat K_0(A)\arrow{d}{D} \\
\arrow{u}{D}\hat K_1(A) &\arrow{l} \hat K_1(A_0)\oplus \hat K_1(A_1) & \arrow{l}\hat K_1(A_0\cap A_1) 
\end{tikzcd}\]
is $\rho$-exact at order $E$.
%where $D$ is a controlled Mayer-Vietoris boundary.\\
\end{thm}

To go back to our example, if $G^{(0)} = V_0\ \cup \ V_1 $, and with the same notations, we simulteanously have two sequences : \\
\begin{itemize}
\item[$\bullet$] the exact sequence in $K$-homology arising from the decomposition of the Rips complex into two open sets 
\[P_E(G) = P_E(G^{V_0^E,(E)}_{V_0^E}) \cup P_E(G^{V_1^E,(E)}_{V_1^E}) ,\] 
\item[$\bullet$] the controlled exact sequence in controlled $K$-theory arising from the controlled Mayer Vietoris pair $(\Delta_{V_0},\Delta_{V_1}, C_r^*(G_0),C_r^*(G_1))$. \\
\end{itemize}
The next step is to show that the controlled assembly map intertwines these two sequences. %The aim of this section is to show that the quantitative assembly maps respects these two exact sequences in a precise way.\\

%Actually, in this particular case, the quantitative Mayer-Vietoris exact sequence should hold at all orders, which would allows us to state a much stronger result, with more interesting applications : a Künneth formula for crossed product algebras of étale groupoids.\\

%\textbf{A remark :} There is a seemingly harmful parallel between the controlled Mayer-Vietoris decomposition and the nuclear dimension of the reduced $C^*$-algebra. Namely, to show that the pair satisfies the Mayer-Vietoris conditions, we make use of the completely positive map induced by 
%\[\left\{\begin{array}{lcr} C_c(G) &\rightarrow & C_c(G) \\ f &\mapsto & \phi_0\circ r\ . f .\ \phi_0\circ s\end{array}\right.\]   
%where $\phi_0$ is any continuous function $X\rightarrow [0,1]$ with support in $V_0$ which is $1$ on some compact $K\subseteq V_0$.\\
%Could we push the analogy further ? 

%\section{Standard modules}

The aim of this section is to develop a notion of non-degenerate standard modules over a groupoid analogous to non-degenerate standard modules over coarse spaces. \\

Let us first recall the coarse case. Let $X$ be a discrete metric space with bounded geometry. 

\begin{definition}
A $X$-module is a Hilbert space $H_X$ equipped with a $*$-representation $\phi : C_0(X) \rightarrow \mathcal L(H_X)$. The $X$-module $(H_X,\phi)$ is said to be :\\

\begin{itemize}
\item[$\bullet$] standard if $\phi(C_0(X))H_X$ is dense in $H_X$,
\item[$\bullet$] non-degenerate if $\forall f \in C_0(X), \phi(f)\in \mathfrak K (H_X) \implies f=0$.
\end{itemize}
\end{definition}

The usefulness of n.d.s. $X$-modules comes from the following lemma :

\begin{lem}
Let $X$ and $Y$ be two discrete metric spaces with bounded geometry, and $h : X\rightarrow Y$ a coarse map. Then, for any two standard modules $H_X$ and $H_Y$ over $X$ and $Y$ respectively, there exists an isometry which covers $h$, i.e. for any $\epsilon >0$, there exists $V\in \mathcal L(H_X,H_Y)$ such that 
\[\text{supp }V \subseteq \{(x,y)\in X\times Y, d(h(x),y)<\epsilon\}.\]
\end{lem}

\textbf{Remark :} We can induce $V$ on the Roe algebras by $\forall T\in C^*(X,H_X), Ad_V(T) := V T V^* \in C^*(Y,H_Y)$, and the preceding lemma entails that \[(Ad_V)_* : K(C^*(X,H_X))\rightarrow K(C^*(Y,H_Y))\] only depends on the coarse class of $h$. We directly see that the $K$-theory of the Roe-algebras of $X$ do not depend on the standard modules if they are n.d.s. : one just need to take an isometry covering the identity. \\

We can actually show that taking the Roe algebra is a functor. Choose, for any coarse space $X$ a n.d.s. $X$-module $H_X$, and consider the category $Coarse$ of coarse spaces with morphisms coarse maps, and the category $KK$ with objects $C^*$-algebras, and morphisms defined by $KK$-theory, $Hom_{KK}(A,B)=KK(A,B)$. Then $ X \mapsto C^*(X,H_X) $ and $\left(h:X \rightarrow Y\right) \mapsto (Ad_V)_*\in KK(C^*(X,H_X),C^*(Y,H_Y))$ defines a functor $Coarse \rightarrow KK$, which does not depend on the choices $X\mapsto H_X$ being made.\\

We will focus ont the following result :

\begin{thm}
Let $X$ be a finite simplicial complex, and $B$ a $C^*$-algebra. There exists $\varepsilon_0\in (0,\frac{1}{4})$ such that for all $\varepsilon \in (0,\varepsilon_0)$, there exists $R_\varepsilon>0 $ s.t. 
\[\forall R\in (0,R_\varepsilon),\  Ind_{X,B}^{\varepsilon, R} : KK(C(X),B)\rightarrow K^{\varepsilon,R}(B\otimes \mathfrak K(H_X))\] 
is an isomorphism.
\end{thm}

We wish to prove a similar result for "nice" groupoids. Let us recall the definition of $G$-simplicial complex from \cite{TuBC2}.

\begin{definition}
Let $G$ be a locally compact groupoid. A $G$-simplicial complex of dimension less than $n$ is a triple $(Z,\Delta, p)$ where
\begin{itemize}
\item[$\bullet$] $Z$ is a locally compact space of vertices and $p: Z \rightarrow G^{(0)}$ a locally injective map, which is an anchor map for an action of $G$, 
\item[$\bullet$] $\Delta$ is a closed $G$-invariant subset of the probability measures $P(Z)$ on $Z$, equipped with the weak-$*$ topology, with the property that every element of $\Delta$ has a support contained in a fiber of $p$ and has at most $n+1$ elements. Such a support is call a simplex of $\Delta$. Moreover, if $\nu\in \Delta,\mu \in P(Z)$ and $supp(\mu)\subseteq supp(\nu)$ then $\mu\in \Delta$.
\end{itemize}
\end{definition}

The first step is to defined what is a s.n.d. $G$-module.

\begin{definition}
A s.n.d. $G$-module is a triple $(E,\phi,V)$ where $E$ is a $C_0(G^{(0)})$-algebra, $\phi : C_0(G^{(0)})\rightarrow \mathcal L(E)$ is a $*$-representation and $V$ is an action of $G$ on $E$ such that
\begin{itemize}
\item[$\bullet$] $E$ decomposes as a external tensor product $E\simeq E^{(0)}\otimes E^{(1)}$ of $G$-modules such that there exists $\phi : C_0(G^{(0)})\rightarrow \mathcal L(E^{(0)})$ and $\phi= \phi_0\otimes id_{E^{(1)}}$
\item[$\bullet$] No non-zero function of $C_0(G^{(0)})$ acts as a compact operator via $\phi_0$, and $\overline{\phi_0(C_0(G^{(0)}))E^{(0)} }= E^{(0)}$
\item[$\bullet$] For all compact subgroupoids $K$ of $G$, there exists an isomorphism of $K$-modules $\text{Res}_K^G E^{(1)} \simeq L^2(K)\otimes H_K$ where $H_K$ is a separable Hilbert space.
\end{itemize}
\end{definition}

Now we want to show the following :

\begin{lem}
Let $G$ be a locally compactly induced groupoid, and $E,E'$ any two s.n.d. $G$-modules, then for any compact subset $K$ of $G$, there exists an isometry $V\in \mathcal L(E,E')$ such that 
\[supp\ V \subseteq (s\times r)(K). \]
\end{lem}

\begin{dem}
Let $E$ a n.d.s. $G$-module, and decompose the base space $G^{(0)}$ into $G$-invariant subset which are locally induced by compact subgroupoids of $G$ :
\[G^{(0)} = \cup_{j=1}^J G\times_{K_j} U_j.\]
By the third hypothesis of being s.n.d., there exist a Hilbert space $H_j$ and an isomorphism of Hilbert modules $F_j :\text{Res}_{K_j}^G E^{(1)} \rightarrow  L^2(K_j)\otimes H_j$, but the induction $\text{Ind}_{K_j}^G L^2(G/K_j)\otimes L^2(K_j)\otimes H_j$ is non canonically isomorphic to $L^2(G)\otimes H_j$, so that $E$ is isomorphic to $E^{(0)}\otimes L^2(G)\otimes H_j$.\\
\qed
\end{dem}

\begin{cor}
Let $G$ and $G'$ two locally compactly induced groupoids, $E,E'$ two s.n.d. modules over $G$ and $G'$ respectively and $(Z,p,p')$ a generalized morphism from $G$ to $G'$ which respects condition ??. For any compact subset $K$, there exists an isometry $V\in \mathcal L(E,E')$ such that
\[supp\ V \subseteq (p\circ (p')^{-1}(s(K))\times r)(K). \]
\end{cor}

























 

      
