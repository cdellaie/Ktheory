\section{Controlled $K$-theory for étale groupoids}

Let $G$ be an étale groupoid and let $\mathcal E$ be  subset of the set of its compact subsets such that :
\begin{itemize}
\item[$\bullet$] if $E\in \mathcal E$, then $E^{-1}\in\mathcal E$,
\item[$\bullet$] if $E\in \mathcal E$ and $E'\in \mathcal E$, then $E\circ E'\in\mathcal E$.
\end{itemize}

For simplicity, we will take $\mathcal E = \{E\subset G \text{ compact s.t. } E=E^{-1}\}$.\\

In this section, we define controlled $K$-theory groups for $G$. By analogy with the terminology used in Coarse Geometry, we will call $E\in \mathcal{E}$ a controlled subset. Notice that if $E,E'\in \mathcal E$, then $E E'\in\mathcal E$, where $EE' := \{ gg' , g\in E,g'\in E'\}$.\\

\begin{definition}
A $C^*$-algebra $A$ is said to be filtered if, for every $E\in \mathcal E$, there exists a linear subspace $A_E$ of $A$ such that :\\
\begin{itemize}
\item[$\bullet$] if $E \subset E'$, then $A_E\subset A_{E'}$, and the family of inclusions $\phi_E^{E'}: A_E\hookrightarrow A_{E'}$ induces an inductive system of linear spaces,
\item[$\bullet$] $A_E$ is stable by involution,
\item[$\bullet$] for all $E,E'\in\mathcal E$, $A_E.A_{E'}\subset A_{EE'}$,
\item[$\bullet$] the union of subspaces is dense in $A$, i.e. $\overline{\cup_{E\in\mathcal E}A_E} = \varinjlim A_E = A$.
\item[$\bullet$] if $A$ is unital, we impose that $1\in A_E,\forall E\in\mathcal E$.\\
\end{itemize}
\end{definition}

If $A$ is a non-unital filtered $C^*$-algebra, we will by default endowed its unitalization $\tilde A$ with the filtration $\tilde A_E = A_E + \C$. A $*$-homomorphism $\phi : A \rightarrow B$ is said to be filtered if $\phi(A_E)\subset B_E$ for all $E\in\mathcal E$.\\

The crucial example for us will be crossed products of $G$-algebras by $G$, but this definition generalizes that of \cite{OY2}. Indeed, as will be recalled later, the Roe algebras can be expressed as a crossed product by a so-called coarse groupoid, which is étale \cite{SkTuYu}, and the definition given here, applied to this groupoid, gives back the filtration in the sense of \cite{OY2}.

\subsection{Almost unitaries and almost projections}

\begin{definition}
Let $A$ be a unital filtered $C^*$-algebra. Let $\varepsilon\in(0,\frac{1}{4})$ and $E\in \mathcal E$ a controlled subset. The set of $\varepsilon$-$E$-unitaries is the set 
\[U^{\varepsilon, E}(A)= \{u\in A_E \text{ s.t. } ||u^*u-1||<\varepsilon\text{ and }||uu^*-1||<\varepsilon \}\]
and the set $\varepsilon$-$E$-projections is the set 
\[P^{\varepsilon, E}(A)= \{p\in A_E \text{ s.t. } p=p^*\text{ and }||p^2-p||<\varepsilon \}.\]
We will use the notation $P_n^{\varepsilon, E}(A)$ for $P^{\varepsilon, E}(M_n(A))$, and $U_n^{\varepsilon, E}(A)$ for $U^{\varepsilon, E}(M_n(A))$. Also, $P_\infty^{\varepsilon, E}(A)$ is the algebraic inductive limit of the $P_n^{\varepsilon, E}(A)$ under the natural inclusions
\[\left\{\begin{array}{rcl}
	P^{\varepsilon,E}_n(A) 		& \rightarrow	& P^{\varepsilon,E}_{n+1}(A)\\ 
	p 		& \mapsto 	& \begin{pmatrix}p& 0 \\ 0&0 \end{pmatrix}
\end{array}\right.\]
and $U_\infty^{\varepsilon, E}(A)$ is the algebraic inductive limit of the $U_n^{\varepsilon, E}(A)$ under the natural inclusions
\[\left\{\begin{array}{rcl}
	U^{\varepsilon,E}_n(A) 		& \rightarrow	& U^{\varepsilon,E}_{n+1}(A)\\ 
	u 		& \mapsto 	& \begin{pmatrix}u & 0 \\ 0& 1 \end{pmatrix}
\end{array}\right. .\]
\end{definition}

\begin{rk}Let $\varepsilon\in (0,\frac{1}{4})$ and $E\in\mathcal E$.\\
\begin{itemize}
\item[$\bullet$] If $p\in P^{\varepsilon,E}(A)$, then $p$ has a spectral gap around $\frac{1}{2}$, and functional calculus allows to define a genuine projection $\kappa_0(p)$, as in \cite{OY2}, by taking $\kappa_0$ to be a continuous function that vanishes inside the spectral gap and that is respectively $0$ and $1$ on the left and right part of the spectrum of $p$. As this projection does not depend on $\kappa_0$, we will always denote it the same, even if the function depends on $p$.\\
\item[$\bullet$] If $u\in U^{\varepsilon,E}(A)$, then $u^* u$ is invertible, and $u(u^* u)^{-1}$ defines a unitary, that we will denote $\kappa_1(u)$.
\end{itemize}
\end{rk}

In order to define controlled $K$-groups, define the following equivalence relations on $P^{\varepsilon, E}_\infty(A)\times \N$ and $U^{\varepsilon,E}_n(A)$.\\

\begin{itemize}

\item[$\bullet$] $(p,l) \sim (q,l')$ if there exists a homotopy of almost projections $h\in P^{\varepsilon, E}_\infty(A[0,1])$ and an integer $k$ such that 
\[h(0)=\begin{pmatrix} p & 0 \\ 0 & 1_{k+l'} \end{pmatrix} \text{ and }
h(1)=\begin{pmatrix} q & 0 \\ 0 & 1_{k+l} \end{pmatrix}\]
\item[$\bullet$] $u \sim v$ if there exists a homotopy of almost unitaries $h\in U^{3\varepsilon, E\circ E}_\infty(A[0,1])$ such that $h(0)= u \text{ and }h(1)=v$.\\
\end{itemize}

Denote $[(p,l)]_{\varepsilon,E}$ and $[u]_{\varepsilon,E}$ for the equivalence classes of almost-projections and almost-unitaries. Then, the same proof as \cite{OY2} shows that $[p,l]_{\varepsilon,E}+[q,l']_{\varepsilon,E}=[diag(p,q),l+l']_{\varepsilon,E}$ and $[u]_{\varepsilon,E}+[v]_{\varepsilon,E}=[diag(u,v)]_{\varepsilon,E}$ induces a group structure on the equivalence classes, that we denote $K_0^{\varepsilon,E}(A) = P^{\varepsilon, E}_\infty(A)\times \N / \sim$ and $K_1^{\varepsilon,E}(A) = U^{\varepsilon, E}_\infty(A) / \sim$.\\

If $A$ is not unital, $K_0^{\varepsilon,E}(A)$ is defined as
\[\{[p,l]_{\varepsilon,E} : p\in P^{\varepsilon,E}_\infty (\tilde A), l\in \N \text{ s.t. rank}(\kappa_0(\rho_A(p)))=l \}\]
and $K_1^{\varepsilon,E}(A)$ is defined as $U_\infty^{\varepsilon,E}(\tilde A)/ \sim_{\varepsilon,E}$.\\

\begin{definition}
The controlled $K$-theory of a filtered $C^*$-algebra $A$ is the family of abelian groups $\hat K_0(A) = (K_0^{\varepsilon,E}(A))_{\varepsilon\in (0,\frac{1}{4}),E\in\mathcal E}$ and $\hat K_1(A) = (K_1^{\varepsilon,E}(A))_{\varepsilon\in (0,\frac{1}{4}),E\in\mathcal E}$ defined above.\\
\end{definition}

We define canonical morphisms : if $\varepsilon, \varepsilon'\in (0,\frac{1}{4})$ and $E,E'\in\mathcal E$ such that $\varepsilon \leq \varepsilon'$ and $E \subset E'$, the natural homomorphism $K_*^{\varepsilon,E}(A)\hookrightarrow K_*^{\varepsilon',E'}(A)$ is denoted by $\iota_{\varepsilon,E}^{\varepsilon',E'}$. Notice that $\iota_{\varepsilon',E'}^{\varepsilon'',E''}\circ\iota_{\varepsilon,E}^{\varepsilon',E'}=\iota_{\varepsilon,E}^{\varepsilon'',E''}$ when this expression makes sense.\\

One has also forgetful morphisms $\iota_{\varepsilon,E} : K_*^{\varepsilon,E}\rightarrow K_*(A)$ given by $[p,l]_{\varepsilon,E} \mapsto [\kappa_0(p)]-[1_l]$ and $[u]_{\varepsilon,E} \mapsto [\kappa_0(u)] $, and 
$\iota_{\varepsilon',E'}\circ\iota_{\varepsilon,E}^{\varepsilon',E'}=\iota_{\varepsilon, E}$ holds. The controlled $K$-theory groups approximate the usual $K$-groups in the sense that $\varinjlim_{E} K_*^{\varepsilon,E}(A) = K_*(A)$ for any $\varepsilon\in (0,\frac{1}{4})$.\\

If $\phi : A\rightarrow B$ is a filtered morphism, then it preserves almost-projections and almost unitaries, and it induces a morphism of abelian groups $\phi_*^{\varepsilon,E} : K_*^{\varepsilon,E}(A) \rightarrow K_*^{\varepsilon,E}(B)$, such that $\phi_*^{\varepsilon,E} \circ \psi_*^{\varepsilon,E} = (\phi\circ\psi)_*^{\varepsilon,E}$.\\

Here is the main example under study : if $A$ is a $G$-algebra, then $A \rtimes_r G$ is naturally filtered. Indeed, for $E\in \mathcal{E}$, denote by $C_E(A,G)$ the linear subspace $\{ f \in C_c(G,A) \text{ s.t. supp }f\subset E\cup E^{-1} \}$. Then $C_E(G,A)$ obviously defines a filtration on $A\rtimes_r G$. A last remark is in order : this remains true for any $C^*$-algebraic completion of $\cup_{E\in\mathcal{E}} C_E(G,A) = C_c(G,A)$, so for maximal or exotic crossed products also. If one allows himself to use controlled $K$-theory for filtered Banach algebras, one can also adapt this definition for Banach completion of $C_c(G,A)$.

\subsection{Quantitative objects}
In order to study functorial properties of controlled $K$-theory, we will adapt and study the notion of quantitative object defined in \cite{OY2}.\\

\begin{definition}
A quantitative object is a family of abelian groups $\hat{\mathcal{O}}=\{\mathcal{O}_{\varepsilon, E}\}_{\varepsilon\in (0,\frac{1}{4}, E\in\mathcal{E}}$ endowed with a family of group homorphisms $\phi_{\varepsilon, E}^{\varepsilon', E'} : \mathcal{O}_E\rightarrow \mathcal{O}_{E'}$ for any $E,E'\in\mathcal E$ and $0<\varepsilon\leq \varepsilon'<\frac{1}{4}$ such that $E\subset E'$, satisfying $\phi_{\varepsilon, E}^{\varepsilon, E}= id_{\mathcal{O}_E}$ and $\phi_{\varepsilon', E'}^{\varepsilon'', E''}\circ \phi_{\varepsilon, E}^{\varepsilon', E'} =\phi_{\varepsilon'', E''}^{\varepsilon, E}$ whenever $E\subset E' \subset E''$ and $\varepsilon<\varepsilon'<\varepsilon''$.
\end{definition}

We need to define controlled morphisms between quantitative objects. We first define control pairs, which are essentially what ensures that the controlled morphisms do not distort too much the propagation.\\

\begin{definition}
A control pair is a couple $\rho=(a,h)$ where $a\in (0,\frac{1}{4})$ and $h : (0,\frac{1}{4a})\rightarrow \N^*$ is a non-decreasing function. 
\end{definition}

Control pairs can be naturally composed, and if $(a,h)$ and $(b,k)$ are two control pairs, then their composition, denoted by $(b,k)\ast(a,h)$, is defined by $(ab,k\ast h)$, where $k\ast h : (0,\frac{1}{4ab})\rightarrow \N^* ; \varepsilon \mapsto k_{a \varepsilon}h_\varepsilon$. \\

Control pairs naturally act on the index subset of the controlled $K$-groups. Indeed, if $\varepsilon\in (0,\frac{1}{4a})$ and $E\in\mathcal E$, $(a,h).(\varepsilon,E)= (a\varepsilon,E^{h_\varepsilon})$ is in $(0,\frac{1}{4})\times\mathcal E$. This allows to define controlled morphims.\\

We can also compare control pairs. Indeed, define the following partial order : $(a,h) \leq (b,k)$ if $a \leq b$ and $h_\varepsilon\leq k_\varepsilon$ for all $\varepsilon\in (0,\frac{1}{4a})$.\\ 

\begin{definition}
Let $\hat{\mathcal O}$ and $\hat{\mathcal O'}$ be two quantitative objects and $\rho=(a,h)$ a control pair. A $\rho$-controlled morphism is a family of groups homomorphims $F_{\varepsilon, E} : \mathcal O_{\varepsilon, E} \rightarrow \mathcal{ O'}_{a\varepsilon, E^{h_\varepsilon}}$ for any $\varepsilon\in(0,\frac{1}{4a})$ and $E\in\mathcal E$, such that
\[\phi^{a\varepsilon',E'^{h_\varepsilon'}}_{a\varepsilon,E^{h_\varepsilon}} \circ F_{\varepsilon, E} =F_{\varepsilon',E'} \circ \phi_{\varepsilon,E}^{\varepsilon',E'}\] for any $0<\varepsilon<\varepsilon'<\frac{1}{4a}$ and $E\subset E'$.
\end{definition}

\begin{rk}
%If we don't specify any control pair, it is implicit and evident from the context, or it is not crucial to specify it. 
When not specified, the control pair is evident from the context. For example, we will often refer to a controlled morphism, meaning a $\alpha$-controlled morphism for some control pair $\alpha$. For a controlled morphism $\hat F : \hat K(A)\rightarrow \hat K(B)$, we will denote $F:K(A)\rightarrow K(B)$ the unique homomorphism it induces in $K$-theory. We will always try to indicate an analogy with the classical case (as opposed to the controlled or quantitative case) by puttig a hat on top of controlled objects that are hopefully inducing a well known object. For example, controlled $K$-theory is $\hat K$, but the controlled assembly map will be denoted $\hat \mu_{G,A}$, etc.\\
\end{rk}

Let $\rho=(\lambda,h),\alpha,\beta$ be control pairs, and $F :  \hat{\mathcal O} \rightarrow \hat{\mathcal O'}$ and $G : \hat{\mathcal O} \rightarrow \hat{\mathcal O'}$ be $\alpha$- and $\beta$-controlled morphisms respectively. We write $F\sim_\rho G$ if :
\begin{itemize}%{$bullets$}
\item[$\bullet$] $\alpha \leq \rho$ and $\beta \leq \rho$,
\item[$\bullet$] for any $\varepsilon\in (0,\frac{1}{4\lambda})$ and $E\in \mathcal E$, the following diagram commutes : 
\[\begin{tikzcd}
 \mathcal O_{\varepsilon,E} \arrow{r}{F_{\varepsilon,E}} \arrow{d}{G_{\varepsilon,E}} & \mathcal O'_{\alpha (\varepsilon,E) } 
\arrow{d}{\iota_{\beta(\varepsilon,E)}^{\rho(\varepsilon,E)}} \\
 \mathcal O'_{\beta (\varepsilon,E) } \arrow{r}{\iota_{\alpha(\varepsilon,E)}^{\rho(\varepsilon,E)}} & \mathcal O'_{\rho(\varepsilon,E)}\\
\end{tikzcd}.\]
\end{itemize}

\begin{definition}
Let $\alpha$ and $\rho$ be control pairs satisfying $\alpha \leq \rho$ and $F : \hat{\mathcal O} \rightarrow \hat{\mathcal O'}$ a $\alpha$-controlled morphism. We say that $F$ is $\rho$-invertible if there exist a controlled morphism $ G : \hat{\mathcal O'} \rightarrow \hat{\mathcal O}$ such that $G \circ F \sim_\rho Id_{\hat{\mathcal O}}$ and $F\circ G \sim_\rho Id_{\hat{\mathcal O'}}$. $G$ is said to be a $\rho$-inverse for $F$.
\end{definition}

As we will see for controlled assembly maps, the correct notions of injectivity and surjectivity for controlled morphisms need to be adpated in the following way.\\

\begin{definition}
Let $\rho=(\lambda,h)$ and $\alpha$ be controlled pairs, and $F : \hat{\mathcal O} \rightarrow \hat{\mathcal O'}$ a $\alpha$-controlled morphism. 
\begin{itemize}
\item[$\bullet$] $F$ is $\rho$-injective if, given any $\varepsilon \in (0,\frac{1}{4\lambda})$ and $E\in \mathcal E$, $\alpha\leq \rho$ and, for all $x\in \mathcal O_{\varepsilon, E}$ such that $F_{\varepsilon, E}(x)=0$, then $\iota_{\varepsilon, E}^{\lambda\varepsilon, h_\varepsilon E}(x)=0$,
\item[$\bullet$] $F$ is $\rho$-surjective if, given any $\varepsilon \in (0,\frac{1}{4\lambda})$ and $E\in \mathcal E$, for any $y\in \mathcal O' _{\varepsilon, E}$, there exists $x\in \mathcal O_{\rho(\varepsilon,E)}$ such that $F_{\rho(\varepsilon,E)}(x)= \iota_{\varepsilon,E}^{\rho(\varepsilon,E)}$,
\item[$\bullet$] $F$ is a $\rho$-isomorphism if $F$ is both $\rho$-injective and $\rho$-surjective.
\end{itemize}
\end{definition}

\subsection{Controlled exact sequences}

\begin{definition}
Let $F : \hat{\mathcal O}\rightarrow \hat{\mathcal O'}$ and $G : \hat{\mathcal O'}\rightarrow \hat{\mathcal O''}$ be $(\alpha,h)$-controlled and a $(\beta,k)$-controlled morphisms respectively. The sequence
\[\begin{tikzcd}[column sep = small] \hat{\mathcal O} \arrow{r}{F} & \hat{\mathcal O'} \arrow{r}{G} & \hat{\mathcal O''} \end{tikzcd}\]
is called $\rho$-exact at $\hat{\mathcal O'}$ if $G\circ F=0$ and if for all $\varepsilon\in (0,\frac{1}{4 \max (\lambda \alpha,\beta)})$, $E\in\mathcal E$ and any $y\in \mathcal O'_{\varepsilon,E}$ such that $G_{\varepsilon,E}(y) = 0$, then there exists $x\in \mathcal O_{\rho(\varepsilon,E)}$ such that $F_{\rho(\varepsilon,E)}(x)=\iota_{\varepsilon,E}^{\rho(\varepsilon,E)} (y)$.\\
A sequence of controlled morphisms 
\[\begin{tikzcd}[column sep = small] ... \arrow{r}{F_{k-2}} & \hat{\mathcal O_{k-2}} \arrow{r}{F_{k-1}} & \hat{\mathcal O_{k-1}} \arrow{r}{F_k} & \hat{\mathcal O_k} \arrow{r}{F_{k+1}} & ... \end{tikzcd}\] 
is said to be $\rho$-exact if the sequence
\[\begin{tikzcd}[column sep = small] \hat{\mathcal O_{j-1}} \arrow{r}{F_{j}} & \hat{\mathcal O_{j}} \arrow{r}{F_{j+1}} & \hat{\mathcal O_{j+1}} \end{tikzcd}\]
is $\rho$-exact at $\hat{\mathcal O_j}$ for all $j$.
\end{definition}

\subsection{Morita equivalence}
A controlled version of the Morita equivalence exists. Indeed, the classical Morita equivalence states that, if $e$ is a rank $1$ projection in $\mathcal L (H)$, the map $A\rightarrow A\otimes \mathfrak K(H) ; a\mapsto a\otimes e$ induces an isomorphism in $K$-theory. But this map preserves propagation, hence the same proof as in \cite{OY2} gives the following proposition.

\begin{prop}
Let $A$ be a filtered $C^*$-algebra and $H$ a separable Hilbert space. Then the $*$-morphism
\[A\rightarrow A\otimes \mathfrak K(H) ; \quad a\mapsto 
\begin{pmatrix}a & & \\  & 0 & \\ & & ... \end{pmatrix}\]
induces a group isomorphism 
\[\mathcal M_A^{\varepsilon,E} : K^{\varepsilon,E}(A)\rightarrow K^{\varepsilon,E}(A\otimes \mathfrak K(H)) \]
for every $\varepsilon\in(0,\frac{1}{4})$ and $E\in\mathcal E$. The family $\mathcal M_A = (\mathcal M_A^{\varepsilon,E} )_{\varepsilon\in(0,\frac{1}{4}),E\in\mathcal E}$ is called the controlled Morita equivalence and is a controlled morphism. It induces the usual Morita equivalence $M_A: K(A)\rightarrow K(A\otimes \mathfrak K(H))$ in $K$-theory. 
\end{prop}

\subsection{Controlled exact sequences}
We will describe the $6$-term controlled exact sequence associated to a completely filtered extension of $C^*$-algebras. For any extension of $C^*$-algebras 
\[\begin{tikzcd}[column sep = small]0\arrow{r} & J\arrow{r} & A\arrow{r} & A/J\arrow{r} & 0 \end{tikzcd},\]
we denote $\partial_{J,A}$ the boundary map $K_*(A/J)\rightarrow K_{*+1}(J)$. The reader can find all the proofs and properties of the following results in \cite{OY2}.

\begin{definition}
Let $A$ a filtered $C^*$-algebra and $J\subset A$ an ideal. If $J_E = A_E\cap J$, the extension
\[\begin{tikzcd}[column sep = small]0\arrow{r} & J\arrow{r} & A\arrow{r} & A/J\arrow{r} & 0 \end{tikzcd}\]
is said to be completely filtered if the continuous linear bijection $A_E/J_E \hookrightarrow (A_E+J)/J$ induced by the inclusion $A_E\hookrightarrow A$ is a complete isometry, i.e.
\[ \inf_{y\in M_n(J_E)} ||x+y|| = \inf_{y\in M_n(J)} ||x+y||\quad,\forall n\in \N,x\in M_n(A_E),E\in \mathcal E.\]
\end{definition}

\begin{prop}
There exists a control pair $(\alpha_D,k_D)$ such that for any completely filtered extension of $C^*$-algebras
\[\begin{tikzcd}[column sep = small]0\arrow{r} & J\arrow{r} & A\arrow{r} & A/J\arrow{r} & 0 \end{tikzcd}\]
there exists a $(\alpha_D,k_D)$-controlled morphism 
\[D_{J,A} : \hat K(A/J)\rightarrow \hat K(J)\]
which induces $\partial_{J,A}$ in $K$-theory.
\end{prop}

\begin{thm}
There exists a universal control pair $(\lambda,h)$, which does not depend on $\mathcal E$, such that for any completely filtered extension of $C^*$-algebras 
\[\begin{tikzcd}[column sep = small]0\arrow{r} & J\arrow{r}{\iota} & A\arrow{r}{q} & A/J\arrow{r} & 0 \end{tikzcd}\]
the following $6$-term exact sequence is $(\lambda,h)$-exact
\[\begin{tikzcd}[column sep = small]
 \hat K(J) \arrow{r}{\iota_*} & \hat K(A) \arrow{r}{q_*} & \hat K(A/J)\arrow{d}{D_{J,A}} \\
 \hat K(A/J) \arrow{u}{D_{J,A}} & \hat K(A) \arrow{l}{q_*} & \hat K(J)\arrow{l}{\iota_*}
\end{tikzcd}.\]
\end{thm}

The following remark can be found in \cite{OY2} (remark $3.8$) and will be used to prove functorial properties of the controlled Roe and Kasparov transformations.

\begin{rk}\label{rk3.8}
Let $A$ and $B$ two $\mathcal E$-filtered $C^*$-algebras, and $\phi : A\rightarrow B$ a filtered $*$-homomorphism. Let $I$ and $J$ be respectively ideals in $A$ and $B$ and assume that :
\begin{itemize}
\item[$\bullet$] $0 \rightarrow I \rightarrow A \rightarrow A/ I \rightarrow 0$ and $0 \rightarrow J \rightarrow B \rightarrow B/J \rightarrow 0$ are completely filtered extensions of $C^*$-algebras,
\item[$\bullet$] $\phi(I)\subset J$,
\end{itemize}
then $D_{J,B}\circ \tilde\phi_* = \phi_* \circ D_{I,A}$.
\end{rk}

\subsection{Tensorisation in $KK$-theory}

If $B$ is a filtered $C^*$-algebra and $A$ any $C^*$-algebra, and if $A\otimes B$ is the spatial tensor product, then $(A\otimes B_E)_{E\in\mathcal E}$ defines a filtration of $A\otimes B$. If $\phi : A_1 \rightarrow A_2$ is a $*$-homomorphism, we use the notation $\phi_B$ for the induced $*$-homomorphism $A_1\otimes B\rightarrow A_2\otimes B$. \\

In \cite{kasparovKKNovikov}, G. Kasparov defined a map
\[\tau_B : KK(A_1,A_2)\rightarrow KK(A_1\otimes B, A_2\otimes B)\]
for any $C^*$-algebras $A_1$ and $A_2$, which is compatible with the Kasparov product. Any $z\in KK(A_1,A_2)$ defines a morphism
\[K(A_1\otimes B)\rightarrow K(A_2\otimes B)\]
which is proved in \cite{OY2} to be induced from a controlled morphism. The following theorem is borrowed from \cite{OY2}.\\

\begin{thm}
There exists a control pair $(\alpha_\tau,k_\tau)$ such that, for any filtered $C^*$-algebra $B$, any $C^*$-algebras $A_1$ and $A_2$ and any $K$-cycle $z\in KK(A_1,A_2)$, there exists a $(\alpha_\tau,k_\tau)$-controlled morphism $\hat \tau_B : \hat K(A_1\otimes B)\rightarrow \hat K(A_2\otimes B)$
such that :
\begin{itemize}\label{tensorization}
\item[$\bullet$] $\hat \tau_B(z)$ induces right-multiplication by $\tau_B(z)$ in $K$-theory,
\item[$\bullet$] for any $K$-cycles $z,z'\in KK(A_1,A_2)$, $\hat \tau_B(z+z')=\hat\tau_B(z)+\hat\tau_B(z')$,
\item[$\bullet$] if $\phi : A_1\rightarrow A'_1$ is a $*$-homomorphism, then $\hat\tau_B(\phi^*(z)) =  \hat\tau_B(z)\circ (\phi_B)_*$ for any $z\in KK(A'_1,A_2)$,
\item[$\bullet$] if $\phi : A_2'\rightarrow A_2$ is a $*$-homomorphism, then $\hat\tau_B(\phi_*(z)) = (\phi_B)_*\circ \hat\tau_B(z)$ for any $z\in KK(A_1,A'_2)$,
\item[$\bullet$] $\hat \tau_B([Id_A])\sim_{(\alpha_\tau,k_\tau)} Id_{\hat K(A\otimes B)}$,
\item[$\bullet$] for any $C^*$-algebra $D$, any $K$-cycle $z\in KK(A_1,A_2)$, $\hat\tau_B (\tau_D(z))= \hat\tau_{B\otimes D}(z)$.
\item[$\bullet$] for any semi-split extension $\begin{tikzcd}[column sep = small] 0 \arrow{r}& J \arrow{r} & A \arrow{r} & A/J\arrow{r} & 0\end{tikzcd}$ with boundary element $[\partial_{J,A}]\in KK_1(A/J,J)$, $\hat\tau_B([\partial_{J,A}])=D_{J\otimes B,A\otimes B}$.
\end{itemize}
\end{thm}

This controlled tensorisation map respects Kasparov product. See \cite{OY2} for a proof.

\begin{thm}
There exists a linear control pair $\lambda$ such that, for any separable $C^*$-algebras $A_1$ and $A_2$, any filtered $C^*$-algebra $B$, the following holds : for any $z\in KK(A_1,A_2)$ and $z'\in KK(A_2,A_3)$,
\[\hat\tau_B(z\otimes z')\sim_\lambda \hat\tau_B(z')\circ\hat\tau_B(z)\]
\end{thm}

This last proposition asserts that the controlled tensorisation map is natural with respect to morphism of filtered $C^*$-algebras. \cite{OY2}

\begin{prop}
Let $B_1$ and $B_2$ be filtered $C^*$-algebras and $(\phi,\rho) :B_1\rightarrow B_2 $ a filtered morphism. Then $ (\phi_{A_2})_*\circ \hat \tau_{B_1}=\hat \tau_{B_2}(z)\circ(\phi_{A_1})_* $ for any $z\in KK(A_1,A_2)$.
\end{prop}




















 
