\section{ Mayer-Vietoris exact sequences and controlled cutting and pasting}

\subsection{Mayer-Vietoris sequence in $K$-homology}

We first recall how to construct a Mayer-Vietoris element out of any pull back diagram of $C^*$-algebras.\\

Let us decompose $X$ into two open sets $U_0$ and $U_1$ and consider the $C^*$-algebras $A=C_0(X)$, $A_j=C_0(U_j)$ where $U_{01}=U_0\cap U_1$, and the cone $C$ of $A_0\oplus A_1$, with the canonical morphisms $\alpha :C \rightarrow A$ and $\beta : C \rightarrow SA_{01}$. The first is an homotopy equivalence, and the Mayer-Vietoris boundary is defined as the dotted arrow in the following commutative diagram
\[\begin{tikzcd}
KK^G_*(C,B) \arrow{r}{\beta^*}\arrow{d}{ \alpha^*}& KK^G_*(SA_{01},B)\arrow{d}{[\partial_{A_{01}}]\otimes -}\\
KK^G_*(A,B) \arrow[dotted]{r}& KK^G_{*+1}(A_{01},B)\\
\end{tikzcd}\]

where the right vertical arrow is the Bott element $[\partial_{A_{01}}]\in KK_1^G(A_{01},SA_{01})$.\\

At the level of controlled $K$-theory, Oyono-Oyono and Yu introduced a notion of Mayer-Vietoris pair in a filtered $C^*$-algebra $A$ weaker than that of a pull-back diagram. Recall that a $R$-controlled weak Mayer-Vietoris pair for $A$ is a quadruple $(\Delta_0,\Delta_1,A_0,A_1)$ such that for some constant $c$:\\

\begin{itemize}
\item[$\bullet$] if $x\in M_n(A_s)$ for $s\leq R$ and any $n>0$ , there exists $x_j\in M_n(\Delta_j\cap A_s)$ such that $x=x_0+x_1$ and $||x_j||\leq c||x||$,
\item[$\bullet$] $A_j$ is filtered by $(A_j\cap A_s)_{s\geq 0}$ and $C^* N_{\Delta_j}^{(R,5R)}\subset A_j$
\item[$\bullet$] for any $\epsilon >0$, if $x\in M_n(A_{0,s})$, $y\in M_n(A_{1,s})$ such that $||x-y||<\epsilon$,  there exists $z\in M_n(A_{0,s}\cap A_{1,s})$ with $||z-x||<\epsilon$ and $||z-y||<\epsilon$. \\
\end{itemize}

For example, take $G$ to be an étale groupoïd with proper length $l$, with compact base space $X$. The convolution algebra $C^*_r G$ is filtered by $(C_c(G_R))_{R>0}$, $G_R=l^{-1}[0,R)$. Fix $R>5r$. If $V$ is an open subset of $X$, set \\

\begin{itemize}
\item[$\bullet$] $V^R= \{r(g) : g\in G_V\cap G_R\}$,
\item[$\bullet$] $\Delta_V = C_0(G_V\cap G_R)$
\item[$\bullet$] $G_{V,R}^V = \langle G_V^V \cap G_R \rangle$\\
\end{itemize} 

then $(\Delta_{V_0},\Delta_{V_1}, C_r^*(G_0),C_r^*(G_1))$ is a $r$-controlled Mayer-Vietoris pair for $C^*_r G$ when $X=V_0\cup V_1$, and $G_j = G_{V_j^R}^{V_j^R , (R)}$.\\

The existence of a controlled Mayer-Vietoris pair is nice, because even if the $C^*$-algebra is simple, it can possess such a decomposition, and the following result gives a way to compute the $K$-theory analogous to the situation of a classical Mayer-Vietoris decomposition :

\begin{thm}
For every positive $c$, there exists a control pair $(\lambda,h)$ such that for any filtered $C^*$-algebra $A$ which has a $R$-controlled Mayer-Vietoris pair $(\Delta_0,\Delta_1,A_0,A_1)$ the following sequence is $(\lambda,h)$-exact at order $R$
\[\begin{tikzcd}
\hat K_0(A_0\cap A_1 ) \arrow{r}& \hat K_0(A_0)\arrow{r}\oplus \hat K_0(A_1) & \hat K_0(A)\arrow{d}{D} \\
\arrow{u}{D}\hat K_1(A) &\arrow{l} \hat K_1(A_0)\oplus \hat K_1(A_1) & \arrow{l}\hat K_1(A_0\cap A_1) 
\end{tikzcd}\]
where $D$ is a controlled Mayer-Vietoris boundary.\\
\end{thm}

To go back to our example, if $X=V_0\cup V_1$, we simulteanously have two decompositions : that arising from the decomposition of the Rips simplex into two open sets $P_d(G^{V^R,(R)}_{V^R})$, and that of the controlled Mayer Vietoris pair. The aim of this section is to show that the quantitative assembly maps respects these two exact sequences in a precise way.\\

Actually, in this particular case, the quantitative Mayer-Vietoris exact sequence should hold at all orders, which would allows us to state a much stronger result, with more interesting applications : a Künneth formula for crossed product algebras of étale groupoids.\\

\textbf{A remark :} There is a seemingly harmful parallel between the controlled Mayer-Vietoris decomposition and the nuclear dimension of the reduced $C^*$-algebra. Namely, to show that the pair satisfies the Mayer-Vietoris conditions, we make use of the completely positive map induced by 
\[\left\{\begin{array}{lcr} C_c(G) &\rightarrow & C_c(G) \\ f &\mapsto & \phi_0\circ r\ . f .\ \phi_0\circ s\end{array}\right.\]   
where $\phi_0$ is any continuous function $X\rightarrow [0,1]$ with support in $V_0$ which is $1$ on some compact $K\subset V_0$.\\
Could we push the analogy further ? 

\subsection{Standard modules}

The aim of this section is to develop a notion of non-degenerate standard module over a groupoid analogous to non-degenerate stadard modules over coarse spaces. \\

Let us first recall the coarse case. Let $X$ be a discrete metric space with bounded geometry. 

\begin{definition}
A $X$-module is a Hilbert space $H_X$ equipped with a $*$-representation $\phi : C_0(X) \rightarrow \mathcal L(H_X)$. The $X$-module $(H_X,\phi)$ is said to be :\\

\begin{itemize}
\item[$\bullet$] standard if $\phi(C_0(X))H_X$ is dense in $H_X$,
\item[$\bullet$] non-degenerate if $\forall f \in C_0(X), \phi(f)\in \mathfrak K (H_X) \implies f=0$.
\end{itemize}
\end{definition}

The usefulness of n.d.s. $X$-modules comes from the following lemma :

\begin{lem}
Let $X$ and $Y$ be two discrete metric spaces with bounded geometry, and $h : X\rightarrow Y$ a coarse map. Then, for any two standard modules $H_X$ and $H_Y$ over $X$ and $Y$ respectively, there exists an isometry which covers $h$, i.e. for any $\epsilon >0$, there exists $V\in \mathcal L(H_X,H_Y)$ such that 
\[\text{supp }V \subset \{(x,y)\in X\times Y, d(h(x),y)<\epsilon\}.\]
\end{lem}

\textbf{Remark :} We can induce $V$ on the Roe algebras by $\forall T\in C^*(X,H_X), Ad_V(T) := V T V^* \in C^*(Y,H_Y)$, and the preceding lemma entails that \[(Ad_V)_* : K(C^*(X,H_X))\rightarrow K(C^*(Y,H_Y))\] only depends on the coarse class of $h$. We directly see that the $K$-theory of the Roe-algebras of $X$ do not depend on the standard modules if they are n.d.s. : one just need to take an isometry covering the identity. \\

We can actually show that taking the Roe algebra is a functor. Choose, for any coarse space $X$ a n.d.s. $X$-module $H_X$, and consider the category $Coarse$ of coarse spaces with morphisms coarse maps, and the category $KK$ with objects $C^*$-algebras, and morphisms defined by $KK$-theory, $Hom_{KK}(A,B)=KK(A,B)$. Then $ X \mapsto C^*(X,H_X) $ and $\left(h:X \rightarrow Y\right) \mapsto (Ad_V)_*\in KK(C^*(X,H_X),C^*(Y,H_Y))$ defines a functor $Coarse \rightarrow KK$, which does not depend on the choices $X\mapsto H_X$ being made.\\

We will focus ont the following result :

\begin{thm}
Let $X$ be a finite simplicial complex, and $B$ a $C^*$-algebra. There exists $\varepsilon_0\in (0,\frac{1}{4})$ such that for all $\varepsilon \in (0,\varepsilon_0)$, there exists $R_\varepsilon>0 $ s.t. 
\[\forall R\in (0,R_\varepsilon),\  Ind_{X,B}^{\varepsilon, R} : KK(C(X),B)\rightarrow K(B\otimes \mathfrak K(H_X))\] 
is an isomorphism.
\end{thm}

We wish to prove a similar result for "nice" groupoids. Let us recall the definition of $G$-simplicial complex from \ref{TuBC2}.

\begin{definition}
Let $G$ be a locally compact groupoid. A $G$-simplicial complex of dimension less than $n$ is a triple $(Z,\Delta, p)$ where
\begin{itemize}
\item[$\bullet$] $Z$ is a locally compact space of vertices and $p: Z \rightarrow G^{(0)}$ a locally injective map, which is an anchor map for an action of $G$, 
\item[$\bullet$] $\Delta$ is a closed $G$-invariant subset of the probability measures $P(Z)$ on $Z$, equipped with the weak-$*$ topology, with the property that every element of $\Delta$ has a support contained in a fiber of $p$ and has at most $n+1$ elements. Such a support is call a simplex of $\Delta$. Moreover, if $\nu\in \Delta,\mu \in P(Z)$ and $supp(\mu)\subset supp(\nu)$ then $\mu\in \Delta$.
\end{itemize}
\end{definition}































 


