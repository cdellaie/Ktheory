\section{K-théorie des $C^*$-algèbres}

%Différentes définitions du foncteur $K_0$:
%\begin{itemize}
%\item groupe de Grothendieck associé au semi-groupe des classes d'équivalences de projections dans $M_\infty (A)$ muni de la somme directe.
%\item groupe de Grothendieck associé  par les sous-modules projectifs fermés de type fini de $\mathcal H_A$
%\end{itemize}
 
\subsection{Définitions et propriétés}
Avant de nous attaquer proprement dit au résultat de l'article de Pimsner et Voiculescu~\cite{PV}, nous allons citer les résultats de $K$-théorie des $C^*$-algèbres dont nous aurons besoin. Les preuves ne seront pas toujours détaillées, et peuvent être trouvées dans n'importe quel livre d'introduction au sujet, par exemple celui de Wegge-Olsen~\cite{WeggeOlsen}.\\

\begin{definition}
Soit $p$ et $q$ deux projecteurs dans une $C^*$-algèbre $A$. On définit trois relations d'équivalences :\\
$p\sim q$ s'il existe une isométrie partielle $u$ de $A$ telle que $p=u^*u $ et $q=uu^*$. ( équivalence de Murray-Von Neumann)\\
$p\sim_u q$ s'il existe un unitaire $u$ de $A^+$ tel que $p=uqu^*$. (Similitude)\\
$p\sim_h q$ s'il existe un chemin continu en norme de projections de $p$ à $q$.(Homotopie)\\
\end{definition}

En général, on a : $\sim_h \Rightarrow \sim_u \Rightarrow \sim$. Pour avoir les implications inverses, on peut se placer dans $M_\infty(A)$. (Doubler la dimension à chaque fois suffit) On peut alors considérer l'ensemble des projections de $M_\infty (A)$ et quotienter par l'unique relation d'équivalence définie ci-dessus. L'ensemble obtenu est un semi-groupe pour l'opération de somme directe de projecteur, nommé $V(A)$. On notera $p_n=\text{diag}(1,...,1,0,0,...)$ l'élément de $V(A)$ avec des $1$ sur les $n$ premiers emplacements diagonaux. \\

\begin{definition}
Le premier groupe de $K$-théorie de $A$ est :\\
le groupe de Grothendieck de $V(A)$ si $A$ est unitale.\\
le noyau de $K_0(A^+)\rightarrow K_0(\C)$ sinon.
\end{definition}

Tout élément de $K_0(A)$ se représente comme la différence de deux classes de projecteurs de $M_k(A^+)$, qui peuvent être choisis dans $M_k(A)$ si $A$ est unitale. Plus précisément, tout élément de $K_0(A)$ peut s'écrire
\[[p]-[p_n]\]
où $p\in M_k(A^+)$ est un projecteur avec $k\leq n$ tel que $p-p_n \in M_k(A)$.\\

$K_0$ est un foncteur covariant de la catégorie des $C^*$-algèbres dans celle des groupes abéliens, c'est pourquoi l'on se permettra de noter, pour tout homomorphisme involutif $\varphi : A \rightarrow B$ entre deux $C^*$-algèbres, $K_0(\varphi)=\varphi_* : K_0(A)\rightarrow K_0(B)$ l'homomorphisme de groupes défini par 
\[\varphi_*([p]-[q])=[\varphi(p)]-[\varphi(q)].\]

Pour passer aux groupes de $K$-théorie d'indices supérieurs de $A$, on se servira du foncteur de suspension $S(A)=A\otimes C_0(\R)$.

\begin{definition}
Pour toute algèbre de Banach unitale, on pose \[GL_\infty(A) = \varinjlim GL_n(A) \quad \text{(limite inductive)}\]
munie de la topologie de la limite inductive.\\
Pour $n\geq 1$, on définit :
\[K_n(A)=\pi_{n-1}\left(GL_\infty (A)\right)\] 
où $\pi_n, n\geq 1$ désigne le $n^{ie}$-groupe d'homotopie, et $\pi_0$ le groupe des composantes connexes.
\end{definition}

Quelques remarques : 
\begin{enumerate}
\item Le groupe $K_1(A)$ est donc généré par les classes $[u]$ où $u$ est un unitaire ou un inversible de $GL_n(A)$, avec la présentation $[1]=0$, $[u]+[v]=[u\oplus v]$ et $[u]=[v]$ si $u$ et $v$ sont reliés par un chemin continu d'unitaires ou d'inversibles.
\item On a en fait la relation suivante \[\forall i\in \mathbb N, \quad K_{i+1}(A)=K_i(S(A)).\]
\item Dans le cas des $C^*$-algèbres comme en $K$-théorie topologique, on a périodicité de Bott : $K_{i+2}(A)\simeq K_i(A)$.\\
\item $K_1(A)$ est bien un groupe abélien. Premièrement, notons que la matrice $\begin{pmatrix}1 & 0\\ 0 & 1\end{pmatrix}$ est connectée à  $\begin{pmatrix}0 & -1\\ 1 & 0\end{pmatrix}$ par l'arc $t\mapsto \begin{pmatrix}\cos t & -\sin t \\ \sin t  & \cos t\end{pmatrix}$ de $GL(2,\C)$. Ceci assure, par multiplication à gauche et à droite par des matrices élémentaires, que l'on peut échanger deux colonnes ou deux lignes d'une matrice sans changer sa classe dans $K_1(A)$. Si $\sim$ signifie "être dans la même composante connexe", alors :
\begin{align*}
\begin{pmatrix} xy & 0\\ 0 & 1\end{pmatrix} & \sim \begin{pmatrix} x & 0\\ 0 & 1\end{pmatrix} \begin{pmatrix}y & 0\\ 0 & 1\end{pmatrix}\\
							&\sim \begin{pmatrix}x & 0\\ 0 & 1\end{pmatrix} \begin{pmatrix}1 & 0\\ 0 & y\end{pmatrix}\\
							&\sim \begin{pmatrix}x & 0\\ 0 & y\end{pmatrix}.
\end{align*}
Le dernier terme étant symétrique, $K_1(A)$ est bien abélien. On a montré au passage que $\begin{pmatrix} xy & 0\\ 0 & 1\end{pmatrix}\sim \begin{pmatrix}x & 0\\ 0 & y\end{pmatrix}$, ce que l'on utilisera fréquemment, par exemple lorsque l'on affirmera que $\begin{pmatrix}u & 0\\ 0 & u^*\end{pmatrix}$ est connectée à l'identité si $u$ est unitaire.
\end{enumerate}

Cette dernière remarque nous amène à un autre résultat utile.
\begin{prop} Soit $A$ une $C^*$-algèbre unitale. Un élément $z\in A$ est inversible et connecté à $1$ par un chemin d'inversibles ssi il existe $a_1,..., a_n\in A$ tels que $z=\prod_j e^{a_j}$.
\end{prop}
\begin{dem}
Toute exponentielle est connectée par un chemin d'inversibles à l'identité par l'arc $t\mapsto e^{t a}, a\in A$ : le sens indirect est prouvé. On note $\exp(A)$ les éléments qui sont des produits d'exponentielles. Soit $z=\prod_j e^{a_j}$ un tel élément. Alors, si $z'$ est assez proche de $z$ : $||z-z'||<\frac{1}{||z^{-1}||}$, le spectre de $z'z^{-1}$ est contenu dans le disque ouvert de rayon $1$ et de centre $1$, on peut donc définir son $\log$. Donc $z'= \exp(\log (z'z^{-1} ) ) z $ est bien un produit d'exponentielles. Comme $\exp(A)$ est ouvert et fermé et qu'il contient $1$, $GL_1(A)_0=\exp(A)$.
\qed
\end{dem}

Cette proposition a pour corollaire un résultat de relèvement important.
\begin{prop}
Soit $\psi : A \rightarrow A''$ un $*$-homomorphisme surjectif entre deux $C^*$-algèbres unitales. Alors tout unitaire (resp. inversible)de $A''$ qui est connecté à l'identité peut se relever en un unitaire (resp. inversible) de $A$. 
\end{prop} 

\begin{dem}Soit $a''\in A''$ un unitaire connecté à l'identité.
La proposition précédente assure que $a'' = \prod_j e^{x_j}$. Par surjectivité de $\psi$, il existe des éléments $y_j$ de $A$ tels que $\psi(y_j)=x_j$. Mais alors $a=\prod_j e^{y_j}$ est un relevé inversible connecté à l'identité qui relève $a''$. Si $a''$ est unitaire, $a(a^*a)^{-\frac{1}{2}}$ convient.
\qed
\end{dem}
%Les groupes de $K$-théorie d'ordre supérieurs de $A$ sont définis par suspension :
%\[\forall i\in \mathbb N, \quad K_i(A)=K_0(S(A)).\]

Ces foncteurs de la catégorie des $C^*$-algèbres dans celle des groupes abéliens sont semi-exacts, i.e. ils transforment toute suite exacte courte en suite exacte très courte. Un point remarquable, et qui sera utilisé plus tard : leur comportement vis à vis de la stabilisation est naturel.\\

Donnons en exemple quelques calculs de groupes de $K$-théorie. \\

Deux projections dans $M_n(\C)$ sont équivalentes précisément lorsque elles ont même rang. Ceci assure que $K_0(\C)=G_{\mathbb N}=\Z$, un générateur étant n'importe quel projecteur de rang $1$. La $K$-théorie étant stable par augmentation et limite inductive, on a aussi : $\Z=K_0(M_n(\C))=K_0(\K)$ avec même générateur. \\

$K_0(C(\mathbb S^1))=\Z$ est généré par la classe du projecteur $z\mapsto 1_\C$, et $K_1(C(\mathbb S^1))=\Z$ avec pour générateur la classe de l'unitaire $z\mapsto  z$. En effet, $C(\mathbb S^1)$ n'admet que $0$ et $1$ comme projecteurs. De plus $K_1(C(\mathbb S^1))=\pi_0(GL_\infty(C(\mathbb S^1)))$ et $\pi_0(GL(n,C(\mathbb S^1) ))=\pi_1(GL(n,\C))=\Z$ pour tout entier $n$.\\

\begin{definition}
Une $C^*$-algèbre $A$ est dite stable si $A\simeq A\otimes \K$, où $A\otimes \K$ est par défintion la stabilisation de $A$. On dit de deux $C^*$-algèbres qu'elles sont stablement isomorphes si $A\otimes \K \simeq B\otimes \K$ .
\end{definition}

L'agèbre des matrices se plonge naturellement (mais non canoniquement) dans l'algèbre de opérateurs bornés $\mathfrak M_n \simeq P_n \B P_n \subset \K $, où $P_n$ est le projecteur sur les $n$ premières coordonnées. Et le diagramme commutatif 

\[\begin{tikzcd}
\mathfrak M_m \arrow{r}{\iota_{nm}} \arrow{drr}{\phi_m}& \mathfrak M_n \arrow{r} \arrow{dr}{\phi_m}& \varinjlim \mathfrak M_n \arrow{d} \\
				&				& \K
\end{tikzcd}\]
permet d'affirmer que l'algèbre des opérateurs compacts est limite inductive  du sytème inductif des matrices $\K \simeq \varinjlim \mathfrak M_n$. Ici, une matrice de taille $m\leq n$ est plongée dans $\mathfrak M_n$ en la positionnant dans le coin haut-gauche. L'injectivité de la flèche verticale provient de celle des $\phi_n$ et la surjectivité de la densité des opérateurs de rang fini dans $\K$. Finalement, comme $M_n(A) \simeq A\otimes \mathfrak M_n$, on obtient que $A\otimes \K = \varinjlim M_n(A)$.\\
Il est facile de voir que homomorphismes $\iota_{nm*}=K_0\iota_{nm}$ sont de isomorphismes, on obtient ainsi un diagramme commutatif pour tout $n\geq m$, que l'universalité de la limite inductive permet de compléter par une unique flèche (en pointillés) :
\[\begin{tikzcd}K_0(M_n(A)) \arrow{r}{\phi_{m*}}& K_0(A\otimes \K)\arrow[dotted]{d} \\
			K_0(M_m(A)) \arrow{u}{\iota_{nm*}}\arrow{ur}{\phi_{n*}}& K_0(A)   \arrow{l}{\iota_{m1 *}}\arrow{ul }{\iota_{n1*}}
\end{tikzcd}.\]
Comme les $\iota_*$ sont des isomorphismes, cette flèche l'est. Mais le diagramme commute, donc son inverse est donnée par $\phi_{n*}\circ\iota_{n1*}=\phi_{1*}$. On vient de montrer que 
\[\left\{\begin{array}{rcl} A & \rightarrow & A\otimes \K \\ a &\mapsto & a\otimes e_{11}\end{array}\right.\]
induit un isomorphisme $K_0(A)\simeq K_0(A\otimes \K)$.\\

Soit $p\in M_n(A^+)$ un projecteur. Posons 
\[f_p(z):= zp-1_n+p\quad, \forall z\in \mathbb S^1.\]
Pour tout $z\in \mathbb S^1$, $f_p(z)$ définit un unitaire, donc $[f_p]_1\in K_1(SA)$. De plus $||f_p-f_q||=\sup_{z\in \mathbb S^1}||(z-1)(p-q)||=2||p-q||$, ce qui montre que $p\mapsto f_p$ et $f_p\mapsto p$ sont continues. Enfin, la relation $f_p f_q=f_{p+q}$ lorsque $p$ et $q$ sont orthogonales assure que $p\mapsto [f_p]_1$ se factorise en un homomorphisme $V(A)\rightarrow K_1(SA)$. 

\begin{definition}
Soit $A$ une $C^*$-algèbre. L'application de Bott est l'homomorphisme de groupe défini par :
\[\beta_A : \left\{ \begin{array}{rcl} K_0(A) & \rightarrow & K_1(SA) \\ 
\ [p]_0-[q]_0 & \mapsto & [f_p f_q^*]_1\end{array}\right.\]
\end{definition}

Soit $u \in \mathcal U_n^+(A)$. Comme 
\[\begin{pmatrix} u & 0 \\ 0 & u^*\end{pmatrix} \sim \begin{pmatrix} uu^* & 0 \\ 0 & 1_n \end{pmatrix},\]
on peut trouver un chemin continu d'unitaires $t\mapsto w_t$ de $1_{2n}$ à $\begin{pmatrix} u & 0 \\ 0 & u^*\end{pmatrix}$. Mais alors $q_t := w_t p_n w_t^* \in M_{2n}(A^+)$ est un lacet continu de projecteurs d'origine $p_n$. Comme $\pi_\C(q_t)=p_n$, $q_t-p_n \in M_{2n}((SA)^+)$ et on peut définir l'application Theta.

\begin{definition} L'application Theta est l'homomorphisme défini par :
\[\theta_A : \left\{ \begin{array}{rcl} K_1(A) & \rightarrow & K_0(SA) \\ 
\ [u]_1 & \mapsto & [q]_0 - [p_n]\end{array}\right.\]
\end{definition}

Ces deux applications, Bott et Theta, sont toutes deux des transformations naturelles entre les foncteurs $K_0$ et $K_1 S$ pour Bott, $K_1$ et $K_0 S $ pour Theta. Cela signifie que tout $*$-homomorphisme $\alpha : A \rightarrow B$ entre deux $C^*$-algèbres induit deux diagrammes commutatifs :
\[\begin{array}{cc}\begin{tikzcd}
K_0 (A)\arrow{r}{\alpha_*}\arrow{d}{\beta_A} &  K_0(B) \arrow{d}{\beta_B}\\
K_1(SA) \arrow{r}{\alpha_*} & K_1(SB) 
\end{tikzcd} & 
\begin{tikzcd}
K_1 (A)\arrow{r}{\alpha_*}\arrow{d}{\theta_A} &  K_1(B) \arrow{d}{\theta_B}\\
K_0(SA) \arrow{r}{\alpha_*} & K_0(SB) 
\end{tikzcd}
\end{array}\]

\subsection{La suite exacte à six termes}
\begin{thm}
Soit \begin{tikzcd}[column sep=small] 0 \arrow{r}  & J \arrow{r}{\iota}& A \arrow{r}{\pi} & B \arrow{r} & 0\end{tikzcd} une suite exacte de $C^*$-algèbres. Alors la suite à six termes suivante est exacte :\\

\[\begin{tikzcd}
 K_0(J) \arrow{r}{\iota_*} & K_0(A)  \arrow{r}{\pi_*}  &    K_0(B)  \arrow{d}{\delta}  \\
 K_1(B) \arrow{u}{\partial} & K_1(A)  \arrow{l}{\pi_*}  &    K_1(J) \arrow{l} {\iota_*}
\end{tikzcd}\]
\end{thm}

C'est l'un des résultats fondamentaux en $K$-théorie, il permet des calculs effectifs. Le premier pas à faire est de construire l'indice associé à toute suite exacte $\partial : K_1(B)\rightarrow K_0(J)$, qui transforme toute suite exacte courte en suite exacte longue. \\

Si $J$ est un idéal bilatère fermé de $A$, et $x=[u]\in K_1(A/J)$ pour un certain $u\in U^+_n(A)$, on choisit $v\in U^+_k(A)$ tel que $\begin{pmatrix}u & 0 \\ 0 & v\end{pmatrix}$ soit connectée à $1_{n+k}$ (par exemple $v=u^*$ convient). Alors on peut relever $\begin{pmatrix}u & 0 \\ 0 & v\end{pmatrix}$ en un unitaire $w\in U_{n+k}(A)$.

\begin{definition}
L'indice d'un élément $x\in K_1(A/J)$ décrit ci-dessus est défini comme  
\[\partial (x) = [wp_n w^*]-[p_n]\in K_0(J).\] 
\end{definition}

Si l'on se donne un relevé $w$ de $\begin{pmatrix}u & 0 \\ 0 &u^* \end{pmatrix}$, alors $v=(1-p_n)w^*$ est une isométrie partielle :
\begin{align*}
vv^* &=1-p_n\\
v^*v &=1-w p_n w^*
\end{align*}
sont en effet des projecteurs. Mais en remplaçant ces expressions dans la formule de l'indice, on obtient
\[\partial [u]=[1-v^* v]-[1-vv^*].\]
Est-on obligé de doubler la dimension ? Non, si $u$ se relève en une isométrie partielle $a$ par exemple. Alors $w=\begin{pmatrix}a & 1-aa^* \\ 1-a^*a & a^*\end{pmatrix}$ est un unitaire qui relève $\begin{pmatrix}u & 0 \\ 0 &u^* \end{pmatrix}$, et on peut calculer l'indice de $[u]$ avec des matrices de taille identique : 
\[\partial [u]=[1-a^* a]-[1-aa^*].\]

\begin{prop}
Si la suite de $C^*$-algèbres \[\begin{tikzcd}[column sep = small] 0\arrow{r} & J\arrow{r} & A\arrow{r} & A/J\arrow{r} & 0\end{tikzcd}\]
est exacte, alors la suite de groupes abéliens
\[\begin{tikzcd}[column sep = small] K_1(J) \arrow{r} & K_1(A)\arrow{r} & K_1(A/J) \arrow{r}{\partial} &  K_0(J) \arrow{r} & K_0(A)\arrow{r} & K_0(A/J) \end{tikzcd}\]
est exacte.
\end{prop}

On peut trouver $2$ isomorphismes naturels qui donnent la périodicité de Bott :
\[K_{i+1}(A)\simeq K_i(A), i=0,1.\]
Ces isomorphismes sont donnés par l'application de Bott $\beta : K_0 \rightarrow K_1 S$ et $\theta :  K_1 \rightarrow K_0 S$. La périodicité permet de conclure en enroulant la suite exacte longue grâce à l'application exponentielle $\delta : K_0(B)\rightarrow K_1(J)$ qui est la composition $\theta_J^{-1}\circ \partial \circ \beta_B$.\\

\begin{prop}[Remarque sur le nom d'application exponentielle] Soit $J$ un idéal bilatère de la $C^*$-algèbre $A$. Si $p-p_n \in M_\infty (A/J)$ et $x\in M_\infty (A^{+})$ est un relevé auto-adjoint de $p$, alors :
\[\delta([p]-[p_n])=[\exp(-2i\pi x)].\]
\label{exp}
\end{prop}
De plus, si toutes les projections de $M_\infty(A/J^{+})$ peuvent se relever en des projections de $M_\infty(A^{+})$, alors l'application exponentielle est triviale :

\[\exp(-2i\pi x )=\sum_{n=0}^\infty \frac{(-2i\pi x)^n}{n!}=1+(e^{-2i\pi}-1) x =1\]
car $x=x^2$.\\

\begin{dem}
Rappelons que $\delta$ est la composée donnée par :
\[
\begin{tikzcd}
K_0(A/J)  \arrow{r}{\delta}\arrow{d}{\beta_{A/J}}	& K_1(J) \arrow{d}{\theta_J} \\
K_1(S A/J) \arrow{r}{\partial} 	& K_0(SJ)
\end{tikzcd}
\]

Soient $p\in A/J$ et $x \in A$ un élément auto-adjoint tel que $\pi(x)=p$. Comme $e^{2i\pi tp}=1+(e^{2i\pi t}-1)p$, $f_x(t):=1+(e^{2i\pi t}-1)x$ relève $f_p(t)=e^{2i\pi tp}$. \\

Notons, dans un premier temps, que tout élément $y$ d'une $C^*$-algèbre tel que le spectre de $y^*y$ soit inclus dans $[0;1]$ produit un unitaire $\begin{pmatrix} y & \sqrt{1-yy^*}\\ -\sqrt{1-y^*y} & y^* \end{pmatrix}$. \\
On peut alors affirmer que 
\[w_{f_x}:=\begin{pmatrix} f_x & \sqrt{1-f_xf_x^*}\\ -\sqrt{1-f_x^*f_x} & f^*_x \end{pmatrix}\]
est un relevé unitaire de $\begin{pmatrix}f_p & 0\\ 0 & f_p^*\end{pmatrix}$, relevé qui nous donne l'indice de $[f_p]_1=\beta_{A/J}[p]_0$ :  
\[\partial [f_p]_1= [w_{f_x} p_n w_{f_x^*}]-[p_n].\]

Soit $g_x(t):=(1-t)1_{A^+}+t e^{2i \pi x}$ un chemin continu entre l'identité et $e^{2i\pi x}$. L'image de $e^{2i\pi x}$ par $\theta_J$ se calcule comme l'indice $[w_{g_x} p_n w_{g^*_x}]-[p_n]$. Montrer que $f_x$ et $g_x$ sont homotopes suffit donc à conclure. \\

Pour cela, remarquons que, $t$ variant de $0$ à $1$ et le spectre de $x$ étant inclu dans $[0,1]$ %VERIFICATION
, les éléments $f_x$ et $g_x$ ne dépendent que des valeurs des fonctions réelles
\[\begin{array}{rl}f(t,x) &=1+(e^{2i\pi t}-1)x \\
g(t,x) &=1-t+te^{2i\pi x}=f(x,t)\end{array}\]
au voisinage du bord du carré $\partial [0;1]\times [0;1]$, homéomorphe au cercle $\mathbb S^1$. Les classes d'homotopie de fonctions continues sur le cercle sont classifiées par leur nombre de tours, voir le livre d'Hatcher par exemple~\cite{Hatcher}, et on vérifie que $f$ et $g$ sont ainsi homotopes en tant que fonctions sur le cercle à valeurs dans le cercle, et donc que :
\[[w_{f_x} p_n w_{f_x^*}]=[w_{g_x} p_n w_{g^*_x}].\]
L'identité $\partial \circ \beta_B= \theta_J\circ \delta$ est démontrée, ce qui conclut.
\qed
\end{dem}


\subsection{Produits croisés de $C^*$-algèbres}

\subsubsection{Suite exacte de Pimsner-Voiculescu}

Voici le résultat central de ce rapport. Il a été demontré par Pimsner et Voiculescu en $1980$.~\cite{PV}
\begin{thm}[Pimsner-Voiculescu]\label{PV}
Soit $A$ une $C^*$-algèbre et $\alpha \in Aut(A)$. Il existe alors une suite exacte à six termes :\\
\[\begin{tikzcd}
 K_0(A) \arrow{r}{1-\alpha_*} & K_0(A)  \arrow{r}{\iota_*}  &    K_0(A\times_\alpha \Z)  \arrow{d}  \\
 K_1(A\times_\alpha \Z) \arrow{u} & K_1(A)  \arrow{l}{\iota_*} &    K_1(A) \arrow{l}{1-\alpha_*} 
\end{tikzcd}.\]
\end{thm}

La première chose que l'on peut, et que l'on va, dire à propos des produits croisés est que les générateurs de leurs groupes de $K$-théorie prennent une forme sympathique, qui va nous permettre de faire des calculs explicites dans la preuve de la suite de Pimsner-Voiculescu.\\

\begin{lem}\label{generateur}
Soit $B$ une $C^*$-algèbre unitale, $1_B\in A$ une sous-$C^*$-algèbre de $B$, et $u$ un unitaire de $B$ tels que $A$ et $u$ engendrent $B$ et $uAu^*=A$.\\
Alors $K_1(B)$ est engendré par les inversibles de la forme :
\[1_B\otimes 1_n +x(u^*\otimes 1_n)\quad , n\in\N,  x\in A\otimes \frak M_n.\]
De plus, si $B=A\times_\alpha \Z$, alors on peut se limiter aux classes d'unitaires de la forme :
\[1_B\otimes 1_n-F+Fx(u^*\otimes 1_n)F\quad F,x\in A\otimes\frak M_n\]
où $F$ désigne une projection auto-adjointe. 
\end{lem}

La remarque suivante est importante pour la preuve du lemme \ref{isom} : dans le cas $B=A\times_\alpha \Z$, les classes concernées sont stables par somme, donc tout élément de $K_1(B)$ est la différence de deux générateurs.\\

\begin{dem}
On note $\Gamma$ le sous-groupe de $K_1(B)$ engendré par les éléments de la forme $1_B\otimes 1_n+x(u^*\otimes 1_n)$.\\

Comme $u$ est unitaire, si $0\leq t\leq 1$, le spectre de $t(1_B+2u^*) + (1-t)u^*=t1_B + (1+t)u^*$ ne contient pas $0$, et on a donc un chemin continu d'inversibles entre $1_B+2u^*$ et $u^*$, d'où :
\begin{equation}[1_B+2u^*]_1=[u^*]_1.\label{homotop}\end{equation}
Les éléments $\sum_{s\leq j\leq t}a_j(u^j\otimes 1_p), a_j \in A\otimes \mathfrak M_p,s,t\in \Z$ sont denses dans $B\otimes \mathfrak M_p$. Il suffit donc de prouver notre assertion pour ce type d'éléments. Mais, d'après l'équation \ref{homotop}, $[u]_1\in \Gamma$, donc $s=0$ suffit.\\

Soit donc $y=\sum_{0\leq j\leq t}a_j(u^j\otimes 1_p)$ un inversible. On pose :
\[S_\epsilon = \begin{pmatrix}0 & 0 &  .&. &. &      \epsilon I \\
				-I & 0 &  & 0 & &		    \\
				 &    -I   &   &  &  &               \\
				&      0 &    &  ... &  &        \\
				&      &     &    &-I & 0
\end{pmatrix} \quad ,\forall \epsilon >0 \]
la matrice avec des $-I:=-1_B\otimes 1_p$ sur la sous-diagonale et un $\epsilon I$ dans le coin haut-droit, et :
\[T = \begin{pmatrix}         a_0 &  a_1 &  .   &.   &.     &      a_t \\
				0 &      I &       & 0 &      &		    \\
				   &        & I     &    &      &               \\
				   &    0  &       &    & ... &        \\
				   &        &       &    &      & I
\end{pmatrix}.  \]

Si l'on note $u_p=u\otimes 1_p$ et $y_k=\sum_{j=0}^{t-k} a_{j+k}(u^j\otimes 1_p)=y_{k+1}(u\otimes 1_p)+a_k$, on obtient l'identité :
\begin{align*}
S_0(u\otimes 1_n)+T   &= 
\begin{pmatrix}                        a_0 & a_1        &  .  &.   &.      &   a_t     \\
				-u_p & I            &      & 0 &        &             \\
				        &   -u_p    & I    &    &        &               \\
				        &   0          &     &     & ...  &               \\
				        &               &     &    &-u_p & I 
\end{pmatrix}
\\ &= \begin{pmatrix}                    I & y_1   &  . &  .  &    .    &      y_t      \\
				0 & I       &     & 0  &        &		  \\
				   &         & I   &     &        &                    \\
				   &      0 &     &     & ...  &                      \\
				   &          &     &    &       & I
\end{pmatrix} 
\begin{pmatrix}y & 0 &  .&. &. &      0 \\
				0 & I &  & 0 & &		    \\
				 &       & I   &  &  &               \\
				&      0 &    &  & ... &        \\
				&      &     &    & & I   
\end{pmatrix}
\begin{pmatrix}                             I &                      &.    &.         &      \\
				-u_p & I                  & 0    &         &		    \\
				         &   -u_p           &     &          &               \\
				         &      0             &     & ...   &        \\
				         &                    &      &-u_p & I
\end{pmatrix}.
\end{align*}

Les matrices de droite et de gauche de la dernière ligne sont unipotentes, leur classe dans $K_1(B)$ est donc nulle. Mais la classe de l'élément central est celle de $y$. Ajoutons à cela l'invariance de $K_1$ par homotopie, nous pouvons alors écrire, pour $\epsilon$ assez petit :
\begin{align*}[y]_1 & =[S_\epsilon(u\otimes 1_n)+T]_1
	\\		&= [u\otimes 1_n +S_\epsilon^{-1}T]_1 \quad \text{car } \ [S_\epsilon]_1=0	
	\\ 		&=[u\otimes 1_n]_1+[1_B\otimes 1_n +S_\epsilon^{-1}T(u^*\otimes 1_n)]_1
	\\		&=n[u]_1+[1_B\otimes 1_n +S_\epsilon^{-1}T(u^*\otimes 1_n)]_1
\end{align*}
et la première partie du lemme est démontrée.\\

Si maintenant $B$ est un produit croisé $A\times_\alpha \Z$, qui est engendré par $A$ et l'unitaire $u$ vérifiant 
\[\alpha^n(x)=u^n x u^{*n},\]
les générateurs peuvent être donnés plus explicitement.\\

On note $\beta : \mathbb S^1 \rightarrow  Aut(B)$ l'automorphisme donné par l'identité sur $A$ et $\beta(z)u=zu$ pour tout $z\in \mathbb S^1$. La $C^*$-algèbre $A$ est exactement l'algèbre des points fixes de $\beta$. Vu la première partie du lemme, il suffit de montrer que, pour tout générateur de la forme $1_B\otimes 1_n +x(u^*\otimes 1_n)$ où $x\in A\otimes\mathfrak M_n$, on peut trouver un unitaire $1_B\otimes 1_n-F+Fx(u^*\otimes 1_n)F$, où $F$ est un projecteur auto-adjoint.\\

Soit donc un inversible $y=1_B\otimes 1_n +x(u^*\otimes 1_n)$ : $-1$ n'est donc pas dans le spectre de $x(u^*\otimes 1_n)$. Comme $\beta$ est un automorphisme qui fixe $A$, il laisse invariant les spectres : le spectre de $x(u^*\otimes 1_n)$ est le même que celui de $(\beta(z)\otimes id_n)x(u^*\otimes 1_n)=\overline z x(u^*\otimes 1_n)$, pour tout $z\in \mathbb S^1$. Donc tout l'orbite de $-1$ sous l'action du cercle, c'est-à-dire le cercle $\mathbb S^1$, est dans la résolvante de $x(u^*\otimes 1_n)$.\\

On note $U_-$ et $U_+$ les composantes connexes respectivement bornée et non-bornée de $\C - \mathbb S^1$, et $T$ l'opérateur $x(u^*\otimes 1_n)$. Alors la fonction $e_-$ qui vaut $1$ sur $U_-$ et $0$ sur $U_-$ est holomorphe sur $\C - \mathbb S^1$, et on peut définir $P_-=e_-(T)$ par calcul holomorphe. De même pour $P_+=e_+(T)$. Ces deux éléments sont des projecteurs (non auto-adjoints) car $e_*(z)^2=e_*(z)$, et sont appelés les projecteurs spectraux de $x(u^*\otimes 1_n)$ associés à $U_-$ et $U_+$.\\

Comme $x(u^*\otimes 1_n)$ et  $\overline z x(u^*\otimes 1_n)$ ont même spectre, les projecteur spectraux associés à $U_+$ et $U_-$, que l'on note $P_{+/-}^z$, peuvent se définir avec les mêmes fonctions holomorphes $e_+$ et $e_-$. Comme elles sont invariantes par rotation,  on obtient que :
\begin{align*}
P_+^z=e_+(\overline z T ) & =\frac{1}{2i\pi} \int_\gamma e_+(w)(w-\overline z T)^{-1} dw\\
				&=\frac{z}{2i\pi} \int_\gamma e_+(w)(zw-\overline T)^{-1} dw \\
				&=z\frac{1}{2i\pi} \int_\gamma e_+(\overline z w)(w-\overline T)^{-1} \frac{dw}{z}\quad \text{en posant } w'=zw\\
				&= P_+.
\end{align*}
($\gamma$ est un lacet entourant le spectre de $T$.)\\
On a donc montré que :
\[(\beta(z)\otimes id_n)P_{+/-}=P_{+/-}.\] 

Ceci assure que $P_{+/-}\in A\otimes \mathfrak M_n$, ce qui n'était pas évident a priori, $x(u^*\otimes 1_n)$ étant élément de $B\otimes\mathfrak M_n$.\\

On définit un chemin d'éléments inversibles par  
\[y_\epsilon = \epsilon P_+ + x(u^*\otimes 1_n) P_+ + P_- + \epsilon x(u^*\otimes 1_n) P_-,\]
pour tout $ 0\leq \epsilon \leq 1$. Pour voir l'inversibilité, il suffit de remarquer que l'on a scindé $x(u^*\otimes 1_n)$ sur l'image de $P_+$ et $P_-$, et que l'on lui a ajouté un scalaire qui ne se trouve pas dans le spectre relatif à ce projecteur.
On a donc $[y_0]=[y_1]=[y]$. Comme $y_0= 1-P_+ + P_+ x(u^*\otimes 1_n)P_+$, il suffit de montrer que l'on peut remplacer $P_+$ par un projecteur auto-adjoint pour conclure.\\

Soit $F$ le projecteur orthogonal sur l'image de $P_+$. Alors il existe $T\in A\otimes \mathfrak M_n$ tel que :
\[P_+=F + FT(1_A\otimes 1_n-F)\] %Pourquoi ne pas prendre T=P_+ ??
(On peut prendre $T=P_+$ par exemple.) Comme $Fx(u^*\otimes 1_n)F=x(u^*\otimes 1_n)F$, on vérifie par un simple calcul que :
\[y_0=1_A\otimes 1_n -F+Fx(u^*\otimes 1_n)F+FS(1_A\otimes 1_n-F),\]
où $S=-T-x(u^*\otimes 1_n)FT$. L'inversibilité de $y_0$ donne celle de 
\[1_A\otimes 1_n -F + F x(u^*\otimes 1_n) F +\epsilon FS (1_A\otimes 1_n-F)\]
pour tout $\epsilon \in \C$, donc 
\[[y]=[1_A\otimes 1_n -F+ F x(u^*\otimes 1_n)F].\]
On aurait pu remplacer $y_0$ par $(y_0 y_0 ^*)^{-\frac{1}{2}} y_0$ qui est unitaire et homotope à $y_0$, ce qui conclut le lemme.

\qed
\end{dem}

%%%%%%%%%%%%%%%%%%%%%%%%%%%%%%%%%%%%%%%%%%%%%%%%%%%%%%%%%%%%%%%%%%%%%%%%%%%%%%%%%%%%%%%%%
\subsubsection{Extension de Toeplitz}


Soient $A$ et $C$ deux $C^*$-algèbres. \\
Par extension de $A$ par $C$, on entend un triplet $(B,\alpha,\beta)$ d'une $C^*$-algèbre et de deux morphismes telle que la suite :
\[\begin{tikzcd}[column sep =small] 
0 \arrow{r}  & A \arrow{r}{\alpha}  &  B \arrow{r}{\beta}  & C \arrow{r}& 0 \\ 
\end{tikzcd}\]
soit exacte.\\

Cette section présente la construction d'une extension de $A\otimes \K$ par $A\times_\alpha \Z$ qui sera utile dans la preuve de l'exactitude de la suite de PV : l'extension de Toeplitz. Dans tout le document $\Hil$ dénote un espace de Hilbert, $l_2$ par exemple, dont on fixe une base hilbertienne $(e_n)$, et $\B$ et $\K$ sont respectivement l'algèbre des opérateurs bornés et compacts sur $\Hil$. $\K$ est un idéal bilatère et :
\[\pi : \B \rightarrow \Cat\]
est la projection naturelle sur l'algèbre de Calkin. \\
$H^2(\mathbb S^1)$ désigne le sous-espace hilbertien de $L^2(\mathbb S ^1)$ engendré par les fonctions $z \mapsto z^n$ pour $n\geq 0$. Lorque l'on prendra $H^2(\mathbb S^1)$ pour $\Hil$, $e_n$ dénotera ces fonctions. Pour $f\in C(\mathbb S^1)$, on désigne par $T_f$ l'opérateur de $H^2(\mathbb S^1)$, appelé opérateur de Toeplitz associé à $f$, défini par $T_f(g)=\mathcal P(fg)$, où $\mathcal P$ est le projecteur orthogonal sur $H^2(\mathbb S^1)$. On appelle $f$ le symbole de $T_f$.\\

Soit $S\in \B$ l'opérateur de shift unilatéral, qui envoie $e_n$ sur $e_{n+1}$. On note $C^*(S)$ la $C^*$-algèbre unitale engendrée par $S$. On voit que $S^*$ envoie $e_1$ sur $0$ et $e_n$ sur $e_{n-1}$ lorsque $n\geq 2$. Si on note $E_{ij}(x)=\langle x,e_j\rangle e_i$, on a :

\[E_{ij} = S^{i-1}S^{*j-1}-S^{i}S^{*j}\in C^*(S)\]
$\K $ est donc un idéal bilatère de $C^*(S)$ et $P=1-SS^* = E_{11}$ est de rang $1$ donc compact.\\

\begin{lem}
L'application 
\[\tau \left\{\begin{array}{rcl} C(\mathbb S^1)  & \rightarrow & B(H^2(\mathbb S^1))/K(H^2(\mathbb S^1)) \\ 
f & \mapsto & \pi(T_f)\end{array}\right.\]
est un $*$-homomorphisme injectif.
\end{lem}

\begin{dem}
Si l'on confond $f\in C(\mathbb S^1)$ avec l'opérateur de multiplication associé dans $L^2(\mathbb S^1)$, alors $f\mathcal P -\mathcal Pf$ est un opérateur compact. En effet, si $f(z)=z$, on a un opérateur de rang $1$, et cette fonction génère $C(\mathbb S^1)$ par théorème de Stone-Weiertrass. \\

Ceci permet d'écrire la relation suivante :
\[T_f T_g =\mathcal P f\mathcal P g =\mathcal P (\mathcal P f + \text{compact}) g = \mathcal P f g + \text{compact}\]
Donc $T_f T_g = T_{fg} \ \text{mod} \  \K$, et comme $T_f^*=T_{\overline f}$, $\tau$ est bien un $*$-homorphisme.\\

Pour l'injectivité, observons le noyau de $\tau$. C'est un idéal bilatère de $C(\mathbb S^1)$, il existe donc un ouvert $X \subset \mathbb S^1$ tel que :
\[\text{ker}\ \tau =\{f \in \mathbb C(S^1) : f(z)=0,\forall z \in X\}\]
Mais si $f\in \text{ker}\ \tau$, alors $z\mapsto f(e^{i\theta}z)$ est aussi dans le noyau pour tout $\theta$, ce qui assure que $X=\mathbb S^1$ ou $\emptyset$. Mais comme $T_z$ n'est pas compact, $X=\mathbb S^1$ et l'injectivité est démontrée.

\qed
\end{dem}
% On en déduit que $S$ est essentiellement normal et 
%\[Spec(\pi S)\subset \mathbb S^1\]
%Montrons que c'est en fait une égalité. Par l'absurde, si l'inclusion est stricte, alors \textbf{A FINIR } \\ %%%%afinir 
Comme $C(\mathbb S^1)$ est généré par $z$, qui s'envoit sur $S$ par $T$, l'image de $T_{.}$ est $C^*(S)$. La remarque précédente permet d'affirmer que $C^*(S)/\K$ est $*$-isomorphe à l'algèbre des fonctions continues sur le tore $C(\mathbb S^1)$, et l'image de $S$ est la fonction identité sur $\mathbb S^1$, noté $z$. On a donc une extension, écrite sous la forme d'une suite exacte :
\[\begin{tikzcd}[column sep =small] 0 \arrow{r} & \K \arrow{r} & C^*(S) \arrow{r} & C(\mathbb S ^1) \arrow{r} & 0 \end{tikzcd}\]

\begin{definition}
On définit l'algèbre de Toeplitz $\mathcal T$ associée à la paire $(A,\alpha)$ comme la $C^*$-sous-algèbre de $(A\times_\alpha \Z)\otimes C^*(S)$ engendré par $A\otimes I$ et $u\otimes S$. 
\end{definition}
Rappelons que l'on voit $A$ comme une sous-$*$-algèbre de $A\times_\alpha \Z$, et que l'on note $u$ l'unitaire qui rend intérieure l'action de $\alpha$ :
\[\forall a\in A, n\in \Z, \quad \alpha(n)a=u^{*n} a u^n\]

Observons maintenant $A\times_\alpha \Z$, dont on va montrer qu'elle se réalise comme un quotient de $\mathcal T$ par un idéal bilatère fermé. Soit donc $J$ l'idéal bilatère fermé engendré par la projection $1\otimes P$. La première chose à remarquer, c'est que l'on a un $*$-morphisme :
\[\phi \left\{\begin{array}{ccc}\K & \rightarrow & \mathcal T\\
				e_{ij} & \rightarrow & S^i P S^{*j}\end{array}\right. .\]

Il est ici défini sur le système d'unités de $\K$, 
\[e_{ij}(x)=\langle x,e_i\rangle e_j\]
ce qui permet facilement de l'étendre à $\K$ entier. \\

L'identité suivante permet d'étendre $\phi$ à $A\otimes \K$ :
\[(u\otimes S)^i (a\otimes P) (u\otimes S)^{*j}=(u^i a u^{*j}) \otimes \phi (e_{ij})\]
définit l'extension $\psi$ de $\phi$ à $A\otimes \K$. Alors $\psi(A\otimes \K)=J\subset \mathcal T$ et $\psi $ est injective.\\

Pimsner et Voiculescu montrent ~\cite{PV} que :
\begin{equation}
\text{im} \ \psi = (A\times _ \alpha \Z )\otimes \phi(\K) \ \cap \ \mathcal T
\label{image}
\end{equation}

En effet, soit $y\in (A\times _ \alpha \Z )\otimes \phi(\K) \ \cap \ \mathcal T$. Comme $y$ est dans $(A\times _ \alpha \Z )\otimes \phi(\K) $,
\[J \ni (1\otimes E_n) y (1\otimes E_n) \underset{n \rightarrow \infty}{\longrightarrow} y \]
où $E_n = 1\otimes\phi(e_{00}+e_{11}+...+e_{nn})=\psi (1\otimes(e_{00}+e_{11}+...+e_{nn}) )\in J$ ( on utilise une unité approchée de $\K$). $J$ étant un idéal fermé, on en déduit que $y\in J$. L'inclusion inverse est directe.\\

Les $C^*$-algèbres $\K$, $C^*(S)$ et $C(\mathbb S^1)$ sont nucléaires car commutative pour $C(\mathbb S^1)$ ou limite inductive de $C^*$-algèbres finie-dimensionnelles pour $\K$. %pour C*(S) ?
Ceci assure qu'il n'y a qu'une seule norme de  $C^*$-algèbre sur leur produit tensoriel avec $A\times _\alpha \Z$. De plus, avec le théorème $T.2.6.26$ de l'appendice T du livre de Wegge-Olsen ~\cite{WeggeOlsen}, on a, sans ambiguité, une suite exacte :
\[\begin{tikzcd}[column sep=small]
0 \arrow{r} & (A\times _\alpha \Z)\otimes \K \arrow{r} &  (A\times _\alpha \Z)\otimes C^*(S) \arrow{r}  &  (A\times _\alpha \Z) \otimes C(\mathbb S^1) \arrow{r} & 0 \\ 
\end{tikzcd}\] 
Une autre méthode pour l'obtenir est d'utiliser la proposition \ref{CPexactness}. En effet, $f\mapsto T_f$ est une section complètement positive de la première suite exacte :
\[\sum_{i,j} b_i T_{f_i f_j^*} b_j^*= (\sum_i b_i T_{f_i})(\sum_i b_i T_{f_i})^*.\]

Cette suite exacte et l'identité \ref{image} permet d'identifier $\mathcal T / J$ à la $C^*$-algèbre engendrée par $A\otimes 1$ et $u\otimes z$ où $z$ est l'inclusion $\mathbb S^1 \rightarrow \C$. Cette dernière étant $*$-isomorphe à $A\times_\alpha \Z$, on en déduit la suite exacte :

\[\begin{tikzcd}[column sep=small]
0 \arrow{r} &  A\otimes \K \arrow{r}{\psi} &  \mathcal T \arrow{r}{\pi}  &  (A\times _\alpha \Z) \arrow{r} & 0. \\ 
\end{tikzcd}\] 
C'est l'extension de Toeplitz associée à $(A,\alpha)$.
%%%%%%%%%%%%%%%%%%%%%%%%%%%%%%%%%%%%%%%%%%%%%%%%%%%%%%%%%%%%%%%%%%%%%%%%%%%%%%%%%%%%%%%%%
