\section{Annexes}
\subsection{Produits tensoriels de $C^*$-algèbres}

\begin{lem}
Soient $E$ et $F$ deux espaces vectoriels sur un corps $K$. S'il existe deux $K$-espaces vectoriels $V_1$ et $V_2$ munis d'applications bilinéaires $\pi_j : E\times F \rightarrow V_j$ telles que, pour tout espace vectoriel $W$, toute application bilinéaire $E\times F \rightarrow W$,  se factorise uniquement via $\pi_1$ et $\pi_2$, alors $V_1$ et $V_2$ sont isomorphes en tant que $K$-espaces vectoriels.
\end{lem}


\[\begin{tikzcd}
V_1 \arrow{dr}& \arrow{l}{\pi_1}	E\times F \arrow{d}\arrow{r}{\pi_2}	& \arrow{dl}V_2 \\
			 & 		W		&	\\
\end{tikzcd}\]

\begin{dem}
En factorisant $\pi_j$ via $\pi_j$, il existe deux uniques applications linéaires $\phi_1 : V_2\rightarrow V_1$ et $\phi_2 : V_1\rightarrow V_2$ telles que :
\[\begin{array}{c}\pi_1=\phi_1\circ \pi_2 \\ \pi_2=\phi_2\circ \pi_1.\end{array}\]

Montrons que ces deux applications sont inverses. Comme :
\[\phi_1\circ \phi_2 \circ \pi_1 = \phi_1\circ \pi_2 =\pi_1 ,\]
$\phi_1\circ \phi_2 = id_{V_1}$ par unicité de la factorisation de $\pi_1$ via $\pi_1$.
Symétriquement, on démontre que : $\phi_2\circ \phi_1 = id_{V_2}$, et le résultat est démontré.\\
\qed
\end{dem}

\subsection{Produits croisés par un groupe discret : version bimodule hilbertien}

Soit $A$ une $C^*$-algèbre et $\Gamma$ un groupe discret. On se donne de plus une action par automorphisme $\alpha : \Gamma \rightarrow Aut(A)$. On peut alors munir l'espace $C_c(\Gamma,A)$ des fonctions à support fini d'un produit de convolution tordu par $\alpha$ :
\[f*_\alpha g = \sum_{s,t \in \Gamma} f(s)\alpha_s(g(t))st.\]

Soit $\lambda_{\Gamma,A}$ la représentation régulière gauche de $C_c(\Gamma,A)$ sur $l^2(\Gamma,A)=\{\eta : \Gamma \rightarrow A : \sum_s \eta^*(s)\eta(s) <\infty\}$ :
\[(\lambda_{\Gamma,A}(f)\eta)(\gamma) = \sum_{s\in \Gamma} \alpha_{\gamma^{-1}}(f(s))\eta(\gamma^{-1}s)\]
pour tous $f\in C_c(\Gamma,A)$,$\eta \in l^2(\Gamma,A)$ et $\gamma \in \Gamma$. \\

Le produit croisé réduit de $A$ par $\Gamma$, noté $A\times_\alpha \Gamma$, est défini comme la fermeture pour la norme d'opérateur de $\lambda_{\Gamma,A}(C_c(\Gamma,A))$ dans $B(l^2(\Gamma,A))$.\\

Les actions habituelles de $A$ et de $\Gamma$ sur $l^2(\Gamma,A)$ sont combinées.
\[(\pi(a)\eta)(s) = \alpha_{s^{-1}}(a)\eta(s)\]
\[(\lambda(\gamma)\eta)(s)=\eta(\gamma^{-1}s)\]
 On parle pour la paire $(\lambda, \pi)$ de représentation covariante du système $\{A,\Gamma,\alpha\}$, car la relation :
\[\lambda(\gamma)\pi(a)\lambda(\gamma^{-1})=\pi(\alpha_\gamma(a))\]
est vérifiée.\\