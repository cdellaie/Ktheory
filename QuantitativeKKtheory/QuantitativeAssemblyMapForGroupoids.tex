\section{English Notes}

In a serie of papers, H. Oyono-Oyono and G. Yu have defined a controlled version of operator $K$-theory, ~\cite{OY1}~\cite{OY2}~\cite{OY3}, that allows them to define a quantitative and local assembly map, and a related Baum-Connes Conjecture. The aim of this work is to extend their work to the realm of groupoids. In a second part, we shall see how this setting can help us understand better the relation between the coarse-Baum-Connes conjecture and the Baum-Connes Conjecture with coefficient for groupoids. Indeed, in ~\cite{TuBC} and ~\cite{SkTuYu}, G. Skandalis, J-L.Tu and G. Yu proved that there is a commutative diagramm
\[\begin{tikzcd}
K_*(X) \arrow{r}{A}\arrow{d}{\simeq} & K_*(C^*X)\arrow{d}{\simeq} \\
K_*(\mathcal E \Gamma)\arrow{r}{\mu_r} & K_*(C_r \Gamma)
\end{tikzcd}\]
where :
\begin{itemize}
\item the left sides are $K$-homology of certain spaces, the left sides are $K$-theory of certain $C^*$-algebras,
\item the first line is the coarse assembly map associated to a coarse space $X$, the second being the assembly map for groupoids associated to the coarse groupoid of $X$, namely $\Gamma = G(X)$,
\item the vertical arrows are isomorphisms on the level of $KK$-groups. The left one will be studied later, the right one derives from an isomomorphism on the level of the $C^*$-algebras. Indeed, it has been shown that the Roe algebra is $*$-isomorphic to the crossed product of $l^\infty (X)$ by $\Gamma$ : $C^*(X) \simeq l^\infty(X)\times \Gamma$.\\
\end{itemize}

We shall see that this relation is already true "locally" and factorises through quantitative $K$-theory. Locally here means that we can factorize the assembly map through $KK(C_0(P_d(X),B)$. Indeed, the $K$-homology can be expressed as an inductive limit 
\[K_*(X,B)=\varinjlim_{d\rightarrow \infty}  KK(C_0(P_d(X)), B ) \quad \text{and}\quad\] 
hence the local. \\

The local quantitative assembly maps defined in \cite{OY2} are of the form
\[KK^F(C_0(P_d(F)), B  ) \rightarrow K^{\epsilon, r}_*(B\times_r \Gamma)\]
where $F$ is a finite group. \\

We will be using the bivariant functor $KK^\Gamma$ introduced by P-Y. Le Gall in his thesis ~\cite{LeGall}, which is a generalization of Kasparov's bifunctor for the case where $\Gamma$ is a Hausdorff locally compact groupoid with Haar system. As for the $KK$-theory for Banach algebras introduced by V. Lafforgue~\cite{Lafforgue}, there is no (not yet?) a Kasparox product in quantitative $K$-theory. The crucial point to define an assembly map is the existence of a morphism 
\[\hat J : KK^{\Gamma}(A,B)\rightarrow Hom^*(\hat K_*(A),\hat K_*(B))\]
which allows us, for every element $x\in K_*(A)$, to consider the related index
\[Ind_x \left\{ \begin{array}{ccc} KK_*(A,B) & \rightarrow  & K_*(B)\\
	z & \mapsto & \hat J(z)(x)
\end{array}\right.\]
to construct $\hat J$, we will mimic the construction for $\mathcal J$ in ~\cite{OY2}, which gives the right morphism when considering finite groups. The starting point is the following 

\begin{lem}
Let $\Gamma$ be a Hausdorff locally compact groupoid with Haar system. Then, if $A$ is a $\Gamma-C^*$-algebra, forming the reduced (and maximal) crossed product $A\times_r \Gamma$ is functorial in $A$. Moreover, if $\Gamma$ is \textit{étale}, then it preserves short semi-split filtered exact sequences of $\Gamma-C^*$-algebras, meaning that if 
\[\begin{tikzcd}[column sep = small] 0\arrow{r} & A' \arrow{r} & A\arrow{r} & A'' \arrow{r} & 0\end{tikzcd}\]
is a semi-split filtered exact sequence of $\Gamma-C^*$-algebras, then 
\[\begin{tikzcd}[column sep = small] 0\arrow{r} & A' \times_r \Gamma\arrow{r} & A\times_r \Gamma\arrow{r} & A'' \times_r \Gamma\arrow{r} & 0\end{tikzcd}\]
is too.
\end{lem}

\begin{dem}
Let 
\[\begin{tikzcd}[column sep = small] 0\arrow{r} & A' \arrow{r}{\Psi} & A\arrow{r}{\Phi} & A'' \arrow{r} & 0\end{tikzcd}\]
be a semi-split filtered exact sequence of $\Gamma-C^*$-algebras.\\

Let $f\in C_c(\Gamma, A)$ such that $\Phi_\Gamma(f)=0$. For every $\gamma \in \Gamma$, there exists a unique $g(\gamma)\in A'$ such that $\Psi(g(\gamma))=f(\gamma)$. But, as $\Psi$ and $\Phi$ commute with the action of $\Gamma$, we have $g(\gamma'^{-1}\gamma)=\gamma' . g(\gamma)$, so that $g$ is continuous.\\

Let $a\in A\times \Gamma$ such that $\Phi_\Gamma(a)=0$. We can approximate $a$ by a sequence of $f_n\in C_c(\Gamma,A)$. But, with the preceeding work on compactly supported continuous functions, there exist a sequence $g_n\in C_c(\Gamma, A')$ such that :
\[f_n = \Psi(g_n)+\sigma\circ \Phi(f_n).\]
Moreover, $\text{Im}\ \Psi$ is a closed ideal in $A$, and $\lim ||f_n-\Psi(g_n)||=0$ so that $a\in \text{Im} \ \Psi$ : the sequence is exact in the middle.
\qed
\end{dem}

\subsection{}
Let $X$ be a discrete metric space with bounded geometry and $\Gamma$ its coarse groupoid. $\C[X]$ is the algebra of locally compacts operators with finite propagation on $l^2(X)\otimes H$. We denote by $\tilde A $ the $C(\Gamma)$-algebra $C_0(P_d(\Gamma))$, and by $A$ the $C^*$-algebra $C_0(P_d(X))$. As $P_d(\Gamma)$ is equivariantly homeomorphic to $\beta X \times P_d(X)$, the fiber $\tilde A_x$ is isomorphic to $A$ as a $C^*$-algebra.\\

\begin{lem}
Let $B$ be a $C^*$-algebra and $x\in X$. The morphism of groupoids $\iota : \{e_x\}\rightarrow \Gamma$ induces an isomorphism in $KK$-theory 
\[\begin{tikzcd}[column sep = small] KK_*^\Gamma(\tilde A, l^\infty(X, B))\arrow{r}{\simeq} & KK_*(A, B)\end{tikzcd}.\]
\end{lem}

\begin{dem}
Let us first check that the fiber over $x\in X$ of $\tilde B :=l^\infty(X,B)$ is $B$ : easily, if $I_x$ denotes the kernel of the evaluation map $\tilde B \rightarrow B$ at $x$, $\tilde B_x = \tilde B / I_x \tilde B\simeq B $. Moreover, as $\tilde A_x \simeq A$, functoriality assures that $\iota$ induces $\iota^* : KK_*^\Gamma(\tilde A, l^\infty(X, B))\rightarrow KK_*(A, B)$.\\

We can now construct an explicit $R$ inverse for $\iota^*$. Let $z=(\hat H_{\tilde B},\Phi, T)\in \mathbb E^\Gamma(\tilde A,\tilde B)$ be a $K$-cycle. By stabilization theorem, we can suppose that the Hilbert $\tilde B$ module is in canonical form, that is the standard separable Hilbert module over $\tilde B$ with usual grading. We will also suppose that $T$ is $\Gamma$-equivariant, which is always true up to compact perturbation. The image of $z$ under $R$ is defined to be its restriction to $x$, $(\hat H_B,\Phi_x,T_x)$. This definition gives immediatly $R\circ \iota^* = 1$. Now let $V : s^* \hat H_{\tilde B}\rightarrow \hat H_{\tilde B}$ be the unitary implementing the action of $\Gamma$ on $\hat H_{\tilde B}$. The equivariance of $T$ can be written as 
\[V_\gamma T_{s(\gamma)} V_\gamma^* = T_{r(\gamma)}\quad,\forall \gamma \in \Gamma. \]
If $T' = \iota^*\circ R(T)$, it is the constant operator with fiber $T_x\in \mathcal L_B(\hat H_B)$ on the trivial Hilbert module $\bigoplus_{y\in X}(H_{\tilde B})_x $. Define $V=\bigoplus_{\gamma \in \Gamma} V_\gamma$. It is unitary and intertwines $T$ and $T'$, i.e. the cycles are homotopic and $\iota^*\circ R = 1$.

\qed
\end{dem}

Another result of interest for our study is the expression of the Roe algebras of our discrete space $X$ as crossed products. It was first proven by G. Yu in the case of $X$ being a finitely generated group with the word metric, the general version can be found in the article of G. Skandalis, J-L. Tu and G. Yu. \cite{SkTuYu}

\begin{lem}
Let $B$ be a coefficient $C^*$-algebra. The uniform Roe algebra of $X$  and the Roe algebra of $X$ are $*$-isomorphic to crossed-product algebras by $\Gamma$, respectively : 
\[\begin{array}{ll} C_u^* (X,B) \simeq l^\infty(X,B) \rtimes_r \Gamma \\ C^* (X,B) \simeq l^\infty(X,\K\otimes B) \rtimes_r \Gamma \end{array}.\]
\end{lem}

\begin{dem}
If $T\in \C[X]$. For $x,y\in X$, $T_{xy}\in \K$ is the compact operator defined by Riesz representation theorem as 
\[\langle T(\delta_x \otimes \xi),\delta_y \otimes \eta\rangle =\langle T_{xy}\xi,\eta\rangle \ ,\xi, \eta\in H.\]

Let us recall how one construct the left regular representation of $\Gamma$. For $x\in \Gamma^{(0)}=\beta X$, $\Gamma^x$ has a natural action on $l^2(\Gamma^x)$ :
\[(\lambda_x(\gamma)\eta)(\gamma')=\eta(\gamma^{-1}\gamma')\quad,\gamma,\gamma'\in \Gamma^x,\eta\in l^2(\Gamma^x) .\] 

To any couple $(x,y)\in \beta X\times \beta X$, there exists a unique $\gamma_x^y\in \Gamma$ such that $s(\gamma)=x$ and $r(\gamma)=y$. The map
\[T\rightarrow \sum_{x,y} T_{xy} \lambda_x(\gamma_x^y)\]
establishes an $*$-isomorphism between $\C[X]$ and the algebra of finitely supported functions from $\Gamma$ to $l^\infty(X,\K)$, its inverse being the $*$-morphism that associates to a function $a$ the multiplication operator : 
\[\big(a(\gamma)(\eta\otimes \xi)\big) (x)= a(\gamma)(x)\eta(x)\xi , \eta\in l^2(X),\xi\in H. \]
That is it of finite propagation and locally compact follows from the finiteness of the support. As this two rules are continuous and their domain are respectivly dense in the Roe algebra and in the crossed product, the morphism extends to the $*$-isomorphism we were looking for.
\qed
\end{dem}

This $*$-isomorphism gives an isomorphism at the level of $K$-theory 
\[ K_*(C^* (X,B))\rightarrow K_*(l^\infty(X,\K\otimes B)\rtimes_r \Gamma).\]
These lemmas allows us to state a reffinement of a result of G. Skandalis, J-L.Tu and G. Yu \cite{SkTuYu}, namely that the two isomorphisms described above intertwine the (local quantitative ?) assembly maps for the groupoid $\Gamma$ with coefficients in $l^\infty(X)$ with the coarse assembly map of $X$. (If local and quantitative :) This result at the level of controlled $K$-theory on the right side entails that of \cite{SkTuYu}

\begin{thm}
The following diagram
\[\begin{tikzcd}[column sep = small]
A : KK_*(A,B)\arrow{r}\arrow{d}{\simeq} & K_*(C^*(X,B)) \arrow{d}{\simeq}\\
\mu_{r} : KK^\Gamma \arrow{r} & K_*(l^\infty(X,\K\otimes B)\rtimes_r \Gamma)
\end{tikzcd}\]
commutes, and the vertical arrows are isomorphisms.
\end{thm}



















