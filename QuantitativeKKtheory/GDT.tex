\section{Séminaire $KK$-théorie et groupe quantique, exposés du 17 et 24 Octobre 2014}

Ce court rapport a pour but de présenter les bases de la $K$-théorie des $C^*$-algèbres. Nous irons des définitions à la suite exacte à $6$ termes. 

\subsection{Motivations}

Comme Lorenzo Pittau vous a tout raconté sur $K_0$ la semaine dernière, je vais me permettre d'aller plus vite sur les détails techniques. \\

Le foncteur $K$ a été introduit par Grothendieck dans sa démonstration du théorème de Riemann-Roch à la fin des années $50$, voir l'article de Borel et Serre~\cite{BorelSerre}. Ces travaux ont inspiré Atiyah et Singer dans la démonstration de leur fameux théorème de l'indice. \\

Voici une version très simplifiée d'un théorème de l'indice trouvée dans un article de Jean Bellissard~\cite{Bellissard}, que je trouve très pédagogique. Soit $f\in C(\mathbb S^1,\C^\times)$. Cette fonction définit un lacet $\gamma$, donc un élément du premier groupe d'homologie du cercle $[\gamma]=f^*(\mathbb S^1)\in H_1(\mathbb S^1)=\Z$, dont la classe est représentée via cet isomorphisme par le degré de $f$. Si l'on suppose que $f$ est une fonction holomorphe, on peut même calculer ce degré : 
\[\text{deg}(f)=\frac{1}{2i\pi}\int_{f(\mathbb S^1)} f(z)\frac{dz}{-z}= -\langle [w],[f]\rangle_{H^1\times H_1},\] 
où $w=\frac{dz}{2i\pi z}$ est une forme fermée qui définit une classe de cohomologie de $\mathbb S^1$, et le crochet est la dualité usuelle entre cohomologie et homologie.\\

Cette formule a aussi une interprétation opératorielle. Soit $\mathcal H$ l'espace des fonctions holomorphes sur le disque unité ouvert $\{z\in \C : |z|<1\}$ possédant un prolongement de carré intégrable sur le bord du disque $\mathbb S^1$. Cet espace $\mathcal H$, appelé espace de Hardy, est un sous-espace fermé de l'espace de Hilbert $L^2(\mathbb S^1)$, et on note $\mathcal P$ la projection orthogonale sur $\mathcal H$. On peut montrer que l'espace de Hardy est celui des fonctions dont les coefficients de Fourier strictement négatifs sont nuls.\\

A toute fonction continue sur le cercle $f\in C(\mathbb S^1)$, on associe un opérateur dit de Toeplitz $T_f\in \mathcal L(\mathcal H)$
\[T_f : g \mapsto \mathcal P fg. \]
Cet opérateur induit un $*$-morphisme de $C(\mathbb S^1)\rightarrow \B/\K$ à valeur dans l'algèbre de Calkin (les opérateurs bronés quotientés par l'idéal des opérateurs compacts), et donc $T_f T_{\overline f}= T_{\overline f} T_f \ \text{mod}\ \K  = 1 \ \text{mod}\ \K$. On a donc un opérateur de Fredholm, dont on sait que le noyau et le conoyau sont de dimension finie : il a un indice
\[\text{Ind}(T_f)=\text{dim}(Ker(T))- \text{dim} (coker(T))\in \Z.\]
Le "théorème-$0$" de l'indice affirme que cette indice est précisément le degré de $f$, ce que l'on peut réécrire comme 
\[\text{Ind}(T_f)= \langle [w],f^*(\mathbb S^1)\rangle_{H^1\times H_1}.\]

Pour le montrer, remarquez d'abord que l'indice est invariant par perturbation compacte et par homotopie. Il suffit alors de le montrer pour les fonctions $z^n$, or $T_{z^n}=S^n$, où $S$ est le shift unilatéral, qui est injectif. Le noyau de $S^{*n}$ étant de dimension $n$, $Ind(T_{z^n})=-n$.\qed \\

Atiyah et Singer ont profondément généralisé ce type de résultat au cadre des fibrés vectoriels. L'exemple typique est celui d'un opérateur de Dirac  sur un variété munie d'une structure $\text{spin}^c$, dont on peut montrer qu'il est inversible modulo les opérateurs pseudo-différentiels réguliers. On verra comment calculer des indices associés à toute extension de $C^*$-algèbre, et ici l'extension pertinente est celle des opérateurs pseudo-différentiels. On a alors un indice associé à cet opérateur de Dirac, qui est un entier. Dans certains cas, lorsque votre variété est un espace homogène par exemple, on peut relever notre opérateur de Dirac, sur le revêtement universel dans notre exemple. Le receptacle pour l'indice de cet opérateur est alors la $K$-théorie : l'indice n'est plus un entier mais un élément d'un certain groupe de $K$-théorie.\\

Il se trouve que la $K$-théorie se généralise bien au cadre non-commutatif. Pour cela, rappelons le théorème de Serre-Swan :

\begin{thm}
Soit $X$ un espace topologique compact et $\Phi$ le foncteur qui va de la catégorie des fibrés vectoriels complexes de base $X$ dans celle des $C(X)$-modules projectifs de type fini qui, à un fibré \begin{tikzcd}[row sep = small]E \arrow{d}{\pi}\\ X\end{tikzcd} associe l'espace des sections continues $\Phi : E\mapsto \Gamma(E)=\{s : X\rightarrow E / \pi\circ s = id\}$.\\
Alors $\Phi$ réalise une équivalence de catégories.
\end{thm}

Ce théorème assure que se donner un fibré, c'est se donner un $C(X)$-module projectif de type fini. C'est ainsi que les algébristes définissent le premier groupe de $K$-théorie d'un anneau $A$ comme le groupe de Grothendieck du monoïde des classes d'équivalence des $A$-modules projectifs de type fini. C'est d'ailleurs une définition que l'on peut prendre pour les $C^*$-algèbres, si l'on utilise la théorie des $C^*$-modules hilbertiens (voir les prochains exposés !). Pour mémoire, rappelons qu'un $A$-module $P$ est dit projectif si pour tout $A$-modules $N$ et $M$ et tout morphisme $f : N\rightarrow P$ et tout épimorphisme $g: M\rightarrow P$, il existe une unique flèche $h : N\rightarrow M$ tel que le diagramme suivant commute :
\[\begin{tikzcd}  . & M \arrow[two heads]{d}{g} \\
	N \arrow[dotted]{ur}{h}\arrow{r}{f} & P
\end{tikzcd}\]

Que se passe t'il pour un $A$-module projectif de type fini $\mathcal E $? On a un morphisme surjectif $g : A^n \rightarrow \mathcal E$, et on peut relever l'identité grâce à la propriété universelle
\[\begin{tikzcd}
 . 		& A^n \arrow[two heads]{d}{g} \\
	\mathcal E \arrow[dotted]{ur}{f}\arrow{r}{id_{\mathcal E}} & \mathcal E
\end{tikzcd}\]
Il existe donc $f : \mathcal E \rightarrow A^n$ telle que $g\circ f = id_{\mathcal E}$. Alors $p = f\circ g$ est un projecteur, et on a un isomorphisme $ \mathcal E \simeq pA^n$. Ainsi, un module projectif de type fini est donné par un projecteur de $\mathfrak M_n (A)$.

\subsection{Propriétés de $K_0$}

On rappelle la description standard du premier groupe de $K$-théorie. 
\begin{definition}
Soit $p$ et $q$ deux projecteurs dans une $C^*$-algèbre $A$. On définit trois relations d'équivalences :\\
$p\sim q$ s'il existe une isométrie partielle $u$ de $A$ telle que $p=u^*u $ et $q=uu^*$. ( équivalence de Murray-Von Neumann)\\
$p\sim_u q$ s'il existe un unitaire $u$ de $A^+$ tel que $p=uqu^*$. (Similitude)\\
$p\sim_h q$ s'il existe un chemin continu en norme de projections de $p$ à $q$.(Homotopie)\\
\end{definition}

\begin{prop}[Description standard]
Soit $A$ une $C^*$-algèbre. Tout élément $x$ de $K_0(A)$ peut être représenté par une différence
\[x=[p]-[p_n]\ \text{tels que } p = p_n \ \text{mod}\ \mathfrak M_k(A)\]
où $k$ et $n$ sont deux entiers tels que $k\geq n$, $p\in\mathfrak M_k(A^+)$ est un projecteur et $p_n$ est l'élément de $\mathfrak M_k(A^+)$ avec des $0$ partout et des $1$ sur les $n$ premiers emplacements diagonaux.
\end{prop}

\subsubsection{Equivalence entre les relations d'équivalences sur les projecteurs modulo stabilisation}
\begin{lem}\label{conj}
Soit $A$ une $C^*$-algèbre unitale et $p$ et $q$ deux projecteurs de $A$ tels que $||p-q||<1$. Alors il existe un unitaire $u\in A$ vérifiant
\[p=uqu^*.\]
\end{lem}

\begin{dem}
Pour tout projecteur $\in A$, on pose $s_p = 2p-1$ : c'est une symmétrie. De plus 
\[s_p-s_q = 2(p-q),\]
ce qui assure que si $||p-q||<1$, $v:= \frac{s_p s_q+1}{2}$ est inversible. Mais $pv=pq$ et donc par symmétrie $qv^* = qp$ d'où $pv=vq$. La décomposition polaire de $v$ assure alors que l'unitaire $v |v|^{-1}$ entrelace $p$ et $q$.
\qed\\
\end{dem}
Une remarque : on vient finalement de montrer qu'il existe une application continue sur un voisinage de $p$ (la boule de centre $p$ et de rayon $||2p-1||$ par exemple,
\[\Psi\left\{\begin{array}{rcl} B_p &\rightarrow & A^{-1} \\
 q & \mapsto &  \frac{s_p s_q +1}{2}|\frac{s_p s_q +1}{2}|^{-1}		
\end{array}\right.\]
telle que $\Psi(q) p \Psi(q)^* = q$. \\

\begin{lem} Soit $v$ une isométrie partielle d'une $C^*$-algèbre unitale $A$ i.e. $v^* v $ et $vv^*$ sont des projecteurs. Alors 
\[\begin{pmatrix}v & 1-vv^* \\ v^*v-1 & v^*\end{pmatrix}\]
est un unitaire de $\mathfrak M_2(A)$ homotope à l'identité $1_{A}\otimes I_2 $.
\end{lem}
\begin{dem}Un calcul suffit à montrer l'unitarité. Le chemin
\[t\mapsto \cos(\frac{\pi}{2} t)\begin{pmatrix}v & 0 \\ 0 & v^*\end{pmatrix} + (1-\sin(\frac{\pi}{2}t))\begin{pmatrix}0 & -vv^* \\ v^*v & 0\end{pmatrix}+\begin{pmatrix} 0 & -1 \\ 1 & 0 \end{pmatrix}\]
connecte la matrice en question à $\begin{pmatrix} 0 & -1 \\ 1 & 0 \end{pmatrix}$ dans les unitaires.\qed
\end{dem}

\begin{prop}
Les relations d'équivalence sur les projecteurs sont ordonnées comme suit : $\sim_h \Rightarrow \sim_u \Rightarrow \sim$. De plus,
\[\text{si }\ p\sim q \ \text{alors }\ \begin{pmatrix}p & 0 \\ 0 & 0 \end{pmatrix}\sim_u \begin{pmatrix}q & 0 \\ 0 & 0 \end{pmatrix} \]
\[\text{et si }\ p\sim_u q \ \text{alors }\ \begin{pmatrix}p & 0 \\ 0 & 0 \end{pmatrix}\sim_h \begin{pmatrix}q & 0 \\ 0 & 0 \end{pmatrix}. \]
\end{prop}
\begin{dem}
Si $p\sim_h q$, il existe un chemin continu de projecteurs $[0,1]\rightarrow P(A);t\mapsto p_t$. En découpant l'intervalle $[0,1]$ en segment assez petits, le lemme \ref{conj} donne l'existence d'un chemin d'unitaires $t\mapsto u_t$ tel que $t\mapsto u_t p u_t^*$ soit un chemin continu de projecteurs de $p$ à $q$. En particulier, $p\sim_u q$.\\

Si $p\sim_u q$, alors il existe un unitaire $u$ vérifiant
\[p= u q u^* = (u q)(u q)^*\]
comme $uq$ est une isométrie partielle, $p\sim q$.\\

Si $p\sim q$. Soit $v$ une isométrie partielle de $A$ telle que $p=v^*v$ et $q = vv^*$. Alors 
\[w=\begin{pmatrix} v & 1-vv^*\\1-v^* v  & v^*\end{pmatrix}\]
est une isométrie de $\mathfrak M_2 (A)$. De plus
\[w\begin{pmatrix}p & 0 \\ 0 & 0 \end{pmatrix}w^*=\begin{pmatrix}q & 0 \\ 0 & 0 \end{pmatrix}.\]

Si $p\sim_u q$, soit $u$ un unitaire qui entrelace $p$ et $q$. Alors 
\[\begin{pmatrix}u & 0 \\ 0 & u^* \end{pmatrix}\]
est un unitaire homotope à $1\otimes I_2$, et un tel chemin continu d'unitaire $t\mapsto w_t$ assure que
\[\begin{pmatrix}p & 0 \\ 0 & 0 \end{pmatrix}\sim_h \begin{pmatrix}q & 0 \\ 0 & 0 \end{pmatrix}.\] 
\qed
\end{dem}

Cette propostion montre que les différentes relations d'équivalence coïncident si l'on se place dans $\mathfrak M_\infty(A)$.

\subsubsection{Limites inductives}

\begin{prop}
Soit $\{A_i, \phi^j_i\}$ un système inductif de $C^*$-algèbres. Alors $\{K_0(A_i), \phi^j_{i*}\}$ est un système inductif de groupes abéliens et
\[K_0( \varinjlim A_j)= \varinjlim K_0(A_j).\]
\end{prop}
\begin{dem}
Par définition, pour tout $i\leq j$, il existe des morphismes $\phi_i$ et $\phi_j$ qui font commuter le diagramme suivant :
\[\begin{tikzcd}A_i \arrow{r}{\phi_i}\arrow{d}{\phi^j_i} & \varinjlim A_j \\
A_j \arrow{ur}{\phi_j}\end{tikzcd}.\]
En passant ce diagramme en $K_0$-théorie, on obtient un second diagramme commutatif que la propriété universelle de la limite inductive nous permet de compléter en pointillés :
\[\begin{tikzcd}K_0(A_i) \arrow{r}{\phi_{i*}}\arrow{d}{\phi^j_ {i*}}\arrow[dotted]{dr} & K_0(\varinjlim A_j) \\
K_0(A_j) \arrow[dotted]{r}\arrow{ur}{\phi_{j*}} & \varinjlim K_0(A_j)\arrow[dotted]{u}{\exists ! \phi}\end{tikzcd}.\]
L'injectivité de $\phi$ est claire puisque tous les $\phi_{j*}$ le sont. Soit $[p]-[p_n]\in K_0(\varinjlim A_j)$. Pour tout $\epsilon >0$, il existe un $j>0$ et élément $a$ autoadjoint de $A_j$ tel que $||a-p||<\epsilon$. Alors le spectre de $a$ est inclus dans la réunion de $2$ intervalles centrés en $0$ et $1$. Par calcul fonctionnel continu, on peut alors définir un projecteur $q=f(a)$ tel que $||q-a||$ soit borné par $O(\epsilon)$. Pour $\epsilon$ assez petit, $||p-q||<1$ et donc $[p]=[q]$ dans $K_0(\varinjlim A)$. Mais $[q]\in K_0(A_j)\subset \varinjlim K_0(A_j)$ : la surjectivité est démontrée.
\qed
\end{dem}

Voici un exercice sur ce thème.\\

On a vu que le foncteur $K_*$ est stable par limite inductive. Pour autant certains exemples demandent un peu d'attention. Soit $A$ une $C^*$-algèbre unitale et $q$ un entier positif. On note $A_j=\mathfrak M_{q^j}(A)$, et $\phi_j^{j+1}:A_j  \rightarrow  A_{j+1}$ défini par
\[\phi_j^{j+1}: 
a  \mapsto  \begin{pmatrix} a & 0 &   &  &   \\
                              0 & a &   &  &   \\
			        &   & a &  &    \\
				&   &   & ..  &   \\
				&   &   &   & a 
\end{pmatrix}=a \otimes 1_q 
\] 
En composant on obtient des $*$-morphismes, où $i\leq j$ :
\[\phi^j_i :\left\{\begin{array}{lcr}A_i & \rightarrow & A_j \\
 a &\mapsto & a\otimes 1_{q^{j-i}}\end{array}\right.\]
qui nous définissent un système inductif $\{A_i, \phi^j_i\}$. Comme $K_*(A_j)=K_*(A)$, on serait tenter de conclure que la $K$-théorie de la limite inductive est celle de $A$. Il n'en n'est rien comme on peut facilement le voir avec $A=\C$. Les morphismes $\phi_{i*}^j$ sont donnés par la multiplication 
\[\phi_{i*}^j\left\{\begin{array}{lcr} K_*(A_i) & \rightarrow & K_*(A_j)\\
			\ [x]   & \mapsto     &  q^{j-i} [x] \end{array}\right.\]
et même si lorsque $A=\C$, $K_0(A_j)=K_0(\C)=\Z$, on obtient $K_0(\varinjlim A_j))=\Z[\frac{1}{q}]$.

\subsubsection{Suites exactes}
\begin{prop}
Soit $\begin{tikzcd}0 \arrow{r}& J\arrow{r}{\iota} & A\arrow{r}{\pi} & A/J\arrow{r} & 0\end{tikzcd}$ une suite exacte de $C^*$-algèbres. Alors la suite de groupes abéliens
\[\begin{tikzcd}K_0(J)\arrow{r}{\iota_*} & K_0(A)\arrow{r}{\pi_*}& K_0(A/J)\end{tikzcd}\]
est exacte.
\end{prop}

\begin{dem}
Soit $[p]-[p_n]\in K_0(J)$ avec $p=p_n \ \text{mod} \ \mathfrak M_k(J), k\geq n $. Cela assure que $\pi(p)=p_n$ et 
\[\pi_*([p]-[p_n])=[\pi(p)]-[p_n]=0\]
donc $\pi_*\circ \iota_*=0$.\\

Soit $[p]-[p_n]\in K_0(A)$ avec $p=p_n \ \text{mod} \ \mathfrak M_k(A), k\geq n $, d'image nulle par $\pi_*$. Alors $[\pi(p)]=[p_n]$. Il existe donc deux entiers $r\leq r'$ et un unitaire $u\in \mathfrak M_{k+r'}((A/J)^+)$ tels que 
\[u\begin{pmatrix}\pi(p) & 0 \\ 0 & p_r \end{pmatrix} u^* = \begin{pmatrix}p_n & 0 \\0 & p_r\end{pmatrix}.\]
Soit $w\in \mathfrak M_{2(k+r')}(A^+)$ un unitaire qui relève $\begin{pmatrix}u & 0 \\ 0 & u^*\end{pmatrix}$. Alors la relation ci-dessus assure que 
\[w(p \oplus p_r )w^* = p_n\oplus p_r \mod \mathfrak M_{2(k+r')}(J)\]
donc $[p]-[p_n]= [w(p\oplus p_r)w^*]-[p_n\oplus p_r]\in K_0(J)$ et donc $\text{ker} \pi_*\subset \text{im} \iota_* $.
\qed
\end{dem}

\subsection{Foncteur $K_1$}
\subsubsection{Définition}
On rappelle que, si $A$ est une $C^*$-algèbre, non nécessairement unitale, 
\[GL_n(A):= \{x\in GL_n(A^+) : x= 1_n \ \text{mod }\mathfrak M_n(A) \},\]
et c'est un sous-groupe fermé distingué de $GL_n(A^+)$. On peut naturellement plonger $GL_n(A)$ dans $GL_{n+1}(A)$ par 
$a\mapsto \begin{pmatrix}a & 0\\0 & 1_{A^+}\end{pmatrix}$, et ces morphismes définissent
\[GL_\infty(A)= \varinjlim GL_n(A),\]
que l'on peut voir comme des matrices infinies à valeurs dans $1_{A^+}+A$ sur la diagonale, dans $A$ ailleurs, et dont tous les termes sont nuls ou égaux à $1_{A^+}$ sur la diagonale, sauf un nombre fini. Les inclusion envoient $GL_n(A)_0$ dans $GL_{n+1}(A)_0$, on a donc 
\[GL_\infty(A)_0=\varinjlim GL_n(A)_0.\]

\begin{definition}
Soit $A$ un $C^*$-algèbre. On définit le groupe
\[\begin{array}{ccc}K_1(A) &= GL_\infty(A)/GL_\infty(A)_0 & = \varinjlim GL_n(A)/GL_n(A)_0 \\
			& = U_\infty(A)/U_\infty(A)_0 & = \varinjlim U_n(A)/U_n(A)_0 
\end{array}\] La seconde ligne provient de la décomposition polaire, qui assure que $U_n(A)$ est un retract de $GL_n(A)$.
\end{definition}

On voit que le groupe $K_1$ est généré par les classes $[u]$ où $u$ est un unitaire de $\mathfrak M_n(A)$, soumis aux relations $[uv]=[u]+[v]=[u\oplus v]$. En effet
\begin{prop}Le groupe $K_1(A)$ est abélien.\end{prop}
\begin{dem}
Le chemin $t\mapsto \begin{pmatrix}\cos t & -\sin t \\ \sin t & \cos t \end{pmatrix}$ connecte l'identité à $\begin{pmatrix}0& -1 \\ 1 & 0\end{pmatrix}$, ce qui assure que l'on peut permuter des lignes et des colonnes (modulo un signe) sans changer la classe dans $K_1(A)$. Si $\sim$ dénote la relation <<être homotope>>, alors pour $x$ et $y$ dans $A$:
\begin{align*}
\begin{pmatrix}xy & 0 \\ 0 & 1\end{pmatrix} & = \begin{pmatrix}x & 0 \\ 0 & 1\end{pmatrix}\begin{pmatrix}y & 0 \\ 0 & 1\end{pmatrix} \\
						&\sim \begin{pmatrix}x & 0 \\ 0 & 1\end{pmatrix}\begin{pmatrix} 1 & 0 \\ 0 & y \end{pmatrix}\\
					& =\begin{pmatrix} x& 0 \\ 0 & y\end{pmatrix}
\end{align*}
Alors, par symmétrie : $[xy]=[yx]=[x]+[y]$.
\qed
\end{dem}

Voici un lemme, et surtout son corollaire, qui sont utiles dans le maniement des suites exactes avec $K_1$.

\begin{lem}
Soit $A$ une $C^*$-algèbre unitale. Alors, si $\langle \exp(A)\rangle$ dénote le sous-groupe multiplicatif de $GL_1(A)$ algébriquement engendré par les exponentielles du type $\exp(a)$, alors
\[GL_1(A)_0 = \langle \exp(A)\rangle. \]
\end{lem}

\begin{dem}
Soit $z=\prod_{j=1}^n e^{a_j}\in \langle \exp(A)\rangle$. Si $||z-z'||<\frac{1}{||z^{-1}||}$, alors le logarithme de $z'z^{-1}$ est defini et donc 
$z'=\exp(\log(z'z^{-1}) ) z $ est dans $\langle \exp(A)\rangle$. Le sous-groupe $\langle \exp(A)\rangle$ est donc un ouvert d'un groupe de Lie, donc fermé, et il contient l'identité : c'est la composante connexe $GL_1(A)_0$.
\qed
\end{dem}
Le corollaire découle directement du lemme.
\begin{cor} Soit $\Psi : A\rightarrow A''$ un $*$-morphisme surjectif. Alors \[\Psi^{-1}(GL_1(A'')_0)\subset GL_1(A)_0.\]
\end{cor}

\subsubsection{Suspension}
\begin{definition}
On définit la suspension d'une $C^{*}$-algèbre $A$ comme la $C^*$-algèbre 
\[SA = C_0(\R) \otimes A.\]
\end{definition}

Comme $C_0(\R)$ est abélienne, elle est nucléaire et le produit tensoriel est unique. De plus, on a directement la 
\begin{prop}
$S$ est un foncteur covariant exact de la catégorie des $C^*$-algèbres dans la catégorie des $C^*$-algèbres non-unitales.
\end{prop}

Bien-sûr, cette définition peut être remplacée par une construction plus à la main, que l'on va décrire pour se familiariser avec cet objet. Cette façon de faire à la mérite de donner directement l'exactitude.\\
La suspension peut aussi se voir comme 
\[SA = \{f : \R\rightarrow A / f \ \text{continue telle que }\lim_{x\rightarrow \infty}||f(x)||=0\},\]
munie des opérations point par point et de la norme sup. Attention, $(SA)^+\neq S(A^+)$. D'ailleurs, cela nous servira beaucoup dans les preuves suivantes, $(SA)^+ \simeq\{f: \mathbb S^1 \rightarrow A^+ : f(z)=\lambda 1_{A^+}+g(z)\ \text{où }g\in SA \}$.

\begin{thm}
Il existe un isomorphisme naturel  entre $K_1$ et $K_0 S$, i.e. pour toute $C^*$-algèbre $A$, il existe un isomorphisme de groupe abéliens $\theta_A : K_1(A) \rightarrow K_0(SA)$ tel que, pour tout $C^*$-algèbre $B$ et tout $*$-morphisme $\phi : A\rightarrow B$, le diagramme 
\[\begin{tikzcd}K_1(A) \arrow{r}{\phi_*}\arrow{d}{\theta_A} & K_1(B) \arrow{d}{\theta_B} \\ K_0(SA)\arrow{r}{\phi_*} & K_0(SB)\end{tikzcd}\]
commute.
\end{thm}

\begin{dem}
Nous allons d'abors définir l'application $\theta_A$. Si $u\in GL_n(A)$, on peut construire un chemin continu $t\mapsto z_t $ d'unitaires dans $GL_{2n}(A)$ connectant $\begin{pmatrix}u & 0 \\ 0 & u^{-1}\end{pmatrix}$ à l'identité $1_{2n}$ et vérifiant $z_t = 1_{2n} \ \text{mod } \mathfrak M_n(A)$. Alors $e_t = z_t p_n z_t^{-1}$ est une boucle de projecteurs de base $p_n$ : $e\in \mathfrak M_{2n}((SA)^+)$. On pose :
\[\theta_A [u] = [e]-[p_n].\]
Cette application est bien définie car si $[u]=[v]$ dans $ K_1(A)$, alors pour un $n$ assez grand, on peut trouver deux chemins d'unitaires
\[\begin{array}{c} a_t : uv^{-1} \sim 1_n \\ b_t : u^{-1}v \sim 1_n\end{array}.\]
Et si $z_t : \begin{pmatrix}u & 0 \\ 0 & u^{-1}\end{pmatrix} \sim 1_{2n} $ et $w_t : \begin{pmatrix}v & 0 \\ 0 & v^{-1}\end{pmatrix} \sim 1_{2n} $ avec $\theta_A [u]= [e]-[p_n]=[zp_n z^{-1}]-[p_n]$ et $\theta_A [v]= [f]-[p_n]=[wp_n w^{-1}]-[p_n]$, on pose $\begin{pmatrix}a_t & 0 \\ 0 & b_t \end{pmatrix}$, alors 
\[e_t = z_t x_t p_n x_t^{-1} z_t^{-1} = y_t f_t y_t^{-1}\] 
où $y_t = z_t x_t w_t^{-1}$ définit un inversible de $\mathfrak M_{2n}((SA)^+)$ car $y_0=y_1=1_{2n}$. Comme $e$ et $f$ sont unitairement équivalents, $\theta_A [u]=\theta_A [v]$.\\

Montrons que $\theta_A$ est injective. Si $\theta_A [u]=\theta_A [v]$, alors, avec les mêmes notations, $[e]=[f]$, donc pour $k$ assez grand, il existe un inversible $x$ de $\mathfrak M_k((SA)^+)$ tel que $xex^{-1}=f$. Cette relation assure que 
\[ x_t z_t p_n z_t^{-1}  = w_t p_n w_t^{-1} x_t \]
donc $y_t= w_t^{-1} x_t z_t $ commute à $p_n$ et est donc de la forme $\begin{pmatrix} a_t & 0 \\ 0 & b_t\end{pmatrix}$. Comme $x_0=x_1=1_k$, $a_t$ connecte $1_n$ à $v^{-1} u$, qui sont donc homotopes : $[u]=[v]$.\\

Montrons la surjectivité. Soit $[f]-[p_n]\in K_0(SA)$ : $f\in \mathfrak M_k ((SA)^+)$ tel que $f=p_n \ \text{mod } \mathfrak M_k(SA)$, $k\geq n$. Chaque $f_t$ est dans $\mathfrak M_k(A^+)$ et $f_0=f_1=p_n$ donc il existe un chemin d'unitaires $t \mapsto w_t$ connectés à l'identité tel que $f_t = w_t p_n w^{-1}_t$. Comme $w_1 p_n = p_n w_1$, $w_1$ est de la forme $\begin{pmatrix}u & 0\\ 0 & v\end{pmatrix}$ où $u \in \mathfrak M_{n}(A^+)$ et $v \in \mathfrak M_{K-n}(A^+)$ sont deux inversibles. Mais $v$ est connecté par un chemin $b_t$ à $\begin{pmatrix}1_{k-n} & 0\\ 0 & u^{-1}\end{pmatrix}$, et on a un chemin $z_t$ qui connecte $1_k$ à $\begin{pmatrix} u & 0 & 0 \\ 0 & u^{-1} & 0\\ 0 & 0 & 1_{k-n}\end{pmatrix}$. Si on pose $e_t = z_t p_n z_t^{-1}$, on obtient que $z_t \begin{pmatrix}1 & 0 \\ 0 & v^{-1} b_t \end{pmatrix}w_t^{-1}$ entrelace $e_t$ et $f_t$. Comme précédement, ses valeurs coïncident en $0$ et $1$ avec $1_{2k}$ : c'est un inversible de $\mathfrak M_k ((SA)^+)$. Donc 
\[\theta_A [u] = [e]-[p_n]= [f]-[p_n],\]
$\theta_A$ est surjective. La naturalité est immédiate, la définition de l'application utilisant des sommes directes.
\qed
\end{dem}

\begin{thm}
Pour toute suite exacte de $C^*$-algèbres \[\begin{tikzcd}[row sep = small] 0\arrow{r} & A' \arrow{r}{\iota} & A\arrow{r}{\pi} & A''\arrow{r} & 0\end{tikzcd},\] la suite de groupes abéliens
\[\begin{tikzcd} K_*(A') \arrow{r}{\iota_*} & K_*(A)\arrow{r}{\pi_*} & K_*(A'')\end{tikzcd}\] est exacte.
\end{thm}

\subsubsection{Indice}
Dans cette section, nous allons définir un morphisme $\delta : K_1(A/J)\rightarrow K_0(J)$, appelé indice, qui rend exacte la suite
\begin{equation}\begin{tikzcd} K_1(J) \arrow{r}{\iota_*} & K_1(A)\arrow{r}{\pi_*} & K_1(A/J)\arrow{r}{\delta} 
& K_0(J) \arrow{r}{\iota_*} & K_0(A)\arrow{r}{\pi_*} & K_0(A/J)
\end{tikzcd}.\label{longue}\end{equation}
\begin{definition}
Soit $u\in GL_n(A/J)$ et $w\in GL_{2n}(A)$ un relevé de $\begin{pmatrix}u & 0 \\ 0 & u^{-1}\end{pmatrix}$. On définit l'indice de $[u]\in K_1(A/J)$ par \[\delta [u]=[wp_n w^{-1}]-[p_n].\]
\end{definition}

\begin{prop}
La suite \ref{longue} est exacte en $K_1(A/J)$.
\end{prop}

\begin{prop}
La suite \ref{longue} est exacte en $K_0(J)$.
\end{prop}

\subsection{Periodicité de Bott}

Dans cette section, nous allons montrer un résultat central en $K$-théorie des $C^*$-algèbres, celui de la périodicité de Bott : il existe un isomorphisme naturel $\beta_A : K_0 (A) \rightarrow K_1(SA)$ pour toute $C^*$-algèbre $A$, et donc la $K$-théorie est $2$-périodique : $K(S^2 A )\simeq K(A)$.\\

La preuve donnée dans le cadre de la $K$-théorie topologique s'adapte facilement au cadre non-commutatif, nous la proposerons sous forme d'exercice. Celle que nous avons choisi de détailler utilise la non-commutativité de l'algèbre de Toeplitz. Elle est due à Cuntz~\cite{Cuntz}, et est détaillée dans le livre de Wegge-Olsen~\cite{WeggeOlsen}. Nous exposerons donc un point de vue moins $C^*$-algébrique, où la $K$-théorie est vue comme une théorie de l'homologie sur la catégorie des $C^*$-algèbres.\\

\subsubsection{Rappels}

\begin{definition}
On définit le cône d'une $C^*$-algèbre $A$ comme 
\[CA= \{f:\R \rightarrow A / f \ \text{continue bornée telle que }\lim_{x\rightarrow -\infty} ||f(x)||=0\}.\]
$SA$ est clairement un idéal de $CA$ et on a une suite exacte 
\[\begin{tikzcd} 0\arrow{r} & SA \arrow{r}& CA \arrow{r}{\epsilon_1} & A \arrow{r} & 0\end{tikzcd}\]
grâce à l'évaluation en $1$, notée $\epsilon_1$.\\
\end{definition}

Le cône $CA$ est toujours contractile : $t\mapsto f(st)$ pour $s$ allant de $0$ à $1$ donne une homotopie entre $0$ et $id_{CA}$.\\

\begin{definition}
Pour tout $*$-morphisme $\pi: A\rightarrow A''$, on peut définir le cône de $\pi$ comme le pullback de $A''$ via $\pi$ et 
$\epsilon_1 : CA''\rightarrow A''$ :
\[\begin{tikzcd}
C_\pi \arrow[dotted]{r}\arrow[dotted]{d}& A \arrow{d}{\pi} \\
CA'' \arrow{r}{\epsilon_1} & A''.
\end{tikzcd}\]
\end{definition}
Attention, $C_\pi$ n'est pas forcément contractile.

Le lemme suivant est simple, mais technique, et il sera utile pour la suite.

\begin{lem}
Soit \[\begin{tikzcd}
0\arrow{r} & A' \arrow{r}{\iota} & A \arrow{r}{\pi} & A'' \arrow{r} & 0
\end{tikzcd}\]
une suite exacte de $C^*$-algèbre. On a alors trois suites exactes :
\[\begin{tikzcd}
0\arrow{r} & SA'' \arrow{r}{\iota_1} & C_\pi \arrow{r}{\pi_1} & A \arrow{r} & 0 \\
0\arrow{r} & A' \arrow{r}{\iota_2} & C_\pi \arrow{r}{\pi_2} & CA'' \arrow{r} & 0\\
0\arrow{r} & CA' \arrow{r}{\iota_3} & D \arrow{r}{\pi_3} & C_\pi \arrow{r} & 0\\
\end{tikzcd}\]
avec $D=\{f:\R\rightarrow A /\ f  \text{ continue bornée telle que }\lim_{x\rightarrow -\infty} f (x) \in \iota ( A')\}$ et 
\begin{align*}
\iota_1(f)=0\oplus f & \pi_1(a\oplus f) = a ,\\
\iota_2(a')=\iota(a')\oplus 0 & \pi_2(a\oplus f )=f,\\
\iota_3(f) = \{t\mapsto \iota(f(1-t))\} & \pi_3(g)=g(1)\oplus (\pi\circ g ) .
\end{align*}
De plus, on a une équivalence d'homotopie $\left\{\begin{array}{ccc} A' &\rightarrow& D\\ a' &\mapsto &\iota(a')\end{array}\right.$ telle que le diagramme suivant commute :
\[\begin{tikzcd}A'\arrow{r}{\iota}\arrow{d} & A\arrow{d}{\pi_1}\\ D \arrow{r}{\pi_3} & C_\pi .\end{tikzcd}\]
\end{lem}

\subsubsection{Homologie}

\begin{definition}
Deux $*$-morphismes $\phi,\Psi : A\rightarrow B$ sont dit homotopes s'il existe une fammile $\phi_t : A\rightarrow B$ de $*$-morphismes telle que $\phi_0=\phi$, $\phi_1=\Psi$ et $t\mapsto \phi_t(a) $ est continue pour tout $a$. On note $\phi\sim\Psi$.\\
Un foncteur $F$ de la catégorie des $C^*$-algèbres dans celle des groupes abéliens est dit foncteur d'homologie s'il est exacte au milieu et invariant par homotopie.
\end{definition}

\begin{prop}
Soit $F$ un foncteur covariant de la catégorie des $C^*$-algèbres dans celle des groupes abéliens qui est invariant par homotopie. Alors toute équivalence $\phi: A \rightarrow B$ induit un isomorphisme de groupe $F(\phi):F(A)\rightarrow F(B)$.
\end{prop}

\begin{dem} Si $\phi$ est une équivalence, il existe $\psi : B\rightarrow A $ telle que $\phi \circ \psi \sim id_B $ et $\psi\circ \phi\sim id_B$. Alors $F(\psi)F(\phi)=id_{F(A)}$ et $F(\phi)F(\psi)=id_{F(B)}$, donc $F(\phi)$ est un isomorphisme. \qed
\end{dem}

Ce lemme assure que :
\begin{itemize}
\item si $B$ est une rétraction de $A$, alors $F(A)\simeq F(B)$,
\item si $A$ est contractile, $F(A)=0$.
\end{itemize}

\begin{definition}
Une théorie de l'homologie sur la catégorie des $C^*$-algèbres est une suite de foncteurs d'homologie $H_n$ telle que, pour toute suite exacte 
\[\begin{tikzcd}
0\arrow{r} & A' \arrow{r}{\iota} & A \arrow{r}{\pi} & A'' \arrow{r} & 0,
\end{tikzcd}\]
il existe des morphismes connectant $\delta_n : H_n(A'')\rightarrow H_{n+1}(A')$ qui rendent la suite longue 
\[\begin{tikzcd}[row sep =small]
...\arrow{r} & H_n( A') \arrow{r}{\iota_*} & H_n(A) \arrow{r}{\pi_*} & H_n(A'') \arrow{r}{\delta_n} & H_{n+1}(A')\arrow{r}{\iota} & H_{n+1}(A) \arrow{r}{\pi_*} & H_{n+1}(A'')\arrow{r}{\delta}& ....
\end{tikzcd}\]
exacte, et telle que chaque $\delta$ soit une application naturelle. \\
On distingue les théories infinies à gauche ($n\leq 0$), à droite ($n\geq 0$), ou doublement infinie. 
\end{definition}

On voit que la partie précédente montre que $H_{-n}(A)=K_0(C_0(\R^n)\otimes A)$ définit une théorie de l'homologie pour les $C^*$-algèbres.\\

\begin{prop}
Soit $(H_n)$ une théorie de l'homologie pour les $C^*$-algèbres, infinie à gauche ou doublement infinie. Alors pour toute suite exacte scindée, l'application connectante est nulle. 
\end{prop}

\begin{prop}
Soit $F$ un foncteur d'homologie et \[\begin{tikzcd}
0\arrow{r} & A' \arrow{r}{\iota} & A \arrow{r}{\pi} & A'' \arrow{r} & 0
\end{tikzcd}\]
une suite exacte. Alors 
\begin{itemize}
\item si $A''$ est contractile, $ \iota_*$ est un isomorphisme,
\item il existe un morphisme de groupes abéliens $\delta : F(SA'')\rightarrow F(A')$ telle que la suite suivante soit exacte :
\[\begin{tikzcd}[row sep = small]
F(SA') \arrow{r}{\iota_*} & F(SA) \arrow{r}{\pi_*} & F(SA'') \arrow{r}{\delta} & F(A)' \arrow{r}{\iota_*} & F(A) \arrow{r}{\pi_*} & F(A'').\\
\end{tikzcd}\] 
\end{itemize}
\end{prop}

\begin{cor}Tout foncteur d'homologie $F$ définit une théorie de l'homologie pour les $C^*$-algèbres par 
\[\left\{\begin{array}{c} H_0(A)=F(A) \\ H_{-n}(A) = F(S^n A)\quad, n\geq 0.\end{array}\right.\]
\end{cor}

\subsubsection{Périodicité de Bott}

\begin{thm}Tout foncteur $F$ de la catégorie des $C^*$-algèbres dans celle des groupes abéliens qui est exact au mileu, invariant par homotopie et stable admet une periodicité de Bott, c'est-à-dire que $F(S^2 A)$ et $F(A)$ sont naturellement isomorphes.\end{thm}


