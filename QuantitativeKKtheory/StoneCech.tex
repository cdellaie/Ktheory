\section{Stone-Cech compactification}

Let $X$ be a topological space. The Stone-Cech compactification of $X$, denoted by $\beta X$, is defined as the compact Hausdorff space, unique up to homeomorphism, together with a $C_b(X)$-embedding $\phi_X: X\rightarrow \beta X$ such that, for any continuous map $f : X\rightarrow K$ in a compact space $K$, there exists a unique continuous map $\tilde f : \beta X\rightarrow K$ that makes the following diagram commutes :

\[\begin{tikzcd} X \arrow{r}{f}\arrow{d}{\phi_X} & K \\
	\beta X \arrow{ur}{ \tilde f}& 
.\end{tikzcd}\]

The universal property of the Stone-Cech compactified makes it a functor from the category of topological spaces to the category of compact Hausdorff spaces. Indeed, it is general property that, if we are given two categories $C$ and $C'$ and a functor $\phi : C \rightarrow $, such that for every functor $F: C \rightarrow C'$, there exists \\

Let $X$ be a compact Hausdorff space. Then the maximal ideals of $C(X)$ are in a one-to-one correspondence with the points of $X$. Explicitely, to a point $p\in X$ corresponds the maximal ideal
\[\mathfrak M_p=\{f\in C(X) : f(p)=0\}.\]
If one endorses the  spaces of maximal ideals of $C(X)$ with the Stone topology, this correspondence $p\mapsto \mathfrak M_p$ is actually a homeomorphism. Now, it is a theorem that when $X$ is just locally compact, $C(\beta X)$ and $C_b(X)$ are homeomorphic, and then we have that $\beta X   \simeq\mathfrak M(C(\beta X)) \simeq \mathfrak M (C_b(X)) $. This amounts saying that we can see $\beta X $ as the spectra of $C_b(X)$, for all locally compact spaces.\\


