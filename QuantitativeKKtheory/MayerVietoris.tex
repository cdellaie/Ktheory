%\section{The Novikov conjecture for groups with finite decomposition complexity (maybe)}

During the past years, there has been a growing interest on the links between several conjectures involving assembly maps. This report will focus on the link between the coarse Baum-Connes conjecture and the Novikov conjecture. If $\Gamma$ is a finitely generated group, the descent principle assures that if the coarse Baum-Connes map for $\Gamma$ as a metric space with the word length is an isomorphism, then the Baum-Connes assembly map for $\Gamma$ is injective, thus the Novikov conjecture holds for $\Gamma$.\\

Following ideas of M. Gromov, G. Yu introduced new coarse concepts in the study of these assembly maps. He was able to prove the coarse Baum-Connes conjecture for proper metric spaces with finite asymptotic dimension, which is a coarse analogue of the topological covering dimension. Later on, in a paper with Guenter and R. Tessera \cite{GTY}, they defined decomposition complexity for metric spaces, which is a broad generalization of asymptotic dimension. In particular, proper metric spaces with finite asymptotic dimension are of finite decomposition complexity. At the end of \cite{GTY}, as concluding remarks, the authors point out that one should be able to derive a new proof of the coarse Baum-Connes conjecture for spaces with finite decomposition complexity. We should emphasize that this is already known : a space which is finitely decomposable has property (A), hence verifies the coarse Baum-Connes conjecture by the work of G. Yu. \cite{Yu2} But the techniques of this proof is highly analytical, it uses a Dirac-Dual Dirac type construction, which involves infinite dimensional analysis. The suggestion of \cite{GTY} is to give a geometrical proof, using a coarse Mayer-Vietoris argument in the spirit of the proof of the Baum-Connes conjecture for spaces with finite asymptotic dimension.\\

Such a proof was given in the setting of algebraic $K$-theory in a paper of D. A. Ramras, R. Tessera and G. Yu where they established the integral Novikov conjecture for algebraic $K$-theory of group rings $R[\Gamma]$ when the group $\Gamma$ has FDC (finite decomposition complexity). Their proof uses the continuously controlled algebraic $K$-theory groups very intensively : their key lemma is a vanishing theorem of these groups. In a series of papers \cite{OY2}\cite{OY3}, H. Oyono-Oyono and G. Yu developed an analogue of this controlled $K$-theory for operator algebras, which they named quantitative $K$-theory. It consists of a family of groups $\hat K (A) = (K^{\epsilon,r}(A))$ for $r\geq 0,0< \epsilon <\frac{1}{4}$ and $A$ a filtered $C^*$-algebra, which we shall describe later. They were able to define quantitative assembly maps that factorize the usual ones, and to give equivalence between isomorphisms of the assembly map and quantitative statements.\\

Following the route of these articles \cite{OY2}\cite{OY3}, we will define quantitative assembly maps for étale groupoids with a proper length. These assembly maps are equivalent to the coarse quantitative assembly maps for proper metric spaces $X$ defined in \cite{OY3} if one takes $G= G(X)$, the coarse groupoid of $X$. We give also quantitative statements equivalent to a certain isomorphism. \textbf{(rerédiger ce paragraphe de façon plus précise une fois les résultats écrits)}\\

\section{Review of quantitative $K$-theory}

This section presents basic constructions of quantitative $K$-theory for operator algebras that we shall use. For more details, see the original article of H. Oyono-Oyono and G. Yu.\cite{OY2} We will refer either to quantitative or controlled $K$-theory for the same object, namely a family of abelian groups $\hat K(A)= (K^{\epsilon,R})$ where $R>0, 0<\epsilon<\frac{1}{4}$, defined for a filtered $C^*$-algebra $A$. The motivating idea is to keep track of propagation of an operator while taking his (possibly higher) index. The main example is that of Roe algebras. \\

\subsection{Roe algebras and filtration}

Let $(X,d)$ be a discrete proper metric space, i.e. its closed ball are compact, that is uniformly bounded, so that for every $R>0$, there exists an integer $N\geq 0$ such that every ball of radius $R$ contains less than $N$ elements. A $X$-module is a hilbert space $H$ equiped with a $C^*$-morphism $\rho : C_0(X)\rightarrow \mathcal L(H)$. To lighten notations, we write $fx$ instead of $\rho(f)x$ if $f\in C_0(X)$ and $x\in X$. All these definitions can be found in \cite{RoeIndex}

\begin{definition}
Let $H$ be a $X$-module.
\begin{itemize}
\item[$\bullet$] An operator $T\in \mathcal L(H)$ is locally compact if for every $f\in C_0(X)$, $fT$ and $Tf$ are compact operators, where $f$ is understood as a multiplication operator.
\item[$\bullet$] An operator $T\in \mathcal L(H)$ is of finite propagation bounded by $R>0$ if for every pair of functions $f,g\in C_0(X)$ such that $d(\text{supp }f , \text{supp }g)>R$, $fTg =0$.
\item[$\bullet$] We denote by $C_R[X]$ the set of locally compact operators with finite propagation bounded by $R$. The Roe algebra of $X$ is $C^*(X)$, the closure of $\cup_{R>0} C_R[X]$ in the operator topology of $\mathcal L(H)$.\\
\end{itemize}
\end{definition} 
 
An simple example is given by $l^2(X)\otimes H$ with $H$ a separable Hilbert space, in which $C_R[X]$ is the algebra of operators $(T_{xy})_{x,y\in X}$ such that $T_{x,y}\in K(H)$ for every $x,y\in X$, and $T_{xy}=0$ as soon as $d(x,y)>R$.\\
Remark : one coulde replace Hilbert spaces by Hilbert modules $E$ over a $C^*$-algebra $B$ in this definition, $\mathcal L(H)$ by adjoinable operators $\mathcal L_B(E)$ and $K(H)$ by compact operators $K_B(E)$, to obtain $C^*(X,B)$, the Roe algebra with coefficient in $B$. The Roe algebra $C^*(X,B)$ enjoys functorial properties in $B$.\\

This example motivates the following definition.\\

\begin{definition}
A $C^*$-algebra $A$ is said to be filtered if there are closed $*$-stable linear subspaces $A_R$ for every $R>0$ such that
\begin{itemize}
\item[$\bullet$] $A_s \subset A_r$ when $s\leq r$,
\item[$\bullet$] $\cup_{R>0} A_R$ is dense in $A$,
\item[$\bullet$] $A_s . A_r \subset A_{s+r}$ for every $r,s \geq 0$,
\item[$\bullet$] $\forall r>0, 1\in A_r$ when $A$ is unital.\\
\end{itemize}

A $C^*$-morphism between filtered $C^*$-algebras $\phi : A \rightarrow B$ is filtered if $\phi(A_R)\subset B_R$ for every $R>0$.
\end{definition}

If $A$ is a non-unital $C^*$-algebra, let $A^+$ be the unital $C^*$-algebra containing $A$ as a two-sided ideal, defined as :

\[\begin{array}{c}A^+=\{(a,\lambda)\in A\times \C \} \\ (a,\lambda)(b,\mu)=(ab+\lambda b +\mu a,\lambda\mu) \\ (a,\lambda)^*=(a^*,\overline\lambda)\end{array}\]
with the norm operator
\[||(a,\lambda)||=\sup \{||ax+\lambda x|| : x\in A , ||x||=1\}.\]

When $A$ is not unital and filtered by $(A_R)_{R>0}$, $A^+$ is filtered by $A_R^+= \{(x,\lambda) : x\in A_R,\lambda\in \C\}$.

\subsection{Definition of quantitative $K$-theory} 

\subsection{Morita equivalence}

As in classical $K$-theory, we have an isomorphism which we call the (controlled) Morita equivalence.
\begin{prop}
Let $A$ be a filtered $C^*$-algebra and $H$ a separable Hilbert space. We denote by $K_A$ the $C^*$-algebra of compact operators of the standard Hilbert module $H_A$, which is $C^*$-isomorphic to $A\otimes K(H)$. Let $e$ be any rank-one projection in $K(H)$. Then the $C^*$-morphism 
\[\begin{array}{lll} A & \rightarrow & K_A \\ a &\mapsto & a\otimes e\end{array}\]
induces an $\Z_2$-graded isomorphism
\[M_A^{\epsilon, R} : K^{\epsilon , R}(A)\rightarrow K^{\epsilon , R}(K_A)\]
for every $R>0$ and $0< \epsilon <\frac{1}{4}$.
\end{prop}

\subsection{Quantitative boundary maps}
\textbf{METTRE LA RQ 3.7 de OY2}

\subsection{A remainder on groupoids action}

\section{Quantitative statements}

The more general setting of the Baum-Connes conjecture \cite{TuBC} is that of a locally compact $\sigma$-compact Hausdorff groupoid $\G$ endowed with a Haar system, together with a coefficient $C^*$-algebra $B$ acted upon by $\G$, which give rise to an assembly map
\[\mu_r : K^{top}_*(\G,B)\rightarrow K_*(B\rtimes_r \G).\]

The left hand side $K^{top}_*(\G,B)$ is the $K$-homology of the classifying space $\mathcal E \G$ for proper actions of $\G$ in coefficient in $B$. We give a sketch of the construction when $\G$ is étale. Let $d\geq 0$ and $P_d(\G)$ be the Rips complex of $\G$, i.e. the space of probabilities supported on a fiber $\G^x$ for a $x\in \G^{(0)}$
\[P_d(\G) = \{p\in \mathcal P(\G) : \exists x\in \G^{(0)}, r^*p=\delta_x, \text{supp }p \subset B(e_x,d)\}.\]
Then $KK^\G( C_0(P_d(\G)),B)$ is defined to be the inductive limite of $KK^\G( C_0(X),B)$ for $X$ $\G$-proper $\G$-spaces (such that $X/G$ is compact). If $d\leq d'$, we have a morphism $KK^\G( C_0(P_d(\G)),B)\rightarrow KK^\G( C_0(P_{d'}(\G)),B)$ naturally induced by the inclusion $P_d(\G)\subset P_{d'}(\G)$, and the $K$-homology of $\G$ is defined as
\[K^{top}_*(\G,B)=\lim_{d\rightarrow \infty} KK^\G( C_0(P_d(\G)),B).\]

In his thesis \cite{LeGall}, P.-Y. Le Gall constructed the Kasparov transform for the action of a groupoid
\[j_\G : KK^\G(A,B)\rightarrow KK(A\rtimes \G,B\rtimes \G)\]
for any $\G$-$C^*$-algebras $A$ and $B$. It is also in this paper that equivariant $KK$-theory for groupoids and the corresponding Kasparov product are defined. One can then give an formula for the assembly map, namely if $z\in KK^G(C_0(X),B)$ for a $\G$-proper $\G$-space $X$ of $P_d(\G)$, then
\[\mu_r(z)=[\mathcal L_X]\otimes_{C_0(X)\rtimes_r \G} j_\G(z) \in K_*(B\rtimes_r \G) \]
holds, where $[\mathcal L_X]$ is the class of a canonical element associated to $X$ which is to be thought of as a Miscenko bundle over $C_0(X)\rtimes_r \G$.\\

The remaining of this section will be devoted to the construction of a controlled Kasparov transformation for every $z\in KK^\G(A,B)$ :
\[J_\G(z) : \hat K(A\rtimes \G)\rightarrow \hat K(B\rtimes \G)\] 
which is of course a controlled morphism which induces right multiplication by $j_\G(z)$ in $K$-theory. This will allow us to define a bunch of quantitative assembly maps
\[\mu_\G^{\epsilon, R}: K^{top}(\G,B)\rightarrow K^{\epsilon,R}(B\rtimes \G)\]
inducing the assembly map in $K$-theory, and to study the relation between the quantitative Baum-Connes conjecture and the classical one for $\G$.

\subsection{Length, propagation and controlled six-terms exact sequence}

Let $\G$ be a locally compact groupoid with base $\G^{(0)}=X$, a compact space, endowed with a Haar system $\lambda=(\lambda^x)_{x\in X}$. We suppose that $\G$ comes with a proper length $l$, that is a family of application $(l^x)_{x\in X}$ defined on the fibers $\G^x$ with values in $\R_+$, such that
\[\begin{array}{l}
l^x(e_x)=0 \\
l^{r(\gamma)}(\gamma)=l^{s(\gamma)}(\gamma^{-1}) \\
l^x (\gamma_1^{-1} \gamma_2)	\leq l^x(\gamma_1)+l^x(\gamma_2) .	
\end{array}\]

That length allows us to define a filtration on crossed-product algebras of $\G$ by
\[(A\rtimes \G)_r = \{f\in C_c(\G, A) : \text{ supp }f \subset \cup_{x\in X} B_x(e_x, r)\}\]
for any $\G$-algebra $A$. Here, $B_x(e_x,r)$ is the ball $\{\gamma \in \G : l^{r(\gamma)}\leq r\}$, and $\rtimes$ can be either the reduced cross-product $\rtimes_r$ or the maximal one $\rtimes_{max}$. Recall that $A\rtimes \G$ is functorial in $A$, from the category of $\G$-$C^*$-algebras with $\G$-equivariant $C^*$-morphisms to the category of $C^*$-algebras with $C^*$-morphisms. For $\phi : A\rightarrow B$ a $\G$-equivariant $C^*$-morphism, we denote by $\phi_\G : A\rtimes\G \rightarrow B\rtimes\G$ the induced $C^*$-morphism.\\

If $\begin{tikzcd}[column sep = small]
0 \arrow{r} & J\arrow{r}{\phi}& A \arrow{r}{\psi} & A/J \arrow{r} & 0
\end{tikzcd}$ is a semi-split exact sequence of $\G$-$C^*$-algebras, then 
$\begin{tikzcd}[column sep = small]
0 \arrow{r} & J\rtimes \G\arrow{r}{\phi_\G}& A\rtimes \G \arrow{r}{\psi_\G} & A/J\rtimes \G \arrow{r} & 0
\end{tikzcd}$ is a flitered semi-split exact sequence. From this, we can state the following proposition.\\

\begin{prop}
There exists a control pair $(\lambda,h)$ such that for every semi-split extension of $\G$-$C^*$-algebras
\[\begin{tikzcd}[column sep = small]
0 \arrow{r} & J\arrow{r}{\phi}& A \arrow{r}{\psi} & A/J \arrow{r} & 0
\end{tikzcd},\]
the following diagrams commutes and are exact 
\[\begin{tikzcd}[column sep = small]
\hat K_0(J\rtimes_r \G) \arrow{r}{\phi_{\G,*}}&\hat K_0( A\rtimes_r \G ) \arrow{r}{\psi_{\G,*}} & \hat K_0( A/J\rtimes_r \G) \arrow{d} \\
\hat K_1(A/J\rtimes_r \G) \arrow{u} \arrow{r}{\phi_{\G,*}}& \hat K_1( A\rtimes_r \G ) \arrow{r}{\psi_{\G,*}} & \hat K_1( J\rtimes_r \G)
\end{tikzcd},\]
\[\begin{tikzcd}[column sep = small]
\hat K_0(J\rtimes_{max} \G) \arrow{r}{\phi_{\G,*}}&\hat K_0( A\rtimes_{max} \G ) \arrow{r}{\psi_{\G,*}} & \hat K_0( A/J\rtimes_{max} \G) \arrow{d} \\
\hat K_1(A/J\rtimes_{max} \G) \arrow{u} \arrow{r}{\phi_{\G,*}}& \hat K_1( A\rtimes_{max} \G ) \arrow{r}{\psi_{\G,*}} & \hat K_1( J\rtimes_{max} \G)
\end{tikzcd}.\]
\end{prop}

\subsection{The Kasparov transform}

Let $A$ and $B$ be two $\G$-$C^*$-algebras, and $H$ a separable Hilbert space, $l^2(\Z)$ for instance, and $H_\G= H\otimes L^2(\G,\lambda)$. The standard Hilbert module over $B$ is denoted by $H_B=H_\G\otimes B$, and $K_B$ is the algebra of compact operators for $H_B$, i.e. $K(H)\otimes L^2(\G,\lambda)\otimes B$. \\

Every $K$-cycle $z\in KK^G(A,B)$ can be represented as a triplet $(H_B, \pi, T)$ where :
\begin{itemize}
\item[$\bullet$]$\pi : A\rightarrow \mathcal L_B(H_B)$ is a $*$-representation of $A$ on $H_B$.
\item[$\bullet$]$T\in \mathcal L_B(H_B)$ is a self-adjoint operator.
\item[$\bullet$] $T$ and $\pi$ verify the $K$-cycle condition, i.e. $[T,\pi(a)]$, $\pi(a)(T^2-id_{H_B})$ and $\pi(a)(g.T-T)$ are compact operator over $H_B$ for all $a\in A, g\in \G$.\\
\end{itemize}

Set $T_\G= T\otimes id_{B\rtimes \G}\in \mathcal L_{B\rtimes \G}(H_B\otimes (B\rtimes \G))\simeq \mathcal L_{B\rtimes \G}(H_{B\rtimes\G})$, and $\pi_G: A\rtimes \G v\rightarrow L_{B\rtimes \G}(H_{B\rtimes\G})$. Then, according to Le Gall \cite{LeGall}, $(H_{B\rtimes\G, \pi_\G, T_\G})$ represents the $K$-cycle $j_\G(z)\in KK(A\rtimes \G,B\rtimes \G)$. Let us construct a controlled morphism associated to $z$,
\[J_\G(z) : \hat K(A\rtimes \G)\rightarrow K(B\rtimes\G), \]
which induces right multiplication by $j_\G(z)$ in $K$-theory.\\

\subsubsection{Odd case}

Let us first do the for work for $z\in KK_1^\G(A,B)$. Let $(H_B,\pi,T)$ be a $K$-cycle representing $z$. Set $P=\frac{1+T}{2}$ and $P_\G=P\otimes id_{B\rtimes \G}$. We define
\[E^{(\pi,T)}=\{(x,P_G\pi_G(x)P_\G + y) : x\in A\rtimes \G, y\in K_{B\rtimes\G}\}\]
a $C^*$-algebra which is filtered by
\[E_R^{(\pi,T)}=\{(x,P_G\pi_G(x)P_\G + y) : x\in (A\rtimes \G)_R, y\in K\otimes (B\rtimes\G)_R\}\]
which gives us a filtered extension
\[\begin{tikzcd}[column sep = small]
0\arrow{r} & K_{B\rtimes_r\G}\arrow{r} & E^{(\pi,T)} \arrow{r} & A\rtimes_r \G \arrow{r}& 0
\end{tikzcd}\]
and semi split by  $s :\left\{\begin{array}{lll}A\rtimes_r \G & \rightarrow & E^{(\pi,T)} \\ x & \mapsto & (x, P_\G \pi_\G(x)P_\G)\end{array}\right.$.\\

Let us show that the controlled boundary map of this extension does not depend on the representant chosen, but only on the class $z$.\\
Let $(H_B, \pi_j,T_j), j=0,1$ two $K$-cycles which are homotopic via $(H_{B[0,1]},\pi,T)$. We denote $e_t$ the evaluation at $t\in[0,1]$ for an element of $B[0,1]$, and set $y_t=e_t(y)$ for such a $y$. The $*$-morphism
\[\phi : \left\{\begin{array}{lll}E^{(\pi,T)} & \rightarrow & E^{(\pi_t,T_t)} \\ (x,y) & \mapsto & (x, y_t)\end{array}\right.\]
satisfies $\phi(K_{B[0,1] \rtimes_r \G})\subset K_{B \rtimes_r \G}$ and makes the following diagram commute
\[\begin{tikzcd}[column sep = small]
0\arrow{r} & K_{B[0,1] \rtimes_r \G}\arrow{r}\arrow{d}{\phi_{|K_{B[0,1] \rtimes_r \G}}} & E^{(\pi,T)} \arrow{r}\arrow{d}{\phi} & A\rtimes_r \G \arrow{r}\arrow{d}{=}& 0 \\
0\arrow{r} & K_{B \rtimes_r \G}\arrow{r} &  E^{(\pi_t,T_t)} \arrow{r} & A\rtimes_r \G \arrow{r} & 0
\end{tikzcd}.\]

According to \cite{OY2}, remark $3.7.$, the following holds
\[D_{K_{B\rtimes_r\G,E^{(\pi_t,T_t)}}} = \phi_* \circ D_{K_{B[0,1]\rtimes_r\G},E^{(\pi,T)}}.\]
As $id \otimes e_t$ gives a homotopy between $id\otimes e_0$ and $id\otimes e_1$, and as if two $*$-morphisms are homotopic, then they are equal in controlled $K$-theory, 
\[D_{K_{B\rtimes_r \G}, E^{(\pi_0,T_0)}}=D_{K_{B\rtimes_r \G}, E^{(\pi_1,T_1)}}\]
holds, and the boundary of the extension $E^{(\pi,T)}$ depends only on $z$.\\

\begin{definition}
The controlled Kasparov transform of an element $z\in KK_1^\G(A,B)$ is defined as the compostion
\[J_{red,\G}(z)=\mathcal M_{B\rtimes_r \G}^{-1}\circ D_{K_{B\rtimes_r \G}, E^{(\pi,T)}}.\]
\end{definition}

\begin{prop}
Let $A$ and $B$ two $\G$-$C^*$-algebras. For every $z\in KK^\G_1(A,B)$, there exists a controlled morphism
\[J_{red,\G}(z) : \hat K_*(A\rtimes_r \G)\rightarrow \hat K_{*+1}(B\rtimes_r \G)\]
such that
\begin{enumerate}
\item[(i)] $J_{red,\G}(z)$ induces right multiplication by $j_{red,\G}(z)$ in $K$-theory ;
\item[(ii)] $J_{red,\G}$ is additive, i.e.
\[J_{red,\G}(z+z')=J_{red,\G}(z)+J_{red,\G}(z').\]
\item[(iii)] For every $\G$-morphism $f : A_1\rightarrow A_2$,
\[J_{red,\G}(f^*(z))=J_{red,\G}(z)\circ f_{\G,red,*}\] for all $z\in KK_1^G(A_2,B)$.
\item[(iv)] For every $\G$-morphism $g : B_1\rightarrow B_2$,
\[J_{red,\G}(g_*(z))= g_{\G,red,*}\circ J_{red,\G}(z)\] for all $z\in KK_1^G(A,B_1)$.
\end{enumerate}
\end{prop}

\begin{dem}
\begin{enumerate}

\item[(i)] Le bord $[\delta_{K_{B\rtimes_r \G},E^{(\pi,T)}}]\in KK_1(A\rtimes_r \G, B\rtimes_r \G)$ associé à l'extension $E^{(\pi,T)}$ induit par définition, modulo équivalence de Morita, l'application $j_{red,\G}$, ce qui assure directement ce point.

\item[(ii)] Si $z,z'$ sont deux eléments de $KK_1^G(A,B)$, repésentés par des $K$-cycles $(H_B,\pi_j,T_j)$, et si l'on note $(H_B,\pi,T)$ un $K$-cycle représentant la somme $z+z'$, alors $E^{(\pi,T)}$ est naturellement isomorphe à l'extension somme des $E_j:=E^{(\pi_j,T_j)}$
\[\begin{tikzcd}[column sep = small]
0\arrow{r} & K_{B\rtimes_r \G} \arrow{r} & D \arrow{r} & A\rtimes_r \G \arrow{r} & 0
\end{tikzcd}\]
où 
\[D=\left\{\begin{pmatrix}x_1 & k_{12}\\ k_{21} & x_2\end{pmatrix} : x_j\in E_j , p_1(x_1)=p_2(x_2), k_{ij}\in K(E_j,E_i)\right\}.\]
Par naturalité du bord contrôlé \cite{OY2}, le bord de la somme de deux extensions est la somme des bords de chaque extension, d'où le résultat.
\item[(iii)] Soit $z\in KK_1^\G(A_2,B)$, représenté par un cycle $(H_B,\pi,T)$. Un représentant de $f^*(z)$ est $(H_B,f^*\pi,T)$ avec bien sûr $f^*\pi=\pi \circ f$. L'application 
\[\phi : \left\{\begin{array}{lll} E^{f^*(\pi,T)} & \rightarrow & E^{(\pi,T)} \\
( x, P_\G(f^*\pi)(x)P_\G+y) & \rightarrow & ( f_\G(x), P_\G(f^*\pi)(x)P_\G+y) \end{array}\right. \]
vérifie
\begin{enumerate}
\item[$\bullet$] $\phi(K_{B\rtimes_r \G})\subset K_{B\rtimes_r \G}$, et s'insère dans le diagramme
\[\begin{tikzcd}[column sep = small]
0\arrow{r} & K_{B\rtimes_r \G}\arrow{r}\arrow{d}{=} & E^{f^*(\pi,T)} \arrow{r}\arrow{d}{\phi}& A_1\rtimes_r \G\arrow{r}\arrow{d}{f_\G} & 0\\
0\arrow{r} & K_{B\rtimes_r \G}\arrow{r} & E^{(\pi,T)} \arrow{r}& A_2\rtimes_r \G\arrow{r} & 0
\end{tikzcd}.\]
\item[$\bullet$] Elle entrelace les sections de ces deux extensions.
\end{enumerate}
La remarque $3.7$ de \cite{OY2} assure donc que \[D_{K_{B\rtimes_r \G}, E^{f^*(\pi,T)} } =  D_{K_{B\rtimes_r \G}, E^{(\pi,T)} }\circ f_{\G,*}\], et l'assertion est claire en composant par $\mathcal M_{B\rtimes_r \G}^{-1}$.

\item[(iv)] Let $\mathcal E = H_{B_1}\otimes_g B_2$, which is a countably generated Hilbert $B_2$-module. The homomorphism $g:B_1\rightarrow B_2$ gives rise to $g_* : \mathcal L_{B_1}(H_{B_1})\rightarrow \mathcal L_{B_2}(\mathcal E)$, which preserves compact operators : $g_*(K_{B_1})\subset K(\mathcal E)$. We have a similar statement for $g_G : B_1\rtimes\G\rightarrow B_2\rtimes\G$. We denote $\mathcal E_G$ the Hilbert $B_2\rtimes\G$-module $\mathcal E\rtimes\G\simeq H_{B_1\rtimes \G}\otimes_g (B_2\rtimes \G)$.\\

Let $z\in KK^\G(A,B_1)$ be represented by the $K$-cycle $(H_{B_1},\pi,T)$. Then $(H_{B_1}\otimes_g B_2,g_*\circ\pi, g_*(T))=(\mathcal E, \tilde\pi,\tilde T)$ represents $g_*(z)$.\\

The map $(x,y)\mapsto (x, (g_G)_*(y))$ induces $\Psi :E^{(\pi,T)}\rightarrow  E^{g_*(\pi,T)} $ such that
\[\Psi(x,P_G \pi_G(x) P_G +y)\mapsto (x,\tilde P_G \tilde\pi_G(x) \tilde P_G+(g_G)_*(y)).\]
Indeed, the crossed-product functor commutes with pull-back by $\G$-morphisms, and $(g_G)*\circ\pi_G=(g_*\circ\pi)_G=\tilde \pi_G$ and $(g_G)_*(P_G) = g_*(P)_G=\tilde P_G$ so that 
\[(g_G)_*(P_G \pi_G(x) P_G)=\tilde P_G \tilde\pi_G(x) \tilde P_G. \]
Now, by the equivariant stabilisation lemma of Le Gall \cite{LeGall}, we know that the countably generated Hilbert module $\mathcal E_G$ sits as a complemented module of $H_{B_2\rtimes\G}$, and there exists a projection $p\in L(H_{B_2\rtimes\G})$ such that $pH_{B_2\rtimes\G}\simeq \mathcal E_\G$ and $pK_{B_2\rtimes\G}p\simeq K(\mathcal E_G)$. Let $\psi$ be the composition $K_{B_1\rtimes \G}\rightarrow_{(g_G)_*} K(\mathcal E_\G)\rightarrow K_{B_2\rtimes \G}$. In this particular case, we can give an explicit description of $\psi$. The map defined on basic tensor products $(x_j)_{j}\otimes b\mapsto (g(x_j)b)_j $ extends to an isometric embedding $\mathcal E_\G \rightarrow H_{B_2\rtimes \G}$, under which $ b\theta_{e_i,e_j}$ is mapped to $g(b)\theta_{u_i,u_j}$, where $\{e_j\}$ and $\{u_j\}$ are respectively the canonical orthogonal basis of $H_{B_1 \rtimes\G}$ and $H_{B_2 \rtimes\G}$. This gives a commutative diagram 
\[\begin{tikzcd}[column sep = small]
0\arrow{r} & K_{B_1\rtimes \G}\arrow{r}\arrow{d}{\psi} & E^{(\pi,T)} \arrow{r}\arrow{d}{\Psi}& A\rtimes_r \G\arrow{r}\arrow{d}{=} & 0\\
0\arrow{r} & K_{B_2\rtimes \G}\arrow{r} & E^{g_*(\pi,T)} \arrow{r}& A\rtimes \G\arrow{r} & 0
\end{tikzcd}.\]
and $\Psi$ intertwines the two filtered sections by the previous relation. Moreover, $\Psi_{|K_{B_1\rtimes \G}}\subset K_{B_2\rtimes\G}$, so that we can again apply the remark $3.7$ of \cite{OY2} to state
\[ D_{K_{B_2\rtimes \G},E^{g_*(\pi,T)}}=\psi_*\circ D_{K_{B_1\rtimes \G},E^{(\pi,T)}},\]
which we compose by the Morita equivalence on the left $M_{B_2\rtimes\G}^{-1}$
\[J_\G(g_*(z)) = M_{B_2\rtimes\G}^{-1}\circ g_{G,*}\circ D_{K_{B_1\rtimes \G},E^{(\pi,T)}}.\]
The homomorphisms inducing the Morita equivalence make the following diagram commutes,
\[\begin{tikzcd}B_1\rtimes\G\arrow{r}{g_\G}\arrow{d} & B_2\rtimes \G\arrow{d} \\ K_{B_1\rtimes\G } \arrow{r}{\psi}& K_{B_2\rtimes\G }\end{tikzcd},\]
and $J_\G(g_*(z))= g_{G,*}\circ M_{B_1\rtimes\G}^{-1}\circ D_{K_{B_1\rtimes \G},E^{(\pi,T)}}=g_{G,*}\circ J_\G(z)$.\\
\qed
\end{enumerate}
\end{dem}

We now show that the controlled Kasparov transform respects in a quantitative way the Kasparov product.

\begin{prop} There exists a control pair $(\alpha,h)$ such that for every $\G$-$C^*$-algebra $A$, $B$ and $C$, and every $z\in KK^\G(A,B),z'\in KK^\G(B,C)$, the controlled equality
\[J_\G(z\otimes_B z') \sim_{\alpha,h} J_\G(z')\circ J_\G(z)\]
holds.
\end{prop}
\begin{dem}
We will use the following fact : there exists a positive integer $d$ such that every cycle $z\in KK^\G(A,B)$ has decomposition property $(d)$. For more details, we send to the appendice of the article of V. Lafforgue \cite{LaffOY} where H. Oyono-Oyono shows that claim. We just need to know that $z$ satisfies the decomposition property $(d)$ if there exist $d+1$ $\G$-$C^*$-algebras $A_j$  and $d$ cycles $\alpha_j\in KK^\G(A_{j-1},A_j), j=1,d$ such that $A_0=A$, $A_d=B$ and each $\alpha_j$ is either coming from a $*$-morphism $A_{j-1}\rightarrow A_j$, or there is a $*$-morphism $\theta_j: A_j\rightarrow A_{j-1}$ such that $\alpha_j \otimes_{A_j} [\theta_j]=1$ in $KK^G(A_{j-1},A_{j-1})$.\\

This property reduces the proof to the special case of $\alpha$ being the inverse of a morphism in $KK^\G$-theory : $\alpha\otimes[\theta]=1$, then :
\[\begin{array}{rcl}
J_\G (\alpha\otimes z) & \sim &  J_\G(\alpha\otimes z)\circ J_\G(\alpha\otimes [\theta]) \\
			& \sim & J_\G(\alpha\otimes z)\circ J_\G(\theta_*(\alpha))\\
			& \sim & J_\G(\alpha\otimes z)\circ \theta_{\G,*}\circ J_\G(\alpha)\\
			& \sim & J_\G(\theta^*(\alpha\otimes z))\circ J_\G(\alpha)\\
			& \sim & J_\G(z)\circ J_\G(\alpha) \\
\end{array}\] 
because $\theta^*(\alpha\otimes z)=\theta^*(\alpha)\otimes z=1\otimes z =z$. As $d$ is uniform for all locally compact groupoids with Haar systems, a simple induction concludes. \textbf{METTRE EXPLICITEMENT LES EXPOSANTS DE PROPAGATION}
\end{dem}

\subsection{Quantitative assembly maps}

Following the article of J.-L. Tu \cite{TuBC}, we recall that a locally compact, $\sigma$-compact and Hausdorff groupoid $G$, endowed with a Haar system $\lambda$, is said to be proper if there exists a cut-off function $c:G^{(0)}\rightarrow \R_+$ continuous such that 
\begin{itemize}
\item[$\bullet$] for all compact subset $K$ of $G^{(0)}$, $\text{supp }c\cap s(G^K)$ is compact,
\item[$\bullet$] $\int_{G^x}c(s(g))d\lambda^x(g)=1,\forall x \in G^{(0)}$.
\end{itemize}
If moreover $G^{(0)}/G$ is also compact, reducing the first condition to "$\text{supp }c$ compact", then $g\mapsto \sqrt{c(r(g))c(s(g))}$ defines a projection in $C_c(G)$ for convolution, which gives an element $[\mathcal L_G] \in K_0(C^*G)$.\\

Now when $X$ is a locally compact space which is $\G$-proper and $\G$-compact, the groupoid $X\rtimes\G$ is proper with a compact orbit base space ($X/\G$ is compact). We can then define $[\mathcal L_X]\in K_0(C_0(X)\rtimes_r \G)$ as the class of the projector for $G=X\rtimes\G$.

\subsubsection{Classifying space for proper actions}
We remind the construction of a classifying space for proper actions for a $\sigma$-compact étale groupoid, which can be found in \cite{TuBC2} and \cite{OY3}.\\
If $d\geq 0$, we set 
\[P_d(\G) = \{p\in Prob(\G) : \exists x\in \G^{(0)}, r^*p=\delta_x \text{ and } l^x(g)\leq d,\ \forall g \in \text{supp }p\}\]
endowed with the $*$-weak topology, and with the natural action of $\G$ by translation.\\
If $p\in P_d(\G)$ such that $r^*p = \delta_x$ for a certain $x\in \G^{(0)}$, we can write
\[p=\sum_{g\in \G^x} \lambda_g(p)\delta_g.\]
If we set $\phi^2(p)=\lambda_{e_x}(p)\geq 0$, we have $\phi \in C_0(P_d(\G))$ and $(g.\phi^2)(p)=\lambda_g(p)$. Now define :
\[\mathcal L_d = \sum_{g\in \G^x} \phi.(g.\phi)\in C(X,C_0(P_d(\G)))\subset C(\G,C_0(P_d(\G))) \]
because $X$ is compact. \textbf{??}
$\mathcal L_d$ is a projection of $C_0(P_d(\G))\rtimes_r \G$ without propagation, and defines a class $[\mathcal L_d]_{\epsilon,R}\in K_0^{\epsilon,R}(C_0(P_d(\G))\rtimes_r \G))$ for all $R>0,0<\epsilon<1/4$.\\

\begin{definition}
Let $B$ be a $\G$-algebra, and $R>0,0<\epsilon<1/4,d>0$. The local quantitative assembly map for $\G$ is defined as the composition of $J_\G$ with the evaluation at $[\mathcal L_d]$ :
\[\mu_B^{\epsilon,R,d}\left\{
\begin{array}{rcl}
KK^\G(C_0(P_d(\G)), B) & \rightarrow & K_*^{\epsilon, R}(B\rtimes \G)\\
z & \mapsto & J_\G^{\epsilon, R}(z)([\mathcal L_d]_{\epsilon , R})
\end{array}\right.\]
\end{definition}

\textit{Remarks}
\begin{itemize}
\item[(1)] The assembly map is defined for all reasonnable crossed-products by $\G$. In particular for the reduced one and the maximal one, so that we have two different assembly, which we would distinguish writing $J_{\G,r}$ and $J_{\G,max}$ if necessary.
\item[(2)] The bunch of assembly maps $\mu_B^{\epsilon,R,d}$ induces the Baum-Connes assembly map for $\G$ in $K$-theory : the following diagram commutes
\[\begin{tikzcd}
KK^G(C_0(P_d(\G)),B) \arrow{r}{\mu_B^{\epsilon,R,d}}\arrow{dr}{\mu_\G^d} &  K_*^{\epsilon, R}(B\rtimes \G)\arrow{d}{\iota_{\epsilon,R}} \\ 
		&  K_*(B\rtimes \G)
\end{tikzcd}\]
because $J_\G(z)$ induces the right multiplication by $j_\G(z)$ and also $\mu_\G^d(z)=[\mathcal L_d]\otimes j_\G(z)$. But, as ${{\mathcal L}_{d'}}_{|P_d(\G)}=\mathcal L_d$ as soon as $d\leq d'$, this diagram commutes with inductive limit over $d$.

\item[(3)] In \cite{OY3}, H. Oyono-Oyono and G. Yu defined a bunch of local quantitative coarse assembly maps for a metric space $X$. For the sake of simplicity, we take $X$ to be discrete and uniformly bounded. Then, for any $C^*$-algebras $A$ and $B$ and a $K$-cycle $z\in KK(A,B)$, they construct a controlled morphism
\[\sigma_X(z) : \hat K(C^*(X,A))\rightarrow \hat K(C^*(X,B)).\]
There exists a projection $P_X$ without propagation, and the local quantitative assembly map is defined as 
\[A_{X,B}^{\epsilon,r,d}(z)=\sigma_X^{\epsilon,r}(z)([P_X]_{\epsilon,r})\] for $z\in KK(C_0(P_d(X)),B)$, where $P_d(X)$ is the classical Rips complex of $X$. This bunch of assembly maps induce the usual coarse assembly map of $X$ 
\[A_{X,B} : KX_*(X,B)\rightarrow K_*(C^*(X,B)\]
in $K$-theory. Now let $\G$ be the coarse groupoid of $X$. It is an étale groupoid with compact base space $\G^{(0)}=\beta X$, the Stone-Cech compactification of $X$. A classical result of G. Skandalis, J.-L. Tu and G. Yu \cite{SkTuYu} claims that the coarse Baum-Connes conjecture for $X$ with coefficients in $B$ is equivalent to the Baum-Connes conjecture for the groupoid $G$ with coefficient in $l^\infty(X,B)$. More precisely, there is an isomorphism of $C^*$-algebras $l^\infty(X,B)\rtimes_r \G \simeq_\phi C^*(X,B)$ and the following diagram commutes :
\[\begin{tikzcd}
KK_*^\G(C_0(P_d(\G),l^\infty(X,B)) \arrow{r}{\mu_{\G,l^\infty(X,B)}^d}\arrow{d}{\iota^*}& K_*(l^\infty(X,B)\rtimes_r G)\arrow{d}{\phi_*}\\
KK_*(C_0(P_d(X),B) \arrow{r}{A_{X,B}^d}& K_*(C^*(X,B))
\end{tikzcd}\]
where the left vertical arrow comes from the inclusion of groupoid $\iota :\{x\}\rightarrow \G$ for any $x\in X$. We claim that we can prove a controlled analogue of this result which induces it in $K$-theory. \textbf{A FINIR} 
\end{itemize}

\subsection{Quantitative statements}

\begin{prop} 
Let A be a $\G$-algebra.\\
If the following statement is true :\\

$\bullet$(Quantitative Injectivity) $\forall d\geq 0$, there exists $\epsilon\in (0,\frac{1}{4})$ such that, for all $r\geq r_{d,\epsilon}$, there exists $d'\geq d$ such that if $x\in KK_*^\G(C_0(P_d(\G)),A)$ satisfies $\mu_\G^{\epsilon,R,d}(x)=0\in K^{\epsilon,R}$, then $x=0$ in $KK^\G(C_0(P_{d'}(\G)),A)$;\\

then $\mu_{\G,A}$ is injective.\\

On the other hand, if this statement is true : \\

$\bullet$(Quantitative Surjectivity) there exists $\epsilon'\in (0,\frac{1}{4})$ such that $\forall r'\geq r_{d,\epsilon},\exists \epsilon,r$ such that $\epsilon'\leq \epsilon<frac{1}{4}$ and $r_{d,\epsilon}\leq r\leq r'$, such that for all $y\in K_*^{\epsilon',r'}(A\rtimes\G),\exists x \in KK_*^\G(C_0(P_d(\G)),A)$ such that $\mu_{\G,A}^{\epsilon',r',d}=\iota_{\epsilon,r}^{\epsilon',r'}(y)$;\\

then $\mu_{\G,A}$ is surjective.
\end{prop}

\begin{dem}
Let $x\in KK(C_0(P_d(\G)),A)$ which satisfies $\mu_{\G,A}(x)=0$, then $\iota_{\epsilon,r}\circ\mu_{\G,A}^{\epsilon,r,d}(x)=0$. By remark $1.18$ of \cite{OY2}, there exists a universal $\lambda>0$ and a certain $r'>0$ such that
\[\begin{array}{lll}0 &  =  & \iota_{\epsilon,r}^{\lambda\epsilon,r'}\circ \mu_{\G,A}^{\epsilon,r,d}(x) \\
			& = & \iota_{\epsilon,r}^{\lambda\epsilon,r'} (J_{\G}^{\epsilon,r}(x)([\mathcal L_d]_{\epsilon,r})) \\
			& = & J_{\G}^{\lambda\epsilon,r'}(x)([\mathcal L_d]_{\lambda\epsilon,r'}) \\
			& = & \mu_{\G,A}^{\lambda\epsilon,r',d}(x).
\end{array}\]
But then the quantitative injectivity condition assures that $x=0$ in $KK^\G(C_0(P_{d'}),A)$ and $x=0$ in the inductive limite over $d$ $K^{top}(G,A)$.\\
The second point is immediate. 
\end{dem}

To prove the theorem, we will need a lemma.\\

\begin{lem}
Let $\G$ be an étale groupoid, $\Delta$ a CW-complex \textbf{conditions}, $\{B_j\}_{j\geq 0}$ a family of $\G$-algebras and $K$ the algebra of compact operators over a separable Hilbert space. Then we have an $\Z_2$-graded ismorphism 
\[KK^\G(C_0(\Delta),\prod_j B_j\otimes K)\simeq \prod_j KK^\G(C_0(\Delta),B_j)\]
\end{lem}

\begin{dem}
\textbf{A FAIRE}
\end{dem}








































