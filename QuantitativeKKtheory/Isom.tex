\section{Assembly maps for groupoids and for coarse spaces}

\subsection{The case of a finitely generated group}
Let $\Gamma$ be a discrete finitely generated group. The word length provides a structure of metric space, of which the class up to coarse equivalence is independent of the set of generators.\\

Denoting $C^*(\Gamma)$ the Roe algebra, i.e. the $C^*$-algebra generated by locally compact operators on $l^2(\Gamma)\otimes H$ with finite propagation, we can show that 
\[C_u^*(\Gamma) = l^\infty (\Gamma) \times_\alpha \Gamma\]
\[C^*(\Gamma) = l^\infty (\Gamma,\mathfrak K (H)) \times_\alpha \Gamma.\]
Here $\alpha \in Aut(A)$ is the automorphism encoding the left action of $\Gamma$ on $A$ :
\[\alpha_\gamma ( a ) = s\rightarrow a_{s\gamma^{-1}}.\]
Let $S_\gamma$ be the operator acting on $l^2(\Gamma)$ as 
\[(S_\gamma \eta)_s=\eta_{s\gamma^{-1}}.\]
We see $l^\infty (\Gamma)$ as an algebra of operator, acting by left multiplication on $l^2(\Gamma)$. Then 
\[S_\gamma aS_\gamma^* = \alpha_\gamma(a),\]
for any $a\in l^\infty(\Gamma)$ and $\gamma\in\Gamma$. The algebra $C^*(\Gamma)$ is generated by finite sums of the form
\[\sum_\gamma a_\gamma S_{\gamma}\]
which are of finite propagation $\max_\gamma \{l(\gamma) : a_\gamma\neq 0\}$ and locally compact.  

\subsection{Relation between the coarse and the groupoid assembly maps}

We have to show that there is an isomorphism 
\[KX_*(X)\rightarrow KK^{top}_*(G(X),l^\infty (X, \mathfrak K)).\]

Let us recall that the Stone-Cech compactification of our coarse groupoid $\Gamma=G(X)$ identified itself to the spectrum of the bounded continuous functions over $X$, which is discrete. We have
\[C(\beta X)\simeq l^\infty(X)\]
and we can think of $C(\beta X)$-algebras as $l^\infty (X)$-algebras.\\

The left handside $KX_*(X)$ is defined as the limit of the directed groups 
\[KK_*(C_0(P_E(X),\mathbb C)\]
when $E$ is an entourage of $X$. Here $P_E(X)$ denotes the Rips complex defined by the entourage $E$, which is the set of simplexes $[x_0,....,x_n]$ such that $(x_i,x_j)\in E$.\\

 Now the classifying space $\mathcal E \Gamma$ of the groupoid $G(X)$ is unique up to homotopy, and can be realised by the space of measures $\mu$ on $G(X)$ which satisfied $s^*  \mu$ is a Dirac measure on $G^{(0)}=\beta X$, and $\frac{1}{2}<|\mu|\leq 1$. Saying that $s^* \mu$ is a Dirac measure is the same as demanding $\mu$ to be supported in a fiber $\Gamma_x$ for some $x\in X$. The abelian group $KK^{top}_*(G(X),l^\infty (X, \mathfrak K))$ is defined as the inductive limit of 
\[KK_{G(X)}\left(C_0(Y),l^\infty (X,\mathfrak K)\right)\]
when $Y$ is a $\Gamma$-compact space of $\mathcal E \Gamma$.\\

Let $E$ be an entourage of $X$. A Fredholm module $(H, \phi,F)$ in $E(C_0(P_E(X)),\mathbb C)$ is defined by a Hilbert space $H$, a $*$-homomorphism $\phi : C_0(P_E(X))\rightarrow \mathcal L(H)$ and an operator $F$ satisfying all definitions. \\
We can form the $l^\infty (X,\mathfrak K)$-module $\mathcal E = H\otimes_{\mathbb C} l^2(\Gamma,\mathfrak K) )$, and extend $\phi$ into $\phi\otimes id : C_0(P_E(X))\rightarrow \mathcal L(H\otimes l^2(\Gamma,\mathfrak K)$. We do the same with $F$ : $\hat F := F\otimes id$.
Then, as $P_E(X)$ identifies itself as a $G$-compact of $\mathcal E G$, $(\mathcal E,\phi\otimes id,F\otimes id)$ defines an element of $KK_{G(X)}\left(C_0(Y),l^\infty (X,\mathfrak K)\right)$.



\section{Correspondance between the coarse $K$-homology of a space and the one of its coarse groupoid}

The aim of this section is to give a proof of a result of ~\cite{TuBC}, in which it is stated that the following diagramm commutes :

\[\begin{tikzcd}KX_* (X,B)\arrow{r}{A}\arrow{d}{\simeq} & K_*(C^*X,B) \arrow{d}{\simeq}\\
K_*(G(X),l^{\infty}(X,B))\arrow{r}{\mu} & K_*(C_r(G(X)),B).
\end{tikzcd}\]

The vertical arrow from the left comes from an isomorphism at the $C^*$-algebraic level, as
\[C^*(X) \simeq l^\infty(X)\times G(X).\] % mieux decrire : il y en a deux. Et le prouver avant ou en appendice
The rest of this section is devoted to describe the vertical arrow from the right in the langage of Kasparov $KK$-theory, i.e.
\[\varinjlim_d KK(C_0(P_d(X)),B)\rightarrow \varinjlim_{Y\subset \mathcal E G(X)} KK(C_0(Y),B),\]
were the inductive limite on the right is taken among the proper $G(X)$-compact subsets $Y$ of the universal classifying space for proper actions of $G(X)$.\\

Recall from ~\cite{TuNovikov} that we can take for $\mathcal E G(X) $ the space $\mathfrak M$ of positive measures $\mu$ on $G(X)$ satisfying :
\begin{itemize}
\item $\frac{1}{2}<\mu(G(X))\leq 1$, 
\item $s^*\mu$ is a Dirac measure, i.e. its support consists of arrows of $G(X)$ that all source from the same base point of $\beta X$.\\
\end{itemize}

If $\mathfrak M_d$ denotes the space of measures $\mu$ of $\mathfrak M$ such that :
\begin{itemize}
\item $\mu$ is a probability measure
\item for all $\gamma$ and $\gamma'$ in the support of $\mu$, $\gamma'\gamma^{-1}$ is $d$-controlled, i.e. $d(r(\gamma),r(\gamma'))\leq d$, 
\end{itemize}
then $\mathfrak M=\varinjlim \mathfrak M_d$.\\ % à vérifier

The Rips complex of $X$, denoted $P_d(X)$, is the topological space of the complexes of diameter less than $d$, identified with probability measures on $X$ with support of diameter less than $d$, with the weak topology coming from $C_c(C)$. We will write $[y,t]$ for a point of a simplex defined by barycentric coordinates of $k$ points $y_1,...,y_k$, ie $\sum t_j \delta_{y_j}$. To such a point $[y,t]$ and an element of the Stone-Cech compactification $w\in \beta X$, we can associate a measure of $\mathfrak M_d$ in the following way. As $G(X)$ is a principal and transitive groupoid, there exists only one arrow $\gamma_j$ such that $s(\gamma_j)=x$ and $r(\gamma_j)=y_j$. To $z=([y,t],w)=(z_w,w)$, we associate 
\[\phi_d(z)=\sum_{j=1,k} t_j \delta_{\gamma_j}\in\mathfrak M_d.\]

\begin{prop}
The map
\[\phi_d: P_d(X)\times \beta X \rightarrow \mathfrak M_d\]
is an homeomorphism.
\end{prop}
\begin{dem}
It is clearly bijective. The bicontinuity comes from the identity :
\[\langle z_w,f\rangle=\langle \phi_d(z),f\circ r\rangle\]
for all $z=(z_w,w)\in P_d(X)\times \beta X$, and $f\in C_c(X)$.
\qed
\end{dem}

This homeomorphism $\phi_d$ gives an $*$-isomorphism at the level of $C^*$-algebras
\[\Psi_d : C_0(\mathfrak M_d)\rightarrow C_0(P_d(X)\times \beta X).\]

Let $(\mathcal E, \pi, F)\in \mathbb E(C_0(P_d(X)),B)$ be an elliptic operator. 
