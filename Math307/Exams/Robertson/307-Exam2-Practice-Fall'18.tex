\documentclass[11pt]{exam}
\RequirePackage{amssymb, amsfonts, amsmath, latexsym, verbatim, xspace, setspace}
\RequirePackage{tikz, pgflibraryplotmarks}
\usepackage[margin=1in]{geometry}
\newcommand{\class}{Math 307	}
\newcommand{\term}{Fall 2018}
\newcommand{\examnum}{Exam 2 }
\newcommand{\examdate}{10/25/18}
\newcommand{\timelimit}{75 Minutes}
\singlespacing
\parindent 0ex
\newcommand{\R}{\mathbb{R}}

\begin{document}
\pagestyle{head}
\firstpageheader{}{}{}
\runningheader{\class}{\examnum\ - Page \thepage\ of \numpages}{\examdate}
\runningheadrule
\begin{flushright}
\begin{tabular}{p{2.8in} r l}
\textbf{\class} & \textbf{Name (Print):} & \makebox[2in]{\hrulefill}\\
\textbf{\term} &&\\
\textbf{\examnum} &&\\
\textbf{\examdate} &&\\
\textbf{Time Limit: \timelimit} 
\end{tabular}\\
\end{flushright}
\rule[1ex]{\textwidth}{.1pt}

% These commands set up the running header on the top of the exam pages
\pagestyle{head}
\firstpageheader{}{}{}
\runningheader{\class}{\examnum\ - Page \thepage\ of \numpages}{\examdate}
\runningheadrule

\begin{flushleft}
\begin{center}

\end{center}
\end{flushleft}

\hfill
\begin{minipage}[t]{4.3in}
\vspace{0pt}
%\cellwidth{3em}
\gradetablestretch{2}
\vqword{Problem}
\addpoints % required here by exam.cls, even though questions haven't started yet.	
\gradetable[v]%[pages]  % Use [pages] to have grading table by page instead of question

\end{minipage}


\pagebreak

\begin{questions}

\addpoints
\question
Let $S: \R^2 \rightarrow R^2$ and $T:R^2 \rightarrow \R^2$ 
be transformations defined by

$$ S \begin{bmatrix} x \\ y \end{bmatrix} = \begin{bmatrix}
2x+y \\ x-y \end{bmatrix},
\quad  \quad T \begin{bmatrix} x \\ y \end{bmatrix} = \begin{bmatrix}
x-4y \\ 3x \end{bmatrix}$$

\begin{parts}

\part[5] Show that $S$ and $T$ are both linear transformations. 

\vfill
\part[5] Find $ST \begin{bmatrix} x \\ y \end{bmatrix}$ and $T^2 \begin{bmatrix} x \\ y \end{bmatrix}$.
\vfill

\part[5] Find the matrices of $S$ and $T$ with respect to the standard basis for $\R^2$.

\end{parts}

\vfill

\newpage
\addpoints
\question
\begin{parts}
\part[10] 
Prove or provide a counter example: If $T$ is a linear transformation, then so is $T+T^2$.


\vfill

\part[10] Let $D : C^{\infty}(-\infty,\infty) \rightarrow C^{\infty}(-\infty,\infty)$ be the usual derivative operator. Find a basis for the kernel of the operator
$$D^2-4D+4$$

\vfill



\end{parts}


\newpage
\addpoints
\question


 Let $\alpha$ be the standard basis for $\R^3$ and $\beta$ the basis consisting of 

$$\begin{bmatrix}
1 \\
0 \\
1
\end{bmatrix},
\begin{bmatrix}
1 \\
1 \\
0
\end{bmatrix},
\begin{bmatrix}
0 \\
0 \\
1
\end{bmatrix}
$$
\begin{parts} 
\part[5] Find the change of basis matrix from $\alpha$ to $\beta$.
\vfill
\part[10] Find the change of basis matrix from $\beta$ to $\alpha$. 
\vfill
\vfill
\part[10] Define $T: \R^3 \rightarrow R^3$ by 
$T\begin{bmatrix}
x \\
y\\
z
\end{bmatrix}
=\begin{bmatrix}
x \\
2y\\
3z
\end{bmatrix}
$, Find $[T]_{\alpha}^{\alpha}$.
\vfill

\part[10] Express $[T]_{\beta}^{\beta}$ as the product of the three matrices found above.  
\vfill
\end{parts}



\newpage
\addpoints
\question
\begin{parts}
\part[10] Let $A$ be an $n \times n$ matrix. Prove that a 
number $\lambda$ is an eigenvalue of $A$ if and only if 
$$\det( \lambda I - A)=0.$$


\vfill
\part[10] Let $A = \begin{bmatrix} 2 & 2 \\ 1 & 3 \end{bmatrix}$. Find the eigenvalue(s) and associated eigenvectors of $A$.   
\vfill

\part[10] Define what it means for $A$ to be similar to a matrix $B$, and what it means for $A$ to be diagonalizable.  Prove that if $A$ is similar to $B$, and $A$ is diagonalizable, then $B$ is also diagonalizable.   

\vfill
\end{parts}

\newpage
\addpoints
\question
Let $A = \begin{bmatrix} 2 & 2 & 0\\ 0 & 3 & 0 \\ 0 & 0 & 1 \end{bmatrix}$
\begin{parts}
\part[10] Find the eigenvalues and associated eigenvectors of $A$.  

\vfill

\part[10] Determine if $A$ is diagonalizable. If it is, give the matrix $P$ and the diagonal matrix $D$ such that $P^{-1} A P = D$.  

\vfill









\end{parts}

\newpage
\addpoints
\question Suppose that $A$ is a matrix with characteristic polynomial $p(\lambda)=(\lambda -3)^2(\lambda-2)^2$. 

\begin{parts}

\part[10] If $\dim(E_3) = 2$ and $\dim(E_2) = 2$ what is the Jordan Normal Form of $A$?

\vfill

\part[10] If $\dim(E_3) = 1$ and $\dim(E_2) = 2$ what is the Jordan Normal Form of $A$?

\vfill

\end{parts}
\newpage
\addpoints
\question[20] Suppose that $A$ is a matrix with characteristic polynomial $p(\lambda)=(\lambda +1)^2(\lambda-5)^4$. If we decide on a Jordan Normal Form, $J$, of $A$ as $$J= \begin{bmatrix}
B_1 & 0 \\
0 & B_2 
\end{bmatrix}$$ where $B_1$ is a $2 \times 2$ matrix and $B_2$ is a $4 \times 4$ matrix, what are the possibilities (up to permutation of the Jordan blocks) of $B_1$ and $B_2$?  

\newpage
\addpoints
\question[20]


Is the following $n \times n$ matrix diagonalizable?


$$
\begin{bmatrix}
    1 & 1 & 0 & \dots  & 0 \\
    0 & 1 & 0 & \dots  & 0 \\
    0 & 0 & 1 & \dots & 0 \\
    \vdots & \vdots & \vdots & \ddots & \vdots \\
    0 & 0 & 0 & \dots  & 1
\end{bmatrix}
$$

\end{questions}
\end{document}