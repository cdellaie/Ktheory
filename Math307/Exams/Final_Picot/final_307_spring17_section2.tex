\documentclass[11pt]{article}
\usepackage{amsmath}
\usepackage{amssymb} % amscd, syntonly
\usepackage{verbatim}
\usepackage{array}
\usepackage{eufrak}  % redundant if amsfonts package is used
\usepackage{amsfonts}
\usepackage{epic}
\usepackage{eepic}
\usepackage[dvips]{graphics}
%\usepackage{geometry}
%\geometry{top=3cm, bottom=1cm}% left=''longueur '', right=''longueur''
\usepackage{multicol}
\setlength{\columnsep}{1.5cm}
\setlength{\columnseprule}{0.2pt}
\addtolength{\textwidth}{7cm}
\addtolength{\oddsidemargin}{-3.8cm}
\addtolength{\textheight }{1.5cm} 
\usepackage{fancyhdr}
\usepackage{comment}
\setlength{\topmargin}{-50pt} % Pas de marge en haut
\fancypagestyle{garde}{% 
\pagestyle{fancy}
\fancyhead[LO]{Last name, first name, ID number : }
\fancyhead[RO]{Score: \hskip.5cm /50}
\fancyhead[CO]{\textbf{Final Exam, Math 307, section 1, Spring 2017 \vskip.3cm}}
%\renewcommand{\headrulewidth}{0cm}
}

\begin{document}

\thispagestyle{garde}
\vskip 1cm

\begin{enumerate}


\item (22 points)Let's consider the non-homogeneous first order linear differential system
\begin{equation*}
  \left\{
        \begin{array}{l}
       y_1'(t)=-4y_1(t)-3y_2(t)+3y_3(t)\\
       y_2'(t)=3y_1(t)+2y_2(t)-3y_3(t)+e^t\\
       y_3'(t)=-3y_1(t)-3y_2(t)+2y_3(t).\\
        \end{array}
    \right.
\end{equation*}


where 

\begin{enumerate}
\item (1 point) Write the system in the matrix form $Y'(t)= AY(t)+G(t)$.
\item (6 points) Find a diagonal matrix $D$ and an invertible matrix $P$ such that $D=P^{-1}AP$.
\item (2 points) Find the general solution to the homogeneous system $Z'(t)=DZ(t)$.
\item (6 points) Use the result of the previous question to find the general solution $Y_H(t)$to the homogeneous system $Y'(t)=AY(t)$ and find a fundamental set $Y_1(t),\ Y_2(t),\ Y_3(t)$ of solutions to this system.
\item (6 points) Using the matrix $M(t)=[Y_1(t) \ Y_2(t) \ Y_3(t)]$, compute a particular solution $Y_p(t)$ to the non-homogeneous system $Y'(t)= AY(t) + G(t)$.
\item (1 point) Use the result of the previous question to find the general solution to the non-homogeneous system $Y'(t)= AY(t) + G(t)$.   
\end{enumerate}

%\begin{comment}
\underline{Solution}
\begin{enumerate}
\item Y'(t)= AY(t) + G(t) with $A= \begin{pmatrix} -4& -3& \phantom{-}3 \\ \phantom{-}3 & \phantom{-}2& -3 \\ -3& -3& \phantom{-}2\\ \end{pmatrix}$ and $G(t)=\begin{pmatrix} 0 \\ e^t \\0  \end{pmatrix}$.
\item  $D= \begin{pmatrix} -1& \phantom{-} 0 & \phantom{-} 0 \\ \phantom{-}0&-1 &\phantom{-}0 \\  \phantom{-}0&  \phantom{-}0&  \phantom{-}2 \end{pmatrix}$,   $P= \begin{pmatrix}  \phantom{-} 1& \phantom{-} 0 & \phantom{-} 1 \\ \phantom{-} 0 &  \phantom{-} 1 &  -1 \\  \phantom{-}1&  \phantom{-} 1&  \phantom{-} 1 \end{pmatrix}$ and $P^{-1}= \begin{pmatrix}  \phantom{-} 2&  \phantom{-} 1 &  -1 \\ - 1 &  \phantom{-} 0 &\phantom{-} 1\\ -1 &  -1  & 1 \end{pmatrix}$

\item $Z(t)= \begin{pmatrix} c_1e^{-t}\\c_2e^{-t}\\ c_3e^{2t} \end{pmatrix}$,  for  $c_1, \ c_2$ and $c_3$ 3 real constants.

\item $Y_H(t)=PZ(t)=\begin{pmatrix} c_1e^{-t}+c_3e^{2t} \\c_2e^{-t}-c_3e^{2t} \\c_1e^{-t}+c_2e^{-t}+c_3e^{2t}  \end{pmatrix}$ for $c_1, \ c_2$  and $c_3$ 3 real constants. Thus
\[
Y_1(t)=\begin{pmatrix} e^{-t}\\0\\e^{-t}\end{pmatrix}, \ Y_2(t)=\begin{pmatrix} 0 \\ e^{-t} \\ e^{-t} \end{pmatrix}, \ Y_3(t)=\begin{pmatrix} \phantom{-}e^{2t} \\ -e^{2t} \\  \phantom{-}e^{2t}  \end{pmatrix}.
\]

\item $M(t)=\begin{pmatrix}  e^{-t} &0 & e^{2t}  \\0&e^{-t}&-e^{2t} \\e^{-t}&e^{-t} &e^{2t}\end{pmatrix} $, $ M^{-1}(t)=\begin{pmatrix} 2e^t& e^t & -e^{t} \\ -e^t &  0 & e^t\\ -e^{-2t} &  -e^{-2t} &  e^{-2t} \end{pmatrix} $, $\int M^{-1}(t) G(t)dt=\begin{pmatrix}\frac{e^{2t}}{2}\\ 0 \\ e^{-t} \end{pmatrix} $ so


 \[ Y_p(t)=M(t)\int M^{-1}(t) G(t) dt=\begin{pmatrix}\frac{3e^t}{2}\\ -e^{t} \\ \frac{3e^t}{2} \end{pmatrix} \]


\item The general solution to the non homogeneous system is 
\[
Y(t)=Y_H(t)+Y_p(t)=\begin{pmatrix} c_1e^{-t}+c_3e^{2t} +\frac{3e^t}{2}\\c_2e^{-t}-c_3e^{2t}-e^{t}\\c_1e^{-t}+c_2e^{-t}+c_3e^{2t}+\frac{3e^t}{2}  \end{pmatrix} 
\]
for $c_1, \ c_2$ and $c_3$ \text{ 3 real constants.}


\end{enumerate}
%\end{comment}

\newpage


 \item (9 points) Consider the homogeneous first order linear differential system
\begin{equation*}
  \left\{
        \begin{array}{l}
       y_1'(t)=y_1(t)-y_2(t)+y_3(t)\\
       y_2'(t)=2y_2(t)+1y_3(t)\\
       y_3'(t)=2y_3(t).\\
        \end{array}
    \right.
\end{equation*}




\begin{enumerate}
\item (1 point) Write the system in the matrix form $Y'(t)= AY(t)+G(t)$.
\item (2 points) Is the matrix $A$ diagonalizable? Explain your answer.
\item (6 points) Find the general solution to the homogeneous system $Y(t)'= AY(t)$. 
%(You will need to use the method of the integrating factor twice.)
\end{enumerate}


%\begin{comment}
\underline{Solution}
\begin{enumerate}
\item Y'(t)= AY(t) 
where $A= \begin{pmatrix}  \phantom{-}1&  -1& \phantom{-}1 \\ \phantom{-} 0& \phantom{-}2&  \phantom{-}1 \\ \phantom{-}0& \phantom{-}0&\phantom{-} 2\\ \end{pmatrix}$.
\item The matrix $A$ has two eigenvalues which are 1 and 2. We find $E_2=\text{Span}\Big \lbrace   \begin{pmatrix}  -1\\ 1 \\ 0 \end{pmatrix}  \Big \rbrace$  and $E_1=\text{Span}\Big \lbrace   \begin{pmatrix}  1\\ 0 \\ 0 \end{pmatrix}  \Big \rbrace$. Therefore, $\text{dim}(E_1)+\text{dim}(E_2)=2\neq 3$ so $A$ is not diagonalizable. 
\item From the third equation, we find $y_3(t)=c_3e^{2t}$. Plugging in the second equation and applying the method of the integrating factor, we find $y_2(t)=c_3te^{2t}+c_2e^{2t}$.  Plugging in the first equation and applying the method of the integrating factor once again, we get $y_1(t)=-c_3te^t+2c_3e^{2t}-c_2e^{2t}+c_1e^t$. Therefore, the general solution to the system is
\[
 Y(t)=\begin{pmatrix} -c_3te^{2t}+2c_3e^{2t}-c_2e^{2t}+c_1e^t\\ c_3te^{2t}+c_2e^{2t}\\ c_3e^{2t} \end{pmatrix}.
\]
for $c_1, \ c_2$ and $c_3$ \text{ 3 real constants.}
\end{enumerate}
%\end{comment} 

\newpage

\item (11 points) Let $T$ be the transformation
\begin{align*}
T:P_2 &\longrightarrow P_2\\
ax^3+bx^2+cx+d &\longrightarrow (c+d)x^3+bx^2 -c-d
\end{align*}
\begin{enumerate}
\item (1 points) Show that $T$ is a linear transformation.
\item (4 points) Find a basis for $\text{Ker}(T)$ and the dimension of  $\text{Ker}(T)$.
\item (3 points) Consider the basis $\alpha = \lbrace x^3+x^2, \ x^3-1, \ x+1, \ x^2+x  \rbrace$. Find $[T]^{\alpha}_{\alpha}.$
\item (3 points) Let $p$ be the polynomial function in $P_3$ whose coordinate vector relatively to $\alpha$ is 
\[ [p]_{\alpha}=\begin{pmatrix} 1\\ 2\\ -1\\ 1 \end{pmatrix}.\]
Find $[T(p)]_{\alpha}.$
\end{enumerate}

\begin{comment}
\underline{Solution:}

\begin{enumerate}
\item Straightforward
\item A polynomial function $ax^3+bx^2+cx+d$ is in  $\text{Ker}(T)$ if and only if 
$c+d=0, \ b=0 \text{ and } -c-d=0$
so if and only if $ \begin{pmatrix} a\\ b \\ c\\ d \end{pmatrix}=a\begin{pmatrix} 1\\ 0 \\ 0\\ 0 \end{pmatrix}+ c\begin{pmatrix} 0\\ 0 \\ 1\\-1 \end{pmatrix}.$ Therefore, the  basis for $\text{Ker}(T)$ consists of the polynomial functions $x^3$ and $x-1$.
\item Set of four linearly independent vectors in a vector space of dimension four so basis.
\item Find $[T]^{\alpha}_{\alpha}=\begin{pmatrix} \frac{1}{2}&0&0&\frac{1}{2}\\ -\frac{1}{2}&1&2&\frac{1}{2} \\ -\frac{1}{2}&0&0&-\frac{1}{2}\\ \frac{1}{2}&0&0&\frac{1}{2} \end{pmatrix}$.
\item $[T(p)]_{\alpha}=\begin{pmatrix} 1 \\ 0 \\ -1 \\ 1 \end{pmatrix}$
\end{enumerate}
\end{comment}


\newpage
\item (13 points) Random questions. You will not get credits if you do not explain your answers.
\begin{enumerate} 
\item (3 points) Consider the linear transformation 
\begin{align*}
T:P_3 &\longrightarrow P_3\\
ax^3+bx^2+cx+d &\longrightarrow (a+2c)x^3-2a-4c.
\end{align*} Is the polynomial function $p(x)=2x^2+1$ an eigenvector of $T$? Explain your answer. \\
%\underline{Solution:} Yes, since $T(p)=0$, it is an eigenvector associated with the eigenvalue $0$.

\vspace{2.5cm}

\item (3 points) Let $v$ be an eigenvector of a matrix $A$ associated with an eigenvalue $\lambda$ and let $c$ be a non-zero scalar. Is the vector $c.v$ an eigenvector of $A$ associated with $\lambda$? \\
%\underline{Solution:} Yes, since $A(c.v)=c.Av=c\lambda v=\lambda.cv$.

 \vspace{2.5cm}


\item (3 points) Let $v_1$ and $v_2$ be two linearly independant vectors in a vector space $V$ and let $c_1$ and $c_2$ be two scalars. What is the dimension of $\text{Span}\lbrace v_1,v_2, c_1 v_1,c_1 v_1+c_2 v_2 \rbrace$?
 \\
%\underline{Solution:} The dimension is 2 since $\text{Span}\lbrace v_1,v_2, \alpha v_1,\alpha v_1+\beta v_2 \rbrace$=$\text{Span}\lbrace v_1,v_2\rbrace$ 

\vspace{2.5cm}

\item (3 points) Find two basis $\alpha$ and $\beta$ for $\mathbb{R}^2$ such that the matrix $\begin{pmatrix} 1 &1 \\ -1 & 1 \end{pmatrix}$ is the change of basis matrix from $\alpha$ to $\beta$.\\
%\underline{Solution:} $\alpha$ is the canonical basis and $\beta= \begin{pmatrix} 1 &1\end{pmatrix}, \  \begin{pmatrix} -1 & 1 \end{pmatrix}$. 

\vspace{2.5cm}


\item (1 point) Write a grammatically correct statement, about linear algebra or differential equations, that you learned this semester. You are not allowed to use any mathematical symbols. 
 \\

\vspace{2cm}

\end{enumerate}


\end{enumerate}

\end{document}
