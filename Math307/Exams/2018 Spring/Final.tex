\documentclass{article}
\usepackage{amsfonts, graphicx, multicol, amssymb, amsmath}
%usepackage{ulem}
\usepackage[left=.4in,top=.4in,right=.4in,nohead,nofoot]{geometry}
\newcommand{\example}{\subsubsection*{Example}}


\newcommand\tr{\textrm{tr}}
\newcommand{\vecxx}[2]{\begin{bmatrix} #1\\ #2\\ \end{bmatrix}}
\newcommand{\vecxxx}[3]{\begin{bmatrix} #1\\ #2\\ #3 \end{bmatrix}}
\newcommand{\vecxxxx}[4]{\begin{bmatrix} #1\\ #2\\ #3\\ #4 \end{bmatrix}}

\newcommand{\svecxx}[2]{\left[\begin{smallmatrix} #1\\ #2\\ \end{smallmatrix}\right]}
\newcommand{\svecxxx}[3]{\left[\begin{smallmatrix} #1\\ #2\\ #3 \end{smallmatrix}\right]}
\newcommand{\svecxxxx}[4]{\left[\begin{smallmatrix} #1\\ #2\\ #3\\ #4 \end{smallmatrix}\right]}

\newcommand\matxx[4]{\begin{bmatrix}#1 & #2\\ #3 & #4\end{bmatrix}}
\newcommand\matxxx[9]{\begin{bmatrix}#1 & #2 & #3\\ #4& #5 & #6\\ #7 & #8 & #9\end{bmatrix}}

\newcommand\diagxx[2]{\begin{bmatrix}#1 & 0\\ 0 & #2\end{bmatrix}}
\newcommand\diagxxx[3]{\begin{bmatrix}#1 & 0 & 0\\ 0& #2 & 0\\ 0 & 0 & #3\end{bmatrix}}
\newcommand\idxx{\begin{bmatrix}1 & 0\\ 0 & 1\end{bmatrix}}
\newcommand\idxxx{\begin{bmatrix}1 & 0 & 0\\ 0 & 1 & 0\\ 0 & 0 & 1\end{bmatrix}}
\newcommand\idxxxx{\begin{bmatrix}1 & 0 & 0 & 0\\ 0 & 1 & 0 & 0\\ 0 & 0 & 1 & 0\\ 0 & 0 & 0 & 1\end{bmatrix}}

\newcommand\bR{\mathbb{R}}
\newcommand\bC{\mathbb{C}}


\begin{document}

%%%%%HEADING
\noindent.\hrulefill.\\
\begin{minipage}{0.5\textwidth}
\noindent \sc{Math 307: Final}
\end{minipage}
\hfill
\begin{minipage}{0.5\textwidth}
\flushright \sc{Reference: Chapter 1 to 6, Peterson, Sochaki,\\ \emph{Linear Algebra and Differential Equations}.}
\end{minipage}
\begin{minipage}{0.5\textwidth}
\vskip 0.2in
Name:\\
%\noindent ID Number:\\
\end{minipage}
\hfill
\begin{minipage}{0.5\textwidth}
\flushright Score: \quad \quad \quad \quad /68 \hspace{2in}
\end{minipage}
\vskip 0.25in
\hrule

%%%%HEADING
\thispagestyle{empty}
\vskip 0.1in
\begin{center}
\fbox{Note: Notes and calculators are forbidden. Phones must be turned off and closed books.}\\
Show all your work, circle or box your answers. Write sentences and explain your line of thought.
\end{center}
\vskip .15in

%%%%%%%%%%%%%%%%%%%%%%%%%%%%%%%
%%%        Questions       %%%%
%%%%%%%%%%%%%%%%%%%%%%%%%%%%%%%

\begin{enumerate}
\item \textbf{General Knowledge} (10 points) You will not get credits if you do not explain your answers.
\begin{enumerate}

\item (3 points) Consider the linear transformation 
\[T: \left\{\begin{array}{rcl}
{\mathbb R}_3 [X] & \rightarrow & {\mathbb R}_3 [X]\\
ax^3+bx^2+cx+d & \mapsto & (a+2c)x^3-2a-4c.
\end{array}\right.\] 
Is the polynomial function $p(x)=2x^2+1$ an eigenvector of $T$? Explain your answer. \\
%\underline{Solution:} Yes, since $T(p)=0$, it is an eigenvector associated with the eigenvalue $0$.

\vspace{2.5cm}

\item (3 points) Let $v$ be an eigenvector of a matrix $A$ associated with an eigenvalue $\lambda$ and let $c$ be a non-zero scalar. Is the vector $c.v$ an eigenvector of $A$ associated with $\lambda$? \\
%\underline{Solution:} Yes, since $A(c.v)=c.Av=c\lambda v=\lambda.cv$.

 \vspace{2.5cm}


\item (3 points) Let $v_1$ and $v_2$ be two linearly independant vectors in a vector space $V$ and let $c_1$ and $c_2$ be two scalars. What is the dimension of $\text{Span}\lbrace v_1,v_2, c_1 v_1,c_1 v_1+c_2 v_2 \rbrace$?
 \\
%\underline{Solution:} The dimension is 2 since $\text{Span}\lbrace v_1,v_2, \alpha v_1,\alpha v_1+\beta v_2 \rbrace$=$\text{Span}\lbrace v_1,v_2\rbrace$ 

\vspace{2.4cm}

%\item (3 points) Find two basis $\alpha$ and $\beta$ for $\mathbb{R}^2$ such that the matrix $\begin{pmatrix} 1 &1 \\ -1 & 1 \end{pmatrix}$ is the change of basis matrix from $\alpha$ to $\beta$.\\
%\underline{Solution:} $\alpha$ is the canonical basis and $\beta= \begin{pmatrix} 1 &1\end{pmatrix}, \  \begin{pmatrix} -1 & 1 \end{pmatrix}$. 

\vspace{2.5cm}


\item (1 point) Write a grammatically correct statement, about linear algebra or differential equations, that you learned this semester. You are not allowed to use any mathematical symbols. 
 \\
\end{enumerate}

%%%%%%%%%%%%%%%%%%%%%%%%%%%%%%%
%%%%        Problems       %%%%
%%%%%%%%%%%%%%%%%%%%%%%%%%%%%%%

\newpage

%%%%%%%%%%%%%%%%%%%%%%%
\item (22 points)Let's consider the non-homogeneous first order linear differential system
\begin{equation*}
  \left\{
        \begin{array}{l}
       x'(t)=-4x(t)-3y(t)+3z(t)\\
       y'(t)=3x(t)+2y(t)-3z(t)+e^t\\
       z'(t)=-3x(t)-3y(t)+2z(t).\\
        \end{array}
    \right.
\end{equation*}

\begin{enumerate}
\item (1 point) Write the system in the matrix form $Y'(t)= AY(t)+G(t)$.
\item (6 points) Find a diagonal matrix $D$ and an invertible matrix $P$ such that $D=P^{-1}AP$.
\item (2 points) Find the general solution to the homogeneous system $Z'(t)=DZ(t)$.
\item (6 points) Use the result of the previous question to find the general solution $Y_H(t)$to the homogeneous system $Y'(t)=AY(t)$ and find a fundamental set $Y_1(t),\ Y_2(t),\ Y_3(t)$ of solutions to this system.
\item (6 points) Using the matrix $M(t)=[Y_1(t) \ Y_2(t) \ Y_3(t)]$, compute a particular solution $Y_p(t)$ to the non-homogeneous system $Y'(t)= AY(t) + G(t)$.
\item (1 point) Use the result of the previous question to find the general solution to the non-homogeneous system $Y'(t)= AY(t) + G(t)$.   
\end{enumerate}

\newpage
\
\newpage

\item (9 points) Consider the homogeneous first order linear differential system
\begin{equation*}
  \left\{
        \begin{array}{l}
       x'(t)=x(t)-y(t)+z(t)\\
       y'(t)=2y(t)+z(t)\\
       z'(t)=2z(t).\\
        \end{array}
    \right.
\end{equation*}

\begin{enumerate}
\item (1 point) Write the system in the matrix form $Y'(t)= AY(t)+G(t)$.
\item (2 points) Is the matrix $A$ diagonalizable? Explain your answer.
\item (6 points) Find the general solution to the homogeneous system $Y(t)'= AY(t)$. 
%(You will need to use the method of the integrating factor twice.)
\end{enumerate}

\newpage 
\
\newpage

\item (12 points) Let $T$ be the transformation
\[T: \left\{\begin{array}{rcl}
{\mathbb R}_2 [X] & \rightarrow & {\mathbb R}_2 [X] \\
ax^3+bx^2+cx+d &\mapsto & (c+d)x^3+bx^2 -c-d
\end{array}\right.\]
\begin{enumerate}
\item (1 points) Show that $T$ is a linear transformation.
\item (1 points) Write the matrix of $T$ in the canonical basis $\mathcal B = \{1,x,x^2, x^3\}$.
\item (4 points) Find a basis for $\text{Ker}(T)$ and the dimension of  $\text{Ker}(T)$.
\item (3 points) Consider the basis $\alpha = \lbrace x^3+x^2, \ x^3-1, \ x+1, \ x^2+x  \rbrace$. Write the matrix of $T$ in the basis $\alpha$, $Mat_\alpha^\alpha (T)$.
\item (3 points) Let $p$ be the polynomial function in $P_3$ whose coordinate vector relatively to $\alpha$ is 
\[ [p]_{\alpha}=\begin{pmatrix} 1\\ 2\\ -1\\ 1 \end{pmatrix}.\]
Find the expression of $T(p)$ in the basis $\alpha$, i.e. $[T(p)]_{\alpha}$.
\end{enumerate}

%%%%%%%%%%%%%%%%%%%%%%%%
\newpage 

\item (7 points) The goal is to solve the following Bernoulli equation
\[y'+ 3y + \cos (t) y^2.\] 
Suppose $y$ is never zero.
\begin{enumerate}
\item (3 points) First show that $x= \frac{1}{y}$ satisfies the non-homogeneous linear differential equation \[x' = 3x + \cos(t).\] 
\item (3 points) Solve $x' = 3x + \cos(t)$.
\item (1 points) Solve $y'+ 3y + \cos (t) y^2$.
\end{enumerate}

\newpage
%^\vspace{3in}

\item (8 points) The goal is to solve the following Ricatti equation
\[y'= y^2 -5y +6.\] 
\begin{enumerate}
\item (1 points) Find two particular solutions. \textbf{Hint:} look for constant solutions $y_p=\alpha$. 
\item (2 points) For each particular solution $y_p$, show that $z= y-y_p$ satisfies a Bernoulli equation.
\item (4 points) Reduce both Bernoulli equation to a linear equation and solve it.
\item (1 points) Give the general solution of $y'= y^2 -5y +6$. 
\end{enumerate}
\newpage
\

\newpage
\


\end{enumerate}
\end{document}
