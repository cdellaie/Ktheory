\documentclass{article}
\usepackage{amsfonts, graphicx, multicol, amssymb, amsmath}
%usepackage{ulem}
\usepackage[left=.4in,top=.4in,right=.4in,nohead,nofoot]{geometry}
\newcommand{\example}{\subsubsection*{Example}}


\newcommand\tr{\textrm{tr}}
\newcommand{\vecxx}[2]{\begin{bmatrix} #1\\ #2\\ \end{bmatrix}}
\newcommand{\vecxxx}[3]{\begin{bmatrix} #1\\ #2\\ #3 \end{bmatrix}}
\newcommand{\vecxxxx}[4]{\begin{bmatrix} #1\\ #2\\ #3\\ #4 \end{bmatrix}}

\newcommand{\svecxx}[2]{\left[\begin{smallmatrix} #1\\ #2\\ \end{smallmatrix}\right]}
\newcommand{\svecxxx}[3]{\left[\begin{smallmatrix} #1\\ #2\\ #3 \end{smallmatrix}\right]}
\newcommand{\svecxxxx}[4]{\left[\begin{smallmatrix} #1\\ #2\\ #3\\ #4 \end{smallmatrix}\right]}

\newcommand\matxx[4]{\begin{bmatrix}#1 & #2\\ #3 & #4\end{bmatrix}}
\newcommand\matxxx[9]{\begin{bmatrix}#1 & #2 & #3\\ #4& #5 & #6\\ #7 & #8 & #9\end{bmatrix}}

\newcommand\diagxx[2]{\begin{bmatrix}#1 & 0\\ 0 & #2\end{bmatrix}}
\newcommand\diagxxx[3]{\begin{bmatrix}#1 & 0 & 0\\ 0& #2 & 0\\ 0 & 0 & #3\end{bmatrix}}
\newcommand\idxx{\begin{bmatrix}1 & 0\\ 0 & 1\end{bmatrix}}
\newcommand\idxxx{\begin{bmatrix}1 & 0 & 0\\ 0 & 1 & 0\\ 0 & 0 & 1\end{bmatrix}}
\newcommand\idxxxx{\begin{bmatrix}1 & 0 & 0 & 0\\ 0 & 1 & 0 & 0\\ 0 & 0 & 1 & 0\\ 0 & 0 & 0 & 1\end{bmatrix}}

\newcommand\bR{\mathbb{R}}
\newcommand\bC{\mathbb{C}}


\begin{document}

%%%%%HEADING
\noindent.\hrulefill.\\
\begin{minipage}{0.5\textwidth}
\noindent \sc{Math 307: Midterm 1}
\end{minipage}
\hfill
\begin{minipage}{0.5\textwidth}
\flushright \sc{Reference: Chapter 1 and 2, Peterson, Sochaki,\\ \emph{Linear Algebra and Differential Equations}.}
\end{minipage}
\begin{minipage}{0.5\textwidth}
\vskip 0.2in
Name:\\
%\noindent ID Number:\\
\end{minipage}
\hfill
\begin{minipage}{0.5\textwidth}
\flushright Score: \hspace{2in}
\end{minipage}
\vskip 0.25in
\hrule

%%%%HEADING
\thispagestyle{empty}
\vskip 0.1in
\begin{center}
\fbox{Note: Notes and calculators are forbidden. Phones must be turned off and closed books.}\\
Show all your work, circle or box your answers. Write sentences and explain your line of thought.
\end{center}
\vskip .15in



%%%%%%%%%%%%%%%%%%%%%%%%%%%%%%%
%%%        Questions       %%%%
%%%%%%%%%%%%%%%%%%%%%%%%%%%%%%%

\begin{enumerate}
\item \textbf{General Knowledge}
\begin{enumerate}
\item Which matrices amongst the following are in echelon form? When a matrix is not in echelon form, justify your answer.
 \[\begin{split}
 A =\begin{pmatrix} 1 & 0 & 3 \\ 0 & 1 & 1 \\ 0 & 0 & 0 \end{pmatrix} & \  B =\begin{pmatrix} 1 & 1 & 0 \\ 0 & 1 & 3 \\ 0 & 0 & 1 \end{pmatrix} \\
C =\begin{pmatrix} 1 & 0 & 2 \\ 0 & 1 & 4 \\ 1 & 0 & 0 \end{pmatrix}  & \  D =\begin{pmatrix} 1 & 0 & -2 \\ 0 & 1 & -1 \\ 0 & 0 & 0 \end{pmatrix} \\
\end{split}\]

\vskip 1in

\item Recall the three steps one has to do to conduct the Gaussian Elimination algorithm (also called the Row Reduction algorithm). 

\vskip 3in

\item Let $V$ and $W$ be two finite-dimensional vector spaces. Recall the dimension formula for a linear map $T: V \rightarrow W$.
\end{enumerate}



%%%%%%%%%%%%%%%%%%%%%%%%%%%%%%%
%%%%        Problems       %%%%
%%%%%%%%%%%%%%%%%%%%%%%%%%%%%%%

\newpage

\item 
\begin{enumerate}
\item If possible, find the inverse of the following matrix:
\[\begin{pmatrix}
2 & 1 & 3 \\
2 & 1 & 1 \\
4 & 5 & 1 \\
\end{pmatrix}.\]

\vskip 3in

\item Solve the system:
\[ \left\{ 
\begin{array}{ccc} 
2x + y + 3 z & = & 6 \\
2x + y +   z & = & -12 \\
4x + 5y + z  & = & 3 \\
\end{array}
\right.
\]
\end{enumerate}

\newpage

\item Determine if the following matrices are invertible. If it is the case, compute the inverse.
\begin{enumerate}
\item \[ A =\begin{pmatrix} 2 & 1 & 3 \\ 1 & -1 & 1 \\ 1 & 1 & 2 \end{pmatrix}\]
\item \[ B =\begin{pmatrix} 1 & -2 & 2 \\ 2 & -3 & 1 \\ 1 & -1 & -1\end{pmatrix} \]
\end{enumerate}

\vskip 3in

\item Use Cramer's rule to solve the following system:
 
\[\left\{\begin{array}{ccc}
x+2y  & = & 3 \\
2x-3y  & = & 5 \\
\end{array}\right.\]
\newpage

\item Give a basis for the image and kernel of the following matrices. 
\begin{enumerate}
\item \[A= \begin{pmatrix} 1 & -1 & 0 \\ 1 & 0 & 3 \\ 2 & 0 & 6 \end{pmatrix} \]
\vskip 3in
\item \[B= \begin{pmatrix} 1 & 0 & 2 \\ 0 & 1 & 1 \\ -1 & -1 & -3 \end{pmatrix} \]
\end{enumerate}

\newpage
\item Let  \[ V = \{ \begin{pmatrix} x \\ y \\ z \end{pmatrix}\in \mathbb{R}^3 \text{ s.t. } 4x+3y+2z =0\}.\]
\begin{enumerate}
\item Show that $V$ is a subspace of $\mathbb{R}^3$.
\item Find a basis for $V$. (And justify your answer)
\item What is the dimension of $V$?
\end{enumerate}

	\thispagestyle{empty}
%	\vskip 0.1in
%\vskip 1.5in

	\thispagestyle{empty}
%\vskip 0.1in
\end{enumerate}
\end{document}
