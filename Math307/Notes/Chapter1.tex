\section{Linear systems}

\begin{itemize}
\item What is a linear system? 
\[ \left\{ 
\begin{array}{ccc} 
a_{11}x_1 + a_{12}x_2 +... + a_{1n}x_n & = & b_1 \\
a_{21}x_1 + a_{22}x_2 +... + a_{2n}x_n & = & b_2 \\
\ & \ & \ \\
a_{m1}x_1 + a_{m2}x_2 +... + a_{mn}x_n & = & b_m \\
\end{array}
\right.
\]
\item What is the augmented matrix of a linear system?
\[ \begin{pmatrix}
a_{11} &  a_{12} & ... & a_{1n}  & b_1 \\
a_{21} &  a_{22} & ... & a_{2n}  &  b_2 \\
\ & \ & \ \\
a_{m1} & a_{m2} & .. & a_{mn} &  b_m \\
\end{pmatrix} \]
\item Echelon form for a matrix (what is a leading 1?)
\begin{enumerate}
\item For every row, the first nonzero entry is a 1 (called a leading one).
\item The leading 1 of every lower row is always further on the right than the leading one of a higher row.
\item For every column containing a leading 1, all other entries are 0.
\end{enumerate}
\[ \begin{pmatrix}
1 &  * & * & ... & *  & *\\
0 &  0 & 1 & ... & *  &  * \\
\ & \ & \ \\
0 &  0 & 0 & ... &  * & * \\
\end{pmatrix} \]
\item Gaussian elimination / Row reduction algorithm: to put a matrix in echelom form.
\begin{enumerate}
\item Look down the first column until you find a nonzero entry (called a pivot). If you cannot find one, look to the next column, etc. 
\item Move to the row containing the pivot to the first position and divide that row by the pivot to make that entry a leading one.
\item Add appropriate multiples of this row to cancel the entries in the first column of each of the other rows.
\end{enumerate}

\item Number of solutions for linear systems depending on their echelon form:
\begin{enumerate}
\item \textbf{inconsistent systems:} there is a zero row on the left side with a nonzero entry on the right side. NO SOLUTION
\item \textbf{consistent systems:} $NZR =$ number of nonzero row , $C =$ number of columns
\begin{enumerate}
\item $NZR < C$ : infinitely many solutions.  
\item $NZR = C$ : one unique solution.
\end{enumerate} 
\end{enumerate}
\end{itemize}

\section{Matrix algebras and determinants}

\begin{itemize}
\item Definition of a matrix, operations on matrices: multiplication by a scalar, addition and multiplication. Matrices are useful to rewrite linear systems of equations.
\item Inverse of a matrix: not all matrices are invertible! Do you have an example? How can we find the inverse of an invertible matrix?  
\[(AB)^{-1} = B^{-1}A^{-1}\]
\item How to find the inverse of a invertible matrix A of size $n$? Apply Gaussian elimination to the left side of the $n\times 2n$ matrix:
\[\begin{pmatrix}
A & I_n \end{pmatrix},\]
and, if the matrix is invertible, the outcome is
\[\begin{pmatrix}
I_n & A^{-1} \end{pmatrix}.\]
\item The transpose of a matrix $A\in M_{m,n}(\R)$ is the matrix $A^T\in M_{n,m}(\R)$ obtained by interchanging all the rows and columns of $A$.
\[(AB)^{T} = B^{T}A^{T}\]
\item Determinants: to every square matrix, its determinant $det(A)$ is a real number.
\end{itemize} 
